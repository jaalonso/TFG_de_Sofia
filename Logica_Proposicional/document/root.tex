\documentclass[12pt,a4paper,fleqn]{book}
\usepackage{isabelle,isabellesym}
\usepackage{ifthen,mathpartir}

% further packages required for unusual symbols (see also
% isabellesym.sty), use only when needed

% Personalización
% \usepackage{color,graphicx}      % Usa figuras.
\usepackage[utf8x]{inputenc}       % Acentos de UTF8
\usepackage[T1]{fontenc}           % Codificación T1 con European Computer
% \usepackage[spanish]{babel}      % Castellanización.
\usepackage{ucs}
\usepackage{mathpazo}            % Tipo de fuente
\usepackage[scaled=.90]{helvet}  % Tipo de fuente
% \usepackage{a4wide}              % Márgenes
\linespread{1.05}                % Distancia entre líneas
\setlength{\parindent}{2em}      % Indentación de comienzo de párrafo

\usepackage[colorinlistoftodos
           , backgroundcolor = yellow
           , textwidth = 4cm
           , shadow
           , spanish]{todonotes}

\setcounter{secnumdepth}{3}
           
\usepackage{amssymb}
  %for \<leadsto>, \<box>, \<diamond>, \<sqsupset>, \<mho>, \<Join>,
  %\<lhd>, \<lesssim>, \<greatersim>, \<lessapprox>, \<greaterapprox>,
  %\<triangleq>, \<yen>, \<lozenge>

%\usepackage{eurosym}
  %for \<euro>

%\usepackage[only,bigsqcap]{stmaryrd}
  %for \<Sqinter>

%\usepackage{eufrak}
  %for \<AA> ... \<ZZ>, \<aa> ... \<zz> (also included in amssymb)

% \usepackage{textcomp}
  %for \<onequarter>, \<onehalf>, \<threequarters>, \<degree>, \<cent>,
  %\<currency>

% this should be the last package used
\usepackage{pdfsetup}

% urls in roman style, theory text in math-similar italics
\urlstyle{rm}
\isabellestyle{it}

% for uniform font size
\renewcommand{\isastyle}{\isastyleminor}

% Nota: Definiciones
\input definiciones
\input castellano

%%%%%%%%%%%%%%%%%%%%%%%%%%%%%%%%%%%%%%%%%%%%%%%%%%%%%%%%%%%%%%%%%%%%%%%%%%%%%%
%% Documento
%%%%%%%%%%%%%%%%%%%%%%%%%%%%%%%%%%%%%%%%%%%%%%%%%%%%%%%%%%%%%%%%%%%%%%%%%%%%%%

\begin{document}

\title{Lógica proposicional en Isabelle/HOL}
\author{Sofía Santiago Fernández}
\date{actualizado el 15 de diciembre de 2019}
\maketitle

\comentario{Falta la introducción.}

% \setcounter{tocdepth}{1}
\tableofcontents

% sane default for proof documents
% \parindent 0pt\parskip 0.5ex
\parindent 2em\parskip 1ex

% generated text of all theories
% %
\begin{isabellebody}%
\setisabellecontext{Sintaxis}%
%
\isadelimtheory
%
\endisadelimtheory
%
\isatagtheory
%
\endisatagtheory
{\isafoldtheory}%
%
\isadelimtheory
%
\endisadelimtheory
%
\isadelimdocument
%
\endisadelimdocument
%
\isatagdocument
%
\isamarkupsection{Fórmulas%
}
\isamarkuptrue%
%
\endisatagdocument
{\isafolddocument}%
%
\isadelimdocument
%
\endisadelimdocument
%
\begin{isamarkuptext}%
\comentario{Explicar la siguiente notación y recolocarla donde se
  use por primera vez.}

  \comentario{He quitado la palabra "continuación" del fichero 
  castellano.tex ya que no dejaba cargar el documento}%
\end{isamarkuptext}\isamarkuptrue%
\isacommand{notation}\isamarkupfalse%
\ insert\ {\isacharparenleft}{\isachardoublequoteopen}{\isacharunderscore}\ {\isasymtriangleright}\ {\isacharunderscore}{\isachardoublequoteclose}\ {\isacharbrackleft}{\isadigit{5}}{\isadigit{6}}{\isacharcomma}{\isadigit{5}}{\isadigit{5}}{\isacharbrackright}\ {\isadigit{5}}{\isadigit{5}}{\isacharparenright}%
\begin{isamarkuptext}%
En esta sección presentaremos una formalización en Isabelle de la 
  sintaxis de la lógica proposicional, junto con resultados y pruebas 
  sobre la misma. En líneas generales, primero daremos las nociones de 
  forma clásica y, a continuación, su correspondiente formalización.

  En primer lugar, supondremos que disponemos de los siguientes 
  elementos:
  \begin{description}
    \item[Alfabeto:] Es una lista infinita de variables proposicionales. 
      También pueden ser llamadas átomos o símbolos proposicionales.
    \item[Conectivas:] Conjunto finito cuyos elementos interactúan con 
      las variables. Pueden ser monarias que afectan a un único elemento 
      o binarias que afectan a dos. En el primer grupo se encuentra le 
      negación (\isa{{\isasymnot}}) y en el segundo la conjunción (\isa{{\isasymand}}), la disyunción 
      (\isa{{\isasymor}}) y la implicación (\isa{{\isasymlongrightarrow}}).
  \end{description}

  A continuación definiremos la estructura de fórmula sobre los 
  elementos anteriores. Para ello daremos una definición recursiva 
  basada en dos elementos: un conjunto de fórmulas básicas y una serie 
  de procedimientos de definición de fórmulas a partir de otras. El 
  conjunto de las fórmulas será el menor conjunto de estructuras 
  sinctáticas con dicho alfabeto y conectivas que contiene a las básicas 
  y es cerrado mediante los procedimientos de definición que mostraremos 
  a continuación.

  \begin{definicion}
    El conjunto de las fórmulas proposicionales está formado por las 
    siguientes:
    \begin{itemize}
      \item Las fórmulas atómicas, constituidas únicamente por una 
        variable del alfabeto. 
      \item La constante \isa{{\isasymbottom}}.
      \item Dada una fórmula \isa{F}, la negación \isa{{\isasymnot}\ F} es una fórmula.
      \item Dadas dos fórmulas \isa{F} y \isa{G}, la conjunción \isa{F\ {\isasymand}\ G} es una
        fórmula.
      \item Dadas dos fórmulas \isa{F} y \isa{G}, la disyunción \isa{F\ {\isasymor}\ G} es una
        fórmula.
      \item Dadas dos fórmulas \isa{F} y \isa{G}, la implicación \isa{F\ {\isasymlongrightarrow}\ G} es 
        una fórmula.
    \end{itemize}
  \end{definicion}

  Intuitivamente, las fórmulas proposicionales son entendidas como un 
  tipo de árbol sintáctico cuyos nodos son las conectivas y sus hojas
  las fórmulas atómicas.

  \comentario{Incluir el árbol de formación.}

  A continuación, veamos su representación en Isabelle%
\end{isamarkuptext}\isamarkuptrue%
\isacommand{datatype}\isamarkupfalse%
\ {\isacharparenleft}atoms{\isacharcolon}\ {\isacharprime}a{\isacharparenright}\ formula\ {\isacharequal}\ \isanewline
\ \ Atom\ {\isacharprime}a\isanewline
{\isacharbar}\ Bot\ \ \ \ \ \ \ \ \ \ \ \ \ \ \ \ \ \ \ \ \ \ \ \ \ \ \ \ \ \ {\isacharparenleft}{\isachardoublequoteopen}{\isasymbottom}{\isachardoublequoteclose}{\isacharparenright}\ \ \isanewline
{\isacharbar}\ Not\ {\isachardoublequoteopen}{\isacharprime}a\ formula{\isachardoublequoteclose}\ \ \ \ \ \ \ \ \ \ \ \ \ \ \ \ \ {\isacharparenleft}{\isachardoublequoteopen}\isactrlbold {\isasymnot}{\isachardoublequoteclose}{\isacharparenright}\isanewline
{\isacharbar}\ And\ {\isachardoublequoteopen}{\isacharprime}a\ formula{\isachardoublequoteclose}\ {\isachardoublequoteopen}{\isacharprime}a\ formula{\isachardoublequoteclose}\ \ \ \ {\isacharparenleft}\isakeyword{infix}\ {\isachardoublequoteopen}\isactrlbold {\isasymand}{\isachardoublequoteclose}\ {\isadigit{6}}{\isadigit{8}}{\isacharparenright}\isanewline
{\isacharbar}\ Or\ {\isachardoublequoteopen}{\isacharprime}a\ formula{\isachardoublequoteclose}\ {\isachardoublequoteopen}{\isacharprime}a\ formula{\isachardoublequoteclose}\ \ \ \ \ {\isacharparenleft}\isakeyword{infix}\ {\isachardoublequoteopen}\isactrlbold {\isasymor}{\isachardoublequoteclose}\ {\isadigit{6}}{\isadigit{8}}{\isacharparenright}\isanewline
{\isacharbar}\ Imp\ {\isachardoublequoteopen}{\isacharprime}a\ formula{\isachardoublequoteclose}\ {\isachardoublequoteopen}{\isacharprime}a\ formula{\isachardoublequoteclose}\ \ \ \ {\isacharparenleft}\isakeyword{infixr}\ {\isachardoublequoteopen}\isactrlbold {\isasymrightarrow}{\isachardoublequoteclose}\ {\isadigit{6}}{\isadigit{8}}{\isacharparenright}%
\begin{isamarkuptext}%
Como podemos observar representamos las fórmulas proposicionales
  mediante un tipo de dato recursivo, \isa{formula}, con los 
  siguientes constructures sobre un tipo \isa{{\isacharprime}a} cualquiera:

  \begin{description}
    \item[Fórmulas básicas:]  
      \begin{itemize}
        \item \isa{Atom\ {\isacharcolon}{\isacharcolon}\ {\isacharprime}a\ {\isasymRightarrow}\ {\isacharprime}a\ formula}
        \item \isa{{\isasymbottom}\ {\isacharcolon}{\isacharcolon}\ {\isacharprime}a\ formula}
      \end{itemize}
    \item [Fórmulas compuestas:]
      \begin{itemize}
        \item \isa{\isactrlbold {\isasymnot}\ {\isacharcolon}{\isacharcolon}\ {\isacharprime}a\ formula\ {\isasymRightarrow}\ {\isacharprime}a\ formula}
        \item \isa{{\isacharparenleft}\isactrlbold {\isasymand}{\isacharparenright}\ {\isacharcolon}{\isacharcolon}\ {\isacharprime}a\ formula\ {\isasymRightarrow}\ {\isacharprime}a\ formula\ {\isasymRightarrow}\ {\isacharprime}a\ formula}
        \item \isa{{\isacharparenleft}\isactrlbold {\isasymor}{\isacharparenright}\ {\isacharcolon}{\isacharcolon}\ {\isacharprime}a\ formula\ {\isasymRightarrow}\ {\isacharprime}a\ formula\ {\isasymRightarrow}\ {\isacharprime}a\ formula}
        \item \isa{{\isacharparenleft}\isactrlbold {\isasymrightarrow}{\isacharparenright}\ {\isacharcolon}{\isacharcolon}\ {\isacharprime}a\ formula\ {\isasymRightarrow}\ {\isacharprime}a\ formula\ {\isasymRightarrow}\ {\isacharprime}a\ formula}
      \end{itemize}
  \end{description}

  Cabe señalar que los términos \isa{infix} e \isa{infixr} nos señalan que 
  los constructores que representan a las conectivas se pueden usar de
  forma infija. En particular, \isa{infixr} se trata de un infijo asociado a 
  la derecha.

  Además se define simultáneamente la función \isa{atoms\ {\isacharcolon}{\isacharcolon}\ {\isacharprime}a\ formula\ {\isasymRightarrow}\ {\isacharprime}a\ set}, que 
  obtiene el conjunto de variables proposicionales de una fórmula. 

  Por otro lado, la definición de \isa{formula} genera 
  automáticamente los siguientes lemas sobre la función de conjuntos 
  \isa{atoms} en Isabelle.
  
  \begin{itemize}
    \item[] \isa{atoms\ {\isacharparenleft}Atom\ x{\isadigit{1}}{\isachardot}{\isadigit{0}}{\isacharparenright}\ {\isacharequal}\ {\isacharbraceleft}x{\isadigit{1}}{\isachardot}{\isadigit{0}}{\isacharbraceright}\isasep\isanewline%
atoms\ {\isasymbottom}\ {\isacharequal}\ {\isasymemptyset}\isasep\isanewline%
atoms\ {\isacharparenleft}\isactrlbold {\isasymnot}\ x{\isadigit{3}}{\isachardot}{\isadigit{0}}{\isacharparenright}\ {\isacharequal}\ atoms\ x{\isadigit{3}}{\isachardot}{\isadigit{0}}\isasep\isanewline%
atoms\ {\isacharparenleft}x{\isadigit{4}}{\isadigit{1}}{\isachardot}{\isadigit{0}}\ \isactrlbold {\isasymand}\ x{\isadigit{4}}{\isadigit{2}}{\isachardot}{\isadigit{0}}{\isacharparenright}\ {\isacharequal}\ atoms\ x{\isadigit{4}}{\isadigit{1}}{\isachardot}{\isadigit{0}}\ {\isasymunion}\ atoms\ x{\isadigit{4}}{\isadigit{2}}{\isachardot}{\isadigit{0}}\isasep\isanewline%
atoms\ {\isacharparenleft}x{\isadigit{5}}{\isadigit{1}}{\isachardot}{\isadigit{0}}\ \isactrlbold {\isasymor}\ x{\isadigit{5}}{\isadigit{2}}{\isachardot}{\isadigit{0}}{\isacharparenright}\ {\isacharequal}\ atoms\ x{\isadigit{5}}{\isadigit{1}}{\isachardot}{\isadigit{0}}\ {\isasymunion}\ atoms\ x{\isadigit{5}}{\isadigit{2}}{\isachardot}{\isadigit{0}}\isasep\isanewline%
atoms\ {\isacharparenleft}x{\isadigit{6}}{\isadigit{1}}{\isachardot}{\isadigit{0}}\ \isactrlbold {\isasymrightarrow}\ x{\isadigit{6}}{\isadigit{2}}{\isachardot}{\isadigit{0}}{\isacharparenright}\ {\isacharequal}\ atoms\ x{\isadigit{6}}{\isadigit{1}}{\isachardot}{\isadigit{0}}\ {\isasymunion}\ atoms\ x{\isadigit{6}}{\isadigit{2}}{\isachardot}{\isadigit{0}}}
  \end{itemize} 

  A continuación veremos varios ejemplos de fórmulas y el conjunto de 
  sus variables proposicionales obtenido mediante \isa{atoms}. Se 
  observa que, por definición de conjunto, no contiene 
  elementos repetidos.%
\end{isamarkuptext}\isamarkuptrue%
\isacommand{notepad}\isamarkupfalse%
\ \isanewline
\isakeyword{begin}\isanewline
%
\isadelimproof
\ \ %
\endisadelimproof
%
\isatagproof
\isacommand{fix}\isamarkupfalse%
\ p\ q\ r\ {\isacharcolon}{\isacharcolon}\ {\isacharprime}a\isanewline
\isanewline
\ \ \isacommand{have}\isamarkupfalse%
\ {\isachardoublequoteopen}atoms\ {\isacharparenleft}Atom\ p{\isacharparenright}\ {\isacharequal}\ {\isacharbraceleft}p{\isacharbraceright}{\isachardoublequoteclose}\isanewline
\ \ \ \ \isacommand{by}\isamarkupfalse%
\ {\isacharparenleft}simp\ only{\isacharcolon}\ formula{\isachardot}set{\isacharparenright}\isanewline
\isanewline
\ \ \isacommand{have}\isamarkupfalse%
\ {\isachardoublequoteopen}atoms\ {\isacharparenleft}\isactrlbold {\isasymnot}\ {\isacharparenleft}Atom\ p{\isacharparenright}{\isacharparenright}\ {\isacharequal}\ {\isacharbraceleft}p{\isacharbraceright}{\isachardoublequoteclose}\isanewline
\ \ \ \ \isacommand{by}\isamarkupfalse%
\ {\isacharparenleft}simp\ only{\isacharcolon}\ formula{\isachardot}set{\isacharparenright}\isanewline
\isanewline
\ \ \isacommand{have}\isamarkupfalse%
\ {\isachardoublequoteopen}atoms\ {\isacharparenleft}{\isacharparenleft}Atom\ p\ \isactrlbold {\isasymrightarrow}\ Atom\ q{\isacharparenright}\ \isactrlbold {\isasymor}\ Atom\ r{\isacharparenright}\ {\isacharequal}\ {\isacharbraceleft}p{\isacharcomma}q{\isacharcomma}r{\isacharbraceright}{\isachardoublequoteclose}\isanewline
\ \ \ \ \isacommand{by}\isamarkupfalse%
\ auto\isanewline
\isanewline
\ \ \isacommand{have}\isamarkupfalse%
\ {\isachardoublequoteopen}atoms\ {\isacharparenleft}{\isacharparenleft}Atom\ p\ \isactrlbold {\isasymrightarrow}\ Atom\ p{\isacharparenright}\ \isactrlbold {\isasymor}\ Atom\ r{\isacharparenright}\ {\isacharequal}\ {\isacharbraceleft}p{\isacharcomma}r{\isacharbraceright}{\isachardoublequoteclose}\isanewline
\ \ \ \ \isacommand{by}\isamarkupfalse%
\ auto%
\endisatagproof
{\isafoldproof}%
%
\isadelimproof
\ \ \isanewline
%
\endisadelimproof
\isacommand{end}\isamarkupfalse%
%
\begin{isamarkuptext}%
En particular, el conjunto de símbolos proposicionales de la 
  fórmula \isa{Bot} es vacío. Además, para calcular esta constante es 
  necesario especificar el tipo sobre el que se construye la fórmula.%
\end{isamarkuptext}\isamarkuptrue%
\isacommand{notepad}\isamarkupfalse%
\ \isanewline
\isakeyword{begin}\isanewline
%
\isadelimproof
\ \ %
\endisadelimproof
%
\isatagproof
\isacommand{fix}\isamarkupfalse%
\ p\ {\isacharcolon}{\isacharcolon}\ {\isacharprime}a\isanewline
\isanewline
\ \ \isacommand{have}\isamarkupfalse%
\ {\isachardoublequoteopen}atoms\ {\isasymbottom}\ {\isacharequal}\ {\isasymemptyset}{\isachardoublequoteclose}\isanewline
\ \ \ \ \isacommand{by}\isamarkupfalse%
\ {\isacharparenleft}simp\ only{\isacharcolon}\ formula{\isachardot}set{\isacharparenright}\isanewline
\isanewline
\ \ \isacommand{have}\isamarkupfalse%
\ {\isachardoublequoteopen}atoms\ {\isacharparenleft}Atom\ p\ \isactrlbold {\isasymor}\ {\isasymbottom}{\isacharparenright}\ {\isacharequal}\ {\isacharbraceleft}p{\isacharbraceright}{\isachardoublequoteclose}\isanewline
\ \ \isacommand{proof}\isamarkupfalse%
\ {\isacharminus}\isanewline
\ \ \ \ \isacommand{have}\isamarkupfalse%
\ {\isachardoublequoteopen}atoms\ {\isacharparenleft}Atom\ p\ \isactrlbold {\isasymor}\ {\isasymbottom}{\isacharparenright}\ {\isacharequal}\ atoms\ {\isacharparenleft}Atom\ p{\isacharparenright}\ {\isasymunion}\ atoms\ Bot{\isachardoublequoteclose}\isanewline
\ \ \ \ \ \ \isacommand{by}\isamarkupfalse%
\ {\isacharparenleft}simp\ only{\isacharcolon}\ formula{\isachardot}set{\isacharparenleft}{\isadigit{5}}{\isacharparenright}{\isacharparenright}\isanewline
\ \ \ \ \isacommand{also}\isamarkupfalse%
\ \isacommand{have}\isamarkupfalse%
\ {\isachardoublequoteopen}{\isasymdots}\ {\isacharequal}\ {\isacharbraceleft}p{\isacharbraceright}\ {\isasymunion}\ atoms\ Bot{\isachardoublequoteclose}\isanewline
\ \ \ \ \ \ \isacommand{by}\isamarkupfalse%
\ {\isacharparenleft}simp\ only{\isacharcolon}\ formula{\isachardot}set{\isacharparenleft}{\isadigit{1}}{\isacharparenright}{\isacharparenright}\isanewline
\ \ \ \ \isacommand{also}\isamarkupfalse%
\ \isacommand{have}\isamarkupfalse%
\ {\isachardoublequoteopen}{\isasymdots}\ {\isacharequal}\ {\isacharbraceleft}p{\isacharbraceright}\ {\isasymunion}\ {\isasymemptyset}{\isachardoublequoteclose}\isanewline
\ \ \ \ \ \ \isacommand{by}\isamarkupfalse%
\ {\isacharparenleft}simp\ only{\isacharcolon}\ formula{\isachardot}set{\isacharparenleft}{\isadigit{2}}{\isacharparenright}{\isacharparenright}\isanewline
\ \ \ \ \isacommand{also}\isamarkupfalse%
\ \isacommand{have}\isamarkupfalse%
\ {\isachardoublequoteopen}{\isasymdots}\ {\isacharequal}\ {\isacharbraceleft}p{\isacharbraceright}{\isachardoublequoteclose}\isanewline
\ \ \ \ \ \ \isacommand{by}\isamarkupfalse%
\ {\isacharparenleft}simp\ only{\isacharcolon}\ Un{\isacharunderscore}empty{\isacharunderscore}right{\isacharparenright}\isanewline
\ \ \ \ \isacommand{finally}\isamarkupfalse%
\ \isacommand{show}\isamarkupfalse%
\ {\isachardoublequoteopen}atoms\ {\isacharparenleft}Atom\ p\ \isactrlbold {\isasymor}\ {\isasymbottom}{\isacharparenright}\ {\isacharequal}\ {\isacharbraceleft}p{\isacharbraceright}{\isachardoublequoteclose}\isanewline
\ \ \ \ \ \ \isacommand{by}\isamarkupfalse%
\ this\isanewline
\ \ \isacommand{qed}\isamarkupfalse%
\isanewline
\isanewline
\ \ \isacommand{have}\isamarkupfalse%
\ {\isachardoublequoteopen}atoms\ {\isacharparenleft}Atom\ p\ \isactrlbold {\isasymor}\ {\isasymbottom}{\isacharparenright}\ {\isacharequal}\ {\isacharbraceleft}p{\isacharbraceright}{\isachardoublequoteclose}\isanewline
\ \ \ \ \isacommand{by}\isamarkupfalse%
\ {\isacharparenleft}simp\ only{\isacharcolon}\ formula{\isachardot}set\ Un{\isacharunderscore}empty{\isacharunderscore}right{\isacharparenright}%
\endisatagproof
{\isafoldproof}%
%
\isadelimproof
\isanewline
%
\endisadelimproof
\isacommand{end}\isamarkupfalse%
\isanewline
\isanewline
\isacommand{value}\isamarkupfalse%
\ {\isachardoublequoteopen}{\isacharparenleft}Bot{\isacharcolon}{\isacharcolon}nat\ formula{\isacharparenright}{\isachardoublequoteclose}%
\begin{isamarkuptext}%
Una vez definida la estructura de las fórmulas, vamos a introducir 
  el método de demostración que seguirán los resultados que aquí 
  presentaremos, tanto en la teoría clásica como en Isabelle. 

  Según la definición recursiva de las fórmulas, dispondremos de un 
  esquema de inducción sobre las mismas:

  \begin{definicion}
    Sea \isa{{\isasymP}} una propiedad sobre fórmulas que verifica las siguientes 
    condiciones:
    \begin{itemize}
      \item Las fórmulas atómicas la cumplen.
      \item La constante \isa{{\isasymbottom}} la cumple.
      \item Dada \isa{F} fórmula que la cumple, entonces \isa{{\isasymnot}\ F} la cumple.
      \item Dadas \isa{F} y \isa{G} fórmulas que la cumplen, entonces \isa{F\ {\isacharasterisk}\ G} la 
        cumple, donde \isa{{\isacharasterisk}} simboliza cualquier conectiva binaria.
    \end{itemize}
    Entonces, todas las fórmulas proposicionales tienen la propiedad 
    \isa{{\isasymP}}.
  \end{definicion}

  Análogamente, como las fórmulas proposicionales están definidas 
  mediante un tipo de datos recursivo, Isabelle genera de forma 
  automática el esquema de inducción correspondiente. De este modo, en 
  las pruebas formalizadas utilizaremos la táctica \isa{induction}, 
  que corresponde al siguiente esquema.

  \comentario{Poner bien cada regla.}

  \begin{itemize}
    \item[] \isa{\mbox{}\inferrule{\mbox{{\isasymAnd}x{\isachardot}\ P\ {\isacharparenleft}Atom\ x{\isacharparenright}}\\\ \mbox{P\ {\isasymbottom}}\\\ \mbox{{\isasymAnd}x{\isachardot}\ \mbox{}\inferrule{\mbox{P\ x}}{\mbox{P\ {\isacharparenleft}\isactrlbold {\isasymnot}\ x{\isacharparenright}}}}\\\ \mbox{{\isasymAnd}x{\isadigit{1}}a\ x{\isadigit{2}}{\isachardot}\ \mbox{}\inferrule{\mbox{P\ x{\isadigit{1}}a\ {\isasymand}\ P\ x{\isadigit{2}}}}{\mbox{P\ {\isacharparenleft}x{\isadigit{1}}a\ \isactrlbold {\isasymand}\ x{\isadigit{2}}{\isacharparenright}}}}\\\ \mbox{{\isasymAnd}x{\isadigit{1}}a\ x{\isadigit{2}}{\isachardot}\ \mbox{}\inferrule{\mbox{P\ x{\isadigit{1}}a\ {\isasymand}\ P\ x{\isadigit{2}}}}{\mbox{P\ {\isacharparenleft}x{\isadigit{1}}a\ \isactrlbold {\isasymor}\ x{\isadigit{2}}{\isacharparenright}}}}\\\ \mbox{{\isasymAnd}x{\isadigit{1}}a\ x{\isadigit{2}}{\isachardot}\ \mbox{}\inferrule{\mbox{P\ x{\isadigit{1}}a\ {\isasymand}\ P\ x{\isadigit{2}}}}{\mbox{P\ {\isacharparenleft}x{\isadigit{1}}a\ \isactrlbold {\isasymrightarrow}\ x{\isadigit{2}}{\isacharparenright}}}}}{\mbox{P\ formula}}}
  \end{itemize} 

  Como hemos señalado, el esquema inductivo se aplicará en cada uno de 
  los casos de los constructores, desglosándose así seis casos distintos 
  como se muestra anteriormente. Además, todas las demostraciones sobre 
  casos de conectivas binarias son equivalentes en esta sección, pues la 
  construcción sintáctica de fórmulas es idéntica entre ellas. Estas se 
  diferencian esencialmente en la connotación semántica que veremos más 
  adelante.

  Llegamos así al primer resultado de este apartado:

  \begin{lema}
    El conjunto de los átomos de una fórmula proposicional es finito.
  \end{lema}

  Para proceder a la demostración, vamos a dar una definición inductiva 
  de conjunto finito. Cabe añadir que la demostración seguirá el esquema 
  inductivo relativo a la estructura de fórmula, y no el que induce esta
  última definición.

  \begin{definicion}
    Los conjuntos finitos son:
      \begin{itemize}
        \item El vacío.
        \item Dado un conjunto finito \isa{A} y un elemento cualquiera \isa{a}, 
          entonces \isa{{\isacharbraceleft}a{\isacharbraceright}\ {\isasymunion}\ A} es finito.
      \end{itemize}
  \end{definicion}

  En Isabelle, podemos formalizar el lema como sigue.%
\end{isamarkuptext}\isamarkuptrue%
\isacommand{lemma}\isamarkupfalse%
\ {\isachardoublequoteopen}finite\ {\isacharparenleft}atoms\ F{\isacharparenright}{\isachardoublequoteclose}\isanewline
%
\isadelimproof
\ \ %
\endisadelimproof
%
\isatagproof
\isacommand{oops}\isamarkupfalse%
%
\endisatagproof
{\isafoldproof}%
%
\isadelimproof
%
\endisadelimproof
%
\begin{isamarkuptext}%
Análogamente, el enunciado formalizado contiene la definición 
  \isa{finite\ S}, perteneciente a la teoría 
  \href{https://n9.cl/x86r}{FiniteSet.thy}.%
\end{isamarkuptext}\isamarkuptrue%
\isacommand{inductive}\isamarkupfalse%
\ finite{\isacharprime}\ {\isacharcolon}{\isacharcolon}\ {\isachardoublequoteopen}{\isacharprime}a\ set\ {\isasymRightarrow}\ bool{\isachardoublequoteclose}\ \isakeyword{where}\isanewline
\ \ emptyI{\isacharprime}\ {\isacharbrackleft}simp{\isacharcomma}\ intro{\isacharbang}{\isacharbrackright}{\isacharcolon}\ {\isachardoublequoteopen}finite{\isacharprime}\ {\isacharbraceleft}{\isacharbraceright}{\isachardoublequoteclose}\isanewline
{\isacharbar}\ insertI{\isacharprime}\ {\isacharbrackleft}simp{\isacharcomma}\ intro{\isacharbang}{\isacharbrackright}{\isacharcolon}\ {\isachardoublequoteopen}finite{\isacharprime}\ A\ {\isasymLongrightarrow}\ finite{\isacharprime}\ {\isacharparenleft}insert\ a\ A{\isacharparenright}{\isachardoublequoteclose}%
\begin{isamarkuptext}%
Observemos que la definición anterior corresponde a 
  \isa{finite{\isacharprime}}. Sin embargo, es análoga a \isa{finite} de la 
  teoría original. Este cambio de notación es necesario para no definir 
  dos veces de manera idéntica la misma noción en Isabelle. Por otra 
  parte, esta definición permitiría la demostración del lema por 
  simplificacion pues, dentro de ella las reglas que especifica se han 
  añadido como tácticas de \isa{simp} e \isa{intro{\isacharbang}}. Sin embargo, conforme al 
  objetivo de este análisis, detallaremos dónde es usada cada una de las 
  reglas en la prueba detallada. 

  A continuación, veamos en primer lugar la demostración clásica del 
  lema. 

  \begin{demostracion}
  La prueba es por inducción sobre el tipo recursivo de las fórmulas. 
  Veamos cada caso.
  
  Consideremos una fórmula atómica \isa{p} cualquiera. Entonces, 
  su conjunto de variables proposicionales es \isa{{\isacharbraceleft}p{\isacharbraceright}}, finito.

  Sea la fórmula \isa{{\isasymbottom}}. Entonces, su conjunto de átomos es vacío y, por 
  lo tanto, finito.
  
  Sea \isa{F} una fórmula cuyo conjunto de variables proposicionales sea 
  finito. Entonces, por definición, \isa{{\isasymnot}\ F} y \isa{F} tienen igual conjunto de
  átomos y, por hipótesis de inducción, es finito.

  Consideremos las fórmulas \isa{F} y \isa{G} cuyos conjuntos de átomos son 
  finitos. Por construcción, el conjunto de variables de \isa{F{\isacharasterisk}G} es la 
  unión de sus respectivos conjuntos de átomos para cualquier \isa{{\isacharasterisk}} 
  conectiva binaria. Por lo tanto, usando la hipótesis de inducción, 
  dicho conjunto es finito. 
  \end{demostracion} 

  Veamos ahora la prueba detallada en Isabelle. Mostraremos con detalle 
  todos los casos de conectivas binarias, aunque se puede observar que 
  son completamente análogos. Para facilitar la lectura, primero 
  demostraremos por separado cada uno de los casos según el esquema 
  inductivo de fórmulas, y finalmente añadiremos la prueba para una 
  fórmula cualquiera a partir de los anteriores.%
\end{isamarkuptext}\isamarkuptrue%
\isacommand{lemma}\isamarkupfalse%
\ atoms{\isacharunderscore}finite{\isacharunderscore}atom{\isacharcolon}\isanewline
\ \ {\isachardoublequoteopen}finite\ {\isacharparenleft}atoms\ {\isacharparenleft}Atom\ x{\isacharparenright}{\isacharparenright}{\isachardoublequoteclose}\isanewline
%
\isadelimproof
%
\endisadelimproof
%
\isatagproof
\isacommand{proof}\isamarkupfalse%
\ {\isacharminus}\isanewline
\ \ \isacommand{have}\isamarkupfalse%
\ {\isachardoublequoteopen}finite\ {\isasymemptyset}{\isachardoublequoteclose}\isanewline
\ \ \ \ \isacommand{by}\isamarkupfalse%
\ {\isacharparenleft}simp\ only{\isacharcolon}\ finite{\isachardot}emptyI{\isacharparenright}\isanewline
\ \ \isacommand{then}\isamarkupfalse%
\ \isacommand{have}\isamarkupfalse%
\ {\isachardoublequoteopen}finite\ {\isacharbraceleft}x{\isacharbraceright}{\isachardoublequoteclose}\isanewline
\ \ \ \ \isacommand{by}\isamarkupfalse%
\ {\isacharparenleft}simp\ only{\isacharcolon}\ finite{\isacharunderscore}insert{\isacharparenright}\isanewline
\ \ \isacommand{then}\isamarkupfalse%
\ \isacommand{show}\isamarkupfalse%
\ {\isachardoublequoteopen}finite\ {\isacharparenleft}atoms\ {\isacharparenleft}Atom\ x{\isacharparenright}{\isacharparenright}{\isachardoublequoteclose}\isanewline
\ \ \ \ \isacommand{by}\isamarkupfalse%
\ {\isacharparenleft}simp\ only{\isacharcolon}\ formula{\isachardot}set{\isacharparenleft}{\isadigit{1}}{\isacharparenright}{\isacharparenright}\ \isanewline
\isacommand{qed}\isamarkupfalse%
%
\endisatagproof
{\isafoldproof}%
%
\isadelimproof
\isanewline
%
\endisadelimproof
\isanewline
\isacommand{lemma}\isamarkupfalse%
\ atoms{\isacharunderscore}finite{\isacharunderscore}bot{\isacharcolon}\isanewline
\ \ {\isachardoublequoteopen}finite\ {\isacharparenleft}atoms\ {\isasymbottom}{\isacharparenright}{\isachardoublequoteclose}\isanewline
%
\isadelimproof
%
\endisadelimproof
%
\isatagproof
\isacommand{proof}\isamarkupfalse%
\ {\isacharminus}\isanewline
\ \ \isacommand{have}\isamarkupfalse%
\ {\isachardoublequoteopen}finite\ {\isasymemptyset}{\isachardoublequoteclose}\isanewline
\ \ \ \ \isacommand{by}\isamarkupfalse%
\ {\isacharparenleft}simp\ only{\isacharcolon}\ finite{\isachardot}emptyI{\isacharparenright}\isanewline
\ \ \isacommand{then}\isamarkupfalse%
\ \isacommand{show}\isamarkupfalse%
\ {\isachardoublequoteopen}finite\ {\isacharparenleft}atoms\ {\isasymbottom}{\isacharparenright}{\isachardoublequoteclose}\isanewline
\ \ \ \ \isacommand{by}\isamarkupfalse%
\ {\isacharparenleft}simp\ only{\isacharcolon}\ formula{\isachardot}set{\isacharparenleft}{\isadigit{2}}{\isacharparenright}{\isacharparenright}\ \isanewline
\isacommand{qed}\isamarkupfalse%
%
\endisatagproof
{\isafoldproof}%
%
\isadelimproof
\isanewline
%
\endisadelimproof
\isanewline
\isacommand{lemma}\isamarkupfalse%
\ atoms{\isacharunderscore}finite{\isacharunderscore}not{\isacharcolon}\isanewline
\ \ \isakeyword{assumes}\ {\isachardoublequoteopen}finite\ {\isacharparenleft}atoms\ F{\isacharparenright}{\isachardoublequoteclose}\ \isanewline
\ \ \isakeyword{shows}\ \ \ {\isachardoublequoteopen}finite\ {\isacharparenleft}atoms\ {\isacharparenleft}\isactrlbold {\isasymnot}\ F{\isacharparenright}{\isacharparenright}{\isachardoublequoteclose}\isanewline
%
\isadelimproof
\ \ %
\endisadelimproof
%
\isatagproof
\isacommand{using}\isamarkupfalse%
\ assms\isanewline
\ \ \isacommand{by}\isamarkupfalse%
\ {\isacharparenleft}simp\ only{\isacharcolon}\ formula{\isachardot}set{\isacharparenleft}{\isadigit{3}}{\isacharparenright}{\isacharparenright}%
\endisatagproof
{\isafoldproof}%
%
\isadelimproof
\ \isanewline
%
\endisadelimproof
\isanewline
\isacommand{lemma}\isamarkupfalse%
\ atoms{\isacharunderscore}finite{\isacharunderscore}and{\isacharcolon}\isanewline
\ \ \isakeyword{assumes}\ {\isachardoublequoteopen}finite\ {\isacharparenleft}atoms\ F{\isadigit{1}}{\isacharparenright}{\isachardoublequoteclose}\isanewline
\ \ \ \ \ \ \ \ \ \ {\isachardoublequoteopen}finite\ {\isacharparenleft}atoms\ F{\isadigit{2}}{\isacharparenright}{\isachardoublequoteclose}\isanewline
\ \ \isakeyword{shows}\ \ \ {\isachardoublequoteopen}finite\ {\isacharparenleft}atoms\ {\isacharparenleft}F{\isadigit{1}}\ \isactrlbold {\isasymand}\ F{\isadigit{2}}{\isacharparenright}{\isacharparenright}{\isachardoublequoteclose}\isanewline
%
\isadelimproof
%
\endisadelimproof
%
\isatagproof
\isacommand{proof}\isamarkupfalse%
\ {\isacharminus}\isanewline
\ \ \isacommand{have}\isamarkupfalse%
\ {\isachardoublequoteopen}finite\ {\isacharparenleft}atoms\ F{\isadigit{1}}\ {\isasymunion}\ atoms\ F{\isadigit{2}}{\isacharparenright}{\isachardoublequoteclose}\isanewline
\ \ \ \ \isacommand{using}\isamarkupfalse%
\ assms\isanewline
\ \ \ \ \isacommand{by}\isamarkupfalse%
\ {\isacharparenleft}simp\ only{\isacharcolon}\ finite{\isacharunderscore}UnI{\isacharparenright}\isanewline
\ \ \isacommand{then}\isamarkupfalse%
\ \isacommand{show}\isamarkupfalse%
\ {\isachardoublequoteopen}finite\ {\isacharparenleft}atoms\ {\isacharparenleft}F{\isadigit{1}}\ \isactrlbold {\isasymand}\ F{\isadigit{2}}{\isacharparenright}{\isacharparenright}{\isachardoublequoteclose}\ \ \isanewline
\ \ \ \ \isacommand{by}\isamarkupfalse%
\ {\isacharparenleft}simp\ only{\isacharcolon}\ formula{\isachardot}set{\isacharparenleft}{\isadigit{4}}{\isacharparenright}{\isacharparenright}\isanewline
\isacommand{qed}\isamarkupfalse%
%
\endisatagproof
{\isafoldproof}%
%
\isadelimproof
\isanewline
%
\endisadelimproof
\isanewline
\isacommand{lemma}\isamarkupfalse%
\ atoms{\isacharunderscore}finite{\isacharunderscore}or{\isacharcolon}\isanewline
\ \ \isakeyword{assumes}\ {\isachardoublequoteopen}finite\ {\isacharparenleft}atoms\ F{\isadigit{1}}{\isacharparenright}{\isachardoublequoteclose}\isanewline
\ \ \ \ \ \ \ \ \ \ {\isachardoublequoteopen}finite\ {\isacharparenleft}atoms\ F{\isadigit{2}}{\isacharparenright}{\isachardoublequoteclose}\isanewline
\ \ \isakeyword{shows}\ \ \ {\isachardoublequoteopen}finite\ {\isacharparenleft}atoms\ {\isacharparenleft}F{\isadigit{1}}\ \isactrlbold {\isasymor}\ F{\isadigit{2}}{\isacharparenright}{\isacharparenright}{\isachardoublequoteclose}\isanewline
%
\isadelimproof
%
\endisadelimproof
%
\isatagproof
\isacommand{proof}\isamarkupfalse%
\ {\isacharminus}\isanewline
\ \ \isacommand{have}\isamarkupfalse%
\ {\isachardoublequoteopen}finite\ {\isacharparenleft}atoms\ F{\isadigit{1}}\ {\isasymunion}\ atoms\ F{\isadigit{2}}{\isacharparenright}{\isachardoublequoteclose}\isanewline
\ \ \ \ \isacommand{using}\isamarkupfalse%
\ assms\isanewline
\ \ \ \ \isacommand{by}\isamarkupfalse%
\ {\isacharparenleft}simp\ only{\isacharcolon}\ finite{\isacharunderscore}UnI{\isacharparenright}\isanewline
\ \ \isacommand{then}\isamarkupfalse%
\ \isacommand{show}\isamarkupfalse%
\ {\isachardoublequoteopen}finite\ {\isacharparenleft}atoms\ {\isacharparenleft}F{\isadigit{1}}\ \isactrlbold {\isasymor}\ F{\isadigit{2}}{\isacharparenright}{\isacharparenright}{\isachardoublequoteclose}\ \ \isanewline
\ \ \ \ \isacommand{by}\isamarkupfalse%
\ {\isacharparenleft}simp\ only{\isacharcolon}\ formula{\isachardot}set{\isacharparenleft}{\isadigit{5}}{\isacharparenright}{\isacharparenright}\isanewline
\isacommand{qed}\isamarkupfalse%
%
\endisatagproof
{\isafoldproof}%
%
\isadelimproof
\isanewline
%
\endisadelimproof
\isanewline
\isacommand{lemma}\isamarkupfalse%
\ atoms{\isacharunderscore}finite{\isacharunderscore}imp{\isacharcolon}\isanewline
\ \ \isakeyword{assumes}\ {\isachardoublequoteopen}finite\ {\isacharparenleft}atoms\ F{\isadigit{1}}{\isacharparenright}{\isachardoublequoteclose}\isanewline
\ \ \ \ \ \ \ \ \ \ {\isachardoublequoteopen}finite\ {\isacharparenleft}atoms\ F{\isadigit{2}}{\isacharparenright}{\isachardoublequoteclose}\isanewline
\ \ \isakeyword{shows}\ \ \ {\isachardoublequoteopen}finite\ {\isacharparenleft}atoms\ {\isacharparenleft}F{\isadigit{1}}\ \isactrlbold {\isasymrightarrow}\ F{\isadigit{2}}{\isacharparenright}{\isacharparenright}{\isachardoublequoteclose}\isanewline
%
\isadelimproof
%
\endisadelimproof
%
\isatagproof
\isacommand{proof}\isamarkupfalse%
\ {\isacharminus}\isanewline
\ \ \isacommand{have}\isamarkupfalse%
\ {\isachardoublequoteopen}finite\ {\isacharparenleft}atoms\ F{\isadigit{1}}\ {\isasymunion}\ atoms\ F{\isadigit{2}}{\isacharparenright}{\isachardoublequoteclose}\isanewline
\ \ \ \ \isacommand{using}\isamarkupfalse%
\ assms\isanewline
\ \ \ \ \isacommand{by}\isamarkupfalse%
\ {\isacharparenleft}simp\ only{\isacharcolon}\ finite{\isacharunderscore}UnI{\isacharparenright}\isanewline
\ \ \isacommand{then}\isamarkupfalse%
\ \isacommand{show}\isamarkupfalse%
\ {\isachardoublequoteopen}finite\ {\isacharparenleft}atoms\ {\isacharparenleft}F{\isadigit{1}}\ \isactrlbold {\isasymrightarrow}\ F{\isadigit{2}}{\isacharparenright}{\isacharparenright}{\isachardoublequoteclose}\ \ \isanewline
\ \ \ \ \isacommand{by}\isamarkupfalse%
\ {\isacharparenleft}simp\ only{\isacharcolon}\ formula{\isachardot}set{\isacharparenleft}{\isadigit{6}}{\isacharparenright}{\isacharparenright}\isanewline
\isacommand{qed}\isamarkupfalse%
%
\endisatagproof
{\isafoldproof}%
%
\isadelimproof
\isanewline
%
\endisadelimproof
\isanewline
\isacommand{lemma}\isamarkupfalse%
\ atoms{\isacharunderscore}finite{\isacharcolon}\ {\isachardoublequoteopen}finite\ {\isacharparenleft}atoms\ F{\isacharparenright}{\isachardoublequoteclose}\isanewline
%
\isadelimproof
%
\endisadelimproof
%
\isatagproof
\isacommand{proof}\isamarkupfalse%
\ {\isacharparenleft}induction\ F{\isacharparenright}\isanewline
\ \ \isacommand{case}\isamarkupfalse%
\ {\isacharparenleft}Atom\ x{\isacharparenright}\isanewline
\ \ \isacommand{then}\isamarkupfalse%
\ \isacommand{show}\isamarkupfalse%
\ {\isacharquery}case\ \isacommand{by}\isamarkupfalse%
\ {\isacharparenleft}simp\ only{\isacharcolon}\ atoms{\isacharunderscore}finite{\isacharunderscore}atom{\isacharparenright}\isanewline
\isacommand{next}\isamarkupfalse%
\isanewline
\ \ \isacommand{case}\isamarkupfalse%
\ Bot\isanewline
\ \ \isacommand{then}\isamarkupfalse%
\ \isacommand{show}\isamarkupfalse%
\ {\isacharquery}case\ \isacommand{by}\isamarkupfalse%
\ {\isacharparenleft}simp\ only{\isacharcolon}\ atoms{\isacharunderscore}finite{\isacharunderscore}bot{\isacharparenright}\isanewline
\isacommand{next}\isamarkupfalse%
\isanewline
\ \ \isacommand{case}\isamarkupfalse%
\ {\isacharparenleft}Not\ F{\isacharparenright}\isanewline
\ \ \isacommand{then}\isamarkupfalse%
\ \isacommand{show}\isamarkupfalse%
\ {\isacharquery}case\ \isacommand{by}\isamarkupfalse%
\ {\isacharparenleft}simp\ only{\isacharcolon}\ atoms{\isacharunderscore}finite{\isacharunderscore}not{\isacharparenright}\isanewline
\isacommand{next}\isamarkupfalse%
\isanewline
\ \ \isacommand{case}\isamarkupfalse%
\ {\isacharparenleft}And\ F{\isadigit{1}}\ F{\isadigit{2}}{\isacharparenright}\isanewline
\ \ \isacommand{then}\isamarkupfalse%
\ \isacommand{show}\isamarkupfalse%
\ {\isacharquery}case\ \isacommand{by}\isamarkupfalse%
\ {\isacharparenleft}simp\ only{\isacharcolon}\ atoms{\isacharunderscore}finite{\isacharunderscore}and{\isacharparenright}\isanewline
\isacommand{next}\isamarkupfalse%
\isanewline
\ \ \isacommand{case}\isamarkupfalse%
\ {\isacharparenleft}Or\ F{\isadigit{1}}\ F{\isadigit{2}}{\isacharparenright}\isanewline
\ \ \isacommand{then}\isamarkupfalse%
\ \isacommand{show}\isamarkupfalse%
\ {\isacharquery}case\ \isacommand{by}\isamarkupfalse%
\ {\isacharparenleft}simp\ only{\isacharcolon}\ atoms{\isacharunderscore}finite{\isacharunderscore}or{\isacharparenright}\isanewline
\isacommand{next}\isamarkupfalse%
\isanewline
\ \ \isacommand{case}\isamarkupfalse%
\ {\isacharparenleft}Imp\ F{\isadigit{1}}\ F{\isadigit{2}}{\isacharparenright}\isanewline
\ \ \isacommand{then}\isamarkupfalse%
\ \isacommand{show}\isamarkupfalse%
\ {\isacharquery}case\ \isacommand{by}\isamarkupfalse%
\ {\isacharparenleft}simp\ only{\isacharcolon}\ atoms{\isacharunderscore}finite{\isacharunderscore}imp{\isacharparenright}\isanewline
\isacommand{qed}\isamarkupfalse%
%
\endisatagproof
{\isafoldproof}%
%
\isadelimproof
%
\endisadelimproof
%
\begin{isamarkuptext}%
Su demostración automática es la siguiente.%
\end{isamarkuptext}\isamarkuptrue%
\isacommand{lemma}\isamarkupfalse%
\ {\isachardoublequoteopen}finite\ {\isacharparenleft}atoms\ F{\isacharparenright}{\isachardoublequoteclose}\ \isanewline
%
\isadelimproof
\ \ %
\endisadelimproof
%
\isatagproof
\isacommand{by}\isamarkupfalse%
\ {\isacharparenleft}induction\ F{\isacharparenright}\ simp{\isacharunderscore}all%
\endisatagproof
{\isafoldproof}%
%
\isadelimproof
%
\endisadelimproof
%
\isadelimdocument
%
\endisadelimdocument
%
\isatagdocument
%
\isamarkupsection{Subfórmulas%
}
\isamarkuptrue%
%
\endisatagdocument
{\isafolddocument}%
%
\isadelimdocument
%
\endisadelimdocument
%
\begin{isamarkuptext}%
Veamos la noción de subfórmulas.

  \begin{definicion}
  El conjunto de subfórmulas de una fórmula \isa{F}, notada \isa{Subf{\isacharparenleft}F{\isacharparenright}}, se 
  define recursivamente como:
    \begin{itemize}
      \item \isa{{\isacharbraceleft}F{\isacharbraceright}} si \isa{F} es una fórmula atómica.
      \item \isa{{\isacharbraceleft}{\isasymbottom}{\isacharbraceright}} si \isa{F} es \isa{{\isasymbottom}}.
      \item \isa{{\isacharbraceleft}F{\isacharbraceright}\ {\isasymunion}\ Subf{\isacharparenleft}G{\isacharparenright}} si \isa{F} es \isa{{\isasymnot}G}.
      \item \isa{{\isacharbraceleft}F{\isacharbraceright}\ {\isasymunion}\ Subf{\isacharparenleft}G{\isacharparenright}\ {\isasymunion}\ Subf{\isacharparenleft}H{\isacharparenright}} si \isa{F} es \isa{G{\isacharasterisk}H} donde \isa{{\isacharasterisk}} es 
        cualquier conectiva binaria.
    \end{itemize}
  \end{definicion}

  Para proceder a la formalización de Isabelle, seguiremos dos etapas. 
  En primer lugar, definimos la función primitiva recursiva 
  \isa{subformulae}. Esta nos devolverá la lista de todas las 
  subfórmulas de una fórmula original obtenidas recursivamente.%
\end{isamarkuptext}\isamarkuptrue%
\isacommand{primrec}\isamarkupfalse%
\ subformulae\ {\isacharcolon}{\isacharcolon}\ {\isachardoublequoteopen}{\isacharprime}a\ formula\ {\isasymRightarrow}\ {\isacharprime}a\ formula\ list{\isachardoublequoteclose}\ \isakeyword{where}\isanewline
\ \ {\isachardoublequoteopen}subformulae\ {\isacharparenleft}Atom\ k{\isacharparenright}\ {\isacharequal}\ {\isacharbrackleft}Atom\ k{\isacharbrackright}{\isachardoublequoteclose}\ \isanewline
{\isacharbar}\ {\isachardoublequoteopen}subformulae\ {\isasymbottom}\ \ \ \ \ \ \ \ {\isacharequal}\ {\isacharbrackleft}{\isasymbottom}{\isacharbrackright}{\isachardoublequoteclose}\ \isanewline
{\isacharbar}\ {\isachardoublequoteopen}subformulae\ {\isacharparenleft}\isactrlbold {\isasymnot}\ F{\isacharparenright}\ \ \ \ {\isacharequal}\ {\isacharparenleft}\isactrlbold {\isasymnot}\ F{\isacharparenright}\ {\isacharhash}\ subformulae\ F{\isachardoublequoteclose}\ \isanewline
{\isacharbar}\ {\isachardoublequoteopen}subformulae\ {\isacharparenleft}F\ \isactrlbold {\isasymand}\ G{\isacharparenright}\ \ {\isacharequal}\ {\isacharparenleft}F\ \isactrlbold {\isasymand}\ G{\isacharparenright}\ {\isacharhash}\ subformulae\ F\ {\isacharat}\ subformulae\ G{\isachardoublequoteclose}\ \isanewline
{\isacharbar}\ {\isachardoublequoteopen}subformulae\ {\isacharparenleft}F\ \isactrlbold {\isasymor}\ G{\isacharparenright}\ \ {\isacharequal}\ {\isacharparenleft}F\ \isactrlbold {\isasymor}\ G{\isacharparenright}\ {\isacharhash}\ subformulae\ F\ {\isacharat}\ subformulae\ G{\isachardoublequoteclose}\isanewline
{\isacharbar}\ {\isachardoublequoteopen}subformulae\ {\isacharparenleft}F\ \isactrlbold {\isasymrightarrow}\ G{\isacharparenright}\ {\isacharequal}\ {\isacharparenleft}F\ \isactrlbold {\isasymrightarrow}\ G{\isacharparenright}\ {\isacharhash}\ subformulae\ F\ {\isacharat}\ subformulae\ G{\isachardoublequoteclose}%
\begin{isamarkuptext}%
Observemos que, en la definición anterior, \isa{{\isacharhash}} es el operador que 
  añade un elemento al comienzo de una lista y \isa{{\isacharat}} concatena varias 
  listas. Siguiendo con los ejemplos, apliquemos \isa{subformulae} en 
  las distintas fórmulas. En particular, al tratarse de una lista pueden 
  aparecer elementos repetidos como se muestra a continuación.%
\end{isamarkuptext}\isamarkuptrue%
\isacommand{notepad}\isamarkupfalse%
\isanewline
\isakeyword{begin}\isanewline
%
\isadelimproof
\ \ %
\endisadelimproof
%
\isatagproof
\isacommand{fix}\isamarkupfalse%
\ p\ {\isacharcolon}{\isacharcolon}\ {\isacharprime}a\isanewline
\isanewline
\ \ \isacommand{have}\isamarkupfalse%
\ {\isachardoublequoteopen}subformulae\ {\isacharparenleft}Atom\ p{\isacharparenright}\ {\isacharequal}\ {\isacharbrackleft}Atom\ p{\isacharbrackright}{\isachardoublequoteclose}\isanewline
\ \ \ \ \isacommand{by}\isamarkupfalse%
\ simp\isanewline
\isanewline
\ \ \isacommand{have}\isamarkupfalse%
\ {\isachardoublequoteopen}subformulae\ {\isacharparenleft}\isactrlbold {\isasymnot}\ {\isacharparenleft}Atom\ p{\isacharparenright}{\isacharparenright}\ {\isacharequal}\ {\isacharbrackleft}\isactrlbold {\isasymnot}\ {\isacharparenleft}Atom\ p{\isacharparenright}{\isacharcomma}\ Atom\ p{\isacharbrackright}{\isachardoublequoteclose}\isanewline
\ \ \ \ \isacommand{by}\isamarkupfalse%
\ simp\isanewline
\isanewline
\ \ \isacommand{have}\isamarkupfalse%
\ {\isachardoublequoteopen}subformulae\ {\isacharparenleft}{\isacharparenleft}Atom\ p\ \isactrlbold {\isasymrightarrow}\ Atom\ q{\isacharparenright}\ \isactrlbold {\isasymor}\ Atom\ r{\isacharparenright}\ {\isacharequal}\ \isanewline
\ \ \ \ \ \ \ {\isacharbrackleft}{\isacharparenleft}Atom\ p\ \isactrlbold {\isasymrightarrow}\ Atom\ q{\isacharparenright}\ \isactrlbold {\isasymor}\ Atom\ r{\isacharcomma}\ Atom\ p\ \isactrlbold {\isasymrightarrow}\ Atom\ q{\isacharcomma}\ Atom\ p{\isacharcomma}\ \isanewline
\ \ \ \ \ \ \ \ Atom\ q{\isacharcomma}\ Atom\ r{\isacharbrackright}{\isachardoublequoteclose}\isanewline
\ \ \ \ \isacommand{by}\isamarkupfalse%
\ simp\isanewline
\isanewline
\ \ \isacommand{have}\isamarkupfalse%
\ {\isachardoublequoteopen}subformulae\ {\isacharparenleft}Atom\ p\ \isactrlbold {\isasymand}\ {\isasymbottom}{\isacharparenright}\ {\isacharequal}\ {\isacharbrackleft}Atom\ p\ \isactrlbold {\isasymand}\ {\isasymbottom}{\isacharcomma}\ Atom\ p{\isacharcomma}\ {\isasymbottom}{\isacharbrackright}{\isachardoublequoteclose}\isanewline
\ \ \ \ \isacommand{by}\isamarkupfalse%
\ simp\isanewline
\isanewline
\ \ \isacommand{have}\isamarkupfalse%
\ {\isachardoublequoteopen}subformulae\ {\isacharparenleft}Atom\ p\ \isactrlbold {\isasymor}\ Atom\ p{\isacharparenright}\ {\isacharequal}\ \isanewline
\ \ \ \ \ \ \ {\isacharbrackleft}Atom\ p\ \isactrlbold {\isasymor}\ Atom\ p{\isacharcomma}\ Atom\ p{\isacharcomma}\ Atom\ p{\isacharbrackright}{\isachardoublequoteclose}\isanewline
\ \ \ \ \isacommand{by}\isamarkupfalse%
\ simp%
\endisatagproof
{\isafoldproof}%
%
\isadelimproof
\isanewline
%
\endisadelimproof
\isacommand{end}\isamarkupfalse%
%
\begin{isamarkuptext}%
En la segunda etapa de formalización, definimos 
  \isa{setSubformulae}, que convierte al tipo conjunto la lista de 
  subfórmulas anterior.%
\end{isamarkuptext}\isamarkuptrue%
\isacommand{abbreviation}\isamarkupfalse%
\ setSubformulae\ {\isacharcolon}{\isacharcolon}\ {\isachardoublequoteopen}{\isacharprime}a\ formula\ {\isasymRightarrow}\ {\isacharprime}a\ formula\ set{\isachardoublequoteclose}\ \isakeyword{where}\isanewline
\ \ {\isachardoublequoteopen}setSubformulae\ F\ {\isasymequiv}\ set\ {\isacharparenleft}subformulae\ F{\isacharparenright}{\isachardoublequoteclose}%
\begin{isamarkuptext}%
De este modo, la función \isa{setSubformulae} es la formalización
  en Isabelle de \isa{Subf{\isacharparenleft}·{\isacharparenright}}. En Isabelle, primero hemos definido la lista 
  de subfórmulas pues, en algunos casos, es más sencilla la prueba de 
  resultados sobre este tipo. Sin embargo, el tipo de conjuntos facilita
  las pruebas de los resultados de esta sección. Algunas de las
  ventajas del tipo conjuntos son la eliminación de elementos repetidos 
  o las operaciones propias de teoría de conjuntos. Observemos los 
  siguientes ejemplos con el tipo de conjuntos.%
\end{isamarkuptext}\isamarkuptrue%
\isacommand{notepad}\isamarkupfalse%
\isanewline
\isakeyword{begin}\isanewline
%
\isadelimproof
\ \ %
\endisadelimproof
%
\isatagproof
\isacommand{fix}\isamarkupfalse%
\ p\ q\ r\ {\isacharcolon}{\isacharcolon}\ {\isacharprime}a\isanewline
\isanewline
\ \ \isacommand{have}\isamarkupfalse%
\ {\isachardoublequoteopen}setSubformulae\ {\isacharparenleft}Atom\ p\ \isactrlbold {\isasymor}\ Atom\ p{\isacharparenright}\ {\isacharequal}\ {\isacharbraceleft}Atom\ p\ \isactrlbold {\isasymor}\ Atom\ p{\isacharcomma}\ Atom\ p{\isacharbraceright}{\isachardoublequoteclose}\isanewline
\ \ \ \ \isacommand{by}\isamarkupfalse%
\ simp\isanewline
\ \ \isanewline
\ \ \isacommand{have}\isamarkupfalse%
\ {\isachardoublequoteopen}setSubformulae\ {\isacharparenleft}{\isacharparenleft}Atom\ p\ \isactrlbold {\isasymrightarrow}\ Atom\ q{\isacharparenright}\ \isactrlbold {\isasymor}\ Atom\ r{\isacharparenright}\ {\isacharequal}\isanewline
\ \ \ \ \ \ \ \ {\isacharbraceleft}{\isacharparenleft}Atom\ p\ \isactrlbold {\isasymrightarrow}\ Atom\ q{\isacharparenright}\ \isactrlbold {\isasymor}\ Atom\ r{\isacharcomma}\ Atom\ p\ \isactrlbold {\isasymrightarrow}\ Atom\ q{\isacharcomma}\ Atom\ p{\isacharcomma}\ \isanewline
\ \ \ \ \ \ \ \ \ Atom\ q{\isacharcomma}\ Atom\ r{\isacharbraceright}{\isachardoublequoteclose}\isanewline
\ \ \isacommand{by}\isamarkupfalse%
\ auto%
\endisatagproof
{\isafoldproof}%
%
\isadelimproof
\ \ \ \isanewline
%
\endisadelimproof
\isacommand{end}\isamarkupfalse%
%
\begin{isamarkuptext}%
Por otro lado, debemos señalar que el uso de 
  \isa{abbreviation} para definir \isa{setSubformulae} no es 
  arbitrario. Esta elección se debe a que el tipo \isa{abbreviation} 
  se trata de un sinónimo para una expresión cuyo tipo ya existe (en 
  nuestro caso, convertir en conjunto la lista obtenida con 
  \isa{subformulae}). No es una definición propiamente dicha, sino 
  una forma de nombrar la composición de las funciones \isa{set} y 
  \isa{subformulae}.

  En primer lugar, veamos que \isa{setSubformulae} es una
  formalización de \isa{Subf} en Isabelle. Para ello 
  utilizaremos el siguiente resultado sobre listas, probado como sigue.%
\end{isamarkuptext}\isamarkuptrue%
\isacommand{lemma}\isamarkupfalse%
\ set{\isacharunderscore}insert{\isacharcolon}\ {\isachardoublequoteopen}set\ {\isacharparenleft}x\ {\isacharhash}\ ys{\isacharparenright}\ {\isacharequal}\ {\isacharbraceleft}x{\isacharbraceright}\ {\isasymunion}\ set\ ys{\isachardoublequoteclose}\isanewline
%
\isadelimproof
\ \ %
\endisadelimproof
%
\isatagproof
\isacommand{by}\isamarkupfalse%
\ {\isacharparenleft}simp\ only{\isacharcolon}\ list{\isachardot}set{\isacharparenleft}{\isadigit{2}}{\isacharparenright}\ Un{\isacharunderscore}insert{\isacharunderscore}left\ sup{\isacharunderscore}bot{\isachardot}left{\isacharunderscore}neutral{\isacharparenright}%
\endisatagproof
{\isafoldproof}%
%
\isadelimproof
%
\endisadelimproof
%
\begin{isamarkuptext}%
Por tanto, obtenemos la equivalencia como resultado de los 
  siguientes lemas, que aparecen demostrados de manera detallada.%
\end{isamarkuptext}\isamarkuptrue%
\isacommand{lemma}\isamarkupfalse%
\ setSubformulae{\isacharunderscore}atom{\isacharcolon}\isanewline
\ \ {\isachardoublequoteopen}setSubformulae\ {\isacharparenleft}Atom\ p{\isacharparenright}\ {\isacharequal}\ {\isacharbraceleft}Atom\ p{\isacharbraceright}{\isachardoublequoteclose}\isanewline
%
\isadelimproof
\ \ \ \ %
\endisadelimproof
%
\isatagproof
\isacommand{by}\isamarkupfalse%
\ {\isacharparenleft}simp\ only{\isacharcolon}\ subformulae{\isachardot}simps{\isacharparenleft}{\isadigit{1}}{\isacharparenright}\ list{\isachardot}set{\isacharparenright}%
\endisatagproof
{\isafoldproof}%
%
\isadelimproof
\isanewline
%
\endisadelimproof
\isanewline
\isacommand{lemma}\isamarkupfalse%
\ setSubformulae{\isacharunderscore}bot{\isacharcolon}\isanewline
\ \ {\isachardoublequoteopen}setSubformulae\ {\isacharparenleft}{\isasymbottom}{\isacharparenright}\ {\isacharequal}\ {\isacharbraceleft}{\isasymbottom}{\isacharbraceright}{\isachardoublequoteclose}\isanewline
%
\isadelimproof
\ \ \ \ %
\endisadelimproof
%
\isatagproof
\isacommand{by}\isamarkupfalse%
\ {\isacharparenleft}simp\ only{\isacharcolon}\ subformulae{\isachardot}simps{\isacharparenleft}{\isadigit{2}}{\isacharparenright}\ list{\isachardot}set{\isacharparenright}%
\endisatagproof
{\isafoldproof}%
%
\isadelimproof
\isanewline
%
\endisadelimproof
\isanewline
\isacommand{lemma}\isamarkupfalse%
\ setSubformulae{\isacharunderscore}not{\isacharcolon}\isanewline
\ \ \isakeyword{shows}\ {\isachardoublequoteopen}setSubformulae\ {\isacharparenleft}\isactrlbold {\isasymnot}\ F{\isacharparenright}\ {\isacharequal}\ {\isacharbraceleft}\isactrlbold {\isasymnot}\ F{\isacharbraceright}\ {\isasymunion}\ setSubformulae\ F{\isachardoublequoteclose}\isanewline
%
\isadelimproof
%
\endisadelimproof
%
\isatagproof
\isacommand{proof}\isamarkupfalse%
\ {\isacharminus}\isanewline
\ \ \isacommand{have}\isamarkupfalse%
\ {\isachardoublequoteopen}setSubformulae\ {\isacharparenleft}\isactrlbold {\isasymnot}\ F{\isacharparenright}\ {\isacharequal}\ set\ {\isacharparenleft}\isactrlbold {\isasymnot}\ F\ {\isacharhash}\ subformulae\ F{\isacharparenright}{\isachardoublequoteclose}\isanewline
\ \ \ \ \isacommand{by}\isamarkupfalse%
\ {\isacharparenleft}simp\ only{\isacharcolon}\ subformulae{\isachardot}simps{\isacharparenleft}{\isadigit{3}}{\isacharparenright}{\isacharparenright}\isanewline
\ \ \isacommand{also}\isamarkupfalse%
\ \isacommand{have}\isamarkupfalse%
\ {\isachardoublequoteopen}{\isasymdots}\ {\isacharequal}\ {\isacharbraceleft}\isactrlbold {\isasymnot}\ F{\isacharbraceright}\ {\isasymunion}\ set\ {\isacharparenleft}subformulae\ F{\isacharparenright}{\isachardoublequoteclose}\isanewline
\ \ \ \ \isacommand{by}\isamarkupfalse%
\ {\isacharparenleft}simp\ only{\isacharcolon}\ set{\isacharunderscore}insert{\isacharparenright}\isanewline
\ \ \isacommand{finally}\isamarkupfalse%
\ \isacommand{show}\isamarkupfalse%
\ {\isacharquery}thesis\isanewline
\ \ \ \ \isacommand{by}\isamarkupfalse%
\ this\isanewline
\isacommand{qed}\isamarkupfalse%
%
\endisatagproof
{\isafoldproof}%
%
\isadelimproof
\isanewline
%
\endisadelimproof
\isanewline
\isacommand{lemma}\isamarkupfalse%
\ setSubformulae{\isacharunderscore}and{\isacharcolon}\ \isanewline
\ \ {\isachardoublequoteopen}setSubformulae\ {\isacharparenleft}F{\isadigit{1}}\ \isactrlbold {\isasymand}\ F{\isadigit{2}}{\isacharparenright}\ \isanewline
\ \ \ {\isacharequal}\ {\isacharbraceleft}F{\isadigit{1}}\ \isactrlbold {\isasymand}\ F{\isadigit{2}}{\isacharbraceright}\ {\isasymunion}\ {\isacharparenleft}setSubformulae\ F{\isadigit{1}}\ {\isasymunion}\ setSubformulae\ F{\isadigit{2}}{\isacharparenright}{\isachardoublequoteclose}\isanewline
%
\isadelimproof
%
\endisadelimproof
%
\isatagproof
\isacommand{proof}\isamarkupfalse%
\ {\isacharminus}\isanewline
\ \ \isacommand{have}\isamarkupfalse%
\ {\isachardoublequoteopen}setSubformulae\ {\isacharparenleft}F{\isadigit{1}}\ \isactrlbold {\isasymand}\ F{\isadigit{2}}{\isacharparenright}\ \isanewline
\ \ \ \ \ \ \ \ {\isacharequal}\ set\ {\isacharparenleft}{\isacharparenleft}F{\isadigit{1}}\ \isactrlbold {\isasymand}\ F{\isadigit{2}}{\isacharparenright}\ {\isacharhash}\ {\isacharparenleft}subformulae\ F{\isadigit{1}}\ {\isacharat}\ subformulae\ F{\isadigit{2}}{\isacharparenright}{\isacharparenright}{\isachardoublequoteclose}\isanewline
\ \ \ \ \isacommand{by}\isamarkupfalse%
\ {\isacharparenleft}simp\ only{\isacharcolon}\ subformulae{\isachardot}simps{\isacharparenleft}{\isadigit{4}}{\isacharparenright}{\isacharparenright}\isanewline
\ \ \isacommand{also}\isamarkupfalse%
\ \isacommand{have}\isamarkupfalse%
\ {\isachardoublequoteopen}{\isasymdots}\ {\isacharequal}\ {\isacharbraceleft}F{\isadigit{1}}\ \isactrlbold {\isasymand}\ F{\isadigit{2}}{\isacharbraceright}\ {\isasymunion}\ {\isacharparenleft}set\ {\isacharparenleft}subformulae\ F{\isadigit{1}}\ {\isacharat}\ subformulae\ F{\isadigit{2}}{\isacharparenright}{\isacharparenright}{\isachardoublequoteclose}\isanewline
\ \ \ \ \isacommand{by}\isamarkupfalse%
\ {\isacharparenleft}simp\ only{\isacharcolon}\ set{\isacharunderscore}insert{\isacharparenright}\isanewline
\ \ \isacommand{also}\isamarkupfalse%
\ \isacommand{have}\isamarkupfalse%
\ {\isachardoublequoteopen}{\isasymdots}\ {\isacharequal}\ {\isacharbraceleft}F{\isadigit{1}}\ \isactrlbold {\isasymand}\ F{\isadigit{2}}{\isacharbraceright}\ {\isasymunion}\ {\isacharparenleft}setSubformulae\ F{\isadigit{1}}\ {\isasymunion}\ setSubformulae\ F{\isadigit{2}}{\isacharparenright}{\isachardoublequoteclose}\isanewline
\ \ \ \ \isacommand{by}\isamarkupfalse%
\ {\isacharparenleft}simp\ only{\isacharcolon}\ set{\isacharunderscore}append{\isacharparenright}\isanewline
\ \ \isacommand{finally}\isamarkupfalse%
\ \isacommand{show}\isamarkupfalse%
\ {\isacharquery}thesis\isanewline
\ \ \ \ \isacommand{by}\isamarkupfalse%
\ this\isanewline
\isacommand{qed}\isamarkupfalse%
%
\endisatagproof
{\isafoldproof}%
%
\isadelimproof
\isanewline
%
\endisadelimproof
\isanewline
\isacommand{lemma}\isamarkupfalse%
\ setSubformulae{\isacharunderscore}or{\isacharcolon}\ \isanewline
\ \ {\isachardoublequoteopen}setSubformulae\ {\isacharparenleft}F{\isadigit{1}}\ \isactrlbold {\isasymor}\ F{\isadigit{2}}{\isacharparenright}\ \isanewline
\ \ \ {\isacharequal}\ {\isacharbraceleft}F{\isadigit{1}}\ \isactrlbold {\isasymor}\ F{\isadigit{2}}{\isacharbraceright}\ {\isasymunion}\ {\isacharparenleft}setSubformulae\ F{\isadigit{1}}\ {\isasymunion}\ setSubformulae\ F{\isadigit{2}}{\isacharparenright}{\isachardoublequoteclose}\isanewline
%
\isadelimproof
%
\endisadelimproof
%
\isatagproof
\isacommand{proof}\isamarkupfalse%
\ {\isacharminus}\isanewline
\ \ \isacommand{have}\isamarkupfalse%
\ {\isachardoublequoteopen}setSubformulae\ {\isacharparenleft}F{\isadigit{1}}\ \isactrlbold {\isasymor}\ F{\isadigit{2}}{\isacharparenright}\ \isanewline
\ \ \ \ \ \ \ \ {\isacharequal}\ set\ {\isacharparenleft}{\isacharparenleft}F{\isadigit{1}}\ \isactrlbold {\isasymor}\ F{\isadigit{2}}{\isacharparenright}\ {\isacharhash}\ {\isacharparenleft}subformulae\ F{\isadigit{1}}\ {\isacharat}\ subformulae\ F{\isadigit{2}}{\isacharparenright}{\isacharparenright}{\isachardoublequoteclose}\isanewline
\ \ \ \ \isacommand{by}\isamarkupfalse%
\ {\isacharparenleft}simp\ only{\isacharcolon}\ subformulae{\isachardot}simps{\isacharparenleft}{\isadigit{5}}{\isacharparenright}{\isacharparenright}\isanewline
\ \ \isacommand{also}\isamarkupfalse%
\ \isacommand{have}\isamarkupfalse%
\ {\isachardoublequoteopen}{\isasymdots}\ {\isacharequal}\ {\isacharbraceleft}F{\isadigit{1}}\ \isactrlbold {\isasymor}\ F{\isadigit{2}}{\isacharbraceright}\ {\isasymunion}\ {\isacharparenleft}set\ {\isacharparenleft}subformulae\ F{\isadigit{1}}\ {\isacharat}\ subformulae\ F{\isadigit{2}}{\isacharparenright}{\isacharparenright}{\isachardoublequoteclose}\isanewline
\ \ \ \ \isacommand{by}\isamarkupfalse%
\ {\isacharparenleft}simp\ only{\isacharcolon}\ set{\isacharunderscore}insert{\isacharparenright}\isanewline
\ \ \isacommand{also}\isamarkupfalse%
\ \isacommand{have}\isamarkupfalse%
\ {\isachardoublequoteopen}{\isasymdots}\ {\isacharequal}\ {\isacharbraceleft}F{\isadigit{1}}\ \isactrlbold {\isasymor}\ F{\isadigit{2}}{\isacharbraceright}\ {\isasymunion}\ {\isacharparenleft}setSubformulae\ F{\isadigit{1}}\ {\isasymunion}\ setSubformulae\ F{\isadigit{2}}{\isacharparenright}{\isachardoublequoteclose}\isanewline
\ \ \ \ \isacommand{by}\isamarkupfalse%
\ {\isacharparenleft}simp\ only{\isacharcolon}\ set{\isacharunderscore}append{\isacharparenright}\isanewline
\ \ \isacommand{finally}\isamarkupfalse%
\ \isacommand{show}\isamarkupfalse%
\ {\isacharquery}thesis\isanewline
\ \ \ \ \isacommand{by}\isamarkupfalse%
\ this\isanewline
\isacommand{qed}\isamarkupfalse%
%
\endisatagproof
{\isafoldproof}%
%
\isadelimproof
\isanewline
%
\endisadelimproof
\isanewline
\isacommand{lemma}\isamarkupfalse%
\ setSubformulae{\isacharunderscore}imp{\isacharcolon}\ \isanewline
\ \ {\isachardoublequoteopen}setSubformulae\ {\isacharparenleft}F{\isadigit{1}}\ \isactrlbold {\isasymrightarrow}\ F{\isadigit{2}}{\isacharparenright}\ \isanewline
\ \ \ {\isacharequal}\ {\isacharbraceleft}F{\isadigit{1}}\ \isactrlbold {\isasymrightarrow}\ F{\isadigit{2}}{\isacharbraceright}\ {\isasymunion}\ {\isacharparenleft}setSubformulae\ F{\isadigit{1}}\ {\isasymunion}\ setSubformulae\ F{\isadigit{2}}{\isacharparenright}{\isachardoublequoteclose}\isanewline
%
\isadelimproof
%
\endisadelimproof
%
\isatagproof
\isacommand{proof}\isamarkupfalse%
\ {\isacharminus}\isanewline
\ \ \isacommand{have}\isamarkupfalse%
\ {\isachardoublequoteopen}setSubformulae\ {\isacharparenleft}F{\isadigit{1}}\ \isactrlbold {\isasymrightarrow}\ F{\isadigit{2}}{\isacharparenright}\ \isanewline
\ \ \ \ \ \ \ \ {\isacharequal}\ set\ {\isacharparenleft}{\isacharparenleft}F{\isadigit{1}}\ \isactrlbold {\isasymrightarrow}\ F{\isadigit{2}}{\isacharparenright}\ {\isacharhash}\ {\isacharparenleft}subformulae\ F{\isadigit{1}}\ {\isacharat}\ subformulae\ F{\isadigit{2}}{\isacharparenright}{\isacharparenright}{\isachardoublequoteclose}\isanewline
\ \ \ \ \isacommand{by}\isamarkupfalse%
\ {\isacharparenleft}simp\ only{\isacharcolon}\ subformulae{\isachardot}simps{\isacharparenleft}{\isadigit{6}}{\isacharparenright}{\isacharparenright}\isanewline
\ \ \isacommand{also}\isamarkupfalse%
\ \isacommand{have}\isamarkupfalse%
\ {\isachardoublequoteopen}{\isasymdots}\ {\isacharequal}\ {\isacharbraceleft}F{\isadigit{1}}\ \isactrlbold {\isasymrightarrow}\ F{\isadigit{2}}{\isacharbraceright}\ {\isasymunion}\ {\isacharparenleft}set\ {\isacharparenleft}subformulae\ F{\isadigit{1}}\ {\isacharat}\ subformulae\ F{\isadigit{2}}{\isacharparenright}{\isacharparenright}{\isachardoublequoteclose}\isanewline
\ \ \ \ \isacommand{by}\isamarkupfalse%
\ {\isacharparenleft}simp\ only{\isacharcolon}\ set{\isacharunderscore}insert{\isacharparenright}\isanewline
\ \ \isacommand{also}\isamarkupfalse%
\ \isacommand{have}\isamarkupfalse%
\ {\isachardoublequoteopen}{\isasymdots}\ {\isacharequal}\ {\isacharbraceleft}F{\isadigit{1}}\ \isactrlbold {\isasymrightarrow}\ F{\isadigit{2}}{\isacharbraceright}\ {\isasymunion}\ {\isacharparenleft}setSubformulae\ F{\isadigit{1}}\ {\isasymunion}\ setSubformulae\ F{\isadigit{2}}{\isacharparenright}{\isachardoublequoteclose}\isanewline
\ \ \ \ \isacommand{by}\isamarkupfalse%
\ {\isacharparenleft}simp\ only{\isacharcolon}\ set{\isacharunderscore}append{\isacharparenright}\isanewline
\ \ \isacommand{finally}\isamarkupfalse%
\ \isacommand{show}\isamarkupfalse%
\ {\isacharquery}thesis\isanewline
\ \ \ \ \isacommand{by}\isamarkupfalse%
\ this\isanewline
\isacommand{qed}\isamarkupfalse%
%
\endisatagproof
{\isafoldproof}%
%
\isadelimproof
%
\endisadelimproof
%
\begin{isamarkuptext}%
Una vez probada la equivalencia, comencemos con los resultados 
  correspondientes a las subfórmulas. En primer lugar, tenemos la 
  siguiente propiedad como consecuencia directa de la equivalencia de 
  funciones anterior.

  \comentario{Reescribir el siguiente enunciado y demostración.}

  \begin{lema}
    \isa{F\ {\isasymin}\ Subf{\isacharparenleft}F{\isacharparenright}}.
  \end{lema}

  \begin{demostracion}
    Por inducción en la estructura de las fórmulas. Se tienen los
    siguientes casos:
  
    Sea \isa{p} fórmula atómica cualquiera. Por definición de \isa{Subf} tenemos 
    que \isa{Subf{\isacharparenleft}p{\isacharparenright}\ {\isacharequal}\ {\isacharbraceleft}p{\isacharbraceright}}, luego se tiene la propiedad.
  
    Sea la fórmula \isa{{\isasymbottom}}. Como \isa{Subf{\isacharparenleft}{\isasymbottom}{\isacharparenright}\ {\isacharequal}\ {\isacharbraceleft}{\isasymbottom}{\isacharbraceright}}, se verifica el resultado.

    Por definición del conjunto de subfórmulas de \isa{Subf{\isacharparenleft}{\isasymnot}\ F{\isacharparenright}} se tiene 
    la propiedad para este caso, pues 
    \isa{Subf{\isacharparenleft}{\isasymnot}\ F{\isacharparenright}\ {\isacharequal}\ {\isacharbraceleft}{\isasymnot}\ F{\isacharbraceright}\ {\isasymunion}\ Subf{\isacharparenleft}F{\isacharparenright}\ {\isasymLongrightarrow}\ {\isasymnot}\ F\ {\isasymin}\ Subf{\isacharparenleft}{\isasymnot}\ F{\isacharparenright}} como queríamos 
    ver.

    Análogamente, para cualquier conectiva binaria \isa{{\isacharasterisk}} y fórmulas \isa{F} y 
    \isa{G} se cumple \isa{Subf{\isacharparenleft}F{\isacharasterisk}G{\isacharparenright}\ {\isacharequal}\ {\isacharbraceleft}F{\isacharasterisk}G{\isacharbraceright}\ {\isasymunion}\ Subf{\isacharparenleft}F{\isacharparenright}\ {\isasymunion}\ Subf{\isacharparenleft}G{\isacharparenright}}, luego se 
    cumple la propiedad.
  \end{demostracion}

  Formalicemos ahora el lema con su correspondiente demostración 
  detallada.%
\end{isamarkuptext}\isamarkuptrue%
\isacommand{lemma}\isamarkupfalse%
\ subformulae{\isacharunderscore}self{\isacharcolon}\ {\isachardoublequoteopen}F\ {\isasymin}\ setSubformulae\ F{\isachardoublequoteclose}\isanewline
%
\isadelimproof
%
\endisadelimproof
%
\isatagproof
\isacommand{proof}\isamarkupfalse%
\ {\isacharparenleft}induction\ F{\isacharparenright}\ \isanewline
\ \ \isacommand{case}\isamarkupfalse%
\ {\isacharparenleft}Atom\ x{\isacharparenright}\ \isanewline
\ \ \isacommand{then}\isamarkupfalse%
\ \isacommand{show}\isamarkupfalse%
\ {\isacharquery}case\ \isanewline
\ \ \ \ \isacommand{by}\isamarkupfalse%
\ {\isacharparenleft}simp\ only{\isacharcolon}\ singletonI\ setSubformulae{\isacharunderscore}atom{\isacharparenright}\isanewline
\isacommand{next}\isamarkupfalse%
\isanewline
\ \ \isacommand{case}\isamarkupfalse%
\ Bot\isanewline
\ \ \isacommand{then}\isamarkupfalse%
\ \isacommand{show}\isamarkupfalse%
\ {\isacharquery}case\ \isanewline
\ \ \ \ \isacommand{by}\isamarkupfalse%
\ {\isacharparenleft}simp\ only{\isacharcolon}\ singletonI\ setSubformulae{\isacharunderscore}bot{\isacharparenright}\isanewline
\isacommand{next}\isamarkupfalse%
\isanewline
\ \ \isacommand{case}\isamarkupfalse%
\ {\isacharparenleft}Not\ F{\isacharparenright}\isanewline
\ \ \isacommand{then}\isamarkupfalse%
\ \isacommand{show}\isamarkupfalse%
\ {\isacharquery}case\ \isanewline
\ \ \ \ \isacommand{by}\isamarkupfalse%
\ {\isacharparenleft}simp\ add{\isacharcolon}\ insertI{\isadigit{1}}\ setSubformulae{\isacharunderscore}not{\isacharparenright}\isanewline
\isacommand{next}\isamarkupfalse%
\isanewline
\isacommand{case}\isamarkupfalse%
\ {\isacharparenleft}And\ F{\isadigit{1}}\ F{\isadigit{2}}{\isacharparenright}\isanewline
\ \ \isacommand{then}\isamarkupfalse%
\ \isacommand{show}\isamarkupfalse%
\ {\isacharquery}case\ \isanewline
\ \ \ \ \isacommand{by}\isamarkupfalse%
\ {\isacharparenleft}simp\ add{\isacharcolon}\ insertI{\isadigit{1}}\ setSubformulae{\isacharunderscore}and{\isacharparenright}\isanewline
\isacommand{next}\isamarkupfalse%
\isanewline
\isacommand{case}\isamarkupfalse%
\ {\isacharparenleft}Or\ F{\isadigit{1}}\ F{\isadigit{2}}{\isacharparenright}\isanewline
\ \ \isacommand{then}\isamarkupfalse%
\ \isacommand{show}\isamarkupfalse%
\ {\isacharquery}case\ \isanewline
\ \ \ \ \isacommand{by}\isamarkupfalse%
\ {\isacharparenleft}simp\ add{\isacharcolon}\ insertI{\isadigit{1}}\ setSubformulae{\isacharunderscore}or{\isacharparenright}\isanewline
\isacommand{next}\isamarkupfalse%
\isanewline
\isacommand{case}\isamarkupfalse%
\ {\isacharparenleft}Imp\ F{\isadigit{1}}\ F{\isadigit{2}}{\isacharparenright}\isanewline
\ \ \isacommand{then}\isamarkupfalse%
\ \isacommand{show}\isamarkupfalse%
\ {\isacharquery}case\ \isanewline
\ \ \ \ \isacommand{by}\isamarkupfalse%
\ {\isacharparenleft}simp\ add{\isacharcolon}\ insertI{\isadigit{1}}\ setSubformulae{\isacharunderscore}imp{\isacharparenright}\isanewline
\isacommand{qed}\isamarkupfalse%
%
\endisatagproof
{\isafoldproof}%
%
\isadelimproof
%
\endisadelimproof
%
\begin{isamarkuptext}%
La demostración automática es la siguiente.%
\end{isamarkuptext}\isamarkuptrue%
\isacommand{lemma}\isamarkupfalse%
\ {\isachardoublequoteopen}F\ {\isasymin}\ setSubformulae\ F{\isachardoublequoteclose}\isanewline
%
\isadelimproof
\ \ %
\endisadelimproof
%
\isatagproof
\isacommand{by}\isamarkupfalse%
\ {\isacharparenleft}induction\ F{\isacharparenright}\ simp{\isacharunderscore}all%
\endisatagproof
{\isafoldproof}%
%
\isadelimproof
%
\endisadelimproof
%
\begin{isamarkuptext}%
Procedamos con los demás resultados de la sección. Como hemos 
  señalado con anterioridad, utilizaremos varias propiedades de 
  conjuntos pertenecientes a la teoría 
  \href{https://n9.cl/qatp}{Set.thy} de Isabelle, que apareceran en 
  el glosario final. 

  Además, definiremos dos reglas adicionales que utilizaremos con 
  frecuencia.%
\end{isamarkuptext}\isamarkuptrue%
\isacommand{lemma}\isamarkupfalse%
\ subContUnionRev{\isadigit{1}}{\isacharcolon}\ \isanewline
\ \ \isakeyword{assumes}\ {\isachardoublequoteopen}A\ {\isasymunion}\ B\ {\isasymsubseteq}\ C{\isachardoublequoteclose}\ \isanewline
\ \ \isakeyword{shows}\ \ \ {\isachardoublequoteopen}A\ {\isasymsubseteq}\ C{\isachardoublequoteclose}\isanewline
%
\isadelimproof
%
\endisadelimproof
%
\isatagproof
\isacommand{proof}\isamarkupfalse%
\ {\isacharminus}\isanewline
\ \ \isacommand{have}\isamarkupfalse%
\ {\isachardoublequoteopen}A\ {\isasymsubseteq}\ C\ {\isasymand}\ B\ {\isasymsubseteq}\ C{\isachardoublequoteclose}\isanewline
\ \ \ \ \isacommand{using}\isamarkupfalse%
\ assms\isanewline
\ \ \ \ \isacommand{by}\isamarkupfalse%
\ {\isacharparenleft}simp\ only{\isacharcolon}\ sup{\isachardot}bounded{\isacharunderscore}iff{\isacharparenright}\isanewline
\ \ \isacommand{then}\isamarkupfalse%
\ \isacommand{show}\isamarkupfalse%
\ {\isachardoublequoteopen}A\ {\isasymsubseteq}\ C{\isachardoublequoteclose}\isanewline
\ \ \ \ \isacommand{by}\isamarkupfalse%
\ {\isacharparenleft}rule\ conjunct{\isadigit{1}}{\isacharparenright}\isanewline
\isacommand{qed}\isamarkupfalse%
%
\endisatagproof
{\isafoldproof}%
%
\isadelimproof
\isanewline
%
\endisadelimproof
\isanewline
\isacommand{lemma}\isamarkupfalse%
\ subContUnionRev{\isadigit{2}}{\isacharcolon}\ \isanewline
\ \ \isakeyword{assumes}\ {\isachardoublequoteopen}A\ {\isasymunion}\ B\ {\isasymsubseteq}\ C{\isachardoublequoteclose}\ \isanewline
\ \ \isakeyword{shows}\ \ \ {\isachardoublequoteopen}B\ {\isasymsubseteq}\ C{\isachardoublequoteclose}\isanewline
%
\isadelimproof
%
\endisadelimproof
%
\isatagproof
\isacommand{proof}\isamarkupfalse%
\ {\isacharminus}\isanewline
\ \ \isacommand{have}\isamarkupfalse%
\ {\isachardoublequoteopen}A\ {\isasymsubseteq}\ C\ {\isasymand}\ B\ {\isasymsubseteq}\ C{\isachardoublequoteclose}\isanewline
\ \ \ \ \isacommand{using}\isamarkupfalse%
\ assms\isanewline
\ \ \ \ \isacommand{by}\isamarkupfalse%
\ {\isacharparenleft}simp\ only{\isacharcolon}\ sup{\isachardot}bounded{\isacharunderscore}iff{\isacharparenright}\isanewline
\ \ \isacommand{then}\isamarkupfalse%
\ \isacommand{show}\isamarkupfalse%
\ {\isachardoublequoteopen}B\ {\isasymsubseteq}\ C{\isachardoublequoteclose}\isanewline
\ \ \ \ \isacommand{by}\isamarkupfalse%
\ {\isacharparenleft}rule\ conjunct{\isadigit{2}}{\isacharparenright}\isanewline
\isacommand{qed}\isamarkupfalse%
%
\endisatagproof
{\isafoldproof}%
%
\isadelimproof
%
\endisadelimproof
%
\begin{isamarkuptext}%
Sus correspondientes demostraciones automáticas se muestran a 
  continuación.%
\end{isamarkuptext}\isamarkuptrue%
\isacommand{lemma}\isamarkupfalse%
\ {\isachardoublequoteopen}A\ {\isasymunion}\ B\ {\isasymsubseteq}\ C\ {\isasymLongrightarrow}\ A\ {\isasymsubseteq}\ C{\isachardoublequoteclose}\isanewline
%
\isadelimproof
\ \ %
\endisadelimproof
%
\isatagproof
\isacommand{by}\isamarkupfalse%
\ simp%
\endisatagproof
{\isafoldproof}%
%
\isadelimproof
\isanewline
%
\endisadelimproof
\isanewline
\isacommand{lemma}\isamarkupfalse%
\ {\isachardoublequoteopen}A\ {\isasymunion}\ B\ {\isasymsubseteq}\ C\ {\isasymLongrightarrow}\ B\ {\isasymsubseteq}\ C{\isachardoublequoteclose}\isanewline
%
\isadelimproof
\ \ %
\endisadelimproof
%
\isatagproof
\isacommand{by}\isamarkupfalse%
\ simp%
\endisatagproof
{\isafoldproof}%
%
\isadelimproof
%
\endisadelimproof
%
\begin{isamarkuptext}%
Veamos ahora los distintos resultados sobre subfórmulas.

  \comentario{Reescribir el siguiente enunciado y su demostración.}

  \begin{lema}
    Sean \isa{F} una fórmula proposicional y \isa{A\isactrlsub F} el conjunto de las 
    fórmulas atómicas formadas a partir de cada elemento del conjunto 
    de variables proposicionales de \isa{F}. 
    Entonces, \isa{A\isactrlsub F\ {\isasymsubseteq}\ Subf{\isacharparenleft}F{\isacharparenright}}.

    Por tanto, las fórmulas atómicas son subfórmulas.
  \end{lema}

  \begin{demostracion}
    La prueba seguirá el esquema inductivo para la estructura de 
    fórmulas. Veamos cada caso:
  
    Consideremos la fórmula atómica \isa{p} cualquiera. Entonces, su
    conjunto de átomos es \isa{{\isacharbraceleft}p{\isacharbraceright}}. De este modo, el conjunto \isa{A\isactrlsub p} 
    correspondiente será \isa{A\isactrlsub p\ {\isacharequal}\ {\isacharbraceleft}p{\isacharbraceright}\ {\isasymsubseteq}\ {\isacharbraceleft}p{\isacharbraceright}\ {\isacharequal}\ Subf{\isacharparenleft}Atom\ p{\isacharparenright}} como 
    queríamos 
    demostrar.

    Sea la fórmula \isa{{\isasymbottom}}. Como su connjunto de átomos es vacío, es claro 
    que \isa{A\isactrlsub {\isasymbottom}\ {\isacharequal}\ {\isasymemptyset}\ {\isasymsubseteq}\ Subf{\isacharparenleft}{\isasymbottom}{\isacharparenright}\ {\isacharequal}\ {\isasymemptyset}}.

    Sea la fórmula \isa{F} tal que \isa{A\isactrlsub F\ {\isasymsubseteq}\ Subf{\isacharparenleft}F{\isacharparenright}}. Probemos el resultado 
    para \isa{{\isasymnot}\ F}. Por definición tenemos que los conjunto de variables 
    proposicionales de \isa{F} y \isa{{\isasymnot}\ F} coinciden, luego \isa{A\isactrlsub {\isasymnot}\isactrlsub F\ {\isacharequal}\ A\isactrlsub F}. Además, 
    \isa{Subf{\isacharparenleft}{\isasymnot}\ F{\isacharparenright}\ {\isacharequal}\ {\isacharbraceleft}{\isasymnot}\ F{\isacharbraceright}\ {\isasymunion}\ Subf{\isacharparenleft}F{\isacharparenright}}. Por tanto, por hipótesis de 
    inducción tenemos:
    \isa{A\isactrlsub {\isasymnot}\isactrlsub F\ {\isacharequal}\ A\isactrlsub F\ {\isasymsubseteq}\ Subf{\isacharparenleft}F{\isacharparenright}\ {\isasymsubseteq}\ {\isacharbraceleft}{\isasymnot}\ F{\isacharbraceright}\ {\isasymunion}\ Subf{\isacharparenleft}F{\isacharparenright}\ {\isacharequal}\ Subf{\isacharparenleft}{\isasymnot}\ F{\isacharparenright}}, luego
    \isa{A\isactrlsub {\isasymnot}\isactrlsub F\ {\isasymsubseteq}\ Subf{\isacharparenleft}{\isasymnot}\ F{\isacharparenright}}.

    Sean las fórmulas \isa{F} y \isa{G} tales que \isa{A\isactrlsub F\ {\isasymsubseteq}\ Subf{\isacharparenleft}F{\isacharparenright}} y 
    \isa{A\isactrlsub G\ {\isasymsubseteq}\ Subf{\isacharparenleft}G{\isacharparenright}}. Probemos ahora \isa{A\isactrlsub F\isactrlsub {\isacharasterisk}\isactrlsub G\ {\isasymsubseteq}\ Subf{\isacharparenleft}F{\isacharasterisk}G{\isacharparenright}} para cualquier 
    conectiva binaria \isa{{\isacharasterisk}}. Por un lado, el conjunto de átomos de \isa{F{\isacharasterisk}G}
    es la unión de sus correspondientes conjuntos de átomos, luego 
    \isa{A\isactrlsub F\isactrlsub {\isacharasterisk}\isactrlsub G\ {\isacharequal}\ A\isactrlsub F\ {\isasymunion}\ A\isactrlsub G}. Por tanto, por hipótesis de inducción y definición 
    del conjunto de subfórmulas, se tiene:
    \isa{A\isactrlsub F\isactrlsub {\isacharasterisk}\isactrlsub G\ {\isacharequal}\ A\isactrlsub F\ {\isasymunion}\ A\isactrlsub G\ {\isasymsubseteq}\ Subf{\isacharparenleft}F{\isacharparenright}\ {\isasymunion}\ Subf{\isacharparenleft}G{\isacharparenright}\ {\isasymsubseteq}\ {\isacharbraceleft}F{\isacharasterisk}G{\isacharbraceright}\ {\isasymunion}\ Subf{\isacharparenleft}F{\isacharparenright}\ {\isasymunion}\ Subf{\isacharparenleft}G{\isacharparenright}\ {\isacharequal}\ Subf{\isacharparenleft}F{\isacharasterisk}G{\isacharparenright}}
    Luego, \isa{A\isactrlsub F\isactrlsub {\isacharasterisk}\isactrlsub G\ {\isasymsubseteq}\ Subf{\isacharparenleft}F{\isacharasterisk}G{\isacharparenright}} como queríamos demostrar.  
  \end{demostracion}

  En Isabelle, se especifica como sigue.%
\end{isamarkuptext}\isamarkuptrue%
\isacommand{lemma}\isamarkupfalse%
\ {\isachardoublequoteopen}Atom\ {\isacharbackquote}\ atoms\ F\ {\isasymsubseteq}\ setSubformulae\ F{\isachardoublequoteclose}\isanewline
%
\isadelimproof
\ \ %
\endisadelimproof
%
\isatagproof
\isacommand{oops}\isamarkupfalse%
%
\endisatagproof
{\isafoldproof}%
%
\isadelimproof
%
\endisadelimproof
%
\begin{isamarkuptext}%
Debemos observar que \isa{Atom\ {\isacharbackquote}\ atoms\ F} construye las fórmulas 
  atómicas a partir de cada uno de los elementos de \isa{atoms\ F}, creando 
  un conjunto de fórmulas atómicas. Dicho conjunto es equivalente al 
  conjunto \isa{A\isactrlsub F} del enunciado del lema. Para ello emplea el infijo \isa{{\isacharbackquote}} 
  definido como notación abreviada de \isa{{\isacharparenleft}{\isacharbackquote}{\isacharparenright}} que calcula la 
  imagen de un conjunto en la teoría \href{https://n9.cl/qatp}{Set.thy}.

  \begin{itemize}
    \item[] \isa{f\ {\isacharbackquote}\ A\ {\isacharequal}\ {\isacharbraceleft}y\ {\isacharbar}\ {\isasymexists}x{\isasymin}A{\isachardot}\ y\ {\isacharequal}\ f\ x{\isacharbraceright}} 
      \hfill (\isa{image{\isacharunderscore}def})
  \end{itemize}

  Para aclarar su funcionamiento, veamos ejemplos para distintos casos 
  de fórmulas.%
\end{isamarkuptext}\isamarkuptrue%
\isacommand{notepad}\isamarkupfalse%
\isanewline
\isakeyword{begin}\isanewline
%
\isadelimproof
\ \ %
\endisadelimproof
%
\isatagproof
\isacommand{fix}\isamarkupfalse%
\ p\ q\ r\ {\isacharcolon}{\isacharcolon}\ {\isacharprime}a\isanewline
\isanewline
\ \ \isacommand{have}\isamarkupfalse%
\ {\isachardoublequoteopen}Atom\ {\isacharbackquote}atoms\ {\isacharparenleft}Atom\ p\ \isactrlbold {\isasymor}\ {\isasymbottom}{\isacharparenright}\ {\isacharequal}\ {\isacharbraceleft}Atom\ p{\isacharbraceright}{\isachardoublequoteclose}\isanewline
\ \ \ \ \isacommand{by}\isamarkupfalse%
\ simp\isanewline
\isanewline
\ \ \isacommand{have}\isamarkupfalse%
\ {\isachardoublequoteopen}Atom\ {\isacharbackquote}atoms\ {\isacharparenleft}{\isacharparenleft}Atom\ p\ \isactrlbold {\isasymrightarrow}\ Atom\ q{\isacharparenright}\ \isactrlbold {\isasymor}\ Atom\ r{\isacharparenright}\ {\isacharequal}\ \isanewline
\ \ \ \ \ \ \ {\isacharbraceleft}Atom\ p{\isacharcomma}\ Atom\ q{\isacharcomma}\ Atom\ r{\isacharbraceright}{\isachardoublequoteclose}\isanewline
\ \ \ \ \isacommand{by}\isamarkupfalse%
\ auto\ \isanewline
\isanewline
\ \ \isacommand{have}\isamarkupfalse%
\ {\isachardoublequoteopen}Atom\ {\isacharbackquote}atoms\ {\isacharparenleft}{\isacharparenleft}Atom\ p\ \isactrlbold {\isasymrightarrow}\ Atom\ p{\isacharparenright}\ \isactrlbold {\isasymor}\ Atom\ r{\isacharparenright}\ {\isacharequal}\ \isanewline
\ \ \ \ \ \ \ {\isacharbraceleft}Atom\ p{\isacharcomma}\ Atom\ r{\isacharbraceright}{\isachardoublequoteclose}\isanewline
\ \ \ \ \isacommand{by}\isamarkupfalse%
\ auto%
\endisatagproof
{\isafoldproof}%
%
\isadelimproof
\isanewline
%
\endisadelimproof
\isacommand{end}\isamarkupfalse%
%
\begin{isamarkuptext}%
Además, esta función tiene las siguientes propiedades sobre 
  conjuntos que utilizaremos en la demostración.

  \begin{itemize}
    \item[] \isa{f\ {\isacharbackquote}\ {\isacharparenleft}A\ {\isasymunion}\ B{\isacharparenright}\ {\isacharequal}\ f\ {\isacharbackquote}\ A\ {\isasymunion}\ f\ {\isacharbackquote}\ B} 
      \hfill (\isa{image{\isacharunderscore}Un})
    \item[] \isa{f\ {\isacharbackquote}\ {\isacharparenleft}{\isacharbraceleft}a{\isacharbraceright}\ {\isasymunion}\ B{\isacharparenright}\ {\isacharequal}\ {\isacharbraceleft}f\ a{\isacharbraceright}\ {\isasymunion}\ f\ {\isacharbackquote}\ B} 
      \hfill (\isa{image{\isacharunderscore}insert})
    \item[] \isa{f\ {\isacharbackquote}\ {\isasymemptyset}\ {\isacharequal}\ {\isasymemptyset}} 
      \hfill (\isa{image{\isacharunderscore}empty})
  \end{itemize}

  Una vez hechas las aclaraciones necesarias, comencemos la demostración 
  estructurada. Esta seguirá el esquema inductivo señalado con 
  anterioridad.%
\end{isamarkuptext}\isamarkuptrue%
\isacommand{lemma}\isamarkupfalse%
\ atoms{\isacharunderscore}are{\isacharunderscore}subformulae{\isacharunderscore}atom{\isacharcolon}\ \isanewline
\ \ {\isachardoublequoteopen}Atom\ {\isacharbackquote}\ atoms\ {\isacharparenleft}Atom\ x{\isacharparenright}\ {\isasymsubseteq}\ setSubformulae\ {\isacharparenleft}Atom\ x{\isacharparenright}{\isachardoublequoteclose}\ \isanewline
%
\isadelimproof
%
\endisadelimproof
%
\isatagproof
\isacommand{proof}\isamarkupfalse%
\ {\isacharminus}\isanewline
\ \ \isacommand{have}\isamarkupfalse%
\ {\isachardoublequoteopen}Atom\ {\isacharbackquote}\ atoms\ {\isacharparenleft}Atom\ x{\isacharparenright}\ {\isacharequal}\ Atom\ {\isacharbackquote}\ {\isacharbraceleft}x{\isacharbraceright}{\isachardoublequoteclose}\isanewline
\ \ \ \ \isacommand{by}\isamarkupfalse%
\ {\isacharparenleft}simp\ only{\isacharcolon}\ formula{\isachardot}set{\isacharparenleft}{\isadigit{1}}{\isacharparenright}{\isacharparenright}\isanewline
\ \ \isacommand{also}\isamarkupfalse%
\ \isacommand{have}\isamarkupfalse%
\ {\isachardoublequoteopen}{\isasymdots}\ {\isacharequal}\ {\isacharbraceleft}Atom\ x{\isacharbraceright}{\isachardoublequoteclose}\isanewline
\ \ \ \ \isacommand{by}\isamarkupfalse%
\ {\isacharparenleft}simp\ only{\isacharcolon}\ image{\isacharunderscore}insert\ image{\isacharunderscore}empty{\isacharparenright}\isanewline
\ \ \isacommand{also}\isamarkupfalse%
\ \isacommand{have}\isamarkupfalse%
\ {\isachardoublequoteopen}{\isasymdots}\ {\isacharequal}\ set\ {\isacharbrackleft}Atom\ x{\isacharbrackright}{\isachardoublequoteclose}\isanewline
\ \ \ \ \isacommand{by}\isamarkupfalse%
\ {\isacharparenleft}simp\ only{\isacharcolon}\ list{\isachardot}set{\isacharparenleft}{\isadigit{1}}{\isacharparenright}\ list{\isachardot}set{\isacharparenleft}{\isadigit{2}}{\isacharparenright}{\isacharparenright}\isanewline
\ \ \isacommand{also}\isamarkupfalse%
\ \isacommand{have}\isamarkupfalse%
\ {\isachardoublequoteopen}{\isasymdots}\ {\isacharequal}\ set\ {\isacharparenleft}subformulae\ {\isacharparenleft}Atom\ x{\isacharparenright}{\isacharparenright}{\isachardoublequoteclose}\isanewline
\ \ \ \ \isacommand{by}\isamarkupfalse%
\ {\isacharparenleft}simp\ only{\isacharcolon}\ subformulae{\isachardot}simps{\isacharparenleft}{\isadigit{1}}{\isacharparenright}{\isacharparenright}\isanewline
\ \ \isacommand{finally}\isamarkupfalse%
\ \isacommand{have}\isamarkupfalse%
\ {\isachardoublequoteopen}Atom\ {\isacharbackquote}\ atoms\ {\isacharparenleft}Atom\ x{\isacharparenright}\ {\isacharequal}\ set\ {\isacharparenleft}subformulae\ {\isacharparenleft}Atom\ x{\isacharparenright}{\isacharparenright}{\isachardoublequoteclose}\isanewline
\ \ \ \ \isacommand{by}\isamarkupfalse%
\ this\isanewline
\ \ \isacommand{then}\isamarkupfalse%
\ \isacommand{show}\isamarkupfalse%
\ {\isacharquery}thesis\ \isanewline
\ \ \ \ \isacommand{by}\isamarkupfalse%
\ {\isacharparenleft}simp\ only{\isacharcolon}\ subset{\isacharunderscore}refl{\isacharparenright}\isanewline
\isacommand{qed}\isamarkupfalse%
%
\endisatagproof
{\isafoldproof}%
%
\isadelimproof
\isanewline
%
\endisadelimproof
\isanewline
\isacommand{lemma}\isamarkupfalse%
\ atoms{\isacharunderscore}are{\isacharunderscore}subformulae{\isacharunderscore}bot{\isacharcolon}\ \isanewline
\ \ {\isachardoublequoteopen}Atom\ {\isacharbackquote}\ atoms\ {\isasymbottom}\ {\isasymsubseteq}\ setSubformulae\ {\isasymbottom}{\isachardoublequoteclose}\ \ \isanewline
%
\isadelimproof
%
\endisadelimproof
%
\isatagproof
\isacommand{proof}\isamarkupfalse%
\ {\isacharminus}\isanewline
\ \ \isacommand{have}\isamarkupfalse%
\ {\isachardoublequoteopen}Atom\ {\isacharbackquote}\ atoms\ {\isasymbottom}\ {\isacharequal}\ Atom\ {\isacharbackquote}\ {\isasymemptyset}{\isachardoublequoteclose}\isanewline
\ \ \ \ \isacommand{by}\isamarkupfalse%
\ {\isacharparenleft}simp\ only{\isacharcolon}\ formula{\isachardot}set{\isacharparenleft}{\isadigit{2}}{\isacharparenright}{\isacharparenright}\isanewline
\ \ \isacommand{also}\isamarkupfalse%
\ \isacommand{have}\isamarkupfalse%
\ {\isachardoublequoteopen}{\isasymdots}\ {\isacharequal}\ {\isasymemptyset}{\isachardoublequoteclose}\isanewline
\ \ \ \ \isacommand{by}\isamarkupfalse%
\ {\isacharparenleft}simp\ only{\isacharcolon}\ image{\isacharunderscore}empty{\isacharparenright}\isanewline
\ \ \isacommand{also}\isamarkupfalse%
\ \isacommand{have}\isamarkupfalse%
\ {\isachardoublequoteopen}{\isasymdots}\ {\isasymsubseteq}\ setSubformulae\ {\isasymbottom}{\isachardoublequoteclose}\isanewline
\ \ \ \ \isacommand{by}\isamarkupfalse%
\ {\isacharparenleft}simp\ only{\isacharcolon}\ empty{\isacharunderscore}subsetI{\isacharparenright}\isanewline
\ \ \isacommand{finally}\isamarkupfalse%
\ \isacommand{show}\isamarkupfalse%
\ {\isacharquery}thesis\isanewline
\ \ \ \ \isacommand{by}\isamarkupfalse%
\ this\isanewline
\isacommand{qed}\isamarkupfalse%
%
\endisatagproof
{\isafoldproof}%
%
\isadelimproof
\isanewline
%
\endisadelimproof
\isanewline
\isacommand{lemma}\isamarkupfalse%
\ atoms{\isacharunderscore}are{\isacharunderscore}subformulae{\isacharunderscore}not{\isacharcolon}\ \isanewline
\ \ \isakeyword{assumes}\ {\isachardoublequoteopen}Atom\ {\isacharbackquote}\ atoms\ F\ {\isasymsubseteq}\ setSubformulae\ F{\isachardoublequoteclose}\ \isanewline
\ \ \isakeyword{shows}\ \ \ {\isachardoublequoteopen}Atom\ {\isacharbackquote}\ atoms\ {\isacharparenleft}\isactrlbold {\isasymnot}\ F{\isacharparenright}\ {\isasymsubseteq}\ setSubformulae\ {\isacharparenleft}\isactrlbold {\isasymnot}\ F{\isacharparenright}{\isachardoublequoteclose}\isanewline
%
\isadelimproof
%
\endisadelimproof
%
\isatagproof
\isacommand{proof}\isamarkupfalse%
\ {\isacharminus}\isanewline
\ \ \isacommand{have}\isamarkupfalse%
\ {\isachardoublequoteopen}Atom\ {\isacharbackquote}\ atoms\ {\isacharparenleft}\isactrlbold {\isasymnot}\ F{\isacharparenright}\ {\isacharequal}\ Atom\ {\isacharbackquote}\ atoms\ F{\isachardoublequoteclose}\isanewline
\ \ \ \ \isacommand{by}\isamarkupfalse%
\ {\isacharparenleft}simp\ only{\isacharcolon}\ formula{\isachardot}set{\isacharparenleft}{\isadigit{3}}{\isacharparenright}{\isacharparenright}\isanewline
\ \ \isacommand{also}\isamarkupfalse%
\ \isacommand{have}\isamarkupfalse%
\ {\isachardoublequoteopen}{\isasymdots}\ {\isasymsubseteq}\ setSubformulae\ F{\isachardoublequoteclose}\isanewline
\ \ \ \ \isacommand{by}\isamarkupfalse%
\ {\isacharparenleft}simp\ only{\isacharcolon}\ assms{\isacharparenright}\isanewline
\ \ \isacommand{also}\isamarkupfalse%
\ \isacommand{have}\isamarkupfalse%
\ {\isachardoublequoteopen}{\isasymdots}\ {\isasymsubseteq}\ {\isacharbraceleft}\isactrlbold {\isasymnot}\ F{\isacharbraceright}\ {\isasymunion}\ setSubformulae\ F{\isachardoublequoteclose}\isanewline
\ \ \ \ \isacommand{by}\isamarkupfalse%
\ {\isacharparenleft}simp\ only{\isacharcolon}\ Un{\isacharunderscore}upper{\isadigit{2}}{\isacharparenright}\isanewline
\ \ \isacommand{also}\isamarkupfalse%
\ \isacommand{have}\isamarkupfalse%
\ {\isachardoublequoteopen}{\isasymdots}\ {\isacharequal}\ setSubformulae\ {\isacharparenleft}\isactrlbold {\isasymnot}\ F{\isacharparenright}{\isachardoublequoteclose}\isanewline
\ \ \ \ \isacommand{by}\isamarkupfalse%
\ {\isacharparenleft}simp\ only{\isacharcolon}\ setSubformulae{\isacharunderscore}not{\isacharparenright}\isanewline
\ \ \isacommand{finally}\isamarkupfalse%
\ \isacommand{show}\isamarkupfalse%
\ {\isacharquery}thesis\isanewline
\ \ \ \ \isacommand{by}\isamarkupfalse%
\ this\isanewline
\isacommand{qed}\isamarkupfalse%
%
\endisatagproof
{\isafoldproof}%
%
\isadelimproof
\isanewline
%
\endisadelimproof
\isanewline
\isacommand{lemma}\isamarkupfalse%
\ atoms{\isacharunderscore}are{\isacharunderscore}subformulae{\isacharunderscore}and{\isacharcolon}\ \isanewline
\ \ \isakeyword{assumes}\ {\isachardoublequoteopen}Atom\ {\isacharbackquote}\ atoms\ F{\isadigit{1}}\ {\isasymsubseteq}\ setSubformulae\ F{\isadigit{1}}{\isachardoublequoteclose}\isanewline
\ \ \ \ \ \ \ \ \ \ {\isachardoublequoteopen}Atom\ {\isacharbackquote}\ atoms\ F{\isadigit{2}}\ {\isasymsubseteq}\ setSubformulae\ F{\isadigit{2}}{\isachardoublequoteclose}\isanewline
\ \ \isakeyword{shows}\ \ \ {\isachardoublequoteopen}Atom\ {\isacharbackquote}\ atoms\ {\isacharparenleft}F{\isadigit{1}}\ \isactrlbold {\isasymand}\ F{\isadigit{2}}{\isacharparenright}\ {\isasymsubseteq}\ setSubformulae\ {\isacharparenleft}F{\isadigit{1}}\ \isactrlbold {\isasymand}\ F{\isadigit{2}}{\isacharparenright}{\isachardoublequoteclose}\isanewline
%
\isadelimproof
%
\endisadelimproof
%
\isatagproof
\isacommand{proof}\isamarkupfalse%
\ {\isacharminus}\isanewline
\ \ \isacommand{have}\isamarkupfalse%
\ {\isachardoublequoteopen}Atom\ {\isacharbackquote}\ atoms\ {\isacharparenleft}F{\isadigit{1}}\ \isactrlbold {\isasymand}\ F{\isadigit{2}}{\isacharparenright}\ {\isacharequal}\ Atom\ {\isacharbackquote}\ {\isacharparenleft}atoms\ F{\isadigit{1}}\ {\isasymunion}\ atoms\ F{\isadigit{2}}{\isacharparenright}{\isachardoublequoteclose}\isanewline
\ \ \ \ \isacommand{by}\isamarkupfalse%
\ {\isacharparenleft}simp\ only{\isacharcolon}\ formula{\isachardot}set{\isacharparenleft}{\isadigit{4}}{\isacharparenright}{\isacharparenright}\isanewline
\ \ \isacommand{also}\isamarkupfalse%
\ \isacommand{have}\isamarkupfalse%
\ {\isachardoublequoteopen}{\isasymdots}\ {\isacharequal}\ Atom\ {\isacharbackquote}\ atoms\ F{\isadigit{1}}\ {\isasymunion}\ Atom\ {\isacharbackquote}\ atoms\ F{\isadigit{2}}{\isachardoublequoteclose}\ \isanewline
\ \ \ \ \isacommand{by}\isamarkupfalse%
\ {\isacharparenleft}rule\ image{\isacharunderscore}Un{\isacharparenright}\isanewline
\ \ \isacommand{also}\isamarkupfalse%
\ \isacommand{have}\isamarkupfalse%
\ {\isachardoublequoteopen}{\isasymdots}\ {\isasymsubseteq}\ setSubformulae\ F{\isadigit{1}}\ {\isasymunion}\ setSubformulae\ F{\isadigit{2}}{\isachardoublequoteclose}\isanewline
\ \ \ \ \isacommand{using}\isamarkupfalse%
\ assms\isanewline
\ \ \ \ \isacommand{by}\isamarkupfalse%
\ {\isacharparenleft}rule\ Un{\isacharunderscore}mono{\isacharparenright}\isanewline
\ \ \isacommand{also}\isamarkupfalse%
\ \isacommand{have}\isamarkupfalse%
\ {\isachardoublequoteopen}{\isasymdots}\ {\isasymsubseteq}\ {\isacharbraceleft}F{\isadigit{1}}\ \isactrlbold {\isasymand}\ F{\isadigit{2}}{\isacharbraceright}\ {\isasymunion}\ {\isacharparenleft}setSubformulae\ F{\isadigit{1}}\ {\isasymunion}\ setSubformulae\ F{\isadigit{2}}{\isacharparenright}{\isachardoublequoteclose}\isanewline
\ \ \ \ \isacommand{by}\isamarkupfalse%
\ {\isacharparenleft}simp\ only{\isacharcolon}\ Un{\isacharunderscore}upper{\isadigit{2}}{\isacharparenright}\isanewline
\ \ \isacommand{also}\isamarkupfalse%
\ \isacommand{have}\isamarkupfalse%
\ {\isachardoublequoteopen}{\isasymdots}\ {\isacharequal}\ setSubformulae\ {\isacharparenleft}F{\isadigit{1}}\ \isactrlbold {\isasymand}\ F{\isadigit{2}}{\isacharparenright}{\isachardoublequoteclose}\isanewline
\ \ \ \ \isacommand{by}\isamarkupfalse%
\ {\isacharparenleft}simp\ only{\isacharcolon}\ setSubformulae{\isacharunderscore}and{\isacharparenright}\isanewline
\ \ \isacommand{finally}\isamarkupfalse%
\ \isacommand{show}\isamarkupfalse%
\ {\isacharquery}thesis\isanewline
\ \ \ \ \isacommand{by}\isamarkupfalse%
\ this\isanewline
\isacommand{qed}\isamarkupfalse%
%
\endisatagproof
{\isafoldproof}%
%
\isadelimproof
\isanewline
%
\endisadelimproof
\isanewline
\isacommand{lemma}\isamarkupfalse%
\ atoms{\isacharunderscore}are{\isacharunderscore}subformulae{\isacharunderscore}or{\isacharcolon}\ \isanewline
\ \ \isakeyword{assumes}\ {\isachardoublequoteopen}Atom\ {\isacharbackquote}\ atoms\ F{\isadigit{1}}\ {\isasymsubseteq}\ setSubformulae\ F{\isadigit{1}}{\isachardoublequoteclose}\isanewline
\ \ \ \ \ \ \ \ \ \ {\isachardoublequoteopen}Atom\ {\isacharbackquote}\ atoms\ F{\isadigit{2}}\ {\isasymsubseteq}\ setSubformulae\ F{\isadigit{2}}{\isachardoublequoteclose}\isanewline
\ \ \isakeyword{shows}\ \ \ {\isachardoublequoteopen}Atom\ {\isacharbackquote}\ atoms\ {\isacharparenleft}F{\isadigit{1}}\ \isactrlbold {\isasymor}\ F{\isadigit{2}}{\isacharparenright}\ {\isasymsubseteq}\ setSubformulae\ {\isacharparenleft}F{\isadigit{1}}\ \isactrlbold {\isasymor}\ F{\isadigit{2}}{\isacharparenright}{\isachardoublequoteclose}\isanewline
%
\isadelimproof
%
\endisadelimproof
%
\isatagproof
\isacommand{proof}\isamarkupfalse%
\ {\isacharminus}\isanewline
\ \ \isacommand{have}\isamarkupfalse%
\ {\isachardoublequoteopen}Atom\ {\isacharbackquote}\ atoms\ {\isacharparenleft}F{\isadigit{1}}\ \isactrlbold {\isasymor}\ F{\isadigit{2}}{\isacharparenright}\ {\isacharequal}\ Atom\ {\isacharbackquote}\ {\isacharparenleft}atoms\ F{\isadigit{1}}\ {\isasymunion}\ atoms\ F{\isadigit{2}}{\isacharparenright}{\isachardoublequoteclose}\isanewline
\ \ \ \ \isacommand{by}\isamarkupfalse%
\ {\isacharparenleft}simp\ only{\isacharcolon}\ formula{\isachardot}set{\isacharparenleft}{\isadigit{5}}{\isacharparenright}{\isacharparenright}\isanewline
\ \ \isacommand{also}\isamarkupfalse%
\ \isacommand{have}\isamarkupfalse%
\ {\isachardoublequoteopen}{\isasymdots}\ {\isacharequal}\ Atom\ {\isacharbackquote}\ atoms\ F{\isadigit{1}}\ {\isasymunion}\ Atom\ {\isacharbackquote}\ atoms\ F{\isadigit{2}}{\isachardoublequoteclose}\ \isanewline
\ \ \ \ \isacommand{by}\isamarkupfalse%
\ {\isacharparenleft}rule\ image{\isacharunderscore}Un{\isacharparenright}\isanewline
\ \ \isacommand{also}\isamarkupfalse%
\ \isacommand{have}\isamarkupfalse%
\ {\isachardoublequoteopen}{\isasymdots}\ {\isasymsubseteq}\ setSubformulae\ F{\isadigit{1}}\ {\isasymunion}\ setSubformulae\ F{\isadigit{2}}{\isachardoublequoteclose}\isanewline
\ \ \ \ \isacommand{using}\isamarkupfalse%
\ assms\isanewline
\ \ \ \ \isacommand{by}\isamarkupfalse%
\ {\isacharparenleft}rule\ Un{\isacharunderscore}mono{\isacharparenright}\isanewline
\ \ \isacommand{also}\isamarkupfalse%
\ \isacommand{have}\isamarkupfalse%
\ {\isachardoublequoteopen}{\isasymdots}\ {\isasymsubseteq}\ {\isacharbraceleft}F{\isadigit{1}}\ \isactrlbold {\isasymor}\ F{\isadigit{2}}{\isacharbraceright}\ {\isasymunion}\ {\isacharparenleft}setSubformulae\ F{\isadigit{1}}\ {\isasymunion}\ setSubformulae\ F{\isadigit{2}}{\isacharparenright}{\isachardoublequoteclose}\isanewline
\ \ \ \ \isacommand{by}\isamarkupfalse%
\ {\isacharparenleft}simp\ only{\isacharcolon}\ Un{\isacharunderscore}upper{\isadigit{2}}{\isacharparenright}\isanewline
\ \ \isacommand{also}\isamarkupfalse%
\ \isacommand{have}\isamarkupfalse%
\ {\isachardoublequoteopen}{\isasymdots}\ {\isacharequal}\ setSubformulae\ {\isacharparenleft}F{\isadigit{1}}\ \isactrlbold {\isasymor}\ F{\isadigit{2}}{\isacharparenright}{\isachardoublequoteclose}\isanewline
\ \ \ \ \isacommand{by}\isamarkupfalse%
\ {\isacharparenleft}simp\ only{\isacharcolon}\ setSubformulae{\isacharunderscore}or{\isacharparenright}\isanewline
\ \ \isacommand{finally}\isamarkupfalse%
\ \isacommand{show}\isamarkupfalse%
\ {\isacharquery}thesis\isanewline
\ \ \ \ \isacommand{by}\isamarkupfalse%
\ this\isanewline
\isacommand{qed}\isamarkupfalse%
%
\endisatagproof
{\isafoldproof}%
%
\isadelimproof
\isanewline
%
\endisadelimproof
\isanewline
\isacommand{lemma}\isamarkupfalse%
\ atoms{\isacharunderscore}are{\isacharunderscore}subformulae{\isacharunderscore}imp{\isacharcolon}\ \isanewline
\ \ \isakeyword{assumes}\ {\isachardoublequoteopen}Atom\ {\isacharbackquote}\ atoms\ F{\isadigit{1}}\ {\isasymsubseteq}\ setSubformulae\ F{\isadigit{1}}{\isachardoublequoteclose}\isanewline
\ \ \ \ \ \ \ \ \ \ {\isachardoublequoteopen}Atom\ {\isacharbackquote}\ atoms\ F{\isadigit{2}}\ {\isasymsubseteq}\ setSubformulae\ F{\isadigit{2}}{\isachardoublequoteclose}\isanewline
\ \ \isakeyword{shows}\ \ \ {\isachardoublequoteopen}Atom\ {\isacharbackquote}\ atoms\ {\isacharparenleft}F{\isadigit{1}}\ \isactrlbold {\isasymrightarrow}\ F{\isadigit{2}}{\isacharparenright}\ {\isasymsubseteq}\ setSubformulae\ {\isacharparenleft}F{\isadigit{1}}\ \isactrlbold {\isasymrightarrow}\ F{\isadigit{2}}{\isacharparenright}{\isachardoublequoteclose}\isanewline
%
\isadelimproof
%
\endisadelimproof
%
\isatagproof
\isacommand{proof}\isamarkupfalse%
\ {\isacharminus}\isanewline
\ \ \isacommand{have}\isamarkupfalse%
\ {\isachardoublequoteopen}Atom\ {\isacharbackquote}\ atoms\ {\isacharparenleft}F{\isadigit{1}}\ \isactrlbold {\isasymrightarrow}\ F{\isadigit{2}}{\isacharparenright}\ {\isacharequal}\ Atom\ {\isacharbackquote}\ {\isacharparenleft}atoms\ F{\isadigit{1}}\ {\isasymunion}\ atoms\ F{\isadigit{2}}{\isacharparenright}{\isachardoublequoteclose}\isanewline
\ \ \ \ \isacommand{by}\isamarkupfalse%
\ {\isacharparenleft}simp\ only{\isacharcolon}\ formula{\isachardot}set{\isacharparenleft}{\isadigit{6}}{\isacharparenright}{\isacharparenright}\isanewline
\ \ \isacommand{also}\isamarkupfalse%
\ \isacommand{have}\isamarkupfalse%
\ {\isachardoublequoteopen}{\isasymdots}\ {\isacharequal}\ Atom\ {\isacharbackquote}\ atoms\ F{\isadigit{1}}\ {\isasymunion}\ Atom\ {\isacharbackquote}\ atoms\ F{\isadigit{2}}{\isachardoublequoteclose}\ \isanewline
\ \ \ \ \isacommand{by}\isamarkupfalse%
\ {\isacharparenleft}rule\ image{\isacharunderscore}Un{\isacharparenright}\isanewline
\ \ \isacommand{also}\isamarkupfalse%
\ \isacommand{have}\isamarkupfalse%
\ {\isachardoublequoteopen}{\isasymdots}\ {\isasymsubseteq}\ setSubformulae\ F{\isadigit{1}}\ {\isasymunion}\ setSubformulae\ F{\isadigit{2}}{\isachardoublequoteclose}\isanewline
\ \ \ \ \isacommand{using}\isamarkupfalse%
\ assms\isanewline
\ \ \ \ \isacommand{by}\isamarkupfalse%
\ {\isacharparenleft}rule\ Un{\isacharunderscore}mono{\isacharparenright}\isanewline
\ \ \isacommand{also}\isamarkupfalse%
\ \isacommand{have}\isamarkupfalse%
\ {\isachardoublequoteopen}{\isasymdots}\ {\isasymsubseteq}\ {\isacharbraceleft}F{\isadigit{1}}\ \isactrlbold {\isasymrightarrow}\ F{\isadigit{2}}{\isacharbraceright}\ {\isasymunion}\ {\isacharparenleft}setSubformulae\ F{\isadigit{1}}\ {\isasymunion}\ setSubformulae\ F{\isadigit{2}}{\isacharparenright}{\isachardoublequoteclose}\isanewline
\ \ \ \ \isacommand{by}\isamarkupfalse%
\ {\isacharparenleft}simp\ only{\isacharcolon}\ Un{\isacharunderscore}upper{\isadigit{2}}{\isacharparenright}\isanewline
\ \ \isacommand{also}\isamarkupfalse%
\ \isacommand{have}\isamarkupfalse%
\ {\isachardoublequoteopen}{\isasymdots}\ {\isacharequal}\ setSubformulae\ {\isacharparenleft}F{\isadigit{1}}\ \isactrlbold {\isasymrightarrow}\ F{\isadigit{2}}{\isacharparenright}{\isachardoublequoteclose}\isanewline
\ \ \ \ \isacommand{by}\isamarkupfalse%
\ {\isacharparenleft}simp\ only{\isacharcolon}\ setSubformulae{\isacharunderscore}imp{\isacharparenright}\isanewline
\ \ \isacommand{finally}\isamarkupfalse%
\ \isacommand{show}\isamarkupfalse%
\ {\isacharquery}thesis\isanewline
\ \ \ \ \isacommand{by}\isamarkupfalse%
\ this\isanewline
\isacommand{qed}\isamarkupfalse%
%
\endisatagproof
{\isafoldproof}%
%
\isadelimproof
\isanewline
%
\endisadelimproof
\isanewline
\isacommand{lemma}\isamarkupfalse%
\ atoms{\isacharunderscore}are{\isacharunderscore}subformulae{\isacharcolon}\ \isanewline
\ \ {\isachardoublequoteopen}Atom\ {\isacharbackquote}\ atoms\ F\ {\isasymsubseteq}\ setSubformulae\ F{\isachardoublequoteclose}\isanewline
%
\isadelimproof
%
\endisadelimproof
%
\isatagproof
\isacommand{proof}\isamarkupfalse%
\ {\isacharparenleft}induction\ F{\isacharparenright}\isanewline
\ \ \isacommand{case}\isamarkupfalse%
\ {\isacharparenleft}Atom\ x{\isacharparenright}\isanewline
\ \ \isacommand{then}\isamarkupfalse%
\ \isacommand{show}\isamarkupfalse%
\ {\isacharquery}case\ \isacommand{by}\isamarkupfalse%
\ {\isacharparenleft}simp\ only{\isacharcolon}\ atoms{\isacharunderscore}are{\isacharunderscore}subformulae{\isacharunderscore}atom{\isacharparenright}\ \isanewline
\isacommand{next}\isamarkupfalse%
\isanewline
\ \ \isacommand{case}\isamarkupfalse%
\ Bot\isanewline
\ \ \isacommand{then}\isamarkupfalse%
\ \isacommand{show}\isamarkupfalse%
\ {\isacharquery}case\ \isacommand{by}\isamarkupfalse%
\ {\isacharparenleft}simp\ only{\isacharcolon}\ atoms{\isacharunderscore}are{\isacharunderscore}subformulae{\isacharunderscore}bot{\isacharparenright}\ \isanewline
\isacommand{next}\isamarkupfalse%
\isanewline
\ \ \isacommand{case}\isamarkupfalse%
\ {\isacharparenleft}Not\ F{\isacharparenright}\isanewline
\ \ \isacommand{then}\isamarkupfalse%
\ \isacommand{show}\isamarkupfalse%
\ {\isacharquery}case\ \isacommand{by}\isamarkupfalse%
\ {\isacharparenleft}simp\ only{\isacharcolon}\ atoms{\isacharunderscore}are{\isacharunderscore}subformulae{\isacharunderscore}not{\isacharparenright}\ \isanewline
\isacommand{next}\isamarkupfalse%
\isanewline
\ \ \isacommand{case}\isamarkupfalse%
\ {\isacharparenleft}And\ F{\isadigit{1}}\ F{\isadigit{2}}{\isacharparenright}\isanewline
\ \ \isacommand{then}\isamarkupfalse%
\ \isacommand{show}\isamarkupfalse%
\ {\isacharquery}case\ \isacommand{by}\isamarkupfalse%
\ {\isacharparenleft}simp\ only{\isacharcolon}\ atoms{\isacharunderscore}are{\isacharunderscore}subformulae{\isacharunderscore}and{\isacharparenright}\ \isanewline
\isacommand{next}\isamarkupfalse%
\isanewline
\ \ \isacommand{case}\isamarkupfalse%
\ {\isacharparenleft}Or\ F{\isadigit{1}}\ F{\isadigit{2}}{\isacharparenright}\isanewline
\ \ \isacommand{then}\isamarkupfalse%
\ \isacommand{show}\isamarkupfalse%
\ {\isacharquery}case\ \isacommand{by}\isamarkupfalse%
\ {\isacharparenleft}simp\ only{\isacharcolon}\ atoms{\isacharunderscore}are{\isacharunderscore}subformulae{\isacharunderscore}or{\isacharparenright}\isanewline
\isacommand{next}\isamarkupfalse%
\isanewline
\ \ \isacommand{case}\isamarkupfalse%
\ {\isacharparenleft}Imp\ F{\isadigit{1}}\ F{\isadigit{2}}{\isacharparenright}\isanewline
\ \ \isacommand{then}\isamarkupfalse%
\ \isacommand{show}\isamarkupfalse%
\ {\isacharquery}case\ \isacommand{by}\isamarkupfalse%
\ {\isacharparenleft}simp\ only{\isacharcolon}\ atoms{\isacharunderscore}are{\isacharunderscore}subformulae{\isacharunderscore}imp{\isacharparenright}\isanewline
\isacommand{qed}\isamarkupfalse%
%
\endisatagproof
{\isafoldproof}%
%
\isadelimproof
%
\endisadelimproof
%
\begin{isamarkuptext}%
La demostración automática queda igualmente expuesta a 
  continuación.%
\end{isamarkuptext}\isamarkuptrue%
\isacommand{lemma}\isamarkupfalse%
\ {\isachardoublequoteopen}Atom\ {\isacharbackquote}\ atoms\ F\ {\isasymsubseteq}\ setSubformulae\ F{\isachardoublequoteclose}\isanewline
%
\isadelimproof
\ \ %
\endisadelimproof
%
\isatagproof
\isacommand{by}\isamarkupfalse%
\ {\isacharparenleft}induction\ F{\isacharparenright}\ \ auto%
\endisatagproof
{\isafoldproof}%
%
\isadelimproof
%
\endisadelimproof
%
\begin{isamarkuptext}%
La siguiente propiedad declara que el conjunto de átomos de una 
  subfórmula está contenido en el conjunto de átomos de la propia 
  fórmula.
  \begin{lema}
    Sea \isa{G\ {\isasymin}\ Subf{\isacharparenleft}F{\isacharparenright}}, entonces el conjunto de átomos de \isa{G} está
    contenido en el de \isa{F}.
  \end{lema}

  \begin{demostracion}
  Procedemos mediante inducción en la estructura de las fórmulas según 
  los distintos casos:

  Sea \isa{p} una fórmula atómica cualquiera. Si \isa{G\ {\isasymin}\ Subf{\isacharparenleft}p{\isacharparenright}}, 
  como su conjunto de variables es \isa{{\isacharbraceleft}p{\isacharbraceright}}, se tiene \isa{G\ {\isacharequal}\ p}. 
  Por tanto, se tiene el resultado.

  Sea la fórmula \isa{{\isasymbottom}}. Si \isa{G\ {\isasymin}\ Subf{\isacharparenleft}{\isasymbottom}{\isacharparenright}}, como  su conjunto de átomos es
  \isa{{\isacharbraceleft}{\isasymbottom}{\isacharbraceright}}, se tiene \isa{G\ {\isacharequal}\ {\isasymbottom}}. Por tanto, se cumple la propiedad.

  Sea la fórmula \isa{F} cualquiera tal que para cualquier subfórmula 
  \isa{G\ {\isasymin}\ Subf{\isacharparenleft}F{\isacharparenright}} se verifica que el conjunto de átomos de \isa{G} está 
  contenido en el de \isa{F}. Supongamos \isa{G{\isacharprime}\ {\isasymin}\ Subf{\isacharparenleft}{\isasymnot}\ F{\isacharparenright}} cualquiera, 
  probemos que \isa{conjAtoms{\isacharparenleft}G{\isacharprime}{\isacharparenright}\ {\isasymsubseteq}\ conjAtoms{\isacharparenleft}{\isasymnot}\ F{\isacharparenright}}.
  Por definición, tenemos que \isa{Subf{\isacharparenleft}{\isasymnot}\ F{\isacharparenright}\ {\isacharequal}\ {\isacharbraceleft}{\isasymnot}\ F{\isacharbraceright}\ {\isasymunion}\ Subf{\isacharparenleft}F{\isacharparenright}}. De este 
  modo, tenemos dos opciones:
  \isa{G{\isacharprime}\ {\isasymin}\ {\isacharbraceleft}{\isasymnot}\ F{\isacharbraceright}} o \isa{G{\isacharprime}\ {\isasymin}\ Subf{\isacharparenleft}F{\isacharparenright}}. Del primer caso se deduce \isa{G{\isacharprime}\ {\isacharequal}\ {\isasymnot}\ F} 
  y, por tanto, se verifica el resultado. Observando el segundo caso, 
  por hipótesis de inducción, se tiene que el conjunto de átomos de \isa{G{\isacharprime}}
  está contenido en el de \isa{F}. Además, como el conjunto de átomos de 
  \isa{F} y \isa{{\isasymnot}\ F} coinciden, se verifica el resultado.

  Sea \isa{F{\isadigit{1}}} fórmula proposicional tal que para cualquier \isa{G\ {\isasymin}\ Subf{\isacharparenleft}F{\isadigit{1}}{\isacharparenright}} 
  se tiene que el conjunto de átomos de \isa{G} está contenido en el de 
  \isa{F{\isadigit{1}}}. Sea también \isa{F{\isadigit{2}}} tal que dada \isa{G\ {\isasymin}\ Subf{\isacharparenleft}F{\isadigit{2}}{\isacharparenright}} cualquiera se 
  verifica también la hipótesis de inducción en su caso. Supongamos 
  \isa{G{\isacharprime}\ {\isasymin}\ Subf{\isacharparenleft}F{\isadigit{1}}{\isacharasterisk}F{\isadigit{2}}{\isacharparenright}} donde \isa{{\isacharasterisk}} es cualquier conectiva binaria. Vamos a 
  probar que el conjunto de átomos de \isa{G} está contenido en el de 
  \isa{F{\isadigit{1}}{\isacharasterisk}F{\isadigit{2}}}.

  En primer lugar, como 
  \isa{Subf{\isacharparenleft}F{\isadigit{1}}{\isacharasterisk}F{\isadigit{2}}{\isacharparenright}\ {\isacharequal}\ {\isacharbraceleft}F{\isadigit{1}}{\isacharasterisk}F{\isadigit{2}}{\isacharbraceright}\ {\isasymunion}\ {\isacharparenleft}Subf{\isacharparenleft}F{\isadigit{1}}{\isacharparenright}\ {\isasymunion}\ Subf{\isacharparenleft}F{\isadigit{2}}{\isacharparenright}{\isacharparenright}}, se desglosan tres
  casos posibles para \isa{G{\isacharprime}}:
  Si \isa{G{\isacharprime}\ {\isasymin}\ {\isacharbraceleft}F{\isadigit{1}}{\isacharasterisk}F{\isadigit{2}}{\isacharbraceright}}, entonces \isa{G{\isacharprime}\ {\isacharequal}\ F{\isadigit{1}}{\isacharasterisk}F{\isadigit{2}}} y se tiene la propiedad.
  Si \isa{G{\isacharprime}\ {\isasymin}\ Subf{\isacharparenleft}F{\isadigit{1}}{\isacharparenright}\ {\isasymunion}\ Subf{\isacharparenleft}F{\isadigit{2}}{\isacharparenright}}, entonces por propiedades de 
  conjuntos:
  \isa{G{\isacharprime}\ {\isasymin}\ Subf{\isacharparenleft}F{\isadigit{1}}{\isacharparenright}} o \isa{G{\isacharprime}\ {\isasymin}\ Subf{\isacharparenleft}F{\isadigit{2}}{\isacharparenright}}. Si \isa{G{\isacharprime}\ {\isasymin}\ Subf{\isacharparenleft}F{\isadigit{1}}{\isacharparenright}}, por hipótesis 
  de inducción se tiene el resultado. Como el conjunto de átomos de
  \isa{F{\isadigit{1}}{\isacharasterisk}F{\isadigit{2}}} es la unión de los conjuntos de átomos de \isa{F{\isadigit{1}}} y \isa{F{\isadigit{2}}}, se 
  obtiene el resultado como consecuencia de la transitividad de 
  contención para conjuntos. El caso \isa{G{\isacharprime}\ {\isasymin}\ Subf{\isacharparenleft}F{\isadigit{2}}{\isacharparenright}} se demuestra de la 
  misma forma.      
  \end{demostracion}

  Formalizado en Isabelle:%
\end{isamarkuptext}\isamarkuptrue%
\isacommand{lemma}\isamarkupfalse%
\ {\isachardoublequoteopen}G\ {\isasymin}\ setSubformulae\ F\ {\isasymLongrightarrow}\ atoms\ G\ {\isasymsubseteq}\ atoms\ F{\isachardoublequoteclose}\isanewline
%
\isadelimproof
\ \ %
\endisadelimproof
%
\isatagproof
\isacommand{oops}\isamarkupfalse%
%
\endisatagproof
{\isafoldproof}%
%
\isadelimproof
%
\endisadelimproof
%
\begin{isamarkuptext}%
Veamos su demostración estructurada.%
\end{isamarkuptext}\isamarkuptrue%
\isacommand{lemma}\isamarkupfalse%
\ subformulas{\isacharunderscore}atoms{\isacharunderscore}atom{\isacharcolon}\isanewline
\ \ \isakeyword{assumes}\ {\isachardoublequoteopen}G\ {\isasymin}\ setSubformulae\ {\isacharparenleft}Atom\ x{\isacharparenright}{\isachardoublequoteclose}\ \isanewline
\ \ \isakeyword{shows}\ \ \ {\isachardoublequoteopen}atoms\ G\ {\isasymsubseteq}\ atoms\ {\isacharparenleft}Atom\ x{\isacharparenright}{\isachardoublequoteclose}\isanewline
%
\isadelimproof
%
\endisadelimproof
%
\isatagproof
\isacommand{proof}\isamarkupfalse%
\ {\isacharminus}\isanewline
\ \ \isacommand{have}\isamarkupfalse%
\ {\isachardoublequoteopen}G\ {\isasymin}\ {\isacharbraceleft}Atom\ x{\isacharbraceright}{\isachardoublequoteclose}\isanewline
\ \ \ \ \isacommand{using}\isamarkupfalse%
\ assms\isanewline
\ \ \ \ \isacommand{by}\isamarkupfalse%
\ {\isacharparenleft}simp\ only{\isacharcolon}\ setSubformulae{\isacharunderscore}atom{\isacharparenright}\isanewline
\ \ \isacommand{then}\isamarkupfalse%
\ \isacommand{have}\isamarkupfalse%
\ {\isachardoublequoteopen}G\ {\isacharequal}\ Atom\ x{\isachardoublequoteclose}\isanewline
\ \ \ \ \isacommand{by}\isamarkupfalse%
\ {\isacharparenleft}simp\ only{\isacharcolon}\ singletonD{\isacharparenright}\isanewline
\ \ \isacommand{then}\isamarkupfalse%
\ \isacommand{show}\isamarkupfalse%
\ {\isacharquery}thesis\isanewline
\ \ \ \ \isacommand{by}\isamarkupfalse%
\ {\isacharparenleft}simp\ only{\isacharcolon}\ subset{\isacharunderscore}refl{\isacharparenright}\isanewline
\isacommand{qed}\isamarkupfalse%
%
\endisatagproof
{\isafoldproof}%
%
\isadelimproof
\isanewline
%
\endisadelimproof
\isanewline
\isacommand{lemma}\isamarkupfalse%
\ subformulas{\isacharunderscore}atoms{\isacharunderscore}bot{\isacharcolon}\isanewline
\ \ \isakeyword{assumes}\ {\isachardoublequoteopen}G\ {\isasymin}\ setSubformulae\ {\isasymbottom}{\isachardoublequoteclose}\ \isanewline
\ \ \isakeyword{shows}\ \ \ {\isachardoublequoteopen}atoms\ G\ {\isasymsubseteq}\ atoms\ {\isasymbottom}{\isachardoublequoteclose}\isanewline
%
\isadelimproof
%
\endisadelimproof
%
\isatagproof
\isacommand{proof}\isamarkupfalse%
\ {\isacharminus}\isanewline
\ \ \isacommand{have}\isamarkupfalse%
\ {\isachardoublequoteopen}G\ {\isasymin}\ {\isacharbraceleft}{\isasymbottom}{\isacharbraceright}{\isachardoublequoteclose}\isanewline
\ \ \ \ \isacommand{using}\isamarkupfalse%
\ assms\isanewline
\ \ \ \ \isacommand{by}\isamarkupfalse%
\ {\isacharparenleft}simp\ only{\isacharcolon}\ setSubformulae{\isacharunderscore}bot{\isacharparenright}\isanewline
\ \ \isacommand{then}\isamarkupfalse%
\ \isacommand{have}\isamarkupfalse%
\ {\isachardoublequoteopen}G\ {\isacharequal}\ {\isasymbottom}{\isachardoublequoteclose}\isanewline
\ \ \ \ \isacommand{by}\isamarkupfalse%
\ {\isacharparenleft}simp\ only{\isacharcolon}\ singletonD{\isacharparenright}\isanewline
\ \ \isacommand{then}\isamarkupfalse%
\ \isacommand{show}\isamarkupfalse%
\ {\isacharquery}thesis\isanewline
\ \ \ \ \isacommand{by}\isamarkupfalse%
\ {\isacharparenleft}simp\ only{\isacharcolon}\ subset{\isacharunderscore}refl{\isacharparenright}\isanewline
\isacommand{qed}\isamarkupfalse%
%
\endisatagproof
{\isafoldproof}%
%
\isadelimproof
\isanewline
%
\endisadelimproof
\isanewline
\isacommand{lemma}\isamarkupfalse%
\ subformulas{\isacharunderscore}atoms{\isacharunderscore}not{\isacharcolon}\isanewline
\ \ \isakeyword{assumes}\ {\isachardoublequoteopen}G\ {\isasymin}\ setSubformulae\ F\ {\isasymLongrightarrow}\ atoms\ G\ {\isasymsubseteq}\ atoms\ F{\isachardoublequoteclose}\isanewline
\ \ \ \ \ \ \ \ \ \ {\isachardoublequoteopen}G\ {\isasymin}\ setSubformulae\ {\isacharparenleft}\isactrlbold {\isasymnot}\ F{\isacharparenright}{\isachardoublequoteclose}\isanewline
\ \ \isakeyword{shows}\ \ \ {\isachardoublequoteopen}atoms\ G\ {\isasymsubseteq}\ atoms\ {\isacharparenleft}\isactrlbold {\isasymnot}\ F{\isacharparenright}{\isachardoublequoteclose}\isanewline
%
\isadelimproof
%
\endisadelimproof
%
\isatagproof
\isacommand{proof}\isamarkupfalse%
\ {\isacharminus}\isanewline
\ \ \isacommand{have}\isamarkupfalse%
\ {\isachardoublequoteopen}G\ {\isasymin}\ {\isacharbraceleft}\isactrlbold {\isasymnot}\ F{\isacharbraceright}\ {\isasymunion}\ setSubformulae\ F{\isachardoublequoteclose}\isanewline
\ \ \ \ \isacommand{using}\isamarkupfalse%
\ assms{\isacharparenleft}{\isadigit{2}}{\isacharparenright}\isanewline
\ \ \ \ \isacommand{by}\isamarkupfalse%
\ {\isacharparenleft}simp\ only{\isacharcolon}\ setSubformulae{\isacharunderscore}not{\isacharparenright}\ \isanewline
\ \ \isacommand{then}\isamarkupfalse%
\ \isacommand{have}\isamarkupfalse%
\ {\isachardoublequoteopen}G\ {\isasymin}\ {\isacharbraceleft}\isactrlbold {\isasymnot}\ F{\isacharbraceright}\ {\isasymor}\ G\ {\isasymin}\ setSubformulae\ F{\isachardoublequoteclose}\isanewline
\ \ \ \ \isacommand{by}\isamarkupfalse%
\ {\isacharparenleft}simp\ only{\isacharcolon}\ Un{\isacharunderscore}iff{\isacharparenright}\isanewline
\ \ \isacommand{then}\isamarkupfalse%
\ \isacommand{show}\isamarkupfalse%
\ {\isachardoublequoteopen}atoms\ G\ {\isasymsubseteq}\ atoms\ {\isacharparenleft}\isactrlbold {\isasymnot}\ F{\isacharparenright}{\isachardoublequoteclose}\isanewline
\ \ \isacommand{proof}\isamarkupfalse%
\isanewline
\ \ \ \ \isacommand{assume}\isamarkupfalse%
\ {\isachardoublequoteopen}G\ {\isasymin}\ {\isacharbraceleft}\isactrlbold {\isasymnot}\ F{\isacharbraceright}{\isachardoublequoteclose}\isanewline
\ \ \ \ \isacommand{then}\isamarkupfalse%
\ \isacommand{have}\isamarkupfalse%
\ {\isachardoublequoteopen}G\ {\isacharequal}\ \isactrlbold {\isasymnot}\ F{\isachardoublequoteclose}\isanewline
\ \ \ \ \ \ \isacommand{by}\isamarkupfalse%
\ {\isacharparenleft}simp\ only{\isacharcolon}\ singletonD{\isacharparenright}\isanewline
\ \ \ \ \isacommand{then}\isamarkupfalse%
\ \isacommand{show}\isamarkupfalse%
\ {\isacharquery}thesis\isanewline
\ \ \ \ \ \ \isacommand{by}\isamarkupfalse%
\ {\isacharparenleft}simp\ only{\isacharcolon}\ subset{\isacharunderscore}refl{\isacharparenright}\isanewline
\ \ \isacommand{next}\isamarkupfalse%
\isanewline
\ \ \ \ \isacommand{assume}\isamarkupfalse%
\ {\isachardoublequoteopen}G\ {\isasymin}\ setSubformulae\ F{\isachardoublequoteclose}\isanewline
\ \ \ \ \isacommand{then}\isamarkupfalse%
\ \isacommand{have}\isamarkupfalse%
\ {\isachardoublequoteopen}atoms\ G\ {\isasymsubseteq}\ atoms\ F{\isachardoublequoteclose}\isanewline
\ \ \ \ \ \ \isacommand{by}\isamarkupfalse%
\ {\isacharparenleft}simp\ only{\isacharcolon}\ assms{\isacharparenleft}{\isadigit{1}}{\isacharparenright}{\isacharparenright}\isanewline
\ \ \ \ \isacommand{also}\isamarkupfalse%
\ \isacommand{have}\isamarkupfalse%
\ {\isachardoublequoteopen}{\isasymdots}\ {\isacharequal}\ atoms\ {\isacharparenleft}\isactrlbold {\isasymnot}\ F{\isacharparenright}{\isachardoublequoteclose}\isanewline
\ \ \ \ \ \ \isacommand{by}\isamarkupfalse%
\ {\isacharparenleft}simp\ only{\isacharcolon}\ formula{\isachardot}set{\isacharparenleft}{\isadigit{3}}{\isacharparenright}{\isacharparenright}\isanewline
\ \ \ \ \isacommand{finally}\isamarkupfalse%
\ \isacommand{show}\isamarkupfalse%
\ {\isacharquery}thesis\isanewline
\ \ \ \ \ \ \isacommand{by}\isamarkupfalse%
\ this\isanewline
\ \ \isacommand{qed}\isamarkupfalse%
\isanewline
\isacommand{qed}\isamarkupfalse%
%
\endisatagproof
{\isafoldproof}%
%
\isadelimproof
\isanewline
%
\endisadelimproof
\isanewline
\isacommand{lemma}\isamarkupfalse%
\ subformulas{\isacharunderscore}atoms{\isacharunderscore}and{\isacharcolon}\isanewline
\ \ \isakeyword{assumes}\ {\isachardoublequoteopen}G\ {\isasymin}\ setSubformulae\ F{\isadigit{1}}\ {\isasymLongrightarrow}\ atoms\ G\ {\isasymsubseteq}\ atoms\ F{\isadigit{1}}{\isachardoublequoteclose}\isanewline
\ \ \ \ \ \ \ \ \ \ {\isachardoublequoteopen}G\ {\isasymin}\ setSubformulae\ F{\isadigit{2}}\ {\isasymLongrightarrow}\ atoms\ G\ {\isasymsubseteq}\ atoms\ F{\isadigit{2}}{\isachardoublequoteclose}\isanewline
\ \ \ \ \ \ \ \ \ \ {\isachardoublequoteopen}G\ {\isasymin}\ setSubformulae\ {\isacharparenleft}F{\isadigit{1}}\ \isactrlbold {\isasymand}\ F{\isadigit{2}}{\isacharparenright}{\isachardoublequoteclose}\isanewline
\ \ \isakeyword{shows}\ \ \ {\isachardoublequoteopen}atoms\ G\ {\isasymsubseteq}\ atoms\ {\isacharparenleft}F{\isadigit{1}}\ \isactrlbold {\isasymand}\ F{\isadigit{2}}{\isacharparenright}{\isachardoublequoteclose}\isanewline
%
\isadelimproof
%
\endisadelimproof
%
\isatagproof
\isacommand{proof}\isamarkupfalse%
\ {\isacharminus}\isanewline
\ \ \isacommand{have}\isamarkupfalse%
\ {\isachardoublequoteopen}G\ {\isasymin}\ {\isacharbraceleft}F{\isadigit{1}}\ \isactrlbold {\isasymand}\ F{\isadigit{2}}{\isacharbraceright}\ {\isasymunion}\ {\isacharparenleft}setSubformulae\ F{\isadigit{1}}\ {\isasymunion}\ setSubformulae\ F{\isadigit{2}}{\isacharparenright}{\isachardoublequoteclose}\isanewline
\ \ \ \ \isacommand{using}\isamarkupfalse%
\ assms{\isacharparenleft}{\isadigit{3}}{\isacharparenright}\ \isanewline
\ \ \ \ \isacommand{by}\isamarkupfalse%
\ {\isacharparenleft}simp\ only{\isacharcolon}\ setSubformulae{\isacharunderscore}and{\isacharparenright}\isanewline
\ \ \isacommand{then}\isamarkupfalse%
\ \isacommand{have}\isamarkupfalse%
\ {\isachardoublequoteopen}G\ {\isasymin}\ {\isacharbraceleft}F{\isadigit{1}}\ \isactrlbold {\isasymand}\ F{\isadigit{2}}{\isacharbraceright}\ {\isasymor}\ G\ {\isasymin}\ setSubformulae\ F{\isadigit{1}}\ {\isasymunion}\ setSubformulae\ F{\isadigit{2}}{\isachardoublequoteclose}\isanewline
\ \ \ \ \isacommand{by}\isamarkupfalse%
\ {\isacharparenleft}simp\ only{\isacharcolon}\ Un{\isacharunderscore}iff{\isacharparenright}\isanewline
\ \ \isacommand{then}\isamarkupfalse%
\ \isacommand{show}\isamarkupfalse%
\ {\isacharquery}thesis\isanewline
\ \ \isacommand{proof}\isamarkupfalse%
\ \isanewline
\ \ \ \ \isacommand{assume}\isamarkupfalse%
\ {\isachardoublequoteopen}G\ {\isasymin}\ {\isacharbraceleft}F{\isadigit{1}}\ \isactrlbold {\isasymand}\ F{\isadigit{2}}{\isacharbraceright}{\isachardoublequoteclose}\isanewline
\ \ \ \ \isacommand{then}\isamarkupfalse%
\ \isacommand{have}\isamarkupfalse%
\ {\isachardoublequoteopen}G\ {\isacharequal}\ F{\isadigit{1}}\ \isactrlbold {\isasymand}\ F{\isadigit{2}}{\isachardoublequoteclose}\isanewline
\ \ \ \ \ \ \isacommand{by}\isamarkupfalse%
\ {\isacharparenleft}simp\ only{\isacharcolon}\ singletonD{\isacharparenright}\isanewline
\ \ \ \ \isacommand{then}\isamarkupfalse%
\ \isacommand{show}\isamarkupfalse%
\ {\isacharquery}thesis\isanewline
\ \ \ \ \ \ \isacommand{by}\isamarkupfalse%
\ {\isacharparenleft}simp\ only{\isacharcolon}\ subset{\isacharunderscore}refl{\isacharparenright}\isanewline
\ \ \isacommand{next}\isamarkupfalse%
\isanewline
\ \ \ \ \isacommand{assume}\isamarkupfalse%
\ {\isachardoublequoteopen}G\ {\isasymin}\ setSubformulae\ F{\isadigit{1}}\ {\isasymunion}\ setSubformulae\ F{\isadigit{2}}{\isachardoublequoteclose}\isanewline
\ \ \ \ \isacommand{then}\isamarkupfalse%
\ \isacommand{have}\isamarkupfalse%
\ {\isachardoublequoteopen}G\ {\isasymin}\ setSubformulae\ F{\isadigit{1}}\ {\isasymor}\ G\ {\isasymin}\ setSubformulae\ F{\isadigit{2}}{\isachardoublequoteclose}\ \ \isanewline
\ \ \ \ \ \ \isacommand{by}\isamarkupfalse%
\ {\isacharparenleft}simp\ only{\isacharcolon}\ Un{\isacharunderscore}iff{\isacharparenright}\isanewline
\ \ \ \ \isacommand{then}\isamarkupfalse%
\ \isacommand{show}\isamarkupfalse%
\ {\isacharquery}thesis\isanewline
\ \ \ \ \isacommand{proof}\isamarkupfalse%
\ \isanewline
\ \ \ \ \ \ \isacommand{assume}\isamarkupfalse%
\ {\isachardoublequoteopen}G\ {\isasymin}\ setSubformulae\ F{\isadigit{1}}{\isachardoublequoteclose}\isanewline
\ \ \ \ \ \ \isacommand{then}\isamarkupfalse%
\ \isacommand{have}\isamarkupfalse%
\ {\isachardoublequoteopen}atoms\ G\ {\isasymsubseteq}\ atoms\ F{\isadigit{1}}{\isachardoublequoteclose}\isanewline
\ \ \ \ \ \ \ \ \isacommand{by}\isamarkupfalse%
\ {\isacharparenleft}rule\ assms{\isacharparenleft}{\isadigit{1}}{\isacharparenright}{\isacharparenright}\isanewline
\ \ \ \ \ \ \isacommand{also}\isamarkupfalse%
\ \isacommand{have}\isamarkupfalse%
\ {\isachardoublequoteopen}{\isasymdots}\ {\isasymsubseteq}\ atoms\ F{\isadigit{1}}\ {\isasymunion}\ atoms\ F{\isadigit{2}}{\isachardoublequoteclose}\isanewline
\ \ \ \ \ \ \ \ \isacommand{by}\isamarkupfalse%
\ {\isacharparenleft}simp\ only{\isacharcolon}\ Un{\isacharunderscore}upper{\isadigit{1}}{\isacharparenright}\isanewline
\ \ \ \ \ \ \isacommand{also}\isamarkupfalse%
\ \isacommand{have}\isamarkupfalse%
\ {\isachardoublequoteopen}{\isasymdots}\ {\isacharequal}\ atoms\ {\isacharparenleft}F{\isadigit{1}}\ \isactrlbold {\isasymand}\ F{\isadigit{2}}{\isacharparenright}{\isachardoublequoteclose}\isanewline
\ \ \ \ \ \ \ \ \isacommand{by}\isamarkupfalse%
\ {\isacharparenleft}simp\ only{\isacharcolon}\ formula{\isachardot}set{\isacharparenleft}{\isadigit{4}}{\isacharparenright}{\isacharparenright}\isanewline
\ \ \ \ \ \ \isacommand{finally}\isamarkupfalse%
\ \isacommand{show}\isamarkupfalse%
\ {\isacharquery}thesis\isanewline
\ \ \ \ \ \ \ \ \isacommand{by}\isamarkupfalse%
\ this\isanewline
\ \ \ \ \isacommand{next}\isamarkupfalse%
\isanewline
\ \ \ \ \ \ \isacommand{assume}\isamarkupfalse%
\ {\isachardoublequoteopen}G\ {\isasymin}\ setSubformulae\ F{\isadigit{2}}{\isachardoublequoteclose}\isanewline
\ \ \ \ \ \ \isacommand{then}\isamarkupfalse%
\ \isacommand{have}\isamarkupfalse%
\ {\isachardoublequoteopen}atoms\ G\ {\isasymsubseteq}\ atoms\ F{\isadigit{2}}{\isachardoublequoteclose}\isanewline
\ \ \ \ \ \ \ \ \isacommand{by}\isamarkupfalse%
\ {\isacharparenleft}rule\ assms{\isacharparenleft}{\isadigit{2}}{\isacharparenright}{\isacharparenright}\isanewline
\ \ \ \ \ \ \isacommand{also}\isamarkupfalse%
\ \isacommand{have}\isamarkupfalse%
\ {\isachardoublequoteopen}{\isasymdots}\ {\isasymsubseteq}\ atoms\ F{\isadigit{1}}\ {\isasymunion}\ atoms\ F{\isadigit{2}}{\isachardoublequoteclose}\isanewline
\ \ \ \ \ \ \ \ \isacommand{by}\isamarkupfalse%
\ {\isacharparenleft}simp\ only{\isacharcolon}\ Un{\isacharunderscore}upper{\isadigit{2}}{\isacharparenright}\isanewline
\ \ \ \ \ \ \isacommand{also}\isamarkupfalse%
\ \isacommand{have}\isamarkupfalse%
\ {\isachardoublequoteopen}{\isasymdots}\ {\isacharequal}\ atoms\ {\isacharparenleft}F{\isadigit{1}}\ \isactrlbold {\isasymand}\ F{\isadigit{2}}{\isacharparenright}{\isachardoublequoteclose}\isanewline
\ \ \ \ \ \ \ \ \isacommand{by}\isamarkupfalse%
\ {\isacharparenleft}simp\ only{\isacharcolon}\ formula{\isachardot}set{\isacharparenleft}{\isadigit{4}}{\isacharparenright}{\isacharparenright}\isanewline
\ \ \ \ \ \ \isacommand{finally}\isamarkupfalse%
\ \isacommand{show}\isamarkupfalse%
\ {\isacharquery}thesis\isanewline
\ \ \ \ \ \ \ \ \isacommand{by}\isamarkupfalse%
\ this\isanewline
\ \ \ \ \isacommand{qed}\isamarkupfalse%
\isanewline
\ \ \isacommand{qed}\isamarkupfalse%
\isanewline
\isacommand{qed}\isamarkupfalse%
%
\endisatagproof
{\isafoldproof}%
%
\isadelimproof
\isanewline
%
\endisadelimproof
\isanewline
\isacommand{lemma}\isamarkupfalse%
\ subformulas{\isacharunderscore}atoms{\isacharunderscore}or{\isacharcolon}\isanewline
\ \ \isakeyword{assumes}\ {\isachardoublequoteopen}G\ {\isasymin}\ setSubformulae\ F{\isadigit{1}}\ {\isasymLongrightarrow}\ atoms\ G\ {\isasymsubseteq}\ atoms\ F{\isadigit{1}}{\isachardoublequoteclose}\isanewline
\ \ \ \ \ \ \ \ \ \ {\isachardoublequoteopen}G\ {\isasymin}\ setSubformulae\ F{\isadigit{2}}\ {\isasymLongrightarrow}\ atoms\ G\ {\isasymsubseteq}\ atoms\ F{\isadigit{2}}{\isachardoublequoteclose}\isanewline
\ \ \ \ \ \ \ \ \ \ {\isachardoublequoteopen}G\ {\isasymin}\ setSubformulae\ {\isacharparenleft}F{\isadigit{1}}\ \isactrlbold {\isasymor}\ F{\isadigit{2}}{\isacharparenright}{\isachardoublequoteclose}\isanewline
\ \ \isakeyword{shows}\ \ \ {\isachardoublequoteopen}atoms\ G\ {\isasymsubseteq}\ atoms\ {\isacharparenleft}F{\isadigit{1}}\ \isactrlbold {\isasymor}\ F{\isadigit{2}}{\isacharparenright}{\isachardoublequoteclose}\isanewline
%
\isadelimproof
%
\endisadelimproof
%
\isatagproof
\isacommand{proof}\isamarkupfalse%
\ {\isacharminus}\isanewline
\ \ \isacommand{have}\isamarkupfalse%
\ {\isachardoublequoteopen}G\ {\isasymin}\ {\isacharbraceleft}F{\isadigit{1}}\ \isactrlbold {\isasymor}\ F{\isadigit{2}}{\isacharbraceright}\ {\isasymunion}\ {\isacharparenleft}setSubformulae\ F{\isadigit{1}}\ {\isasymunion}\ setSubformulae\ F{\isadigit{2}}{\isacharparenright}{\isachardoublequoteclose}\isanewline
\ \ \ \ \isacommand{using}\isamarkupfalse%
\ assms{\isacharparenleft}{\isadigit{3}}{\isacharparenright}\ \isanewline
\ \ \ \ \isacommand{by}\isamarkupfalse%
\ {\isacharparenleft}simp\ only{\isacharcolon}\ setSubformulae{\isacharunderscore}or{\isacharparenright}\isanewline
\ \ \isacommand{then}\isamarkupfalse%
\ \isacommand{have}\isamarkupfalse%
\ {\isachardoublequoteopen}G\ {\isasymin}\ {\isacharbraceleft}F{\isadigit{1}}\ \isactrlbold {\isasymor}\ F{\isadigit{2}}{\isacharbraceright}\ {\isasymor}\ G\ {\isasymin}\ setSubformulae\ F{\isadigit{1}}\ {\isasymunion}\ setSubformulae\ F{\isadigit{2}}{\isachardoublequoteclose}\isanewline
\ \ \ \ \isacommand{by}\isamarkupfalse%
\ {\isacharparenleft}simp\ only{\isacharcolon}\ Un{\isacharunderscore}iff{\isacharparenright}\isanewline
\ \ \isacommand{then}\isamarkupfalse%
\ \isacommand{show}\isamarkupfalse%
\ {\isacharquery}thesis\isanewline
\ \ \isacommand{proof}\isamarkupfalse%
\ \isanewline
\ \ \ \ \isacommand{assume}\isamarkupfalse%
\ {\isachardoublequoteopen}G\ {\isasymin}\ {\isacharbraceleft}F{\isadigit{1}}\ \isactrlbold {\isasymor}\ F{\isadigit{2}}{\isacharbraceright}{\isachardoublequoteclose}\isanewline
\ \ \ \ \isacommand{then}\isamarkupfalse%
\ \isacommand{have}\isamarkupfalse%
\ {\isachardoublequoteopen}G\ {\isacharequal}\ F{\isadigit{1}}\ \isactrlbold {\isasymor}\ F{\isadigit{2}}{\isachardoublequoteclose}\isanewline
\ \ \ \ \ \ \isacommand{by}\isamarkupfalse%
\ {\isacharparenleft}simp\ only{\isacharcolon}\ singletonD{\isacharparenright}\isanewline
\ \ \ \ \isacommand{then}\isamarkupfalse%
\ \isacommand{show}\isamarkupfalse%
\ {\isacharquery}thesis\isanewline
\ \ \ \ \ \ \isacommand{by}\isamarkupfalse%
\ {\isacharparenleft}simp\ only{\isacharcolon}\ subset{\isacharunderscore}refl{\isacharparenright}\isanewline
\ \ \isacommand{next}\isamarkupfalse%
\isanewline
\ \ \ \ \isacommand{assume}\isamarkupfalse%
\ {\isachardoublequoteopen}G\ {\isasymin}\ setSubformulae\ F{\isadigit{1}}\ {\isasymunion}\ setSubformulae\ F{\isadigit{2}}{\isachardoublequoteclose}\isanewline
\ \ \ \ \isacommand{then}\isamarkupfalse%
\ \isacommand{have}\isamarkupfalse%
\ {\isachardoublequoteopen}G\ {\isasymin}\ setSubformulae\ F{\isadigit{1}}\ {\isasymor}\ G\ {\isasymin}\ setSubformulae\ F{\isadigit{2}}{\isachardoublequoteclose}\ \ \isanewline
\ \ \ \ \ \ \isacommand{by}\isamarkupfalse%
\ {\isacharparenleft}simp\ only{\isacharcolon}\ Un{\isacharunderscore}iff{\isacharparenright}\isanewline
\ \ \ \ \isacommand{then}\isamarkupfalse%
\ \isacommand{show}\isamarkupfalse%
\ {\isacharquery}thesis\isanewline
\ \ \ \ \isacommand{proof}\isamarkupfalse%
\ \isanewline
\ \ \ \ \ \ \isacommand{assume}\isamarkupfalse%
\ {\isachardoublequoteopen}G\ {\isasymin}\ setSubformulae\ F{\isadigit{1}}{\isachardoublequoteclose}\isanewline
\ \ \ \ \ \ \isacommand{then}\isamarkupfalse%
\ \isacommand{have}\isamarkupfalse%
\ {\isachardoublequoteopen}atoms\ G\ {\isasymsubseteq}\ atoms\ F{\isadigit{1}}{\isachardoublequoteclose}\isanewline
\ \ \ \ \ \ \ \ \isacommand{by}\isamarkupfalse%
\ {\isacharparenleft}rule\ assms{\isacharparenleft}{\isadigit{1}}{\isacharparenright}{\isacharparenright}\isanewline
\ \ \ \ \ \ \isacommand{also}\isamarkupfalse%
\ \isacommand{have}\isamarkupfalse%
\ {\isachardoublequoteopen}{\isasymdots}\ {\isasymsubseteq}\ atoms\ F{\isadigit{1}}\ {\isasymunion}\ atoms\ F{\isadigit{2}}{\isachardoublequoteclose}\isanewline
\ \ \ \ \ \ \ \ \isacommand{by}\isamarkupfalse%
\ {\isacharparenleft}simp\ only{\isacharcolon}\ Un{\isacharunderscore}upper{\isadigit{1}}{\isacharparenright}\isanewline
\ \ \ \ \ \ \isacommand{also}\isamarkupfalse%
\ \isacommand{have}\isamarkupfalse%
\ {\isachardoublequoteopen}{\isasymdots}\ {\isacharequal}\ atoms\ {\isacharparenleft}F{\isadigit{1}}\ \isactrlbold {\isasymor}\ F{\isadigit{2}}{\isacharparenright}{\isachardoublequoteclose}\isanewline
\ \ \ \ \ \ \ \ \isacommand{by}\isamarkupfalse%
\ {\isacharparenleft}simp\ only{\isacharcolon}\ formula{\isachardot}set{\isacharparenleft}{\isadigit{5}}{\isacharparenright}{\isacharparenright}\isanewline
\ \ \ \ \ \ \isacommand{finally}\isamarkupfalse%
\ \isacommand{show}\isamarkupfalse%
\ {\isacharquery}thesis\isanewline
\ \ \ \ \ \ \ \ \isacommand{by}\isamarkupfalse%
\ this\isanewline
\ \ \ \ \isacommand{next}\isamarkupfalse%
\isanewline
\ \ \ \ \ \ \isacommand{assume}\isamarkupfalse%
\ {\isachardoublequoteopen}G\ {\isasymin}\ setSubformulae\ F{\isadigit{2}}{\isachardoublequoteclose}\isanewline
\ \ \ \ \ \ \isacommand{then}\isamarkupfalse%
\ \isacommand{have}\isamarkupfalse%
\ {\isachardoublequoteopen}atoms\ G\ {\isasymsubseteq}\ atoms\ F{\isadigit{2}}{\isachardoublequoteclose}\isanewline
\ \ \ \ \ \ \ \ \isacommand{by}\isamarkupfalse%
\ {\isacharparenleft}rule\ assms{\isacharparenleft}{\isadigit{2}}{\isacharparenright}{\isacharparenright}\isanewline
\ \ \ \ \ \ \isacommand{also}\isamarkupfalse%
\ \isacommand{have}\isamarkupfalse%
\ {\isachardoublequoteopen}{\isasymdots}\ {\isasymsubseteq}\ atoms\ F{\isadigit{1}}\ {\isasymunion}\ atoms\ F{\isadigit{2}}{\isachardoublequoteclose}\isanewline
\ \ \ \ \ \ \ \ \isacommand{by}\isamarkupfalse%
\ {\isacharparenleft}simp\ only{\isacharcolon}\ Un{\isacharunderscore}upper{\isadigit{2}}{\isacharparenright}\isanewline
\ \ \ \ \ \ \isacommand{also}\isamarkupfalse%
\ \isacommand{have}\isamarkupfalse%
\ {\isachardoublequoteopen}{\isasymdots}\ {\isacharequal}\ atoms\ {\isacharparenleft}F{\isadigit{1}}\ \isactrlbold {\isasymor}\ F{\isadigit{2}}{\isacharparenright}{\isachardoublequoteclose}\isanewline
\ \ \ \ \ \ \ \ \isacommand{by}\isamarkupfalse%
\ {\isacharparenleft}simp\ only{\isacharcolon}\ formula{\isachardot}set{\isacharparenleft}{\isadigit{5}}{\isacharparenright}{\isacharparenright}\isanewline
\ \ \ \ \ \ \isacommand{finally}\isamarkupfalse%
\ \isacommand{show}\isamarkupfalse%
\ {\isacharquery}thesis\isanewline
\ \ \ \ \ \ \ \ \isacommand{by}\isamarkupfalse%
\ this\isanewline
\ \ \ \ \isacommand{qed}\isamarkupfalse%
\isanewline
\ \ \isacommand{qed}\isamarkupfalse%
\isanewline
\isacommand{qed}\isamarkupfalse%
%
\endisatagproof
{\isafoldproof}%
%
\isadelimproof
\isanewline
%
\endisadelimproof
\isanewline
\isacommand{lemma}\isamarkupfalse%
\ subformulas{\isacharunderscore}atoms{\isacharunderscore}imp{\isacharcolon}\isanewline
\ \ \isakeyword{assumes}\ {\isachardoublequoteopen}G\ {\isasymin}\ setSubformulae\ F{\isadigit{1}}\ {\isasymLongrightarrow}\ atoms\ G\ {\isasymsubseteq}\ atoms\ F{\isadigit{1}}{\isachardoublequoteclose}\isanewline
\ \ \ \ \ \ \ \ \ \ {\isachardoublequoteopen}G\ {\isasymin}\ setSubformulae\ F{\isadigit{2}}\ {\isasymLongrightarrow}\ atoms\ G\ {\isasymsubseteq}\ atoms\ F{\isadigit{2}}{\isachardoublequoteclose}\isanewline
\ \ \ \ \ \ \ \ \ \ {\isachardoublequoteopen}G\ {\isasymin}\ setSubformulae\ {\isacharparenleft}F{\isadigit{1}}\ \isactrlbold {\isasymrightarrow}\ F{\isadigit{2}}{\isacharparenright}{\isachardoublequoteclose}\isanewline
\ \ \isakeyword{shows}\ \ \ {\isachardoublequoteopen}atoms\ G\ {\isasymsubseteq}\ atoms\ {\isacharparenleft}F{\isadigit{1}}\ \isactrlbold {\isasymrightarrow}\ F{\isadigit{2}}{\isacharparenright}{\isachardoublequoteclose}\isanewline
%
\isadelimproof
%
\endisadelimproof
%
\isatagproof
\isacommand{proof}\isamarkupfalse%
\ {\isacharminus}\isanewline
\ \ \isacommand{have}\isamarkupfalse%
\ {\isachardoublequoteopen}G\ {\isasymin}\ {\isacharbraceleft}F{\isadigit{1}}\ \isactrlbold {\isasymrightarrow}\ F{\isadigit{2}}{\isacharbraceright}\ {\isasymunion}\ {\isacharparenleft}setSubformulae\ F{\isadigit{1}}\ {\isasymunion}\ setSubformulae\ F{\isadigit{2}}{\isacharparenright}{\isachardoublequoteclose}\isanewline
\ \ \ \ \isacommand{using}\isamarkupfalse%
\ assms{\isacharparenleft}{\isadigit{3}}{\isacharparenright}\ \isanewline
\ \ \ \ \isacommand{by}\isamarkupfalse%
\ {\isacharparenleft}simp\ only{\isacharcolon}\ setSubformulae{\isacharunderscore}imp{\isacharparenright}\isanewline
\ \ \isacommand{then}\isamarkupfalse%
\ \isacommand{have}\isamarkupfalse%
\ {\isachardoublequoteopen}G\ {\isasymin}\ {\isacharbraceleft}F{\isadigit{1}}\ \isactrlbold {\isasymrightarrow}\ F{\isadigit{2}}{\isacharbraceright}\ {\isasymor}\ G\ {\isasymin}\ setSubformulae\ F{\isadigit{1}}\ {\isasymunion}\ setSubformulae\ F{\isadigit{2}}{\isachardoublequoteclose}\isanewline
\ \ \ \ \isacommand{by}\isamarkupfalse%
\ {\isacharparenleft}simp\ only{\isacharcolon}\ Un{\isacharunderscore}iff{\isacharparenright}\isanewline
\ \ \isacommand{then}\isamarkupfalse%
\ \isacommand{show}\isamarkupfalse%
\ {\isacharquery}thesis\isanewline
\ \ \isacommand{proof}\isamarkupfalse%
\ \isanewline
\ \ \ \ \isacommand{assume}\isamarkupfalse%
\ {\isachardoublequoteopen}G\ {\isasymin}\ {\isacharbraceleft}F{\isadigit{1}}\ \isactrlbold {\isasymrightarrow}\ F{\isadigit{2}}{\isacharbraceright}{\isachardoublequoteclose}\isanewline
\ \ \ \ \isacommand{then}\isamarkupfalse%
\ \isacommand{have}\isamarkupfalse%
\ {\isachardoublequoteopen}G\ {\isacharequal}\ F{\isadigit{1}}\ \isactrlbold {\isasymrightarrow}\ F{\isadigit{2}}{\isachardoublequoteclose}\isanewline
\ \ \ \ \ \ \isacommand{by}\isamarkupfalse%
\ {\isacharparenleft}simp\ only{\isacharcolon}\ singletonD{\isacharparenright}\isanewline
\ \ \ \ \isacommand{then}\isamarkupfalse%
\ \isacommand{show}\isamarkupfalse%
\ {\isacharquery}thesis\isanewline
\ \ \ \ \ \ \isacommand{by}\isamarkupfalse%
\ {\isacharparenleft}simp\ only{\isacharcolon}\ subset{\isacharunderscore}refl{\isacharparenright}\isanewline
\ \ \isacommand{next}\isamarkupfalse%
\isanewline
\ \ \ \ \isacommand{assume}\isamarkupfalse%
\ {\isachardoublequoteopen}G\ {\isasymin}\ setSubformulae\ F{\isadigit{1}}\ {\isasymunion}\ setSubformulae\ F{\isadigit{2}}{\isachardoublequoteclose}\isanewline
\ \ \ \ \isacommand{then}\isamarkupfalse%
\ \isacommand{have}\isamarkupfalse%
\ {\isachardoublequoteopen}G\ {\isasymin}\ setSubformulae\ F{\isadigit{1}}\ {\isasymor}\ G\ {\isasymin}\ setSubformulae\ F{\isadigit{2}}{\isachardoublequoteclose}\ \ \isanewline
\ \ \ \ \ \ \isacommand{by}\isamarkupfalse%
\ {\isacharparenleft}simp\ only{\isacharcolon}\ Un{\isacharunderscore}iff{\isacharparenright}\isanewline
\ \ \ \ \isacommand{then}\isamarkupfalse%
\ \isacommand{show}\isamarkupfalse%
\ {\isacharquery}thesis\isanewline
\ \ \ \ \isacommand{proof}\isamarkupfalse%
\ \isanewline
\ \ \ \ \ \ \isacommand{assume}\isamarkupfalse%
\ {\isachardoublequoteopen}G\ {\isasymin}\ setSubformulae\ F{\isadigit{1}}{\isachardoublequoteclose}\isanewline
\ \ \ \ \ \ \isacommand{then}\isamarkupfalse%
\ \isacommand{have}\isamarkupfalse%
\ {\isachardoublequoteopen}atoms\ G\ {\isasymsubseteq}\ atoms\ F{\isadigit{1}}{\isachardoublequoteclose}\isanewline
\ \ \ \ \ \ \ \ \isacommand{by}\isamarkupfalse%
\ {\isacharparenleft}rule\ assms{\isacharparenleft}{\isadigit{1}}{\isacharparenright}{\isacharparenright}\isanewline
\ \ \ \ \ \ \isacommand{also}\isamarkupfalse%
\ \isacommand{have}\isamarkupfalse%
\ {\isachardoublequoteopen}{\isasymdots}\ {\isasymsubseteq}\ atoms\ F{\isadigit{1}}\ {\isasymunion}\ atoms\ F{\isadigit{2}}{\isachardoublequoteclose}\isanewline
\ \ \ \ \ \ \ \ \isacommand{by}\isamarkupfalse%
\ {\isacharparenleft}simp\ only{\isacharcolon}\ Un{\isacharunderscore}upper{\isadigit{1}}{\isacharparenright}\isanewline
\ \ \ \ \ \ \isacommand{also}\isamarkupfalse%
\ \isacommand{have}\isamarkupfalse%
\ {\isachardoublequoteopen}{\isasymdots}\ {\isacharequal}\ atoms\ {\isacharparenleft}F{\isadigit{1}}\ \isactrlbold {\isasymrightarrow}\ F{\isadigit{2}}{\isacharparenright}{\isachardoublequoteclose}\isanewline
\ \ \ \ \ \ \ \ \isacommand{by}\isamarkupfalse%
\ {\isacharparenleft}simp\ only{\isacharcolon}\ formula{\isachardot}set{\isacharparenleft}{\isadigit{6}}{\isacharparenright}{\isacharparenright}\isanewline
\ \ \ \ \ \ \isacommand{finally}\isamarkupfalse%
\ \isacommand{show}\isamarkupfalse%
\ {\isacharquery}thesis\isanewline
\ \ \ \ \ \ \ \ \isacommand{by}\isamarkupfalse%
\ this\isanewline
\ \ \ \ \isacommand{next}\isamarkupfalse%
\isanewline
\ \ \ \ \ \ \isacommand{assume}\isamarkupfalse%
\ {\isachardoublequoteopen}G\ {\isasymin}\ setSubformulae\ F{\isadigit{2}}{\isachardoublequoteclose}\isanewline
\ \ \ \ \ \ \isacommand{then}\isamarkupfalse%
\ \isacommand{have}\isamarkupfalse%
\ {\isachardoublequoteopen}atoms\ G\ {\isasymsubseteq}\ atoms\ F{\isadigit{2}}{\isachardoublequoteclose}\isanewline
\ \ \ \ \ \ \ \ \isacommand{by}\isamarkupfalse%
\ {\isacharparenleft}rule\ assms{\isacharparenleft}{\isadigit{2}}{\isacharparenright}{\isacharparenright}\isanewline
\ \ \ \ \ \ \isacommand{also}\isamarkupfalse%
\ \isacommand{have}\isamarkupfalse%
\ {\isachardoublequoteopen}{\isasymdots}\ {\isasymsubseteq}\ atoms\ F{\isadigit{1}}\ {\isasymunion}\ atoms\ F{\isadigit{2}}{\isachardoublequoteclose}\isanewline
\ \ \ \ \ \ \ \ \isacommand{by}\isamarkupfalse%
\ {\isacharparenleft}simp\ only{\isacharcolon}\ Un{\isacharunderscore}upper{\isadigit{2}}{\isacharparenright}\isanewline
\ \ \ \ \ \ \isacommand{also}\isamarkupfalse%
\ \isacommand{have}\isamarkupfalse%
\ {\isachardoublequoteopen}{\isasymdots}\ {\isacharequal}\ atoms\ {\isacharparenleft}F{\isadigit{1}}\ \isactrlbold {\isasymrightarrow}\ F{\isadigit{2}}{\isacharparenright}{\isachardoublequoteclose}\isanewline
\ \ \ \ \ \ \ \ \isacommand{by}\isamarkupfalse%
\ {\isacharparenleft}simp\ only{\isacharcolon}\ formula{\isachardot}set{\isacharparenleft}{\isadigit{6}}{\isacharparenright}{\isacharparenright}\isanewline
\ \ \ \ \ \ \isacommand{finally}\isamarkupfalse%
\ \isacommand{show}\isamarkupfalse%
\ {\isacharquery}thesis\isanewline
\ \ \ \ \ \ \ \ \isacommand{by}\isamarkupfalse%
\ this\isanewline
\ \ \ \ \isacommand{qed}\isamarkupfalse%
\isanewline
\ \ \isacommand{qed}\isamarkupfalse%
\isanewline
\isacommand{qed}\isamarkupfalse%
%
\endisatagproof
{\isafoldproof}%
%
\isadelimproof
\isanewline
%
\endisadelimproof
\isanewline
\isacommand{lemma}\isamarkupfalse%
\ subformulae{\isacharunderscore}atoms{\isacharcolon}\ {\isachardoublequoteopen}G\ {\isasymin}\ setSubformulae\ F\ {\isasymLongrightarrow}\ atoms\ G\ {\isasymsubseteq}\ atoms\ F{\isachardoublequoteclose}\isanewline
%
\isadelimproof
%
\endisadelimproof
%
\isatagproof
\isacommand{proof}\isamarkupfalse%
\ {\isacharparenleft}induction\ F{\isacharparenright}\isanewline
\ \ \isacommand{case}\isamarkupfalse%
\ {\isacharparenleft}Atom\ x{\isacharparenright}\isanewline
\ \ \isacommand{then}\isamarkupfalse%
\ \isacommand{show}\isamarkupfalse%
\ {\isacharquery}case\ \isacommand{by}\isamarkupfalse%
\ {\isacharparenleft}simp\ only{\isacharcolon}\ subformulas{\isacharunderscore}atoms{\isacharunderscore}atom{\isacharparenright}\ \isanewline
\isacommand{next}\isamarkupfalse%
\isanewline
\ \ \isacommand{case}\isamarkupfalse%
\ Bot\isanewline
\ \ \isacommand{then}\isamarkupfalse%
\ \isacommand{show}\isamarkupfalse%
\ {\isacharquery}case\ \isacommand{by}\isamarkupfalse%
\ {\isacharparenleft}simp\ only{\isacharcolon}\ subformulas{\isacharunderscore}atoms{\isacharunderscore}bot{\isacharparenright}\isanewline
\isacommand{next}\isamarkupfalse%
\isanewline
\ \ \isacommand{case}\isamarkupfalse%
\ {\isacharparenleft}Not\ F{\isacharparenright}\isanewline
\ \ \isacommand{then}\isamarkupfalse%
\ \isacommand{show}\isamarkupfalse%
\ {\isacharquery}case\ \isacommand{by}\isamarkupfalse%
\ {\isacharparenleft}simp\ only{\isacharcolon}\ subformulas{\isacharunderscore}atoms{\isacharunderscore}not{\isacharparenright}\isanewline
\isacommand{next}\isamarkupfalse%
\isanewline
\ \ \isacommand{case}\isamarkupfalse%
\ {\isacharparenleft}And\ F{\isadigit{1}}\ F{\isadigit{2}}{\isacharparenright}\isanewline
\ \ \isacommand{then}\isamarkupfalse%
\ \isacommand{show}\isamarkupfalse%
\ {\isacharquery}case\ \isacommand{by}\isamarkupfalse%
\ {\isacharparenleft}simp\ only{\isacharcolon}\ subformulas{\isacharunderscore}atoms{\isacharunderscore}and{\isacharparenright}\isanewline
\isacommand{next}\isamarkupfalse%
\isanewline
\ \ \isacommand{case}\isamarkupfalse%
\ {\isacharparenleft}Or\ F{\isadigit{1}}\ F{\isadigit{2}}{\isacharparenright}\isanewline
\ \ \isacommand{then}\isamarkupfalse%
\ \isacommand{show}\isamarkupfalse%
\ {\isacharquery}case\ \isacommand{by}\isamarkupfalse%
\ {\isacharparenleft}simp\ only{\isacharcolon}\ subformulas{\isacharunderscore}atoms{\isacharunderscore}or{\isacharparenright}\isanewline
\isacommand{next}\isamarkupfalse%
\isanewline
\ \ \isacommand{case}\isamarkupfalse%
\ {\isacharparenleft}Imp\ F{\isadigit{1}}\ F{\isadigit{2}}{\isacharparenright}\isanewline
\ \ \isacommand{then}\isamarkupfalse%
\ \isacommand{show}\isamarkupfalse%
\ {\isacharquery}case\ \isacommand{by}\isamarkupfalse%
\ {\isacharparenleft}simp\ only{\isacharcolon}\ subformulas{\isacharunderscore}atoms{\isacharunderscore}imp{\isacharparenright}\isanewline
\isacommand{qed}\isamarkupfalse%
%
\endisatagproof
{\isafoldproof}%
%
\isadelimproof
%
\endisadelimproof
%
\begin{isamarkuptext}%
Por último, su demostración aplicativa automática.%
\end{isamarkuptext}\isamarkuptrue%
\isacommand{lemma}\isamarkupfalse%
\ {\isachardoublequoteopen}G\ {\isasymin}\ setSubformulae\ F\ {\isasymLongrightarrow}\ atoms\ G\ {\isasymsubseteq}\ atoms\ F{\isachardoublequoteclose}\isanewline
%
\isadelimproof
\ \ %
\endisadelimproof
%
\isatagproof
\isacommand{by}\isamarkupfalse%
\ {\isacharparenleft}induction\ F{\isacharparenright}\ auto%
\endisatagproof
{\isafoldproof}%
%
\isadelimproof
%
\endisadelimproof
%
\begin{isamarkuptext}%
A continuación voy a introducir un lema que no pertenece a la 
  teoría original de Isabelle pero facilita las siguientes 
  demostraciones detalladas mediante contenciones en cadena.

  \begin{lema}
    Sea \isa{G} subfórmula de \isa{F}, entonces el conjunto de subfórmulas de 
    \isa{G} está contenido en el de \isa{F}.
  \end{lema} 

  \begin{demostracion}
  La prueba es por inducción en la estructura de fórmula.
  
  Sea \isa{p} una fórmula atómica cualquiera. Entonces, bajo las
  condiciones del lema se tiene que \isa{G\ {\isacharequal}\ p}. Por lo tanto, tienen igual
  conjunto de subfórmulas.

  Sea la fórmula \isa{{\isasymbottom}}. Entonces, \isa{G\ {\isacharequal}\ {\isasymbottom}} y tienen igual conjunto de
  subfórmulas.

  Sea una fórmula \isa{F} tal que para toda subfórmula \isa{G}, se tiene que el
  conjunto de subfórmulas de \isa{G} está contenido en el de \isa{F}. Veamos la
  propiedad para \isa{{\isasymnot}\ F}. Sea \isa{G{\isacharprime}\ {\isasymin}\ Subf{\isacharparenleft}{\isasymnot}\ F{\isacharparenright}\ {\isacharequal}\ {\isacharbraceleft}{\isasymnot}\ F{\isacharbraceright}\ {\isasymunion}\ Subf{\isacharparenleft}F{\isacharparenright}}. 
  Entonces \isa{G{\isacharprime}\ {\isasymin}\ {\isacharbraceleft}{\isasymnot}\ F{\isacharbraceright}} o \isa{G{\isacharprime}\ {\isasymin}\ Subf{\isacharparenleft}F{\isacharparenright}}. 
  Del primer caso se obtiene que \isa{G{\isacharprime}\ {\isacharequal}\ {\isasymnot}\ F} y, por tanto, tienen igual 
  conjunto de subfórmulas. Del segundo caso se tiene \isa{G{\isacharprime}\ {\isasymin}\ Subf{\isacharparenleft}F{\isacharparenright}} y, 
  por hipótesis de inducción, el conjunto de subfórmulas de \isa{G{\isacharprime}} está 
  contenido en el de \isa{F}. Como, a su vez, el conjunto de subfórmulas 
  de \isa{F} está contenido en el de \isa{{\isasymnot}\ F} por definición, se verifica la
  propiedad como consecuencia de la transitividad de la contención.

  Sean las fórmulas \isa{F{\isadigit{1}}} y \isa{F{\isadigit{2}}} tales que para cualquier subfórmula \isa{G{\isadigit{1}}}
  de \isa{F{\isadigit{1}}} el conjunto de subfórmulas de \isa{G{\isadigit{1}}} está contenido en el de 
  \isa{F{\isadigit{1}}}, y para cualquier subfórmula \isa{G{\isadigit{2}}} de \isa{F{\isadigit{2}}} el conjunto de 
  subfórmulas de \isa{G{\isadigit{2}}} está contenido en el de \isa{F{\isadigit{2}}}. Veamos que se 
  verifica la propiedad para \isa{F{\isadigit{1}}{\isacharasterisk}F{\isadigit{2}}} donde \isa{{\isacharasterisk}} es cualquier conectiva 
  binaria. 
  Sea \isa{G{\isacharprime}\ {\isasymin}\ Subf{\isacharparenleft}F{\isadigit{1}}{\isacharasterisk}F{\isadigit{2}}{\isacharparenright}\ {\isacharequal}\ {\isacharbraceleft}F{\isadigit{1}}{\isacharasterisk}F{\isadigit{2}}{\isacharbraceright}\ {\isasymunion}\ Subf{\isacharparenleft}F{\isadigit{1}}{\isacharparenright}\ {\isasymunion}\ Subf{\isacharparenleft}F{\isadigit{2}}{\isacharparenright}}. De este modo,
  tenemos tres casos: \isa{G{\isacharprime}\ {\isasymin}\ {\isacharbraceleft}F{\isadigit{1}}{\isacharasterisk}F{\isadigit{2}}{\isacharbraceright}} o \isa{G{\isacharprime}\ {\isasymin}\ Subf{\isacharparenleft}F{\isadigit{1}}{\isacharparenright}} o 
  \isa{G{\isacharprime}\ {\isasymin}\ Subf{\isacharparenleft}F{\isadigit{2}}{\isacharparenright}}. De la primera opción se deduce \isa{G{\isacharprime}\ {\isacharequal}\ F{\isadigit{1}}{\isacharasterisk}F{\isadigit{2}}} y, por
  tanto, tienen igual conjunto de subfórmulas. Por otro lado, si 
  \isa{G{\isacharprime}\ {\isasymin}\ Subf{\isacharparenleft}F{\isadigit{1}}{\isacharparenright}}, por hipótesis de inducción se tiene que el conjunto
  de subfórmulas de \isa{G{\isacharprime}} está contenido en el de \isa{F{\isadigit{1}}}. Por tanto, 
  como el conjunto de subfórmulas de \isa{F{\isadigit{1}}} está a su vez contenido en el 
  de \isa{F{\isadigit{1}}{\isacharasterisk}F{\isadigit{2}}}, se verifica la propiedad por la transitividad de la 
  contención en cadena. El caso \isa{G{\isacharprime}\ {\isasymin}\ Subf{\isacharparenleft}F{\isadigit{2}}{\isacharparenright}} es análogo cambiando el 
  índice de la fórmula.   
  \end{demostracion}%
\end{isamarkuptext}\isamarkuptrue%
%
\begin{isamarkuptext}%
Veamos su formalización en Isabelle junto con su demostración 
  estructurada.%
\end{isamarkuptext}\isamarkuptrue%
\isacommand{lemma}\isamarkupfalse%
\ subContsubformulae{\isacharunderscore}atom{\isacharcolon}\ \isanewline
\ \ \isakeyword{assumes}\ {\isachardoublequoteopen}G\ {\isasymin}\ setSubformulae\ {\isacharparenleft}Atom\ x{\isacharparenright}{\isachardoublequoteclose}\ \isanewline
\ \ \isakeyword{shows}\ {\isachardoublequoteopen}setSubformulae\ G\ {\isasymsubseteq}\ setSubformulae\ {\isacharparenleft}Atom\ x{\isacharparenright}{\isachardoublequoteclose}\isanewline
%
\isadelimproof
%
\endisadelimproof
%
\isatagproof
\isacommand{proof}\isamarkupfalse%
\ {\isacharminus}\ \isanewline
\ \ \isacommand{have}\isamarkupfalse%
\ {\isachardoublequoteopen}G\ {\isasymin}\ {\isacharbraceleft}Atom\ x{\isacharbraceright}{\isachardoublequoteclose}\ \isacommand{using}\isamarkupfalse%
\ assms\ \isanewline
\ \ \ \ \isacommand{by}\isamarkupfalse%
\ {\isacharparenleft}simp\ only{\isacharcolon}\ setSubformulae{\isacharunderscore}atom{\isacharparenright}\isanewline
\ \ \isacommand{then}\isamarkupfalse%
\ \isacommand{have}\isamarkupfalse%
\ {\isachardoublequoteopen}G\ {\isacharequal}\ Atom\ x{\isachardoublequoteclose}\isanewline
\ \ \ \ \isacommand{by}\isamarkupfalse%
\ {\isacharparenleft}simp\ only{\isacharcolon}\ singletonD{\isacharparenright}\isanewline
\ \ \isacommand{then}\isamarkupfalse%
\ \isacommand{show}\isamarkupfalse%
\ {\isacharquery}thesis\isanewline
\ \ \ \ \isacommand{by}\isamarkupfalse%
\ {\isacharparenleft}simp\ only{\isacharcolon}\ subset{\isacharunderscore}refl{\isacharparenright}\isanewline
\isacommand{qed}\isamarkupfalse%
%
\endisatagproof
{\isafoldproof}%
%
\isadelimproof
\isanewline
%
\endisadelimproof
\isanewline
\isacommand{lemma}\isamarkupfalse%
\ subContsubformulae{\isacharunderscore}bot{\isacharcolon}\isanewline
\ \ \isakeyword{assumes}\ {\isachardoublequoteopen}G\ {\isasymin}\ setSubformulae\ {\isasymbottom}{\isachardoublequoteclose}\ \isanewline
\ \ \isakeyword{shows}\ \ \ {\isachardoublequoteopen}setSubformulae\ G\ {\isasymsubseteq}\ setSubformulae\ {\isasymbottom}{\isachardoublequoteclose}\isanewline
%
\isadelimproof
%
\endisadelimproof
%
\isatagproof
\isacommand{proof}\isamarkupfalse%
\ {\isacharminus}\isanewline
\ \ \isacommand{have}\isamarkupfalse%
\ {\isachardoublequoteopen}G\ {\isasymin}\ {\isacharbraceleft}{\isasymbottom}{\isacharbraceright}{\isachardoublequoteclose}\isanewline
\ \ \ \ \isacommand{using}\isamarkupfalse%
\ assms\isanewline
\ \ \ \ \isacommand{by}\isamarkupfalse%
\ {\isacharparenleft}simp\ only{\isacharcolon}\ setSubformulae{\isacharunderscore}bot{\isacharparenright}\isanewline
\ \ \isacommand{then}\isamarkupfalse%
\ \isacommand{have}\isamarkupfalse%
\ {\isachardoublequoteopen}G\ {\isacharequal}\ {\isasymbottom}{\isachardoublequoteclose}\isanewline
\ \ \ \ \isacommand{by}\isamarkupfalse%
\ {\isacharparenleft}simp\ only{\isacharcolon}\ singletonD{\isacharparenright}\isanewline
\ \ \isacommand{then}\isamarkupfalse%
\ \isacommand{show}\isamarkupfalse%
\ {\isacharquery}thesis\isanewline
\ \ \ \ \isacommand{by}\isamarkupfalse%
\ {\isacharparenleft}simp\ only{\isacharcolon}\ subset{\isacharunderscore}refl{\isacharparenright}\isanewline
\isacommand{qed}\isamarkupfalse%
%
\endisatagproof
{\isafoldproof}%
%
\isadelimproof
\isanewline
%
\endisadelimproof
\isanewline
\isacommand{lemma}\isamarkupfalse%
\ subContsubformulae{\isacharunderscore}not{\isacharcolon}\isanewline
\ \ \isakeyword{assumes}\ {\isachardoublequoteopen}G\ {\isasymin}\ setSubformulae\ F\ {\isasymLongrightarrow}\ setSubformulae\ G\ {\isasymsubseteq}\ setSubformulae\ F{\isachardoublequoteclose}\isanewline
\ \ \ \ \ \ \ \ \ \ {\isachardoublequoteopen}G\ {\isasymin}\ setSubformulae\ {\isacharparenleft}\isactrlbold {\isasymnot}\ F{\isacharparenright}{\isachardoublequoteclose}\isanewline
\ \ \isakeyword{shows}\ \ \ {\isachardoublequoteopen}setSubformulae\ G\ {\isasymsubseteq}\ setSubformulae\ {\isacharparenleft}\isactrlbold {\isasymnot}\ F{\isacharparenright}{\isachardoublequoteclose}\isanewline
%
\isadelimproof
%
\endisadelimproof
%
\isatagproof
\isacommand{proof}\isamarkupfalse%
\ {\isacharminus}\isanewline
\ \ \isacommand{have}\isamarkupfalse%
\ {\isachardoublequoteopen}G\ {\isasymin}\ {\isacharbraceleft}\isactrlbold {\isasymnot}\ F{\isacharbraceright}\ {\isasymunion}\ setSubformulae\ F{\isachardoublequoteclose}\isanewline
\ \ \ \ \isacommand{using}\isamarkupfalse%
\ assms{\isacharparenleft}{\isadigit{2}}{\isacharparenright}\isanewline
\ \ \ \ \isacommand{by}\isamarkupfalse%
\ {\isacharparenleft}simp\ only{\isacharcolon}\ setSubformulae{\isacharunderscore}not{\isacharparenright}\ \isanewline
\ \ \isacommand{then}\isamarkupfalse%
\ \isacommand{have}\isamarkupfalse%
\ {\isachardoublequoteopen}G\ {\isasymin}\ {\isacharbraceleft}\isactrlbold {\isasymnot}\ F{\isacharbraceright}\ {\isasymor}\ G\ {\isasymin}\ setSubformulae\ F{\isachardoublequoteclose}\isanewline
\ \ \ \ \isacommand{by}\isamarkupfalse%
\ {\isacharparenleft}simp\ only{\isacharcolon}\ Un{\isacharunderscore}iff{\isacharparenright}\isanewline
\ \ \isacommand{then}\isamarkupfalse%
\ \isacommand{show}\isamarkupfalse%
\ {\isachardoublequoteopen}setSubformulae\ G\ {\isasymsubseteq}\ setSubformulae\ {\isacharparenleft}\isactrlbold {\isasymnot}\ F{\isacharparenright}{\isachardoublequoteclose}\isanewline
\ \ \isacommand{proof}\isamarkupfalse%
\isanewline
\ \ \ \ \isacommand{assume}\isamarkupfalse%
\ {\isachardoublequoteopen}G\ {\isasymin}\ {\isacharbraceleft}\isactrlbold {\isasymnot}\ F{\isacharbraceright}{\isachardoublequoteclose}\isanewline
\ \ \ \ \isacommand{then}\isamarkupfalse%
\ \isacommand{have}\isamarkupfalse%
\ {\isachardoublequoteopen}G\ {\isacharequal}\ \isactrlbold {\isasymnot}\ F{\isachardoublequoteclose}\isanewline
\ \ \ \ \ \ \isacommand{by}\isamarkupfalse%
\ {\isacharparenleft}simp\ only{\isacharcolon}\ singletonD{\isacharparenright}\isanewline
\ \ \ \ \isacommand{then}\isamarkupfalse%
\ \isacommand{show}\isamarkupfalse%
\ {\isacharquery}thesis\isanewline
\ \ \ \ \ \ \isacommand{by}\isamarkupfalse%
\ {\isacharparenleft}simp\ only{\isacharcolon}\ subset{\isacharunderscore}refl{\isacharparenright}\isanewline
\ \ \isacommand{next}\isamarkupfalse%
\isanewline
\ \ \ \ \isacommand{assume}\isamarkupfalse%
\ {\isachardoublequoteopen}G\ {\isasymin}\ setSubformulae\ F{\isachardoublequoteclose}\isanewline
\ \ \ \ \isacommand{then}\isamarkupfalse%
\ \isacommand{have}\isamarkupfalse%
\ {\isadigit{1}}{\isacharcolon}{\isachardoublequoteopen}setSubformulae\ G\ {\isasymsubseteq}\ setSubformulae\ F{\isachardoublequoteclose}\isanewline
\ \ \ \ \ \ \isacommand{by}\isamarkupfalse%
\ {\isacharparenleft}simp\ only{\isacharcolon}\ assms{\isacharparenleft}{\isadigit{1}}{\isacharparenright}{\isacharparenright}\isanewline
\ \ \ \ \isacommand{also}\isamarkupfalse%
\ \isacommand{have}\isamarkupfalse%
\ {\isadigit{2}}{\isacharcolon}{\isachardoublequoteopen}setSubformulae\ F\ {\isasymsubseteq}\ setSubformulae\ {\isacharparenleft}\isactrlbold {\isasymnot}\ F{\isacharparenright}{\isachardoublequoteclose}\isanewline
\ \ \ \ \ \ \isacommand{by}\isamarkupfalse%
\ {\isacharparenleft}simp\ only{\isacharcolon}\ setSubformulae{\isacharunderscore}not\ Un{\isacharunderscore}upper{\isadigit{2}}{\isacharparenright}\isanewline
\ \ \ \ \isacommand{finally}\isamarkupfalse%
\ \isacommand{show}\isamarkupfalse%
\ {\isacharquery}thesis\isanewline
\ \ \ \ \ \ \isacommand{using}\isamarkupfalse%
\ {\isadigit{1}}\ {\isadigit{2}}\ \isacommand{by}\isamarkupfalse%
\ {\isacharparenleft}simp\ only{\isacharcolon}\ subset{\isacharunderscore}trans{\isacharparenright}\isanewline
\ \ \isacommand{qed}\isamarkupfalse%
\isanewline
\isacommand{qed}\isamarkupfalse%
%
\endisatagproof
{\isafoldproof}%
%
\isadelimproof
\isanewline
%
\endisadelimproof
\isanewline
\isacommand{lemma}\isamarkupfalse%
\ subContsubformulae{\isacharunderscore}and{\isacharcolon}\isanewline
\ \ \isakeyword{assumes}\ {\isachardoublequoteopen}G\ {\isasymin}\ setSubformulae\ F{\isadigit{1}}\ \isanewline
\ \ \ \ \ \ \ \ \ \ \ \ {\isasymLongrightarrow}\ setSubformulae\ G\ {\isasymsubseteq}\ setSubformulae\ F{\isadigit{1}}{\isachardoublequoteclose}\isanewline
\ \ \ \ \ \ \ \ \ \ {\isachardoublequoteopen}G\ {\isasymin}\ setSubformulae\ F{\isadigit{2}}\ \isanewline
\ \ \ \ \ \ \ \ \ \ \ \ {\isasymLongrightarrow}\ setSubformulae\ G\ {\isasymsubseteq}\ setSubformulae\ F{\isadigit{2}}{\isachardoublequoteclose}\isanewline
\ \ \ \ \ \ \ \ \ \ {\isachardoublequoteopen}G\ {\isasymin}\ setSubformulae\ {\isacharparenleft}F{\isadigit{1}}\ \isactrlbold {\isasymand}\ F{\isadigit{2}}{\isacharparenright}{\isachardoublequoteclose}\isanewline
\ \ \isakeyword{shows}\ \ \ {\isachardoublequoteopen}setSubformulae\ G\ {\isasymsubseteq}\ setSubformulae\ {\isacharparenleft}F{\isadigit{1}}\ \isactrlbold {\isasymand}\ F{\isadigit{2}}{\isacharparenright}{\isachardoublequoteclose}\isanewline
%
\isadelimproof
%
\endisadelimproof
%
\isatagproof
\isacommand{proof}\isamarkupfalse%
\ {\isacharminus}\isanewline
\ \ \isacommand{have}\isamarkupfalse%
\ {\isachardoublequoteopen}G\ {\isasymin}\ {\isacharbraceleft}F{\isadigit{1}}\ \isactrlbold {\isasymand}\ F{\isadigit{2}}{\isacharbraceright}\ {\isasymunion}\ {\isacharparenleft}setSubformulae\ F{\isadigit{1}}\ {\isasymunion}\ setSubformulae\ F{\isadigit{2}}{\isacharparenright}{\isachardoublequoteclose}\isanewline
\ \ \ \ \isacommand{using}\isamarkupfalse%
\ assms{\isacharparenleft}{\isadigit{3}}{\isacharparenright}\ \isanewline
\ \ \ \ \isacommand{by}\isamarkupfalse%
\ {\isacharparenleft}simp\ only{\isacharcolon}\ setSubformulae{\isacharunderscore}and{\isacharparenright}\isanewline
\ \ \isacommand{then}\isamarkupfalse%
\ \isacommand{have}\isamarkupfalse%
\ {\isachardoublequoteopen}G\ {\isasymin}\ {\isacharbraceleft}F{\isadigit{1}}\ \isactrlbold {\isasymand}\ F{\isadigit{2}}{\isacharbraceright}\ {\isasymor}\ G\ {\isasymin}\ setSubformulae\ F{\isadigit{1}}\ {\isasymunion}\ setSubformulae\ F{\isadigit{2}}{\isachardoublequoteclose}\isanewline
\ \ \ \ \isacommand{by}\isamarkupfalse%
\ {\isacharparenleft}simp\ only{\isacharcolon}\ Un{\isacharunderscore}iff{\isacharparenright}\isanewline
\ \ \isacommand{then}\isamarkupfalse%
\ \isacommand{show}\isamarkupfalse%
\ {\isacharquery}thesis\isanewline
\ \ \isacommand{proof}\isamarkupfalse%
\ \isanewline
\ \ \ \ \isacommand{assume}\isamarkupfalse%
\ {\isachardoublequoteopen}G\ {\isasymin}\ {\isacharbraceleft}F{\isadigit{1}}\ \isactrlbold {\isasymand}\ F{\isadigit{2}}{\isacharbraceright}{\isachardoublequoteclose}\isanewline
\ \ \ \ \isacommand{then}\isamarkupfalse%
\ \isacommand{have}\isamarkupfalse%
\ {\isachardoublequoteopen}G\ {\isacharequal}\ F{\isadigit{1}}\ \isactrlbold {\isasymand}\ F{\isadigit{2}}{\isachardoublequoteclose}\isanewline
\ \ \ \ \ \ \isacommand{by}\isamarkupfalse%
\ {\isacharparenleft}simp\ only{\isacharcolon}\ singletonD{\isacharparenright}\isanewline
\ \ \ \ \isacommand{then}\isamarkupfalse%
\ \isacommand{show}\isamarkupfalse%
\ {\isacharquery}thesis\isanewline
\ \ \ \ \ \ \isacommand{by}\isamarkupfalse%
\ {\isacharparenleft}simp\ only{\isacharcolon}\ subset{\isacharunderscore}refl{\isacharparenright}\isanewline
\ \ \isacommand{next}\isamarkupfalse%
\isanewline
\ \ \ \ \isacommand{assume}\isamarkupfalse%
\ {\isachardoublequoteopen}G\ {\isasymin}\ setSubformulae\ F{\isadigit{1}}\ {\isasymunion}\ setSubformulae\ F{\isadigit{2}}{\isachardoublequoteclose}\isanewline
\ \ \ \ \isacommand{then}\isamarkupfalse%
\ \isacommand{have}\isamarkupfalse%
\ {\isachardoublequoteopen}G\ {\isasymin}\ setSubformulae\ F{\isadigit{1}}\ {\isasymor}\ G\ {\isasymin}\ setSubformulae\ F{\isadigit{2}}{\isachardoublequoteclose}\ \ \isanewline
\ \ \ \ \ \ \isacommand{by}\isamarkupfalse%
\ {\isacharparenleft}simp\ only{\isacharcolon}\ Un{\isacharunderscore}iff{\isacharparenright}\isanewline
\ \ \ \ \isacommand{then}\isamarkupfalse%
\ \isacommand{show}\isamarkupfalse%
\ {\isacharquery}thesis\isanewline
\ \ \ \ \isacommand{proof}\isamarkupfalse%
\ \isanewline
\ \ \ \ \ \ \isacommand{assume}\isamarkupfalse%
\ {\isachardoublequoteopen}G\ {\isasymin}\ setSubformulae\ F{\isadigit{1}}{\isachardoublequoteclose}\isanewline
\ \ \ \ \ \ \isacommand{then}\isamarkupfalse%
\ \isacommand{have}\isamarkupfalse%
\ {\isachardoublequoteopen}setSubformulae\ G\ {\isasymsubseteq}\ setSubformulae\ F{\isadigit{1}}{\isachardoublequoteclose}\isanewline
\ \ \ \ \ \ \ \ \isacommand{by}\isamarkupfalse%
\ {\isacharparenleft}simp\ only{\isacharcolon}\ assms{\isacharparenleft}{\isadigit{1}}{\isacharparenright}{\isacharparenright}\isanewline
\ \ \ \ \ \ \isacommand{also}\isamarkupfalse%
\ \isacommand{have}\isamarkupfalse%
\ {\isachardoublequoteopen}{\isasymdots}\ {\isasymsubseteq}\ setSubformulae\ F{\isadigit{1}}\ {\isasymunion}\ setSubformulae\ F{\isadigit{2}}{\isachardoublequoteclose}\isanewline
\ \ \ \ \ \ \ \ \isacommand{by}\isamarkupfalse%
\ {\isacharparenleft}simp\ only{\isacharcolon}\ Un{\isacharunderscore}upper{\isadigit{1}}{\isacharparenright}\isanewline
\ \ \ \ \ \ \isacommand{also}\isamarkupfalse%
\ \isacommand{have}\isamarkupfalse%
\ {\isachardoublequoteopen}{\isasymdots}\ {\isasymsubseteq}\ setSubformulae\ {\isacharparenleft}F{\isadigit{1}}\ \isactrlbold {\isasymand}\ F{\isadigit{2}}{\isacharparenright}{\isachardoublequoteclose}\isanewline
\ \ \ \ \ \ \ \ \isacommand{by}\isamarkupfalse%
\ {\isacharparenleft}simp\ only{\isacharcolon}\ setSubformulae{\isacharunderscore}and\ Un{\isacharunderscore}upper{\isadigit{2}}{\isacharparenright}\isanewline
\ \ \ \ \ \ \isacommand{finally}\isamarkupfalse%
\ \isacommand{show}\isamarkupfalse%
\ {\isacharquery}thesis\isanewline
\ \ \ \ \ \ \ \ \isacommand{by}\isamarkupfalse%
\ this\isanewline
\ \ \ \ \isacommand{next}\isamarkupfalse%
\isanewline
\ \ \ \ \ \ \isacommand{assume}\isamarkupfalse%
\ {\isachardoublequoteopen}G\ {\isasymin}\ setSubformulae\ F{\isadigit{2}}{\isachardoublequoteclose}\isanewline
\ \ \ \ \ \ \isacommand{then}\isamarkupfalse%
\ \isacommand{have}\isamarkupfalse%
\ {\isachardoublequoteopen}setSubformulae\ G\ {\isasymsubseteq}\ setSubformulae\ F{\isadigit{2}}{\isachardoublequoteclose}\isanewline
\ \ \ \ \ \ \ \ \isacommand{by}\isamarkupfalse%
\ {\isacharparenleft}rule\ assms{\isacharparenleft}{\isadigit{2}}{\isacharparenright}{\isacharparenright}\isanewline
\ \ \ \ \ \ \isacommand{also}\isamarkupfalse%
\ \isacommand{have}\isamarkupfalse%
\ {\isachardoublequoteopen}{\isasymdots}\ {\isasymsubseteq}\ setSubformulae\ F{\isadigit{1}}\ {\isasymunion}\ setSubformulae\ F{\isadigit{2}}{\isachardoublequoteclose}\isanewline
\ \ \ \ \ \ \ \ \isacommand{by}\isamarkupfalse%
\ {\isacharparenleft}simp\ only{\isacharcolon}\ Un{\isacharunderscore}upper{\isadigit{2}}{\isacharparenright}\isanewline
\ \ \ \ \ \ \isacommand{also}\isamarkupfalse%
\ \isacommand{have}\isamarkupfalse%
\ {\isachardoublequoteopen}{\isasymdots}\ {\isasymsubseteq}\ setSubformulae\ {\isacharparenleft}F{\isadigit{1}}\ \isactrlbold {\isasymand}\ F{\isadigit{2}}{\isacharparenright}{\isachardoublequoteclose}\isanewline
\ \ \ \ \ \ \ \ \isacommand{by}\isamarkupfalse%
\ {\isacharparenleft}simp\ only{\isacharcolon}\ setSubformulae{\isacharunderscore}and\ Un{\isacharunderscore}upper{\isadigit{2}}{\isacharparenright}\isanewline
\ \ \ \ \ \ \isacommand{finally}\isamarkupfalse%
\ \isacommand{show}\isamarkupfalse%
\ {\isacharquery}thesis\isanewline
\ \ \ \ \ \ \ \ \isacommand{by}\isamarkupfalse%
\ this\isanewline
\ \ \ \ \isacommand{qed}\isamarkupfalse%
\isanewline
\ \ \isacommand{qed}\isamarkupfalse%
\isanewline
\isacommand{qed}\isamarkupfalse%
%
\endisatagproof
{\isafoldproof}%
%
\isadelimproof
\isanewline
%
\endisadelimproof
\isanewline
\isacommand{lemma}\isamarkupfalse%
\ subContsubformulae{\isacharunderscore}or{\isacharcolon}\isanewline
\ \ \isakeyword{assumes}\ {\isachardoublequoteopen}G\ {\isasymin}\ setSubformulae\ F{\isadigit{1}}\ \isanewline
\ \ \ \ \ \ \ \ \ \ \ \ {\isasymLongrightarrow}\ setSubformulae\ G\ {\isasymsubseteq}\ setSubformulae\ F{\isadigit{1}}{\isachardoublequoteclose}\isanewline
\ \ \ \ \ \ \ \ \ \ {\isachardoublequoteopen}G\ {\isasymin}\ setSubformulae\ F{\isadigit{2}}\ \isanewline
\ \ \ \ \ \ \ \ \ \ \ \ {\isasymLongrightarrow}\ setSubformulae\ G\ {\isasymsubseteq}\ setSubformulae\ F{\isadigit{2}}{\isachardoublequoteclose}\isanewline
\ \ \ \ \ \ \ \ \ \ {\isachardoublequoteopen}G\ {\isasymin}\ setSubformulae\ {\isacharparenleft}F{\isadigit{1}}\ \isactrlbold {\isasymor}\ F{\isadigit{2}}{\isacharparenright}{\isachardoublequoteclose}\isanewline
\ \ \isakeyword{shows}\ \ \ {\isachardoublequoteopen}setSubformulae\ G\ {\isasymsubseteq}\ setSubformulae\ {\isacharparenleft}F{\isadigit{1}}\ \isactrlbold {\isasymor}\ F{\isadigit{2}}{\isacharparenright}{\isachardoublequoteclose}\isanewline
%
\isadelimproof
%
\endisadelimproof
%
\isatagproof
\isacommand{proof}\isamarkupfalse%
\ {\isacharminus}\isanewline
\ \ \isacommand{have}\isamarkupfalse%
\ {\isachardoublequoteopen}G\ {\isasymin}\ {\isacharbraceleft}F{\isadigit{1}}\ \isactrlbold {\isasymor}\ F{\isadigit{2}}{\isacharbraceright}\ {\isasymunion}\ {\isacharparenleft}setSubformulae\ F{\isadigit{1}}\ {\isasymunion}\ setSubformulae\ F{\isadigit{2}}{\isacharparenright}{\isachardoublequoteclose}\isanewline
\ \ \ \ \isacommand{using}\isamarkupfalse%
\ assms{\isacharparenleft}{\isadigit{3}}{\isacharparenright}\ \isanewline
\ \ \ \ \isacommand{by}\isamarkupfalse%
\ {\isacharparenleft}simp\ only{\isacharcolon}\ setSubformulae{\isacharunderscore}or{\isacharparenright}\isanewline
\ \ \isacommand{then}\isamarkupfalse%
\ \isacommand{have}\isamarkupfalse%
\ {\isachardoublequoteopen}G\ {\isasymin}\ {\isacharbraceleft}F{\isadigit{1}}\ \isactrlbold {\isasymor}\ F{\isadigit{2}}{\isacharbraceright}\ {\isasymor}\ G\ {\isasymin}\ setSubformulae\ F{\isadigit{1}}\ {\isasymunion}\ setSubformulae\ F{\isadigit{2}}{\isachardoublequoteclose}\isanewline
\ \ \ \ \isacommand{by}\isamarkupfalse%
\ {\isacharparenleft}simp\ only{\isacharcolon}\ Un{\isacharunderscore}iff{\isacharparenright}\isanewline
\ \ \isacommand{then}\isamarkupfalse%
\ \isacommand{show}\isamarkupfalse%
\ {\isacharquery}thesis\isanewline
\ \ \isacommand{proof}\isamarkupfalse%
\ \isanewline
\ \ \ \ \isacommand{assume}\isamarkupfalse%
\ {\isachardoublequoteopen}G\ {\isasymin}\ {\isacharbraceleft}F{\isadigit{1}}\ \isactrlbold {\isasymor}\ F{\isadigit{2}}{\isacharbraceright}{\isachardoublequoteclose}\isanewline
\ \ \ \ \isacommand{then}\isamarkupfalse%
\ \isacommand{have}\isamarkupfalse%
\ {\isachardoublequoteopen}G\ {\isacharequal}\ F{\isadigit{1}}\ \isactrlbold {\isasymor}\ F{\isadigit{2}}{\isachardoublequoteclose}\isanewline
\ \ \ \ \ \ \isacommand{by}\isamarkupfalse%
\ {\isacharparenleft}simp\ only{\isacharcolon}\ singletonD{\isacharparenright}\isanewline
\ \ \ \ \isacommand{then}\isamarkupfalse%
\ \isacommand{show}\isamarkupfalse%
\ {\isacharquery}thesis\isanewline
\ \ \ \ \ \ \isacommand{by}\isamarkupfalse%
\ {\isacharparenleft}simp\ only{\isacharcolon}\ subset{\isacharunderscore}refl{\isacharparenright}\isanewline
\ \ \isacommand{next}\isamarkupfalse%
\isanewline
\ \ \ \ \isacommand{assume}\isamarkupfalse%
\ {\isachardoublequoteopen}G\ {\isasymin}\ setSubformulae\ F{\isadigit{1}}\ {\isasymunion}\ setSubformulae\ F{\isadigit{2}}{\isachardoublequoteclose}\isanewline
\ \ \ \ \isacommand{then}\isamarkupfalse%
\ \isacommand{have}\isamarkupfalse%
\ {\isachardoublequoteopen}G\ {\isasymin}\ setSubformulae\ F{\isadigit{1}}\ {\isasymor}\ G\ {\isasymin}\ setSubformulae\ F{\isadigit{2}}{\isachardoublequoteclose}\ \ \isanewline
\ \ \ \ \ \ \isacommand{by}\isamarkupfalse%
\ {\isacharparenleft}simp\ only{\isacharcolon}\ Un{\isacharunderscore}iff{\isacharparenright}\isanewline
\ \ \ \ \isacommand{then}\isamarkupfalse%
\ \isacommand{show}\isamarkupfalse%
\ {\isacharquery}thesis\isanewline
\ \ \ \ \isacommand{proof}\isamarkupfalse%
\ \isanewline
\ \ \ \ \ \ \isacommand{assume}\isamarkupfalse%
\ {\isachardoublequoteopen}G\ {\isasymin}\ setSubformulae\ F{\isadigit{1}}{\isachardoublequoteclose}\isanewline
\ \ \ \ \ \ \isacommand{then}\isamarkupfalse%
\ \isacommand{have}\isamarkupfalse%
\ {\isachardoublequoteopen}setSubformulae\ G\ {\isasymsubseteq}\ setSubformulae\ F{\isadigit{1}}{\isachardoublequoteclose}\isanewline
\ \ \ \ \ \ \ \ \isacommand{by}\isamarkupfalse%
\ {\isacharparenleft}simp\ only{\isacharcolon}\ assms{\isacharparenleft}{\isadigit{1}}{\isacharparenright}{\isacharparenright}\isanewline
\ \ \ \ \ \ \isacommand{also}\isamarkupfalse%
\ \isacommand{have}\isamarkupfalse%
\ {\isachardoublequoteopen}{\isasymdots}\ {\isasymsubseteq}\ setSubformulae\ F{\isadigit{1}}\ {\isasymunion}\ setSubformulae\ F{\isadigit{2}}{\isachardoublequoteclose}\isanewline
\ \ \ \ \ \ \ \ \isacommand{by}\isamarkupfalse%
\ {\isacharparenleft}simp\ only{\isacharcolon}\ Un{\isacharunderscore}upper{\isadigit{1}}{\isacharparenright}\isanewline
\ \ \ \ \ \ \isacommand{also}\isamarkupfalse%
\ \isacommand{have}\isamarkupfalse%
\ {\isachardoublequoteopen}{\isasymdots}\ {\isasymsubseteq}\ setSubformulae\ {\isacharparenleft}F{\isadigit{1}}\ \isactrlbold {\isasymor}\ F{\isadigit{2}}{\isacharparenright}{\isachardoublequoteclose}\isanewline
\ \ \ \ \ \ \ \ \isacommand{by}\isamarkupfalse%
\ {\isacharparenleft}simp\ only{\isacharcolon}\ setSubformulae{\isacharunderscore}or\ Un{\isacharunderscore}upper{\isadigit{2}}{\isacharparenright}\isanewline
\ \ \ \ \ \ \isacommand{finally}\isamarkupfalse%
\ \isacommand{show}\isamarkupfalse%
\ {\isacharquery}thesis\isanewline
\ \ \ \ \ \ \ \ \isacommand{by}\isamarkupfalse%
\ this\isanewline
\ \ \ \ \isacommand{next}\isamarkupfalse%
\isanewline
\ \ \ \ \ \ \isacommand{assume}\isamarkupfalse%
\ {\isachardoublequoteopen}G\ {\isasymin}\ setSubformulae\ F{\isadigit{2}}{\isachardoublequoteclose}\isanewline
\ \ \ \ \ \ \isacommand{then}\isamarkupfalse%
\ \isacommand{have}\isamarkupfalse%
\ {\isachardoublequoteopen}setSubformulae\ G\ {\isasymsubseteq}\ setSubformulae\ F{\isadigit{2}}{\isachardoublequoteclose}\isanewline
\ \ \ \ \ \ \ \ \isacommand{by}\isamarkupfalse%
\ {\isacharparenleft}rule\ assms{\isacharparenleft}{\isadigit{2}}{\isacharparenright}{\isacharparenright}\isanewline
\ \ \ \ \ \ \isacommand{also}\isamarkupfalse%
\ \isacommand{have}\isamarkupfalse%
\ {\isachardoublequoteopen}{\isasymdots}\ {\isasymsubseteq}\ setSubformulae\ F{\isadigit{1}}\ {\isasymunion}\ setSubformulae\ F{\isadigit{2}}{\isachardoublequoteclose}\isanewline
\ \ \ \ \ \ \ \ \isacommand{by}\isamarkupfalse%
\ {\isacharparenleft}simp\ only{\isacharcolon}\ Un{\isacharunderscore}upper{\isadigit{2}}{\isacharparenright}\isanewline
\ \ \ \ \ \ \isacommand{also}\isamarkupfalse%
\ \isacommand{have}\isamarkupfalse%
\ {\isachardoublequoteopen}{\isasymdots}\ {\isasymsubseteq}\ setSubformulae\ {\isacharparenleft}F{\isadigit{1}}\ \isactrlbold {\isasymor}\ F{\isadigit{2}}{\isacharparenright}{\isachardoublequoteclose}\isanewline
\ \ \ \ \ \ \ \ \isacommand{by}\isamarkupfalse%
\ {\isacharparenleft}simp\ only{\isacharcolon}\ setSubformulae{\isacharunderscore}or\ Un{\isacharunderscore}upper{\isadigit{2}}{\isacharparenright}\isanewline
\ \ \ \ \ \ \isacommand{finally}\isamarkupfalse%
\ \isacommand{show}\isamarkupfalse%
\ {\isacharquery}thesis\isanewline
\ \ \ \ \ \ \ \ \isacommand{by}\isamarkupfalse%
\ this\isanewline
\ \ \ \ \isacommand{qed}\isamarkupfalse%
\isanewline
\ \ \isacommand{qed}\isamarkupfalse%
\isanewline
\isacommand{qed}\isamarkupfalse%
%
\endisatagproof
{\isafoldproof}%
%
\isadelimproof
\isanewline
%
\endisadelimproof
\isanewline
\isacommand{lemma}\isamarkupfalse%
\ subContsubformulae{\isacharunderscore}imp{\isacharcolon}\isanewline
\ \ \isakeyword{assumes}\ {\isachardoublequoteopen}G\ {\isasymin}\ setSubformulae\ F{\isadigit{1}}\ \isanewline
\ \ \ \ \ \ \ \ \ \ \ \ {\isasymLongrightarrow}\ setSubformulae\ G\ {\isasymsubseteq}\ setSubformulae\ F{\isadigit{1}}{\isachardoublequoteclose}\isanewline
\ \ \ \ \ \ \ \ \ \ {\isachardoublequoteopen}G\ {\isasymin}\ setSubformulae\ F{\isadigit{2}}\ \isanewline
\ \ \ \ \ \ \ \ \ \ \ \ {\isasymLongrightarrow}\ setSubformulae\ G\ {\isasymsubseteq}\ setSubformulae\ F{\isadigit{2}}{\isachardoublequoteclose}\isanewline
\ \ \ \ \ \ \ \ \ \ {\isachardoublequoteopen}G\ {\isasymin}\ setSubformulae\ {\isacharparenleft}F{\isadigit{1}}\ \isactrlbold {\isasymrightarrow}\ F{\isadigit{2}}{\isacharparenright}{\isachardoublequoteclose}\isanewline
\ \ \isakeyword{shows}\ \ \ {\isachardoublequoteopen}setSubformulae\ G\ {\isasymsubseteq}\ setSubformulae\ {\isacharparenleft}F{\isadigit{1}}\ \isactrlbold {\isasymrightarrow}\ F{\isadigit{2}}{\isacharparenright}{\isachardoublequoteclose}\isanewline
%
\isadelimproof
%
\endisadelimproof
%
\isatagproof
\isacommand{proof}\isamarkupfalse%
\ {\isacharminus}\isanewline
\ \ \isacommand{have}\isamarkupfalse%
\ {\isachardoublequoteopen}G\ {\isasymin}\ {\isacharbraceleft}F{\isadigit{1}}\ \isactrlbold {\isasymrightarrow}\ F{\isadigit{2}}{\isacharbraceright}\ {\isasymunion}\ {\isacharparenleft}setSubformulae\ F{\isadigit{1}}\ {\isasymunion}\ setSubformulae\ F{\isadigit{2}}{\isacharparenright}{\isachardoublequoteclose}\isanewline
\ \ \ \ \isacommand{using}\isamarkupfalse%
\ assms{\isacharparenleft}{\isadigit{3}}{\isacharparenright}\ \isanewline
\ \ \ \ \isacommand{by}\isamarkupfalse%
\ {\isacharparenleft}simp\ only{\isacharcolon}\ setSubformulae{\isacharunderscore}imp{\isacharparenright}\isanewline
\ \ \isacommand{then}\isamarkupfalse%
\ \isacommand{have}\isamarkupfalse%
\ {\isachardoublequoteopen}G\ {\isasymin}\ {\isacharbraceleft}F{\isadigit{1}}\ \isactrlbold {\isasymrightarrow}\ F{\isadigit{2}}{\isacharbraceright}\ {\isasymor}\ G\ {\isasymin}\ setSubformulae\ F{\isadigit{1}}\ {\isasymunion}\ setSubformulae\ F{\isadigit{2}}{\isachardoublequoteclose}\isanewline
\ \ \ \ \isacommand{by}\isamarkupfalse%
\ {\isacharparenleft}simp\ only{\isacharcolon}\ Un{\isacharunderscore}iff{\isacharparenright}\isanewline
\ \ \isacommand{then}\isamarkupfalse%
\ \isacommand{show}\isamarkupfalse%
\ {\isacharquery}thesis\isanewline
\ \ \isacommand{proof}\isamarkupfalse%
\ \isanewline
\ \ \ \ \isacommand{assume}\isamarkupfalse%
\ {\isachardoublequoteopen}G\ {\isasymin}\ {\isacharbraceleft}F{\isadigit{1}}\ \isactrlbold {\isasymrightarrow}\ F{\isadigit{2}}{\isacharbraceright}{\isachardoublequoteclose}\isanewline
\ \ \ \ \isacommand{then}\isamarkupfalse%
\ \isacommand{have}\isamarkupfalse%
\ {\isachardoublequoteopen}G\ {\isacharequal}\ F{\isadigit{1}}\ \isactrlbold {\isasymrightarrow}\ F{\isadigit{2}}{\isachardoublequoteclose}\isanewline
\ \ \ \ \ \ \isacommand{by}\isamarkupfalse%
\ {\isacharparenleft}simp\ only{\isacharcolon}\ singletonD{\isacharparenright}\isanewline
\ \ \ \ \isacommand{then}\isamarkupfalse%
\ \isacommand{show}\isamarkupfalse%
\ {\isacharquery}thesis\isanewline
\ \ \ \ \ \ \isacommand{by}\isamarkupfalse%
\ {\isacharparenleft}simp\ only{\isacharcolon}\ subset{\isacharunderscore}refl{\isacharparenright}\isanewline
\ \ \isacommand{next}\isamarkupfalse%
\isanewline
\ \ \ \ \isacommand{assume}\isamarkupfalse%
\ {\isachardoublequoteopen}G\ {\isasymin}\ setSubformulae\ F{\isadigit{1}}\ {\isasymunion}\ setSubformulae\ F{\isadigit{2}}{\isachardoublequoteclose}\isanewline
\ \ \ \ \isacommand{then}\isamarkupfalse%
\ \isacommand{have}\isamarkupfalse%
\ {\isachardoublequoteopen}G\ {\isasymin}\ setSubformulae\ F{\isadigit{1}}\ {\isasymor}\ G\ {\isasymin}\ setSubformulae\ F{\isadigit{2}}{\isachardoublequoteclose}\ \ \isanewline
\ \ \ \ \ \ \isacommand{by}\isamarkupfalse%
\ {\isacharparenleft}simp\ only{\isacharcolon}\ Un{\isacharunderscore}iff{\isacharparenright}\isanewline
\ \ \ \ \isacommand{then}\isamarkupfalse%
\ \isacommand{show}\isamarkupfalse%
\ {\isacharquery}thesis\isanewline
\ \ \ \ \isacommand{proof}\isamarkupfalse%
\ \isanewline
\ \ \ \ \ \ \isacommand{assume}\isamarkupfalse%
\ {\isachardoublequoteopen}G\ {\isasymin}\ setSubformulae\ F{\isadigit{1}}{\isachardoublequoteclose}\isanewline
\ \ \ \ \ \ \isacommand{then}\isamarkupfalse%
\ \isacommand{have}\isamarkupfalse%
\ {\isachardoublequoteopen}setSubformulae\ G\ {\isasymsubseteq}\ setSubformulae\ F{\isadigit{1}}{\isachardoublequoteclose}\isanewline
\ \ \ \ \ \ \ \ \isacommand{by}\isamarkupfalse%
\ {\isacharparenleft}simp\ only{\isacharcolon}\ assms{\isacharparenleft}{\isadigit{1}}{\isacharparenright}{\isacharparenright}\isanewline
\ \ \ \ \ \ \isacommand{also}\isamarkupfalse%
\ \isacommand{have}\isamarkupfalse%
\ {\isachardoublequoteopen}{\isasymdots}\ {\isasymsubseteq}\ setSubformulae\ F{\isadigit{1}}\ {\isasymunion}\ setSubformulae\ F{\isadigit{2}}{\isachardoublequoteclose}\isanewline
\ \ \ \ \ \ \ \ \isacommand{by}\isamarkupfalse%
\ {\isacharparenleft}simp\ only{\isacharcolon}\ Un{\isacharunderscore}upper{\isadigit{1}}{\isacharparenright}\isanewline
\ \ \ \ \ \ \isacommand{also}\isamarkupfalse%
\ \isacommand{have}\isamarkupfalse%
\ {\isachardoublequoteopen}{\isasymdots}\ {\isasymsubseteq}\ setSubformulae\ {\isacharparenleft}F{\isadigit{1}}\ \isactrlbold {\isasymrightarrow}\ F{\isadigit{2}}{\isacharparenright}{\isachardoublequoteclose}\isanewline
\ \ \ \ \ \ \ \ \isacommand{by}\isamarkupfalse%
\ {\isacharparenleft}simp\ only{\isacharcolon}\ setSubformulae{\isacharunderscore}imp\ Un{\isacharunderscore}upper{\isadigit{2}}{\isacharparenright}\isanewline
\ \ \ \ \ \ \isacommand{finally}\isamarkupfalse%
\ \isacommand{show}\isamarkupfalse%
\ {\isacharquery}thesis\isanewline
\ \ \ \ \ \ \ \ \isacommand{by}\isamarkupfalse%
\ this\isanewline
\ \ \ \ \isacommand{next}\isamarkupfalse%
\isanewline
\ \ \ \ \ \ \isacommand{assume}\isamarkupfalse%
\ {\isachardoublequoteopen}G\ {\isasymin}\ setSubformulae\ F{\isadigit{2}}{\isachardoublequoteclose}\isanewline
\ \ \ \ \ \ \isacommand{then}\isamarkupfalse%
\ \isacommand{have}\isamarkupfalse%
\ {\isachardoublequoteopen}setSubformulae\ G\ {\isasymsubseteq}\ setSubformulae\ F{\isadigit{2}}{\isachardoublequoteclose}\isanewline
\ \ \ \ \ \ \ \ \isacommand{by}\isamarkupfalse%
\ {\isacharparenleft}rule\ assms{\isacharparenleft}{\isadigit{2}}{\isacharparenright}{\isacharparenright}\isanewline
\ \ \ \ \ \ \isacommand{also}\isamarkupfalse%
\ \isacommand{have}\isamarkupfalse%
\ {\isachardoublequoteopen}{\isasymdots}\ {\isasymsubseteq}\ setSubformulae\ F{\isadigit{1}}\ {\isasymunion}\ setSubformulae\ F{\isadigit{2}}{\isachardoublequoteclose}\isanewline
\ \ \ \ \ \ \ \ \isacommand{by}\isamarkupfalse%
\ {\isacharparenleft}simp\ only{\isacharcolon}\ Un{\isacharunderscore}upper{\isadigit{2}}{\isacharparenright}\isanewline
\ \ \ \ \ \ \isacommand{also}\isamarkupfalse%
\ \isacommand{have}\isamarkupfalse%
\ {\isachardoublequoteopen}{\isasymdots}\ {\isasymsubseteq}\ setSubformulae\ {\isacharparenleft}F{\isadigit{1}}\ \isactrlbold {\isasymrightarrow}\ F{\isadigit{2}}{\isacharparenright}{\isachardoublequoteclose}\isanewline
\ \ \ \ \ \ \ \ \isacommand{by}\isamarkupfalse%
\ {\isacharparenleft}simp\ only{\isacharcolon}\ setSubformulae{\isacharunderscore}imp\ Un{\isacharunderscore}upper{\isadigit{2}}{\isacharparenright}\isanewline
\ \ \ \ \ \ \isacommand{finally}\isamarkupfalse%
\ \isacommand{show}\isamarkupfalse%
\ {\isacharquery}thesis\isanewline
\ \ \ \ \ \ \ \ \isacommand{by}\isamarkupfalse%
\ this\isanewline
\ \ \ \ \isacommand{qed}\isamarkupfalse%
\isanewline
\ \ \isacommand{qed}\isamarkupfalse%
\isanewline
\isacommand{qed}\isamarkupfalse%
%
\endisatagproof
{\isafoldproof}%
%
\isadelimproof
\isanewline
%
\endisadelimproof
\isanewline
\isacommand{lemma}\isamarkupfalse%
\isanewline
\ \ {\isachardoublequoteopen}G\ {\isasymin}\ setSubformulae\ F\ {\isasymLongrightarrow}\ setSubformulae\ G\ {\isasymsubseteq}\ setSubformulae\ F{\isachardoublequoteclose}\isanewline
%
\isadelimproof
%
\endisadelimproof
%
\isatagproof
\isacommand{proof}\isamarkupfalse%
\ {\isacharparenleft}induction\ F{\isacharparenright}\isanewline
\isacommand{case}\isamarkupfalse%
\ {\isacharparenleft}Atom\ x{\isacharparenright}\isanewline
\ \ \isacommand{then}\isamarkupfalse%
\ \isacommand{show}\isamarkupfalse%
\ {\isacharquery}case\ \isacommand{by}\isamarkupfalse%
\ {\isacharparenleft}rule\ subContsubformulae{\isacharunderscore}atom{\isacharparenright}\isanewline
\isacommand{next}\isamarkupfalse%
\isanewline
\ \ \isacommand{case}\isamarkupfalse%
\ Bot\isanewline
\ \ \isacommand{then}\isamarkupfalse%
\ \isacommand{show}\isamarkupfalse%
\ {\isacharquery}case\ \isacommand{by}\isamarkupfalse%
\ {\isacharparenleft}rule\ subContsubformulae{\isacharunderscore}bot{\isacharparenright}\isanewline
\isacommand{next}\isamarkupfalse%
\isanewline
\isacommand{case}\isamarkupfalse%
\ {\isacharparenleft}Not\ F{\isacharparenright}\isanewline
\ \ \isacommand{then}\isamarkupfalse%
\ \isacommand{show}\isamarkupfalse%
\ {\isacharquery}case\ \isacommand{by}\isamarkupfalse%
\ {\isacharparenleft}rule\ subContsubformulae{\isacharunderscore}not{\isacharparenright}\isanewline
\isacommand{next}\isamarkupfalse%
\isanewline
\ \ \isacommand{case}\isamarkupfalse%
\ {\isacharparenleft}And\ F{\isadigit{1}}\ F{\isadigit{2}}{\isacharparenright}\isanewline
\ \ \isacommand{then}\isamarkupfalse%
\ \isacommand{show}\isamarkupfalse%
\ {\isacharquery}case\ \isacommand{by}\isamarkupfalse%
\ {\isacharparenleft}rule\ subContsubformulae{\isacharunderscore}and{\isacharparenright}\isanewline
\isacommand{next}\isamarkupfalse%
\isanewline
\ \ \isacommand{case}\isamarkupfalse%
\ {\isacharparenleft}Or\ F{\isadigit{1}}\ F{\isadigit{2}}{\isacharparenright}\isanewline
\ \ \isacommand{then}\isamarkupfalse%
\ \isacommand{show}\isamarkupfalse%
\ {\isacharquery}case\ \isacommand{by}\isamarkupfalse%
\ {\isacharparenleft}rule\ subContsubformulae{\isacharunderscore}or{\isacharparenright}\isanewline
\isacommand{next}\isamarkupfalse%
\isanewline
\ \ \isacommand{case}\isamarkupfalse%
\ {\isacharparenleft}Imp\ F{\isadigit{1}}\ F{\isadigit{2}}{\isacharparenright}\isanewline
\ \ \isacommand{then}\isamarkupfalse%
\ \isacommand{show}\isamarkupfalse%
\ {\isacharquery}case\ \isacommand{by}\isamarkupfalse%
\ {\isacharparenleft}rule\ subContsubformulae{\isacharunderscore}imp{\isacharparenright}\isanewline
\isacommand{qed}\isamarkupfalse%
%
\endisatagproof
{\isafoldproof}%
%
\isadelimproof
%
\endisadelimproof
%
\begin{isamarkuptext}%
Finalmente, su demostración automática se muestra a continuación.%
\end{isamarkuptext}\isamarkuptrue%
\isacommand{lemma}\isamarkupfalse%
\ subContsubformulae{\isacharcolon}\isanewline
\ \ {\isachardoublequoteopen}G\ {\isasymin}\ setSubformulae\ F\ {\isasymLongrightarrow}\ setSubformulae\ G\ {\isasymsubseteq}\ setSubformulae\ F{\isachardoublequoteclose}\isanewline
%
\isadelimproof
\ \ %
\endisadelimproof
%
\isatagproof
\isacommand{by}\isamarkupfalse%
\ {\isacharparenleft}induction\ F{\isacharparenright}\ auto%
\endisatagproof
{\isafoldproof}%
%
\isadelimproof
%
\endisadelimproof
%
\begin{isamarkuptext}%
El siguiente lema nos da la noción de transitividad de contención 
  en cadena de las subfórmulas, de modo que la subfórmula de una 
  subfórmula es del mismo modo subfórmula de la mayor.

  \begin{lema}
    Sean \isa{G} subfórmula de \isa{F} y \isa{H} subfórmula de \isa{G}, entonces 
    \isa{H} es subfórmula de \isa{F}.
  \end{lema}

  \begin{demostracion}
  La prueba está basada en el lema anterior. Hemos demostrado que como 
  \isa{G} es subfórmula de \isa{F}, entonces el conjunto de átomos de \isa{G} está
  contenido en el de \isa{F}. Del mismo modo, como \isa{H} es subfórmula de
  \isa{G}, su conjunto de átomos está contenido en el de \isa{G}. Por la
  transitividad de la contención, tenemos que el conjunto de átomos de 
  \isa{H} está contenido en el de \isa{F}. Por otro lema anterior, tenemos que
  \isa{H} pertenece a su propio conjunto de subfórmulas. Por tanto,
  \isa{H\ {\isasymin}\ Subf{\isacharparenleft}H{\isacharparenright}\ {\isasymsubseteq}\ Subf{\isacharparenleft}F{\isacharparenright}\ {\isasymLongrightarrow}\ H\ {\isasymin}\ Subf{\isacharparenleft}F{\isacharparenright}}.
  \end{demostracion}

  Veamos su formalización y prueba estructurada en Isabelle.

  Veamos su prueba según las distintas tácticas.%
\end{isamarkuptext}\isamarkuptrue%
\isacommand{lemma}\isamarkupfalse%
\isanewline
\ \ \isakeyword{assumes}\ {\isachardoublequoteopen}G\ {\isasymin}\ setSubformulae\ F{\isachardoublequoteclose}\ \isanewline
\ \ \ \ \ \ \ \ \ \ {\isachardoublequoteopen}H\ {\isasymin}\ setSubformulae\ G{\isachardoublequoteclose}\isanewline
\ \ \isakeyword{shows}\ \ \ {\isachardoublequoteopen}H\ {\isasymin}\ setSubformulae\ F{\isachardoublequoteclose}\isanewline
%
\isadelimproof
%
\endisadelimproof
%
\isatagproof
\isacommand{proof}\isamarkupfalse%
\ {\isacharminus}\isanewline
\ \ \isacommand{have}\isamarkupfalse%
\ {\isadigit{1}}{\isacharcolon}{\isachardoublequoteopen}setSubformulae\ G\ {\isasymsubseteq}\ setSubformulae\ F{\isachardoublequoteclose}\ \isacommand{using}\isamarkupfalse%
\ assms{\isacharparenleft}{\isadigit{1}}{\isacharparenright}\ \isanewline
\ \ \ \ \isacommand{by}\isamarkupfalse%
\ {\isacharparenleft}rule\ subContsubformulae{\isacharparenright}\isanewline
\ \ \isacommand{have}\isamarkupfalse%
\ {\isachardoublequoteopen}setSubformulae\ H\ {\isasymsubseteq}\ setSubformulae\ G{\isachardoublequoteclose}\ \isacommand{using}\isamarkupfalse%
\ assms{\isacharparenleft}{\isadigit{2}}{\isacharparenright}\ \isanewline
\ \ \ \ \isacommand{by}\isamarkupfalse%
\ {\isacharparenleft}rule\ subContsubformulae{\isacharparenright}\isanewline
\ \ \isacommand{then}\isamarkupfalse%
\ \isacommand{have}\isamarkupfalse%
\ {\isadigit{2}}{\isacharcolon}{\isachardoublequoteopen}setSubformulae\ H\ {\isasymsubseteq}\ setSubformulae\ F{\isachardoublequoteclose}\ \isacommand{using}\isamarkupfalse%
\ {\isadigit{1}}\ \isanewline
\ \ \ \ \isacommand{by}\isamarkupfalse%
\ {\isacharparenleft}rule\ subset{\isacharunderscore}trans{\isacharparenright}\isanewline
\ \ \isacommand{have}\isamarkupfalse%
\ {\isachardoublequoteopen}H\ {\isasymin}\ setSubformulae\ H{\isachardoublequoteclose}\ \isanewline
\ \ \ \ \isacommand{by}\isamarkupfalse%
\ {\isacharparenleft}simp\ only{\isacharcolon}\ subformulae{\isacharunderscore}self{\isacharparenright}\isanewline
\ \ \isacommand{then}\isamarkupfalse%
\ \isacommand{show}\isamarkupfalse%
\ {\isachardoublequoteopen}H\ {\isasymin}\ setSubformulae\ F{\isachardoublequoteclose}\ \isanewline
\ \ \ \ \isacommand{using}\isamarkupfalse%
\ {\isadigit{2}}\ \isanewline
\ \ \ \ \isacommand{by}\isamarkupfalse%
\ {\isacharparenleft}rule\ rev{\isacharunderscore}subsetD{\isacharparenright}\isanewline
\isacommand{qed}\isamarkupfalse%
%
\endisatagproof
{\isafoldproof}%
%
\isadelimproof
\isanewline
%
\endisadelimproof
\isanewline
\isacommand{lemma}\isamarkupfalse%
\ subsubformulae{\isacharcolon}\ \isanewline
\ \ {\isachardoublequoteopen}G\ {\isasymin}\ setSubformulae\ F\ \isanewline
\ \ \ {\isasymLongrightarrow}\ H\ {\isasymin}\ setSubformulae\ G\ \isanewline
\ \ \ {\isasymLongrightarrow}\ H\ {\isasymin}\ setSubformulae\ F{\isachardoublequoteclose}\isanewline
%
\isadelimproof
\ \ %
\endisadelimproof
%
\isatagproof
\isacommand{using}\isamarkupfalse%
\ subContsubformulae\ \isacommand{by}\isamarkupfalse%
\ blast%
\endisatagproof
{\isafoldproof}%
%
\isadelimproof
%
\endisadelimproof
%
\begin{isamarkuptext}%
comentario{Explicar el cambio de enunciado}%
\end{isamarkuptext}\isamarkuptrue%
%
\begin{isamarkuptext}%
A continuación presentamos otro resultado que relaciona los 
  conjuntos de subfórmulas según las conectivas que operen.%
\end{isamarkuptext}\isamarkuptrue%
\isacommand{lemma}\isamarkupfalse%
\ subformulas{\isacharunderscore}in{\isacharunderscore}subformulas{\isacharcolon}\isanewline
\ \ {\isachardoublequoteopen}G\ \isactrlbold {\isasymand}\ H\ {\isasymin}\ setSubformulae\ F\ \isanewline
\ \ {\isasymLongrightarrow}\ G\ {\isasymin}\ setSubformulae\ F\ {\isasymand}\ H\ {\isasymin}\ setSubformulae\ F{\isachardoublequoteclose}\isanewline
\ \ {\isachardoublequoteopen}G\ \isactrlbold {\isasymor}\ H\ {\isasymin}\ setSubformulae\ F\ \isanewline
\ \ {\isasymLongrightarrow}\ G\ {\isasymin}\ setSubformulae\ F\ {\isasymand}\ H\ {\isasymin}\ setSubformulae\ F{\isachardoublequoteclose}\isanewline
\ \ {\isachardoublequoteopen}G\ \isactrlbold {\isasymrightarrow}\ H\ {\isasymin}\ setSubformulae\ F\ \isanewline
\ \ {\isasymLongrightarrow}\ G\ {\isasymin}\ setSubformulae\ F\ {\isasymand}\ H\ {\isasymin}\ setSubformulae\ F{\isachardoublequoteclose}\isanewline
\ \ {\isachardoublequoteopen}\isactrlbold {\isasymnot}\ G\ {\isasymin}\ setSubformulae\ F\ {\isasymLongrightarrow}\ G\ {\isasymin}\ setSubformulae\ F{\isachardoublequoteclose}\isanewline
%
\isadelimproof
\ \ %
\endisadelimproof
%
\isatagproof
\isacommand{oops}\isamarkupfalse%
%
\endisatagproof
{\isafoldproof}%
%
\isadelimproof
%
\endisadelimproof
%
\begin{isamarkuptext}%
Como podemos observar, el resultado es análogo en todas las 
  conectivas binarias aunque aparezcan definidas por separado, por tanto 
  haré la demostración estructurada para una de ellas pues el resto son 
  análogas. 

  Nos basaremos en el lema anterior \isa{subsubformulae}.%
\end{isamarkuptext}\isamarkuptrue%
\isacommand{lemma}\isamarkupfalse%
\ subformulas{\isacharunderscore}in{\isacharunderscore}subformulas{\isacharunderscore}not{\isacharcolon}\isanewline
\ \ \isakeyword{assumes}\ {\isachardoublequoteopen}Not\ G\ {\isasymin}\ setSubformulae\ F{\isachardoublequoteclose}\isanewline
\ \ \isakeyword{shows}\ {\isachardoublequoteopen}G\ {\isasymin}\ setSubformulae\ F{\isachardoublequoteclose}\isanewline
%
\isadelimproof
%
\endisadelimproof
%
\isatagproof
\isacommand{proof}\isamarkupfalse%
\ {\isacharminus}\isanewline
\ \ \isacommand{have}\isamarkupfalse%
\ {\isachardoublequoteopen}setSubformulae\ {\isacharparenleft}Not\ G{\isacharparenright}\ {\isacharequal}\ {\isacharbraceleft}Not\ G{\isacharbraceright}\ {\isasymunion}\ setSubformulae\ G{\isachardoublequoteclose}\ \isanewline
\ \ \ \ \isacommand{by}\isamarkupfalse%
\ simp\ %
\isamarkupcmt{Pendiente%
}\isanewline
\ \ \isacommand{then}\isamarkupfalse%
\ \isacommand{have}\isamarkupfalse%
\ {\isadigit{1}}{\isacharcolon}{\isachardoublequoteopen}G\ {\isasymin}\ setSubformulae\ {\isacharparenleft}Not\ G{\isacharparenright}{\isachardoublequoteclose}\ \isanewline
\ \ \ \ \isacommand{by}\isamarkupfalse%
\ {\isacharparenleft}simp\ add{\isacharcolon}\ subformulae{\isacharunderscore}self{\isacharparenright}\ %
\isamarkupcmt{Pendiente%
}\isanewline
\ \ \isacommand{show}\isamarkupfalse%
\ {\isachardoublequoteopen}G\ {\isasymin}\ setSubformulae\ F{\isachardoublequoteclose}\ \isacommand{using}\isamarkupfalse%
\ assms\ {\isadigit{1}}\ \isanewline
\ \ \ \ \isacommand{by}\isamarkupfalse%
\ {\isacharparenleft}rule\ subsubformulae{\isacharparenright}\isanewline
\isacommand{qed}\isamarkupfalse%
%
\endisatagproof
{\isafoldproof}%
%
\isadelimproof
\isanewline
%
\endisadelimproof
\isanewline
\isacommand{lemma}\isamarkupfalse%
\ subformulas{\isacharunderscore}in{\isacharunderscore}subformulas{\isacharunderscore}and{\isacharcolon}\isanewline
\ \ \isakeyword{assumes}\ {\isachardoublequoteopen}G\ \isactrlbold {\isasymand}\ H\ {\isasymin}\ setSubformulae\ F{\isachardoublequoteclose}\ \isanewline
\ \ \isakeyword{shows}\ {\isachardoublequoteopen}G\ {\isasymin}\ setSubformulae\ F\ {\isasymand}\ H\ {\isasymin}\ setSubformulae\ F{\isachardoublequoteclose}\isanewline
%
\isadelimproof
%
\endisadelimproof
%
\isatagproof
\isacommand{proof}\isamarkupfalse%
\ {\isacharparenleft}rule\ conjI{\isacharparenright}\isanewline
\ \ \isacommand{have}\isamarkupfalse%
\ {\isadigit{1}}{\isacharcolon}\ {\isachardoublequoteopen}setSubformulae\ {\isacharparenleft}G\ \isactrlbold {\isasymand}\ H{\isacharparenright}\ {\isacharequal}\ \isanewline
\ \ \ \ \ \ \ \ \ \ {\isacharbraceleft}G\ \isactrlbold {\isasymand}\ H{\isacharbraceright}\ {\isasymunion}\ {\isacharparenleft}setSubformulae\ G\ {\isasymunion}\ setSubformulae\ H{\isacharparenright}{\isachardoublequoteclose}\ \isanewline
\ \ \ \ \isacommand{by}\isamarkupfalse%
\ {\isacharparenleft}simp\ only{\isacharcolon}\ setSubformulae{\isacharunderscore}and{\isacharparenright}\isanewline
\ \ \isacommand{then}\isamarkupfalse%
\ \isacommand{have}\isamarkupfalse%
\ {\isadigit{2}}{\isacharcolon}\ {\isachardoublequoteopen}G\ {\isasymin}\ setSubformulae\ {\isacharparenleft}G\ \isactrlbold {\isasymand}\ H{\isacharparenright}{\isachardoublequoteclose}\ \isanewline
\ \ \ \ \isacommand{by}\isamarkupfalse%
\ {\isacharparenleft}simp\ add{\isacharcolon}\ subformulae{\isacharunderscore}self{\isacharparenright}\ %
\isamarkupcmt{Pendiente%
}\ \isanewline
\ \ \isacommand{have}\isamarkupfalse%
\ {\isadigit{3}}{\isacharcolon}\ {\isachardoublequoteopen}H\ {\isasymin}\ setSubformulae\ {\isacharparenleft}G\ \isactrlbold {\isasymand}\ H{\isacharparenright}{\isachardoublequoteclose}\ \isanewline
\ \ \ \ \isacommand{using}\isamarkupfalse%
\ {\isadigit{1}}\ \isanewline
\ \ \ \ \isacommand{by}\isamarkupfalse%
\ {\isacharparenleft}simp\ add{\isacharcolon}\ subformulae{\isacharunderscore}self{\isacharparenright}\ %
\isamarkupcmt{Pendiente%
}\ \isanewline
\ \ \isacommand{show}\isamarkupfalse%
\ {\isachardoublequoteopen}G\ {\isasymin}\ setSubformulae\ F{\isachardoublequoteclose}\ \isacommand{using}\isamarkupfalse%
\ assms\ {\isadigit{2}}\ \isacommand{by}\isamarkupfalse%
\ {\isacharparenleft}rule\ subsubformulae{\isacharparenright}\isanewline
\ \ \isacommand{show}\isamarkupfalse%
\ {\isachardoublequoteopen}H\ {\isasymin}\ setSubformulae\ F{\isachardoublequoteclose}\ \isacommand{using}\isamarkupfalse%
\ assms\ {\isadigit{3}}\ \isacommand{by}\isamarkupfalse%
\ {\isacharparenleft}rule\ subsubformulae{\isacharparenright}\isanewline
\isacommand{qed}\isamarkupfalse%
%
\endisatagproof
{\isafoldproof}%
%
\isadelimproof
%
\endisadelimproof
%
\begin{isamarkuptext}%
Mostremos ahora la demostración automática.%
\end{isamarkuptext}\isamarkuptrue%
\isacommand{lemma}\isamarkupfalse%
\ subformulas{\isacharunderscore}in{\isacharunderscore}subformulas{\isacharcolon}\isanewline
\ \ {\isachardoublequoteopen}G\ \isactrlbold {\isasymand}\ H\ {\isasymin}\ setSubformulae\ F\ \isanewline
\ \ \ {\isasymLongrightarrow}\ G\ {\isasymin}\ setSubformulae\ F\ {\isasymand}\ H\ {\isasymin}\ setSubformulae\ F{\isachardoublequoteclose}\isanewline
\ \ {\isachardoublequoteopen}G\ \isactrlbold {\isasymor}\ H\ {\isasymin}\ setSubformulae\ F\ \isanewline
\ \ \ {\isasymLongrightarrow}\ G\ {\isasymin}\ setSubformulae\ F\ {\isasymand}\ H\ {\isasymin}\ setSubformulae\ F{\isachardoublequoteclose}\isanewline
\ \ {\isachardoublequoteopen}G\ \isactrlbold {\isasymrightarrow}\ H\ {\isasymin}\ setSubformulae\ F\ \isanewline
\ \ \ {\isasymLongrightarrow}\ G\ {\isasymin}\ setSubformulae\ F\ {\isasymand}\ H\ {\isasymin}\ setSubformulae\ F{\isachardoublequoteclose}\isanewline
\ \ {\isachardoublequoteopen}\isactrlbold {\isasymnot}\ G\ {\isasymin}\ setSubformulae\ F\ {\isasymLongrightarrow}\ G\ {\isasymin}\ setSubformulae\ F{\isachardoublequoteclose}\isanewline
%
\isadelimproof
\ \ %
\endisadelimproof
%
\isatagproof
\isacommand{using}\isamarkupfalse%
\ subformulae{\isacharunderscore}self\ subsubformulae\ \isacommand{apply}\isamarkupfalse%
\ force\isanewline
\ \ \isacommand{oops}\isamarkupfalse%
%
\endisatagproof
{\isafoldproof}%
%
\isadelimproof
%
\endisadelimproof
%
\begin{isamarkuptext}%
\comentario{Completar la prueba anterior.}%
\end{isamarkuptext}\isamarkuptrue%
%
\begin{isamarkuptext}%
\comentario{Completar lo que falta de sección}%
\end{isamarkuptext}\isamarkuptrue%
%
\begin{isamarkuptext}%
Concluimos la sección de subfórmulas con un resultado que 
  relaciona varias funciones sobre la longitud de la lista 
  \isa{subformulae\ F} de una fórmula \isa{F} cualquiera.%
\end{isamarkuptext}\isamarkuptrue%
\isacommand{lemma}\isamarkupfalse%
\ length{\isacharunderscore}subformulae{\isacharcolon}\ {\isachardoublequoteopen}length\ {\isacharparenleft}subformulae\ F{\isacharparenright}\ {\isacharequal}\ size\ F{\isachardoublequoteclose}\ \isanewline
%
\isadelimproof
\ \ %
\endisadelimproof
%
\isatagproof
\isacommand{oops}\isamarkupfalse%
%
\endisatagproof
{\isafoldproof}%
%
\isadelimproof
%
\endisadelimproof
%
\begin{isamarkuptext}%
En primer lugar aparece la clase \isa{size} de la teoría de 
  números naturales ....

  Vamos a definir \isa{size{\isadigit{1}}} de manera idéntica a como aparece 
  \isa{size} en la teoría.%
\end{isamarkuptext}\isamarkuptrue%
\isacommand{class}\isamarkupfalse%
\ size{\isadigit{1}}\ {\isacharequal}\isanewline
\ \ \isakeyword{fixes}\ size{\isadigit{1}}\ {\isacharcolon}{\isacharcolon}\ {\isachardoublequoteopen}{\isacharprime}a\ {\isasymRightarrow}\ nat{\isachardoublequoteclose}\ \isanewline
\isanewline
\isacommand{instantiation}\isamarkupfalse%
\ nat\ {\isacharcolon}{\isacharcolon}\ size{\isadigit{1}}\isanewline
\isakeyword{begin}\isanewline
\isanewline
\isacommand{definition}\isamarkupfalse%
\ size{\isadigit{1}}{\isacharunderscore}nat\ \isakeyword{where}\ {\isacharbrackleft}simp{\isacharcomma}\ code{\isacharbrackright}{\isacharcolon}\ {\isachardoublequoteopen}size{\isadigit{1}}\ {\isacharparenleft}n{\isacharcolon}{\isacharcolon}nat{\isacharparenright}\ {\isacharequal}\ n{\isachardoublequoteclose}\isanewline
\isanewline
\isacommand{instance}\isamarkupfalse%
%
\isadelimproof
\ %
\endisadelimproof
%
\isatagproof
\isacommand{{\isachardot}{\isachardot}}\isamarkupfalse%
%
\endisatagproof
{\isafoldproof}%
%
\isadelimproof
%
\endisadelimproof
\isanewline
\isanewline
\isacommand{end}\isamarkupfalse%
%
\begin{isamarkuptext}%
Como podemos observar, se trata de una clase que actúa sobre un 
  parámetro global de tipo \isa{{\isacharprime}a} cualquiera. Por otro lado, 
  \isa{instantation} define una clase de parámetros, en este caso los 
  números naturales \isa{nat} que devuelve como resultado. Incluye una 
  definición concreta del operador \isa{size{\isadigit{1}}} sobre dichos parámetros. 
  Además, el último \isa{instance} abre una prueba que afirma que los 
  parámetros dados conforman la clase especificada en la definición. 
  Esta prueba que nos afirma que está bien definida la clase aparece
  omitida utilizando \isa{{\isachardot}{\isachardot}} .

  En particular, sobre una fórmula nos devuelve el número de elementos 
  de la lista cuyos elementos son los nodos y las hojas de su árbol de 
  formación.%
\end{isamarkuptext}\isamarkuptrue%
\isacommand{value}\isamarkupfalse%
\ {\isachardoublequoteopen}size\ {\isacharparenleft}n{\isacharcolon}{\isacharcolon}nat{\isacharparenright}\ {\isacharequal}\ n{\isachardoublequoteclose}\isanewline
\isacommand{value}\isamarkupfalse%
\ {\isachardoublequoteopen}size\ {\isacharparenleft}{\isadigit{5}}{\isacharcolon}{\isacharcolon}nat{\isacharparenright}\ {\isacharequal}\ {\isadigit{5}}{\isachardoublequoteclose}%
\begin{isamarkuptext}%
Por otro lado, la función \isa{length} de la teoría 
  \href{http://cort.as/-Stfm}{List.thy} nos indica la longitud de una 
  lista cualquiera de elementos, definiéndose utilizando el comando
  \isa{size} visto anteriormente.%
\end{isamarkuptext}\isamarkuptrue%
\isacommand{abbreviation}\isamarkupfalse%
\ length{\isacharprime}\ {\isacharcolon}{\isacharcolon}\ {\isachardoublequoteopen}{\isacharprime}a\ list\ {\isasymRightarrow}\ nat{\isachardoublequoteclose}\ \isakeyword{where}\isanewline
\ \ {\isachardoublequoteopen}length{\isacharprime}\ {\isasymequiv}\ size{\isachardoublequoteclose}%
\begin{isamarkuptext}%
Las demostración del resultado se vuelve a basar en la inducción 
  que nos despliega seis casos. 

  La prueba estructurada no resulta interesante, pues todos los casos 
  son inmediatos por simplificación como en el primer lema de esta 
  sección. 

  Incluimos a continuación la prueba automática.%
\end{isamarkuptext}\isamarkuptrue%
\isacommand{lemma}\isamarkupfalse%
\ length{\isacharunderscore}subformulae{\isacharcolon}\ {\isachardoublequoteopen}length\ {\isacharparenleft}subformulae\ F{\isacharparenright}\ {\isacharequal}\ size\ F{\isachardoublequoteclose}\ \isanewline
%
\isadelimproof
\ \ %
\endisadelimproof
%
\isatagproof
\isacommand{by}\isamarkupfalse%
\ {\isacharparenleft}induction\ F{\isacharsemicolon}\ simp{\isacharparenright}%
\endisatagproof
{\isafoldproof}%
%
\isadelimproof
%
\endisadelimproof
%
\begin{isamarkuptext}%
\comentario{Hacer la prueba detallada para mostrar los teoremas 
  utilizados.}%
\end{isamarkuptext}\isamarkuptrue%
%
\isadelimdocument
%
\endisadelimdocument
%
\isatagdocument
%
\isamarkupsection{Conectivas derivadas%
}
\isamarkuptrue%
%
\endisatagdocument
{\isafolddocument}%
%
\isadelimdocument
%
\endisadelimdocument
%
\begin{isamarkuptext}%
En esta sección definiremos nuevas conectivas y fórmulas a partir 
  de los ya definidos en el apartado anterior, junto con varios 
  resultados sobre los mismos. Veamos el primero.

  \begin{definicion}
    Se define la fórmula \isa{{\isasymtop}} como la implicación \isa{{\isasymbottom}\ {\isasymlongrightarrow}\ {\isasymbottom}}.
  \end{definicion}

  Se formaliza del siguiente modo.%
\end{isamarkuptext}\isamarkuptrue%
\isacommand{definition}\isamarkupfalse%
\ Top\ {\isacharparenleft}{\isachardoublequoteopen}{\isasymtop}{\isachardoublequoteclose}{\isacharparenright}\ \isakeyword{where}\isanewline
\ \ {\isachardoublequoteopen}{\isasymtop}\ {\isasymequiv}\ {\isasymbottom}\ \isactrlbold {\isasymrightarrow}\ {\isasymbottom}{\isachardoublequoteclose}%
\begin{isamarkuptext}%
Como podemos observar, se define mediante una relación de 
  equivalencia con otra fórmula ya conocida. El uso de dicha 
  equivalencia justifica el tipo \isa{definition} empleado en este 
  caso. 

  Por la propia definición, es claro que \isa{{\isasymtop}} no contiene ninguna
  variable proposicional, como se verifica a continuación en Isabelle.%
\end{isamarkuptext}\isamarkuptrue%
\isacommand{lemma}\isamarkupfalse%
\ {\isachardoublequoteopen}atoms\ {\isasymtop}\ {\isacharequal}\ {\isasymemptyset}{\isachardoublequoteclose}\isanewline
%
\isadelimproof
\ \ \ %
\endisadelimproof
%
\isatagproof
\isacommand{by}\isamarkupfalse%
\ {\isacharparenleft}simp\ only{\isacharcolon}\ Top{\isacharunderscore}def\ formula{\isachardot}set\ Un{\isacharunderscore}absorb{\isacharparenright}%
\endisatagproof
{\isafoldproof}%
%
\isadelimproof
%
\endisadelimproof
%
\begin{isamarkuptext}%
\comentario{Añadir regla al glosario.}%
\end{isamarkuptext}\isamarkuptrue%
%
\begin{isamarkuptext}%
\comentario{Añadir la doble implicación como conectiva derivada.}%
\end{isamarkuptext}\isamarkuptrue%
%
\begin{isamarkuptext}%
A continuación vamos a definir dos conectivas que generalizan la 
  conjunción y la disyunción para una lista finita de fórmulas. En
  particular, al ser aplicadas a listas, se definen conforme a la 
  estructura recursiva de las mismas que se muestra a continuación. 
  
  \begin{definicion}
    Las listas de fórmulas se definen recursivamente como sigue.
    \begin{itemize}
      \item La lista vacía es una lista.
      \item Sea \isa{F} una fórmula y \isa{Fs} una lista de fórmulas. Entonces,
        \isa{F{\isacharhash}Fs} es una lista de fórmulas.
    \end{itemize}
  \end{definicion} 

\comentario{Esta definición es un caso particular de listas. 
  No se si incluir la definicion de estructura e inducción general}

  De este modo, se definen las conectivas plurales de acuerdo a la 
  estructura recursiva anterior. Notemos que al referirnos simplemente 
  a disyunción o conjunción nos referiremos a la de dos elementos.

  \begin{definicion}
  La conjunción plural de una lista de fórmulas se define recursivamente
  como:
    \begin{itemize}
      \item La conjunción plural de la lista vacía es \isa{{\isasymnot}{\isasymbottom}}.
      \item Sea \isa{F} una fórmula y \isa{Fs} una lista de fórmulas. Entonces,
  la conjunción plural de \isa{F{\isacharhash}Fs} es la conjunción de \isa{F} con la 
  conjunción plural de \isa{Fs}.
    \end{itemize}
  \end{definicion}

  \begin{definicion}
  La disyunción plural de una lista de fórmulas se define recursivamente
  como:
    \begin{itemize}
      \item La disyunción plural de la lista vacía es \isa{{\isasymnot}{\isasymbottom}}.
      \item Sea \isa{F} una fórmula y \isa{Fs} una lista de fórmulas. Entonces,
  la disyunción plural de \isa{F{\isacharhash}Fs} es la disyunción de \isa{F} con la 
  disyunción plural de \isa{Fs}.
    \end{itemize}
  \end{definicion}

  Su formalización en Isabelle es la siguiente.

  \comentario{Da error que no localizo}%
\end{isamarkuptext}\isamarkuptrue%
%
\begin{isamarkuptext}%
Ambas nuevas conectivas se definen con el tipo funciones 
  primitivas recursivas. Estas se basan en los dos casos descritos
  anteriormente: la lista vacía representada como \isa{Nil} y la lista
  construida añadiendo una fórmula a una lista de fórmulas. 
  Además, se observa en cada conectiva plural el nuevo símbolo de 
  notación que aparece entre paréntesis.

  Por otro lado, como es habitual, estas definiciones recursivas llevan
  asociado el correspondiente esquema inductivo de demostración. En
  este caso, se trata de la inducción para la estructura de lista de 
  fórmulas.

  \begin{definicion}
    Sea \isa{{\isasymP}} una propiedad sobre lista de fórmulas proposicionales que 
    verifica las siguientes condiciones:
    \begin{itemize}
     \item La lista vacía la verifica.
     \item Dada una fórmula \isa{F} y una lista de fórmulas \isa{Fs} que la
      verifican, entonces \isa{F{\isacharhash}Fs} la verifica.
    \end{itemize}
    Entonces, todas las listas de fórmulas proposicionales tienen la 
    propiedad \isa{{\isasymP}}. 
  \end{definicion}

  La conjunción plural nos da el siguiente resultado.

\comentario{Añadir lema a mano y demostración. Falta demostración en Isabelle.}%
\end{isamarkuptext}\isamarkuptrue%
%
\isadelimtheory
%
\endisadelimtheory
%
\isatagtheory
%
\endisatagtheory
{\isafoldtheory}%
%
\isadelimtheory
%
\endisadelimtheory
%
\end{isabellebody}%
\endinput
%:%file=~/TFG/Logica_Proposicional/Sintaxis.thy%:%
%:%24=13%:%
%:%36=15%:%
%:%37=16%:%
%:%38=17%:%
%:%39=18%:%
%:%40=19%:%
%:%42=21%:%
%:%43=21%:%
%:%45=23%:%
%:%46=24%:%
%:%47=25%:%
%:%48=26%:%
%:%49=27%:%
%:%50=28%:%
%:%51=29%:%
%:%52=30%:%
%:%53=31%:%
%:%54=32%:%
%:%55=33%:%
%:%56=34%:%
%:%57=35%:%
%:%58=36%:%
%:%59=37%:%
%:%60=38%:%
%:%61=39%:%
%:%62=40%:%
%:%63=41%:%
%:%64=42%:%
%:%65=43%:%
%:%66=44%:%
%:%67=45%:%
%:%68=46%:%
%:%69=47%:%
%:%70=48%:%
%:%71=49%:%
%:%72=50%:%
%:%73=51%:%
%:%74=52%:%
%:%75=53%:%
%:%76=54%:%
%:%77=55%:%
%:%78=56%:%
%:%79=57%:%
%:%80=58%:%
%:%81=59%:%
%:%82=60%:%
%:%83=61%:%
%:%84=62%:%
%:%85=63%:%
%:%86=64%:%
%:%87=65%:%
%:%88=66%:%
%:%89=67%:%
%:%90=68%:%
%:%91=69%:%
%:%92=70%:%
%:%93=71%:%
%:%94=72%:%
%:%96=74%:%
%:%97=74%:%
%:%98=75%:%
%:%99=76%:%
%:%100=77%:%
%:%101=78%:%
%:%102=79%:%
%:%103=80%:%
%:%105=82%:%
%:%106=83%:%
%:%107=84%:%
%:%108=85%:%
%:%109=86%:%
%:%110=87%:%
%:%111=88%:%
%:%112=89%:%
%:%113=90%:%
%:%114=91%:%
%:%115=92%:%
%:%116=93%:%
%:%117=94%:%
%:%118=95%:%
%:%119=96%:%
%:%120=97%:%
%:%121=98%:%
%:%122=99%:%
%:%123=100%:%
%:%124=101%:%
%:%125=102%:%
%:%126=103%:%
%:%127=104%:%
%:%128=105%:%
%:%129=106%:%
%:%130=107%:%
%:%131=108%:%
%:%132=109%:%
%:%133=110%:%
%:%134=111%:%
%:%135=112%:%
%:%136=113%:%
%:%137=114%:%
%:%142=114%:%
%:%143=115%:%
%:%144=116%:%
%:%145=117%:%
%:%146=118%:%
%:%147=119%:%
%:%148=120%:%
%:%150=122%:%
%:%151=122%:%
%:%152=123%:%
%:%155=124%:%
%:%159=124%:%
%:%160=124%:%
%:%161=125%:%
%:%162=126%:%
%:%163=126%:%
%:%164=127%:%
%:%165=127%:%
%:%166=128%:%
%:%167=129%:%
%:%168=129%:%
%:%169=130%:%
%:%170=130%:%
%:%171=131%:%
%:%172=132%:%
%:%173=132%:%
%:%174=133%:%
%:%175=133%:%
%:%176=134%:%
%:%177=135%:%
%:%178=135%:%
%:%179=136%:%
%:%180=136%:%
%:%185=136%:%
%:%188=137%:%
%:%191=139%:%
%:%192=140%:%
%:%193=141%:%
%:%195=143%:%
%:%196=143%:%
%:%197=144%:%
%:%200=145%:%
%:%204=145%:%
%:%205=145%:%
%:%206=146%:%
%:%207=147%:%
%:%208=147%:%
%:%209=148%:%
%:%210=148%:%
%:%211=149%:%
%:%212=150%:%
%:%213=150%:%
%:%214=151%:%
%:%215=151%:%
%:%216=152%:%
%:%217=152%:%
%:%218=153%:%
%:%219=153%:%
%:%220=154%:%
%:%221=154%:%
%:%222=154%:%
%:%223=155%:%
%:%224=155%:%
%:%225=156%:%
%:%226=156%:%
%:%227=156%:%
%:%228=157%:%
%:%229=157%:%
%:%230=158%:%
%:%231=158%:%
%:%232=158%:%
%:%233=159%:%
%:%234=159%:%
%:%235=160%:%
%:%236=160%:%
%:%237=160%:%
%:%238=161%:%
%:%239=161%:%
%:%240=162%:%
%:%241=162%:%
%:%242=163%:%
%:%243=164%:%
%:%244=164%:%
%:%245=165%:%
%:%246=165%:%
%:%251=165%:%
%:%254=166%:%
%:%255=166%:%
%:%256=167%:%
%:%257=168%:%
%:%258=168%:%
%:%260=170%:%
%:%261=171%:%
%:%262=172%:%
%:%263=173%:%
%:%264=174%:%
%:%265=175%:%
%:%266=176%:%
%:%267=177%:%
%:%268=178%:%
%:%269=179%:%
%:%270=180%:%
%:%271=181%:%
%:%272=182%:%
%:%273=183%:%
%:%274=184%:%
%:%275=185%:%
%:%276=186%:%
%:%277=187%:%
%:%278=188%:%
%:%279=189%:%
%:%280=190%:%
%:%281=191%:%
%:%282=192%:%
%:%283=193%:%
%:%284=194%:%
%:%285=195%:%
%:%286=196%:%
%:%287=197%:%
%:%288=198%:%
%:%289=199%:%
%:%290=200%:%
%:%291=201%:%
%:%292=202%:%
%:%293=203%:%
%:%294=204%:%
%:%295=205%:%
%:%296=206%:%
%:%297=207%:%
%:%298=208%:%
%:%299=209%:%
%:%300=210%:%
%:%301=211%:%
%:%302=212%:%
%:%303=213%:%
%:%304=214%:%
%:%305=215%:%
%:%306=216%:%
%:%307=217%:%
%:%308=218%:%
%:%309=219%:%
%:%310=220%:%
%:%311=221%:%
%:%312=222%:%
%:%313=223%:%
%:%314=224%:%
%:%315=225%:%
%:%316=226%:%
%:%317=227%:%
%:%318=228%:%
%:%319=229%:%
%:%320=230%:%
%:%321=231%:%
%:%323=233%:%
%:%324=233%:%
%:%327=234%:%
%:%331=234%:%
%:%341=236%:%
%:%342=237%:%
%:%343=238%:%
%:%345=240%:%
%:%346=240%:%
%:%347=241%:%
%:%348=242%:%
%:%350=244%:%
%:%351=245%:%
%:%352=246%:%
%:%353=247%:%
%:%354=248%:%
%:%355=249%:%
%:%356=250%:%
%:%357=251%:%
%:%358=252%:%
%:%359=253%:%
%:%360=254%:%
%:%361=255%:%
%:%362=256%:%
%:%363=257%:%
%:%364=258%:%
%:%365=259%:%
%:%366=260%:%
%:%367=261%:%
%:%368=262%:%
%:%369=263%:%
%:%370=264%:%
%:%371=265%:%
%:%372=266%:%
%:%373=267%:%
%:%374=268%:%
%:%375=269%:%
%:%376=270%:%
%:%377=271%:%
%:%378=272%:%
%:%379=273%:%
%:%380=274%:%
%:%381=275%:%
%:%382=276%:%
%:%383=277%:%
%:%384=278%:%
%:%385=279%:%
%:%386=280%:%
%:%387=281%:%
%:%388=282%:%
%:%389=283%:%
%:%391=285%:%
%:%392=285%:%
%:%393=286%:%
%:%400=287%:%
%:%401=287%:%
%:%402=288%:%
%:%403=288%:%
%:%404=289%:%
%:%405=289%:%
%:%406=290%:%
%:%407=290%:%
%:%408=290%:%
%:%409=291%:%
%:%410=291%:%
%:%411=292%:%
%:%412=292%:%
%:%413=292%:%
%:%414=293%:%
%:%415=293%:%
%:%416=294%:%
%:%422=294%:%
%:%425=295%:%
%:%426=296%:%
%:%427=296%:%
%:%428=297%:%
%:%435=298%:%
%:%436=298%:%
%:%437=299%:%
%:%438=299%:%
%:%439=300%:%
%:%440=300%:%
%:%441=301%:%
%:%442=301%:%
%:%443=301%:%
%:%444=302%:%
%:%445=302%:%
%:%446=303%:%
%:%452=303%:%
%:%455=304%:%
%:%456=305%:%
%:%457=305%:%
%:%458=306%:%
%:%459=307%:%
%:%462=308%:%
%:%466=308%:%
%:%467=308%:%
%:%468=309%:%
%:%469=309%:%
%:%474=309%:%
%:%477=310%:%
%:%478=311%:%
%:%479=311%:%
%:%480=312%:%
%:%481=313%:%
%:%482=314%:%
%:%489=315%:%
%:%490=315%:%
%:%491=316%:%
%:%492=316%:%
%:%493=317%:%
%:%494=317%:%
%:%495=318%:%
%:%496=318%:%
%:%497=319%:%
%:%498=319%:%
%:%499=319%:%
%:%500=320%:%
%:%501=320%:%
%:%502=321%:%
%:%508=321%:%
%:%511=322%:%
%:%512=323%:%
%:%513=323%:%
%:%514=324%:%
%:%515=325%:%
%:%516=326%:%
%:%523=327%:%
%:%524=327%:%
%:%525=328%:%
%:%526=328%:%
%:%527=329%:%
%:%528=329%:%
%:%529=330%:%
%:%530=330%:%
%:%531=331%:%
%:%532=331%:%
%:%533=331%:%
%:%534=332%:%
%:%535=332%:%
%:%536=333%:%
%:%542=333%:%
%:%545=334%:%
%:%546=335%:%
%:%547=335%:%
%:%548=336%:%
%:%549=337%:%
%:%550=338%:%
%:%557=339%:%
%:%558=339%:%
%:%559=340%:%
%:%560=340%:%
%:%561=341%:%
%:%562=341%:%
%:%563=342%:%
%:%564=342%:%
%:%565=343%:%
%:%566=343%:%
%:%567=343%:%
%:%568=344%:%
%:%569=344%:%
%:%570=345%:%
%:%576=345%:%
%:%579=346%:%
%:%580=347%:%
%:%581=347%:%
%:%588=348%:%
%:%589=348%:%
%:%590=349%:%
%:%591=349%:%
%:%592=350%:%
%:%593=350%:%
%:%594=350%:%
%:%595=350%:%
%:%596=351%:%
%:%597=351%:%
%:%598=352%:%
%:%599=352%:%
%:%600=353%:%
%:%601=353%:%
%:%602=353%:%
%:%603=353%:%
%:%604=354%:%
%:%605=354%:%
%:%606=355%:%
%:%607=355%:%
%:%608=356%:%
%:%609=356%:%
%:%610=356%:%
%:%611=356%:%
%:%612=357%:%
%:%613=357%:%
%:%614=358%:%
%:%615=358%:%
%:%616=359%:%
%:%617=359%:%
%:%618=359%:%
%:%619=359%:%
%:%620=360%:%
%:%621=360%:%
%:%622=361%:%
%:%623=361%:%
%:%624=362%:%
%:%625=362%:%
%:%626=362%:%
%:%627=362%:%
%:%628=363%:%
%:%629=363%:%
%:%630=364%:%
%:%631=364%:%
%:%632=365%:%
%:%633=365%:%
%:%634=365%:%
%:%635=365%:%
%:%636=366%:%
%:%646=368%:%
%:%648=370%:%
%:%649=370%:%
%:%652=371%:%
%:%656=371%:%
%:%657=371%:%
%:%671=373%:%
%:%683=375%:%
%:%684=376%:%
%:%685=377%:%
%:%686=378%:%
%:%687=379%:%
%:%688=380%:%
%:%689=381%:%
%:%690=382%:%
%:%691=383%:%
%:%692=384%:%
%:%693=385%:%
%:%694=386%:%
%:%695=387%:%
%:%696=388%:%
%:%697=389%:%
%:%698=390%:%
%:%699=391%:%
%:%700=392%:%
%:%702=394%:%
%:%703=394%:%
%:%704=395%:%
%:%705=396%:%
%:%706=397%:%
%:%707=398%:%
%:%708=399%:%
%:%709=400%:%
%:%711=402%:%
%:%712=403%:%
%:%713=404%:%
%:%714=405%:%
%:%715=406%:%
%:%717=408%:%
%:%718=408%:%
%:%719=409%:%
%:%722=410%:%
%:%726=410%:%
%:%727=410%:%
%:%728=411%:%
%:%729=412%:%
%:%730=412%:%
%:%731=413%:%
%:%732=413%:%
%:%733=414%:%
%:%734=415%:%
%:%735=415%:%
%:%736=416%:%
%:%737=416%:%
%:%738=417%:%
%:%739=418%:%
%:%740=418%:%
%:%742=420%:%
%:%743=421%:%
%:%744=421%:%
%:%745=422%:%
%:%746=423%:%
%:%747=423%:%
%:%748=424%:%
%:%749=424%:%
%:%750=425%:%
%:%751=426%:%
%:%752=426%:%
%:%753=427%:%
%:%754=428%:%
%:%755=428%:%
%:%760=428%:%
%:%763=429%:%
%:%766=431%:%
%:%767=432%:%
%:%768=433%:%
%:%770=435%:%
%:%771=435%:%
%:%772=436%:%
%:%774=438%:%
%:%775=439%:%
%:%776=440%:%
%:%777=441%:%
%:%778=442%:%
%:%779=443%:%
%:%780=444%:%
%:%781=445%:%
%:%783=447%:%
%:%784=447%:%
%:%785=448%:%
%:%788=449%:%
%:%792=449%:%
%:%793=449%:%
%:%794=450%:%
%:%795=451%:%
%:%796=451%:%
%:%797=452%:%
%:%798=452%:%
%:%799=453%:%
%:%800=454%:%
%:%801=454%:%
%:%803=456%:%
%:%804=457%:%
%:%805=457%:%
%:%810=457%:%
%:%813=458%:%
%:%816=460%:%
%:%817=461%:%
%:%818=462%:%
%:%819=463%:%
%:%820=464%:%
%:%821=465%:%
%:%822=466%:%
%:%823=467%:%
%:%824=468%:%
%:%825=469%:%
%:%826=470%:%
%:%827=471%:%
%:%829=473%:%
%:%830=473%:%
%:%833=474%:%
%:%837=474%:%
%:%838=474%:%
%:%847=476%:%
%:%848=477%:%
%:%850=479%:%
%:%851=479%:%
%:%852=480%:%
%:%855=481%:%
%:%859=481%:%
%:%860=481%:%
%:%865=481%:%
%:%868=482%:%
%:%869=483%:%
%:%870=483%:%
%:%871=484%:%
%:%874=485%:%
%:%878=485%:%
%:%879=485%:%
%:%884=485%:%
%:%887=486%:%
%:%888=487%:%
%:%889=487%:%
%:%890=488%:%
%:%897=489%:%
%:%898=489%:%
%:%899=490%:%
%:%900=490%:%
%:%901=491%:%
%:%902=491%:%
%:%903=492%:%
%:%904=492%:%
%:%905=492%:%
%:%906=493%:%
%:%907=493%:%
%:%908=494%:%
%:%909=494%:%
%:%910=494%:%
%:%911=495%:%
%:%912=495%:%
%:%913=496%:%
%:%919=496%:%
%:%922=497%:%
%:%923=498%:%
%:%924=498%:%
%:%925=499%:%
%:%926=500%:%
%:%933=501%:%
%:%934=501%:%
%:%935=502%:%
%:%936=502%:%
%:%937=503%:%
%:%938=504%:%
%:%939=504%:%
%:%940=505%:%
%:%941=505%:%
%:%942=505%:%
%:%943=506%:%
%:%944=506%:%
%:%945=507%:%
%:%946=507%:%
%:%947=507%:%
%:%948=508%:%
%:%949=508%:%
%:%950=509%:%
%:%951=509%:%
%:%952=509%:%
%:%953=510%:%
%:%954=510%:%
%:%955=511%:%
%:%961=511%:%
%:%964=512%:%
%:%965=513%:%
%:%966=513%:%
%:%967=514%:%
%:%968=515%:%
%:%975=516%:%
%:%976=516%:%
%:%977=517%:%
%:%978=517%:%
%:%979=518%:%
%:%980=519%:%
%:%981=519%:%
%:%982=520%:%
%:%983=520%:%
%:%984=520%:%
%:%985=521%:%
%:%986=521%:%
%:%987=522%:%
%:%988=522%:%
%:%989=522%:%
%:%990=523%:%
%:%991=523%:%
%:%992=524%:%
%:%993=524%:%
%:%994=524%:%
%:%995=525%:%
%:%996=525%:%
%:%997=526%:%
%:%1003=526%:%
%:%1006=527%:%
%:%1007=528%:%
%:%1008=528%:%
%:%1009=529%:%
%:%1010=530%:%
%:%1017=531%:%
%:%1018=531%:%
%:%1019=532%:%
%:%1020=532%:%
%:%1021=533%:%
%:%1022=534%:%
%:%1023=534%:%
%:%1024=535%:%
%:%1025=535%:%
%:%1026=535%:%
%:%1027=536%:%
%:%1028=536%:%
%:%1029=537%:%
%:%1030=537%:%
%:%1031=537%:%
%:%1032=538%:%
%:%1033=538%:%
%:%1034=539%:%
%:%1035=539%:%
%:%1036=539%:%
%:%1037=540%:%
%:%1038=540%:%
%:%1039=541%:%
%:%1049=543%:%
%:%1050=544%:%
%:%1051=545%:%
%:%1052=546%:%
%:%1053=547%:%
%:%1054=548%:%
%:%1055=549%:%
%:%1056=550%:%
%:%1057=551%:%
%:%1058=552%:%
%:%1059=553%:%
%:%1060=554%:%
%:%1061=555%:%
%:%1062=556%:%
%:%1063=557%:%
%:%1064=558%:%
%:%1065=559%:%
%:%1066=560%:%
%:%1067=561%:%
%:%1068=562%:%
%:%1069=563%:%
%:%1070=564%:%
%:%1071=565%:%
%:%1072=566%:%
%:%1073=567%:%
%:%1074=568%:%
%:%1075=569%:%
%:%1076=570%:%
%:%1077=571%:%
%:%1078=572%:%
%:%1079=573%:%
%:%1080=574%:%
%:%1082=576%:%
%:%1083=576%:%
%:%1090=577%:%
%:%1091=577%:%
%:%1092=578%:%
%:%1093=578%:%
%:%1094=579%:%
%:%1095=579%:%
%:%1096=579%:%
%:%1097=580%:%
%:%1098=580%:%
%:%1099=581%:%
%:%1100=581%:%
%:%1101=582%:%
%:%1102=582%:%
%:%1103=583%:%
%:%1104=583%:%
%:%1105=583%:%
%:%1106=584%:%
%:%1107=584%:%
%:%1108=585%:%
%:%1109=585%:%
%:%1110=586%:%
%:%1111=586%:%
%:%1112=587%:%
%:%1113=587%:%
%:%1114=587%:%
%:%1115=588%:%
%:%1116=588%:%
%:%1117=589%:%
%:%1118=589%:%
%:%1119=590%:%
%:%1120=590%:%
%:%1121=591%:%
%:%1122=591%:%
%:%1123=591%:%
%:%1124=592%:%
%:%1125=592%:%
%:%1126=593%:%
%:%1127=593%:%
%:%1128=594%:%
%:%1129=594%:%
%:%1130=595%:%
%:%1131=595%:%
%:%1132=595%:%
%:%1133=596%:%
%:%1134=596%:%
%:%1135=597%:%
%:%1136=597%:%
%:%1137=598%:%
%:%1138=598%:%
%:%1139=599%:%
%:%1140=599%:%
%:%1141=599%:%
%:%1142=600%:%
%:%1143=600%:%
%:%1144=601%:%
%:%1154=603%:%
%:%1156=605%:%
%:%1157=605%:%
%:%1160=606%:%
%:%1164=606%:%
%:%1165=606%:%
%:%1174=608%:%
%:%1175=609%:%
%:%1176=610%:%
%:%1177=611%:%
%:%1178=612%:%
%:%1179=613%:%
%:%1180=614%:%
%:%1181=615%:%
%:%1183=617%:%
%:%1184=617%:%
%:%1185=618%:%
%:%1186=619%:%
%:%1193=620%:%
%:%1194=620%:%
%:%1195=621%:%
%:%1196=621%:%
%:%1197=622%:%
%:%1198=622%:%
%:%1199=623%:%
%:%1200=623%:%
%:%1201=624%:%
%:%1202=624%:%
%:%1203=624%:%
%:%1204=625%:%
%:%1205=625%:%
%:%1206=626%:%
%:%1212=626%:%
%:%1215=627%:%
%:%1216=628%:%
%:%1217=628%:%
%:%1218=629%:%
%:%1219=630%:%
%:%1226=631%:%
%:%1227=631%:%
%:%1228=632%:%
%:%1229=632%:%
%:%1230=633%:%
%:%1231=633%:%
%:%1232=634%:%
%:%1233=634%:%
%:%1234=635%:%
%:%1235=635%:%
%:%1236=635%:%
%:%1237=636%:%
%:%1238=636%:%
%:%1239=637%:%
%:%1249=639%:%
%:%1250=640%:%
%:%1252=642%:%
%:%1253=642%:%
%:%1256=643%:%
%:%1260=643%:%
%:%1261=643%:%
%:%1266=643%:%
%:%1269=644%:%
%:%1270=645%:%
%:%1271=645%:%
%:%1274=646%:%
%:%1278=646%:%
%:%1279=646%:%
%:%1288=648%:%
%:%1289=649%:%
%:%1290=650%:%
%:%1291=651%:%
%:%1292=652%:%
%:%1293=653%:%
%:%1294=654%:%
%:%1295=655%:%
%:%1296=656%:%
%:%1297=657%:%
%:%1298=658%:%
%:%1299=659%:%
%:%1300=660%:%
%:%1301=661%:%
%:%1302=662%:%
%:%1303=663%:%
%:%1304=664%:%
%:%1305=665%:%
%:%1306=666%:%
%:%1307=667%:%
%:%1308=668%:%
%:%1309=669%:%
%:%1310=670%:%
%:%1311=671%:%
%:%1312=672%:%
%:%1313=673%:%
%:%1314=674%:%
%:%1315=675%:%
%:%1316=676%:%
%:%1317=677%:%
%:%1318=678%:%
%:%1319=679%:%
%:%1320=680%:%
%:%1321=681%:%
%:%1322=682%:%
%:%1323=683%:%
%:%1324=684%:%
%:%1325=685%:%
%:%1326=686%:%
%:%1327=687%:%
%:%1328=688%:%
%:%1328=689%:%
%:%1329=690%:%
%:%1330=691%:%
%:%1331=692%:%
%:%1332=693%:%
%:%1334=695%:%
%:%1335=695%:%
%:%1338=696%:%
%:%1342=696%:%
%:%1352=698%:%
%:%1353=699%:%
%:%1354=700%:%
%:%1355=701%:%
%:%1356=702%:%
%:%1357=703%:%
%:%1358=704%:%
%:%1359=705%:%
%:%1360=706%:%
%:%1361=707%:%
%:%1362=708%:%
%:%1363=709%:%
%:%1364=710%:%
%:%1365=711%:%
%:%1367=713%:%
%:%1368=713%:%
%:%1369=714%:%
%:%1372=715%:%
%:%1376=715%:%
%:%1377=715%:%
%:%1378=716%:%
%:%1379=717%:%
%:%1380=717%:%
%:%1381=718%:%
%:%1382=718%:%
%:%1383=719%:%
%:%1384=720%:%
%:%1385=720%:%
%:%1386=721%:%
%:%1387=722%:%
%:%1388=722%:%
%:%1389=723%:%
%:%1390=724%:%
%:%1391=724%:%
%:%1392=725%:%
%:%1393=726%:%
%:%1394=726%:%
%:%1399=726%:%
%:%1402=727%:%
%:%1405=729%:%
%:%1406=730%:%
%:%1407=731%:%
%:%1408=732%:%
%:%1409=733%:%
%:%1410=734%:%
%:%1411=735%:%
%:%1412=736%:%
%:%1413=737%:%
%:%1414=738%:%
%:%1415=739%:%
%:%1416=740%:%
%:%1417=741%:%
%:%1418=742%:%
%:%1419=743%:%
%:%1421=745%:%
%:%1422=745%:%
%:%1423=746%:%
%:%1430=747%:%
%:%1431=747%:%
%:%1432=748%:%
%:%1433=748%:%
%:%1434=749%:%
%:%1435=749%:%
%:%1436=750%:%
%:%1437=750%:%
%:%1438=750%:%
%:%1439=751%:%
%:%1440=751%:%
%:%1441=752%:%
%:%1442=752%:%
%:%1443=752%:%
%:%1444=753%:%
%:%1445=753%:%
%:%1446=754%:%
%:%1447=754%:%
%:%1448=754%:%
%:%1449=755%:%
%:%1450=755%:%
%:%1451=756%:%
%:%1452=756%:%
%:%1453=756%:%
%:%1454=757%:%
%:%1455=757%:%
%:%1456=758%:%
%:%1457=758%:%
%:%1458=758%:%
%:%1459=759%:%
%:%1460=759%:%
%:%1461=760%:%
%:%1467=760%:%
%:%1470=761%:%
%:%1471=762%:%
%:%1472=762%:%
%:%1473=763%:%
%:%1480=764%:%
%:%1481=764%:%
%:%1482=765%:%
%:%1483=765%:%
%:%1484=766%:%
%:%1485=766%:%
%:%1486=767%:%
%:%1487=767%:%
%:%1488=767%:%
%:%1489=768%:%
%:%1490=768%:%
%:%1491=769%:%
%:%1492=769%:%
%:%1493=769%:%
%:%1494=770%:%
%:%1495=770%:%
%:%1496=771%:%
%:%1497=771%:%
%:%1498=771%:%
%:%1499=772%:%
%:%1500=772%:%
%:%1501=773%:%
%:%1507=773%:%
%:%1510=774%:%
%:%1511=775%:%
%:%1512=775%:%
%:%1513=776%:%
%:%1514=777%:%
%:%1521=778%:%
%:%1522=778%:%
%:%1523=779%:%
%:%1524=779%:%
%:%1525=780%:%
%:%1526=780%:%
%:%1527=781%:%
%:%1528=781%:%
%:%1529=781%:%
%:%1530=782%:%
%:%1531=782%:%
%:%1532=783%:%
%:%1533=783%:%
%:%1534=783%:%
%:%1535=784%:%
%:%1536=784%:%
%:%1537=785%:%
%:%1538=785%:%
%:%1539=785%:%
%:%1540=786%:%
%:%1541=786%:%
%:%1542=787%:%
%:%1543=787%:%
%:%1544=787%:%
%:%1545=788%:%
%:%1546=788%:%
%:%1547=789%:%
%:%1553=789%:%
%:%1556=790%:%
%:%1557=791%:%
%:%1558=791%:%
%:%1559=792%:%
%:%1560=793%:%
%:%1561=794%:%
%:%1568=795%:%
%:%1569=795%:%
%:%1570=796%:%
%:%1571=796%:%
%:%1572=797%:%
%:%1573=797%:%
%:%1574=798%:%
%:%1575=798%:%
%:%1576=798%:%
%:%1577=799%:%
%:%1578=799%:%
%:%1579=800%:%
%:%1580=800%:%
%:%1581=800%:%
%:%1582=801%:%
%:%1583=801%:%
%:%1584=802%:%
%:%1585=802%:%
%:%1586=803%:%
%:%1587=803%:%
%:%1588=803%:%
%:%1589=804%:%
%:%1590=804%:%
%:%1591=805%:%
%:%1592=805%:%
%:%1593=805%:%
%:%1594=806%:%
%:%1595=806%:%
%:%1596=807%:%
%:%1597=807%:%
%:%1598=807%:%
%:%1599=808%:%
%:%1600=808%:%
%:%1601=809%:%
%:%1607=809%:%
%:%1610=810%:%
%:%1611=811%:%
%:%1612=811%:%
%:%1613=812%:%
%:%1614=813%:%
%:%1615=814%:%
%:%1622=815%:%
%:%1623=815%:%
%:%1624=816%:%
%:%1625=816%:%
%:%1626=817%:%
%:%1627=817%:%
%:%1628=818%:%
%:%1629=818%:%
%:%1630=818%:%
%:%1631=819%:%
%:%1632=819%:%
%:%1633=820%:%
%:%1634=820%:%
%:%1635=820%:%
%:%1636=821%:%
%:%1637=821%:%
%:%1638=822%:%
%:%1639=822%:%
%:%1640=823%:%
%:%1641=823%:%
%:%1642=823%:%
%:%1643=824%:%
%:%1644=824%:%
%:%1645=825%:%
%:%1646=825%:%
%:%1647=825%:%
%:%1648=826%:%
%:%1649=826%:%
%:%1650=827%:%
%:%1651=827%:%
%:%1652=827%:%
%:%1653=828%:%
%:%1654=828%:%
%:%1655=829%:%
%:%1661=829%:%
%:%1664=830%:%
%:%1665=831%:%
%:%1666=831%:%
%:%1667=832%:%
%:%1668=833%:%
%:%1669=834%:%
%:%1676=835%:%
%:%1677=835%:%
%:%1678=836%:%
%:%1679=836%:%
%:%1680=837%:%
%:%1681=837%:%
%:%1682=838%:%
%:%1683=838%:%
%:%1684=838%:%
%:%1685=839%:%
%:%1686=839%:%
%:%1687=840%:%
%:%1688=840%:%
%:%1689=840%:%
%:%1690=841%:%
%:%1691=841%:%
%:%1692=842%:%
%:%1693=842%:%
%:%1694=843%:%
%:%1695=843%:%
%:%1696=843%:%
%:%1697=844%:%
%:%1698=844%:%
%:%1699=845%:%
%:%1700=845%:%
%:%1701=845%:%
%:%1702=846%:%
%:%1703=846%:%
%:%1704=847%:%
%:%1705=847%:%
%:%1706=847%:%
%:%1707=848%:%
%:%1708=848%:%
%:%1709=849%:%
%:%1715=849%:%
%:%1718=850%:%
%:%1719=851%:%
%:%1720=851%:%
%:%1721=852%:%
%:%1728=853%:%
%:%1729=853%:%
%:%1730=854%:%
%:%1731=854%:%
%:%1732=855%:%
%:%1733=855%:%
%:%1734=855%:%
%:%1735=855%:%
%:%1736=856%:%
%:%1737=856%:%
%:%1738=857%:%
%:%1739=857%:%
%:%1740=858%:%
%:%1741=858%:%
%:%1742=858%:%
%:%1743=858%:%
%:%1744=859%:%
%:%1745=859%:%
%:%1746=860%:%
%:%1747=860%:%
%:%1748=861%:%
%:%1749=861%:%
%:%1750=861%:%
%:%1751=861%:%
%:%1752=862%:%
%:%1753=862%:%
%:%1754=863%:%
%:%1755=863%:%
%:%1756=864%:%
%:%1757=864%:%
%:%1758=864%:%
%:%1759=864%:%
%:%1760=865%:%
%:%1761=865%:%
%:%1762=866%:%
%:%1763=866%:%
%:%1764=867%:%
%:%1765=867%:%
%:%1766=867%:%
%:%1767=867%:%
%:%1768=868%:%
%:%1769=868%:%
%:%1770=869%:%
%:%1771=869%:%
%:%1772=870%:%
%:%1773=870%:%
%:%1774=870%:%
%:%1775=870%:%
%:%1776=871%:%
%:%1786=873%:%
%:%1787=874%:%
%:%1789=876%:%
%:%1790=876%:%
%:%1793=877%:%
%:%1797=877%:%
%:%1798=877%:%
%:%1807=879%:%
%:%1808=880%:%
%:%1809=881%:%
%:%1810=882%:%
%:%1811=883%:%
%:%1812=884%:%
%:%1813=885%:%
%:%1814=886%:%
%:%1815=887%:%
%:%1816=888%:%
%:%1817=889%:%
%:%1818=890%:%
%:%1819=891%:%
%:%1820=892%:%
%:%1821=893%:%
%:%1822=894%:%
%:%1823=895%:%
%:%1824=896%:%
%:%1825=897%:%
%:%1826=898%:%
%:%1827=899%:%
%:%1828=900%:%
%:%1829=901%:%
%:%1830=902%:%
%:%1831=903%:%
%:%1832=904%:%
%:%1833=905%:%
%:%1834=906%:%
%:%1835=907%:%
%:%1836=908%:%
%:%1837=909%:%
%:%1838=910%:%
%:%1839=911%:%
%:%1840=912%:%
%:%1841=913%:%
%:%1842=914%:%
%:%1843=915%:%
%:%1844=916%:%
%:%1845=917%:%
%:%1846=918%:%
%:%1847=919%:%
%:%1848=920%:%
%:%1849=921%:%
%:%1850=922%:%
%:%1851=923%:%
%:%1852=924%:%
%:%1853=925%:%
%:%1854=926%:%
%:%1855=927%:%
%:%1856=928%:%
%:%1857=929%:%
%:%1858=930%:%
%:%1859=931%:%
%:%1860=932%:%
%:%1862=934%:%
%:%1863=934%:%
%:%1866=935%:%
%:%1870=935%:%
%:%1880=937%:%
%:%1882=939%:%
%:%1883=939%:%
%:%1884=940%:%
%:%1885=941%:%
%:%1892=942%:%
%:%1893=942%:%
%:%1894=943%:%
%:%1895=943%:%
%:%1896=944%:%
%:%1897=944%:%
%:%1898=945%:%
%:%1899=945%:%
%:%1900=946%:%
%:%1901=946%:%
%:%1902=946%:%
%:%1903=947%:%
%:%1904=947%:%
%:%1905=948%:%
%:%1906=948%:%
%:%1907=948%:%
%:%1908=949%:%
%:%1909=949%:%
%:%1910=950%:%
%:%1916=950%:%
%:%1919=951%:%
%:%1920=952%:%
%:%1921=952%:%
%:%1922=953%:%
%:%1923=954%:%
%:%1930=955%:%
%:%1931=955%:%
%:%1932=956%:%
%:%1933=956%:%
%:%1934=957%:%
%:%1935=957%:%
%:%1936=958%:%
%:%1937=958%:%
%:%1938=959%:%
%:%1939=959%:%
%:%1940=959%:%
%:%1941=960%:%
%:%1942=960%:%
%:%1943=961%:%
%:%1944=961%:%
%:%1945=961%:%
%:%1946=962%:%
%:%1947=962%:%
%:%1948=963%:%
%:%1954=963%:%
%:%1957=964%:%
%:%1958=965%:%
%:%1959=965%:%
%:%1960=966%:%
%:%1961=967%:%
%:%1962=968%:%
%:%1969=969%:%
%:%1970=969%:%
%:%1971=970%:%
%:%1972=970%:%
%:%1973=971%:%
%:%1974=971%:%
%:%1975=972%:%
%:%1976=972%:%
%:%1977=973%:%
%:%1978=973%:%
%:%1979=973%:%
%:%1980=974%:%
%:%1981=974%:%
%:%1982=975%:%
%:%1983=975%:%
%:%1984=975%:%
%:%1985=976%:%
%:%1986=976%:%
%:%1987=977%:%
%:%1988=977%:%
%:%1989=978%:%
%:%1990=978%:%
%:%1991=978%:%
%:%1992=979%:%
%:%1993=979%:%
%:%1994=980%:%
%:%1995=980%:%
%:%1996=980%:%
%:%1997=981%:%
%:%1998=981%:%
%:%1999=982%:%
%:%2000=982%:%
%:%2001=983%:%
%:%2002=983%:%
%:%2003=984%:%
%:%2004=984%:%
%:%2005=984%:%
%:%2006=985%:%
%:%2007=985%:%
%:%2008=986%:%
%:%2009=986%:%
%:%2010=986%:%
%:%2011=987%:%
%:%2012=987%:%
%:%2013=988%:%
%:%2014=988%:%
%:%2015=988%:%
%:%2016=989%:%
%:%2017=989%:%
%:%2018=990%:%
%:%2019=990%:%
%:%2020=991%:%
%:%2026=991%:%
%:%2029=992%:%
%:%2030=993%:%
%:%2031=993%:%
%:%2032=994%:%
%:%2033=995%:%
%:%2034=996%:%
%:%2035=997%:%
%:%2042=998%:%
%:%2043=998%:%
%:%2044=999%:%
%:%2045=999%:%
%:%2046=1000%:%
%:%2047=1000%:%
%:%2048=1001%:%
%:%2049=1001%:%
%:%2050=1002%:%
%:%2051=1002%:%
%:%2052=1002%:%
%:%2053=1003%:%
%:%2054=1003%:%
%:%2055=1004%:%
%:%2056=1004%:%
%:%2057=1004%:%
%:%2058=1005%:%
%:%2059=1005%:%
%:%2060=1006%:%
%:%2061=1006%:%
%:%2062=1007%:%
%:%2063=1007%:%
%:%2064=1007%:%
%:%2065=1008%:%
%:%2066=1008%:%
%:%2067=1009%:%
%:%2068=1009%:%
%:%2069=1009%:%
%:%2070=1010%:%
%:%2071=1010%:%
%:%2072=1011%:%
%:%2073=1011%:%
%:%2074=1012%:%
%:%2075=1012%:%
%:%2076=1013%:%
%:%2077=1013%:%
%:%2078=1013%:%
%:%2079=1014%:%
%:%2080=1014%:%
%:%2081=1015%:%
%:%2082=1015%:%
%:%2083=1015%:%
%:%2084=1016%:%
%:%2085=1016%:%
%:%2086=1017%:%
%:%2087=1017%:%
%:%2088=1018%:%
%:%2089=1018%:%
%:%2090=1018%:%
%:%2091=1019%:%
%:%2092=1019%:%
%:%2093=1020%:%
%:%2094=1020%:%
%:%2095=1020%:%
%:%2096=1021%:%
%:%2097=1021%:%
%:%2098=1022%:%
%:%2099=1022%:%
%:%2100=1022%:%
%:%2101=1023%:%
%:%2102=1023%:%
%:%2103=1024%:%
%:%2104=1024%:%
%:%2105=1024%:%
%:%2106=1025%:%
%:%2107=1025%:%
%:%2108=1026%:%
%:%2109=1026%:%
%:%2110=1027%:%
%:%2111=1027%:%
%:%2112=1028%:%
%:%2113=1028%:%
%:%2114=1028%:%
%:%2115=1029%:%
%:%2116=1029%:%
%:%2117=1030%:%
%:%2118=1030%:%
%:%2119=1030%:%
%:%2120=1031%:%
%:%2121=1031%:%
%:%2122=1032%:%
%:%2123=1032%:%
%:%2124=1032%:%
%:%2125=1033%:%
%:%2126=1033%:%
%:%2127=1034%:%
%:%2128=1034%:%
%:%2129=1034%:%
%:%2130=1035%:%
%:%2131=1035%:%
%:%2132=1036%:%
%:%2133=1036%:%
%:%2134=1037%:%
%:%2135=1037%:%
%:%2136=1038%:%
%:%2142=1038%:%
%:%2145=1039%:%
%:%2146=1040%:%
%:%2147=1040%:%
%:%2148=1041%:%
%:%2149=1042%:%
%:%2150=1043%:%
%:%2151=1044%:%
%:%2158=1045%:%
%:%2159=1045%:%
%:%2160=1046%:%
%:%2161=1046%:%
%:%2162=1047%:%
%:%2163=1047%:%
%:%2164=1048%:%
%:%2165=1048%:%
%:%2166=1049%:%
%:%2167=1049%:%
%:%2168=1049%:%
%:%2169=1050%:%
%:%2170=1050%:%
%:%2171=1051%:%
%:%2172=1051%:%
%:%2173=1051%:%
%:%2174=1052%:%
%:%2175=1052%:%
%:%2176=1053%:%
%:%2177=1053%:%
%:%2178=1054%:%
%:%2179=1054%:%
%:%2180=1054%:%
%:%2181=1055%:%
%:%2182=1055%:%
%:%2183=1056%:%
%:%2184=1056%:%
%:%2185=1056%:%
%:%2186=1057%:%
%:%2187=1057%:%
%:%2188=1058%:%
%:%2189=1058%:%
%:%2190=1059%:%
%:%2191=1059%:%
%:%2192=1060%:%
%:%2193=1060%:%
%:%2194=1060%:%
%:%2195=1061%:%
%:%2196=1061%:%
%:%2197=1062%:%
%:%2198=1062%:%
%:%2199=1062%:%
%:%2200=1063%:%
%:%2201=1063%:%
%:%2202=1064%:%
%:%2203=1064%:%
%:%2204=1065%:%
%:%2205=1065%:%
%:%2206=1065%:%
%:%2207=1066%:%
%:%2208=1066%:%
%:%2209=1067%:%
%:%2210=1067%:%
%:%2211=1067%:%
%:%2212=1068%:%
%:%2213=1068%:%
%:%2214=1069%:%
%:%2215=1069%:%
%:%2216=1069%:%
%:%2217=1070%:%
%:%2218=1070%:%
%:%2219=1071%:%
%:%2220=1071%:%
%:%2221=1071%:%
%:%2222=1072%:%
%:%2223=1072%:%
%:%2224=1073%:%
%:%2225=1073%:%
%:%2226=1074%:%
%:%2227=1074%:%
%:%2228=1075%:%
%:%2229=1075%:%
%:%2230=1075%:%
%:%2231=1076%:%
%:%2232=1076%:%
%:%2233=1077%:%
%:%2234=1077%:%
%:%2235=1077%:%
%:%2236=1078%:%
%:%2237=1078%:%
%:%2238=1079%:%
%:%2239=1079%:%
%:%2240=1079%:%
%:%2241=1080%:%
%:%2242=1080%:%
%:%2243=1081%:%
%:%2244=1081%:%
%:%2245=1081%:%
%:%2246=1082%:%
%:%2247=1082%:%
%:%2248=1083%:%
%:%2249=1083%:%
%:%2250=1084%:%
%:%2251=1084%:%
%:%2252=1085%:%
%:%2258=1085%:%
%:%2261=1086%:%
%:%2262=1087%:%
%:%2263=1087%:%
%:%2264=1088%:%
%:%2265=1089%:%
%:%2266=1090%:%
%:%2267=1091%:%
%:%2274=1092%:%
%:%2275=1092%:%
%:%2276=1093%:%
%:%2277=1093%:%
%:%2278=1094%:%
%:%2279=1094%:%
%:%2280=1095%:%
%:%2281=1095%:%
%:%2282=1096%:%
%:%2283=1096%:%
%:%2284=1096%:%
%:%2285=1097%:%
%:%2286=1097%:%
%:%2287=1098%:%
%:%2288=1098%:%
%:%2289=1098%:%
%:%2290=1099%:%
%:%2291=1099%:%
%:%2292=1100%:%
%:%2293=1100%:%
%:%2294=1101%:%
%:%2295=1101%:%
%:%2296=1101%:%
%:%2297=1102%:%
%:%2298=1102%:%
%:%2299=1103%:%
%:%2300=1103%:%
%:%2301=1103%:%
%:%2302=1104%:%
%:%2303=1104%:%
%:%2304=1105%:%
%:%2305=1105%:%
%:%2306=1106%:%
%:%2307=1106%:%
%:%2308=1107%:%
%:%2309=1107%:%
%:%2310=1107%:%
%:%2311=1108%:%
%:%2312=1108%:%
%:%2313=1109%:%
%:%2314=1109%:%
%:%2315=1109%:%
%:%2316=1110%:%
%:%2317=1110%:%
%:%2318=1111%:%
%:%2319=1111%:%
%:%2320=1112%:%
%:%2321=1112%:%
%:%2322=1112%:%
%:%2323=1113%:%
%:%2324=1113%:%
%:%2325=1114%:%
%:%2326=1114%:%
%:%2327=1114%:%
%:%2328=1115%:%
%:%2329=1115%:%
%:%2330=1116%:%
%:%2331=1116%:%
%:%2332=1116%:%
%:%2333=1117%:%
%:%2334=1117%:%
%:%2335=1118%:%
%:%2336=1118%:%
%:%2337=1118%:%
%:%2338=1119%:%
%:%2339=1119%:%
%:%2340=1120%:%
%:%2341=1120%:%
%:%2342=1121%:%
%:%2343=1121%:%
%:%2344=1122%:%
%:%2345=1122%:%
%:%2346=1122%:%
%:%2347=1123%:%
%:%2348=1123%:%
%:%2349=1124%:%
%:%2350=1124%:%
%:%2351=1124%:%
%:%2352=1125%:%
%:%2353=1125%:%
%:%2354=1126%:%
%:%2355=1126%:%
%:%2356=1126%:%
%:%2357=1127%:%
%:%2358=1127%:%
%:%2359=1128%:%
%:%2360=1128%:%
%:%2361=1128%:%
%:%2362=1129%:%
%:%2363=1129%:%
%:%2364=1130%:%
%:%2365=1130%:%
%:%2366=1131%:%
%:%2367=1131%:%
%:%2368=1132%:%
%:%2374=1132%:%
%:%2377=1133%:%
%:%2378=1134%:%
%:%2379=1134%:%
%:%2386=1135%:%
%:%2387=1135%:%
%:%2388=1136%:%
%:%2389=1136%:%
%:%2390=1137%:%
%:%2391=1137%:%
%:%2392=1137%:%
%:%2393=1137%:%
%:%2394=1138%:%
%:%2395=1138%:%
%:%2396=1139%:%
%:%2397=1139%:%
%:%2398=1140%:%
%:%2399=1140%:%
%:%2400=1140%:%
%:%2401=1140%:%
%:%2402=1141%:%
%:%2403=1141%:%
%:%2404=1142%:%
%:%2405=1142%:%
%:%2406=1143%:%
%:%2407=1143%:%
%:%2408=1143%:%
%:%2409=1143%:%
%:%2410=1144%:%
%:%2411=1144%:%
%:%2412=1145%:%
%:%2413=1145%:%
%:%2414=1146%:%
%:%2415=1146%:%
%:%2416=1146%:%
%:%2417=1146%:%
%:%2418=1147%:%
%:%2419=1147%:%
%:%2420=1148%:%
%:%2421=1148%:%
%:%2422=1149%:%
%:%2423=1149%:%
%:%2424=1149%:%
%:%2425=1149%:%
%:%2426=1150%:%
%:%2427=1150%:%
%:%2428=1151%:%
%:%2429=1151%:%
%:%2430=1152%:%
%:%2431=1152%:%
%:%2432=1152%:%
%:%2433=1152%:%
%:%2434=1153%:%
%:%2444=1155%:%
%:%2446=1157%:%
%:%2447=1157%:%
%:%2450=1158%:%
%:%2454=1158%:%
%:%2455=1158%:%
%:%2464=1160%:%
%:%2465=1161%:%
%:%2466=1162%:%
%:%2467=1163%:%
%:%2468=1164%:%
%:%2469=1165%:%
%:%2470=1166%:%
%:%2471=1167%:%
%:%2472=1168%:%
%:%2473=1169%:%
%:%2474=1170%:%
%:%2475=1171%:%
%:%2476=1172%:%
%:%2477=1173%:%
%:%2478=1174%:%
%:%2479=1175%:%
%:%2480=1176%:%
%:%2481=1177%:%
%:%2482=1178%:%
%:%2483=1179%:%
%:%2484=1180%:%
%:%2485=1181%:%
%:%2486=1182%:%
%:%2487=1183%:%
%:%2488=1184%:%
%:%2489=1185%:%
%:%2490=1186%:%
%:%2491=1187%:%
%:%2492=1188%:%
%:%2493=1189%:%
%:%2494=1190%:%
%:%2495=1191%:%
%:%2496=1192%:%
%:%2497=1193%:%
%:%2498=1194%:%
%:%2499=1195%:%
%:%2500=1196%:%
%:%2501=1197%:%
%:%2502=1198%:%
%:%2503=1199%:%
%:%2504=1200%:%
%:%2505=1201%:%
%:%2506=1202%:%
%:%2507=1203%:%
%:%2508=1204%:%
%:%2509=1205%:%
%:%2510=1206%:%
%:%2514=1208%:%
%:%2515=1209%:%
%:%2517=1211%:%
%:%2518=1211%:%
%:%2519=1212%:%
%:%2520=1213%:%
%:%2527=1214%:%
%:%2528=1214%:%
%:%2529=1215%:%
%:%2530=1215%:%
%:%2531=1215%:%
%:%2532=1216%:%
%:%2533=1216%:%
%:%2534=1217%:%
%:%2535=1217%:%
%:%2536=1217%:%
%:%2537=1218%:%
%:%2538=1218%:%
%:%2539=1219%:%
%:%2540=1219%:%
%:%2541=1219%:%
%:%2542=1220%:%
%:%2543=1220%:%
%:%2544=1221%:%
%:%2550=1221%:%
%:%2553=1222%:%
%:%2554=1223%:%
%:%2555=1223%:%
%:%2556=1224%:%
%:%2557=1225%:%
%:%2564=1226%:%
%:%2565=1226%:%
%:%2566=1227%:%
%:%2567=1227%:%
%:%2568=1228%:%
%:%2569=1228%:%
%:%2570=1229%:%
%:%2571=1229%:%
%:%2572=1230%:%
%:%2573=1230%:%
%:%2574=1230%:%
%:%2575=1231%:%
%:%2576=1231%:%
%:%2577=1232%:%
%:%2578=1232%:%
%:%2579=1232%:%
%:%2580=1233%:%
%:%2581=1233%:%
%:%2582=1234%:%
%:%2588=1234%:%
%:%2591=1235%:%
%:%2592=1236%:%
%:%2593=1236%:%
%:%2594=1237%:%
%:%2595=1238%:%
%:%2596=1239%:%
%:%2603=1240%:%
%:%2604=1240%:%
%:%2605=1241%:%
%:%2606=1241%:%
%:%2607=1242%:%
%:%2608=1242%:%
%:%2609=1243%:%
%:%2610=1243%:%
%:%2611=1244%:%
%:%2612=1244%:%
%:%2613=1244%:%
%:%2614=1245%:%
%:%2615=1245%:%
%:%2616=1246%:%
%:%2617=1246%:%
%:%2618=1246%:%
%:%2619=1247%:%
%:%2620=1247%:%
%:%2621=1248%:%
%:%2622=1248%:%
%:%2623=1249%:%
%:%2624=1249%:%
%:%2625=1249%:%
%:%2626=1250%:%
%:%2627=1250%:%
%:%2628=1251%:%
%:%2629=1251%:%
%:%2630=1251%:%
%:%2631=1252%:%
%:%2632=1252%:%
%:%2633=1253%:%
%:%2634=1253%:%
%:%2635=1254%:%
%:%2636=1254%:%
%:%2637=1255%:%
%:%2638=1255%:%
%:%2639=1255%:%
%:%2640=1256%:%
%:%2641=1256%:%
%:%2642=1257%:%
%:%2643=1257%:%
%:%2644=1257%:%
%:%2645=1258%:%
%:%2646=1258%:%
%:%2647=1259%:%
%:%2648=1259%:%
%:%2649=1259%:%
%:%2650=1260%:%
%:%2651=1260%:%
%:%2652=1260%:%
%:%2653=1261%:%
%:%2654=1261%:%
%:%2655=1262%:%
%:%2661=1262%:%
%:%2664=1263%:%
%:%2665=1264%:%
%:%2666=1264%:%
%:%2667=1265%:%
%:%2668=1266%:%
%:%2669=1267%:%
%:%2670=1268%:%
%:%2671=1269%:%
%:%2672=1270%:%
%:%2679=1271%:%
%:%2680=1271%:%
%:%2681=1272%:%
%:%2682=1272%:%
%:%2683=1273%:%
%:%2684=1273%:%
%:%2685=1274%:%
%:%2686=1274%:%
%:%2687=1275%:%
%:%2688=1275%:%
%:%2689=1275%:%
%:%2690=1276%:%
%:%2691=1276%:%
%:%2692=1277%:%
%:%2693=1277%:%
%:%2694=1277%:%
%:%2695=1278%:%
%:%2696=1278%:%
%:%2697=1279%:%
%:%2698=1279%:%
%:%2699=1280%:%
%:%2700=1280%:%
%:%2701=1280%:%
%:%2702=1281%:%
%:%2703=1281%:%
%:%2704=1282%:%
%:%2705=1282%:%
%:%2706=1282%:%
%:%2707=1283%:%
%:%2708=1283%:%
%:%2709=1284%:%
%:%2710=1284%:%
%:%2711=1285%:%
%:%2712=1285%:%
%:%2713=1286%:%
%:%2714=1286%:%
%:%2715=1286%:%
%:%2716=1287%:%
%:%2717=1287%:%
%:%2718=1288%:%
%:%2719=1288%:%
%:%2720=1288%:%
%:%2721=1289%:%
%:%2722=1289%:%
%:%2723=1290%:%
%:%2724=1290%:%
%:%2725=1291%:%
%:%2726=1291%:%
%:%2727=1291%:%
%:%2728=1292%:%
%:%2729=1292%:%
%:%2730=1293%:%
%:%2731=1293%:%
%:%2732=1293%:%
%:%2733=1294%:%
%:%2734=1294%:%
%:%2735=1295%:%
%:%2736=1295%:%
%:%2737=1295%:%
%:%2738=1296%:%
%:%2739=1296%:%
%:%2740=1297%:%
%:%2741=1297%:%
%:%2742=1297%:%
%:%2743=1298%:%
%:%2744=1298%:%
%:%2745=1299%:%
%:%2746=1299%:%
%:%2747=1300%:%
%:%2748=1300%:%
%:%2749=1301%:%
%:%2750=1301%:%
%:%2751=1301%:%
%:%2752=1302%:%
%:%2753=1302%:%
%:%2754=1303%:%
%:%2755=1303%:%
%:%2756=1303%:%
%:%2757=1304%:%
%:%2758=1304%:%
%:%2759=1305%:%
%:%2760=1305%:%
%:%2761=1305%:%
%:%2762=1306%:%
%:%2763=1306%:%
%:%2764=1307%:%
%:%2765=1307%:%
%:%2766=1307%:%
%:%2767=1308%:%
%:%2768=1308%:%
%:%2769=1309%:%
%:%2770=1309%:%
%:%2771=1310%:%
%:%2772=1310%:%
%:%2773=1311%:%
%:%2779=1311%:%
%:%2782=1312%:%
%:%2783=1313%:%
%:%2784=1313%:%
%:%2785=1314%:%
%:%2786=1315%:%
%:%2787=1316%:%
%:%2788=1317%:%
%:%2789=1318%:%
%:%2790=1319%:%
%:%2797=1320%:%
%:%2798=1320%:%
%:%2799=1321%:%
%:%2800=1321%:%
%:%2801=1322%:%
%:%2802=1322%:%
%:%2803=1323%:%
%:%2804=1323%:%
%:%2805=1324%:%
%:%2806=1324%:%
%:%2807=1324%:%
%:%2808=1325%:%
%:%2809=1325%:%
%:%2810=1326%:%
%:%2811=1326%:%
%:%2812=1326%:%
%:%2813=1327%:%
%:%2814=1327%:%
%:%2815=1328%:%
%:%2816=1328%:%
%:%2817=1329%:%
%:%2818=1329%:%
%:%2819=1329%:%
%:%2820=1330%:%
%:%2821=1330%:%
%:%2822=1331%:%
%:%2823=1331%:%
%:%2824=1331%:%
%:%2825=1332%:%
%:%2826=1332%:%
%:%2827=1333%:%
%:%2828=1333%:%
%:%2829=1334%:%
%:%2830=1334%:%
%:%2831=1335%:%
%:%2832=1335%:%
%:%2833=1335%:%
%:%2834=1336%:%
%:%2835=1336%:%
%:%2836=1337%:%
%:%2837=1337%:%
%:%2838=1337%:%
%:%2839=1338%:%
%:%2840=1338%:%
%:%2841=1339%:%
%:%2842=1339%:%
%:%2843=1340%:%
%:%2844=1340%:%
%:%2845=1340%:%
%:%2846=1341%:%
%:%2847=1341%:%
%:%2848=1342%:%
%:%2849=1342%:%
%:%2850=1342%:%
%:%2851=1343%:%
%:%2852=1343%:%
%:%2853=1344%:%
%:%2854=1344%:%
%:%2855=1344%:%
%:%2856=1345%:%
%:%2857=1345%:%
%:%2858=1346%:%
%:%2859=1346%:%
%:%2860=1346%:%
%:%2861=1347%:%
%:%2862=1347%:%
%:%2863=1348%:%
%:%2864=1348%:%
%:%2865=1349%:%
%:%2866=1349%:%
%:%2867=1350%:%
%:%2868=1350%:%
%:%2869=1350%:%
%:%2870=1351%:%
%:%2871=1351%:%
%:%2872=1352%:%
%:%2873=1352%:%
%:%2874=1352%:%
%:%2875=1353%:%
%:%2876=1353%:%
%:%2877=1354%:%
%:%2878=1354%:%
%:%2879=1354%:%
%:%2880=1355%:%
%:%2881=1355%:%
%:%2882=1356%:%
%:%2883=1356%:%
%:%2884=1356%:%
%:%2885=1357%:%
%:%2886=1357%:%
%:%2887=1358%:%
%:%2888=1358%:%
%:%2889=1359%:%
%:%2890=1359%:%
%:%2891=1360%:%
%:%2897=1360%:%
%:%2900=1361%:%
%:%2901=1362%:%
%:%2902=1362%:%
%:%2903=1363%:%
%:%2904=1364%:%
%:%2905=1365%:%
%:%2906=1366%:%
%:%2907=1367%:%
%:%2908=1368%:%
%:%2915=1369%:%
%:%2916=1369%:%
%:%2917=1370%:%
%:%2918=1370%:%
%:%2919=1371%:%
%:%2920=1371%:%
%:%2921=1372%:%
%:%2922=1372%:%
%:%2923=1373%:%
%:%2924=1373%:%
%:%2925=1373%:%
%:%2926=1374%:%
%:%2927=1374%:%
%:%2928=1375%:%
%:%2929=1375%:%
%:%2930=1375%:%
%:%2931=1376%:%
%:%2932=1376%:%
%:%2933=1377%:%
%:%2934=1377%:%
%:%2935=1378%:%
%:%2936=1378%:%
%:%2937=1378%:%
%:%2938=1379%:%
%:%2939=1379%:%
%:%2940=1380%:%
%:%2941=1380%:%
%:%2942=1380%:%
%:%2943=1381%:%
%:%2944=1381%:%
%:%2945=1382%:%
%:%2946=1382%:%
%:%2947=1383%:%
%:%2948=1383%:%
%:%2949=1384%:%
%:%2950=1384%:%
%:%2951=1384%:%
%:%2952=1385%:%
%:%2953=1385%:%
%:%2954=1386%:%
%:%2955=1386%:%
%:%2956=1386%:%
%:%2957=1387%:%
%:%2958=1387%:%
%:%2959=1388%:%
%:%2960=1388%:%
%:%2961=1389%:%
%:%2962=1389%:%
%:%2963=1389%:%
%:%2964=1390%:%
%:%2965=1390%:%
%:%2966=1391%:%
%:%2967=1391%:%
%:%2968=1391%:%
%:%2969=1392%:%
%:%2970=1392%:%
%:%2971=1393%:%
%:%2972=1393%:%
%:%2973=1393%:%
%:%2974=1394%:%
%:%2975=1394%:%
%:%2976=1395%:%
%:%2977=1395%:%
%:%2978=1395%:%
%:%2979=1396%:%
%:%2980=1396%:%
%:%2981=1397%:%
%:%2982=1397%:%
%:%2983=1398%:%
%:%2984=1398%:%
%:%2985=1399%:%
%:%2986=1399%:%
%:%2987=1399%:%
%:%2988=1400%:%
%:%2989=1400%:%
%:%2990=1401%:%
%:%2991=1401%:%
%:%2992=1401%:%
%:%2993=1402%:%
%:%2994=1402%:%
%:%2995=1403%:%
%:%2996=1403%:%
%:%2997=1403%:%
%:%2998=1404%:%
%:%2999=1404%:%
%:%3000=1405%:%
%:%3001=1405%:%
%:%3002=1405%:%
%:%3003=1406%:%
%:%3004=1406%:%
%:%3005=1407%:%
%:%3006=1407%:%
%:%3007=1408%:%
%:%3008=1408%:%
%:%3009=1409%:%
%:%3015=1409%:%
%:%3018=1410%:%
%:%3019=1411%:%
%:%3020=1411%:%
%:%3021=1412%:%
%:%3028=1413%:%
%:%3029=1413%:%
%:%3030=1414%:%
%:%3031=1414%:%
%:%3032=1415%:%
%:%3033=1415%:%
%:%3034=1415%:%
%:%3035=1415%:%
%:%3036=1416%:%
%:%3037=1416%:%
%:%3038=1417%:%
%:%3039=1417%:%
%:%3040=1418%:%
%:%3041=1418%:%
%:%3042=1418%:%
%:%3043=1418%:%
%:%3044=1419%:%
%:%3045=1419%:%
%:%3046=1420%:%
%:%3047=1420%:%
%:%3048=1421%:%
%:%3049=1421%:%
%:%3050=1421%:%
%:%3051=1421%:%
%:%3052=1422%:%
%:%3053=1422%:%
%:%3054=1423%:%
%:%3055=1423%:%
%:%3056=1424%:%
%:%3057=1424%:%
%:%3058=1424%:%
%:%3059=1424%:%
%:%3060=1425%:%
%:%3061=1425%:%
%:%3062=1426%:%
%:%3063=1426%:%
%:%3064=1427%:%
%:%3065=1427%:%
%:%3066=1427%:%
%:%3067=1427%:%
%:%3068=1428%:%
%:%3069=1428%:%
%:%3070=1429%:%
%:%3071=1429%:%
%:%3072=1430%:%
%:%3073=1430%:%
%:%3074=1430%:%
%:%3075=1430%:%
%:%3076=1431%:%
%:%3086=1433%:%
%:%3088=1435%:%
%:%3089=1435%:%
%:%3090=1436%:%
%:%3093=1437%:%
%:%3097=1437%:%
%:%3098=1437%:%
%:%3107=1439%:%
%:%3108=1440%:%
%:%3109=1441%:%
%:%3110=1442%:%
%:%3111=1443%:%
%:%3112=1444%:%
%:%3113=1445%:%
%:%3114=1446%:%
%:%3115=1447%:%
%:%3116=1448%:%
%:%3117=1449%:%
%:%3118=1450%:%
%:%3119=1451%:%
%:%3120=1452%:%
%:%3121=1453%:%
%:%3122=1454%:%
%:%3123=1455%:%
%:%3124=1456%:%
%:%3125=1457%:%
%:%3126=1458%:%
%:%3127=1459%:%
%:%3128=1460%:%
%:%3129=1461%:%
%:%3131=1463%:%
%:%3132=1463%:%
%:%3133=1464%:%
%:%3134=1465%:%
%:%3135=1466%:%
%:%3142=1467%:%
%:%3143=1467%:%
%:%3144=1468%:%
%:%3145=1468%:%
%:%3146=1468%:%
%:%3147=1469%:%
%:%3148=1469%:%
%:%3149=1470%:%
%:%3150=1470%:%
%:%3151=1470%:%
%:%3152=1471%:%
%:%3153=1471%:%
%:%3154=1472%:%
%:%3155=1472%:%
%:%3156=1472%:%
%:%3157=1472%:%
%:%3158=1473%:%
%:%3159=1473%:%
%:%3160=1474%:%
%:%3161=1474%:%
%:%3162=1475%:%
%:%3163=1475%:%
%:%3164=1476%:%
%:%3165=1476%:%
%:%3166=1476%:%
%:%3167=1477%:%
%:%3168=1477%:%
%:%3169=1478%:%
%:%3170=1478%:%
%:%3171=1479%:%
%:%3177=1479%:%
%:%3180=1480%:%
%:%3181=1481%:%
%:%3182=1481%:%
%:%3183=1482%:%
%:%3185=1484%:%
%:%3188=1485%:%
%:%3192=1485%:%
%:%3193=1485%:%
%:%3194=1485%:%
%:%3203=1487%:%
%:%3207=1489%:%
%:%3208=1490%:%
%:%3210=1492%:%
%:%3211=1492%:%
%:%3212=1493%:%
%:%3213=1494%:%
%:%3214=1495%:%
%:%3215=1496%:%
%:%3216=1497%:%
%:%3217=1498%:%
%:%3218=1499%:%
%:%3221=1500%:%
%:%3225=1500%:%
%:%3235=1502%:%
%:%3236=1503%:%
%:%3237=1504%:%
%:%3238=1505%:%
%:%3239=1506%:%
%:%3240=1507%:%
%:%3242=1509%:%
%:%3243=1509%:%
%:%3244=1510%:%
%:%3245=1511%:%
%:%3252=1512%:%
%:%3253=1512%:%
%:%3254=1513%:%
%:%3255=1513%:%
%:%3256=1514%:%
%:%3257=1514%:%
%:%3258=1514%:%
%:%3259=1514%:%
%:%3260=1515%:%
%:%3261=1515%:%
%:%3262=1515%:%
%:%3263=1516%:%
%:%3264=1516%:%
%:%3265=1516%:%
%:%3266=1516%:%
%:%3267=1517%:%
%:%3268=1517%:%
%:%3269=1517%:%
%:%3270=1518%:%
%:%3271=1518%:%
%:%3272=1519%:%
%:%3278=1519%:%
%:%3281=1520%:%
%:%3282=1521%:%
%:%3283=1521%:%
%:%3284=1522%:%
%:%3285=1523%:%
%:%3292=1524%:%
%:%3293=1524%:%
%:%3294=1525%:%
%:%3295=1525%:%
%:%3296=1526%:%
%:%3297=1527%:%
%:%3298=1527%:%
%:%3299=1528%:%
%:%3300=1528%:%
%:%3301=1528%:%
%:%3302=1529%:%
%:%3303=1529%:%
%:%3304=1529%:%
%:%3305=1529%:%
%:%3306=1530%:%
%:%3307=1530%:%
%:%3308=1531%:%
%:%3309=1531%:%
%:%3310=1532%:%
%:%3311=1532%:%
%:%3312=1532%:%
%:%3313=1532%:%
%:%3314=1533%:%
%:%3315=1533%:%
%:%3316=1533%:%
%:%3317=1533%:%
%:%3318=1534%:%
%:%3319=1534%:%
%:%3320=1534%:%
%:%3321=1534%:%
%:%3322=1535%:%
%:%3332=1537%:%
%:%3334=1539%:%
%:%3335=1539%:%
%:%3336=1540%:%
%:%3337=1541%:%
%:%3338=1542%:%
%:%3339=1543%:%
%:%3340=1544%:%
%:%3341=1545%:%
%:%3342=1546%:%
%:%3345=1547%:%
%:%3349=1547%:%
%:%3350=1547%:%
%:%3351=1547%:%
%:%3352=1548%:%
%:%3362=1550%:%
%:%3366=1552%:%
%:%3370=1554%:%
%:%3371=1555%:%
%:%3372=1556%:%
%:%3374=1558%:%
%:%3375=1558%:%
%:%3378=1559%:%
%:%3382=1559%:%
%:%3392=1561%:%
%:%3393=1562%:%
%:%3394=1563%:%
%:%3395=1564%:%
%:%3396=1565%:%
%:%3398=1567%:%
%:%3399=1567%:%
%:%3400=1568%:%
%:%3401=1569%:%
%:%3402=1570%:%
%:%3403=1570%:%
%:%3404=1571%:%
%:%3405=1572%:%
%:%3406=1573%:%
%:%3407=1573%:%
%:%3408=1574%:%
%:%3409=1575%:%
%:%3412=1575%:%
%:%3416=1575%:%
%:%3424=1575%:%
%:%3425=1576%:%
%:%3426=1577%:%
%:%3429=1579%:%
%:%3430=1580%:%
%:%3431=1581%:%
%:%3432=1582%:%
%:%3433=1583%:%
%:%3434=1584%:%
%:%3435=1585%:%
%:%3436=1586%:%
%:%3437=1587%:%
%:%3438=1588%:%
%:%3439=1589%:%
%:%3440=1590%:%
%:%3441=1591%:%
%:%3443=1593%:%
%:%3444=1593%:%
%:%3445=1594%:%
%:%3446=1594%:%
%:%3448=1597%:%
%:%3449=1598%:%
%:%3450=1599%:%
%:%3451=1600%:%
%:%3453=1602%:%
%:%3454=1602%:%
%:%3455=1603%:%
%:%3457=1605%:%
%:%3458=1606%:%
%:%3459=1607%:%
%:%3460=1608%:%
%:%3461=1609%:%
%:%3462=1610%:%
%:%3463=1611%:%
%:%3464=1612%:%
%:%3466=1614%:%
%:%3467=1614%:%
%:%3470=1615%:%
%:%3474=1615%:%
%:%3475=1615%:%
%:%3484=1617%:%
%:%3485=1618%:%
%:%3494=1620%:%
%:%3506=1622%:%
%:%3507=1623%:%
%:%3508=1624%:%
%:%3509=1625%:%
%:%3510=1626%:%
%:%3511=1627%:%
%:%3512=1628%:%
%:%3513=1629%:%
%:%3514=1630%:%
%:%3516=1632%:%
%:%3517=1632%:%
%:%3518=1633%:%
%:%3520=1635%:%
%:%3521=1636%:%
%:%3522=1637%:%
%:%3523=1638%:%
%:%3524=1639%:%
%:%3525=1640%:%
%:%3526=1641%:%
%:%3528=1643%:%
%:%3529=1643%:%
%:%3532=1644%:%
%:%3536=1644%:%
%:%3537=1644%:%
%:%3546=1646%:%
%:%3550=1648%:%
%:%3554=1650%:%
%:%3555=1651%:%
%:%3556=1652%:%
%:%3557=1653%:%
%:%3558=1654%:%
%:%3559=1655%:%
%:%3560=1656%:%
%:%3561=1657%:%
%:%3562=1658%:%
%:%3563=1659%:%
%:%3564=1660%:%
%:%3565=1661%:%
%:%3566=1662%:%
%:%3567=1663%:%
%:%3568=1664%:%
%:%3569=1665%:%
%:%3570=1666%:%
%:%3571=1667%:%
%:%3572=1668%:%
%:%3573=1669%:%
%:%3574=1670%:%
%:%3575=1671%:%
%:%3576=1672%:%
%:%3577=1673%:%
%:%3578=1674%:%
%:%3579=1675%:%
%:%3580=1676%:%
%:%3581=1677%:%
%:%3582=1678%:%
%:%3583=1679%:%
%:%3584=1680%:%
%:%3585=1681%:%
%:%3586=1682%:%
%:%3587=1683%:%
%:%3588=1684%:%
%:%3589=1685%:%
%:%3590=1686%:%
%:%3591=1687%:%
%:%3592=1688%:%
%:%3593=1689%:%
%:%3594=1690%:%
%:%3595=1691%:%
%:%3596=1692%:%
%:%3597=1693%:%
%:%3598=1694%:%
%:%3599=1695%:%
%:%3603=1705%:%
%:%3604=1706%:%
%:%3605=1707%:%
%:%3606=1708%:%
%:%3607=1709%:%
%:%3608=1710%:%
%:%3609=1711%:%
%:%3610=1712%:%
%:%3611=1713%:%
%:%3612=1714%:%
%:%3613=1715%:%
%:%3614=1716%:%
%:%3615=1717%:%
%:%3616=1718%:%
%:%3617=1719%:%
%:%3618=1720%:%
%:%3619=1721%:%
%:%3620=1722%:%
%:%3621=1723%:%
%:%3622=1724%:%
%:%3623=1725%:%
%:%3624=1726%:%
%:%3625=1727%:%
%:%3626=1728%:%
%:%3627=1729%:%
%:%3628=1730%:%
%:%3629=1731%:%



\chapter{Sintaxis}
%
\begin{isabellebody}%
\setisabellecontext{Sintaxis}%
%
\isadelimtheory
%
\endisadelimtheory
%
\isatagtheory
%
\endisatagtheory
{\isafoldtheory}%
%
\isadelimtheory
%
\endisadelimtheory
%
\isadelimdocument
%
\endisadelimdocument
%
\isatagdocument
%
\isamarkupsection{Fórmulas%
}
\isamarkuptrue%
%
\endisatagdocument
{\isafolddocument}%
%
\isadelimdocument
%
\endisadelimdocument
%
\begin{isamarkuptext}%
\comentario{Explicar la siguiente notación y recolocarla donde se
  use por primera vez.}

  \comentario{He quitado la palabra "continuación" del fichero 
  castellano.tex ya que no dejaba cargar el documento}%
\end{isamarkuptext}\isamarkuptrue%
\isacommand{notation}\isamarkupfalse%
\ insert\ {\isacharparenleft}{\isachardoublequoteopen}{\isacharunderscore}\ {\isasymtriangleright}\ {\isacharunderscore}{\isachardoublequoteclose}\ {\isacharbrackleft}{\isadigit{5}}{\isadigit{6}}{\isacharcomma}{\isadigit{5}}{\isadigit{5}}{\isacharbrackright}\ {\isadigit{5}}{\isadigit{5}}{\isacharparenright}%
\begin{isamarkuptext}%
En esta sección presentaremos una formalización en Isabelle de la 
  sintaxis de la lógica proposicional, junto con resultados y pruebas 
  sobre la misma. En líneas generales, primero daremos las nociones de 
  forma clásica y, a continuación, su correspondiente formalización.

  En primer lugar, supondremos que disponemos de los siguientes 
  elementos:
  \begin{description}
    \item[Alfabeto:] Es una lista infinita de variables proposicionales. 
      También pueden ser llamadas átomos o símbolos proposicionales.
    \item[Conectivas:] Conjunto finito cuyos elementos interactúan con 
      las variables. Pueden ser monarias que afectan a un único elemento 
      o binarias que afectan a dos. En el primer grupo se encuentra le 
      negación (\isa{{\isasymnot}}) y en el segundo la conjunción (\isa{{\isasymand}}), la disyunción 
      (\isa{{\isasymor}}) y la implicación (\isa{{\isasymlongrightarrow}}).
  \end{description}

  A continuación definiremos la estructura de fórmula sobre los 
  elementos anteriores. Para ello daremos una definición recursiva 
  basada en dos elementos: un conjunto de fórmulas básicas y una serie 
  de procedimientos de definición de fórmulas a partir de otras. El 
  conjunto de las fórmulas será el menor conjunto de estructuras 
  sinctáticas con dicho alfabeto y conectivas que contiene a las básicas 
  y es cerrado mediante los procedimientos de definición que mostraremos 
  a continuación.

  \begin{definicion}
    El conjunto de las fórmulas proposicionales está formado por las 
    siguientes:
    \begin{itemize}
      \item Las fórmulas atómicas, constituidas únicamente por una 
        variable del alfabeto. 
      \item La constante \isa{{\isasymbottom}}.
      \item Dada una fórmula \isa{F}, la negación \isa{{\isasymnot}\ F} es una fórmula.
      \item Dadas dos fórmulas \isa{F} y \isa{G}, la conjunción \isa{F\ {\isasymand}\ G} es una
        fórmula.
      \item Dadas dos fórmulas \isa{F} y \isa{G}, la disyunción \isa{F\ {\isasymor}\ G} es una
        fórmula.
      \item Dadas dos fórmulas \isa{F} y \isa{G}, la implicación \isa{F\ {\isasymlongrightarrow}\ G} es 
        una fórmula.
    \end{itemize}
  \end{definicion}

  Intuitivamente, las fórmulas proposicionales son entendidas como un 
  tipo de árbol sintáctico cuyos nodos son las conectivas y sus hojas
  las fórmulas atómicas.

  \comentario{Incluir el árbol de formación.}

  A continuación, veamos su representación en Isabelle%
\end{isamarkuptext}\isamarkuptrue%
\isacommand{datatype}\isamarkupfalse%
\ {\isacharparenleft}atoms{\isacharcolon}\ {\isacharprime}a{\isacharparenright}\ formula\ {\isacharequal}\ \isanewline
\ \ Atom\ {\isacharprime}a\isanewline
{\isacharbar}\ Bot\ \ \ \ \ \ \ \ \ \ \ \ \ \ \ \ \ \ \ \ \ \ \ \ \ \ \ \ \ \ {\isacharparenleft}{\isachardoublequoteopen}{\isasymbottom}{\isachardoublequoteclose}{\isacharparenright}\ \ \isanewline
{\isacharbar}\ Not\ {\isachardoublequoteopen}{\isacharprime}a\ formula{\isachardoublequoteclose}\ \ \ \ \ \ \ \ \ \ \ \ \ \ \ \ \ {\isacharparenleft}{\isachardoublequoteopen}\isactrlbold {\isasymnot}{\isachardoublequoteclose}{\isacharparenright}\isanewline
{\isacharbar}\ And\ {\isachardoublequoteopen}{\isacharprime}a\ formula{\isachardoublequoteclose}\ {\isachardoublequoteopen}{\isacharprime}a\ formula{\isachardoublequoteclose}\ \ \ \ {\isacharparenleft}\isakeyword{infix}\ {\isachardoublequoteopen}\isactrlbold {\isasymand}{\isachardoublequoteclose}\ {\isadigit{6}}{\isadigit{8}}{\isacharparenright}\isanewline
{\isacharbar}\ Or\ {\isachardoublequoteopen}{\isacharprime}a\ formula{\isachardoublequoteclose}\ {\isachardoublequoteopen}{\isacharprime}a\ formula{\isachardoublequoteclose}\ \ \ \ \ {\isacharparenleft}\isakeyword{infix}\ {\isachardoublequoteopen}\isactrlbold {\isasymor}{\isachardoublequoteclose}\ {\isadigit{6}}{\isadigit{8}}{\isacharparenright}\isanewline
{\isacharbar}\ Imp\ {\isachardoublequoteopen}{\isacharprime}a\ formula{\isachardoublequoteclose}\ {\isachardoublequoteopen}{\isacharprime}a\ formula{\isachardoublequoteclose}\ \ \ \ {\isacharparenleft}\isakeyword{infixr}\ {\isachardoublequoteopen}\isactrlbold {\isasymrightarrow}{\isachardoublequoteclose}\ {\isadigit{6}}{\isadigit{8}}{\isacharparenright}%
\begin{isamarkuptext}%
Como podemos observar representamos las fórmulas proposicionales
  mediante un tipo de dato recursivo, \isa{formula}, con los 
  siguientes constructures sobre un tipo \isa{{\isacharprime}a} cualquiera:

  \begin{description}
    \item[Fórmulas básicas:]  
      \begin{itemize}
        \item \isa{Atom\ {\isacharcolon}{\isacharcolon}\ {\isacharprime}a\ {\isasymRightarrow}\ {\isacharprime}a\ formula}
        \item \isa{{\isasymbottom}\ {\isacharcolon}{\isacharcolon}\ {\isacharprime}a\ formula}
      \end{itemize}
    \item [Fórmulas compuestas:]
      \begin{itemize}
        \item \isa{\isactrlbold {\isasymnot}\ {\isacharcolon}{\isacharcolon}\ {\isacharprime}a\ formula\ {\isasymRightarrow}\ {\isacharprime}a\ formula}
        \item \isa{{\isacharparenleft}\isactrlbold {\isasymand}{\isacharparenright}\ {\isacharcolon}{\isacharcolon}\ {\isacharprime}a\ formula\ {\isasymRightarrow}\ {\isacharprime}a\ formula\ {\isasymRightarrow}\ {\isacharprime}a\ formula}
        \item \isa{{\isacharparenleft}\isactrlbold {\isasymor}{\isacharparenright}\ {\isacharcolon}{\isacharcolon}\ {\isacharprime}a\ formula\ {\isasymRightarrow}\ {\isacharprime}a\ formula\ {\isasymRightarrow}\ {\isacharprime}a\ formula}
        \item \isa{{\isacharparenleft}\isactrlbold {\isasymrightarrow}{\isacharparenright}\ {\isacharcolon}{\isacharcolon}\ {\isacharprime}a\ formula\ {\isasymRightarrow}\ {\isacharprime}a\ formula\ {\isasymRightarrow}\ {\isacharprime}a\ formula}
      \end{itemize}
  \end{description}

  Cabe señalar que los términos \isa{infix} e \isa{infixr} nos señalan que 
  los constructores que representan a las conectivas se pueden usar de
  forma infija. En particular, \isa{infixr} se trata de un infijo asociado a 
  la derecha.

  Además se define simultáneamente la función \isa{atoms\ {\isacharcolon}{\isacharcolon}\ {\isacharprime}a\ formula\ {\isasymRightarrow}\ {\isacharprime}a\ set}, que 
  obtiene el conjunto de variables proposicionales de una fórmula. 

  Por otro lado, la definición de \isa{formula} genera 
  automáticamente los siguientes lemas sobre la función de conjuntos 
  \isa{atoms} en Isabelle.
  
  \begin{itemize}
    \item[] \isa{atoms\ {\isacharparenleft}Atom\ x{\isadigit{1}}{\isachardot}{\isadigit{0}}{\isacharparenright}\ {\isacharequal}\ {\isacharbraceleft}x{\isadigit{1}}{\isachardot}{\isadigit{0}}{\isacharbraceright}\isasep\isanewline%
atoms\ {\isasymbottom}\ {\isacharequal}\ {\isasymemptyset}\isasep\isanewline%
atoms\ {\isacharparenleft}\isactrlbold {\isasymnot}\ x{\isadigit{3}}{\isachardot}{\isadigit{0}}{\isacharparenright}\ {\isacharequal}\ atoms\ x{\isadigit{3}}{\isachardot}{\isadigit{0}}\isasep\isanewline%
atoms\ {\isacharparenleft}x{\isadigit{4}}{\isadigit{1}}{\isachardot}{\isadigit{0}}\ \isactrlbold {\isasymand}\ x{\isadigit{4}}{\isadigit{2}}{\isachardot}{\isadigit{0}}{\isacharparenright}\ {\isacharequal}\ atoms\ x{\isadigit{4}}{\isadigit{1}}{\isachardot}{\isadigit{0}}\ {\isasymunion}\ atoms\ x{\isadigit{4}}{\isadigit{2}}{\isachardot}{\isadigit{0}}\isasep\isanewline%
atoms\ {\isacharparenleft}x{\isadigit{5}}{\isadigit{1}}{\isachardot}{\isadigit{0}}\ \isactrlbold {\isasymor}\ x{\isadigit{5}}{\isadigit{2}}{\isachardot}{\isadigit{0}}{\isacharparenright}\ {\isacharequal}\ atoms\ x{\isadigit{5}}{\isadigit{1}}{\isachardot}{\isadigit{0}}\ {\isasymunion}\ atoms\ x{\isadigit{5}}{\isadigit{2}}{\isachardot}{\isadigit{0}}\isasep\isanewline%
atoms\ {\isacharparenleft}x{\isadigit{6}}{\isadigit{1}}{\isachardot}{\isadigit{0}}\ \isactrlbold {\isasymrightarrow}\ x{\isadigit{6}}{\isadigit{2}}{\isachardot}{\isadigit{0}}{\isacharparenright}\ {\isacharequal}\ atoms\ x{\isadigit{6}}{\isadigit{1}}{\isachardot}{\isadigit{0}}\ {\isasymunion}\ atoms\ x{\isadigit{6}}{\isadigit{2}}{\isachardot}{\isadigit{0}}}
  \end{itemize} 

  A continuación veremos varios ejemplos de fórmulas y el conjunto de 
  sus variables proposicionales obtenido mediante \isa{atoms}. Se 
  observa que, por definición de conjunto, no contiene 
  elementos repetidos.%
\end{isamarkuptext}\isamarkuptrue%
\isacommand{notepad}\isamarkupfalse%
\ \isanewline
\isakeyword{begin}\isanewline
%
\isadelimproof
\ \ %
\endisadelimproof
%
\isatagproof
\isacommand{fix}\isamarkupfalse%
\ p\ q\ r\ {\isacharcolon}{\isacharcolon}\ {\isacharprime}a\isanewline
\isanewline
\ \ \isacommand{have}\isamarkupfalse%
\ {\isachardoublequoteopen}atoms\ {\isacharparenleft}Atom\ p{\isacharparenright}\ {\isacharequal}\ {\isacharbraceleft}p{\isacharbraceright}{\isachardoublequoteclose}\isanewline
\ \ \ \ \isacommand{by}\isamarkupfalse%
\ {\isacharparenleft}simp\ only{\isacharcolon}\ formula{\isachardot}set{\isacharparenright}\isanewline
\isanewline
\ \ \isacommand{have}\isamarkupfalse%
\ {\isachardoublequoteopen}atoms\ {\isacharparenleft}\isactrlbold {\isasymnot}\ {\isacharparenleft}Atom\ p{\isacharparenright}{\isacharparenright}\ {\isacharequal}\ {\isacharbraceleft}p{\isacharbraceright}{\isachardoublequoteclose}\isanewline
\ \ \ \ \isacommand{by}\isamarkupfalse%
\ {\isacharparenleft}simp\ only{\isacharcolon}\ formula{\isachardot}set{\isacharparenright}\isanewline
\isanewline
\ \ \isacommand{have}\isamarkupfalse%
\ {\isachardoublequoteopen}atoms\ {\isacharparenleft}{\isacharparenleft}Atom\ p\ \isactrlbold {\isasymrightarrow}\ Atom\ q{\isacharparenright}\ \isactrlbold {\isasymor}\ Atom\ r{\isacharparenright}\ {\isacharequal}\ {\isacharbraceleft}p{\isacharcomma}q{\isacharcomma}r{\isacharbraceright}{\isachardoublequoteclose}\isanewline
\ \ \ \ \isacommand{by}\isamarkupfalse%
\ auto\isanewline
\isanewline
\ \ \isacommand{have}\isamarkupfalse%
\ {\isachardoublequoteopen}atoms\ {\isacharparenleft}{\isacharparenleft}Atom\ p\ \isactrlbold {\isasymrightarrow}\ Atom\ p{\isacharparenright}\ \isactrlbold {\isasymor}\ Atom\ r{\isacharparenright}\ {\isacharequal}\ {\isacharbraceleft}p{\isacharcomma}r{\isacharbraceright}{\isachardoublequoteclose}\isanewline
\ \ \ \ \isacommand{by}\isamarkupfalse%
\ auto%
\endisatagproof
{\isafoldproof}%
%
\isadelimproof
\ \ \isanewline
%
\endisadelimproof
\isacommand{end}\isamarkupfalse%
%
\begin{isamarkuptext}%
En particular, el conjunto de símbolos proposicionales de la 
  fórmula \isa{Bot} es vacío. Además, para calcular esta constante es 
  necesario especificar el tipo sobre el que se construye la fórmula.%
\end{isamarkuptext}\isamarkuptrue%
\isacommand{notepad}\isamarkupfalse%
\ \isanewline
\isakeyword{begin}\isanewline
%
\isadelimproof
\ \ %
\endisadelimproof
%
\isatagproof
\isacommand{fix}\isamarkupfalse%
\ p\ {\isacharcolon}{\isacharcolon}\ {\isacharprime}a\isanewline
\isanewline
\ \ \isacommand{have}\isamarkupfalse%
\ {\isachardoublequoteopen}atoms\ {\isasymbottom}\ {\isacharequal}\ {\isasymemptyset}{\isachardoublequoteclose}\isanewline
\ \ \ \ \isacommand{by}\isamarkupfalse%
\ {\isacharparenleft}simp\ only{\isacharcolon}\ formula{\isachardot}set{\isacharparenright}\isanewline
\isanewline
\ \ \isacommand{have}\isamarkupfalse%
\ {\isachardoublequoteopen}atoms\ {\isacharparenleft}Atom\ p\ \isactrlbold {\isasymor}\ {\isasymbottom}{\isacharparenright}\ {\isacharequal}\ {\isacharbraceleft}p{\isacharbraceright}{\isachardoublequoteclose}\isanewline
\ \ \isacommand{proof}\isamarkupfalse%
\ {\isacharminus}\isanewline
\ \ \ \ \isacommand{have}\isamarkupfalse%
\ {\isachardoublequoteopen}atoms\ {\isacharparenleft}Atom\ p\ \isactrlbold {\isasymor}\ {\isasymbottom}{\isacharparenright}\ {\isacharequal}\ atoms\ {\isacharparenleft}Atom\ p{\isacharparenright}\ {\isasymunion}\ atoms\ Bot{\isachardoublequoteclose}\isanewline
\ \ \ \ \ \ \isacommand{by}\isamarkupfalse%
\ {\isacharparenleft}simp\ only{\isacharcolon}\ formula{\isachardot}set{\isacharparenleft}{\isadigit{5}}{\isacharparenright}{\isacharparenright}\isanewline
\ \ \ \ \isacommand{also}\isamarkupfalse%
\ \isacommand{have}\isamarkupfalse%
\ {\isachardoublequoteopen}{\isasymdots}\ {\isacharequal}\ {\isacharbraceleft}p{\isacharbraceright}\ {\isasymunion}\ atoms\ Bot{\isachardoublequoteclose}\isanewline
\ \ \ \ \ \ \isacommand{by}\isamarkupfalse%
\ {\isacharparenleft}simp\ only{\isacharcolon}\ formula{\isachardot}set{\isacharparenleft}{\isadigit{1}}{\isacharparenright}{\isacharparenright}\isanewline
\ \ \ \ \isacommand{also}\isamarkupfalse%
\ \isacommand{have}\isamarkupfalse%
\ {\isachardoublequoteopen}{\isasymdots}\ {\isacharequal}\ {\isacharbraceleft}p{\isacharbraceright}\ {\isasymunion}\ {\isasymemptyset}{\isachardoublequoteclose}\isanewline
\ \ \ \ \ \ \isacommand{by}\isamarkupfalse%
\ {\isacharparenleft}simp\ only{\isacharcolon}\ formula{\isachardot}set{\isacharparenleft}{\isadigit{2}}{\isacharparenright}{\isacharparenright}\isanewline
\ \ \ \ \isacommand{also}\isamarkupfalse%
\ \isacommand{have}\isamarkupfalse%
\ {\isachardoublequoteopen}{\isasymdots}\ {\isacharequal}\ {\isacharbraceleft}p{\isacharbraceright}{\isachardoublequoteclose}\isanewline
\ \ \ \ \ \ \isacommand{by}\isamarkupfalse%
\ {\isacharparenleft}simp\ only{\isacharcolon}\ Un{\isacharunderscore}empty{\isacharunderscore}right{\isacharparenright}\isanewline
\ \ \ \ \isacommand{finally}\isamarkupfalse%
\ \isacommand{show}\isamarkupfalse%
\ {\isachardoublequoteopen}atoms\ {\isacharparenleft}Atom\ p\ \isactrlbold {\isasymor}\ {\isasymbottom}{\isacharparenright}\ {\isacharequal}\ {\isacharbraceleft}p{\isacharbraceright}{\isachardoublequoteclose}\isanewline
\ \ \ \ \ \ \isacommand{by}\isamarkupfalse%
\ this\isanewline
\ \ \isacommand{qed}\isamarkupfalse%
\isanewline
\isanewline
\ \ \isacommand{have}\isamarkupfalse%
\ {\isachardoublequoteopen}atoms\ {\isacharparenleft}Atom\ p\ \isactrlbold {\isasymor}\ {\isasymbottom}{\isacharparenright}\ {\isacharequal}\ {\isacharbraceleft}p{\isacharbraceright}{\isachardoublequoteclose}\isanewline
\ \ \ \ \isacommand{by}\isamarkupfalse%
\ {\isacharparenleft}simp\ only{\isacharcolon}\ formula{\isachardot}set\ Un{\isacharunderscore}empty{\isacharunderscore}right{\isacharparenright}%
\endisatagproof
{\isafoldproof}%
%
\isadelimproof
\isanewline
%
\endisadelimproof
\isacommand{end}\isamarkupfalse%
\isanewline
\isanewline
\isacommand{value}\isamarkupfalse%
\ {\isachardoublequoteopen}{\isacharparenleft}Bot{\isacharcolon}{\isacharcolon}nat\ formula{\isacharparenright}{\isachardoublequoteclose}%
\begin{isamarkuptext}%
Una vez definida la estructura de las fórmulas, vamos a introducir 
  el método de demostración que seguirán los resultados que aquí 
  presentaremos, tanto en la teoría clásica como en Isabelle. 

  Según la definición recursiva de las fórmulas, dispondremos de un 
  esquema de inducción sobre las mismas:

  \begin{definicion}
    Sea \isa{{\isasymP}} una propiedad sobre fórmulas que verifica las siguientes 
    condiciones:
    \begin{itemize}
      \item Las fórmulas atómicas la cumplen.
      \item La constante \isa{{\isasymbottom}} la cumple.
      \item Dada \isa{F} fórmula que la cumple, entonces \isa{{\isasymnot}\ F} la cumple.
      \item Dadas \isa{F} y \isa{G} fórmulas que la cumplen, entonces \isa{F\ {\isacharasterisk}\ G} la 
        cumple, donde \isa{{\isacharasterisk}} simboliza cualquier conectiva binaria.
    \end{itemize}
    Entonces, todas las fórmulas proposicionales tienen la propiedad 
    \isa{{\isasymP}}.
  \end{definicion}

  Análogamente, como las fórmulas proposicionales están definidas 
  mediante un tipo de datos recursivo, Isabelle genera de forma 
  automática el esquema de inducción correspondiente. De este modo, en 
  las pruebas formalizadas utilizaremos la táctica \isa{induction}, 
  que corresponde al siguiente esquema.

  \comentario{Poner bien cada regla.}

  \begin{itemize}
    \item[] \isa{\mbox{}\inferrule{\mbox{{\isasymAnd}x{\isachardot}\ P\ {\isacharparenleft}Atom\ x{\isacharparenright}}\\\ \mbox{P\ {\isasymbottom}}\\\ \mbox{{\isasymAnd}x{\isachardot}\ \mbox{}\inferrule{\mbox{P\ x}}{\mbox{P\ {\isacharparenleft}\isactrlbold {\isasymnot}\ x{\isacharparenright}}}}\\\ \mbox{{\isasymAnd}x{\isadigit{1}}a\ x{\isadigit{2}}{\isachardot}\ \mbox{}\inferrule{\mbox{P\ x{\isadigit{1}}a\ {\isasymand}\ P\ x{\isadigit{2}}}}{\mbox{P\ {\isacharparenleft}x{\isadigit{1}}a\ \isactrlbold {\isasymand}\ x{\isadigit{2}}{\isacharparenright}}}}\\\ \mbox{{\isasymAnd}x{\isadigit{1}}a\ x{\isadigit{2}}{\isachardot}\ \mbox{}\inferrule{\mbox{P\ x{\isadigit{1}}a\ {\isasymand}\ P\ x{\isadigit{2}}}}{\mbox{P\ {\isacharparenleft}x{\isadigit{1}}a\ \isactrlbold {\isasymor}\ x{\isadigit{2}}{\isacharparenright}}}}\\\ \mbox{{\isasymAnd}x{\isadigit{1}}a\ x{\isadigit{2}}{\isachardot}\ \mbox{}\inferrule{\mbox{P\ x{\isadigit{1}}a\ {\isasymand}\ P\ x{\isadigit{2}}}}{\mbox{P\ {\isacharparenleft}x{\isadigit{1}}a\ \isactrlbold {\isasymrightarrow}\ x{\isadigit{2}}{\isacharparenright}}}}}{\mbox{P\ formula}}}
  \end{itemize} 

  Como hemos señalado, el esquema inductivo se aplicará en cada uno de 
  los casos de los constructores, desglosándose así seis casos distintos 
  como se muestra anteriormente. Además, todas las demostraciones sobre 
  casos de conectivas binarias son equivalentes en esta sección, pues la 
  construcción sintáctica de fórmulas es idéntica entre ellas. Estas se 
  diferencian esencialmente en la connotación semántica que veremos más 
  adelante.

  Llegamos así al primer resultado de este apartado:

  \begin{lema}
    El conjunto de los átomos de una fórmula proposicional es finito.
  \end{lema}

  Para proceder a la demostración, vamos a dar una definición inductiva 
  de conjunto finito. Cabe añadir que la demostración seguirá el esquema 
  inductivo relativo a la estructura de fórmula, y no el que induce esta
  última definición.

  \begin{definicion}
    Los conjuntos finitos son:
      \begin{itemize}
        \item El vacío.
        \item Dado un conjunto finito \isa{A} y un elemento cualquiera \isa{a}, 
          entonces \isa{{\isacharbraceleft}a{\isacharbraceright}\ {\isasymunion}\ A} es finito.
      \end{itemize}
  \end{definicion}

  En Isabelle, podemos formalizar el lema como sigue.%
\end{isamarkuptext}\isamarkuptrue%
\isacommand{lemma}\isamarkupfalse%
\ {\isachardoublequoteopen}finite\ {\isacharparenleft}atoms\ F{\isacharparenright}{\isachardoublequoteclose}\isanewline
%
\isadelimproof
\ \ %
\endisadelimproof
%
\isatagproof
\isacommand{oops}\isamarkupfalse%
%
\endisatagproof
{\isafoldproof}%
%
\isadelimproof
%
\endisadelimproof
%
\begin{isamarkuptext}%
Análogamente, el enunciado formalizado contiene la definición 
  \isa{finite\ S}, perteneciente a la teoría 
  \href{https://n9.cl/x86r}{FiniteSet.thy}.%
\end{isamarkuptext}\isamarkuptrue%
\isacommand{inductive}\isamarkupfalse%
\ finite{\isacharprime}\ {\isacharcolon}{\isacharcolon}\ {\isachardoublequoteopen}{\isacharprime}a\ set\ {\isasymRightarrow}\ bool{\isachardoublequoteclose}\ \isakeyword{where}\isanewline
\ \ emptyI{\isacharprime}\ {\isacharbrackleft}simp{\isacharcomma}\ intro{\isacharbang}{\isacharbrackright}{\isacharcolon}\ {\isachardoublequoteopen}finite{\isacharprime}\ {\isacharbraceleft}{\isacharbraceright}{\isachardoublequoteclose}\isanewline
{\isacharbar}\ insertI{\isacharprime}\ {\isacharbrackleft}simp{\isacharcomma}\ intro{\isacharbang}{\isacharbrackright}{\isacharcolon}\ {\isachardoublequoteopen}finite{\isacharprime}\ A\ {\isasymLongrightarrow}\ finite{\isacharprime}\ {\isacharparenleft}insert\ a\ A{\isacharparenright}{\isachardoublequoteclose}%
\begin{isamarkuptext}%
Observemos que la definición anterior corresponde a 
  \isa{finite{\isacharprime}}. Sin embargo, es análoga a \isa{finite} de la 
  teoría original. Este cambio de notación es necesario para no definir 
  dos veces de manera idéntica la misma noción en Isabelle. Por otra 
  parte, esta definición permitiría la demostración del lema por 
  simplificacion pues, dentro de ella las reglas que especifica se han 
  añadido como tácticas de \isa{simp} e \isa{intro{\isacharbang}}. Sin embargo, conforme al 
  objetivo de este análisis, detallaremos dónde es usada cada una de las 
  reglas en la prueba detallada. 

  A continuación, veamos en primer lugar la demostración clásica del 
  lema. 

  \begin{demostracion}
  La prueba es por inducción sobre el tipo recursivo de las fórmulas. 
  Veamos cada caso.
  
  Consideremos una fórmula atómica \isa{p} cualquiera. Entonces, 
  su conjunto de variables proposicionales es \isa{{\isacharbraceleft}p{\isacharbraceright}}, finito.

  Sea la fórmula \isa{{\isasymbottom}}. Entonces, su conjunto de átomos es vacío y, por 
  lo tanto, finito.
  
  Sea \isa{F} una fórmula cuyo conjunto de variables proposicionales sea 
  finito. Entonces, por definición, \isa{{\isasymnot}\ F} y \isa{F} tienen igual conjunto de
  átomos y, por hipótesis de inducción, es finito.

  Consideremos las fórmulas \isa{F} y \isa{G} cuyos conjuntos de átomos son 
  finitos. Por construcción, el conjunto de variables de \isa{F{\isacharasterisk}G} es la 
  unión de sus respectivos conjuntos de átomos para cualquier \isa{{\isacharasterisk}} 
  conectiva binaria. Por lo tanto, usando la hipótesis de inducción, 
  dicho conjunto es finito. 
  \end{demostracion} 

  Veamos ahora la prueba detallada en Isabelle. Mostraremos con detalle 
  todos los casos de conectivas binarias, aunque se puede observar que 
  son completamente análogos. Para facilitar la lectura, primero 
  demostraremos por separado cada uno de los casos según el esquema 
  inductivo de fórmulas, y finalmente añadiremos la prueba para una 
  fórmula cualquiera a partir de los anteriores.%
\end{isamarkuptext}\isamarkuptrue%
\isacommand{lemma}\isamarkupfalse%
\ atoms{\isacharunderscore}finite{\isacharunderscore}atom{\isacharcolon}\isanewline
\ \ {\isachardoublequoteopen}finite\ {\isacharparenleft}atoms\ {\isacharparenleft}Atom\ x{\isacharparenright}{\isacharparenright}{\isachardoublequoteclose}\isanewline
%
\isadelimproof
%
\endisadelimproof
%
\isatagproof
\isacommand{proof}\isamarkupfalse%
\ {\isacharminus}\isanewline
\ \ \isacommand{have}\isamarkupfalse%
\ {\isachardoublequoteopen}finite\ {\isasymemptyset}{\isachardoublequoteclose}\isanewline
\ \ \ \ \isacommand{by}\isamarkupfalse%
\ {\isacharparenleft}simp\ only{\isacharcolon}\ finite{\isachardot}emptyI{\isacharparenright}\isanewline
\ \ \isacommand{then}\isamarkupfalse%
\ \isacommand{have}\isamarkupfalse%
\ {\isachardoublequoteopen}finite\ {\isacharbraceleft}x{\isacharbraceright}{\isachardoublequoteclose}\isanewline
\ \ \ \ \isacommand{by}\isamarkupfalse%
\ {\isacharparenleft}simp\ only{\isacharcolon}\ finite{\isacharunderscore}insert{\isacharparenright}\isanewline
\ \ \isacommand{then}\isamarkupfalse%
\ \isacommand{show}\isamarkupfalse%
\ {\isachardoublequoteopen}finite\ {\isacharparenleft}atoms\ {\isacharparenleft}Atom\ x{\isacharparenright}{\isacharparenright}{\isachardoublequoteclose}\isanewline
\ \ \ \ \isacommand{by}\isamarkupfalse%
\ {\isacharparenleft}simp\ only{\isacharcolon}\ formula{\isachardot}set{\isacharparenleft}{\isadigit{1}}{\isacharparenright}{\isacharparenright}\ \isanewline
\isacommand{qed}\isamarkupfalse%
%
\endisatagproof
{\isafoldproof}%
%
\isadelimproof
\isanewline
%
\endisadelimproof
\isanewline
\isacommand{lemma}\isamarkupfalse%
\ atoms{\isacharunderscore}finite{\isacharunderscore}bot{\isacharcolon}\isanewline
\ \ {\isachardoublequoteopen}finite\ {\isacharparenleft}atoms\ {\isasymbottom}{\isacharparenright}{\isachardoublequoteclose}\isanewline
%
\isadelimproof
%
\endisadelimproof
%
\isatagproof
\isacommand{proof}\isamarkupfalse%
\ {\isacharminus}\isanewline
\ \ \isacommand{have}\isamarkupfalse%
\ {\isachardoublequoteopen}finite\ {\isasymemptyset}{\isachardoublequoteclose}\isanewline
\ \ \ \ \isacommand{by}\isamarkupfalse%
\ {\isacharparenleft}simp\ only{\isacharcolon}\ finite{\isachardot}emptyI{\isacharparenright}\isanewline
\ \ \isacommand{then}\isamarkupfalse%
\ \isacommand{show}\isamarkupfalse%
\ {\isachardoublequoteopen}finite\ {\isacharparenleft}atoms\ {\isasymbottom}{\isacharparenright}{\isachardoublequoteclose}\isanewline
\ \ \ \ \isacommand{by}\isamarkupfalse%
\ {\isacharparenleft}simp\ only{\isacharcolon}\ formula{\isachardot}set{\isacharparenleft}{\isadigit{2}}{\isacharparenright}{\isacharparenright}\ \isanewline
\isacommand{qed}\isamarkupfalse%
%
\endisatagproof
{\isafoldproof}%
%
\isadelimproof
\isanewline
%
\endisadelimproof
\isanewline
\isacommand{lemma}\isamarkupfalse%
\ atoms{\isacharunderscore}finite{\isacharunderscore}not{\isacharcolon}\isanewline
\ \ \isakeyword{assumes}\ {\isachardoublequoteopen}finite\ {\isacharparenleft}atoms\ F{\isacharparenright}{\isachardoublequoteclose}\ \isanewline
\ \ \isakeyword{shows}\ \ \ {\isachardoublequoteopen}finite\ {\isacharparenleft}atoms\ {\isacharparenleft}\isactrlbold {\isasymnot}\ F{\isacharparenright}{\isacharparenright}{\isachardoublequoteclose}\isanewline
%
\isadelimproof
\ \ %
\endisadelimproof
%
\isatagproof
\isacommand{using}\isamarkupfalse%
\ assms\isanewline
\ \ \isacommand{by}\isamarkupfalse%
\ {\isacharparenleft}simp\ only{\isacharcolon}\ formula{\isachardot}set{\isacharparenleft}{\isadigit{3}}{\isacharparenright}{\isacharparenright}%
\endisatagproof
{\isafoldproof}%
%
\isadelimproof
\ \isanewline
%
\endisadelimproof
\isanewline
\isacommand{lemma}\isamarkupfalse%
\ atoms{\isacharunderscore}finite{\isacharunderscore}and{\isacharcolon}\isanewline
\ \ \isakeyword{assumes}\ {\isachardoublequoteopen}finite\ {\isacharparenleft}atoms\ F{\isadigit{1}}{\isacharparenright}{\isachardoublequoteclose}\isanewline
\ \ \ \ \ \ \ \ \ \ {\isachardoublequoteopen}finite\ {\isacharparenleft}atoms\ F{\isadigit{2}}{\isacharparenright}{\isachardoublequoteclose}\isanewline
\ \ \isakeyword{shows}\ \ \ {\isachardoublequoteopen}finite\ {\isacharparenleft}atoms\ {\isacharparenleft}F{\isadigit{1}}\ \isactrlbold {\isasymand}\ F{\isadigit{2}}{\isacharparenright}{\isacharparenright}{\isachardoublequoteclose}\isanewline
%
\isadelimproof
%
\endisadelimproof
%
\isatagproof
\isacommand{proof}\isamarkupfalse%
\ {\isacharminus}\isanewline
\ \ \isacommand{have}\isamarkupfalse%
\ {\isachardoublequoteopen}finite\ {\isacharparenleft}atoms\ F{\isadigit{1}}\ {\isasymunion}\ atoms\ F{\isadigit{2}}{\isacharparenright}{\isachardoublequoteclose}\isanewline
\ \ \ \ \isacommand{using}\isamarkupfalse%
\ assms\isanewline
\ \ \ \ \isacommand{by}\isamarkupfalse%
\ {\isacharparenleft}simp\ only{\isacharcolon}\ finite{\isacharunderscore}UnI{\isacharparenright}\isanewline
\ \ \isacommand{then}\isamarkupfalse%
\ \isacommand{show}\isamarkupfalse%
\ {\isachardoublequoteopen}finite\ {\isacharparenleft}atoms\ {\isacharparenleft}F{\isadigit{1}}\ \isactrlbold {\isasymand}\ F{\isadigit{2}}{\isacharparenright}{\isacharparenright}{\isachardoublequoteclose}\ \ \isanewline
\ \ \ \ \isacommand{by}\isamarkupfalse%
\ {\isacharparenleft}simp\ only{\isacharcolon}\ formula{\isachardot}set{\isacharparenleft}{\isadigit{4}}{\isacharparenright}{\isacharparenright}\isanewline
\isacommand{qed}\isamarkupfalse%
%
\endisatagproof
{\isafoldproof}%
%
\isadelimproof
\isanewline
%
\endisadelimproof
\isanewline
\isacommand{lemma}\isamarkupfalse%
\ atoms{\isacharunderscore}finite{\isacharunderscore}or{\isacharcolon}\isanewline
\ \ \isakeyword{assumes}\ {\isachardoublequoteopen}finite\ {\isacharparenleft}atoms\ F{\isadigit{1}}{\isacharparenright}{\isachardoublequoteclose}\isanewline
\ \ \ \ \ \ \ \ \ \ {\isachardoublequoteopen}finite\ {\isacharparenleft}atoms\ F{\isadigit{2}}{\isacharparenright}{\isachardoublequoteclose}\isanewline
\ \ \isakeyword{shows}\ \ \ {\isachardoublequoteopen}finite\ {\isacharparenleft}atoms\ {\isacharparenleft}F{\isadigit{1}}\ \isactrlbold {\isasymor}\ F{\isadigit{2}}{\isacharparenright}{\isacharparenright}{\isachardoublequoteclose}\isanewline
%
\isadelimproof
%
\endisadelimproof
%
\isatagproof
\isacommand{proof}\isamarkupfalse%
\ {\isacharminus}\isanewline
\ \ \isacommand{have}\isamarkupfalse%
\ {\isachardoublequoteopen}finite\ {\isacharparenleft}atoms\ F{\isadigit{1}}\ {\isasymunion}\ atoms\ F{\isadigit{2}}{\isacharparenright}{\isachardoublequoteclose}\isanewline
\ \ \ \ \isacommand{using}\isamarkupfalse%
\ assms\isanewline
\ \ \ \ \isacommand{by}\isamarkupfalse%
\ {\isacharparenleft}simp\ only{\isacharcolon}\ finite{\isacharunderscore}UnI{\isacharparenright}\isanewline
\ \ \isacommand{then}\isamarkupfalse%
\ \isacommand{show}\isamarkupfalse%
\ {\isachardoublequoteopen}finite\ {\isacharparenleft}atoms\ {\isacharparenleft}F{\isadigit{1}}\ \isactrlbold {\isasymor}\ F{\isadigit{2}}{\isacharparenright}{\isacharparenright}{\isachardoublequoteclose}\ \ \isanewline
\ \ \ \ \isacommand{by}\isamarkupfalse%
\ {\isacharparenleft}simp\ only{\isacharcolon}\ formula{\isachardot}set{\isacharparenleft}{\isadigit{5}}{\isacharparenright}{\isacharparenright}\isanewline
\isacommand{qed}\isamarkupfalse%
%
\endisatagproof
{\isafoldproof}%
%
\isadelimproof
\isanewline
%
\endisadelimproof
\isanewline
\isacommand{lemma}\isamarkupfalse%
\ atoms{\isacharunderscore}finite{\isacharunderscore}imp{\isacharcolon}\isanewline
\ \ \isakeyword{assumes}\ {\isachardoublequoteopen}finite\ {\isacharparenleft}atoms\ F{\isadigit{1}}{\isacharparenright}{\isachardoublequoteclose}\isanewline
\ \ \ \ \ \ \ \ \ \ {\isachardoublequoteopen}finite\ {\isacharparenleft}atoms\ F{\isadigit{2}}{\isacharparenright}{\isachardoublequoteclose}\isanewline
\ \ \isakeyword{shows}\ \ \ {\isachardoublequoteopen}finite\ {\isacharparenleft}atoms\ {\isacharparenleft}F{\isadigit{1}}\ \isactrlbold {\isasymrightarrow}\ F{\isadigit{2}}{\isacharparenright}{\isacharparenright}{\isachardoublequoteclose}\isanewline
%
\isadelimproof
%
\endisadelimproof
%
\isatagproof
\isacommand{proof}\isamarkupfalse%
\ {\isacharminus}\isanewline
\ \ \isacommand{have}\isamarkupfalse%
\ {\isachardoublequoteopen}finite\ {\isacharparenleft}atoms\ F{\isadigit{1}}\ {\isasymunion}\ atoms\ F{\isadigit{2}}{\isacharparenright}{\isachardoublequoteclose}\isanewline
\ \ \ \ \isacommand{using}\isamarkupfalse%
\ assms\isanewline
\ \ \ \ \isacommand{by}\isamarkupfalse%
\ {\isacharparenleft}simp\ only{\isacharcolon}\ finite{\isacharunderscore}UnI{\isacharparenright}\isanewline
\ \ \isacommand{then}\isamarkupfalse%
\ \isacommand{show}\isamarkupfalse%
\ {\isachardoublequoteopen}finite\ {\isacharparenleft}atoms\ {\isacharparenleft}F{\isadigit{1}}\ \isactrlbold {\isasymrightarrow}\ F{\isadigit{2}}{\isacharparenright}{\isacharparenright}{\isachardoublequoteclose}\ \ \isanewline
\ \ \ \ \isacommand{by}\isamarkupfalse%
\ {\isacharparenleft}simp\ only{\isacharcolon}\ formula{\isachardot}set{\isacharparenleft}{\isadigit{6}}{\isacharparenright}{\isacharparenright}\isanewline
\isacommand{qed}\isamarkupfalse%
%
\endisatagproof
{\isafoldproof}%
%
\isadelimproof
\isanewline
%
\endisadelimproof
\isanewline
\isacommand{lemma}\isamarkupfalse%
\ atoms{\isacharunderscore}finite{\isacharcolon}\ {\isachardoublequoteopen}finite\ {\isacharparenleft}atoms\ F{\isacharparenright}{\isachardoublequoteclose}\isanewline
%
\isadelimproof
%
\endisadelimproof
%
\isatagproof
\isacommand{proof}\isamarkupfalse%
\ {\isacharparenleft}induction\ F{\isacharparenright}\isanewline
\ \ \isacommand{case}\isamarkupfalse%
\ {\isacharparenleft}Atom\ x{\isacharparenright}\isanewline
\ \ \isacommand{then}\isamarkupfalse%
\ \isacommand{show}\isamarkupfalse%
\ {\isacharquery}case\ \isacommand{by}\isamarkupfalse%
\ {\isacharparenleft}simp\ only{\isacharcolon}\ atoms{\isacharunderscore}finite{\isacharunderscore}atom{\isacharparenright}\isanewline
\isacommand{next}\isamarkupfalse%
\isanewline
\ \ \isacommand{case}\isamarkupfalse%
\ Bot\isanewline
\ \ \isacommand{then}\isamarkupfalse%
\ \isacommand{show}\isamarkupfalse%
\ {\isacharquery}case\ \isacommand{by}\isamarkupfalse%
\ {\isacharparenleft}simp\ only{\isacharcolon}\ atoms{\isacharunderscore}finite{\isacharunderscore}bot{\isacharparenright}\isanewline
\isacommand{next}\isamarkupfalse%
\isanewline
\ \ \isacommand{case}\isamarkupfalse%
\ {\isacharparenleft}Not\ F{\isacharparenright}\isanewline
\ \ \isacommand{then}\isamarkupfalse%
\ \isacommand{show}\isamarkupfalse%
\ {\isacharquery}case\ \isacommand{by}\isamarkupfalse%
\ {\isacharparenleft}simp\ only{\isacharcolon}\ atoms{\isacharunderscore}finite{\isacharunderscore}not{\isacharparenright}\isanewline
\isacommand{next}\isamarkupfalse%
\isanewline
\ \ \isacommand{case}\isamarkupfalse%
\ {\isacharparenleft}And\ F{\isadigit{1}}\ F{\isadigit{2}}{\isacharparenright}\isanewline
\ \ \isacommand{then}\isamarkupfalse%
\ \isacommand{show}\isamarkupfalse%
\ {\isacharquery}case\ \isacommand{by}\isamarkupfalse%
\ {\isacharparenleft}simp\ only{\isacharcolon}\ atoms{\isacharunderscore}finite{\isacharunderscore}and{\isacharparenright}\isanewline
\isacommand{next}\isamarkupfalse%
\isanewline
\ \ \isacommand{case}\isamarkupfalse%
\ {\isacharparenleft}Or\ F{\isadigit{1}}\ F{\isadigit{2}}{\isacharparenright}\isanewline
\ \ \isacommand{then}\isamarkupfalse%
\ \isacommand{show}\isamarkupfalse%
\ {\isacharquery}case\ \isacommand{by}\isamarkupfalse%
\ {\isacharparenleft}simp\ only{\isacharcolon}\ atoms{\isacharunderscore}finite{\isacharunderscore}or{\isacharparenright}\isanewline
\isacommand{next}\isamarkupfalse%
\isanewline
\ \ \isacommand{case}\isamarkupfalse%
\ {\isacharparenleft}Imp\ F{\isadigit{1}}\ F{\isadigit{2}}{\isacharparenright}\isanewline
\ \ \isacommand{then}\isamarkupfalse%
\ \isacommand{show}\isamarkupfalse%
\ {\isacharquery}case\ \isacommand{by}\isamarkupfalse%
\ {\isacharparenleft}simp\ only{\isacharcolon}\ atoms{\isacharunderscore}finite{\isacharunderscore}imp{\isacharparenright}\isanewline
\isacommand{qed}\isamarkupfalse%
%
\endisatagproof
{\isafoldproof}%
%
\isadelimproof
%
\endisadelimproof
%
\begin{isamarkuptext}%
Su demostración automática es la siguiente.%
\end{isamarkuptext}\isamarkuptrue%
\isacommand{lemma}\isamarkupfalse%
\ {\isachardoublequoteopen}finite\ {\isacharparenleft}atoms\ F{\isacharparenright}{\isachardoublequoteclose}\ \isanewline
%
\isadelimproof
\ \ %
\endisadelimproof
%
\isatagproof
\isacommand{by}\isamarkupfalse%
\ {\isacharparenleft}induction\ F{\isacharparenright}\ simp{\isacharunderscore}all%
\endisatagproof
{\isafoldproof}%
%
\isadelimproof
%
\endisadelimproof
%
\isadelimdocument
%
\endisadelimdocument
%
\isatagdocument
%
\isamarkupsection{Subfórmulas%
}
\isamarkuptrue%
%
\endisatagdocument
{\isafolddocument}%
%
\isadelimdocument
%
\endisadelimdocument
%
\begin{isamarkuptext}%
Veamos la noción de subfórmulas.

  \begin{definicion}
  El conjunto de subfórmulas de una fórmula \isa{F}, notada \isa{Subf{\isacharparenleft}F{\isacharparenright}}, se 
  define recursivamente como:
    \begin{itemize}
      \item \isa{{\isacharbraceleft}F{\isacharbraceright}} si \isa{F} es una fórmula atómica.
      \item \isa{{\isacharbraceleft}{\isasymbottom}{\isacharbraceright}} si \isa{F} es \isa{{\isasymbottom}}.
      \item \isa{{\isacharbraceleft}F{\isacharbraceright}\ {\isasymunion}\ Subf{\isacharparenleft}G{\isacharparenright}} si \isa{F} es \isa{{\isasymnot}G}.
      \item \isa{{\isacharbraceleft}F{\isacharbraceright}\ {\isasymunion}\ Subf{\isacharparenleft}G{\isacharparenright}\ {\isasymunion}\ Subf{\isacharparenleft}H{\isacharparenright}} si \isa{F} es \isa{G{\isacharasterisk}H} donde \isa{{\isacharasterisk}} es 
        cualquier conectiva binaria.
    \end{itemize}
  \end{definicion}

  Para proceder a la formalización de Isabelle, seguiremos dos etapas. 
  En primer lugar, definimos la función primitiva recursiva 
  \isa{subformulae}. Esta nos devolverá la lista de todas las 
  subfórmulas de una fórmula original obtenidas recursivamente.%
\end{isamarkuptext}\isamarkuptrue%
\isacommand{primrec}\isamarkupfalse%
\ subformulae\ {\isacharcolon}{\isacharcolon}\ {\isachardoublequoteopen}{\isacharprime}a\ formula\ {\isasymRightarrow}\ {\isacharprime}a\ formula\ list{\isachardoublequoteclose}\ \isakeyword{where}\isanewline
\ \ {\isachardoublequoteopen}subformulae\ {\isacharparenleft}Atom\ k{\isacharparenright}\ {\isacharequal}\ {\isacharbrackleft}Atom\ k{\isacharbrackright}{\isachardoublequoteclose}\ \isanewline
{\isacharbar}\ {\isachardoublequoteopen}subformulae\ {\isasymbottom}\ \ \ \ \ \ \ \ {\isacharequal}\ {\isacharbrackleft}{\isasymbottom}{\isacharbrackright}{\isachardoublequoteclose}\ \isanewline
{\isacharbar}\ {\isachardoublequoteopen}subformulae\ {\isacharparenleft}\isactrlbold {\isasymnot}\ F{\isacharparenright}\ \ \ \ {\isacharequal}\ {\isacharparenleft}\isactrlbold {\isasymnot}\ F{\isacharparenright}\ {\isacharhash}\ subformulae\ F{\isachardoublequoteclose}\ \isanewline
{\isacharbar}\ {\isachardoublequoteopen}subformulae\ {\isacharparenleft}F\ \isactrlbold {\isasymand}\ G{\isacharparenright}\ \ {\isacharequal}\ {\isacharparenleft}F\ \isactrlbold {\isasymand}\ G{\isacharparenright}\ {\isacharhash}\ subformulae\ F\ {\isacharat}\ subformulae\ G{\isachardoublequoteclose}\ \isanewline
{\isacharbar}\ {\isachardoublequoteopen}subformulae\ {\isacharparenleft}F\ \isactrlbold {\isasymor}\ G{\isacharparenright}\ \ {\isacharequal}\ {\isacharparenleft}F\ \isactrlbold {\isasymor}\ G{\isacharparenright}\ {\isacharhash}\ subformulae\ F\ {\isacharat}\ subformulae\ G{\isachardoublequoteclose}\isanewline
{\isacharbar}\ {\isachardoublequoteopen}subformulae\ {\isacharparenleft}F\ \isactrlbold {\isasymrightarrow}\ G{\isacharparenright}\ {\isacharequal}\ {\isacharparenleft}F\ \isactrlbold {\isasymrightarrow}\ G{\isacharparenright}\ {\isacharhash}\ subformulae\ F\ {\isacharat}\ subformulae\ G{\isachardoublequoteclose}%
\begin{isamarkuptext}%
Observemos que, en la definición anterior, \isa{{\isacharhash}} es el operador que 
  añade un elemento al comienzo de una lista y \isa{{\isacharat}} concatena varias 
  listas. Siguiendo con los ejemplos, apliquemos \isa{subformulae} en 
  las distintas fórmulas. En particular, al tratarse de una lista pueden 
  aparecer elementos repetidos como se muestra a continuación.%
\end{isamarkuptext}\isamarkuptrue%
\isacommand{notepad}\isamarkupfalse%
\isanewline
\isakeyword{begin}\isanewline
%
\isadelimproof
\ \ %
\endisadelimproof
%
\isatagproof
\isacommand{fix}\isamarkupfalse%
\ p\ {\isacharcolon}{\isacharcolon}\ {\isacharprime}a\isanewline
\isanewline
\ \ \isacommand{have}\isamarkupfalse%
\ {\isachardoublequoteopen}subformulae\ {\isacharparenleft}Atom\ p{\isacharparenright}\ {\isacharequal}\ {\isacharbrackleft}Atom\ p{\isacharbrackright}{\isachardoublequoteclose}\isanewline
\ \ \ \ \isacommand{by}\isamarkupfalse%
\ simp\isanewline
\isanewline
\ \ \isacommand{have}\isamarkupfalse%
\ {\isachardoublequoteopen}subformulae\ {\isacharparenleft}\isactrlbold {\isasymnot}\ {\isacharparenleft}Atom\ p{\isacharparenright}{\isacharparenright}\ {\isacharequal}\ {\isacharbrackleft}\isactrlbold {\isasymnot}\ {\isacharparenleft}Atom\ p{\isacharparenright}{\isacharcomma}\ Atom\ p{\isacharbrackright}{\isachardoublequoteclose}\isanewline
\ \ \ \ \isacommand{by}\isamarkupfalse%
\ simp\isanewline
\isanewline
\ \ \isacommand{have}\isamarkupfalse%
\ {\isachardoublequoteopen}subformulae\ {\isacharparenleft}{\isacharparenleft}Atom\ p\ \isactrlbold {\isasymrightarrow}\ Atom\ q{\isacharparenright}\ \isactrlbold {\isasymor}\ Atom\ r{\isacharparenright}\ {\isacharequal}\ \isanewline
\ \ \ \ \ \ \ {\isacharbrackleft}{\isacharparenleft}Atom\ p\ \isactrlbold {\isasymrightarrow}\ Atom\ q{\isacharparenright}\ \isactrlbold {\isasymor}\ Atom\ r{\isacharcomma}\ Atom\ p\ \isactrlbold {\isasymrightarrow}\ Atom\ q{\isacharcomma}\ Atom\ p{\isacharcomma}\ \isanewline
\ \ \ \ \ \ \ \ Atom\ q{\isacharcomma}\ Atom\ r{\isacharbrackright}{\isachardoublequoteclose}\isanewline
\ \ \ \ \isacommand{by}\isamarkupfalse%
\ simp\isanewline
\isanewline
\ \ \isacommand{have}\isamarkupfalse%
\ {\isachardoublequoteopen}subformulae\ {\isacharparenleft}Atom\ p\ \isactrlbold {\isasymand}\ {\isasymbottom}{\isacharparenright}\ {\isacharequal}\ {\isacharbrackleft}Atom\ p\ \isactrlbold {\isasymand}\ {\isasymbottom}{\isacharcomma}\ Atom\ p{\isacharcomma}\ {\isasymbottom}{\isacharbrackright}{\isachardoublequoteclose}\isanewline
\ \ \ \ \isacommand{by}\isamarkupfalse%
\ simp\isanewline
\isanewline
\ \ \isacommand{have}\isamarkupfalse%
\ {\isachardoublequoteopen}subformulae\ {\isacharparenleft}Atom\ p\ \isactrlbold {\isasymor}\ Atom\ p{\isacharparenright}\ {\isacharequal}\ \isanewline
\ \ \ \ \ \ \ {\isacharbrackleft}Atom\ p\ \isactrlbold {\isasymor}\ Atom\ p{\isacharcomma}\ Atom\ p{\isacharcomma}\ Atom\ p{\isacharbrackright}{\isachardoublequoteclose}\isanewline
\ \ \ \ \isacommand{by}\isamarkupfalse%
\ simp%
\endisatagproof
{\isafoldproof}%
%
\isadelimproof
\isanewline
%
\endisadelimproof
\isacommand{end}\isamarkupfalse%
%
\begin{isamarkuptext}%
En la segunda etapa de formalización, definimos 
  \isa{setSubformulae}, que convierte al tipo conjunto la lista de 
  subfórmulas anterior.%
\end{isamarkuptext}\isamarkuptrue%
\isacommand{abbreviation}\isamarkupfalse%
\ setSubformulae\ {\isacharcolon}{\isacharcolon}\ {\isachardoublequoteopen}{\isacharprime}a\ formula\ {\isasymRightarrow}\ {\isacharprime}a\ formula\ set{\isachardoublequoteclose}\ \isakeyword{where}\isanewline
\ \ {\isachardoublequoteopen}setSubformulae\ F\ {\isasymequiv}\ set\ {\isacharparenleft}subformulae\ F{\isacharparenright}{\isachardoublequoteclose}%
\begin{isamarkuptext}%
De este modo, la función \isa{setSubformulae} es la formalización
  en Isabelle de \isa{Subf{\isacharparenleft}·{\isacharparenright}}. En Isabelle, primero hemos definido la lista 
  de subfórmulas pues, en algunos casos, es más sencilla la prueba de 
  resultados sobre este tipo. Sin embargo, el tipo de conjuntos facilita
  las pruebas de los resultados de esta sección. Algunas de las
  ventajas del tipo conjuntos son la eliminación de elementos repetidos 
  o las operaciones propias de teoría de conjuntos. Observemos los 
  siguientes ejemplos con el tipo de conjuntos.%
\end{isamarkuptext}\isamarkuptrue%
\isacommand{notepad}\isamarkupfalse%
\isanewline
\isakeyword{begin}\isanewline
%
\isadelimproof
\ \ %
\endisadelimproof
%
\isatagproof
\isacommand{fix}\isamarkupfalse%
\ p\ q\ r\ {\isacharcolon}{\isacharcolon}\ {\isacharprime}a\isanewline
\isanewline
\ \ \isacommand{have}\isamarkupfalse%
\ {\isachardoublequoteopen}setSubformulae\ {\isacharparenleft}Atom\ p\ \isactrlbold {\isasymor}\ Atom\ p{\isacharparenright}\ {\isacharequal}\ {\isacharbraceleft}Atom\ p\ \isactrlbold {\isasymor}\ Atom\ p{\isacharcomma}\ Atom\ p{\isacharbraceright}{\isachardoublequoteclose}\isanewline
\ \ \ \ \isacommand{by}\isamarkupfalse%
\ simp\isanewline
\ \ \isanewline
\ \ \isacommand{have}\isamarkupfalse%
\ {\isachardoublequoteopen}setSubformulae\ {\isacharparenleft}{\isacharparenleft}Atom\ p\ \isactrlbold {\isasymrightarrow}\ Atom\ q{\isacharparenright}\ \isactrlbold {\isasymor}\ Atom\ r{\isacharparenright}\ {\isacharequal}\isanewline
\ \ \ \ \ \ \ \ {\isacharbraceleft}{\isacharparenleft}Atom\ p\ \isactrlbold {\isasymrightarrow}\ Atom\ q{\isacharparenright}\ \isactrlbold {\isasymor}\ Atom\ r{\isacharcomma}\ Atom\ p\ \isactrlbold {\isasymrightarrow}\ Atom\ q{\isacharcomma}\ Atom\ p{\isacharcomma}\ \isanewline
\ \ \ \ \ \ \ \ \ Atom\ q{\isacharcomma}\ Atom\ r{\isacharbraceright}{\isachardoublequoteclose}\isanewline
\ \ \isacommand{by}\isamarkupfalse%
\ auto%
\endisatagproof
{\isafoldproof}%
%
\isadelimproof
\ \ \ \isanewline
%
\endisadelimproof
\isacommand{end}\isamarkupfalse%
%
\begin{isamarkuptext}%
Por otro lado, debemos señalar que el uso de 
  \isa{abbreviation} para definir \isa{setSubformulae} no es 
  arbitrario. Esta elección se debe a que el tipo \isa{abbreviation} 
  se trata de un sinónimo para una expresión cuyo tipo ya existe (en 
  nuestro caso, convertir en conjunto la lista obtenida con 
  \isa{subformulae}). No es una definición propiamente dicha, sino 
  una forma de nombrar la composición de las funciones \isa{set} y 
  \isa{subformulae}.

  En primer lugar, veamos que \isa{setSubformulae} es una
  formalización de \isa{Subf} en Isabelle. Para ello 
  utilizaremos el siguiente resultado sobre listas, probado como sigue.%
\end{isamarkuptext}\isamarkuptrue%
\isacommand{lemma}\isamarkupfalse%
\ set{\isacharunderscore}insert{\isacharcolon}\ {\isachardoublequoteopen}set\ {\isacharparenleft}x\ {\isacharhash}\ ys{\isacharparenright}\ {\isacharequal}\ {\isacharbraceleft}x{\isacharbraceright}\ {\isasymunion}\ set\ ys{\isachardoublequoteclose}\isanewline
%
\isadelimproof
\ \ %
\endisadelimproof
%
\isatagproof
\isacommand{by}\isamarkupfalse%
\ {\isacharparenleft}simp\ only{\isacharcolon}\ list{\isachardot}set{\isacharparenleft}{\isadigit{2}}{\isacharparenright}\ Un{\isacharunderscore}insert{\isacharunderscore}left\ sup{\isacharunderscore}bot{\isachardot}left{\isacharunderscore}neutral{\isacharparenright}%
\endisatagproof
{\isafoldproof}%
%
\isadelimproof
%
\endisadelimproof
%
\begin{isamarkuptext}%
Por tanto, obtenemos la equivalencia como resultado de los 
  siguientes lemas, que aparecen demostrados de manera detallada.%
\end{isamarkuptext}\isamarkuptrue%
\isacommand{lemma}\isamarkupfalse%
\ setSubformulae{\isacharunderscore}atom{\isacharcolon}\isanewline
\ \ {\isachardoublequoteopen}setSubformulae\ {\isacharparenleft}Atom\ p{\isacharparenright}\ {\isacharequal}\ {\isacharbraceleft}Atom\ p{\isacharbraceright}{\isachardoublequoteclose}\isanewline
%
\isadelimproof
\ \ \ \ %
\endisadelimproof
%
\isatagproof
\isacommand{by}\isamarkupfalse%
\ {\isacharparenleft}simp\ only{\isacharcolon}\ subformulae{\isachardot}simps{\isacharparenleft}{\isadigit{1}}{\isacharparenright}\ list{\isachardot}set{\isacharparenright}%
\endisatagproof
{\isafoldproof}%
%
\isadelimproof
\isanewline
%
\endisadelimproof
\isanewline
\isacommand{lemma}\isamarkupfalse%
\ setSubformulae{\isacharunderscore}bot{\isacharcolon}\isanewline
\ \ {\isachardoublequoteopen}setSubformulae\ {\isacharparenleft}{\isasymbottom}{\isacharparenright}\ {\isacharequal}\ {\isacharbraceleft}{\isasymbottom}{\isacharbraceright}{\isachardoublequoteclose}\isanewline
%
\isadelimproof
\ \ \ \ %
\endisadelimproof
%
\isatagproof
\isacommand{by}\isamarkupfalse%
\ {\isacharparenleft}simp\ only{\isacharcolon}\ subformulae{\isachardot}simps{\isacharparenleft}{\isadigit{2}}{\isacharparenright}\ list{\isachardot}set{\isacharparenright}%
\endisatagproof
{\isafoldproof}%
%
\isadelimproof
\isanewline
%
\endisadelimproof
\isanewline
\isacommand{lemma}\isamarkupfalse%
\ setSubformulae{\isacharunderscore}not{\isacharcolon}\isanewline
\ \ \isakeyword{shows}\ {\isachardoublequoteopen}setSubformulae\ {\isacharparenleft}\isactrlbold {\isasymnot}\ F{\isacharparenright}\ {\isacharequal}\ {\isacharbraceleft}\isactrlbold {\isasymnot}\ F{\isacharbraceright}\ {\isasymunion}\ setSubformulae\ F{\isachardoublequoteclose}\isanewline
%
\isadelimproof
%
\endisadelimproof
%
\isatagproof
\isacommand{proof}\isamarkupfalse%
\ {\isacharminus}\isanewline
\ \ \isacommand{have}\isamarkupfalse%
\ {\isachardoublequoteopen}setSubformulae\ {\isacharparenleft}\isactrlbold {\isasymnot}\ F{\isacharparenright}\ {\isacharequal}\ set\ {\isacharparenleft}\isactrlbold {\isasymnot}\ F\ {\isacharhash}\ subformulae\ F{\isacharparenright}{\isachardoublequoteclose}\isanewline
\ \ \ \ \isacommand{by}\isamarkupfalse%
\ {\isacharparenleft}simp\ only{\isacharcolon}\ subformulae{\isachardot}simps{\isacharparenleft}{\isadigit{3}}{\isacharparenright}{\isacharparenright}\isanewline
\ \ \isacommand{also}\isamarkupfalse%
\ \isacommand{have}\isamarkupfalse%
\ {\isachardoublequoteopen}{\isasymdots}\ {\isacharequal}\ {\isacharbraceleft}\isactrlbold {\isasymnot}\ F{\isacharbraceright}\ {\isasymunion}\ set\ {\isacharparenleft}subformulae\ F{\isacharparenright}{\isachardoublequoteclose}\isanewline
\ \ \ \ \isacommand{by}\isamarkupfalse%
\ {\isacharparenleft}simp\ only{\isacharcolon}\ set{\isacharunderscore}insert{\isacharparenright}\isanewline
\ \ \isacommand{finally}\isamarkupfalse%
\ \isacommand{show}\isamarkupfalse%
\ {\isacharquery}thesis\isanewline
\ \ \ \ \isacommand{by}\isamarkupfalse%
\ this\isanewline
\isacommand{qed}\isamarkupfalse%
%
\endisatagproof
{\isafoldproof}%
%
\isadelimproof
\isanewline
%
\endisadelimproof
\isanewline
\isacommand{lemma}\isamarkupfalse%
\ setSubformulae{\isacharunderscore}and{\isacharcolon}\ \isanewline
\ \ {\isachardoublequoteopen}setSubformulae\ {\isacharparenleft}F{\isadigit{1}}\ \isactrlbold {\isasymand}\ F{\isadigit{2}}{\isacharparenright}\ \isanewline
\ \ \ {\isacharequal}\ {\isacharbraceleft}F{\isadigit{1}}\ \isactrlbold {\isasymand}\ F{\isadigit{2}}{\isacharbraceright}\ {\isasymunion}\ {\isacharparenleft}setSubformulae\ F{\isadigit{1}}\ {\isasymunion}\ setSubformulae\ F{\isadigit{2}}{\isacharparenright}{\isachardoublequoteclose}\isanewline
%
\isadelimproof
%
\endisadelimproof
%
\isatagproof
\isacommand{proof}\isamarkupfalse%
\ {\isacharminus}\isanewline
\ \ \isacommand{have}\isamarkupfalse%
\ {\isachardoublequoteopen}setSubformulae\ {\isacharparenleft}F{\isadigit{1}}\ \isactrlbold {\isasymand}\ F{\isadigit{2}}{\isacharparenright}\ \isanewline
\ \ \ \ \ \ \ \ {\isacharequal}\ set\ {\isacharparenleft}{\isacharparenleft}F{\isadigit{1}}\ \isactrlbold {\isasymand}\ F{\isadigit{2}}{\isacharparenright}\ {\isacharhash}\ {\isacharparenleft}subformulae\ F{\isadigit{1}}\ {\isacharat}\ subformulae\ F{\isadigit{2}}{\isacharparenright}{\isacharparenright}{\isachardoublequoteclose}\isanewline
\ \ \ \ \isacommand{by}\isamarkupfalse%
\ {\isacharparenleft}simp\ only{\isacharcolon}\ subformulae{\isachardot}simps{\isacharparenleft}{\isadigit{4}}{\isacharparenright}{\isacharparenright}\isanewline
\ \ \isacommand{also}\isamarkupfalse%
\ \isacommand{have}\isamarkupfalse%
\ {\isachardoublequoteopen}{\isasymdots}\ {\isacharequal}\ {\isacharbraceleft}F{\isadigit{1}}\ \isactrlbold {\isasymand}\ F{\isadigit{2}}{\isacharbraceright}\ {\isasymunion}\ {\isacharparenleft}set\ {\isacharparenleft}subformulae\ F{\isadigit{1}}\ {\isacharat}\ subformulae\ F{\isadigit{2}}{\isacharparenright}{\isacharparenright}{\isachardoublequoteclose}\isanewline
\ \ \ \ \isacommand{by}\isamarkupfalse%
\ {\isacharparenleft}simp\ only{\isacharcolon}\ set{\isacharunderscore}insert{\isacharparenright}\isanewline
\ \ \isacommand{also}\isamarkupfalse%
\ \isacommand{have}\isamarkupfalse%
\ {\isachardoublequoteopen}{\isasymdots}\ {\isacharequal}\ {\isacharbraceleft}F{\isadigit{1}}\ \isactrlbold {\isasymand}\ F{\isadigit{2}}{\isacharbraceright}\ {\isasymunion}\ {\isacharparenleft}setSubformulae\ F{\isadigit{1}}\ {\isasymunion}\ setSubformulae\ F{\isadigit{2}}{\isacharparenright}{\isachardoublequoteclose}\isanewline
\ \ \ \ \isacommand{by}\isamarkupfalse%
\ {\isacharparenleft}simp\ only{\isacharcolon}\ set{\isacharunderscore}append{\isacharparenright}\isanewline
\ \ \isacommand{finally}\isamarkupfalse%
\ \isacommand{show}\isamarkupfalse%
\ {\isacharquery}thesis\isanewline
\ \ \ \ \isacommand{by}\isamarkupfalse%
\ this\isanewline
\isacommand{qed}\isamarkupfalse%
%
\endisatagproof
{\isafoldproof}%
%
\isadelimproof
\isanewline
%
\endisadelimproof
\isanewline
\isacommand{lemma}\isamarkupfalse%
\ setSubformulae{\isacharunderscore}or{\isacharcolon}\ \isanewline
\ \ {\isachardoublequoteopen}setSubformulae\ {\isacharparenleft}F{\isadigit{1}}\ \isactrlbold {\isasymor}\ F{\isadigit{2}}{\isacharparenright}\ \isanewline
\ \ \ {\isacharequal}\ {\isacharbraceleft}F{\isadigit{1}}\ \isactrlbold {\isasymor}\ F{\isadigit{2}}{\isacharbraceright}\ {\isasymunion}\ {\isacharparenleft}setSubformulae\ F{\isadigit{1}}\ {\isasymunion}\ setSubformulae\ F{\isadigit{2}}{\isacharparenright}{\isachardoublequoteclose}\isanewline
%
\isadelimproof
%
\endisadelimproof
%
\isatagproof
\isacommand{proof}\isamarkupfalse%
\ {\isacharminus}\isanewline
\ \ \isacommand{have}\isamarkupfalse%
\ {\isachardoublequoteopen}setSubformulae\ {\isacharparenleft}F{\isadigit{1}}\ \isactrlbold {\isasymor}\ F{\isadigit{2}}{\isacharparenright}\ \isanewline
\ \ \ \ \ \ \ \ {\isacharequal}\ set\ {\isacharparenleft}{\isacharparenleft}F{\isadigit{1}}\ \isactrlbold {\isasymor}\ F{\isadigit{2}}{\isacharparenright}\ {\isacharhash}\ {\isacharparenleft}subformulae\ F{\isadigit{1}}\ {\isacharat}\ subformulae\ F{\isadigit{2}}{\isacharparenright}{\isacharparenright}{\isachardoublequoteclose}\isanewline
\ \ \ \ \isacommand{by}\isamarkupfalse%
\ {\isacharparenleft}simp\ only{\isacharcolon}\ subformulae{\isachardot}simps{\isacharparenleft}{\isadigit{5}}{\isacharparenright}{\isacharparenright}\isanewline
\ \ \isacommand{also}\isamarkupfalse%
\ \isacommand{have}\isamarkupfalse%
\ {\isachardoublequoteopen}{\isasymdots}\ {\isacharequal}\ {\isacharbraceleft}F{\isadigit{1}}\ \isactrlbold {\isasymor}\ F{\isadigit{2}}{\isacharbraceright}\ {\isasymunion}\ {\isacharparenleft}set\ {\isacharparenleft}subformulae\ F{\isadigit{1}}\ {\isacharat}\ subformulae\ F{\isadigit{2}}{\isacharparenright}{\isacharparenright}{\isachardoublequoteclose}\isanewline
\ \ \ \ \isacommand{by}\isamarkupfalse%
\ {\isacharparenleft}simp\ only{\isacharcolon}\ set{\isacharunderscore}insert{\isacharparenright}\isanewline
\ \ \isacommand{also}\isamarkupfalse%
\ \isacommand{have}\isamarkupfalse%
\ {\isachardoublequoteopen}{\isasymdots}\ {\isacharequal}\ {\isacharbraceleft}F{\isadigit{1}}\ \isactrlbold {\isasymor}\ F{\isadigit{2}}{\isacharbraceright}\ {\isasymunion}\ {\isacharparenleft}setSubformulae\ F{\isadigit{1}}\ {\isasymunion}\ setSubformulae\ F{\isadigit{2}}{\isacharparenright}{\isachardoublequoteclose}\isanewline
\ \ \ \ \isacommand{by}\isamarkupfalse%
\ {\isacharparenleft}simp\ only{\isacharcolon}\ set{\isacharunderscore}append{\isacharparenright}\isanewline
\ \ \isacommand{finally}\isamarkupfalse%
\ \isacommand{show}\isamarkupfalse%
\ {\isacharquery}thesis\isanewline
\ \ \ \ \isacommand{by}\isamarkupfalse%
\ this\isanewline
\isacommand{qed}\isamarkupfalse%
%
\endisatagproof
{\isafoldproof}%
%
\isadelimproof
\isanewline
%
\endisadelimproof
\isanewline
\isacommand{lemma}\isamarkupfalse%
\ setSubformulae{\isacharunderscore}imp{\isacharcolon}\ \isanewline
\ \ {\isachardoublequoteopen}setSubformulae\ {\isacharparenleft}F{\isadigit{1}}\ \isactrlbold {\isasymrightarrow}\ F{\isadigit{2}}{\isacharparenright}\ \isanewline
\ \ \ {\isacharequal}\ {\isacharbraceleft}F{\isadigit{1}}\ \isactrlbold {\isasymrightarrow}\ F{\isadigit{2}}{\isacharbraceright}\ {\isasymunion}\ {\isacharparenleft}setSubformulae\ F{\isadigit{1}}\ {\isasymunion}\ setSubformulae\ F{\isadigit{2}}{\isacharparenright}{\isachardoublequoteclose}\isanewline
%
\isadelimproof
%
\endisadelimproof
%
\isatagproof
\isacommand{proof}\isamarkupfalse%
\ {\isacharminus}\isanewline
\ \ \isacommand{have}\isamarkupfalse%
\ {\isachardoublequoteopen}setSubformulae\ {\isacharparenleft}F{\isadigit{1}}\ \isactrlbold {\isasymrightarrow}\ F{\isadigit{2}}{\isacharparenright}\ \isanewline
\ \ \ \ \ \ \ \ {\isacharequal}\ set\ {\isacharparenleft}{\isacharparenleft}F{\isadigit{1}}\ \isactrlbold {\isasymrightarrow}\ F{\isadigit{2}}{\isacharparenright}\ {\isacharhash}\ {\isacharparenleft}subformulae\ F{\isadigit{1}}\ {\isacharat}\ subformulae\ F{\isadigit{2}}{\isacharparenright}{\isacharparenright}{\isachardoublequoteclose}\isanewline
\ \ \ \ \isacommand{by}\isamarkupfalse%
\ {\isacharparenleft}simp\ only{\isacharcolon}\ subformulae{\isachardot}simps{\isacharparenleft}{\isadigit{6}}{\isacharparenright}{\isacharparenright}\isanewline
\ \ \isacommand{also}\isamarkupfalse%
\ \isacommand{have}\isamarkupfalse%
\ {\isachardoublequoteopen}{\isasymdots}\ {\isacharequal}\ {\isacharbraceleft}F{\isadigit{1}}\ \isactrlbold {\isasymrightarrow}\ F{\isadigit{2}}{\isacharbraceright}\ {\isasymunion}\ {\isacharparenleft}set\ {\isacharparenleft}subformulae\ F{\isadigit{1}}\ {\isacharat}\ subformulae\ F{\isadigit{2}}{\isacharparenright}{\isacharparenright}{\isachardoublequoteclose}\isanewline
\ \ \ \ \isacommand{by}\isamarkupfalse%
\ {\isacharparenleft}simp\ only{\isacharcolon}\ set{\isacharunderscore}insert{\isacharparenright}\isanewline
\ \ \isacommand{also}\isamarkupfalse%
\ \isacommand{have}\isamarkupfalse%
\ {\isachardoublequoteopen}{\isasymdots}\ {\isacharequal}\ {\isacharbraceleft}F{\isadigit{1}}\ \isactrlbold {\isasymrightarrow}\ F{\isadigit{2}}{\isacharbraceright}\ {\isasymunion}\ {\isacharparenleft}setSubformulae\ F{\isadigit{1}}\ {\isasymunion}\ setSubformulae\ F{\isadigit{2}}{\isacharparenright}{\isachardoublequoteclose}\isanewline
\ \ \ \ \isacommand{by}\isamarkupfalse%
\ {\isacharparenleft}simp\ only{\isacharcolon}\ set{\isacharunderscore}append{\isacharparenright}\isanewline
\ \ \isacommand{finally}\isamarkupfalse%
\ \isacommand{show}\isamarkupfalse%
\ {\isacharquery}thesis\isanewline
\ \ \ \ \isacommand{by}\isamarkupfalse%
\ this\isanewline
\isacommand{qed}\isamarkupfalse%
%
\endisatagproof
{\isafoldproof}%
%
\isadelimproof
%
\endisadelimproof
%
\begin{isamarkuptext}%
Una vez probada la equivalencia, comencemos con los resultados 
  correspondientes a las subfórmulas. En primer lugar, tenemos la 
  siguiente propiedad como consecuencia directa de la equivalencia de 
  funciones anterior.

  \comentario{Reescribir el siguiente enunciado y demostración.}

  \begin{lema}
    \isa{F\ {\isasymin}\ Subf{\isacharparenleft}F{\isacharparenright}}.
  \end{lema}

  \begin{demostracion}
    Por inducción en la estructura de las fórmulas. Se tienen los
    siguientes casos:
  
    Sea \isa{p} fórmula atómica cualquiera. Por definición de \isa{Subf} tenemos 
    que \isa{Subf{\isacharparenleft}p{\isacharparenright}\ {\isacharequal}\ {\isacharbraceleft}p{\isacharbraceright}}, luego se tiene la propiedad.
  
    Sea la fórmula \isa{{\isasymbottom}}. Como \isa{Subf{\isacharparenleft}{\isasymbottom}{\isacharparenright}\ {\isacharequal}\ {\isacharbraceleft}{\isasymbottom}{\isacharbraceright}}, se verifica el resultado.

    Por definición del conjunto de subfórmulas de \isa{Subf{\isacharparenleft}{\isasymnot}\ F{\isacharparenright}} se tiene 
    la propiedad para este caso, pues 
    \isa{Subf{\isacharparenleft}{\isasymnot}\ F{\isacharparenright}\ {\isacharequal}\ {\isacharbraceleft}{\isasymnot}\ F{\isacharbraceright}\ {\isasymunion}\ Subf{\isacharparenleft}F{\isacharparenright}\ {\isasymLongrightarrow}\ {\isasymnot}\ F\ {\isasymin}\ Subf{\isacharparenleft}{\isasymnot}\ F{\isacharparenright}} como queríamos 
    ver.

    Análogamente, para cualquier conectiva binaria \isa{{\isacharasterisk}} y fórmulas \isa{F} y 
    \isa{G} se cumple \isa{Subf{\isacharparenleft}F{\isacharasterisk}G{\isacharparenright}\ {\isacharequal}\ {\isacharbraceleft}F{\isacharasterisk}G{\isacharbraceright}\ {\isasymunion}\ Subf{\isacharparenleft}F{\isacharparenright}\ {\isasymunion}\ Subf{\isacharparenleft}G{\isacharparenright}}, luego se 
    cumple la propiedad.
  \end{demostracion}

  Formalicemos ahora el lema con su correspondiente demostración 
  detallada.%
\end{isamarkuptext}\isamarkuptrue%
\isacommand{lemma}\isamarkupfalse%
\ subformulae{\isacharunderscore}self{\isacharcolon}\ {\isachardoublequoteopen}F\ {\isasymin}\ setSubformulae\ F{\isachardoublequoteclose}\isanewline
%
\isadelimproof
%
\endisadelimproof
%
\isatagproof
\isacommand{proof}\isamarkupfalse%
\ {\isacharparenleft}induction\ F{\isacharparenright}\ \isanewline
\ \ \isacommand{case}\isamarkupfalse%
\ {\isacharparenleft}Atom\ x{\isacharparenright}\ \isanewline
\ \ \isacommand{then}\isamarkupfalse%
\ \isacommand{show}\isamarkupfalse%
\ {\isacharquery}case\ \isanewline
\ \ \ \ \isacommand{by}\isamarkupfalse%
\ {\isacharparenleft}simp\ only{\isacharcolon}\ singletonI\ setSubformulae{\isacharunderscore}atom{\isacharparenright}\isanewline
\isacommand{next}\isamarkupfalse%
\isanewline
\ \ \isacommand{case}\isamarkupfalse%
\ Bot\isanewline
\ \ \isacommand{then}\isamarkupfalse%
\ \isacommand{show}\isamarkupfalse%
\ {\isacharquery}case\ \isanewline
\ \ \ \ \isacommand{by}\isamarkupfalse%
\ {\isacharparenleft}simp\ only{\isacharcolon}\ singletonI\ setSubformulae{\isacharunderscore}bot{\isacharparenright}\isanewline
\isacommand{next}\isamarkupfalse%
\isanewline
\ \ \isacommand{case}\isamarkupfalse%
\ {\isacharparenleft}Not\ F{\isacharparenright}\isanewline
\ \ \isacommand{then}\isamarkupfalse%
\ \isacommand{show}\isamarkupfalse%
\ {\isacharquery}case\ \isanewline
\ \ \ \ \isacommand{by}\isamarkupfalse%
\ {\isacharparenleft}simp\ add{\isacharcolon}\ insertI{\isadigit{1}}\ setSubformulae{\isacharunderscore}not{\isacharparenright}\isanewline
\isacommand{next}\isamarkupfalse%
\isanewline
\isacommand{case}\isamarkupfalse%
\ {\isacharparenleft}And\ F{\isadigit{1}}\ F{\isadigit{2}}{\isacharparenright}\isanewline
\ \ \isacommand{then}\isamarkupfalse%
\ \isacommand{show}\isamarkupfalse%
\ {\isacharquery}case\ \isanewline
\ \ \ \ \isacommand{by}\isamarkupfalse%
\ {\isacharparenleft}simp\ add{\isacharcolon}\ insertI{\isadigit{1}}\ setSubformulae{\isacharunderscore}and{\isacharparenright}\isanewline
\isacommand{next}\isamarkupfalse%
\isanewline
\isacommand{case}\isamarkupfalse%
\ {\isacharparenleft}Or\ F{\isadigit{1}}\ F{\isadigit{2}}{\isacharparenright}\isanewline
\ \ \isacommand{then}\isamarkupfalse%
\ \isacommand{show}\isamarkupfalse%
\ {\isacharquery}case\ \isanewline
\ \ \ \ \isacommand{by}\isamarkupfalse%
\ {\isacharparenleft}simp\ add{\isacharcolon}\ insertI{\isadigit{1}}\ setSubformulae{\isacharunderscore}or{\isacharparenright}\isanewline
\isacommand{next}\isamarkupfalse%
\isanewline
\isacommand{case}\isamarkupfalse%
\ {\isacharparenleft}Imp\ F{\isadigit{1}}\ F{\isadigit{2}}{\isacharparenright}\isanewline
\ \ \isacommand{then}\isamarkupfalse%
\ \isacommand{show}\isamarkupfalse%
\ {\isacharquery}case\ \isanewline
\ \ \ \ \isacommand{by}\isamarkupfalse%
\ {\isacharparenleft}simp\ add{\isacharcolon}\ insertI{\isadigit{1}}\ setSubformulae{\isacharunderscore}imp{\isacharparenright}\isanewline
\isacommand{qed}\isamarkupfalse%
%
\endisatagproof
{\isafoldproof}%
%
\isadelimproof
%
\endisadelimproof
%
\begin{isamarkuptext}%
La demostración automática es la siguiente.%
\end{isamarkuptext}\isamarkuptrue%
\isacommand{lemma}\isamarkupfalse%
\ {\isachardoublequoteopen}F\ {\isasymin}\ setSubformulae\ F{\isachardoublequoteclose}\isanewline
%
\isadelimproof
\ \ %
\endisadelimproof
%
\isatagproof
\isacommand{by}\isamarkupfalse%
\ {\isacharparenleft}induction\ F{\isacharparenright}\ simp{\isacharunderscore}all%
\endisatagproof
{\isafoldproof}%
%
\isadelimproof
%
\endisadelimproof
%
\begin{isamarkuptext}%
Procedamos con los demás resultados de la sección. Como hemos 
  señalado con anterioridad, utilizaremos varias propiedades de 
  conjuntos pertenecientes a la teoría 
  \href{https://n9.cl/qatp}{Set.thy} de Isabelle, que apareceran en 
  el glosario final. 

  Además, definiremos dos reglas adicionales que utilizaremos con 
  frecuencia.%
\end{isamarkuptext}\isamarkuptrue%
\isacommand{lemma}\isamarkupfalse%
\ subContUnionRev{\isadigit{1}}{\isacharcolon}\ \isanewline
\ \ \isakeyword{assumes}\ {\isachardoublequoteopen}A\ {\isasymunion}\ B\ {\isasymsubseteq}\ C{\isachardoublequoteclose}\ \isanewline
\ \ \isakeyword{shows}\ \ \ {\isachardoublequoteopen}A\ {\isasymsubseteq}\ C{\isachardoublequoteclose}\isanewline
%
\isadelimproof
%
\endisadelimproof
%
\isatagproof
\isacommand{proof}\isamarkupfalse%
\ {\isacharminus}\isanewline
\ \ \isacommand{have}\isamarkupfalse%
\ {\isachardoublequoteopen}A\ {\isasymsubseteq}\ C\ {\isasymand}\ B\ {\isasymsubseteq}\ C{\isachardoublequoteclose}\isanewline
\ \ \ \ \isacommand{using}\isamarkupfalse%
\ assms\isanewline
\ \ \ \ \isacommand{by}\isamarkupfalse%
\ {\isacharparenleft}simp\ only{\isacharcolon}\ sup{\isachardot}bounded{\isacharunderscore}iff{\isacharparenright}\isanewline
\ \ \isacommand{then}\isamarkupfalse%
\ \isacommand{show}\isamarkupfalse%
\ {\isachardoublequoteopen}A\ {\isasymsubseteq}\ C{\isachardoublequoteclose}\isanewline
\ \ \ \ \isacommand{by}\isamarkupfalse%
\ {\isacharparenleft}rule\ conjunct{\isadigit{1}}{\isacharparenright}\isanewline
\isacommand{qed}\isamarkupfalse%
%
\endisatagproof
{\isafoldproof}%
%
\isadelimproof
\isanewline
%
\endisadelimproof
\isanewline
\isacommand{lemma}\isamarkupfalse%
\ subContUnionRev{\isadigit{2}}{\isacharcolon}\ \isanewline
\ \ \isakeyword{assumes}\ {\isachardoublequoteopen}A\ {\isasymunion}\ B\ {\isasymsubseteq}\ C{\isachardoublequoteclose}\ \isanewline
\ \ \isakeyword{shows}\ \ \ {\isachardoublequoteopen}B\ {\isasymsubseteq}\ C{\isachardoublequoteclose}\isanewline
%
\isadelimproof
%
\endisadelimproof
%
\isatagproof
\isacommand{proof}\isamarkupfalse%
\ {\isacharminus}\isanewline
\ \ \isacommand{have}\isamarkupfalse%
\ {\isachardoublequoteopen}A\ {\isasymsubseteq}\ C\ {\isasymand}\ B\ {\isasymsubseteq}\ C{\isachardoublequoteclose}\isanewline
\ \ \ \ \isacommand{using}\isamarkupfalse%
\ assms\isanewline
\ \ \ \ \isacommand{by}\isamarkupfalse%
\ {\isacharparenleft}simp\ only{\isacharcolon}\ sup{\isachardot}bounded{\isacharunderscore}iff{\isacharparenright}\isanewline
\ \ \isacommand{then}\isamarkupfalse%
\ \isacommand{show}\isamarkupfalse%
\ {\isachardoublequoteopen}B\ {\isasymsubseteq}\ C{\isachardoublequoteclose}\isanewline
\ \ \ \ \isacommand{by}\isamarkupfalse%
\ {\isacharparenleft}rule\ conjunct{\isadigit{2}}{\isacharparenright}\isanewline
\isacommand{qed}\isamarkupfalse%
%
\endisatagproof
{\isafoldproof}%
%
\isadelimproof
%
\endisadelimproof
%
\begin{isamarkuptext}%
Sus correspondientes demostraciones automáticas se muestran a 
  continuación.%
\end{isamarkuptext}\isamarkuptrue%
\isacommand{lemma}\isamarkupfalse%
\ {\isachardoublequoteopen}A\ {\isasymunion}\ B\ {\isasymsubseteq}\ C\ {\isasymLongrightarrow}\ A\ {\isasymsubseteq}\ C{\isachardoublequoteclose}\isanewline
%
\isadelimproof
\ \ %
\endisadelimproof
%
\isatagproof
\isacommand{by}\isamarkupfalse%
\ simp%
\endisatagproof
{\isafoldproof}%
%
\isadelimproof
\isanewline
%
\endisadelimproof
\isanewline
\isacommand{lemma}\isamarkupfalse%
\ {\isachardoublequoteopen}A\ {\isasymunion}\ B\ {\isasymsubseteq}\ C\ {\isasymLongrightarrow}\ B\ {\isasymsubseteq}\ C{\isachardoublequoteclose}\isanewline
%
\isadelimproof
\ \ %
\endisadelimproof
%
\isatagproof
\isacommand{by}\isamarkupfalse%
\ simp%
\endisatagproof
{\isafoldproof}%
%
\isadelimproof
%
\endisadelimproof
%
\begin{isamarkuptext}%
Veamos ahora los distintos resultados sobre subfórmulas.

  \comentario{Reescribir el siguiente enunciado y su demostración.}

  \begin{lema}
    Sean \isa{F} una fórmula proposicional y \isa{A\isactrlsub F} el conjunto de las 
    fórmulas atómicas formadas a partir de cada elemento del conjunto 
    de variables proposicionales de \isa{F}. 
    Entonces, \isa{A\isactrlsub F\ {\isasymsubseteq}\ Subf{\isacharparenleft}F{\isacharparenright}}.

    Por tanto, las fórmulas atómicas son subfórmulas.
  \end{lema}

  \begin{demostracion}
    La prueba seguirá el esquema inductivo para la estructura de 
    fórmulas. Veamos cada caso:
  
    Consideremos la fórmula atómica \isa{p} cualquiera. Entonces, su
    conjunto de átomos es \isa{{\isacharbraceleft}p{\isacharbraceright}}. De este modo, el conjunto \isa{A\isactrlsub p} 
    correspondiente será \isa{A\isactrlsub p\ {\isacharequal}\ {\isacharbraceleft}p{\isacharbraceright}\ {\isasymsubseteq}\ {\isacharbraceleft}p{\isacharbraceright}\ {\isacharequal}\ Subf{\isacharparenleft}Atom\ p{\isacharparenright}} como 
    queríamos 
    demostrar.

    Sea la fórmula \isa{{\isasymbottom}}. Como su connjunto de átomos es vacío, es claro 
    que \isa{A\isactrlsub {\isasymbottom}\ {\isacharequal}\ {\isasymemptyset}\ {\isasymsubseteq}\ Subf{\isacharparenleft}{\isasymbottom}{\isacharparenright}\ {\isacharequal}\ {\isasymemptyset}}.

    Sea la fórmula \isa{F} tal que \isa{A\isactrlsub F\ {\isasymsubseteq}\ Subf{\isacharparenleft}F{\isacharparenright}}. Probemos el resultado 
    para \isa{{\isasymnot}\ F}. Por definición tenemos que los conjunto de variables 
    proposicionales de \isa{F} y \isa{{\isasymnot}\ F} coinciden, luego \isa{A\isactrlsub {\isasymnot}\isactrlsub F\ {\isacharequal}\ A\isactrlsub F}. Además, 
    \isa{Subf{\isacharparenleft}{\isasymnot}\ F{\isacharparenright}\ {\isacharequal}\ {\isacharbraceleft}{\isasymnot}\ F{\isacharbraceright}\ {\isasymunion}\ Subf{\isacharparenleft}F{\isacharparenright}}. Por tanto, por hipótesis de 
    inducción tenemos:
    \isa{A\isactrlsub {\isasymnot}\isactrlsub F\ {\isacharequal}\ A\isactrlsub F\ {\isasymsubseteq}\ Subf{\isacharparenleft}F{\isacharparenright}\ {\isasymsubseteq}\ {\isacharbraceleft}{\isasymnot}\ F{\isacharbraceright}\ {\isasymunion}\ Subf{\isacharparenleft}F{\isacharparenright}\ {\isacharequal}\ Subf{\isacharparenleft}{\isasymnot}\ F{\isacharparenright}}, luego
    \isa{A\isactrlsub {\isasymnot}\isactrlsub F\ {\isasymsubseteq}\ Subf{\isacharparenleft}{\isasymnot}\ F{\isacharparenright}}.

    Sean las fórmulas \isa{F} y \isa{G} tales que \isa{A\isactrlsub F\ {\isasymsubseteq}\ Subf{\isacharparenleft}F{\isacharparenright}} y 
    \isa{A\isactrlsub G\ {\isasymsubseteq}\ Subf{\isacharparenleft}G{\isacharparenright}}. Probemos ahora \isa{A\isactrlsub F\isactrlsub {\isacharasterisk}\isactrlsub G\ {\isasymsubseteq}\ Subf{\isacharparenleft}F{\isacharasterisk}G{\isacharparenright}} para cualquier 
    conectiva binaria \isa{{\isacharasterisk}}. Por un lado, el conjunto de átomos de \isa{F{\isacharasterisk}G}
    es la unión de sus correspondientes conjuntos de átomos, luego 
    \isa{A\isactrlsub F\isactrlsub {\isacharasterisk}\isactrlsub G\ {\isacharequal}\ A\isactrlsub F\ {\isasymunion}\ A\isactrlsub G}. Por tanto, por hipótesis de inducción y definición 
    del conjunto de subfórmulas, se tiene:
    \isa{A\isactrlsub F\isactrlsub {\isacharasterisk}\isactrlsub G\ {\isacharequal}\ A\isactrlsub F\ {\isasymunion}\ A\isactrlsub G\ {\isasymsubseteq}\ Subf{\isacharparenleft}F{\isacharparenright}\ {\isasymunion}\ Subf{\isacharparenleft}G{\isacharparenright}\ {\isasymsubseteq}\ {\isacharbraceleft}F{\isacharasterisk}G{\isacharbraceright}\ {\isasymunion}\ Subf{\isacharparenleft}F{\isacharparenright}\ {\isasymunion}\ Subf{\isacharparenleft}G{\isacharparenright}\ {\isacharequal}\ Subf{\isacharparenleft}F{\isacharasterisk}G{\isacharparenright}}
    Luego, \isa{A\isactrlsub F\isactrlsub {\isacharasterisk}\isactrlsub G\ {\isasymsubseteq}\ Subf{\isacharparenleft}F{\isacharasterisk}G{\isacharparenright}} como queríamos demostrar.  
  \end{demostracion}

  En Isabelle, se especifica como sigue.%
\end{isamarkuptext}\isamarkuptrue%
\isacommand{lemma}\isamarkupfalse%
\ {\isachardoublequoteopen}Atom\ {\isacharbackquote}\ atoms\ F\ {\isasymsubseteq}\ setSubformulae\ F{\isachardoublequoteclose}\isanewline
%
\isadelimproof
\ \ %
\endisadelimproof
%
\isatagproof
\isacommand{oops}\isamarkupfalse%
%
\endisatagproof
{\isafoldproof}%
%
\isadelimproof
%
\endisadelimproof
%
\begin{isamarkuptext}%
Debemos observar que \isa{Atom\ {\isacharbackquote}\ atoms\ F} construye las fórmulas 
  atómicas a partir de cada uno de los elementos de \isa{atoms\ F}, creando 
  un conjunto de fórmulas atómicas. Dicho conjunto es equivalente al 
  conjunto \isa{A\isactrlsub F} del enunciado del lema. Para ello emplea el infijo \isa{{\isacharbackquote}} 
  definido como notación abreviada de \isa{{\isacharparenleft}{\isacharbackquote}{\isacharparenright}} que calcula la 
  imagen de un conjunto en la teoría \href{https://n9.cl/qatp}{Set.thy}.

  \begin{itemize}
    \item[] \isa{f\ {\isacharbackquote}\ A\ {\isacharequal}\ {\isacharbraceleft}y\ {\isacharbar}\ {\isasymexists}x{\isasymin}A{\isachardot}\ y\ {\isacharequal}\ f\ x{\isacharbraceright}} 
      \hfill (\isa{image{\isacharunderscore}def})
  \end{itemize}

  Para aclarar su funcionamiento, veamos ejemplos para distintos casos 
  de fórmulas.%
\end{isamarkuptext}\isamarkuptrue%
\isacommand{notepad}\isamarkupfalse%
\isanewline
\isakeyword{begin}\isanewline
%
\isadelimproof
\ \ %
\endisadelimproof
%
\isatagproof
\isacommand{fix}\isamarkupfalse%
\ p\ q\ r\ {\isacharcolon}{\isacharcolon}\ {\isacharprime}a\isanewline
\isanewline
\ \ \isacommand{have}\isamarkupfalse%
\ {\isachardoublequoteopen}Atom\ {\isacharbackquote}atoms\ {\isacharparenleft}Atom\ p\ \isactrlbold {\isasymor}\ {\isasymbottom}{\isacharparenright}\ {\isacharequal}\ {\isacharbraceleft}Atom\ p{\isacharbraceright}{\isachardoublequoteclose}\isanewline
\ \ \ \ \isacommand{by}\isamarkupfalse%
\ simp\isanewline
\isanewline
\ \ \isacommand{have}\isamarkupfalse%
\ {\isachardoublequoteopen}Atom\ {\isacharbackquote}atoms\ {\isacharparenleft}{\isacharparenleft}Atom\ p\ \isactrlbold {\isasymrightarrow}\ Atom\ q{\isacharparenright}\ \isactrlbold {\isasymor}\ Atom\ r{\isacharparenright}\ {\isacharequal}\ \isanewline
\ \ \ \ \ \ \ {\isacharbraceleft}Atom\ p{\isacharcomma}\ Atom\ q{\isacharcomma}\ Atom\ r{\isacharbraceright}{\isachardoublequoteclose}\isanewline
\ \ \ \ \isacommand{by}\isamarkupfalse%
\ auto\ \isanewline
\isanewline
\ \ \isacommand{have}\isamarkupfalse%
\ {\isachardoublequoteopen}Atom\ {\isacharbackquote}atoms\ {\isacharparenleft}{\isacharparenleft}Atom\ p\ \isactrlbold {\isasymrightarrow}\ Atom\ p{\isacharparenright}\ \isactrlbold {\isasymor}\ Atom\ r{\isacharparenright}\ {\isacharequal}\ \isanewline
\ \ \ \ \ \ \ {\isacharbraceleft}Atom\ p{\isacharcomma}\ Atom\ r{\isacharbraceright}{\isachardoublequoteclose}\isanewline
\ \ \ \ \isacommand{by}\isamarkupfalse%
\ auto%
\endisatagproof
{\isafoldproof}%
%
\isadelimproof
\isanewline
%
\endisadelimproof
\isacommand{end}\isamarkupfalse%
%
\begin{isamarkuptext}%
Además, esta función tiene las siguientes propiedades sobre 
  conjuntos que utilizaremos en la demostración.

  \begin{itemize}
    \item[] \isa{f\ {\isacharbackquote}\ {\isacharparenleft}A\ {\isasymunion}\ B{\isacharparenright}\ {\isacharequal}\ f\ {\isacharbackquote}\ A\ {\isasymunion}\ f\ {\isacharbackquote}\ B} 
      \hfill (\isa{image{\isacharunderscore}Un})
    \item[] \isa{f\ {\isacharbackquote}\ {\isacharparenleft}{\isacharbraceleft}a{\isacharbraceright}\ {\isasymunion}\ B{\isacharparenright}\ {\isacharequal}\ {\isacharbraceleft}f\ a{\isacharbraceright}\ {\isasymunion}\ f\ {\isacharbackquote}\ B} 
      \hfill (\isa{image{\isacharunderscore}insert})
    \item[] \isa{f\ {\isacharbackquote}\ {\isasymemptyset}\ {\isacharequal}\ {\isasymemptyset}} 
      \hfill (\isa{image{\isacharunderscore}empty})
  \end{itemize}

  Una vez hechas las aclaraciones necesarias, comencemos la demostración 
  estructurada. Esta seguirá el esquema inductivo señalado con 
  anterioridad.%
\end{isamarkuptext}\isamarkuptrue%
\isacommand{lemma}\isamarkupfalse%
\ atoms{\isacharunderscore}are{\isacharunderscore}subformulae{\isacharunderscore}atom{\isacharcolon}\ \isanewline
\ \ {\isachardoublequoteopen}Atom\ {\isacharbackquote}\ atoms\ {\isacharparenleft}Atom\ x{\isacharparenright}\ {\isasymsubseteq}\ setSubformulae\ {\isacharparenleft}Atom\ x{\isacharparenright}{\isachardoublequoteclose}\ \isanewline
%
\isadelimproof
%
\endisadelimproof
%
\isatagproof
\isacommand{proof}\isamarkupfalse%
\ {\isacharminus}\isanewline
\ \ \isacommand{have}\isamarkupfalse%
\ {\isachardoublequoteopen}Atom\ {\isacharbackquote}\ atoms\ {\isacharparenleft}Atom\ x{\isacharparenright}\ {\isacharequal}\ Atom\ {\isacharbackquote}\ {\isacharbraceleft}x{\isacharbraceright}{\isachardoublequoteclose}\isanewline
\ \ \ \ \isacommand{by}\isamarkupfalse%
\ {\isacharparenleft}simp\ only{\isacharcolon}\ formula{\isachardot}set{\isacharparenleft}{\isadigit{1}}{\isacharparenright}{\isacharparenright}\isanewline
\ \ \isacommand{also}\isamarkupfalse%
\ \isacommand{have}\isamarkupfalse%
\ {\isachardoublequoteopen}{\isasymdots}\ {\isacharequal}\ {\isacharbraceleft}Atom\ x{\isacharbraceright}{\isachardoublequoteclose}\isanewline
\ \ \ \ \isacommand{by}\isamarkupfalse%
\ {\isacharparenleft}simp\ only{\isacharcolon}\ image{\isacharunderscore}insert\ image{\isacharunderscore}empty{\isacharparenright}\isanewline
\ \ \isacommand{also}\isamarkupfalse%
\ \isacommand{have}\isamarkupfalse%
\ {\isachardoublequoteopen}{\isasymdots}\ {\isacharequal}\ set\ {\isacharbrackleft}Atom\ x{\isacharbrackright}{\isachardoublequoteclose}\isanewline
\ \ \ \ \isacommand{by}\isamarkupfalse%
\ {\isacharparenleft}simp\ only{\isacharcolon}\ list{\isachardot}set{\isacharparenleft}{\isadigit{1}}{\isacharparenright}\ list{\isachardot}set{\isacharparenleft}{\isadigit{2}}{\isacharparenright}{\isacharparenright}\isanewline
\ \ \isacommand{also}\isamarkupfalse%
\ \isacommand{have}\isamarkupfalse%
\ {\isachardoublequoteopen}{\isasymdots}\ {\isacharequal}\ set\ {\isacharparenleft}subformulae\ {\isacharparenleft}Atom\ x{\isacharparenright}{\isacharparenright}{\isachardoublequoteclose}\isanewline
\ \ \ \ \isacommand{by}\isamarkupfalse%
\ {\isacharparenleft}simp\ only{\isacharcolon}\ subformulae{\isachardot}simps{\isacharparenleft}{\isadigit{1}}{\isacharparenright}{\isacharparenright}\isanewline
\ \ \isacommand{finally}\isamarkupfalse%
\ \isacommand{have}\isamarkupfalse%
\ {\isachardoublequoteopen}Atom\ {\isacharbackquote}\ atoms\ {\isacharparenleft}Atom\ x{\isacharparenright}\ {\isacharequal}\ set\ {\isacharparenleft}subformulae\ {\isacharparenleft}Atom\ x{\isacharparenright}{\isacharparenright}{\isachardoublequoteclose}\isanewline
\ \ \ \ \isacommand{by}\isamarkupfalse%
\ this\isanewline
\ \ \isacommand{then}\isamarkupfalse%
\ \isacommand{show}\isamarkupfalse%
\ {\isacharquery}thesis\ \isanewline
\ \ \ \ \isacommand{by}\isamarkupfalse%
\ {\isacharparenleft}simp\ only{\isacharcolon}\ subset{\isacharunderscore}refl{\isacharparenright}\isanewline
\isacommand{qed}\isamarkupfalse%
%
\endisatagproof
{\isafoldproof}%
%
\isadelimproof
\isanewline
%
\endisadelimproof
\isanewline
\isacommand{lemma}\isamarkupfalse%
\ atoms{\isacharunderscore}are{\isacharunderscore}subformulae{\isacharunderscore}bot{\isacharcolon}\ \isanewline
\ \ {\isachardoublequoteopen}Atom\ {\isacharbackquote}\ atoms\ {\isasymbottom}\ {\isasymsubseteq}\ setSubformulae\ {\isasymbottom}{\isachardoublequoteclose}\ \ \isanewline
%
\isadelimproof
%
\endisadelimproof
%
\isatagproof
\isacommand{proof}\isamarkupfalse%
\ {\isacharminus}\isanewline
\ \ \isacommand{have}\isamarkupfalse%
\ {\isachardoublequoteopen}Atom\ {\isacharbackquote}\ atoms\ {\isasymbottom}\ {\isacharequal}\ Atom\ {\isacharbackquote}\ {\isasymemptyset}{\isachardoublequoteclose}\isanewline
\ \ \ \ \isacommand{by}\isamarkupfalse%
\ {\isacharparenleft}simp\ only{\isacharcolon}\ formula{\isachardot}set{\isacharparenleft}{\isadigit{2}}{\isacharparenright}{\isacharparenright}\isanewline
\ \ \isacommand{also}\isamarkupfalse%
\ \isacommand{have}\isamarkupfalse%
\ {\isachardoublequoteopen}{\isasymdots}\ {\isacharequal}\ {\isasymemptyset}{\isachardoublequoteclose}\isanewline
\ \ \ \ \isacommand{by}\isamarkupfalse%
\ {\isacharparenleft}simp\ only{\isacharcolon}\ image{\isacharunderscore}empty{\isacharparenright}\isanewline
\ \ \isacommand{also}\isamarkupfalse%
\ \isacommand{have}\isamarkupfalse%
\ {\isachardoublequoteopen}{\isasymdots}\ {\isasymsubseteq}\ setSubformulae\ {\isasymbottom}{\isachardoublequoteclose}\isanewline
\ \ \ \ \isacommand{by}\isamarkupfalse%
\ {\isacharparenleft}simp\ only{\isacharcolon}\ empty{\isacharunderscore}subsetI{\isacharparenright}\isanewline
\ \ \isacommand{finally}\isamarkupfalse%
\ \isacommand{show}\isamarkupfalse%
\ {\isacharquery}thesis\isanewline
\ \ \ \ \isacommand{by}\isamarkupfalse%
\ this\isanewline
\isacommand{qed}\isamarkupfalse%
%
\endisatagproof
{\isafoldproof}%
%
\isadelimproof
\isanewline
%
\endisadelimproof
\isanewline
\isacommand{lemma}\isamarkupfalse%
\ atoms{\isacharunderscore}are{\isacharunderscore}subformulae{\isacharunderscore}not{\isacharcolon}\ \isanewline
\ \ \isakeyword{assumes}\ {\isachardoublequoteopen}Atom\ {\isacharbackquote}\ atoms\ F\ {\isasymsubseteq}\ setSubformulae\ F{\isachardoublequoteclose}\ \isanewline
\ \ \isakeyword{shows}\ \ \ {\isachardoublequoteopen}Atom\ {\isacharbackquote}\ atoms\ {\isacharparenleft}\isactrlbold {\isasymnot}\ F{\isacharparenright}\ {\isasymsubseteq}\ setSubformulae\ {\isacharparenleft}\isactrlbold {\isasymnot}\ F{\isacharparenright}{\isachardoublequoteclose}\isanewline
%
\isadelimproof
%
\endisadelimproof
%
\isatagproof
\isacommand{proof}\isamarkupfalse%
\ {\isacharminus}\isanewline
\ \ \isacommand{have}\isamarkupfalse%
\ {\isachardoublequoteopen}Atom\ {\isacharbackquote}\ atoms\ {\isacharparenleft}\isactrlbold {\isasymnot}\ F{\isacharparenright}\ {\isacharequal}\ Atom\ {\isacharbackquote}\ atoms\ F{\isachardoublequoteclose}\isanewline
\ \ \ \ \isacommand{by}\isamarkupfalse%
\ {\isacharparenleft}simp\ only{\isacharcolon}\ formula{\isachardot}set{\isacharparenleft}{\isadigit{3}}{\isacharparenright}{\isacharparenright}\isanewline
\ \ \isacommand{also}\isamarkupfalse%
\ \isacommand{have}\isamarkupfalse%
\ {\isachardoublequoteopen}{\isasymdots}\ {\isasymsubseteq}\ setSubformulae\ F{\isachardoublequoteclose}\isanewline
\ \ \ \ \isacommand{by}\isamarkupfalse%
\ {\isacharparenleft}simp\ only{\isacharcolon}\ assms{\isacharparenright}\isanewline
\ \ \isacommand{also}\isamarkupfalse%
\ \isacommand{have}\isamarkupfalse%
\ {\isachardoublequoteopen}{\isasymdots}\ {\isasymsubseteq}\ {\isacharbraceleft}\isactrlbold {\isasymnot}\ F{\isacharbraceright}\ {\isasymunion}\ setSubformulae\ F{\isachardoublequoteclose}\isanewline
\ \ \ \ \isacommand{by}\isamarkupfalse%
\ {\isacharparenleft}simp\ only{\isacharcolon}\ Un{\isacharunderscore}upper{\isadigit{2}}{\isacharparenright}\isanewline
\ \ \isacommand{also}\isamarkupfalse%
\ \isacommand{have}\isamarkupfalse%
\ {\isachardoublequoteopen}{\isasymdots}\ {\isacharequal}\ setSubformulae\ {\isacharparenleft}\isactrlbold {\isasymnot}\ F{\isacharparenright}{\isachardoublequoteclose}\isanewline
\ \ \ \ \isacommand{by}\isamarkupfalse%
\ {\isacharparenleft}simp\ only{\isacharcolon}\ setSubformulae{\isacharunderscore}not{\isacharparenright}\isanewline
\ \ \isacommand{finally}\isamarkupfalse%
\ \isacommand{show}\isamarkupfalse%
\ {\isacharquery}thesis\isanewline
\ \ \ \ \isacommand{by}\isamarkupfalse%
\ this\isanewline
\isacommand{qed}\isamarkupfalse%
%
\endisatagproof
{\isafoldproof}%
%
\isadelimproof
\isanewline
%
\endisadelimproof
\isanewline
\isacommand{lemma}\isamarkupfalse%
\ atoms{\isacharunderscore}are{\isacharunderscore}subformulae{\isacharunderscore}and{\isacharcolon}\ \isanewline
\ \ \isakeyword{assumes}\ {\isachardoublequoteopen}Atom\ {\isacharbackquote}\ atoms\ F{\isadigit{1}}\ {\isasymsubseteq}\ setSubformulae\ F{\isadigit{1}}{\isachardoublequoteclose}\isanewline
\ \ \ \ \ \ \ \ \ \ {\isachardoublequoteopen}Atom\ {\isacharbackquote}\ atoms\ F{\isadigit{2}}\ {\isasymsubseteq}\ setSubformulae\ F{\isadigit{2}}{\isachardoublequoteclose}\isanewline
\ \ \isakeyword{shows}\ \ \ {\isachardoublequoteopen}Atom\ {\isacharbackquote}\ atoms\ {\isacharparenleft}F{\isadigit{1}}\ \isactrlbold {\isasymand}\ F{\isadigit{2}}{\isacharparenright}\ {\isasymsubseteq}\ setSubformulae\ {\isacharparenleft}F{\isadigit{1}}\ \isactrlbold {\isasymand}\ F{\isadigit{2}}{\isacharparenright}{\isachardoublequoteclose}\isanewline
%
\isadelimproof
%
\endisadelimproof
%
\isatagproof
\isacommand{proof}\isamarkupfalse%
\ {\isacharminus}\isanewline
\ \ \isacommand{have}\isamarkupfalse%
\ {\isachardoublequoteopen}Atom\ {\isacharbackquote}\ atoms\ {\isacharparenleft}F{\isadigit{1}}\ \isactrlbold {\isasymand}\ F{\isadigit{2}}{\isacharparenright}\ {\isacharequal}\ Atom\ {\isacharbackquote}\ {\isacharparenleft}atoms\ F{\isadigit{1}}\ {\isasymunion}\ atoms\ F{\isadigit{2}}{\isacharparenright}{\isachardoublequoteclose}\isanewline
\ \ \ \ \isacommand{by}\isamarkupfalse%
\ {\isacharparenleft}simp\ only{\isacharcolon}\ formula{\isachardot}set{\isacharparenleft}{\isadigit{4}}{\isacharparenright}{\isacharparenright}\isanewline
\ \ \isacommand{also}\isamarkupfalse%
\ \isacommand{have}\isamarkupfalse%
\ {\isachardoublequoteopen}{\isasymdots}\ {\isacharequal}\ Atom\ {\isacharbackquote}\ atoms\ F{\isadigit{1}}\ {\isasymunion}\ Atom\ {\isacharbackquote}\ atoms\ F{\isadigit{2}}{\isachardoublequoteclose}\ \isanewline
\ \ \ \ \isacommand{by}\isamarkupfalse%
\ {\isacharparenleft}rule\ image{\isacharunderscore}Un{\isacharparenright}\isanewline
\ \ \isacommand{also}\isamarkupfalse%
\ \isacommand{have}\isamarkupfalse%
\ {\isachardoublequoteopen}{\isasymdots}\ {\isasymsubseteq}\ setSubformulae\ F{\isadigit{1}}\ {\isasymunion}\ setSubformulae\ F{\isadigit{2}}{\isachardoublequoteclose}\isanewline
\ \ \ \ \isacommand{using}\isamarkupfalse%
\ assms\isanewline
\ \ \ \ \isacommand{by}\isamarkupfalse%
\ {\isacharparenleft}rule\ Un{\isacharunderscore}mono{\isacharparenright}\isanewline
\ \ \isacommand{also}\isamarkupfalse%
\ \isacommand{have}\isamarkupfalse%
\ {\isachardoublequoteopen}{\isasymdots}\ {\isasymsubseteq}\ {\isacharbraceleft}F{\isadigit{1}}\ \isactrlbold {\isasymand}\ F{\isadigit{2}}{\isacharbraceright}\ {\isasymunion}\ {\isacharparenleft}setSubformulae\ F{\isadigit{1}}\ {\isasymunion}\ setSubformulae\ F{\isadigit{2}}{\isacharparenright}{\isachardoublequoteclose}\isanewline
\ \ \ \ \isacommand{by}\isamarkupfalse%
\ {\isacharparenleft}simp\ only{\isacharcolon}\ Un{\isacharunderscore}upper{\isadigit{2}}{\isacharparenright}\isanewline
\ \ \isacommand{also}\isamarkupfalse%
\ \isacommand{have}\isamarkupfalse%
\ {\isachardoublequoteopen}{\isasymdots}\ {\isacharequal}\ setSubformulae\ {\isacharparenleft}F{\isadigit{1}}\ \isactrlbold {\isasymand}\ F{\isadigit{2}}{\isacharparenright}{\isachardoublequoteclose}\isanewline
\ \ \ \ \isacommand{by}\isamarkupfalse%
\ {\isacharparenleft}simp\ only{\isacharcolon}\ setSubformulae{\isacharunderscore}and{\isacharparenright}\isanewline
\ \ \isacommand{finally}\isamarkupfalse%
\ \isacommand{show}\isamarkupfalse%
\ {\isacharquery}thesis\isanewline
\ \ \ \ \isacommand{by}\isamarkupfalse%
\ this\isanewline
\isacommand{qed}\isamarkupfalse%
%
\endisatagproof
{\isafoldproof}%
%
\isadelimproof
\isanewline
%
\endisadelimproof
\isanewline
\isacommand{lemma}\isamarkupfalse%
\ atoms{\isacharunderscore}are{\isacharunderscore}subformulae{\isacharunderscore}or{\isacharcolon}\ \isanewline
\ \ \isakeyword{assumes}\ {\isachardoublequoteopen}Atom\ {\isacharbackquote}\ atoms\ F{\isadigit{1}}\ {\isasymsubseteq}\ setSubformulae\ F{\isadigit{1}}{\isachardoublequoteclose}\isanewline
\ \ \ \ \ \ \ \ \ \ {\isachardoublequoteopen}Atom\ {\isacharbackquote}\ atoms\ F{\isadigit{2}}\ {\isasymsubseteq}\ setSubformulae\ F{\isadigit{2}}{\isachardoublequoteclose}\isanewline
\ \ \isakeyword{shows}\ \ \ {\isachardoublequoteopen}Atom\ {\isacharbackquote}\ atoms\ {\isacharparenleft}F{\isadigit{1}}\ \isactrlbold {\isasymor}\ F{\isadigit{2}}{\isacharparenright}\ {\isasymsubseteq}\ setSubformulae\ {\isacharparenleft}F{\isadigit{1}}\ \isactrlbold {\isasymor}\ F{\isadigit{2}}{\isacharparenright}{\isachardoublequoteclose}\isanewline
%
\isadelimproof
%
\endisadelimproof
%
\isatagproof
\isacommand{proof}\isamarkupfalse%
\ {\isacharminus}\isanewline
\ \ \isacommand{have}\isamarkupfalse%
\ {\isachardoublequoteopen}Atom\ {\isacharbackquote}\ atoms\ {\isacharparenleft}F{\isadigit{1}}\ \isactrlbold {\isasymor}\ F{\isadigit{2}}{\isacharparenright}\ {\isacharequal}\ Atom\ {\isacharbackquote}\ {\isacharparenleft}atoms\ F{\isadigit{1}}\ {\isasymunion}\ atoms\ F{\isadigit{2}}{\isacharparenright}{\isachardoublequoteclose}\isanewline
\ \ \ \ \isacommand{by}\isamarkupfalse%
\ {\isacharparenleft}simp\ only{\isacharcolon}\ formula{\isachardot}set{\isacharparenleft}{\isadigit{5}}{\isacharparenright}{\isacharparenright}\isanewline
\ \ \isacommand{also}\isamarkupfalse%
\ \isacommand{have}\isamarkupfalse%
\ {\isachardoublequoteopen}{\isasymdots}\ {\isacharequal}\ Atom\ {\isacharbackquote}\ atoms\ F{\isadigit{1}}\ {\isasymunion}\ Atom\ {\isacharbackquote}\ atoms\ F{\isadigit{2}}{\isachardoublequoteclose}\ \isanewline
\ \ \ \ \isacommand{by}\isamarkupfalse%
\ {\isacharparenleft}rule\ image{\isacharunderscore}Un{\isacharparenright}\isanewline
\ \ \isacommand{also}\isamarkupfalse%
\ \isacommand{have}\isamarkupfalse%
\ {\isachardoublequoteopen}{\isasymdots}\ {\isasymsubseteq}\ setSubformulae\ F{\isadigit{1}}\ {\isasymunion}\ setSubformulae\ F{\isadigit{2}}{\isachardoublequoteclose}\isanewline
\ \ \ \ \isacommand{using}\isamarkupfalse%
\ assms\isanewline
\ \ \ \ \isacommand{by}\isamarkupfalse%
\ {\isacharparenleft}rule\ Un{\isacharunderscore}mono{\isacharparenright}\isanewline
\ \ \isacommand{also}\isamarkupfalse%
\ \isacommand{have}\isamarkupfalse%
\ {\isachardoublequoteopen}{\isasymdots}\ {\isasymsubseteq}\ {\isacharbraceleft}F{\isadigit{1}}\ \isactrlbold {\isasymor}\ F{\isadigit{2}}{\isacharbraceright}\ {\isasymunion}\ {\isacharparenleft}setSubformulae\ F{\isadigit{1}}\ {\isasymunion}\ setSubformulae\ F{\isadigit{2}}{\isacharparenright}{\isachardoublequoteclose}\isanewline
\ \ \ \ \isacommand{by}\isamarkupfalse%
\ {\isacharparenleft}simp\ only{\isacharcolon}\ Un{\isacharunderscore}upper{\isadigit{2}}{\isacharparenright}\isanewline
\ \ \isacommand{also}\isamarkupfalse%
\ \isacommand{have}\isamarkupfalse%
\ {\isachardoublequoteopen}{\isasymdots}\ {\isacharequal}\ setSubformulae\ {\isacharparenleft}F{\isadigit{1}}\ \isactrlbold {\isasymor}\ F{\isadigit{2}}{\isacharparenright}{\isachardoublequoteclose}\isanewline
\ \ \ \ \isacommand{by}\isamarkupfalse%
\ {\isacharparenleft}simp\ only{\isacharcolon}\ setSubformulae{\isacharunderscore}or{\isacharparenright}\isanewline
\ \ \isacommand{finally}\isamarkupfalse%
\ \isacommand{show}\isamarkupfalse%
\ {\isacharquery}thesis\isanewline
\ \ \ \ \isacommand{by}\isamarkupfalse%
\ this\isanewline
\isacommand{qed}\isamarkupfalse%
%
\endisatagproof
{\isafoldproof}%
%
\isadelimproof
\isanewline
%
\endisadelimproof
\isanewline
\isacommand{lemma}\isamarkupfalse%
\ atoms{\isacharunderscore}are{\isacharunderscore}subformulae{\isacharunderscore}imp{\isacharcolon}\ \isanewline
\ \ \isakeyword{assumes}\ {\isachardoublequoteopen}Atom\ {\isacharbackquote}\ atoms\ F{\isadigit{1}}\ {\isasymsubseteq}\ setSubformulae\ F{\isadigit{1}}{\isachardoublequoteclose}\isanewline
\ \ \ \ \ \ \ \ \ \ {\isachardoublequoteopen}Atom\ {\isacharbackquote}\ atoms\ F{\isadigit{2}}\ {\isasymsubseteq}\ setSubformulae\ F{\isadigit{2}}{\isachardoublequoteclose}\isanewline
\ \ \isakeyword{shows}\ \ \ {\isachardoublequoteopen}Atom\ {\isacharbackquote}\ atoms\ {\isacharparenleft}F{\isadigit{1}}\ \isactrlbold {\isasymrightarrow}\ F{\isadigit{2}}{\isacharparenright}\ {\isasymsubseteq}\ setSubformulae\ {\isacharparenleft}F{\isadigit{1}}\ \isactrlbold {\isasymrightarrow}\ F{\isadigit{2}}{\isacharparenright}{\isachardoublequoteclose}\isanewline
%
\isadelimproof
%
\endisadelimproof
%
\isatagproof
\isacommand{proof}\isamarkupfalse%
\ {\isacharminus}\isanewline
\ \ \isacommand{have}\isamarkupfalse%
\ {\isachardoublequoteopen}Atom\ {\isacharbackquote}\ atoms\ {\isacharparenleft}F{\isadigit{1}}\ \isactrlbold {\isasymrightarrow}\ F{\isadigit{2}}{\isacharparenright}\ {\isacharequal}\ Atom\ {\isacharbackquote}\ {\isacharparenleft}atoms\ F{\isadigit{1}}\ {\isasymunion}\ atoms\ F{\isadigit{2}}{\isacharparenright}{\isachardoublequoteclose}\isanewline
\ \ \ \ \isacommand{by}\isamarkupfalse%
\ {\isacharparenleft}simp\ only{\isacharcolon}\ formula{\isachardot}set{\isacharparenleft}{\isadigit{6}}{\isacharparenright}{\isacharparenright}\isanewline
\ \ \isacommand{also}\isamarkupfalse%
\ \isacommand{have}\isamarkupfalse%
\ {\isachardoublequoteopen}{\isasymdots}\ {\isacharequal}\ Atom\ {\isacharbackquote}\ atoms\ F{\isadigit{1}}\ {\isasymunion}\ Atom\ {\isacharbackquote}\ atoms\ F{\isadigit{2}}{\isachardoublequoteclose}\ \isanewline
\ \ \ \ \isacommand{by}\isamarkupfalse%
\ {\isacharparenleft}rule\ image{\isacharunderscore}Un{\isacharparenright}\isanewline
\ \ \isacommand{also}\isamarkupfalse%
\ \isacommand{have}\isamarkupfalse%
\ {\isachardoublequoteopen}{\isasymdots}\ {\isasymsubseteq}\ setSubformulae\ F{\isadigit{1}}\ {\isasymunion}\ setSubformulae\ F{\isadigit{2}}{\isachardoublequoteclose}\isanewline
\ \ \ \ \isacommand{using}\isamarkupfalse%
\ assms\isanewline
\ \ \ \ \isacommand{by}\isamarkupfalse%
\ {\isacharparenleft}rule\ Un{\isacharunderscore}mono{\isacharparenright}\isanewline
\ \ \isacommand{also}\isamarkupfalse%
\ \isacommand{have}\isamarkupfalse%
\ {\isachardoublequoteopen}{\isasymdots}\ {\isasymsubseteq}\ {\isacharbraceleft}F{\isadigit{1}}\ \isactrlbold {\isasymrightarrow}\ F{\isadigit{2}}{\isacharbraceright}\ {\isasymunion}\ {\isacharparenleft}setSubformulae\ F{\isadigit{1}}\ {\isasymunion}\ setSubformulae\ F{\isadigit{2}}{\isacharparenright}{\isachardoublequoteclose}\isanewline
\ \ \ \ \isacommand{by}\isamarkupfalse%
\ {\isacharparenleft}simp\ only{\isacharcolon}\ Un{\isacharunderscore}upper{\isadigit{2}}{\isacharparenright}\isanewline
\ \ \isacommand{also}\isamarkupfalse%
\ \isacommand{have}\isamarkupfalse%
\ {\isachardoublequoteopen}{\isasymdots}\ {\isacharequal}\ setSubformulae\ {\isacharparenleft}F{\isadigit{1}}\ \isactrlbold {\isasymrightarrow}\ F{\isadigit{2}}{\isacharparenright}{\isachardoublequoteclose}\isanewline
\ \ \ \ \isacommand{by}\isamarkupfalse%
\ {\isacharparenleft}simp\ only{\isacharcolon}\ setSubformulae{\isacharunderscore}imp{\isacharparenright}\isanewline
\ \ \isacommand{finally}\isamarkupfalse%
\ \isacommand{show}\isamarkupfalse%
\ {\isacharquery}thesis\isanewline
\ \ \ \ \isacommand{by}\isamarkupfalse%
\ this\isanewline
\isacommand{qed}\isamarkupfalse%
%
\endisatagproof
{\isafoldproof}%
%
\isadelimproof
\isanewline
%
\endisadelimproof
\isanewline
\isacommand{lemma}\isamarkupfalse%
\ atoms{\isacharunderscore}are{\isacharunderscore}subformulae{\isacharcolon}\ \isanewline
\ \ {\isachardoublequoteopen}Atom\ {\isacharbackquote}\ atoms\ F\ {\isasymsubseteq}\ setSubformulae\ F{\isachardoublequoteclose}\isanewline
%
\isadelimproof
%
\endisadelimproof
%
\isatagproof
\isacommand{proof}\isamarkupfalse%
\ {\isacharparenleft}induction\ F{\isacharparenright}\isanewline
\ \ \isacommand{case}\isamarkupfalse%
\ {\isacharparenleft}Atom\ x{\isacharparenright}\isanewline
\ \ \isacommand{then}\isamarkupfalse%
\ \isacommand{show}\isamarkupfalse%
\ {\isacharquery}case\ \isacommand{by}\isamarkupfalse%
\ {\isacharparenleft}simp\ only{\isacharcolon}\ atoms{\isacharunderscore}are{\isacharunderscore}subformulae{\isacharunderscore}atom{\isacharparenright}\ \isanewline
\isacommand{next}\isamarkupfalse%
\isanewline
\ \ \isacommand{case}\isamarkupfalse%
\ Bot\isanewline
\ \ \isacommand{then}\isamarkupfalse%
\ \isacommand{show}\isamarkupfalse%
\ {\isacharquery}case\ \isacommand{by}\isamarkupfalse%
\ {\isacharparenleft}simp\ only{\isacharcolon}\ atoms{\isacharunderscore}are{\isacharunderscore}subformulae{\isacharunderscore}bot{\isacharparenright}\ \isanewline
\isacommand{next}\isamarkupfalse%
\isanewline
\ \ \isacommand{case}\isamarkupfalse%
\ {\isacharparenleft}Not\ F{\isacharparenright}\isanewline
\ \ \isacommand{then}\isamarkupfalse%
\ \isacommand{show}\isamarkupfalse%
\ {\isacharquery}case\ \isacommand{by}\isamarkupfalse%
\ {\isacharparenleft}simp\ only{\isacharcolon}\ atoms{\isacharunderscore}are{\isacharunderscore}subformulae{\isacharunderscore}not{\isacharparenright}\ \isanewline
\isacommand{next}\isamarkupfalse%
\isanewline
\ \ \isacommand{case}\isamarkupfalse%
\ {\isacharparenleft}And\ F{\isadigit{1}}\ F{\isadigit{2}}{\isacharparenright}\isanewline
\ \ \isacommand{then}\isamarkupfalse%
\ \isacommand{show}\isamarkupfalse%
\ {\isacharquery}case\ \isacommand{by}\isamarkupfalse%
\ {\isacharparenleft}simp\ only{\isacharcolon}\ atoms{\isacharunderscore}are{\isacharunderscore}subformulae{\isacharunderscore}and{\isacharparenright}\ \isanewline
\isacommand{next}\isamarkupfalse%
\isanewline
\ \ \isacommand{case}\isamarkupfalse%
\ {\isacharparenleft}Or\ F{\isadigit{1}}\ F{\isadigit{2}}{\isacharparenright}\isanewline
\ \ \isacommand{then}\isamarkupfalse%
\ \isacommand{show}\isamarkupfalse%
\ {\isacharquery}case\ \isacommand{by}\isamarkupfalse%
\ {\isacharparenleft}simp\ only{\isacharcolon}\ atoms{\isacharunderscore}are{\isacharunderscore}subformulae{\isacharunderscore}or{\isacharparenright}\isanewline
\isacommand{next}\isamarkupfalse%
\isanewline
\ \ \isacommand{case}\isamarkupfalse%
\ {\isacharparenleft}Imp\ F{\isadigit{1}}\ F{\isadigit{2}}{\isacharparenright}\isanewline
\ \ \isacommand{then}\isamarkupfalse%
\ \isacommand{show}\isamarkupfalse%
\ {\isacharquery}case\ \isacommand{by}\isamarkupfalse%
\ {\isacharparenleft}simp\ only{\isacharcolon}\ atoms{\isacharunderscore}are{\isacharunderscore}subformulae{\isacharunderscore}imp{\isacharparenright}\isanewline
\isacommand{qed}\isamarkupfalse%
%
\endisatagproof
{\isafoldproof}%
%
\isadelimproof
%
\endisadelimproof
%
\begin{isamarkuptext}%
La demostración automática queda igualmente expuesta a 
  continuación.%
\end{isamarkuptext}\isamarkuptrue%
\isacommand{lemma}\isamarkupfalse%
\ {\isachardoublequoteopen}Atom\ {\isacharbackquote}\ atoms\ F\ {\isasymsubseteq}\ setSubformulae\ F{\isachardoublequoteclose}\isanewline
%
\isadelimproof
\ \ %
\endisadelimproof
%
\isatagproof
\isacommand{by}\isamarkupfalse%
\ {\isacharparenleft}induction\ F{\isacharparenright}\ \ auto%
\endisatagproof
{\isafoldproof}%
%
\isadelimproof
%
\endisadelimproof
%
\begin{isamarkuptext}%
La siguiente propiedad declara que el conjunto de átomos de una 
  subfórmula está contenido en el conjunto de átomos de la propia 
  fórmula.
  \begin{lema}
    Sea \isa{G\ {\isasymin}\ Subf{\isacharparenleft}F{\isacharparenright}}, entonces el conjunto de átomos de \isa{G} está
    contenido en el de \isa{F}.
  \end{lema}

  \begin{demostracion}
  Procedemos mediante inducción en la estructura de las fórmulas según 
  los distintos casos:

  Sea \isa{p} una fórmula atómica cualquiera. Si \isa{G\ {\isasymin}\ Subf{\isacharparenleft}p{\isacharparenright}}, 
  como su conjunto de variables es \isa{{\isacharbraceleft}p{\isacharbraceright}}, se tiene \isa{G\ {\isacharequal}\ p}. 
  Por tanto, se tiene el resultado.

  Sea la fórmula \isa{{\isasymbottom}}. Si \isa{G\ {\isasymin}\ Subf{\isacharparenleft}{\isasymbottom}{\isacharparenright}}, como  su conjunto de átomos es
  \isa{{\isacharbraceleft}{\isasymbottom}{\isacharbraceright}}, se tiene \isa{G\ {\isacharequal}\ {\isasymbottom}}. Por tanto, se cumple la propiedad.

  Sea la fórmula \isa{F} cualquiera tal que para cualquier subfórmula 
  \isa{G\ {\isasymin}\ Subf{\isacharparenleft}F{\isacharparenright}} se verifica que el conjunto de átomos de \isa{G} está 
  contenido en el de \isa{F}. Supongamos \isa{G{\isacharprime}\ {\isasymin}\ Subf{\isacharparenleft}{\isasymnot}\ F{\isacharparenright}} cualquiera, 
  probemos que \isa{conjAtoms{\isacharparenleft}G{\isacharprime}{\isacharparenright}\ {\isasymsubseteq}\ conjAtoms{\isacharparenleft}{\isasymnot}\ F{\isacharparenright}}.
  Por definición, tenemos que \isa{Subf{\isacharparenleft}{\isasymnot}\ F{\isacharparenright}\ {\isacharequal}\ {\isacharbraceleft}{\isasymnot}\ F{\isacharbraceright}\ {\isasymunion}\ Subf{\isacharparenleft}F{\isacharparenright}}. De este 
  modo, tenemos dos opciones:
  \isa{G{\isacharprime}\ {\isasymin}\ {\isacharbraceleft}{\isasymnot}\ F{\isacharbraceright}} o \isa{G{\isacharprime}\ {\isasymin}\ Subf{\isacharparenleft}F{\isacharparenright}}. Del primer caso se deduce \isa{G{\isacharprime}\ {\isacharequal}\ {\isasymnot}\ F} 
  y, por tanto, se verifica el resultado. Observando el segundo caso, 
  por hipótesis de inducción, se tiene que el conjunto de átomos de \isa{G{\isacharprime}}
  está contenido en el de \isa{F}. Además, como el conjunto de átomos de 
  \isa{F} y \isa{{\isasymnot}\ F} coinciden, se verifica el resultado.

  Sea \isa{F{\isadigit{1}}} fórmula proposicional tal que para cualquier \isa{G\ {\isasymin}\ Subf{\isacharparenleft}F{\isadigit{1}}{\isacharparenright}} 
  se tiene que el conjunto de átomos de \isa{G} está contenido en el de 
  \isa{F{\isadigit{1}}}. Sea también \isa{F{\isadigit{2}}} tal que dada \isa{G\ {\isasymin}\ Subf{\isacharparenleft}F{\isadigit{2}}{\isacharparenright}} cualquiera se 
  verifica también la hipótesis de inducción en su caso. Supongamos 
  \isa{G{\isacharprime}\ {\isasymin}\ Subf{\isacharparenleft}F{\isadigit{1}}{\isacharasterisk}F{\isadigit{2}}{\isacharparenright}} donde \isa{{\isacharasterisk}} es cualquier conectiva binaria. Vamos a 
  probar que el conjunto de átomos de \isa{G} está contenido en el de 
  \isa{F{\isadigit{1}}{\isacharasterisk}F{\isadigit{2}}}.

  En primer lugar, como 
  \isa{Subf{\isacharparenleft}F{\isadigit{1}}{\isacharasterisk}F{\isadigit{2}}{\isacharparenright}\ {\isacharequal}\ {\isacharbraceleft}F{\isadigit{1}}{\isacharasterisk}F{\isadigit{2}}{\isacharbraceright}\ {\isasymunion}\ {\isacharparenleft}Subf{\isacharparenleft}F{\isadigit{1}}{\isacharparenright}\ {\isasymunion}\ Subf{\isacharparenleft}F{\isadigit{2}}{\isacharparenright}{\isacharparenright}}, se desglosan tres
  casos posibles para \isa{G{\isacharprime}}:
  Si \isa{G{\isacharprime}\ {\isasymin}\ {\isacharbraceleft}F{\isadigit{1}}{\isacharasterisk}F{\isadigit{2}}{\isacharbraceright}}, entonces \isa{G{\isacharprime}\ {\isacharequal}\ F{\isadigit{1}}{\isacharasterisk}F{\isadigit{2}}} y se tiene la propiedad.
  Si \isa{G{\isacharprime}\ {\isasymin}\ Subf{\isacharparenleft}F{\isadigit{1}}{\isacharparenright}\ {\isasymunion}\ Subf{\isacharparenleft}F{\isadigit{2}}{\isacharparenright}}, entonces por propiedades de 
  conjuntos:
  \isa{G{\isacharprime}\ {\isasymin}\ Subf{\isacharparenleft}F{\isadigit{1}}{\isacharparenright}} o \isa{G{\isacharprime}\ {\isasymin}\ Subf{\isacharparenleft}F{\isadigit{2}}{\isacharparenright}}. Si \isa{G{\isacharprime}\ {\isasymin}\ Subf{\isacharparenleft}F{\isadigit{1}}{\isacharparenright}}, por hipótesis 
  de inducción se tiene el resultado. Como el conjunto de átomos de
  \isa{F{\isadigit{1}}{\isacharasterisk}F{\isadigit{2}}} es la unión de los conjuntos de átomos de \isa{F{\isadigit{1}}} y \isa{F{\isadigit{2}}}, se 
  obtiene el resultado como consecuencia de la transitividad de 
  contención para conjuntos. El caso \isa{G{\isacharprime}\ {\isasymin}\ Subf{\isacharparenleft}F{\isadigit{2}}{\isacharparenright}} se demuestra de la 
  misma forma.      
  \end{demostracion}

  Formalizado en Isabelle:%
\end{isamarkuptext}\isamarkuptrue%
\isacommand{lemma}\isamarkupfalse%
\ {\isachardoublequoteopen}G\ {\isasymin}\ setSubformulae\ F\ {\isasymLongrightarrow}\ atoms\ G\ {\isasymsubseteq}\ atoms\ F{\isachardoublequoteclose}\isanewline
%
\isadelimproof
\ \ %
\endisadelimproof
%
\isatagproof
\isacommand{oops}\isamarkupfalse%
%
\endisatagproof
{\isafoldproof}%
%
\isadelimproof
%
\endisadelimproof
%
\begin{isamarkuptext}%
Veamos su demostración estructurada.%
\end{isamarkuptext}\isamarkuptrue%
\isacommand{lemma}\isamarkupfalse%
\ subformulas{\isacharunderscore}atoms{\isacharunderscore}atom{\isacharcolon}\isanewline
\ \ \isakeyword{assumes}\ {\isachardoublequoteopen}G\ {\isasymin}\ setSubformulae\ {\isacharparenleft}Atom\ x{\isacharparenright}{\isachardoublequoteclose}\ \isanewline
\ \ \isakeyword{shows}\ \ \ {\isachardoublequoteopen}atoms\ G\ {\isasymsubseteq}\ atoms\ {\isacharparenleft}Atom\ x{\isacharparenright}{\isachardoublequoteclose}\isanewline
%
\isadelimproof
%
\endisadelimproof
%
\isatagproof
\isacommand{proof}\isamarkupfalse%
\ {\isacharminus}\isanewline
\ \ \isacommand{have}\isamarkupfalse%
\ {\isachardoublequoteopen}G\ {\isasymin}\ {\isacharbraceleft}Atom\ x{\isacharbraceright}{\isachardoublequoteclose}\isanewline
\ \ \ \ \isacommand{using}\isamarkupfalse%
\ assms\isanewline
\ \ \ \ \isacommand{by}\isamarkupfalse%
\ {\isacharparenleft}simp\ only{\isacharcolon}\ setSubformulae{\isacharunderscore}atom{\isacharparenright}\isanewline
\ \ \isacommand{then}\isamarkupfalse%
\ \isacommand{have}\isamarkupfalse%
\ {\isachardoublequoteopen}G\ {\isacharequal}\ Atom\ x{\isachardoublequoteclose}\isanewline
\ \ \ \ \isacommand{by}\isamarkupfalse%
\ {\isacharparenleft}simp\ only{\isacharcolon}\ singletonD{\isacharparenright}\isanewline
\ \ \isacommand{then}\isamarkupfalse%
\ \isacommand{show}\isamarkupfalse%
\ {\isacharquery}thesis\isanewline
\ \ \ \ \isacommand{by}\isamarkupfalse%
\ {\isacharparenleft}simp\ only{\isacharcolon}\ subset{\isacharunderscore}refl{\isacharparenright}\isanewline
\isacommand{qed}\isamarkupfalse%
%
\endisatagproof
{\isafoldproof}%
%
\isadelimproof
\isanewline
%
\endisadelimproof
\isanewline
\isacommand{lemma}\isamarkupfalse%
\ subformulas{\isacharunderscore}atoms{\isacharunderscore}bot{\isacharcolon}\isanewline
\ \ \isakeyword{assumes}\ {\isachardoublequoteopen}G\ {\isasymin}\ setSubformulae\ {\isasymbottom}{\isachardoublequoteclose}\ \isanewline
\ \ \isakeyword{shows}\ \ \ {\isachardoublequoteopen}atoms\ G\ {\isasymsubseteq}\ atoms\ {\isasymbottom}{\isachardoublequoteclose}\isanewline
%
\isadelimproof
%
\endisadelimproof
%
\isatagproof
\isacommand{proof}\isamarkupfalse%
\ {\isacharminus}\isanewline
\ \ \isacommand{have}\isamarkupfalse%
\ {\isachardoublequoteopen}G\ {\isasymin}\ {\isacharbraceleft}{\isasymbottom}{\isacharbraceright}{\isachardoublequoteclose}\isanewline
\ \ \ \ \isacommand{using}\isamarkupfalse%
\ assms\isanewline
\ \ \ \ \isacommand{by}\isamarkupfalse%
\ {\isacharparenleft}simp\ only{\isacharcolon}\ setSubformulae{\isacharunderscore}bot{\isacharparenright}\isanewline
\ \ \isacommand{then}\isamarkupfalse%
\ \isacommand{have}\isamarkupfalse%
\ {\isachardoublequoteopen}G\ {\isacharequal}\ {\isasymbottom}{\isachardoublequoteclose}\isanewline
\ \ \ \ \isacommand{by}\isamarkupfalse%
\ {\isacharparenleft}simp\ only{\isacharcolon}\ singletonD{\isacharparenright}\isanewline
\ \ \isacommand{then}\isamarkupfalse%
\ \isacommand{show}\isamarkupfalse%
\ {\isacharquery}thesis\isanewline
\ \ \ \ \isacommand{by}\isamarkupfalse%
\ {\isacharparenleft}simp\ only{\isacharcolon}\ subset{\isacharunderscore}refl{\isacharparenright}\isanewline
\isacommand{qed}\isamarkupfalse%
%
\endisatagproof
{\isafoldproof}%
%
\isadelimproof
\isanewline
%
\endisadelimproof
\isanewline
\isacommand{lemma}\isamarkupfalse%
\ subformulas{\isacharunderscore}atoms{\isacharunderscore}not{\isacharcolon}\isanewline
\ \ \isakeyword{assumes}\ {\isachardoublequoteopen}G\ {\isasymin}\ setSubformulae\ F\ {\isasymLongrightarrow}\ atoms\ G\ {\isasymsubseteq}\ atoms\ F{\isachardoublequoteclose}\isanewline
\ \ \ \ \ \ \ \ \ \ {\isachardoublequoteopen}G\ {\isasymin}\ setSubformulae\ {\isacharparenleft}\isactrlbold {\isasymnot}\ F{\isacharparenright}{\isachardoublequoteclose}\isanewline
\ \ \isakeyword{shows}\ \ \ {\isachardoublequoteopen}atoms\ G\ {\isasymsubseteq}\ atoms\ {\isacharparenleft}\isactrlbold {\isasymnot}\ F{\isacharparenright}{\isachardoublequoteclose}\isanewline
%
\isadelimproof
%
\endisadelimproof
%
\isatagproof
\isacommand{proof}\isamarkupfalse%
\ {\isacharminus}\isanewline
\ \ \isacommand{have}\isamarkupfalse%
\ {\isachardoublequoteopen}G\ {\isasymin}\ {\isacharbraceleft}\isactrlbold {\isasymnot}\ F{\isacharbraceright}\ {\isasymunion}\ setSubformulae\ F{\isachardoublequoteclose}\isanewline
\ \ \ \ \isacommand{using}\isamarkupfalse%
\ assms{\isacharparenleft}{\isadigit{2}}{\isacharparenright}\isanewline
\ \ \ \ \isacommand{by}\isamarkupfalse%
\ {\isacharparenleft}simp\ only{\isacharcolon}\ setSubformulae{\isacharunderscore}not{\isacharparenright}\ \isanewline
\ \ \isacommand{then}\isamarkupfalse%
\ \isacommand{have}\isamarkupfalse%
\ {\isachardoublequoteopen}G\ {\isasymin}\ {\isacharbraceleft}\isactrlbold {\isasymnot}\ F{\isacharbraceright}\ {\isasymor}\ G\ {\isasymin}\ setSubformulae\ F{\isachardoublequoteclose}\isanewline
\ \ \ \ \isacommand{by}\isamarkupfalse%
\ {\isacharparenleft}simp\ only{\isacharcolon}\ Un{\isacharunderscore}iff{\isacharparenright}\isanewline
\ \ \isacommand{then}\isamarkupfalse%
\ \isacommand{show}\isamarkupfalse%
\ {\isachardoublequoteopen}atoms\ G\ {\isasymsubseteq}\ atoms\ {\isacharparenleft}\isactrlbold {\isasymnot}\ F{\isacharparenright}{\isachardoublequoteclose}\isanewline
\ \ \isacommand{proof}\isamarkupfalse%
\isanewline
\ \ \ \ \isacommand{assume}\isamarkupfalse%
\ {\isachardoublequoteopen}G\ {\isasymin}\ {\isacharbraceleft}\isactrlbold {\isasymnot}\ F{\isacharbraceright}{\isachardoublequoteclose}\isanewline
\ \ \ \ \isacommand{then}\isamarkupfalse%
\ \isacommand{have}\isamarkupfalse%
\ {\isachardoublequoteopen}G\ {\isacharequal}\ \isactrlbold {\isasymnot}\ F{\isachardoublequoteclose}\isanewline
\ \ \ \ \ \ \isacommand{by}\isamarkupfalse%
\ {\isacharparenleft}simp\ only{\isacharcolon}\ singletonD{\isacharparenright}\isanewline
\ \ \ \ \isacommand{then}\isamarkupfalse%
\ \isacommand{show}\isamarkupfalse%
\ {\isacharquery}thesis\isanewline
\ \ \ \ \ \ \isacommand{by}\isamarkupfalse%
\ {\isacharparenleft}simp\ only{\isacharcolon}\ subset{\isacharunderscore}refl{\isacharparenright}\isanewline
\ \ \isacommand{next}\isamarkupfalse%
\isanewline
\ \ \ \ \isacommand{assume}\isamarkupfalse%
\ {\isachardoublequoteopen}G\ {\isasymin}\ setSubformulae\ F{\isachardoublequoteclose}\isanewline
\ \ \ \ \isacommand{then}\isamarkupfalse%
\ \isacommand{have}\isamarkupfalse%
\ {\isachardoublequoteopen}atoms\ G\ {\isasymsubseteq}\ atoms\ F{\isachardoublequoteclose}\isanewline
\ \ \ \ \ \ \isacommand{by}\isamarkupfalse%
\ {\isacharparenleft}simp\ only{\isacharcolon}\ assms{\isacharparenleft}{\isadigit{1}}{\isacharparenright}{\isacharparenright}\isanewline
\ \ \ \ \isacommand{also}\isamarkupfalse%
\ \isacommand{have}\isamarkupfalse%
\ {\isachardoublequoteopen}{\isasymdots}\ {\isacharequal}\ atoms\ {\isacharparenleft}\isactrlbold {\isasymnot}\ F{\isacharparenright}{\isachardoublequoteclose}\isanewline
\ \ \ \ \ \ \isacommand{by}\isamarkupfalse%
\ {\isacharparenleft}simp\ only{\isacharcolon}\ formula{\isachardot}set{\isacharparenleft}{\isadigit{3}}{\isacharparenright}{\isacharparenright}\isanewline
\ \ \ \ \isacommand{finally}\isamarkupfalse%
\ \isacommand{show}\isamarkupfalse%
\ {\isacharquery}thesis\isanewline
\ \ \ \ \ \ \isacommand{by}\isamarkupfalse%
\ this\isanewline
\ \ \isacommand{qed}\isamarkupfalse%
\isanewline
\isacommand{qed}\isamarkupfalse%
%
\endisatagproof
{\isafoldproof}%
%
\isadelimproof
\isanewline
%
\endisadelimproof
\isanewline
\isacommand{lemma}\isamarkupfalse%
\ subformulas{\isacharunderscore}atoms{\isacharunderscore}and{\isacharcolon}\isanewline
\ \ \isakeyword{assumes}\ {\isachardoublequoteopen}G\ {\isasymin}\ setSubformulae\ F{\isadigit{1}}\ {\isasymLongrightarrow}\ atoms\ G\ {\isasymsubseteq}\ atoms\ F{\isadigit{1}}{\isachardoublequoteclose}\isanewline
\ \ \ \ \ \ \ \ \ \ {\isachardoublequoteopen}G\ {\isasymin}\ setSubformulae\ F{\isadigit{2}}\ {\isasymLongrightarrow}\ atoms\ G\ {\isasymsubseteq}\ atoms\ F{\isadigit{2}}{\isachardoublequoteclose}\isanewline
\ \ \ \ \ \ \ \ \ \ {\isachardoublequoteopen}G\ {\isasymin}\ setSubformulae\ {\isacharparenleft}F{\isadigit{1}}\ \isactrlbold {\isasymand}\ F{\isadigit{2}}{\isacharparenright}{\isachardoublequoteclose}\isanewline
\ \ \isakeyword{shows}\ \ \ {\isachardoublequoteopen}atoms\ G\ {\isasymsubseteq}\ atoms\ {\isacharparenleft}F{\isadigit{1}}\ \isactrlbold {\isasymand}\ F{\isadigit{2}}{\isacharparenright}{\isachardoublequoteclose}\isanewline
%
\isadelimproof
%
\endisadelimproof
%
\isatagproof
\isacommand{proof}\isamarkupfalse%
\ {\isacharminus}\isanewline
\ \ \isacommand{have}\isamarkupfalse%
\ {\isachardoublequoteopen}G\ {\isasymin}\ {\isacharbraceleft}F{\isadigit{1}}\ \isactrlbold {\isasymand}\ F{\isadigit{2}}{\isacharbraceright}\ {\isasymunion}\ {\isacharparenleft}setSubformulae\ F{\isadigit{1}}\ {\isasymunion}\ setSubformulae\ F{\isadigit{2}}{\isacharparenright}{\isachardoublequoteclose}\isanewline
\ \ \ \ \isacommand{using}\isamarkupfalse%
\ assms{\isacharparenleft}{\isadigit{3}}{\isacharparenright}\ \isanewline
\ \ \ \ \isacommand{by}\isamarkupfalse%
\ {\isacharparenleft}simp\ only{\isacharcolon}\ setSubformulae{\isacharunderscore}and{\isacharparenright}\isanewline
\ \ \isacommand{then}\isamarkupfalse%
\ \isacommand{have}\isamarkupfalse%
\ {\isachardoublequoteopen}G\ {\isasymin}\ {\isacharbraceleft}F{\isadigit{1}}\ \isactrlbold {\isasymand}\ F{\isadigit{2}}{\isacharbraceright}\ {\isasymor}\ G\ {\isasymin}\ setSubformulae\ F{\isadigit{1}}\ {\isasymunion}\ setSubformulae\ F{\isadigit{2}}{\isachardoublequoteclose}\isanewline
\ \ \ \ \isacommand{by}\isamarkupfalse%
\ {\isacharparenleft}simp\ only{\isacharcolon}\ Un{\isacharunderscore}iff{\isacharparenright}\isanewline
\ \ \isacommand{then}\isamarkupfalse%
\ \isacommand{show}\isamarkupfalse%
\ {\isacharquery}thesis\isanewline
\ \ \isacommand{proof}\isamarkupfalse%
\ \isanewline
\ \ \ \ \isacommand{assume}\isamarkupfalse%
\ {\isachardoublequoteopen}G\ {\isasymin}\ {\isacharbraceleft}F{\isadigit{1}}\ \isactrlbold {\isasymand}\ F{\isadigit{2}}{\isacharbraceright}{\isachardoublequoteclose}\isanewline
\ \ \ \ \isacommand{then}\isamarkupfalse%
\ \isacommand{have}\isamarkupfalse%
\ {\isachardoublequoteopen}G\ {\isacharequal}\ F{\isadigit{1}}\ \isactrlbold {\isasymand}\ F{\isadigit{2}}{\isachardoublequoteclose}\isanewline
\ \ \ \ \ \ \isacommand{by}\isamarkupfalse%
\ {\isacharparenleft}simp\ only{\isacharcolon}\ singletonD{\isacharparenright}\isanewline
\ \ \ \ \isacommand{then}\isamarkupfalse%
\ \isacommand{show}\isamarkupfalse%
\ {\isacharquery}thesis\isanewline
\ \ \ \ \ \ \isacommand{by}\isamarkupfalse%
\ {\isacharparenleft}simp\ only{\isacharcolon}\ subset{\isacharunderscore}refl{\isacharparenright}\isanewline
\ \ \isacommand{next}\isamarkupfalse%
\isanewline
\ \ \ \ \isacommand{assume}\isamarkupfalse%
\ {\isachardoublequoteopen}G\ {\isasymin}\ setSubformulae\ F{\isadigit{1}}\ {\isasymunion}\ setSubformulae\ F{\isadigit{2}}{\isachardoublequoteclose}\isanewline
\ \ \ \ \isacommand{then}\isamarkupfalse%
\ \isacommand{have}\isamarkupfalse%
\ {\isachardoublequoteopen}G\ {\isasymin}\ setSubformulae\ F{\isadigit{1}}\ {\isasymor}\ G\ {\isasymin}\ setSubformulae\ F{\isadigit{2}}{\isachardoublequoteclose}\ \ \isanewline
\ \ \ \ \ \ \isacommand{by}\isamarkupfalse%
\ {\isacharparenleft}simp\ only{\isacharcolon}\ Un{\isacharunderscore}iff{\isacharparenright}\isanewline
\ \ \ \ \isacommand{then}\isamarkupfalse%
\ \isacommand{show}\isamarkupfalse%
\ {\isacharquery}thesis\isanewline
\ \ \ \ \isacommand{proof}\isamarkupfalse%
\ \isanewline
\ \ \ \ \ \ \isacommand{assume}\isamarkupfalse%
\ {\isachardoublequoteopen}G\ {\isasymin}\ setSubformulae\ F{\isadigit{1}}{\isachardoublequoteclose}\isanewline
\ \ \ \ \ \ \isacommand{then}\isamarkupfalse%
\ \isacommand{have}\isamarkupfalse%
\ {\isachardoublequoteopen}atoms\ G\ {\isasymsubseteq}\ atoms\ F{\isadigit{1}}{\isachardoublequoteclose}\isanewline
\ \ \ \ \ \ \ \ \isacommand{by}\isamarkupfalse%
\ {\isacharparenleft}rule\ assms{\isacharparenleft}{\isadigit{1}}{\isacharparenright}{\isacharparenright}\isanewline
\ \ \ \ \ \ \isacommand{also}\isamarkupfalse%
\ \isacommand{have}\isamarkupfalse%
\ {\isachardoublequoteopen}{\isasymdots}\ {\isasymsubseteq}\ atoms\ F{\isadigit{1}}\ {\isasymunion}\ atoms\ F{\isadigit{2}}{\isachardoublequoteclose}\isanewline
\ \ \ \ \ \ \ \ \isacommand{by}\isamarkupfalse%
\ {\isacharparenleft}simp\ only{\isacharcolon}\ Un{\isacharunderscore}upper{\isadigit{1}}{\isacharparenright}\isanewline
\ \ \ \ \ \ \isacommand{also}\isamarkupfalse%
\ \isacommand{have}\isamarkupfalse%
\ {\isachardoublequoteopen}{\isasymdots}\ {\isacharequal}\ atoms\ {\isacharparenleft}F{\isadigit{1}}\ \isactrlbold {\isasymand}\ F{\isadigit{2}}{\isacharparenright}{\isachardoublequoteclose}\isanewline
\ \ \ \ \ \ \ \ \isacommand{by}\isamarkupfalse%
\ {\isacharparenleft}simp\ only{\isacharcolon}\ formula{\isachardot}set{\isacharparenleft}{\isadigit{4}}{\isacharparenright}{\isacharparenright}\isanewline
\ \ \ \ \ \ \isacommand{finally}\isamarkupfalse%
\ \isacommand{show}\isamarkupfalse%
\ {\isacharquery}thesis\isanewline
\ \ \ \ \ \ \ \ \isacommand{by}\isamarkupfalse%
\ this\isanewline
\ \ \ \ \isacommand{next}\isamarkupfalse%
\isanewline
\ \ \ \ \ \ \isacommand{assume}\isamarkupfalse%
\ {\isachardoublequoteopen}G\ {\isasymin}\ setSubformulae\ F{\isadigit{2}}{\isachardoublequoteclose}\isanewline
\ \ \ \ \ \ \isacommand{then}\isamarkupfalse%
\ \isacommand{have}\isamarkupfalse%
\ {\isachardoublequoteopen}atoms\ G\ {\isasymsubseteq}\ atoms\ F{\isadigit{2}}{\isachardoublequoteclose}\isanewline
\ \ \ \ \ \ \ \ \isacommand{by}\isamarkupfalse%
\ {\isacharparenleft}rule\ assms{\isacharparenleft}{\isadigit{2}}{\isacharparenright}{\isacharparenright}\isanewline
\ \ \ \ \ \ \isacommand{also}\isamarkupfalse%
\ \isacommand{have}\isamarkupfalse%
\ {\isachardoublequoteopen}{\isasymdots}\ {\isasymsubseteq}\ atoms\ F{\isadigit{1}}\ {\isasymunion}\ atoms\ F{\isadigit{2}}{\isachardoublequoteclose}\isanewline
\ \ \ \ \ \ \ \ \isacommand{by}\isamarkupfalse%
\ {\isacharparenleft}simp\ only{\isacharcolon}\ Un{\isacharunderscore}upper{\isadigit{2}}{\isacharparenright}\isanewline
\ \ \ \ \ \ \isacommand{also}\isamarkupfalse%
\ \isacommand{have}\isamarkupfalse%
\ {\isachardoublequoteopen}{\isasymdots}\ {\isacharequal}\ atoms\ {\isacharparenleft}F{\isadigit{1}}\ \isactrlbold {\isasymand}\ F{\isadigit{2}}{\isacharparenright}{\isachardoublequoteclose}\isanewline
\ \ \ \ \ \ \ \ \isacommand{by}\isamarkupfalse%
\ {\isacharparenleft}simp\ only{\isacharcolon}\ formula{\isachardot}set{\isacharparenleft}{\isadigit{4}}{\isacharparenright}{\isacharparenright}\isanewline
\ \ \ \ \ \ \isacommand{finally}\isamarkupfalse%
\ \isacommand{show}\isamarkupfalse%
\ {\isacharquery}thesis\isanewline
\ \ \ \ \ \ \ \ \isacommand{by}\isamarkupfalse%
\ this\isanewline
\ \ \ \ \isacommand{qed}\isamarkupfalse%
\isanewline
\ \ \isacommand{qed}\isamarkupfalse%
\isanewline
\isacommand{qed}\isamarkupfalse%
%
\endisatagproof
{\isafoldproof}%
%
\isadelimproof
\isanewline
%
\endisadelimproof
\isanewline
\isacommand{lemma}\isamarkupfalse%
\ subformulas{\isacharunderscore}atoms{\isacharunderscore}or{\isacharcolon}\isanewline
\ \ \isakeyword{assumes}\ {\isachardoublequoteopen}G\ {\isasymin}\ setSubformulae\ F{\isadigit{1}}\ {\isasymLongrightarrow}\ atoms\ G\ {\isasymsubseteq}\ atoms\ F{\isadigit{1}}{\isachardoublequoteclose}\isanewline
\ \ \ \ \ \ \ \ \ \ {\isachardoublequoteopen}G\ {\isasymin}\ setSubformulae\ F{\isadigit{2}}\ {\isasymLongrightarrow}\ atoms\ G\ {\isasymsubseteq}\ atoms\ F{\isadigit{2}}{\isachardoublequoteclose}\isanewline
\ \ \ \ \ \ \ \ \ \ {\isachardoublequoteopen}G\ {\isasymin}\ setSubformulae\ {\isacharparenleft}F{\isadigit{1}}\ \isactrlbold {\isasymor}\ F{\isadigit{2}}{\isacharparenright}{\isachardoublequoteclose}\isanewline
\ \ \isakeyword{shows}\ \ \ {\isachardoublequoteopen}atoms\ G\ {\isasymsubseteq}\ atoms\ {\isacharparenleft}F{\isadigit{1}}\ \isactrlbold {\isasymor}\ F{\isadigit{2}}{\isacharparenright}{\isachardoublequoteclose}\isanewline
%
\isadelimproof
%
\endisadelimproof
%
\isatagproof
\isacommand{proof}\isamarkupfalse%
\ {\isacharminus}\isanewline
\ \ \isacommand{have}\isamarkupfalse%
\ {\isachardoublequoteopen}G\ {\isasymin}\ {\isacharbraceleft}F{\isadigit{1}}\ \isactrlbold {\isasymor}\ F{\isadigit{2}}{\isacharbraceright}\ {\isasymunion}\ {\isacharparenleft}setSubformulae\ F{\isadigit{1}}\ {\isasymunion}\ setSubformulae\ F{\isadigit{2}}{\isacharparenright}{\isachardoublequoteclose}\isanewline
\ \ \ \ \isacommand{using}\isamarkupfalse%
\ assms{\isacharparenleft}{\isadigit{3}}{\isacharparenright}\ \isanewline
\ \ \ \ \isacommand{by}\isamarkupfalse%
\ {\isacharparenleft}simp\ only{\isacharcolon}\ setSubformulae{\isacharunderscore}or{\isacharparenright}\isanewline
\ \ \isacommand{then}\isamarkupfalse%
\ \isacommand{have}\isamarkupfalse%
\ {\isachardoublequoteopen}G\ {\isasymin}\ {\isacharbraceleft}F{\isadigit{1}}\ \isactrlbold {\isasymor}\ F{\isadigit{2}}{\isacharbraceright}\ {\isasymor}\ G\ {\isasymin}\ setSubformulae\ F{\isadigit{1}}\ {\isasymunion}\ setSubformulae\ F{\isadigit{2}}{\isachardoublequoteclose}\isanewline
\ \ \ \ \isacommand{by}\isamarkupfalse%
\ {\isacharparenleft}simp\ only{\isacharcolon}\ Un{\isacharunderscore}iff{\isacharparenright}\isanewline
\ \ \isacommand{then}\isamarkupfalse%
\ \isacommand{show}\isamarkupfalse%
\ {\isacharquery}thesis\isanewline
\ \ \isacommand{proof}\isamarkupfalse%
\ \isanewline
\ \ \ \ \isacommand{assume}\isamarkupfalse%
\ {\isachardoublequoteopen}G\ {\isasymin}\ {\isacharbraceleft}F{\isadigit{1}}\ \isactrlbold {\isasymor}\ F{\isadigit{2}}{\isacharbraceright}{\isachardoublequoteclose}\isanewline
\ \ \ \ \isacommand{then}\isamarkupfalse%
\ \isacommand{have}\isamarkupfalse%
\ {\isachardoublequoteopen}G\ {\isacharequal}\ F{\isadigit{1}}\ \isactrlbold {\isasymor}\ F{\isadigit{2}}{\isachardoublequoteclose}\isanewline
\ \ \ \ \ \ \isacommand{by}\isamarkupfalse%
\ {\isacharparenleft}simp\ only{\isacharcolon}\ singletonD{\isacharparenright}\isanewline
\ \ \ \ \isacommand{then}\isamarkupfalse%
\ \isacommand{show}\isamarkupfalse%
\ {\isacharquery}thesis\isanewline
\ \ \ \ \ \ \isacommand{by}\isamarkupfalse%
\ {\isacharparenleft}simp\ only{\isacharcolon}\ subset{\isacharunderscore}refl{\isacharparenright}\isanewline
\ \ \isacommand{next}\isamarkupfalse%
\isanewline
\ \ \ \ \isacommand{assume}\isamarkupfalse%
\ {\isachardoublequoteopen}G\ {\isasymin}\ setSubformulae\ F{\isadigit{1}}\ {\isasymunion}\ setSubformulae\ F{\isadigit{2}}{\isachardoublequoteclose}\isanewline
\ \ \ \ \isacommand{then}\isamarkupfalse%
\ \isacommand{have}\isamarkupfalse%
\ {\isachardoublequoteopen}G\ {\isasymin}\ setSubformulae\ F{\isadigit{1}}\ {\isasymor}\ G\ {\isasymin}\ setSubformulae\ F{\isadigit{2}}{\isachardoublequoteclose}\ \ \isanewline
\ \ \ \ \ \ \isacommand{by}\isamarkupfalse%
\ {\isacharparenleft}simp\ only{\isacharcolon}\ Un{\isacharunderscore}iff{\isacharparenright}\isanewline
\ \ \ \ \isacommand{then}\isamarkupfalse%
\ \isacommand{show}\isamarkupfalse%
\ {\isacharquery}thesis\isanewline
\ \ \ \ \isacommand{proof}\isamarkupfalse%
\ \isanewline
\ \ \ \ \ \ \isacommand{assume}\isamarkupfalse%
\ {\isachardoublequoteopen}G\ {\isasymin}\ setSubformulae\ F{\isadigit{1}}{\isachardoublequoteclose}\isanewline
\ \ \ \ \ \ \isacommand{then}\isamarkupfalse%
\ \isacommand{have}\isamarkupfalse%
\ {\isachardoublequoteopen}atoms\ G\ {\isasymsubseteq}\ atoms\ F{\isadigit{1}}{\isachardoublequoteclose}\isanewline
\ \ \ \ \ \ \ \ \isacommand{by}\isamarkupfalse%
\ {\isacharparenleft}rule\ assms{\isacharparenleft}{\isadigit{1}}{\isacharparenright}{\isacharparenright}\isanewline
\ \ \ \ \ \ \isacommand{also}\isamarkupfalse%
\ \isacommand{have}\isamarkupfalse%
\ {\isachardoublequoteopen}{\isasymdots}\ {\isasymsubseteq}\ atoms\ F{\isadigit{1}}\ {\isasymunion}\ atoms\ F{\isadigit{2}}{\isachardoublequoteclose}\isanewline
\ \ \ \ \ \ \ \ \isacommand{by}\isamarkupfalse%
\ {\isacharparenleft}simp\ only{\isacharcolon}\ Un{\isacharunderscore}upper{\isadigit{1}}{\isacharparenright}\isanewline
\ \ \ \ \ \ \isacommand{also}\isamarkupfalse%
\ \isacommand{have}\isamarkupfalse%
\ {\isachardoublequoteopen}{\isasymdots}\ {\isacharequal}\ atoms\ {\isacharparenleft}F{\isadigit{1}}\ \isactrlbold {\isasymor}\ F{\isadigit{2}}{\isacharparenright}{\isachardoublequoteclose}\isanewline
\ \ \ \ \ \ \ \ \isacommand{by}\isamarkupfalse%
\ {\isacharparenleft}simp\ only{\isacharcolon}\ formula{\isachardot}set{\isacharparenleft}{\isadigit{5}}{\isacharparenright}{\isacharparenright}\isanewline
\ \ \ \ \ \ \isacommand{finally}\isamarkupfalse%
\ \isacommand{show}\isamarkupfalse%
\ {\isacharquery}thesis\isanewline
\ \ \ \ \ \ \ \ \isacommand{by}\isamarkupfalse%
\ this\isanewline
\ \ \ \ \isacommand{next}\isamarkupfalse%
\isanewline
\ \ \ \ \ \ \isacommand{assume}\isamarkupfalse%
\ {\isachardoublequoteopen}G\ {\isasymin}\ setSubformulae\ F{\isadigit{2}}{\isachardoublequoteclose}\isanewline
\ \ \ \ \ \ \isacommand{then}\isamarkupfalse%
\ \isacommand{have}\isamarkupfalse%
\ {\isachardoublequoteopen}atoms\ G\ {\isasymsubseteq}\ atoms\ F{\isadigit{2}}{\isachardoublequoteclose}\isanewline
\ \ \ \ \ \ \ \ \isacommand{by}\isamarkupfalse%
\ {\isacharparenleft}rule\ assms{\isacharparenleft}{\isadigit{2}}{\isacharparenright}{\isacharparenright}\isanewline
\ \ \ \ \ \ \isacommand{also}\isamarkupfalse%
\ \isacommand{have}\isamarkupfalse%
\ {\isachardoublequoteopen}{\isasymdots}\ {\isasymsubseteq}\ atoms\ F{\isadigit{1}}\ {\isasymunion}\ atoms\ F{\isadigit{2}}{\isachardoublequoteclose}\isanewline
\ \ \ \ \ \ \ \ \isacommand{by}\isamarkupfalse%
\ {\isacharparenleft}simp\ only{\isacharcolon}\ Un{\isacharunderscore}upper{\isadigit{2}}{\isacharparenright}\isanewline
\ \ \ \ \ \ \isacommand{also}\isamarkupfalse%
\ \isacommand{have}\isamarkupfalse%
\ {\isachardoublequoteopen}{\isasymdots}\ {\isacharequal}\ atoms\ {\isacharparenleft}F{\isadigit{1}}\ \isactrlbold {\isasymor}\ F{\isadigit{2}}{\isacharparenright}{\isachardoublequoteclose}\isanewline
\ \ \ \ \ \ \ \ \isacommand{by}\isamarkupfalse%
\ {\isacharparenleft}simp\ only{\isacharcolon}\ formula{\isachardot}set{\isacharparenleft}{\isadigit{5}}{\isacharparenright}{\isacharparenright}\isanewline
\ \ \ \ \ \ \isacommand{finally}\isamarkupfalse%
\ \isacommand{show}\isamarkupfalse%
\ {\isacharquery}thesis\isanewline
\ \ \ \ \ \ \ \ \isacommand{by}\isamarkupfalse%
\ this\isanewline
\ \ \ \ \isacommand{qed}\isamarkupfalse%
\isanewline
\ \ \isacommand{qed}\isamarkupfalse%
\isanewline
\isacommand{qed}\isamarkupfalse%
%
\endisatagproof
{\isafoldproof}%
%
\isadelimproof
\isanewline
%
\endisadelimproof
\isanewline
\isacommand{lemma}\isamarkupfalse%
\ subformulas{\isacharunderscore}atoms{\isacharunderscore}imp{\isacharcolon}\isanewline
\ \ \isakeyword{assumes}\ {\isachardoublequoteopen}G\ {\isasymin}\ setSubformulae\ F{\isadigit{1}}\ {\isasymLongrightarrow}\ atoms\ G\ {\isasymsubseteq}\ atoms\ F{\isadigit{1}}{\isachardoublequoteclose}\isanewline
\ \ \ \ \ \ \ \ \ \ {\isachardoublequoteopen}G\ {\isasymin}\ setSubformulae\ F{\isadigit{2}}\ {\isasymLongrightarrow}\ atoms\ G\ {\isasymsubseteq}\ atoms\ F{\isadigit{2}}{\isachardoublequoteclose}\isanewline
\ \ \ \ \ \ \ \ \ \ {\isachardoublequoteopen}G\ {\isasymin}\ setSubformulae\ {\isacharparenleft}F{\isadigit{1}}\ \isactrlbold {\isasymrightarrow}\ F{\isadigit{2}}{\isacharparenright}{\isachardoublequoteclose}\isanewline
\ \ \isakeyword{shows}\ \ \ {\isachardoublequoteopen}atoms\ G\ {\isasymsubseteq}\ atoms\ {\isacharparenleft}F{\isadigit{1}}\ \isactrlbold {\isasymrightarrow}\ F{\isadigit{2}}{\isacharparenright}{\isachardoublequoteclose}\isanewline
%
\isadelimproof
%
\endisadelimproof
%
\isatagproof
\isacommand{proof}\isamarkupfalse%
\ {\isacharminus}\isanewline
\ \ \isacommand{have}\isamarkupfalse%
\ {\isachardoublequoteopen}G\ {\isasymin}\ {\isacharbraceleft}F{\isadigit{1}}\ \isactrlbold {\isasymrightarrow}\ F{\isadigit{2}}{\isacharbraceright}\ {\isasymunion}\ {\isacharparenleft}setSubformulae\ F{\isadigit{1}}\ {\isasymunion}\ setSubformulae\ F{\isadigit{2}}{\isacharparenright}{\isachardoublequoteclose}\isanewline
\ \ \ \ \isacommand{using}\isamarkupfalse%
\ assms{\isacharparenleft}{\isadigit{3}}{\isacharparenright}\ \isanewline
\ \ \ \ \isacommand{by}\isamarkupfalse%
\ {\isacharparenleft}simp\ only{\isacharcolon}\ setSubformulae{\isacharunderscore}imp{\isacharparenright}\isanewline
\ \ \isacommand{then}\isamarkupfalse%
\ \isacommand{have}\isamarkupfalse%
\ {\isachardoublequoteopen}G\ {\isasymin}\ {\isacharbraceleft}F{\isadigit{1}}\ \isactrlbold {\isasymrightarrow}\ F{\isadigit{2}}{\isacharbraceright}\ {\isasymor}\ G\ {\isasymin}\ setSubformulae\ F{\isadigit{1}}\ {\isasymunion}\ setSubformulae\ F{\isadigit{2}}{\isachardoublequoteclose}\isanewline
\ \ \ \ \isacommand{by}\isamarkupfalse%
\ {\isacharparenleft}simp\ only{\isacharcolon}\ Un{\isacharunderscore}iff{\isacharparenright}\isanewline
\ \ \isacommand{then}\isamarkupfalse%
\ \isacommand{show}\isamarkupfalse%
\ {\isacharquery}thesis\isanewline
\ \ \isacommand{proof}\isamarkupfalse%
\ \isanewline
\ \ \ \ \isacommand{assume}\isamarkupfalse%
\ {\isachardoublequoteopen}G\ {\isasymin}\ {\isacharbraceleft}F{\isadigit{1}}\ \isactrlbold {\isasymrightarrow}\ F{\isadigit{2}}{\isacharbraceright}{\isachardoublequoteclose}\isanewline
\ \ \ \ \isacommand{then}\isamarkupfalse%
\ \isacommand{have}\isamarkupfalse%
\ {\isachardoublequoteopen}G\ {\isacharequal}\ F{\isadigit{1}}\ \isactrlbold {\isasymrightarrow}\ F{\isadigit{2}}{\isachardoublequoteclose}\isanewline
\ \ \ \ \ \ \isacommand{by}\isamarkupfalse%
\ {\isacharparenleft}simp\ only{\isacharcolon}\ singletonD{\isacharparenright}\isanewline
\ \ \ \ \isacommand{then}\isamarkupfalse%
\ \isacommand{show}\isamarkupfalse%
\ {\isacharquery}thesis\isanewline
\ \ \ \ \ \ \isacommand{by}\isamarkupfalse%
\ {\isacharparenleft}simp\ only{\isacharcolon}\ subset{\isacharunderscore}refl{\isacharparenright}\isanewline
\ \ \isacommand{next}\isamarkupfalse%
\isanewline
\ \ \ \ \isacommand{assume}\isamarkupfalse%
\ {\isachardoublequoteopen}G\ {\isasymin}\ setSubformulae\ F{\isadigit{1}}\ {\isasymunion}\ setSubformulae\ F{\isadigit{2}}{\isachardoublequoteclose}\isanewline
\ \ \ \ \isacommand{then}\isamarkupfalse%
\ \isacommand{have}\isamarkupfalse%
\ {\isachardoublequoteopen}G\ {\isasymin}\ setSubformulae\ F{\isadigit{1}}\ {\isasymor}\ G\ {\isasymin}\ setSubformulae\ F{\isadigit{2}}{\isachardoublequoteclose}\ \ \isanewline
\ \ \ \ \ \ \isacommand{by}\isamarkupfalse%
\ {\isacharparenleft}simp\ only{\isacharcolon}\ Un{\isacharunderscore}iff{\isacharparenright}\isanewline
\ \ \ \ \isacommand{then}\isamarkupfalse%
\ \isacommand{show}\isamarkupfalse%
\ {\isacharquery}thesis\isanewline
\ \ \ \ \isacommand{proof}\isamarkupfalse%
\ \isanewline
\ \ \ \ \ \ \isacommand{assume}\isamarkupfalse%
\ {\isachardoublequoteopen}G\ {\isasymin}\ setSubformulae\ F{\isadigit{1}}{\isachardoublequoteclose}\isanewline
\ \ \ \ \ \ \isacommand{then}\isamarkupfalse%
\ \isacommand{have}\isamarkupfalse%
\ {\isachardoublequoteopen}atoms\ G\ {\isasymsubseteq}\ atoms\ F{\isadigit{1}}{\isachardoublequoteclose}\isanewline
\ \ \ \ \ \ \ \ \isacommand{by}\isamarkupfalse%
\ {\isacharparenleft}rule\ assms{\isacharparenleft}{\isadigit{1}}{\isacharparenright}{\isacharparenright}\isanewline
\ \ \ \ \ \ \isacommand{also}\isamarkupfalse%
\ \isacommand{have}\isamarkupfalse%
\ {\isachardoublequoteopen}{\isasymdots}\ {\isasymsubseteq}\ atoms\ F{\isadigit{1}}\ {\isasymunion}\ atoms\ F{\isadigit{2}}{\isachardoublequoteclose}\isanewline
\ \ \ \ \ \ \ \ \isacommand{by}\isamarkupfalse%
\ {\isacharparenleft}simp\ only{\isacharcolon}\ Un{\isacharunderscore}upper{\isadigit{1}}{\isacharparenright}\isanewline
\ \ \ \ \ \ \isacommand{also}\isamarkupfalse%
\ \isacommand{have}\isamarkupfalse%
\ {\isachardoublequoteopen}{\isasymdots}\ {\isacharequal}\ atoms\ {\isacharparenleft}F{\isadigit{1}}\ \isactrlbold {\isasymrightarrow}\ F{\isadigit{2}}{\isacharparenright}{\isachardoublequoteclose}\isanewline
\ \ \ \ \ \ \ \ \isacommand{by}\isamarkupfalse%
\ {\isacharparenleft}simp\ only{\isacharcolon}\ formula{\isachardot}set{\isacharparenleft}{\isadigit{6}}{\isacharparenright}{\isacharparenright}\isanewline
\ \ \ \ \ \ \isacommand{finally}\isamarkupfalse%
\ \isacommand{show}\isamarkupfalse%
\ {\isacharquery}thesis\isanewline
\ \ \ \ \ \ \ \ \isacommand{by}\isamarkupfalse%
\ this\isanewline
\ \ \ \ \isacommand{next}\isamarkupfalse%
\isanewline
\ \ \ \ \ \ \isacommand{assume}\isamarkupfalse%
\ {\isachardoublequoteopen}G\ {\isasymin}\ setSubformulae\ F{\isadigit{2}}{\isachardoublequoteclose}\isanewline
\ \ \ \ \ \ \isacommand{then}\isamarkupfalse%
\ \isacommand{have}\isamarkupfalse%
\ {\isachardoublequoteopen}atoms\ G\ {\isasymsubseteq}\ atoms\ F{\isadigit{2}}{\isachardoublequoteclose}\isanewline
\ \ \ \ \ \ \ \ \isacommand{by}\isamarkupfalse%
\ {\isacharparenleft}rule\ assms{\isacharparenleft}{\isadigit{2}}{\isacharparenright}{\isacharparenright}\isanewline
\ \ \ \ \ \ \isacommand{also}\isamarkupfalse%
\ \isacommand{have}\isamarkupfalse%
\ {\isachardoublequoteopen}{\isasymdots}\ {\isasymsubseteq}\ atoms\ F{\isadigit{1}}\ {\isasymunion}\ atoms\ F{\isadigit{2}}{\isachardoublequoteclose}\isanewline
\ \ \ \ \ \ \ \ \isacommand{by}\isamarkupfalse%
\ {\isacharparenleft}simp\ only{\isacharcolon}\ Un{\isacharunderscore}upper{\isadigit{2}}{\isacharparenright}\isanewline
\ \ \ \ \ \ \isacommand{also}\isamarkupfalse%
\ \isacommand{have}\isamarkupfalse%
\ {\isachardoublequoteopen}{\isasymdots}\ {\isacharequal}\ atoms\ {\isacharparenleft}F{\isadigit{1}}\ \isactrlbold {\isasymrightarrow}\ F{\isadigit{2}}{\isacharparenright}{\isachardoublequoteclose}\isanewline
\ \ \ \ \ \ \ \ \isacommand{by}\isamarkupfalse%
\ {\isacharparenleft}simp\ only{\isacharcolon}\ formula{\isachardot}set{\isacharparenleft}{\isadigit{6}}{\isacharparenright}{\isacharparenright}\isanewline
\ \ \ \ \ \ \isacommand{finally}\isamarkupfalse%
\ \isacommand{show}\isamarkupfalse%
\ {\isacharquery}thesis\isanewline
\ \ \ \ \ \ \ \ \isacommand{by}\isamarkupfalse%
\ this\isanewline
\ \ \ \ \isacommand{qed}\isamarkupfalse%
\isanewline
\ \ \isacommand{qed}\isamarkupfalse%
\isanewline
\isacommand{qed}\isamarkupfalse%
%
\endisatagproof
{\isafoldproof}%
%
\isadelimproof
\isanewline
%
\endisadelimproof
\isanewline
\isacommand{lemma}\isamarkupfalse%
\ subformulae{\isacharunderscore}atoms{\isacharcolon}\ {\isachardoublequoteopen}G\ {\isasymin}\ setSubformulae\ F\ {\isasymLongrightarrow}\ atoms\ G\ {\isasymsubseteq}\ atoms\ F{\isachardoublequoteclose}\isanewline
%
\isadelimproof
%
\endisadelimproof
%
\isatagproof
\isacommand{proof}\isamarkupfalse%
\ {\isacharparenleft}induction\ F{\isacharparenright}\isanewline
\ \ \isacommand{case}\isamarkupfalse%
\ {\isacharparenleft}Atom\ x{\isacharparenright}\isanewline
\ \ \isacommand{then}\isamarkupfalse%
\ \isacommand{show}\isamarkupfalse%
\ {\isacharquery}case\ \isacommand{by}\isamarkupfalse%
\ {\isacharparenleft}simp\ only{\isacharcolon}\ subformulas{\isacharunderscore}atoms{\isacharunderscore}atom{\isacharparenright}\ \isanewline
\isacommand{next}\isamarkupfalse%
\isanewline
\ \ \isacommand{case}\isamarkupfalse%
\ Bot\isanewline
\ \ \isacommand{then}\isamarkupfalse%
\ \isacommand{show}\isamarkupfalse%
\ {\isacharquery}case\ \isacommand{by}\isamarkupfalse%
\ {\isacharparenleft}simp\ only{\isacharcolon}\ subformulas{\isacharunderscore}atoms{\isacharunderscore}bot{\isacharparenright}\isanewline
\isacommand{next}\isamarkupfalse%
\isanewline
\ \ \isacommand{case}\isamarkupfalse%
\ {\isacharparenleft}Not\ F{\isacharparenright}\isanewline
\ \ \isacommand{then}\isamarkupfalse%
\ \isacommand{show}\isamarkupfalse%
\ {\isacharquery}case\ \isacommand{by}\isamarkupfalse%
\ {\isacharparenleft}simp\ only{\isacharcolon}\ subformulas{\isacharunderscore}atoms{\isacharunderscore}not{\isacharparenright}\isanewline
\isacommand{next}\isamarkupfalse%
\isanewline
\ \ \isacommand{case}\isamarkupfalse%
\ {\isacharparenleft}And\ F{\isadigit{1}}\ F{\isadigit{2}}{\isacharparenright}\isanewline
\ \ \isacommand{then}\isamarkupfalse%
\ \isacommand{show}\isamarkupfalse%
\ {\isacharquery}case\ \isacommand{by}\isamarkupfalse%
\ {\isacharparenleft}simp\ only{\isacharcolon}\ subformulas{\isacharunderscore}atoms{\isacharunderscore}and{\isacharparenright}\isanewline
\isacommand{next}\isamarkupfalse%
\isanewline
\ \ \isacommand{case}\isamarkupfalse%
\ {\isacharparenleft}Or\ F{\isadigit{1}}\ F{\isadigit{2}}{\isacharparenright}\isanewline
\ \ \isacommand{then}\isamarkupfalse%
\ \isacommand{show}\isamarkupfalse%
\ {\isacharquery}case\ \isacommand{by}\isamarkupfalse%
\ {\isacharparenleft}simp\ only{\isacharcolon}\ subformulas{\isacharunderscore}atoms{\isacharunderscore}or{\isacharparenright}\isanewline
\isacommand{next}\isamarkupfalse%
\isanewline
\ \ \isacommand{case}\isamarkupfalse%
\ {\isacharparenleft}Imp\ F{\isadigit{1}}\ F{\isadigit{2}}{\isacharparenright}\isanewline
\ \ \isacommand{then}\isamarkupfalse%
\ \isacommand{show}\isamarkupfalse%
\ {\isacharquery}case\ \isacommand{by}\isamarkupfalse%
\ {\isacharparenleft}simp\ only{\isacharcolon}\ subformulas{\isacharunderscore}atoms{\isacharunderscore}imp{\isacharparenright}\isanewline
\isacommand{qed}\isamarkupfalse%
%
\endisatagproof
{\isafoldproof}%
%
\isadelimproof
%
\endisadelimproof
%
\begin{isamarkuptext}%
Por último, su demostración aplicativa automática.%
\end{isamarkuptext}\isamarkuptrue%
\isacommand{lemma}\isamarkupfalse%
\ {\isachardoublequoteopen}G\ {\isasymin}\ setSubformulae\ F\ {\isasymLongrightarrow}\ atoms\ G\ {\isasymsubseteq}\ atoms\ F{\isachardoublequoteclose}\isanewline
%
\isadelimproof
\ \ %
\endisadelimproof
%
\isatagproof
\isacommand{by}\isamarkupfalse%
\ {\isacharparenleft}induction\ F{\isacharparenright}\ auto%
\endisatagproof
{\isafoldproof}%
%
\isadelimproof
%
\endisadelimproof
%
\begin{isamarkuptext}%
A continuación voy a introducir un lema que no pertenece a la 
  teoría original de Isabelle pero facilita las siguientes 
  demostraciones detalladas mediante contenciones en cadena.

  \begin{lema}
    Sea \isa{G} subfórmula de \isa{F}, entonces el conjunto de subfórmulas de 
    \isa{G} está contenido en el de \isa{F}.
  \end{lema} 

  \begin{demostracion}
  La prueba es por inducción en la estructura de fórmula.
  
  Sea \isa{p} una fórmula atómica cualquiera. Entonces, bajo las
  condiciones del lema se tiene que \isa{G\ {\isacharequal}\ p}. Por lo tanto, tienen igual
  conjunto de subfórmulas.

  Sea la fórmula \isa{{\isasymbottom}}. Entonces, \isa{G\ {\isacharequal}\ {\isasymbottom}} y tienen igual conjunto de
  subfórmulas.

  Sea una fórmula \isa{F} tal que para toda subfórmula \isa{G}, se tiene que el
  conjunto de subfórmulas de \isa{G} está contenido en el de \isa{F}. Veamos la
  propiedad para \isa{{\isasymnot}\ F}. Sea \isa{G{\isacharprime}\ {\isasymin}\ Subf{\isacharparenleft}{\isasymnot}\ F{\isacharparenright}\ {\isacharequal}\ {\isacharbraceleft}{\isasymnot}\ F{\isacharbraceright}\ {\isasymunion}\ Subf{\isacharparenleft}F{\isacharparenright}}. 
  Entonces \isa{G{\isacharprime}\ {\isasymin}\ {\isacharbraceleft}{\isasymnot}\ F{\isacharbraceright}} o \isa{G{\isacharprime}\ {\isasymin}\ Subf{\isacharparenleft}F{\isacharparenright}}. 
  Del primer caso se obtiene que \isa{G{\isacharprime}\ {\isacharequal}\ {\isasymnot}\ F} y, por tanto, tienen igual 
  conjunto de subfórmulas. Del segundo caso se tiene \isa{G{\isacharprime}\ {\isasymin}\ Subf{\isacharparenleft}F{\isacharparenright}} y, 
  por hipótesis de inducción, el conjunto de subfórmulas de \isa{G{\isacharprime}} está 
  contenido en el de \isa{F}. Como, a su vez, el conjunto de subfórmulas 
  de \isa{F} está contenido en el de \isa{{\isasymnot}\ F} por definición, se verifica la
  propiedad como consecuencia de la transitividad de la contención.

  Sean las fórmulas \isa{F{\isadigit{1}}} y \isa{F{\isadigit{2}}} tales que para cualquier subfórmula \isa{G{\isadigit{1}}}
  de \isa{F{\isadigit{1}}} el conjunto de subfórmulas de \isa{G{\isadigit{1}}} está contenido en el de 
  \isa{F{\isadigit{1}}}, y para cualquier subfórmula \isa{G{\isadigit{2}}} de \isa{F{\isadigit{2}}} el conjunto de 
  subfórmulas de \isa{G{\isadigit{2}}} está contenido en el de \isa{F{\isadigit{2}}}. Veamos que se 
  verifica la propiedad para \isa{F{\isadigit{1}}{\isacharasterisk}F{\isadigit{2}}} donde \isa{{\isacharasterisk}} es cualquier conectiva 
  binaria. 
  Sea \isa{G{\isacharprime}\ {\isasymin}\ Subf{\isacharparenleft}F{\isadigit{1}}{\isacharasterisk}F{\isadigit{2}}{\isacharparenright}\ {\isacharequal}\ {\isacharbraceleft}F{\isadigit{1}}{\isacharasterisk}F{\isadigit{2}}{\isacharbraceright}\ {\isasymunion}\ Subf{\isacharparenleft}F{\isadigit{1}}{\isacharparenright}\ {\isasymunion}\ Subf{\isacharparenleft}F{\isadigit{2}}{\isacharparenright}}. De este modo,
  tenemos tres casos: \isa{G{\isacharprime}\ {\isasymin}\ {\isacharbraceleft}F{\isadigit{1}}{\isacharasterisk}F{\isadigit{2}}{\isacharbraceright}} o \isa{G{\isacharprime}\ {\isasymin}\ Subf{\isacharparenleft}F{\isadigit{1}}{\isacharparenright}} o 
  \isa{G{\isacharprime}\ {\isasymin}\ Subf{\isacharparenleft}F{\isadigit{2}}{\isacharparenright}}. De la primera opción se deduce \isa{G{\isacharprime}\ {\isacharequal}\ F{\isadigit{1}}{\isacharasterisk}F{\isadigit{2}}} y, por
  tanto, tienen igual conjunto de subfórmulas. Por otro lado, si 
  \isa{G{\isacharprime}\ {\isasymin}\ Subf{\isacharparenleft}F{\isadigit{1}}{\isacharparenright}}, por hipótesis de inducción se tiene que el conjunto
  de subfórmulas de \isa{G{\isacharprime}} está contenido en el de \isa{F{\isadigit{1}}}. Por tanto, 
  como el conjunto de subfórmulas de \isa{F{\isadigit{1}}} está a su vez contenido en el 
  de \isa{F{\isadigit{1}}{\isacharasterisk}F{\isadigit{2}}}, se verifica la propiedad por la transitividad de la 
  contención en cadena. El caso \isa{G{\isacharprime}\ {\isasymin}\ Subf{\isacharparenleft}F{\isadigit{2}}{\isacharparenright}} es análogo cambiando el 
  índice de la fórmula.   
  \end{demostracion}%
\end{isamarkuptext}\isamarkuptrue%
%
\begin{isamarkuptext}%
Veamos su formalización en Isabelle junto con su demostración 
  estructurada.%
\end{isamarkuptext}\isamarkuptrue%
\isacommand{lemma}\isamarkupfalse%
\ subContsubformulae{\isacharunderscore}atom{\isacharcolon}\ \isanewline
\ \ \isakeyword{assumes}\ {\isachardoublequoteopen}G\ {\isasymin}\ setSubformulae\ {\isacharparenleft}Atom\ x{\isacharparenright}{\isachardoublequoteclose}\ \isanewline
\ \ \isakeyword{shows}\ {\isachardoublequoteopen}setSubformulae\ G\ {\isasymsubseteq}\ setSubformulae\ {\isacharparenleft}Atom\ x{\isacharparenright}{\isachardoublequoteclose}\isanewline
%
\isadelimproof
%
\endisadelimproof
%
\isatagproof
\isacommand{proof}\isamarkupfalse%
\ {\isacharminus}\ \isanewline
\ \ \isacommand{have}\isamarkupfalse%
\ {\isachardoublequoteopen}G\ {\isasymin}\ {\isacharbraceleft}Atom\ x{\isacharbraceright}{\isachardoublequoteclose}\ \isacommand{using}\isamarkupfalse%
\ assms\ \isanewline
\ \ \ \ \isacommand{by}\isamarkupfalse%
\ {\isacharparenleft}simp\ only{\isacharcolon}\ setSubformulae{\isacharunderscore}atom{\isacharparenright}\isanewline
\ \ \isacommand{then}\isamarkupfalse%
\ \isacommand{have}\isamarkupfalse%
\ {\isachardoublequoteopen}G\ {\isacharequal}\ Atom\ x{\isachardoublequoteclose}\isanewline
\ \ \ \ \isacommand{by}\isamarkupfalse%
\ {\isacharparenleft}simp\ only{\isacharcolon}\ singletonD{\isacharparenright}\isanewline
\ \ \isacommand{then}\isamarkupfalse%
\ \isacommand{show}\isamarkupfalse%
\ {\isacharquery}thesis\isanewline
\ \ \ \ \isacommand{by}\isamarkupfalse%
\ {\isacharparenleft}simp\ only{\isacharcolon}\ subset{\isacharunderscore}refl{\isacharparenright}\isanewline
\isacommand{qed}\isamarkupfalse%
%
\endisatagproof
{\isafoldproof}%
%
\isadelimproof
\isanewline
%
\endisadelimproof
\isanewline
\isacommand{lemma}\isamarkupfalse%
\ subContsubformulae{\isacharunderscore}bot{\isacharcolon}\isanewline
\ \ \isakeyword{assumes}\ {\isachardoublequoteopen}G\ {\isasymin}\ setSubformulae\ {\isasymbottom}{\isachardoublequoteclose}\ \isanewline
\ \ \isakeyword{shows}\ \ \ {\isachardoublequoteopen}setSubformulae\ G\ {\isasymsubseteq}\ setSubformulae\ {\isasymbottom}{\isachardoublequoteclose}\isanewline
%
\isadelimproof
%
\endisadelimproof
%
\isatagproof
\isacommand{proof}\isamarkupfalse%
\ {\isacharminus}\isanewline
\ \ \isacommand{have}\isamarkupfalse%
\ {\isachardoublequoteopen}G\ {\isasymin}\ {\isacharbraceleft}{\isasymbottom}{\isacharbraceright}{\isachardoublequoteclose}\isanewline
\ \ \ \ \isacommand{using}\isamarkupfalse%
\ assms\isanewline
\ \ \ \ \isacommand{by}\isamarkupfalse%
\ {\isacharparenleft}simp\ only{\isacharcolon}\ setSubformulae{\isacharunderscore}bot{\isacharparenright}\isanewline
\ \ \isacommand{then}\isamarkupfalse%
\ \isacommand{have}\isamarkupfalse%
\ {\isachardoublequoteopen}G\ {\isacharequal}\ {\isasymbottom}{\isachardoublequoteclose}\isanewline
\ \ \ \ \isacommand{by}\isamarkupfalse%
\ {\isacharparenleft}simp\ only{\isacharcolon}\ singletonD{\isacharparenright}\isanewline
\ \ \isacommand{then}\isamarkupfalse%
\ \isacommand{show}\isamarkupfalse%
\ {\isacharquery}thesis\isanewline
\ \ \ \ \isacommand{by}\isamarkupfalse%
\ {\isacharparenleft}simp\ only{\isacharcolon}\ subset{\isacharunderscore}refl{\isacharparenright}\isanewline
\isacommand{qed}\isamarkupfalse%
%
\endisatagproof
{\isafoldproof}%
%
\isadelimproof
\isanewline
%
\endisadelimproof
\isanewline
\isacommand{lemma}\isamarkupfalse%
\ subContsubformulae{\isacharunderscore}not{\isacharcolon}\isanewline
\ \ \isakeyword{assumes}\ {\isachardoublequoteopen}G\ {\isasymin}\ setSubformulae\ F\ {\isasymLongrightarrow}\ setSubformulae\ G\ {\isasymsubseteq}\ setSubformulae\ F{\isachardoublequoteclose}\isanewline
\ \ \ \ \ \ \ \ \ \ {\isachardoublequoteopen}G\ {\isasymin}\ setSubformulae\ {\isacharparenleft}\isactrlbold {\isasymnot}\ F{\isacharparenright}{\isachardoublequoteclose}\isanewline
\ \ \isakeyword{shows}\ \ \ {\isachardoublequoteopen}setSubformulae\ G\ {\isasymsubseteq}\ setSubformulae\ {\isacharparenleft}\isactrlbold {\isasymnot}\ F{\isacharparenright}{\isachardoublequoteclose}\isanewline
%
\isadelimproof
%
\endisadelimproof
%
\isatagproof
\isacommand{proof}\isamarkupfalse%
\ {\isacharminus}\isanewline
\ \ \isacommand{have}\isamarkupfalse%
\ {\isachardoublequoteopen}G\ {\isasymin}\ {\isacharbraceleft}\isactrlbold {\isasymnot}\ F{\isacharbraceright}\ {\isasymunion}\ setSubformulae\ F{\isachardoublequoteclose}\isanewline
\ \ \ \ \isacommand{using}\isamarkupfalse%
\ assms{\isacharparenleft}{\isadigit{2}}{\isacharparenright}\isanewline
\ \ \ \ \isacommand{by}\isamarkupfalse%
\ {\isacharparenleft}simp\ only{\isacharcolon}\ setSubformulae{\isacharunderscore}not{\isacharparenright}\ \isanewline
\ \ \isacommand{then}\isamarkupfalse%
\ \isacommand{have}\isamarkupfalse%
\ {\isachardoublequoteopen}G\ {\isasymin}\ {\isacharbraceleft}\isactrlbold {\isasymnot}\ F{\isacharbraceright}\ {\isasymor}\ G\ {\isasymin}\ setSubformulae\ F{\isachardoublequoteclose}\isanewline
\ \ \ \ \isacommand{by}\isamarkupfalse%
\ {\isacharparenleft}simp\ only{\isacharcolon}\ Un{\isacharunderscore}iff{\isacharparenright}\isanewline
\ \ \isacommand{then}\isamarkupfalse%
\ \isacommand{show}\isamarkupfalse%
\ {\isachardoublequoteopen}setSubformulae\ G\ {\isasymsubseteq}\ setSubformulae\ {\isacharparenleft}\isactrlbold {\isasymnot}\ F{\isacharparenright}{\isachardoublequoteclose}\isanewline
\ \ \isacommand{proof}\isamarkupfalse%
\isanewline
\ \ \ \ \isacommand{assume}\isamarkupfalse%
\ {\isachardoublequoteopen}G\ {\isasymin}\ {\isacharbraceleft}\isactrlbold {\isasymnot}\ F{\isacharbraceright}{\isachardoublequoteclose}\isanewline
\ \ \ \ \isacommand{then}\isamarkupfalse%
\ \isacommand{have}\isamarkupfalse%
\ {\isachardoublequoteopen}G\ {\isacharequal}\ \isactrlbold {\isasymnot}\ F{\isachardoublequoteclose}\isanewline
\ \ \ \ \ \ \isacommand{by}\isamarkupfalse%
\ {\isacharparenleft}simp\ only{\isacharcolon}\ singletonD{\isacharparenright}\isanewline
\ \ \ \ \isacommand{then}\isamarkupfalse%
\ \isacommand{show}\isamarkupfalse%
\ {\isacharquery}thesis\isanewline
\ \ \ \ \ \ \isacommand{by}\isamarkupfalse%
\ {\isacharparenleft}simp\ only{\isacharcolon}\ subset{\isacharunderscore}refl{\isacharparenright}\isanewline
\ \ \isacommand{next}\isamarkupfalse%
\isanewline
\ \ \ \ \isacommand{assume}\isamarkupfalse%
\ {\isachardoublequoteopen}G\ {\isasymin}\ setSubformulae\ F{\isachardoublequoteclose}\isanewline
\ \ \ \ \isacommand{then}\isamarkupfalse%
\ \isacommand{have}\isamarkupfalse%
\ {\isadigit{1}}{\isacharcolon}{\isachardoublequoteopen}setSubformulae\ G\ {\isasymsubseteq}\ setSubformulae\ F{\isachardoublequoteclose}\isanewline
\ \ \ \ \ \ \isacommand{by}\isamarkupfalse%
\ {\isacharparenleft}simp\ only{\isacharcolon}\ assms{\isacharparenleft}{\isadigit{1}}{\isacharparenright}{\isacharparenright}\isanewline
\ \ \ \ \isacommand{also}\isamarkupfalse%
\ \isacommand{have}\isamarkupfalse%
\ {\isadigit{2}}{\isacharcolon}{\isachardoublequoteopen}setSubformulae\ F\ {\isasymsubseteq}\ setSubformulae\ {\isacharparenleft}\isactrlbold {\isasymnot}\ F{\isacharparenright}{\isachardoublequoteclose}\isanewline
\ \ \ \ \ \ \isacommand{by}\isamarkupfalse%
\ {\isacharparenleft}simp\ only{\isacharcolon}\ setSubformulae{\isacharunderscore}not\ Un{\isacharunderscore}upper{\isadigit{2}}{\isacharparenright}\isanewline
\ \ \ \ \isacommand{finally}\isamarkupfalse%
\ \isacommand{show}\isamarkupfalse%
\ {\isacharquery}thesis\isanewline
\ \ \ \ \ \ \isacommand{using}\isamarkupfalse%
\ {\isadigit{1}}\ {\isadigit{2}}\ \isacommand{by}\isamarkupfalse%
\ {\isacharparenleft}simp\ only{\isacharcolon}\ subset{\isacharunderscore}trans{\isacharparenright}\isanewline
\ \ \isacommand{qed}\isamarkupfalse%
\isanewline
\isacommand{qed}\isamarkupfalse%
%
\endisatagproof
{\isafoldproof}%
%
\isadelimproof
\isanewline
%
\endisadelimproof
\isanewline
\isacommand{lemma}\isamarkupfalse%
\ subContsubformulae{\isacharunderscore}and{\isacharcolon}\isanewline
\ \ \isakeyword{assumes}\ {\isachardoublequoteopen}G\ {\isasymin}\ setSubformulae\ F{\isadigit{1}}\ \isanewline
\ \ \ \ \ \ \ \ \ \ \ \ {\isasymLongrightarrow}\ setSubformulae\ G\ {\isasymsubseteq}\ setSubformulae\ F{\isadigit{1}}{\isachardoublequoteclose}\isanewline
\ \ \ \ \ \ \ \ \ \ {\isachardoublequoteopen}G\ {\isasymin}\ setSubformulae\ F{\isadigit{2}}\ \isanewline
\ \ \ \ \ \ \ \ \ \ \ \ {\isasymLongrightarrow}\ setSubformulae\ G\ {\isasymsubseteq}\ setSubformulae\ F{\isadigit{2}}{\isachardoublequoteclose}\isanewline
\ \ \ \ \ \ \ \ \ \ {\isachardoublequoteopen}G\ {\isasymin}\ setSubformulae\ {\isacharparenleft}F{\isadigit{1}}\ \isactrlbold {\isasymand}\ F{\isadigit{2}}{\isacharparenright}{\isachardoublequoteclose}\isanewline
\ \ \isakeyword{shows}\ \ \ {\isachardoublequoteopen}setSubformulae\ G\ {\isasymsubseteq}\ setSubformulae\ {\isacharparenleft}F{\isadigit{1}}\ \isactrlbold {\isasymand}\ F{\isadigit{2}}{\isacharparenright}{\isachardoublequoteclose}\isanewline
%
\isadelimproof
%
\endisadelimproof
%
\isatagproof
\isacommand{proof}\isamarkupfalse%
\ {\isacharminus}\isanewline
\ \ \isacommand{have}\isamarkupfalse%
\ {\isachardoublequoteopen}G\ {\isasymin}\ {\isacharbraceleft}F{\isadigit{1}}\ \isactrlbold {\isasymand}\ F{\isadigit{2}}{\isacharbraceright}\ {\isasymunion}\ {\isacharparenleft}setSubformulae\ F{\isadigit{1}}\ {\isasymunion}\ setSubformulae\ F{\isadigit{2}}{\isacharparenright}{\isachardoublequoteclose}\isanewline
\ \ \ \ \isacommand{using}\isamarkupfalse%
\ assms{\isacharparenleft}{\isadigit{3}}{\isacharparenright}\ \isanewline
\ \ \ \ \isacommand{by}\isamarkupfalse%
\ {\isacharparenleft}simp\ only{\isacharcolon}\ setSubformulae{\isacharunderscore}and{\isacharparenright}\isanewline
\ \ \isacommand{then}\isamarkupfalse%
\ \isacommand{have}\isamarkupfalse%
\ {\isachardoublequoteopen}G\ {\isasymin}\ {\isacharbraceleft}F{\isadigit{1}}\ \isactrlbold {\isasymand}\ F{\isadigit{2}}{\isacharbraceright}\ {\isasymor}\ G\ {\isasymin}\ setSubformulae\ F{\isadigit{1}}\ {\isasymunion}\ setSubformulae\ F{\isadigit{2}}{\isachardoublequoteclose}\isanewline
\ \ \ \ \isacommand{by}\isamarkupfalse%
\ {\isacharparenleft}simp\ only{\isacharcolon}\ Un{\isacharunderscore}iff{\isacharparenright}\isanewline
\ \ \isacommand{then}\isamarkupfalse%
\ \isacommand{show}\isamarkupfalse%
\ {\isacharquery}thesis\isanewline
\ \ \isacommand{proof}\isamarkupfalse%
\ \isanewline
\ \ \ \ \isacommand{assume}\isamarkupfalse%
\ {\isachardoublequoteopen}G\ {\isasymin}\ {\isacharbraceleft}F{\isadigit{1}}\ \isactrlbold {\isasymand}\ F{\isadigit{2}}{\isacharbraceright}{\isachardoublequoteclose}\isanewline
\ \ \ \ \isacommand{then}\isamarkupfalse%
\ \isacommand{have}\isamarkupfalse%
\ {\isachardoublequoteopen}G\ {\isacharequal}\ F{\isadigit{1}}\ \isactrlbold {\isasymand}\ F{\isadigit{2}}{\isachardoublequoteclose}\isanewline
\ \ \ \ \ \ \isacommand{by}\isamarkupfalse%
\ {\isacharparenleft}simp\ only{\isacharcolon}\ singletonD{\isacharparenright}\isanewline
\ \ \ \ \isacommand{then}\isamarkupfalse%
\ \isacommand{show}\isamarkupfalse%
\ {\isacharquery}thesis\isanewline
\ \ \ \ \ \ \isacommand{by}\isamarkupfalse%
\ {\isacharparenleft}simp\ only{\isacharcolon}\ subset{\isacharunderscore}refl{\isacharparenright}\isanewline
\ \ \isacommand{next}\isamarkupfalse%
\isanewline
\ \ \ \ \isacommand{assume}\isamarkupfalse%
\ {\isachardoublequoteopen}G\ {\isasymin}\ setSubformulae\ F{\isadigit{1}}\ {\isasymunion}\ setSubformulae\ F{\isadigit{2}}{\isachardoublequoteclose}\isanewline
\ \ \ \ \isacommand{then}\isamarkupfalse%
\ \isacommand{have}\isamarkupfalse%
\ {\isachardoublequoteopen}G\ {\isasymin}\ setSubformulae\ F{\isadigit{1}}\ {\isasymor}\ G\ {\isasymin}\ setSubformulae\ F{\isadigit{2}}{\isachardoublequoteclose}\ \ \isanewline
\ \ \ \ \ \ \isacommand{by}\isamarkupfalse%
\ {\isacharparenleft}simp\ only{\isacharcolon}\ Un{\isacharunderscore}iff{\isacharparenright}\isanewline
\ \ \ \ \isacommand{then}\isamarkupfalse%
\ \isacommand{show}\isamarkupfalse%
\ {\isacharquery}thesis\isanewline
\ \ \ \ \isacommand{proof}\isamarkupfalse%
\ \isanewline
\ \ \ \ \ \ \isacommand{assume}\isamarkupfalse%
\ {\isachardoublequoteopen}G\ {\isasymin}\ setSubformulae\ F{\isadigit{1}}{\isachardoublequoteclose}\isanewline
\ \ \ \ \ \ \isacommand{then}\isamarkupfalse%
\ \isacommand{have}\isamarkupfalse%
\ {\isachardoublequoteopen}setSubformulae\ G\ {\isasymsubseteq}\ setSubformulae\ F{\isadigit{1}}{\isachardoublequoteclose}\isanewline
\ \ \ \ \ \ \ \ \isacommand{by}\isamarkupfalse%
\ {\isacharparenleft}simp\ only{\isacharcolon}\ assms{\isacharparenleft}{\isadigit{1}}{\isacharparenright}{\isacharparenright}\isanewline
\ \ \ \ \ \ \isacommand{also}\isamarkupfalse%
\ \isacommand{have}\isamarkupfalse%
\ {\isachardoublequoteopen}{\isasymdots}\ {\isasymsubseteq}\ setSubformulae\ F{\isadigit{1}}\ {\isasymunion}\ setSubformulae\ F{\isadigit{2}}{\isachardoublequoteclose}\isanewline
\ \ \ \ \ \ \ \ \isacommand{by}\isamarkupfalse%
\ {\isacharparenleft}simp\ only{\isacharcolon}\ Un{\isacharunderscore}upper{\isadigit{1}}{\isacharparenright}\isanewline
\ \ \ \ \ \ \isacommand{also}\isamarkupfalse%
\ \isacommand{have}\isamarkupfalse%
\ {\isachardoublequoteopen}{\isasymdots}\ {\isasymsubseteq}\ setSubformulae\ {\isacharparenleft}F{\isadigit{1}}\ \isactrlbold {\isasymand}\ F{\isadigit{2}}{\isacharparenright}{\isachardoublequoteclose}\isanewline
\ \ \ \ \ \ \ \ \isacommand{by}\isamarkupfalse%
\ {\isacharparenleft}simp\ only{\isacharcolon}\ setSubformulae{\isacharunderscore}and\ Un{\isacharunderscore}upper{\isadigit{2}}{\isacharparenright}\isanewline
\ \ \ \ \ \ \isacommand{finally}\isamarkupfalse%
\ \isacommand{show}\isamarkupfalse%
\ {\isacharquery}thesis\isanewline
\ \ \ \ \ \ \ \ \isacommand{by}\isamarkupfalse%
\ this\isanewline
\ \ \ \ \isacommand{next}\isamarkupfalse%
\isanewline
\ \ \ \ \ \ \isacommand{assume}\isamarkupfalse%
\ {\isachardoublequoteopen}G\ {\isasymin}\ setSubformulae\ F{\isadigit{2}}{\isachardoublequoteclose}\isanewline
\ \ \ \ \ \ \isacommand{then}\isamarkupfalse%
\ \isacommand{have}\isamarkupfalse%
\ {\isachardoublequoteopen}setSubformulae\ G\ {\isasymsubseteq}\ setSubformulae\ F{\isadigit{2}}{\isachardoublequoteclose}\isanewline
\ \ \ \ \ \ \ \ \isacommand{by}\isamarkupfalse%
\ {\isacharparenleft}rule\ assms{\isacharparenleft}{\isadigit{2}}{\isacharparenright}{\isacharparenright}\isanewline
\ \ \ \ \ \ \isacommand{also}\isamarkupfalse%
\ \isacommand{have}\isamarkupfalse%
\ {\isachardoublequoteopen}{\isasymdots}\ {\isasymsubseteq}\ setSubformulae\ F{\isadigit{1}}\ {\isasymunion}\ setSubformulae\ F{\isadigit{2}}{\isachardoublequoteclose}\isanewline
\ \ \ \ \ \ \ \ \isacommand{by}\isamarkupfalse%
\ {\isacharparenleft}simp\ only{\isacharcolon}\ Un{\isacharunderscore}upper{\isadigit{2}}{\isacharparenright}\isanewline
\ \ \ \ \ \ \isacommand{also}\isamarkupfalse%
\ \isacommand{have}\isamarkupfalse%
\ {\isachardoublequoteopen}{\isasymdots}\ {\isasymsubseteq}\ setSubformulae\ {\isacharparenleft}F{\isadigit{1}}\ \isactrlbold {\isasymand}\ F{\isadigit{2}}{\isacharparenright}{\isachardoublequoteclose}\isanewline
\ \ \ \ \ \ \ \ \isacommand{by}\isamarkupfalse%
\ {\isacharparenleft}simp\ only{\isacharcolon}\ setSubformulae{\isacharunderscore}and\ Un{\isacharunderscore}upper{\isadigit{2}}{\isacharparenright}\isanewline
\ \ \ \ \ \ \isacommand{finally}\isamarkupfalse%
\ \isacommand{show}\isamarkupfalse%
\ {\isacharquery}thesis\isanewline
\ \ \ \ \ \ \ \ \isacommand{by}\isamarkupfalse%
\ this\isanewline
\ \ \ \ \isacommand{qed}\isamarkupfalse%
\isanewline
\ \ \isacommand{qed}\isamarkupfalse%
\isanewline
\isacommand{qed}\isamarkupfalse%
%
\endisatagproof
{\isafoldproof}%
%
\isadelimproof
\isanewline
%
\endisadelimproof
\isanewline
\isacommand{lemma}\isamarkupfalse%
\ subContsubformulae{\isacharunderscore}or{\isacharcolon}\isanewline
\ \ \isakeyword{assumes}\ {\isachardoublequoteopen}G\ {\isasymin}\ setSubformulae\ F{\isadigit{1}}\ \isanewline
\ \ \ \ \ \ \ \ \ \ \ \ {\isasymLongrightarrow}\ setSubformulae\ G\ {\isasymsubseteq}\ setSubformulae\ F{\isadigit{1}}{\isachardoublequoteclose}\isanewline
\ \ \ \ \ \ \ \ \ \ {\isachardoublequoteopen}G\ {\isasymin}\ setSubformulae\ F{\isadigit{2}}\ \isanewline
\ \ \ \ \ \ \ \ \ \ \ \ {\isasymLongrightarrow}\ setSubformulae\ G\ {\isasymsubseteq}\ setSubformulae\ F{\isadigit{2}}{\isachardoublequoteclose}\isanewline
\ \ \ \ \ \ \ \ \ \ {\isachardoublequoteopen}G\ {\isasymin}\ setSubformulae\ {\isacharparenleft}F{\isadigit{1}}\ \isactrlbold {\isasymor}\ F{\isadigit{2}}{\isacharparenright}{\isachardoublequoteclose}\isanewline
\ \ \isakeyword{shows}\ \ \ {\isachardoublequoteopen}setSubformulae\ G\ {\isasymsubseteq}\ setSubformulae\ {\isacharparenleft}F{\isadigit{1}}\ \isactrlbold {\isasymor}\ F{\isadigit{2}}{\isacharparenright}{\isachardoublequoteclose}\isanewline
%
\isadelimproof
%
\endisadelimproof
%
\isatagproof
\isacommand{proof}\isamarkupfalse%
\ {\isacharminus}\isanewline
\ \ \isacommand{have}\isamarkupfalse%
\ {\isachardoublequoteopen}G\ {\isasymin}\ {\isacharbraceleft}F{\isadigit{1}}\ \isactrlbold {\isasymor}\ F{\isadigit{2}}{\isacharbraceright}\ {\isasymunion}\ {\isacharparenleft}setSubformulae\ F{\isadigit{1}}\ {\isasymunion}\ setSubformulae\ F{\isadigit{2}}{\isacharparenright}{\isachardoublequoteclose}\isanewline
\ \ \ \ \isacommand{using}\isamarkupfalse%
\ assms{\isacharparenleft}{\isadigit{3}}{\isacharparenright}\ \isanewline
\ \ \ \ \isacommand{by}\isamarkupfalse%
\ {\isacharparenleft}simp\ only{\isacharcolon}\ setSubformulae{\isacharunderscore}or{\isacharparenright}\isanewline
\ \ \isacommand{then}\isamarkupfalse%
\ \isacommand{have}\isamarkupfalse%
\ {\isachardoublequoteopen}G\ {\isasymin}\ {\isacharbraceleft}F{\isadigit{1}}\ \isactrlbold {\isasymor}\ F{\isadigit{2}}{\isacharbraceright}\ {\isasymor}\ G\ {\isasymin}\ setSubformulae\ F{\isadigit{1}}\ {\isasymunion}\ setSubformulae\ F{\isadigit{2}}{\isachardoublequoteclose}\isanewline
\ \ \ \ \isacommand{by}\isamarkupfalse%
\ {\isacharparenleft}simp\ only{\isacharcolon}\ Un{\isacharunderscore}iff{\isacharparenright}\isanewline
\ \ \isacommand{then}\isamarkupfalse%
\ \isacommand{show}\isamarkupfalse%
\ {\isacharquery}thesis\isanewline
\ \ \isacommand{proof}\isamarkupfalse%
\ \isanewline
\ \ \ \ \isacommand{assume}\isamarkupfalse%
\ {\isachardoublequoteopen}G\ {\isasymin}\ {\isacharbraceleft}F{\isadigit{1}}\ \isactrlbold {\isasymor}\ F{\isadigit{2}}{\isacharbraceright}{\isachardoublequoteclose}\isanewline
\ \ \ \ \isacommand{then}\isamarkupfalse%
\ \isacommand{have}\isamarkupfalse%
\ {\isachardoublequoteopen}G\ {\isacharequal}\ F{\isadigit{1}}\ \isactrlbold {\isasymor}\ F{\isadigit{2}}{\isachardoublequoteclose}\isanewline
\ \ \ \ \ \ \isacommand{by}\isamarkupfalse%
\ {\isacharparenleft}simp\ only{\isacharcolon}\ singletonD{\isacharparenright}\isanewline
\ \ \ \ \isacommand{then}\isamarkupfalse%
\ \isacommand{show}\isamarkupfalse%
\ {\isacharquery}thesis\isanewline
\ \ \ \ \ \ \isacommand{by}\isamarkupfalse%
\ {\isacharparenleft}simp\ only{\isacharcolon}\ subset{\isacharunderscore}refl{\isacharparenright}\isanewline
\ \ \isacommand{next}\isamarkupfalse%
\isanewline
\ \ \ \ \isacommand{assume}\isamarkupfalse%
\ {\isachardoublequoteopen}G\ {\isasymin}\ setSubformulae\ F{\isadigit{1}}\ {\isasymunion}\ setSubformulae\ F{\isadigit{2}}{\isachardoublequoteclose}\isanewline
\ \ \ \ \isacommand{then}\isamarkupfalse%
\ \isacommand{have}\isamarkupfalse%
\ {\isachardoublequoteopen}G\ {\isasymin}\ setSubformulae\ F{\isadigit{1}}\ {\isasymor}\ G\ {\isasymin}\ setSubformulae\ F{\isadigit{2}}{\isachardoublequoteclose}\ \ \isanewline
\ \ \ \ \ \ \isacommand{by}\isamarkupfalse%
\ {\isacharparenleft}simp\ only{\isacharcolon}\ Un{\isacharunderscore}iff{\isacharparenright}\isanewline
\ \ \ \ \isacommand{then}\isamarkupfalse%
\ \isacommand{show}\isamarkupfalse%
\ {\isacharquery}thesis\isanewline
\ \ \ \ \isacommand{proof}\isamarkupfalse%
\ \isanewline
\ \ \ \ \ \ \isacommand{assume}\isamarkupfalse%
\ {\isachardoublequoteopen}G\ {\isasymin}\ setSubformulae\ F{\isadigit{1}}{\isachardoublequoteclose}\isanewline
\ \ \ \ \ \ \isacommand{then}\isamarkupfalse%
\ \isacommand{have}\isamarkupfalse%
\ {\isachardoublequoteopen}setSubformulae\ G\ {\isasymsubseteq}\ setSubformulae\ F{\isadigit{1}}{\isachardoublequoteclose}\isanewline
\ \ \ \ \ \ \ \ \isacommand{by}\isamarkupfalse%
\ {\isacharparenleft}simp\ only{\isacharcolon}\ assms{\isacharparenleft}{\isadigit{1}}{\isacharparenright}{\isacharparenright}\isanewline
\ \ \ \ \ \ \isacommand{also}\isamarkupfalse%
\ \isacommand{have}\isamarkupfalse%
\ {\isachardoublequoteopen}{\isasymdots}\ {\isasymsubseteq}\ setSubformulae\ F{\isadigit{1}}\ {\isasymunion}\ setSubformulae\ F{\isadigit{2}}{\isachardoublequoteclose}\isanewline
\ \ \ \ \ \ \ \ \isacommand{by}\isamarkupfalse%
\ {\isacharparenleft}simp\ only{\isacharcolon}\ Un{\isacharunderscore}upper{\isadigit{1}}{\isacharparenright}\isanewline
\ \ \ \ \ \ \isacommand{also}\isamarkupfalse%
\ \isacommand{have}\isamarkupfalse%
\ {\isachardoublequoteopen}{\isasymdots}\ {\isasymsubseteq}\ setSubformulae\ {\isacharparenleft}F{\isadigit{1}}\ \isactrlbold {\isasymor}\ F{\isadigit{2}}{\isacharparenright}{\isachardoublequoteclose}\isanewline
\ \ \ \ \ \ \ \ \isacommand{by}\isamarkupfalse%
\ {\isacharparenleft}simp\ only{\isacharcolon}\ setSubformulae{\isacharunderscore}or\ Un{\isacharunderscore}upper{\isadigit{2}}{\isacharparenright}\isanewline
\ \ \ \ \ \ \isacommand{finally}\isamarkupfalse%
\ \isacommand{show}\isamarkupfalse%
\ {\isacharquery}thesis\isanewline
\ \ \ \ \ \ \ \ \isacommand{by}\isamarkupfalse%
\ this\isanewline
\ \ \ \ \isacommand{next}\isamarkupfalse%
\isanewline
\ \ \ \ \ \ \isacommand{assume}\isamarkupfalse%
\ {\isachardoublequoteopen}G\ {\isasymin}\ setSubformulae\ F{\isadigit{2}}{\isachardoublequoteclose}\isanewline
\ \ \ \ \ \ \isacommand{then}\isamarkupfalse%
\ \isacommand{have}\isamarkupfalse%
\ {\isachardoublequoteopen}setSubformulae\ G\ {\isasymsubseteq}\ setSubformulae\ F{\isadigit{2}}{\isachardoublequoteclose}\isanewline
\ \ \ \ \ \ \ \ \isacommand{by}\isamarkupfalse%
\ {\isacharparenleft}rule\ assms{\isacharparenleft}{\isadigit{2}}{\isacharparenright}{\isacharparenright}\isanewline
\ \ \ \ \ \ \isacommand{also}\isamarkupfalse%
\ \isacommand{have}\isamarkupfalse%
\ {\isachardoublequoteopen}{\isasymdots}\ {\isasymsubseteq}\ setSubformulae\ F{\isadigit{1}}\ {\isasymunion}\ setSubformulae\ F{\isadigit{2}}{\isachardoublequoteclose}\isanewline
\ \ \ \ \ \ \ \ \isacommand{by}\isamarkupfalse%
\ {\isacharparenleft}simp\ only{\isacharcolon}\ Un{\isacharunderscore}upper{\isadigit{2}}{\isacharparenright}\isanewline
\ \ \ \ \ \ \isacommand{also}\isamarkupfalse%
\ \isacommand{have}\isamarkupfalse%
\ {\isachardoublequoteopen}{\isasymdots}\ {\isasymsubseteq}\ setSubformulae\ {\isacharparenleft}F{\isadigit{1}}\ \isactrlbold {\isasymor}\ F{\isadigit{2}}{\isacharparenright}{\isachardoublequoteclose}\isanewline
\ \ \ \ \ \ \ \ \isacommand{by}\isamarkupfalse%
\ {\isacharparenleft}simp\ only{\isacharcolon}\ setSubformulae{\isacharunderscore}or\ Un{\isacharunderscore}upper{\isadigit{2}}{\isacharparenright}\isanewline
\ \ \ \ \ \ \isacommand{finally}\isamarkupfalse%
\ \isacommand{show}\isamarkupfalse%
\ {\isacharquery}thesis\isanewline
\ \ \ \ \ \ \ \ \isacommand{by}\isamarkupfalse%
\ this\isanewline
\ \ \ \ \isacommand{qed}\isamarkupfalse%
\isanewline
\ \ \isacommand{qed}\isamarkupfalse%
\isanewline
\isacommand{qed}\isamarkupfalse%
%
\endisatagproof
{\isafoldproof}%
%
\isadelimproof
\isanewline
%
\endisadelimproof
\isanewline
\isacommand{lemma}\isamarkupfalse%
\ subContsubformulae{\isacharunderscore}imp{\isacharcolon}\isanewline
\ \ \isakeyword{assumes}\ {\isachardoublequoteopen}G\ {\isasymin}\ setSubformulae\ F{\isadigit{1}}\ \isanewline
\ \ \ \ \ \ \ \ \ \ \ \ {\isasymLongrightarrow}\ setSubformulae\ G\ {\isasymsubseteq}\ setSubformulae\ F{\isadigit{1}}{\isachardoublequoteclose}\isanewline
\ \ \ \ \ \ \ \ \ \ {\isachardoublequoteopen}G\ {\isasymin}\ setSubformulae\ F{\isadigit{2}}\ \isanewline
\ \ \ \ \ \ \ \ \ \ \ \ {\isasymLongrightarrow}\ setSubformulae\ G\ {\isasymsubseteq}\ setSubformulae\ F{\isadigit{2}}{\isachardoublequoteclose}\isanewline
\ \ \ \ \ \ \ \ \ \ {\isachardoublequoteopen}G\ {\isasymin}\ setSubformulae\ {\isacharparenleft}F{\isadigit{1}}\ \isactrlbold {\isasymrightarrow}\ F{\isadigit{2}}{\isacharparenright}{\isachardoublequoteclose}\isanewline
\ \ \isakeyword{shows}\ \ \ {\isachardoublequoteopen}setSubformulae\ G\ {\isasymsubseteq}\ setSubformulae\ {\isacharparenleft}F{\isadigit{1}}\ \isactrlbold {\isasymrightarrow}\ F{\isadigit{2}}{\isacharparenright}{\isachardoublequoteclose}\isanewline
%
\isadelimproof
%
\endisadelimproof
%
\isatagproof
\isacommand{proof}\isamarkupfalse%
\ {\isacharminus}\isanewline
\ \ \isacommand{have}\isamarkupfalse%
\ {\isachardoublequoteopen}G\ {\isasymin}\ {\isacharbraceleft}F{\isadigit{1}}\ \isactrlbold {\isasymrightarrow}\ F{\isadigit{2}}{\isacharbraceright}\ {\isasymunion}\ {\isacharparenleft}setSubformulae\ F{\isadigit{1}}\ {\isasymunion}\ setSubformulae\ F{\isadigit{2}}{\isacharparenright}{\isachardoublequoteclose}\isanewline
\ \ \ \ \isacommand{using}\isamarkupfalse%
\ assms{\isacharparenleft}{\isadigit{3}}{\isacharparenright}\ \isanewline
\ \ \ \ \isacommand{by}\isamarkupfalse%
\ {\isacharparenleft}simp\ only{\isacharcolon}\ setSubformulae{\isacharunderscore}imp{\isacharparenright}\isanewline
\ \ \isacommand{then}\isamarkupfalse%
\ \isacommand{have}\isamarkupfalse%
\ {\isachardoublequoteopen}G\ {\isasymin}\ {\isacharbraceleft}F{\isadigit{1}}\ \isactrlbold {\isasymrightarrow}\ F{\isadigit{2}}{\isacharbraceright}\ {\isasymor}\ G\ {\isasymin}\ setSubformulae\ F{\isadigit{1}}\ {\isasymunion}\ setSubformulae\ F{\isadigit{2}}{\isachardoublequoteclose}\isanewline
\ \ \ \ \isacommand{by}\isamarkupfalse%
\ {\isacharparenleft}simp\ only{\isacharcolon}\ Un{\isacharunderscore}iff{\isacharparenright}\isanewline
\ \ \isacommand{then}\isamarkupfalse%
\ \isacommand{show}\isamarkupfalse%
\ {\isacharquery}thesis\isanewline
\ \ \isacommand{proof}\isamarkupfalse%
\ \isanewline
\ \ \ \ \isacommand{assume}\isamarkupfalse%
\ {\isachardoublequoteopen}G\ {\isasymin}\ {\isacharbraceleft}F{\isadigit{1}}\ \isactrlbold {\isasymrightarrow}\ F{\isadigit{2}}{\isacharbraceright}{\isachardoublequoteclose}\isanewline
\ \ \ \ \isacommand{then}\isamarkupfalse%
\ \isacommand{have}\isamarkupfalse%
\ {\isachardoublequoteopen}G\ {\isacharequal}\ F{\isadigit{1}}\ \isactrlbold {\isasymrightarrow}\ F{\isadigit{2}}{\isachardoublequoteclose}\isanewline
\ \ \ \ \ \ \isacommand{by}\isamarkupfalse%
\ {\isacharparenleft}simp\ only{\isacharcolon}\ singletonD{\isacharparenright}\isanewline
\ \ \ \ \isacommand{then}\isamarkupfalse%
\ \isacommand{show}\isamarkupfalse%
\ {\isacharquery}thesis\isanewline
\ \ \ \ \ \ \isacommand{by}\isamarkupfalse%
\ {\isacharparenleft}simp\ only{\isacharcolon}\ subset{\isacharunderscore}refl{\isacharparenright}\isanewline
\ \ \isacommand{next}\isamarkupfalse%
\isanewline
\ \ \ \ \isacommand{assume}\isamarkupfalse%
\ {\isachardoublequoteopen}G\ {\isasymin}\ setSubformulae\ F{\isadigit{1}}\ {\isasymunion}\ setSubformulae\ F{\isadigit{2}}{\isachardoublequoteclose}\isanewline
\ \ \ \ \isacommand{then}\isamarkupfalse%
\ \isacommand{have}\isamarkupfalse%
\ {\isachardoublequoteopen}G\ {\isasymin}\ setSubformulae\ F{\isadigit{1}}\ {\isasymor}\ G\ {\isasymin}\ setSubformulae\ F{\isadigit{2}}{\isachardoublequoteclose}\ \ \isanewline
\ \ \ \ \ \ \isacommand{by}\isamarkupfalse%
\ {\isacharparenleft}simp\ only{\isacharcolon}\ Un{\isacharunderscore}iff{\isacharparenright}\isanewline
\ \ \ \ \isacommand{then}\isamarkupfalse%
\ \isacommand{show}\isamarkupfalse%
\ {\isacharquery}thesis\isanewline
\ \ \ \ \isacommand{proof}\isamarkupfalse%
\ \isanewline
\ \ \ \ \ \ \isacommand{assume}\isamarkupfalse%
\ {\isachardoublequoteopen}G\ {\isasymin}\ setSubformulae\ F{\isadigit{1}}{\isachardoublequoteclose}\isanewline
\ \ \ \ \ \ \isacommand{then}\isamarkupfalse%
\ \isacommand{have}\isamarkupfalse%
\ {\isachardoublequoteopen}setSubformulae\ G\ {\isasymsubseteq}\ setSubformulae\ F{\isadigit{1}}{\isachardoublequoteclose}\isanewline
\ \ \ \ \ \ \ \ \isacommand{by}\isamarkupfalse%
\ {\isacharparenleft}simp\ only{\isacharcolon}\ assms{\isacharparenleft}{\isadigit{1}}{\isacharparenright}{\isacharparenright}\isanewline
\ \ \ \ \ \ \isacommand{also}\isamarkupfalse%
\ \isacommand{have}\isamarkupfalse%
\ {\isachardoublequoteopen}{\isasymdots}\ {\isasymsubseteq}\ setSubformulae\ F{\isadigit{1}}\ {\isasymunion}\ setSubformulae\ F{\isadigit{2}}{\isachardoublequoteclose}\isanewline
\ \ \ \ \ \ \ \ \isacommand{by}\isamarkupfalse%
\ {\isacharparenleft}simp\ only{\isacharcolon}\ Un{\isacharunderscore}upper{\isadigit{1}}{\isacharparenright}\isanewline
\ \ \ \ \ \ \isacommand{also}\isamarkupfalse%
\ \isacommand{have}\isamarkupfalse%
\ {\isachardoublequoteopen}{\isasymdots}\ {\isasymsubseteq}\ setSubformulae\ {\isacharparenleft}F{\isadigit{1}}\ \isactrlbold {\isasymrightarrow}\ F{\isadigit{2}}{\isacharparenright}{\isachardoublequoteclose}\isanewline
\ \ \ \ \ \ \ \ \isacommand{by}\isamarkupfalse%
\ {\isacharparenleft}simp\ only{\isacharcolon}\ setSubformulae{\isacharunderscore}imp\ Un{\isacharunderscore}upper{\isadigit{2}}{\isacharparenright}\isanewline
\ \ \ \ \ \ \isacommand{finally}\isamarkupfalse%
\ \isacommand{show}\isamarkupfalse%
\ {\isacharquery}thesis\isanewline
\ \ \ \ \ \ \ \ \isacommand{by}\isamarkupfalse%
\ this\isanewline
\ \ \ \ \isacommand{next}\isamarkupfalse%
\isanewline
\ \ \ \ \ \ \isacommand{assume}\isamarkupfalse%
\ {\isachardoublequoteopen}G\ {\isasymin}\ setSubformulae\ F{\isadigit{2}}{\isachardoublequoteclose}\isanewline
\ \ \ \ \ \ \isacommand{then}\isamarkupfalse%
\ \isacommand{have}\isamarkupfalse%
\ {\isachardoublequoteopen}setSubformulae\ G\ {\isasymsubseteq}\ setSubformulae\ F{\isadigit{2}}{\isachardoublequoteclose}\isanewline
\ \ \ \ \ \ \ \ \isacommand{by}\isamarkupfalse%
\ {\isacharparenleft}rule\ assms{\isacharparenleft}{\isadigit{2}}{\isacharparenright}{\isacharparenright}\isanewline
\ \ \ \ \ \ \isacommand{also}\isamarkupfalse%
\ \isacommand{have}\isamarkupfalse%
\ {\isachardoublequoteopen}{\isasymdots}\ {\isasymsubseteq}\ setSubformulae\ F{\isadigit{1}}\ {\isasymunion}\ setSubformulae\ F{\isadigit{2}}{\isachardoublequoteclose}\isanewline
\ \ \ \ \ \ \ \ \isacommand{by}\isamarkupfalse%
\ {\isacharparenleft}simp\ only{\isacharcolon}\ Un{\isacharunderscore}upper{\isadigit{2}}{\isacharparenright}\isanewline
\ \ \ \ \ \ \isacommand{also}\isamarkupfalse%
\ \isacommand{have}\isamarkupfalse%
\ {\isachardoublequoteopen}{\isasymdots}\ {\isasymsubseteq}\ setSubformulae\ {\isacharparenleft}F{\isadigit{1}}\ \isactrlbold {\isasymrightarrow}\ F{\isadigit{2}}{\isacharparenright}{\isachardoublequoteclose}\isanewline
\ \ \ \ \ \ \ \ \isacommand{by}\isamarkupfalse%
\ {\isacharparenleft}simp\ only{\isacharcolon}\ setSubformulae{\isacharunderscore}imp\ Un{\isacharunderscore}upper{\isadigit{2}}{\isacharparenright}\isanewline
\ \ \ \ \ \ \isacommand{finally}\isamarkupfalse%
\ \isacommand{show}\isamarkupfalse%
\ {\isacharquery}thesis\isanewline
\ \ \ \ \ \ \ \ \isacommand{by}\isamarkupfalse%
\ this\isanewline
\ \ \ \ \isacommand{qed}\isamarkupfalse%
\isanewline
\ \ \isacommand{qed}\isamarkupfalse%
\isanewline
\isacommand{qed}\isamarkupfalse%
%
\endisatagproof
{\isafoldproof}%
%
\isadelimproof
\isanewline
%
\endisadelimproof
\isanewline
\isacommand{lemma}\isamarkupfalse%
\isanewline
\ \ {\isachardoublequoteopen}G\ {\isasymin}\ setSubformulae\ F\ {\isasymLongrightarrow}\ setSubformulae\ G\ {\isasymsubseteq}\ setSubformulae\ F{\isachardoublequoteclose}\isanewline
%
\isadelimproof
%
\endisadelimproof
%
\isatagproof
\isacommand{proof}\isamarkupfalse%
\ {\isacharparenleft}induction\ F{\isacharparenright}\isanewline
\isacommand{case}\isamarkupfalse%
\ {\isacharparenleft}Atom\ x{\isacharparenright}\isanewline
\ \ \isacommand{then}\isamarkupfalse%
\ \isacommand{show}\isamarkupfalse%
\ {\isacharquery}case\ \isacommand{by}\isamarkupfalse%
\ {\isacharparenleft}rule\ subContsubformulae{\isacharunderscore}atom{\isacharparenright}\isanewline
\isacommand{next}\isamarkupfalse%
\isanewline
\ \ \isacommand{case}\isamarkupfalse%
\ Bot\isanewline
\ \ \isacommand{then}\isamarkupfalse%
\ \isacommand{show}\isamarkupfalse%
\ {\isacharquery}case\ \isacommand{by}\isamarkupfalse%
\ {\isacharparenleft}rule\ subContsubformulae{\isacharunderscore}bot{\isacharparenright}\isanewline
\isacommand{next}\isamarkupfalse%
\isanewline
\isacommand{case}\isamarkupfalse%
\ {\isacharparenleft}Not\ F{\isacharparenright}\isanewline
\ \ \isacommand{then}\isamarkupfalse%
\ \isacommand{show}\isamarkupfalse%
\ {\isacharquery}case\ \isacommand{by}\isamarkupfalse%
\ {\isacharparenleft}rule\ subContsubformulae{\isacharunderscore}not{\isacharparenright}\isanewline
\isacommand{next}\isamarkupfalse%
\isanewline
\ \ \isacommand{case}\isamarkupfalse%
\ {\isacharparenleft}And\ F{\isadigit{1}}\ F{\isadigit{2}}{\isacharparenright}\isanewline
\ \ \isacommand{then}\isamarkupfalse%
\ \isacommand{show}\isamarkupfalse%
\ {\isacharquery}case\ \isacommand{by}\isamarkupfalse%
\ {\isacharparenleft}rule\ subContsubformulae{\isacharunderscore}and{\isacharparenright}\isanewline
\isacommand{next}\isamarkupfalse%
\isanewline
\ \ \isacommand{case}\isamarkupfalse%
\ {\isacharparenleft}Or\ F{\isadigit{1}}\ F{\isadigit{2}}{\isacharparenright}\isanewline
\ \ \isacommand{then}\isamarkupfalse%
\ \isacommand{show}\isamarkupfalse%
\ {\isacharquery}case\ \isacommand{by}\isamarkupfalse%
\ {\isacharparenleft}rule\ subContsubformulae{\isacharunderscore}or{\isacharparenright}\isanewline
\isacommand{next}\isamarkupfalse%
\isanewline
\ \ \isacommand{case}\isamarkupfalse%
\ {\isacharparenleft}Imp\ F{\isadigit{1}}\ F{\isadigit{2}}{\isacharparenright}\isanewline
\ \ \isacommand{then}\isamarkupfalse%
\ \isacommand{show}\isamarkupfalse%
\ {\isacharquery}case\ \isacommand{by}\isamarkupfalse%
\ {\isacharparenleft}rule\ subContsubformulae{\isacharunderscore}imp{\isacharparenright}\isanewline
\isacommand{qed}\isamarkupfalse%
%
\endisatagproof
{\isafoldproof}%
%
\isadelimproof
%
\endisadelimproof
%
\begin{isamarkuptext}%
Finalmente, su demostración automática se muestra a continuación.%
\end{isamarkuptext}\isamarkuptrue%
\isacommand{lemma}\isamarkupfalse%
\ subContsubformulae{\isacharcolon}\isanewline
\ \ {\isachardoublequoteopen}G\ {\isasymin}\ setSubformulae\ F\ {\isasymLongrightarrow}\ setSubformulae\ G\ {\isasymsubseteq}\ setSubformulae\ F{\isachardoublequoteclose}\isanewline
%
\isadelimproof
\ \ %
\endisadelimproof
%
\isatagproof
\isacommand{by}\isamarkupfalse%
\ {\isacharparenleft}induction\ F{\isacharparenright}\ auto%
\endisatagproof
{\isafoldproof}%
%
\isadelimproof
%
\endisadelimproof
%
\begin{isamarkuptext}%
El siguiente lema nos da la noción de transitividad de contención 
  en cadena de las subfórmulas, de modo que la subfórmula de una 
  subfórmula es del mismo modo subfórmula de la mayor.

  \begin{lema}
    Sean \isa{G} subfórmula de \isa{F} y \isa{H} subfórmula de \isa{G}, entonces 
    \isa{H} es subfórmula de \isa{F}.
  \end{lema}

  \begin{demostracion}
  La prueba está basada en el lema anterior. Hemos demostrado que como 
  \isa{G} es subfórmula de \isa{F}, entonces el conjunto de átomos de \isa{G} está
  contenido en el de \isa{F}. Del mismo modo, como \isa{H} es subfórmula de
  \isa{G}, su conjunto de átomos está contenido en el de \isa{G}. Por la
  transitividad de la contención, tenemos que el conjunto de átomos de 
  \isa{H} está contenido en el de \isa{F}. Por otro lema anterior, tenemos que
  \isa{H} pertenece a su propio conjunto de subfórmulas. Por tanto,
  \isa{H\ {\isasymin}\ Subf{\isacharparenleft}H{\isacharparenright}\ {\isasymsubseteq}\ Subf{\isacharparenleft}F{\isacharparenright}\ {\isasymLongrightarrow}\ H\ {\isasymin}\ Subf{\isacharparenleft}F{\isacharparenright}}.
  \end{demostracion}

  Veamos su formalización y prueba estructurada en Isabelle.

  Veamos su prueba según las distintas tácticas.%
\end{isamarkuptext}\isamarkuptrue%
\isacommand{lemma}\isamarkupfalse%
\isanewline
\ \ \isakeyword{assumes}\ {\isachardoublequoteopen}G\ {\isasymin}\ setSubformulae\ F{\isachardoublequoteclose}\ \isanewline
\ \ \ \ \ \ \ \ \ \ {\isachardoublequoteopen}H\ {\isasymin}\ setSubformulae\ G{\isachardoublequoteclose}\isanewline
\ \ \isakeyword{shows}\ \ \ {\isachardoublequoteopen}H\ {\isasymin}\ setSubformulae\ F{\isachardoublequoteclose}\isanewline
%
\isadelimproof
%
\endisadelimproof
%
\isatagproof
\isacommand{proof}\isamarkupfalse%
\ {\isacharminus}\isanewline
\ \ \isacommand{have}\isamarkupfalse%
\ {\isadigit{1}}{\isacharcolon}{\isachardoublequoteopen}setSubformulae\ G\ {\isasymsubseteq}\ setSubformulae\ F{\isachardoublequoteclose}\ \isacommand{using}\isamarkupfalse%
\ assms{\isacharparenleft}{\isadigit{1}}{\isacharparenright}\ \isanewline
\ \ \ \ \isacommand{by}\isamarkupfalse%
\ {\isacharparenleft}rule\ subContsubformulae{\isacharparenright}\isanewline
\ \ \isacommand{have}\isamarkupfalse%
\ {\isachardoublequoteopen}setSubformulae\ H\ {\isasymsubseteq}\ setSubformulae\ G{\isachardoublequoteclose}\ \isacommand{using}\isamarkupfalse%
\ assms{\isacharparenleft}{\isadigit{2}}{\isacharparenright}\ \isanewline
\ \ \ \ \isacommand{by}\isamarkupfalse%
\ {\isacharparenleft}rule\ subContsubformulae{\isacharparenright}\isanewline
\ \ \isacommand{then}\isamarkupfalse%
\ \isacommand{have}\isamarkupfalse%
\ {\isadigit{2}}{\isacharcolon}{\isachardoublequoteopen}setSubformulae\ H\ {\isasymsubseteq}\ setSubformulae\ F{\isachardoublequoteclose}\ \isacommand{using}\isamarkupfalse%
\ {\isadigit{1}}\ \isanewline
\ \ \ \ \isacommand{by}\isamarkupfalse%
\ {\isacharparenleft}rule\ subset{\isacharunderscore}trans{\isacharparenright}\isanewline
\ \ \isacommand{have}\isamarkupfalse%
\ {\isachardoublequoteopen}H\ {\isasymin}\ setSubformulae\ H{\isachardoublequoteclose}\ \isanewline
\ \ \ \ \isacommand{by}\isamarkupfalse%
\ {\isacharparenleft}simp\ only{\isacharcolon}\ subformulae{\isacharunderscore}self{\isacharparenright}\isanewline
\ \ \isacommand{then}\isamarkupfalse%
\ \isacommand{show}\isamarkupfalse%
\ {\isachardoublequoteopen}H\ {\isasymin}\ setSubformulae\ F{\isachardoublequoteclose}\ \isanewline
\ \ \ \ \isacommand{using}\isamarkupfalse%
\ {\isadigit{2}}\ \isanewline
\ \ \ \ \isacommand{by}\isamarkupfalse%
\ {\isacharparenleft}rule\ rev{\isacharunderscore}subsetD{\isacharparenright}\isanewline
\isacommand{qed}\isamarkupfalse%
%
\endisatagproof
{\isafoldproof}%
%
\isadelimproof
\isanewline
%
\endisadelimproof
\isanewline
\isacommand{lemma}\isamarkupfalse%
\ subsubformulae{\isacharcolon}\ \isanewline
\ \ {\isachardoublequoteopen}G\ {\isasymin}\ setSubformulae\ F\ \isanewline
\ \ \ {\isasymLongrightarrow}\ H\ {\isasymin}\ setSubformulae\ G\ \isanewline
\ \ \ {\isasymLongrightarrow}\ H\ {\isasymin}\ setSubformulae\ F{\isachardoublequoteclose}\isanewline
%
\isadelimproof
\ \ %
\endisadelimproof
%
\isatagproof
\isacommand{using}\isamarkupfalse%
\ subContsubformulae\ \isacommand{by}\isamarkupfalse%
\ blast%
\endisatagproof
{\isafoldproof}%
%
\isadelimproof
%
\endisadelimproof
%
\begin{isamarkuptext}%
comentario{Explicar el cambio de enunciado}%
\end{isamarkuptext}\isamarkuptrue%
%
\begin{isamarkuptext}%
A continuación presentamos otro resultado que relaciona los 
  conjuntos de subfórmulas según las conectivas que operen.%
\end{isamarkuptext}\isamarkuptrue%
\isacommand{lemma}\isamarkupfalse%
\ subformulas{\isacharunderscore}in{\isacharunderscore}subformulas{\isacharcolon}\isanewline
\ \ {\isachardoublequoteopen}G\ \isactrlbold {\isasymand}\ H\ {\isasymin}\ setSubformulae\ F\ \isanewline
\ \ {\isasymLongrightarrow}\ G\ {\isasymin}\ setSubformulae\ F\ {\isasymand}\ H\ {\isasymin}\ setSubformulae\ F{\isachardoublequoteclose}\isanewline
\ \ {\isachardoublequoteopen}G\ \isactrlbold {\isasymor}\ H\ {\isasymin}\ setSubformulae\ F\ \isanewline
\ \ {\isasymLongrightarrow}\ G\ {\isasymin}\ setSubformulae\ F\ {\isasymand}\ H\ {\isasymin}\ setSubformulae\ F{\isachardoublequoteclose}\isanewline
\ \ {\isachardoublequoteopen}G\ \isactrlbold {\isasymrightarrow}\ H\ {\isasymin}\ setSubformulae\ F\ \isanewline
\ \ {\isasymLongrightarrow}\ G\ {\isasymin}\ setSubformulae\ F\ {\isasymand}\ H\ {\isasymin}\ setSubformulae\ F{\isachardoublequoteclose}\isanewline
\ \ {\isachardoublequoteopen}\isactrlbold {\isasymnot}\ G\ {\isasymin}\ setSubformulae\ F\ {\isasymLongrightarrow}\ G\ {\isasymin}\ setSubformulae\ F{\isachardoublequoteclose}\isanewline
%
\isadelimproof
\ \ %
\endisadelimproof
%
\isatagproof
\isacommand{oops}\isamarkupfalse%
%
\endisatagproof
{\isafoldproof}%
%
\isadelimproof
%
\endisadelimproof
%
\begin{isamarkuptext}%
Como podemos observar, el resultado es análogo en todas las 
  conectivas binarias aunque aparezcan definidas por separado, por tanto 
  haré la demostración estructurada para una de ellas pues el resto son 
  análogas. 

  Nos basaremos en el lema anterior \isa{subsubformulae}.%
\end{isamarkuptext}\isamarkuptrue%
\isacommand{lemma}\isamarkupfalse%
\ subformulas{\isacharunderscore}in{\isacharunderscore}subformulas{\isacharunderscore}not{\isacharcolon}\isanewline
\ \ \isakeyword{assumes}\ {\isachardoublequoteopen}Not\ G\ {\isasymin}\ setSubformulae\ F{\isachardoublequoteclose}\isanewline
\ \ \isakeyword{shows}\ {\isachardoublequoteopen}G\ {\isasymin}\ setSubformulae\ F{\isachardoublequoteclose}\isanewline
%
\isadelimproof
%
\endisadelimproof
%
\isatagproof
\isacommand{proof}\isamarkupfalse%
\ {\isacharminus}\isanewline
\ \ \isacommand{have}\isamarkupfalse%
\ {\isachardoublequoteopen}setSubformulae\ {\isacharparenleft}Not\ G{\isacharparenright}\ {\isacharequal}\ {\isacharbraceleft}Not\ G{\isacharbraceright}\ {\isasymunion}\ setSubformulae\ G{\isachardoublequoteclose}\ \isanewline
\ \ \ \ \isacommand{by}\isamarkupfalse%
\ simp\ %
\isamarkupcmt{Pendiente%
}\isanewline
\ \ \isacommand{then}\isamarkupfalse%
\ \isacommand{have}\isamarkupfalse%
\ {\isadigit{1}}{\isacharcolon}{\isachardoublequoteopen}G\ {\isasymin}\ setSubformulae\ {\isacharparenleft}Not\ G{\isacharparenright}{\isachardoublequoteclose}\ \isanewline
\ \ \ \ \isacommand{by}\isamarkupfalse%
\ {\isacharparenleft}simp\ add{\isacharcolon}\ subformulae{\isacharunderscore}self{\isacharparenright}\ %
\isamarkupcmt{Pendiente%
}\isanewline
\ \ \isacommand{show}\isamarkupfalse%
\ {\isachardoublequoteopen}G\ {\isasymin}\ setSubformulae\ F{\isachardoublequoteclose}\ \isacommand{using}\isamarkupfalse%
\ assms\ {\isadigit{1}}\ \isanewline
\ \ \ \ \isacommand{by}\isamarkupfalse%
\ {\isacharparenleft}rule\ subsubformulae{\isacharparenright}\isanewline
\isacommand{qed}\isamarkupfalse%
%
\endisatagproof
{\isafoldproof}%
%
\isadelimproof
\isanewline
%
\endisadelimproof
\isanewline
\isacommand{lemma}\isamarkupfalse%
\ subformulas{\isacharunderscore}in{\isacharunderscore}subformulas{\isacharunderscore}and{\isacharcolon}\isanewline
\ \ \isakeyword{assumes}\ {\isachardoublequoteopen}G\ \isactrlbold {\isasymand}\ H\ {\isasymin}\ setSubformulae\ F{\isachardoublequoteclose}\ \isanewline
\ \ \isakeyword{shows}\ {\isachardoublequoteopen}G\ {\isasymin}\ setSubformulae\ F\ {\isasymand}\ H\ {\isasymin}\ setSubformulae\ F{\isachardoublequoteclose}\isanewline
%
\isadelimproof
%
\endisadelimproof
%
\isatagproof
\isacommand{proof}\isamarkupfalse%
\ {\isacharparenleft}rule\ conjI{\isacharparenright}\isanewline
\ \ \isacommand{have}\isamarkupfalse%
\ {\isadigit{1}}{\isacharcolon}\ {\isachardoublequoteopen}setSubformulae\ {\isacharparenleft}G\ \isactrlbold {\isasymand}\ H{\isacharparenright}\ {\isacharequal}\ \isanewline
\ \ \ \ \ \ \ \ \ \ {\isacharbraceleft}G\ \isactrlbold {\isasymand}\ H{\isacharbraceright}\ {\isasymunion}\ {\isacharparenleft}setSubformulae\ G\ {\isasymunion}\ setSubformulae\ H{\isacharparenright}{\isachardoublequoteclose}\ \isanewline
\ \ \ \ \isacommand{by}\isamarkupfalse%
\ {\isacharparenleft}simp\ only{\isacharcolon}\ setSubformulae{\isacharunderscore}and{\isacharparenright}\isanewline
\ \ \isacommand{then}\isamarkupfalse%
\ \isacommand{have}\isamarkupfalse%
\ {\isadigit{2}}{\isacharcolon}\ {\isachardoublequoteopen}G\ {\isasymin}\ setSubformulae\ {\isacharparenleft}G\ \isactrlbold {\isasymand}\ H{\isacharparenright}{\isachardoublequoteclose}\ \isanewline
\ \ \ \ \isacommand{by}\isamarkupfalse%
\ {\isacharparenleft}simp\ add{\isacharcolon}\ subformulae{\isacharunderscore}self{\isacharparenright}\ %
\isamarkupcmt{Pendiente%
}\ \isanewline
\ \ \isacommand{have}\isamarkupfalse%
\ {\isadigit{3}}{\isacharcolon}\ {\isachardoublequoteopen}H\ {\isasymin}\ setSubformulae\ {\isacharparenleft}G\ \isactrlbold {\isasymand}\ H{\isacharparenright}{\isachardoublequoteclose}\ \isanewline
\ \ \ \ \isacommand{using}\isamarkupfalse%
\ {\isadigit{1}}\ \isanewline
\ \ \ \ \isacommand{by}\isamarkupfalse%
\ {\isacharparenleft}simp\ add{\isacharcolon}\ subformulae{\isacharunderscore}self{\isacharparenright}\ %
\isamarkupcmt{Pendiente%
}\ \isanewline
\ \ \isacommand{show}\isamarkupfalse%
\ {\isachardoublequoteopen}G\ {\isasymin}\ setSubformulae\ F{\isachardoublequoteclose}\ \isacommand{using}\isamarkupfalse%
\ assms\ {\isadigit{2}}\ \isacommand{by}\isamarkupfalse%
\ {\isacharparenleft}rule\ subsubformulae{\isacharparenright}\isanewline
\ \ \isacommand{show}\isamarkupfalse%
\ {\isachardoublequoteopen}H\ {\isasymin}\ setSubformulae\ F{\isachardoublequoteclose}\ \isacommand{using}\isamarkupfalse%
\ assms\ {\isadigit{3}}\ \isacommand{by}\isamarkupfalse%
\ {\isacharparenleft}rule\ subsubformulae{\isacharparenright}\isanewline
\isacommand{qed}\isamarkupfalse%
%
\endisatagproof
{\isafoldproof}%
%
\isadelimproof
%
\endisadelimproof
%
\begin{isamarkuptext}%
Mostremos ahora la demostración automática.%
\end{isamarkuptext}\isamarkuptrue%
\isacommand{lemma}\isamarkupfalse%
\ subformulas{\isacharunderscore}in{\isacharunderscore}subformulas{\isacharcolon}\isanewline
\ \ {\isachardoublequoteopen}G\ \isactrlbold {\isasymand}\ H\ {\isasymin}\ setSubformulae\ F\ \isanewline
\ \ \ {\isasymLongrightarrow}\ G\ {\isasymin}\ setSubformulae\ F\ {\isasymand}\ H\ {\isasymin}\ setSubformulae\ F{\isachardoublequoteclose}\isanewline
\ \ {\isachardoublequoteopen}G\ \isactrlbold {\isasymor}\ H\ {\isasymin}\ setSubformulae\ F\ \isanewline
\ \ \ {\isasymLongrightarrow}\ G\ {\isasymin}\ setSubformulae\ F\ {\isasymand}\ H\ {\isasymin}\ setSubformulae\ F{\isachardoublequoteclose}\isanewline
\ \ {\isachardoublequoteopen}G\ \isactrlbold {\isasymrightarrow}\ H\ {\isasymin}\ setSubformulae\ F\ \isanewline
\ \ \ {\isasymLongrightarrow}\ G\ {\isasymin}\ setSubformulae\ F\ {\isasymand}\ H\ {\isasymin}\ setSubformulae\ F{\isachardoublequoteclose}\isanewline
\ \ {\isachardoublequoteopen}\isactrlbold {\isasymnot}\ G\ {\isasymin}\ setSubformulae\ F\ {\isasymLongrightarrow}\ G\ {\isasymin}\ setSubformulae\ F{\isachardoublequoteclose}\isanewline
%
\isadelimproof
\ \ %
\endisadelimproof
%
\isatagproof
\isacommand{using}\isamarkupfalse%
\ subformulae{\isacharunderscore}self\ subsubformulae\ \isacommand{apply}\isamarkupfalse%
\ force\isanewline
\ \ \isacommand{oops}\isamarkupfalse%
%
\endisatagproof
{\isafoldproof}%
%
\isadelimproof
%
\endisadelimproof
%
\begin{isamarkuptext}%
\comentario{Completar la prueba anterior.}%
\end{isamarkuptext}\isamarkuptrue%
%
\begin{isamarkuptext}%
\comentario{Completar lo que falta de sección}%
\end{isamarkuptext}\isamarkuptrue%
%
\begin{isamarkuptext}%
Concluimos la sección de subfórmulas con un resultado que 
  relaciona varias funciones sobre la longitud de la lista 
  \isa{subformulae\ F} de una fórmula \isa{F} cualquiera.%
\end{isamarkuptext}\isamarkuptrue%
\isacommand{lemma}\isamarkupfalse%
\ length{\isacharunderscore}subformulae{\isacharcolon}\ {\isachardoublequoteopen}length\ {\isacharparenleft}subformulae\ F{\isacharparenright}\ {\isacharequal}\ size\ F{\isachardoublequoteclose}\ \isanewline
%
\isadelimproof
\ \ %
\endisadelimproof
%
\isatagproof
\isacommand{oops}\isamarkupfalse%
%
\endisatagproof
{\isafoldproof}%
%
\isadelimproof
%
\endisadelimproof
%
\begin{isamarkuptext}%
En primer lugar aparece la clase \isa{size} de la teoría de 
  números naturales ....

  Vamos a definir \isa{size{\isadigit{1}}} de manera idéntica a como aparece 
  \isa{size} en la teoría.%
\end{isamarkuptext}\isamarkuptrue%
\isacommand{class}\isamarkupfalse%
\ size{\isadigit{1}}\ {\isacharequal}\isanewline
\ \ \isakeyword{fixes}\ size{\isadigit{1}}\ {\isacharcolon}{\isacharcolon}\ {\isachardoublequoteopen}{\isacharprime}a\ {\isasymRightarrow}\ nat{\isachardoublequoteclose}\ \isanewline
\isanewline
\isacommand{instantiation}\isamarkupfalse%
\ nat\ {\isacharcolon}{\isacharcolon}\ size{\isadigit{1}}\isanewline
\isakeyword{begin}\isanewline
\isanewline
\isacommand{definition}\isamarkupfalse%
\ size{\isadigit{1}}{\isacharunderscore}nat\ \isakeyword{where}\ {\isacharbrackleft}simp{\isacharcomma}\ code{\isacharbrackright}{\isacharcolon}\ {\isachardoublequoteopen}size{\isadigit{1}}\ {\isacharparenleft}n{\isacharcolon}{\isacharcolon}nat{\isacharparenright}\ {\isacharequal}\ n{\isachardoublequoteclose}\isanewline
\isanewline
\isacommand{instance}\isamarkupfalse%
%
\isadelimproof
\ %
\endisadelimproof
%
\isatagproof
\isacommand{{\isachardot}{\isachardot}}\isamarkupfalse%
%
\endisatagproof
{\isafoldproof}%
%
\isadelimproof
%
\endisadelimproof
\isanewline
\isanewline
\isacommand{end}\isamarkupfalse%
%
\begin{isamarkuptext}%
Como podemos observar, se trata de una clase que actúa sobre un 
  parámetro global de tipo \isa{{\isacharprime}a} cualquiera. Por otro lado, 
  \isa{instantation} define una clase de parámetros, en este caso los 
  números naturales \isa{nat} que devuelve como resultado. Incluye una 
  definición concreta del operador \isa{size{\isadigit{1}}} sobre dichos parámetros. 
  Además, el último \isa{instance} abre una prueba que afirma que los 
  parámetros dados conforman la clase especificada en la definición. 
  Esta prueba que nos afirma que está bien definida la clase aparece
  omitida utilizando \isa{{\isachardot}{\isachardot}} .

  En particular, sobre una fórmula nos devuelve el número de elementos 
  de la lista cuyos elementos son los nodos y las hojas de su árbol de 
  formación.%
\end{isamarkuptext}\isamarkuptrue%
\isacommand{value}\isamarkupfalse%
\ {\isachardoublequoteopen}size\ {\isacharparenleft}n{\isacharcolon}{\isacharcolon}nat{\isacharparenright}\ {\isacharequal}\ n{\isachardoublequoteclose}\isanewline
\isacommand{value}\isamarkupfalse%
\ {\isachardoublequoteopen}size\ {\isacharparenleft}{\isadigit{5}}{\isacharcolon}{\isacharcolon}nat{\isacharparenright}\ {\isacharequal}\ {\isadigit{5}}{\isachardoublequoteclose}%
\begin{isamarkuptext}%
Por otro lado, la función \isa{length} de la teoría 
  \href{http://cort.as/-Stfm}{List.thy} nos indica la longitud de una 
  lista cualquiera de elementos, definiéndose utilizando el comando
  \isa{size} visto anteriormente.%
\end{isamarkuptext}\isamarkuptrue%
\isacommand{abbreviation}\isamarkupfalse%
\ length{\isacharprime}\ {\isacharcolon}{\isacharcolon}\ {\isachardoublequoteopen}{\isacharprime}a\ list\ {\isasymRightarrow}\ nat{\isachardoublequoteclose}\ \isakeyword{where}\isanewline
\ \ {\isachardoublequoteopen}length{\isacharprime}\ {\isasymequiv}\ size{\isachardoublequoteclose}%
\begin{isamarkuptext}%
Las demostración del resultado se vuelve a basar en la inducción 
  que nos despliega seis casos. 

  La prueba estructurada no resulta interesante, pues todos los casos 
  son inmediatos por simplificación como en el primer lema de esta 
  sección. 

  Incluimos a continuación la prueba automática.%
\end{isamarkuptext}\isamarkuptrue%
\isacommand{lemma}\isamarkupfalse%
\ length{\isacharunderscore}subformulae{\isacharcolon}\ {\isachardoublequoteopen}length\ {\isacharparenleft}subformulae\ F{\isacharparenright}\ {\isacharequal}\ size\ F{\isachardoublequoteclose}\ \isanewline
%
\isadelimproof
\ \ %
\endisadelimproof
%
\isatagproof
\isacommand{by}\isamarkupfalse%
\ {\isacharparenleft}induction\ F{\isacharsemicolon}\ simp{\isacharparenright}%
\endisatagproof
{\isafoldproof}%
%
\isadelimproof
%
\endisadelimproof
%
\begin{isamarkuptext}%
\comentario{Hacer la prueba detallada para mostrar los teoremas 
  utilizados.}%
\end{isamarkuptext}\isamarkuptrue%
%
\isadelimdocument
%
\endisadelimdocument
%
\isatagdocument
%
\isamarkupsection{Conectivas derivadas%
}
\isamarkuptrue%
%
\endisatagdocument
{\isafolddocument}%
%
\isadelimdocument
%
\endisadelimdocument
%
\begin{isamarkuptext}%
En esta sección definiremos nuevas conectivas y fórmulas a partir 
  de los ya definidos en el apartado anterior, junto con varios 
  resultados sobre los mismos. Veamos el primero.

  \begin{definicion}
    Se define la fórmula \isa{{\isasymtop}} como la implicación \isa{{\isasymbottom}\ {\isasymlongrightarrow}\ {\isasymbottom}}.
  \end{definicion}

  Se formaliza del siguiente modo.%
\end{isamarkuptext}\isamarkuptrue%
\isacommand{definition}\isamarkupfalse%
\ Top\ {\isacharparenleft}{\isachardoublequoteopen}{\isasymtop}{\isachardoublequoteclose}{\isacharparenright}\ \isakeyword{where}\isanewline
\ \ {\isachardoublequoteopen}{\isasymtop}\ {\isasymequiv}\ {\isasymbottom}\ \isactrlbold {\isasymrightarrow}\ {\isasymbottom}{\isachardoublequoteclose}%
\begin{isamarkuptext}%
Como podemos observar, se define mediante una relación de 
  equivalencia con otra fórmula ya conocida. El uso de dicha 
  equivalencia justifica el tipo \isa{definition} empleado en este 
  caso. 

  Por la propia definición, es claro que \isa{{\isasymtop}} no contiene ninguna
  variable proposicional, como se verifica a continuación en Isabelle.%
\end{isamarkuptext}\isamarkuptrue%
\isacommand{lemma}\isamarkupfalse%
\ {\isachardoublequoteopen}atoms\ {\isasymtop}\ {\isacharequal}\ {\isasymemptyset}{\isachardoublequoteclose}\isanewline
%
\isadelimproof
\ \ \ %
\endisadelimproof
%
\isatagproof
\isacommand{by}\isamarkupfalse%
\ {\isacharparenleft}simp\ only{\isacharcolon}\ Top{\isacharunderscore}def\ formula{\isachardot}set\ Un{\isacharunderscore}absorb{\isacharparenright}%
\endisatagproof
{\isafoldproof}%
%
\isadelimproof
%
\endisadelimproof
%
\begin{isamarkuptext}%
\comentario{Añadir regla al glosario.}%
\end{isamarkuptext}\isamarkuptrue%
%
\begin{isamarkuptext}%
\comentario{Añadir la doble implicación como conectiva derivada.}%
\end{isamarkuptext}\isamarkuptrue%
%
\begin{isamarkuptext}%
A continuación vamos a definir dos conectivas que generalizan la 
  conjunción y la disyunción para una lista finita de fórmulas. En
  particular, al ser aplicadas a listas, se definen conforme a la 
  estructura recursiva de las mismas que se muestra a continuación. 
  
  \begin{definicion}
    Las listas de fórmulas se definen recursivamente como sigue.
    \begin{itemize}
      \item La lista vacía es una lista.
      \item Sea \isa{F} una fórmula y \isa{Fs} una lista de fórmulas. Entonces,
        \isa{F{\isacharhash}Fs} es una lista de fórmulas.
    \end{itemize}
  \end{definicion} 

\comentario{Esta definición es un caso particular de listas. 
  No se si incluir la definicion de estructura e inducción general}

  De este modo, se definen las conectivas plurales de acuerdo a la 
  estructura recursiva anterior. Notemos que al referirnos simplemente 
  a disyunción o conjunción nos referiremos a la de dos elementos.

  \begin{definicion}
  La conjunción plural de una lista de fórmulas se define recursivamente
  como:
    \begin{itemize}
      \item La conjunción plural de la lista vacía es \isa{{\isasymnot}{\isasymbottom}}.
      \item Sea \isa{F} una fórmula y \isa{Fs} una lista de fórmulas. Entonces,
  la conjunción plural de \isa{F{\isacharhash}Fs} es la conjunción de \isa{F} con la 
  conjunción plural de \isa{Fs}.
    \end{itemize}
  \end{definicion}

  \begin{definicion}
  La disyunción plural de una lista de fórmulas se define recursivamente
  como:
    \begin{itemize}
      \item La disyunción plural de la lista vacía es \isa{{\isasymnot}{\isasymbottom}}.
      \item Sea \isa{F} una fórmula y \isa{Fs} una lista de fórmulas. Entonces,
  la disyunción plural de \isa{F{\isacharhash}Fs} es la disyunción de \isa{F} con la 
  disyunción plural de \isa{Fs}.
    \end{itemize}
  \end{definicion}

  Su formalización en Isabelle es la siguiente.

  \comentario{Da error que no localizo}%
\end{isamarkuptext}\isamarkuptrue%
%
\begin{isamarkuptext}%
Ambas nuevas conectivas se definen con el tipo funciones 
  primitivas recursivas. Estas se basan en los dos casos descritos
  anteriormente: la lista vacía representada como \isa{Nil} y la lista
  construida añadiendo una fórmula a una lista de fórmulas. 
  Además, se observa en cada conectiva plural el nuevo símbolo de 
  notación que aparece entre paréntesis.

  Por otro lado, como es habitual, estas definiciones recursivas llevan
  asociado el correspondiente esquema inductivo de demostración. En
  este caso, se trata de la inducción para la estructura de lista de 
  fórmulas.

  \begin{definicion}
    Sea \isa{{\isasymP}} una propiedad sobre lista de fórmulas proposicionales que 
    verifica las siguientes condiciones:
    \begin{itemize}
     \item La lista vacía la verifica.
     \item Dada una fórmula \isa{F} y una lista de fórmulas \isa{Fs} que la
      verifican, entonces \isa{F{\isacharhash}Fs} la verifica.
    \end{itemize}
    Entonces, todas las listas de fórmulas proposicionales tienen la 
    propiedad \isa{{\isasymP}}. 
  \end{definicion}

  La conjunción plural nos da el siguiente resultado.

\comentario{Añadir lema a mano y demostración. Falta demostración en Isabelle.}%
\end{isamarkuptext}\isamarkuptrue%
%
\isadelimtheory
%
\endisadelimtheory
%
\isatagtheory
%
\endisatagtheory
{\isafoldtheory}%
%
\isadelimtheory
%
\endisadelimtheory
%
\end{isabellebody}%
\endinput
%:%file=~/TFG/Logica_Proposicional/Sintaxis.thy%:%
%:%24=13%:%
%:%36=15%:%
%:%37=16%:%
%:%38=17%:%
%:%39=18%:%
%:%40=19%:%
%:%42=21%:%
%:%43=21%:%
%:%45=23%:%
%:%46=24%:%
%:%47=25%:%
%:%48=26%:%
%:%49=27%:%
%:%50=28%:%
%:%51=29%:%
%:%52=30%:%
%:%53=31%:%
%:%54=32%:%
%:%55=33%:%
%:%56=34%:%
%:%57=35%:%
%:%58=36%:%
%:%59=37%:%
%:%60=38%:%
%:%61=39%:%
%:%62=40%:%
%:%63=41%:%
%:%64=42%:%
%:%65=43%:%
%:%66=44%:%
%:%67=45%:%
%:%68=46%:%
%:%69=47%:%
%:%70=48%:%
%:%71=49%:%
%:%72=50%:%
%:%73=51%:%
%:%74=52%:%
%:%75=53%:%
%:%76=54%:%
%:%77=55%:%
%:%78=56%:%
%:%79=57%:%
%:%80=58%:%
%:%81=59%:%
%:%82=60%:%
%:%83=61%:%
%:%84=62%:%
%:%85=63%:%
%:%86=64%:%
%:%87=65%:%
%:%88=66%:%
%:%89=67%:%
%:%90=68%:%
%:%91=69%:%
%:%92=70%:%
%:%93=71%:%
%:%94=72%:%
%:%96=74%:%
%:%97=74%:%
%:%98=75%:%
%:%99=76%:%
%:%100=77%:%
%:%101=78%:%
%:%102=79%:%
%:%103=80%:%
%:%105=82%:%
%:%106=83%:%
%:%107=84%:%
%:%108=85%:%
%:%109=86%:%
%:%110=87%:%
%:%111=88%:%
%:%112=89%:%
%:%113=90%:%
%:%114=91%:%
%:%115=92%:%
%:%116=93%:%
%:%117=94%:%
%:%118=95%:%
%:%119=96%:%
%:%120=97%:%
%:%121=98%:%
%:%122=99%:%
%:%123=100%:%
%:%124=101%:%
%:%125=102%:%
%:%126=103%:%
%:%127=104%:%
%:%128=105%:%
%:%129=106%:%
%:%130=107%:%
%:%131=108%:%
%:%132=109%:%
%:%133=110%:%
%:%134=111%:%
%:%135=112%:%
%:%136=113%:%
%:%137=114%:%
%:%142=114%:%
%:%143=115%:%
%:%144=116%:%
%:%145=117%:%
%:%146=118%:%
%:%147=119%:%
%:%148=120%:%
%:%150=122%:%
%:%151=122%:%
%:%152=123%:%
%:%155=124%:%
%:%159=124%:%
%:%160=124%:%
%:%161=125%:%
%:%162=126%:%
%:%163=126%:%
%:%164=127%:%
%:%165=127%:%
%:%166=128%:%
%:%167=129%:%
%:%168=129%:%
%:%169=130%:%
%:%170=130%:%
%:%171=131%:%
%:%172=132%:%
%:%173=132%:%
%:%174=133%:%
%:%175=133%:%
%:%176=134%:%
%:%177=135%:%
%:%178=135%:%
%:%179=136%:%
%:%180=136%:%
%:%185=136%:%
%:%188=137%:%
%:%191=139%:%
%:%192=140%:%
%:%193=141%:%
%:%195=143%:%
%:%196=143%:%
%:%197=144%:%
%:%200=145%:%
%:%204=145%:%
%:%205=145%:%
%:%206=146%:%
%:%207=147%:%
%:%208=147%:%
%:%209=148%:%
%:%210=148%:%
%:%211=149%:%
%:%212=150%:%
%:%213=150%:%
%:%214=151%:%
%:%215=151%:%
%:%216=152%:%
%:%217=152%:%
%:%218=153%:%
%:%219=153%:%
%:%220=154%:%
%:%221=154%:%
%:%222=154%:%
%:%223=155%:%
%:%224=155%:%
%:%225=156%:%
%:%226=156%:%
%:%227=156%:%
%:%228=157%:%
%:%229=157%:%
%:%230=158%:%
%:%231=158%:%
%:%232=158%:%
%:%233=159%:%
%:%234=159%:%
%:%235=160%:%
%:%236=160%:%
%:%237=160%:%
%:%238=161%:%
%:%239=161%:%
%:%240=162%:%
%:%241=162%:%
%:%242=163%:%
%:%243=164%:%
%:%244=164%:%
%:%245=165%:%
%:%246=165%:%
%:%251=165%:%
%:%254=166%:%
%:%255=166%:%
%:%256=167%:%
%:%257=168%:%
%:%258=168%:%
%:%260=170%:%
%:%261=171%:%
%:%262=172%:%
%:%263=173%:%
%:%264=174%:%
%:%265=175%:%
%:%266=176%:%
%:%267=177%:%
%:%268=178%:%
%:%269=179%:%
%:%270=180%:%
%:%271=181%:%
%:%272=182%:%
%:%273=183%:%
%:%274=184%:%
%:%275=185%:%
%:%276=186%:%
%:%277=187%:%
%:%278=188%:%
%:%279=189%:%
%:%280=190%:%
%:%281=191%:%
%:%282=192%:%
%:%283=193%:%
%:%284=194%:%
%:%285=195%:%
%:%286=196%:%
%:%287=197%:%
%:%288=198%:%
%:%289=199%:%
%:%290=200%:%
%:%291=201%:%
%:%292=202%:%
%:%293=203%:%
%:%294=204%:%
%:%295=205%:%
%:%296=206%:%
%:%297=207%:%
%:%298=208%:%
%:%299=209%:%
%:%300=210%:%
%:%301=211%:%
%:%302=212%:%
%:%303=213%:%
%:%304=214%:%
%:%305=215%:%
%:%306=216%:%
%:%307=217%:%
%:%308=218%:%
%:%309=219%:%
%:%310=220%:%
%:%311=221%:%
%:%312=222%:%
%:%313=223%:%
%:%314=224%:%
%:%315=225%:%
%:%316=226%:%
%:%317=227%:%
%:%318=228%:%
%:%319=229%:%
%:%320=230%:%
%:%321=231%:%
%:%323=233%:%
%:%324=233%:%
%:%327=234%:%
%:%331=234%:%
%:%341=236%:%
%:%342=237%:%
%:%343=238%:%
%:%345=240%:%
%:%346=240%:%
%:%347=241%:%
%:%348=242%:%
%:%350=244%:%
%:%351=245%:%
%:%352=246%:%
%:%353=247%:%
%:%354=248%:%
%:%355=249%:%
%:%356=250%:%
%:%357=251%:%
%:%358=252%:%
%:%359=253%:%
%:%360=254%:%
%:%361=255%:%
%:%362=256%:%
%:%363=257%:%
%:%364=258%:%
%:%365=259%:%
%:%366=260%:%
%:%367=261%:%
%:%368=262%:%
%:%369=263%:%
%:%370=264%:%
%:%371=265%:%
%:%372=266%:%
%:%373=267%:%
%:%374=268%:%
%:%375=269%:%
%:%376=270%:%
%:%377=271%:%
%:%378=272%:%
%:%379=273%:%
%:%380=274%:%
%:%381=275%:%
%:%382=276%:%
%:%383=277%:%
%:%384=278%:%
%:%385=279%:%
%:%386=280%:%
%:%387=281%:%
%:%388=282%:%
%:%389=283%:%
%:%391=285%:%
%:%392=285%:%
%:%393=286%:%
%:%400=287%:%
%:%401=287%:%
%:%402=288%:%
%:%403=288%:%
%:%404=289%:%
%:%405=289%:%
%:%406=290%:%
%:%407=290%:%
%:%408=290%:%
%:%409=291%:%
%:%410=291%:%
%:%411=292%:%
%:%412=292%:%
%:%413=292%:%
%:%414=293%:%
%:%415=293%:%
%:%416=294%:%
%:%422=294%:%
%:%425=295%:%
%:%426=296%:%
%:%427=296%:%
%:%428=297%:%
%:%435=298%:%
%:%436=298%:%
%:%437=299%:%
%:%438=299%:%
%:%439=300%:%
%:%440=300%:%
%:%441=301%:%
%:%442=301%:%
%:%443=301%:%
%:%444=302%:%
%:%445=302%:%
%:%446=303%:%
%:%452=303%:%
%:%455=304%:%
%:%456=305%:%
%:%457=305%:%
%:%458=306%:%
%:%459=307%:%
%:%462=308%:%
%:%466=308%:%
%:%467=308%:%
%:%468=309%:%
%:%469=309%:%
%:%474=309%:%
%:%477=310%:%
%:%478=311%:%
%:%479=311%:%
%:%480=312%:%
%:%481=313%:%
%:%482=314%:%
%:%489=315%:%
%:%490=315%:%
%:%491=316%:%
%:%492=316%:%
%:%493=317%:%
%:%494=317%:%
%:%495=318%:%
%:%496=318%:%
%:%497=319%:%
%:%498=319%:%
%:%499=319%:%
%:%500=320%:%
%:%501=320%:%
%:%502=321%:%
%:%508=321%:%
%:%511=322%:%
%:%512=323%:%
%:%513=323%:%
%:%514=324%:%
%:%515=325%:%
%:%516=326%:%
%:%523=327%:%
%:%524=327%:%
%:%525=328%:%
%:%526=328%:%
%:%527=329%:%
%:%528=329%:%
%:%529=330%:%
%:%530=330%:%
%:%531=331%:%
%:%532=331%:%
%:%533=331%:%
%:%534=332%:%
%:%535=332%:%
%:%536=333%:%
%:%542=333%:%
%:%545=334%:%
%:%546=335%:%
%:%547=335%:%
%:%548=336%:%
%:%549=337%:%
%:%550=338%:%
%:%557=339%:%
%:%558=339%:%
%:%559=340%:%
%:%560=340%:%
%:%561=341%:%
%:%562=341%:%
%:%563=342%:%
%:%564=342%:%
%:%565=343%:%
%:%566=343%:%
%:%567=343%:%
%:%568=344%:%
%:%569=344%:%
%:%570=345%:%
%:%576=345%:%
%:%579=346%:%
%:%580=347%:%
%:%581=347%:%
%:%588=348%:%
%:%589=348%:%
%:%590=349%:%
%:%591=349%:%
%:%592=350%:%
%:%593=350%:%
%:%594=350%:%
%:%595=350%:%
%:%596=351%:%
%:%597=351%:%
%:%598=352%:%
%:%599=352%:%
%:%600=353%:%
%:%601=353%:%
%:%602=353%:%
%:%603=353%:%
%:%604=354%:%
%:%605=354%:%
%:%606=355%:%
%:%607=355%:%
%:%608=356%:%
%:%609=356%:%
%:%610=356%:%
%:%611=356%:%
%:%612=357%:%
%:%613=357%:%
%:%614=358%:%
%:%615=358%:%
%:%616=359%:%
%:%617=359%:%
%:%618=359%:%
%:%619=359%:%
%:%620=360%:%
%:%621=360%:%
%:%622=361%:%
%:%623=361%:%
%:%624=362%:%
%:%625=362%:%
%:%626=362%:%
%:%627=362%:%
%:%628=363%:%
%:%629=363%:%
%:%630=364%:%
%:%631=364%:%
%:%632=365%:%
%:%633=365%:%
%:%634=365%:%
%:%635=365%:%
%:%636=366%:%
%:%646=368%:%
%:%648=370%:%
%:%649=370%:%
%:%652=371%:%
%:%656=371%:%
%:%657=371%:%
%:%671=373%:%
%:%683=375%:%
%:%684=376%:%
%:%685=377%:%
%:%686=378%:%
%:%687=379%:%
%:%688=380%:%
%:%689=381%:%
%:%690=382%:%
%:%691=383%:%
%:%692=384%:%
%:%693=385%:%
%:%694=386%:%
%:%695=387%:%
%:%696=388%:%
%:%697=389%:%
%:%698=390%:%
%:%699=391%:%
%:%700=392%:%
%:%702=394%:%
%:%703=394%:%
%:%704=395%:%
%:%705=396%:%
%:%706=397%:%
%:%707=398%:%
%:%708=399%:%
%:%709=400%:%
%:%711=402%:%
%:%712=403%:%
%:%713=404%:%
%:%714=405%:%
%:%715=406%:%
%:%717=408%:%
%:%718=408%:%
%:%719=409%:%
%:%722=410%:%
%:%726=410%:%
%:%727=410%:%
%:%728=411%:%
%:%729=412%:%
%:%730=412%:%
%:%731=413%:%
%:%732=413%:%
%:%733=414%:%
%:%734=415%:%
%:%735=415%:%
%:%736=416%:%
%:%737=416%:%
%:%738=417%:%
%:%739=418%:%
%:%740=418%:%
%:%742=420%:%
%:%743=421%:%
%:%744=421%:%
%:%745=422%:%
%:%746=423%:%
%:%747=423%:%
%:%748=424%:%
%:%749=424%:%
%:%750=425%:%
%:%751=426%:%
%:%752=426%:%
%:%753=427%:%
%:%754=428%:%
%:%755=428%:%
%:%760=428%:%
%:%763=429%:%
%:%766=431%:%
%:%767=432%:%
%:%768=433%:%
%:%770=435%:%
%:%771=435%:%
%:%772=436%:%
%:%774=438%:%
%:%775=439%:%
%:%776=440%:%
%:%777=441%:%
%:%778=442%:%
%:%779=443%:%
%:%780=444%:%
%:%781=445%:%
%:%783=447%:%
%:%784=447%:%
%:%785=448%:%
%:%788=449%:%
%:%792=449%:%
%:%793=449%:%
%:%794=450%:%
%:%795=451%:%
%:%796=451%:%
%:%797=452%:%
%:%798=452%:%
%:%799=453%:%
%:%800=454%:%
%:%801=454%:%
%:%803=456%:%
%:%804=457%:%
%:%805=457%:%
%:%810=457%:%
%:%813=458%:%
%:%816=460%:%
%:%817=461%:%
%:%818=462%:%
%:%819=463%:%
%:%820=464%:%
%:%821=465%:%
%:%822=466%:%
%:%823=467%:%
%:%824=468%:%
%:%825=469%:%
%:%826=470%:%
%:%827=471%:%
%:%829=473%:%
%:%830=473%:%
%:%833=474%:%
%:%837=474%:%
%:%838=474%:%
%:%847=476%:%
%:%848=477%:%
%:%850=479%:%
%:%851=479%:%
%:%852=480%:%
%:%855=481%:%
%:%859=481%:%
%:%860=481%:%
%:%865=481%:%
%:%868=482%:%
%:%869=483%:%
%:%870=483%:%
%:%871=484%:%
%:%874=485%:%
%:%878=485%:%
%:%879=485%:%
%:%884=485%:%
%:%887=486%:%
%:%888=487%:%
%:%889=487%:%
%:%890=488%:%
%:%897=489%:%
%:%898=489%:%
%:%899=490%:%
%:%900=490%:%
%:%901=491%:%
%:%902=491%:%
%:%903=492%:%
%:%904=492%:%
%:%905=492%:%
%:%906=493%:%
%:%907=493%:%
%:%908=494%:%
%:%909=494%:%
%:%910=494%:%
%:%911=495%:%
%:%912=495%:%
%:%913=496%:%
%:%919=496%:%
%:%922=497%:%
%:%923=498%:%
%:%924=498%:%
%:%925=499%:%
%:%926=500%:%
%:%933=501%:%
%:%934=501%:%
%:%935=502%:%
%:%936=502%:%
%:%937=503%:%
%:%938=504%:%
%:%939=504%:%
%:%940=505%:%
%:%941=505%:%
%:%942=505%:%
%:%943=506%:%
%:%944=506%:%
%:%945=507%:%
%:%946=507%:%
%:%947=507%:%
%:%948=508%:%
%:%949=508%:%
%:%950=509%:%
%:%951=509%:%
%:%952=509%:%
%:%953=510%:%
%:%954=510%:%
%:%955=511%:%
%:%961=511%:%
%:%964=512%:%
%:%965=513%:%
%:%966=513%:%
%:%967=514%:%
%:%968=515%:%
%:%975=516%:%
%:%976=516%:%
%:%977=517%:%
%:%978=517%:%
%:%979=518%:%
%:%980=519%:%
%:%981=519%:%
%:%982=520%:%
%:%983=520%:%
%:%984=520%:%
%:%985=521%:%
%:%986=521%:%
%:%987=522%:%
%:%988=522%:%
%:%989=522%:%
%:%990=523%:%
%:%991=523%:%
%:%992=524%:%
%:%993=524%:%
%:%994=524%:%
%:%995=525%:%
%:%996=525%:%
%:%997=526%:%
%:%1003=526%:%
%:%1006=527%:%
%:%1007=528%:%
%:%1008=528%:%
%:%1009=529%:%
%:%1010=530%:%
%:%1017=531%:%
%:%1018=531%:%
%:%1019=532%:%
%:%1020=532%:%
%:%1021=533%:%
%:%1022=534%:%
%:%1023=534%:%
%:%1024=535%:%
%:%1025=535%:%
%:%1026=535%:%
%:%1027=536%:%
%:%1028=536%:%
%:%1029=537%:%
%:%1030=537%:%
%:%1031=537%:%
%:%1032=538%:%
%:%1033=538%:%
%:%1034=539%:%
%:%1035=539%:%
%:%1036=539%:%
%:%1037=540%:%
%:%1038=540%:%
%:%1039=541%:%
%:%1049=543%:%
%:%1050=544%:%
%:%1051=545%:%
%:%1052=546%:%
%:%1053=547%:%
%:%1054=548%:%
%:%1055=549%:%
%:%1056=550%:%
%:%1057=551%:%
%:%1058=552%:%
%:%1059=553%:%
%:%1060=554%:%
%:%1061=555%:%
%:%1062=556%:%
%:%1063=557%:%
%:%1064=558%:%
%:%1065=559%:%
%:%1066=560%:%
%:%1067=561%:%
%:%1068=562%:%
%:%1069=563%:%
%:%1070=564%:%
%:%1071=565%:%
%:%1072=566%:%
%:%1073=567%:%
%:%1074=568%:%
%:%1075=569%:%
%:%1076=570%:%
%:%1077=571%:%
%:%1078=572%:%
%:%1079=573%:%
%:%1080=574%:%
%:%1082=576%:%
%:%1083=576%:%
%:%1090=577%:%
%:%1091=577%:%
%:%1092=578%:%
%:%1093=578%:%
%:%1094=579%:%
%:%1095=579%:%
%:%1096=579%:%
%:%1097=580%:%
%:%1098=580%:%
%:%1099=581%:%
%:%1100=581%:%
%:%1101=582%:%
%:%1102=582%:%
%:%1103=583%:%
%:%1104=583%:%
%:%1105=583%:%
%:%1106=584%:%
%:%1107=584%:%
%:%1108=585%:%
%:%1109=585%:%
%:%1110=586%:%
%:%1111=586%:%
%:%1112=587%:%
%:%1113=587%:%
%:%1114=587%:%
%:%1115=588%:%
%:%1116=588%:%
%:%1117=589%:%
%:%1118=589%:%
%:%1119=590%:%
%:%1120=590%:%
%:%1121=591%:%
%:%1122=591%:%
%:%1123=591%:%
%:%1124=592%:%
%:%1125=592%:%
%:%1126=593%:%
%:%1127=593%:%
%:%1128=594%:%
%:%1129=594%:%
%:%1130=595%:%
%:%1131=595%:%
%:%1132=595%:%
%:%1133=596%:%
%:%1134=596%:%
%:%1135=597%:%
%:%1136=597%:%
%:%1137=598%:%
%:%1138=598%:%
%:%1139=599%:%
%:%1140=599%:%
%:%1141=599%:%
%:%1142=600%:%
%:%1143=600%:%
%:%1144=601%:%
%:%1154=603%:%
%:%1156=605%:%
%:%1157=605%:%
%:%1160=606%:%
%:%1164=606%:%
%:%1165=606%:%
%:%1174=608%:%
%:%1175=609%:%
%:%1176=610%:%
%:%1177=611%:%
%:%1178=612%:%
%:%1179=613%:%
%:%1180=614%:%
%:%1181=615%:%
%:%1183=617%:%
%:%1184=617%:%
%:%1185=618%:%
%:%1186=619%:%
%:%1193=620%:%
%:%1194=620%:%
%:%1195=621%:%
%:%1196=621%:%
%:%1197=622%:%
%:%1198=622%:%
%:%1199=623%:%
%:%1200=623%:%
%:%1201=624%:%
%:%1202=624%:%
%:%1203=624%:%
%:%1204=625%:%
%:%1205=625%:%
%:%1206=626%:%
%:%1212=626%:%
%:%1215=627%:%
%:%1216=628%:%
%:%1217=628%:%
%:%1218=629%:%
%:%1219=630%:%
%:%1226=631%:%
%:%1227=631%:%
%:%1228=632%:%
%:%1229=632%:%
%:%1230=633%:%
%:%1231=633%:%
%:%1232=634%:%
%:%1233=634%:%
%:%1234=635%:%
%:%1235=635%:%
%:%1236=635%:%
%:%1237=636%:%
%:%1238=636%:%
%:%1239=637%:%
%:%1249=639%:%
%:%1250=640%:%
%:%1252=642%:%
%:%1253=642%:%
%:%1256=643%:%
%:%1260=643%:%
%:%1261=643%:%
%:%1266=643%:%
%:%1269=644%:%
%:%1270=645%:%
%:%1271=645%:%
%:%1274=646%:%
%:%1278=646%:%
%:%1279=646%:%
%:%1288=648%:%
%:%1289=649%:%
%:%1290=650%:%
%:%1291=651%:%
%:%1292=652%:%
%:%1293=653%:%
%:%1294=654%:%
%:%1295=655%:%
%:%1296=656%:%
%:%1297=657%:%
%:%1298=658%:%
%:%1299=659%:%
%:%1300=660%:%
%:%1301=661%:%
%:%1302=662%:%
%:%1303=663%:%
%:%1304=664%:%
%:%1305=665%:%
%:%1306=666%:%
%:%1307=667%:%
%:%1308=668%:%
%:%1309=669%:%
%:%1310=670%:%
%:%1311=671%:%
%:%1312=672%:%
%:%1313=673%:%
%:%1314=674%:%
%:%1315=675%:%
%:%1316=676%:%
%:%1317=677%:%
%:%1318=678%:%
%:%1319=679%:%
%:%1320=680%:%
%:%1321=681%:%
%:%1322=682%:%
%:%1323=683%:%
%:%1324=684%:%
%:%1325=685%:%
%:%1326=686%:%
%:%1327=687%:%
%:%1328=688%:%
%:%1328=689%:%
%:%1329=690%:%
%:%1330=691%:%
%:%1331=692%:%
%:%1332=693%:%
%:%1334=695%:%
%:%1335=695%:%
%:%1338=696%:%
%:%1342=696%:%
%:%1352=698%:%
%:%1353=699%:%
%:%1354=700%:%
%:%1355=701%:%
%:%1356=702%:%
%:%1357=703%:%
%:%1358=704%:%
%:%1359=705%:%
%:%1360=706%:%
%:%1361=707%:%
%:%1362=708%:%
%:%1363=709%:%
%:%1364=710%:%
%:%1365=711%:%
%:%1367=713%:%
%:%1368=713%:%
%:%1369=714%:%
%:%1372=715%:%
%:%1376=715%:%
%:%1377=715%:%
%:%1378=716%:%
%:%1379=717%:%
%:%1380=717%:%
%:%1381=718%:%
%:%1382=718%:%
%:%1383=719%:%
%:%1384=720%:%
%:%1385=720%:%
%:%1386=721%:%
%:%1387=722%:%
%:%1388=722%:%
%:%1389=723%:%
%:%1390=724%:%
%:%1391=724%:%
%:%1392=725%:%
%:%1393=726%:%
%:%1394=726%:%
%:%1399=726%:%
%:%1402=727%:%
%:%1405=729%:%
%:%1406=730%:%
%:%1407=731%:%
%:%1408=732%:%
%:%1409=733%:%
%:%1410=734%:%
%:%1411=735%:%
%:%1412=736%:%
%:%1413=737%:%
%:%1414=738%:%
%:%1415=739%:%
%:%1416=740%:%
%:%1417=741%:%
%:%1418=742%:%
%:%1419=743%:%
%:%1421=745%:%
%:%1422=745%:%
%:%1423=746%:%
%:%1430=747%:%
%:%1431=747%:%
%:%1432=748%:%
%:%1433=748%:%
%:%1434=749%:%
%:%1435=749%:%
%:%1436=750%:%
%:%1437=750%:%
%:%1438=750%:%
%:%1439=751%:%
%:%1440=751%:%
%:%1441=752%:%
%:%1442=752%:%
%:%1443=752%:%
%:%1444=753%:%
%:%1445=753%:%
%:%1446=754%:%
%:%1447=754%:%
%:%1448=754%:%
%:%1449=755%:%
%:%1450=755%:%
%:%1451=756%:%
%:%1452=756%:%
%:%1453=756%:%
%:%1454=757%:%
%:%1455=757%:%
%:%1456=758%:%
%:%1457=758%:%
%:%1458=758%:%
%:%1459=759%:%
%:%1460=759%:%
%:%1461=760%:%
%:%1467=760%:%
%:%1470=761%:%
%:%1471=762%:%
%:%1472=762%:%
%:%1473=763%:%
%:%1480=764%:%
%:%1481=764%:%
%:%1482=765%:%
%:%1483=765%:%
%:%1484=766%:%
%:%1485=766%:%
%:%1486=767%:%
%:%1487=767%:%
%:%1488=767%:%
%:%1489=768%:%
%:%1490=768%:%
%:%1491=769%:%
%:%1492=769%:%
%:%1493=769%:%
%:%1494=770%:%
%:%1495=770%:%
%:%1496=771%:%
%:%1497=771%:%
%:%1498=771%:%
%:%1499=772%:%
%:%1500=772%:%
%:%1501=773%:%
%:%1507=773%:%
%:%1510=774%:%
%:%1511=775%:%
%:%1512=775%:%
%:%1513=776%:%
%:%1514=777%:%
%:%1521=778%:%
%:%1522=778%:%
%:%1523=779%:%
%:%1524=779%:%
%:%1525=780%:%
%:%1526=780%:%
%:%1527=781%:%
%:%1528=781%:%
%:%1529=781%:%
%:%1530=782%:%
%:%1531=782%:%
%:%1532=783%:%
%:%1533=783%:%
%:%1534=783%:%
%:%1535=784%:%
%:%1536=784%:%
%:%1537=785%:%
%:%1538=785%:%
%:%1539=785%:%
%:%1540=786%:%
%:%1541=786%:%
%:%1542=787%:%
%:%1543=787%:%
%:%1544=787%:%
%:%1545=788%:%
%:%1546=788%:%
%:%1547=789%:%
%:%1553=789%:%
%:%1556=790%:%
%:%1557=791%:%
%:%1558=791%:%
%:%1559=792%:%
%:%1560=793%:%
%:%1561=794%:%
%:%1568=795%:%
%:%1569=795%:%
%:%1570=796%:%
%:%1571=796%:%
%:%1572=797%:%
%:%1573=797%:%
%:%1574=798%:%
%:%1575=798%:%
%:%1576=798%:%
%:%1577=799%:%
%:%1578=799%:%
%:%1579=800%:%
%:%1580=800%:%
%:%1581=800%:%
%:%1582=801%:%
%:%1583=801%:%
%:%1584=802%:%
%:%1585=802%:%
%:%1586=803%:%
%:%1587=803%:%
%:%1588=803%:%
%:%1589=804%:%
%:%1590=804%:%
%:%1591=805%:%
%:%1592=805%:%
%:%1593=805%:%
%:%1594=806%:%
%:%1595=806%:%
%:%1596=807%:%
%:%1597=807%:%
%:%1598=807%:%
%:%1599=808%:%
%:%1600=808%:%
%:%1601=809%:%
%:%1607=809%:%
%:%1610=810%:%
%:%1611=811%:%
%:%1612=811%:%
%:%1613=812%:%
%:%1614=813%:%
%:%1615=814%:%
%:%1622=815%:%
%:%1623=815%:%
%:%1624=816%:%
%:%1625=816%:%
%:%1626=817%:%
%:%1627=817%:%
%:%1628=818%:%
%:%1629=818%:%
%:%1630=818%:%
%:%1631=819%:%
%:%1632=819%:%
%:%1633=820%:%
%:%1634=820%:%
%:%1635=820%:%
%:%1636=821%:%
%:%1637=821%:%
%:%1638=822%:%
%:%1639=822%:%
%:%1640=823%:%
%:%1641=823%:%
%:%1642=823%:%
%:%1643=824%:%
%:%1644=824%:%
%:%1645=825%:%
%:%1646=825%:%
%:%1647=825%:%
%:%1648=826%:%
%:%1649=826%:%
%:%1650=827%:%
%:%1651=827%:%
%:%1652=827%:%
%:%1653=828%:%
%:%1654=828%:%
%:%1655=829%:%
%:%1661=829%:%
%:%1664=830%:%
%:%1665=831%:%
%:%1666=831%:%
%:%1667=832%:%
%:%1668=833%:%
%:%1669=834%:%
%:%1676=835%:%
%:%1677=835%:%
%:%1678=836%:%
%:%1679=836%:%
%:%1680=837%:%
%:%1681=837%:%
%:%1682=838%:%
%:%1683=838%:%
%:%1684=838%:%
%:%1685=839%:%
%:%1686=839%:%
%:%1687=840%:%
%:%1688=840%:%
%:%1689=840%:%
%:%1690=841%:%
%:%1691=841%:%
%:%1692=842%:%
%:%1693=842%:%
%:%1694=843%:%
%:%1695=843%:%
%:%1696=843%:%
%:%1697=844%:%
%:%1698=844%:%
%:%1699=845%:%
%:%1700=845%:%
%:%1701=845%:%
%:%1702=846%:%
%:%1703=846%:%
%:%1704=847%:%
%:%1705=847%:%
%:%1706=847%:%
%:%1707=848%:%
%:%1708=848%:%
%:%1709=849%:%
%:%1715=849%:%
%:%1718=850%:%
%:%1719=851%:%
%:%1720=851%:%
%:%1721=852%:%
%:%1728=853%:%
%:%1729=853%:%
%:%1730=854%:%
%:%1731=854%:%
%:%1732=855%:%
%:%1733=855%:%
%:%1734=855%:%
%:%1735=855%:%
%:%1736=856%:%
%:%1737=856%:%
%:%1738=857%:%
%:%1739=857%:%
%:%1740=858%:%
%:%1741=858%:%
%:%1742=858%:%
%:%1743=858%:%
%:%1744=859%:%
%:%1745=859%:%
%:%1746=860%:%
%:%1747=860%:%
%:%1748=861%:%
%:%1749=861%:%
%:%1750=861%:%
%:%1751=861%:%
%:%1752=862%:%
%:%1753=862%:%
%:%1754=863%:%
%:%1755=863%:%
%:%1756=864%:%
%:%1757=864%:%
%:%1758=864%:%
%:%1759=864%:%
%:%1760=865%:%
%:%1761=865%:%
%:%1762=866%:%
%:%1763=866%:%
%:%1764=867%:%
%:%1765=867%:%
%:%1766=867%:%
%:%1767=867%:%
%:%1768=868%:%
%:%1769=868%:%
%:%1770=869%:%
%:%1771=869%:%
%:%1772=870%:%
%:%1773=870%:%
%:%1774=870%:%
%:%1775=870%:%
%:%1776=871%:%
%:%1786=873%:%
%:%1787=874%:%
%:%1789=876%:%
%:%1790=876%:%
%:%1793=877%:%
%:%1797=877%:%
%:%1798=877%:%
%:%1807=879%:%
%:%1808=880%:%
%:%1809=881%:%
%:%1810=882%:%
%:%1811=883%:%
%:%1812=884%:%
%:%1813=885%:%
%:%1814=886%:%
%:%1815=887%:%
%:%1816=888%:%
%:%1817=889%:%
%:%1818=890%:%
%:%1819=891%:%
%:%1820=892%:%
%:%1821=893%:%
%:%1822=894%:%
%:%1823=895%:%
%:%1824=896%:%
%:%1825=897%:%
%:%1826=898%:%
%:%1827=899%:%
%:%1828=900%:%
%:%1829=901%:%
%:%1830=902%:%
%:%1831=903%:%
%:%1832=904%:%
%:%1833=905%:%
%:%1834=906%:%
%:%1835=907%:%
%:%1836=908%:%
%:%1837=909%:%
%:%1838=910%:%
%:%1839=911%:%
%:%1840=912%:%
%:%1841=913%:%
%:%1842=914%:%
%:%1843=915%:%
%:%1844=916%:%
%:%1845=917%:%
%:%1846=918%:%
%:%1847=919%:%
%:%1848=920%:%
%:%1849=921%:%
%:%1850=922%:%
%:%1851=923%:%
%:%1852=924%:%
%:%1853=925%:%
%:%1854=926%:%
%:%1855=927%:%
%:%1856=928%:%
%:%1857=929%:%
%:%1858=930%:%
%:%1859=931%:%
%:%1860=932%:%
%:%1862=934%:%
%:%1863=934%:%
%:%1866=935%:%
%:%1870=935%:%
%:%1880=937%:%
%:%1882=939%:%
%:%1883=939%:%
%:%1884=940%:%
%:%1885=941%:%
%:%1892=942%:%
%:%1893=942%:%
%:%1894=943%:%
%:%1895=943%:%
%:%1896=944%:%
%:%1897=944%:%
%:%1898=945%:%
%:%1899=945%:%
%:%1900=946%:%
%:%1901=946%:%
%:%1902=946%:%
%:%1903=947%:%
%:%1904=947%:%
%:%1905=948%:%
%:%1906=948%:%
%:%1907=948%:%
%:%1908=949%:%
%:%1909=949%:%
%:%1910=950%:%
%:%1916=950%:%
%:%1919=951%:%
%:%1920=952%:%
%:%1921=952%:%
%:%1922=953%:%
%:%1923=954%:%
%:%1930=955%:%
%:%1931=955%:%
%:%1932=956%:%
%:%1933=956%:%
%:%1934=957%:%
%:%1935=957%:%
%:%1936=958%:%
%:%1937=958%:%
%:%1938=959%:%
%:%1939=959%:%
%:%1940=959%:%
%:%1941=960%:%
%:%1942=960%:%
%:%1943=961%:%
%:%1944=961%:%
%:%1945=961%:%
%:%1946=962%:%
%:%1947=962%:%
%:%1948=963%:%
%:%1954=963%:%
%:%1957=964%:%
%:%1958=965%:%
%:%1959=965%:%
%:%1960=966%:%
%:%1961=967%:%
%:%1962=968%:%
%:%1969=969%:%
%:%1970=969%:%
%:%1971=970%:%
%:%1972=970%:%
%:%1973=971%:%
%:%1974=971%:%
%:%1975=972%:%
%:%1976=972%:%
%:%1977=973%:%
%:%1978=973%:%
%:%1979=973%:%
%:%1980=974%:%
%:%1981=974%:%
%:%1982=975%:%
%:%1983=975%:%
%:%1984=975%:%
%:%1985=976%:%
%:%1986=976%:%
%:%1987=977%:%
%:%1988=977%:%
%:%1989=978%:%
%:%1990=978%:%
%:%1991=978%:%
%:%1992=979%:%
%:%1993=979%:%
%:%1994=980%:%
%:%1995=980%:%
%:%1996=980%:%
%:%1997=981%:%
%:%1998=981%:%
%:%1999=982%:%
%:%2000=982%:%
%:%2001=983%:%
%:%2002=983%:%
%:%2003=984%:%
%:%2004=984%:%
%:%2005=984%:%
%:%2006=985%:%
%:%2007=985%:%
%:%2008=986%:%
%:%2009=986%:%
%:%2010=986%:%
%:%2011=987%:%
%:%2012=987%:%
%:%2013=988%:%
%:%2014=988%:%
%:%2015=988%:%
%:%2016=989%:%
%:%2017=989%:%
%:%2018=990%:%
%:%2019=990%:%
%:%2020=991%:%
%:%2026=991%:%
%:%2029=992%:%
%:%2030=993%:%
%:%2031=993%:%
%:%2032=994%:%
%:%2033=995%:%
%:%2034=996%:%
%:%2035=997%:%
%:%2042=998%:%
%:%2043=998%:%
%:%2044=999%:%
%:%2045=999%:%
%:%2046=1000%:%
%:%2047=1000%:%
%:%2048=1001%:%
%:%2049=1001%:%
%:%2050=1002%:%
%:%2051=1002%:%
%:%2052=1002%:%
%:%2053=1003%:%
%:%2054=1003%:%
%:%2055=1004%:%
%:%2056=1004%:%
%:%2057=1004%:%
%:%2058=1005%:%
%:%2059=1005%:%
%:%2060=1006%:%
%:%2061=1006%:%
%:%2062=1007%:%
%:%2063=1007%:%
%:%2064=1007%:%
%:%2065=1008%:%
%:%2066=1008%:%
%:%2067=1009%:%
%:%2068=1009%:%
%:%2069=1009%:%
%:%2070=1010%:%
%:%2071=1010%:%
%:%2072=1011%:%
%:%2073=1011%:%
%:%2074=1012%:%
%:%2075=1012%:%
%:%2076=1013%:%
%:%2077=1013%:%
%:%2078=1013%:%
%:%2079=1014%:%
%:%2080=1014%:%
%:%2081=1015%:%
%:%2082=1015%:%
%:%2083=1015%:%
%:%2084=1016%:%
%:%2085=1016%:%
%:%2086=1017%:%
%:%2087=1017%:%
%:%2088=1018%:%
%:%2089=1018%:%
%:%2090=1018%:%
%:%2091=1019%:%
%:%2092=1019%:%
%:%2093=1020%:%
%:%2094=1020%:%
%:%2095=1020%:%
%:%2096=1021%:%
%:%2097=1021%:%
%:%2098=1022%:%
%:%2099=1022%:%
%:%2100=1022%:%
%:%2101=1023%:%
%:%2102=1023%:%
%:%2103=1024%:%
%:%2104=1024%:%
%:%2105=1024%:%
%:%2106=1025%:%
%:%2107=1025%:%
%:%2108=1026%:%
%:%2109=1026%:%
%:%2110=1027%:%
%:%2111=1027%:%
%:%2112=1028%:%
%:%2113=1028%:%
%:%2114=1028%:%
%:%2115=1029%:%
%:%2116=1029%:%
%:%2117=1030%:%
%:%2118=1030%:%
%:%2119=1030%:%
%:%2120=1031%:%
%:%2121=1031%:%
%:%2122=1032%:%
%:%2123=1032%:%
%:%2124=1032%:%
%:%2125=1033%:%
%:%2126=1033%:%
%:%2127=1034%:%
%:%2128=1034%:%
%:%2129=1034%:%
%:%2130=1035%:%
%:%2131=1035%:%
%:%2132=1036%:%
%:%2133=1036%:%
%:%2134=1037%:%
%:%2135=1037%:%
%:%2136=1038%:%
%:%2142=1038%:%
%:%2145=1039%:%
%:%2146=1040%:%
%:%2147=1040%:%
%:%2148=1041%:%
%:%2149=1042%:%
%:%2150=1043%:%
%:%2151=1044%:%
%:%2158=1045%:%
%:%2159=1045%:%
%:%2160=1046%:%
%:%2161=1046%:%
%:%2162=1047%:%
%:%2163=1047%:%
%:%2164=1048%:%
%:%2165=1048%:%
%:%2166=1049%:%
%:%2167=1049%:%
%:%2168=1049%:%
%:%2169=1050%:%
%:%2170=1050%:%
%:%2171=1051%:%
%:%2172=1051%:%
%:%2173=1051%:%
%:%2174=1052%:%
%:%2175=1052%:%
%:%2176=1053%:%
%:%2177=1053%:%
%:%2178=1054%:%
%:%2179=1054%:%
%:%2180=1054%:%
%:%2181=1055%:%
%:%2182=1055%:%
%:%2183=1056%:%
%:%2184=1056%:%
%:%2185=1056%:%
%:%2186=1057%:%
%:%2187=1057%:%
%:%2188=1058%:%
%:%2189=1058%:%
%:%2190=1059%:%
%:%2191=1059%:%
%:%2192=1060%:%
%:%2193=1060%:%
%:%2194=1060%:%
%:%2195=1061%:%
%:%2196=1061%:%
%:%2197=1062%:%
%:%2198=1062%:%
%:%2199=1062%:%
%:%2200=1063%:%
%:%2201=1063%:%
%:%2202=1064%:%
%:%2203=1064%:%
%:%2204=1065%:%
%:%2205=1065%:%
%:%2206=1065%:%
%:%2207=1066%:%
%:%2208=1066%:%
%:%2209=1067%:%
%:%2210=1067%:%
%:%2211=1067%:%
%:%2212=1068%:%
%:%2213=1068%:%
%:%2214=1069%:%
%:%2215=1069%:%
%:%2216=1069%:%
%:%2217=1070%:%
%:%2218=1070%:%
%:%2219=1071%:%
%:%2220=1071%:%
%:%2221=1071%:%
%:%2222=1072%:%
%:%2223=1072%:%
%:%2224=1073%:%
%:%2225=1073%:%
%:%2226=1074%:%
%:%2227=1074%:%
%:%2228=1075%:%
%:%2229=1075%:%
%:%2230=1075%:%
%:%2231=1076%:%
%:%2232=1076%:%
%:%2233=1077%:%
%:%2234=1077%:%
%:%2235=1077%:%
%:%2236=1078%:%
%:%2237=1078%:%
%:%2238=1079%:%
%:%2239=1079%:%
%:%2240=1079%:%
%:%2241=1080%:%
%:%2242=1080%:%
%:%2243=1081%:%
%:%2244=1081%:%
%:%2245=1081%:%
%:%2246=1082%:%
%:%2247=1082%:%
%:%2248=1083%:%
%:%2249=1083%:%
%:%2250=1084%:%
%:%2251=1084%:%
%:%2252=1085%:%
%:%2258=1085%:%
%:%2261=1086%:%
%:%2262=1087%:%
%:%2263=1087%:%
%:%2264=1088%:%
%:%2265=1089%:%
%:%2266=1090%:%
%:%2267=1091%:%
%:%2274=1092%:%
%:%2275=1092%:%
%:%2276=1093%:%
%:%2277=1093%:%
%:%2278=1094%:%
%:%2279=1094%:%
%:%2280=1095%:%
%:%2281=1095%:%
%:%2282=1096%:%
%:%2283=1096%:%
%:%2284=1096%:%
%:%2285=1097%:%
%:%2286=1097%:%
%:%2287=1098%:%
%:%2288=1098%:%
%:%2289=1098%:%
%:%2290=1099%:%
%:%2291=1099%:%
%:%2292=1100%:%
%:%2293=1100%:%
%:%2294=1101%:%
%:%2295=1101%:%
%:%2296=1101%:%
%:%2297=1102%:%
%:%2298=1102%:%
%:%2299=1103%:%
%:%2300=1103%:%
%:%2301=1103%:%
%:%2302=1104%:%
%:%2303=1104%:%
%:%2304=1105%:%
%:%2305=1105%:%
%:%2306=1106%:%
%:%2307=1106%:%
%:%2308=1107%:%
%:%2309=1107%:%
%:%2310=1107%:%
%:%2311=1108%:%
%:%2312=1108%:%
%:%2313=1109%:%
%:%2314=1109%:%
%:%2315=1109%:%
%:%2316=1110%:%
%:%2317=1110%:%
%:%2318=1111%:%
%:%2319=1111%:%
%:%2320=1112%:%
%:%2321=1112%:%
%:%2322=1112%:%
%:%2323=1113%:%
%:%2324=1113%:%
%:%2325=1114%:%
%:%2326=1114%:%
%:%2327=1114%:%
%:%2328=1115%:%
%:%2329=1115%:%
%:%2330=1116%:%
%:%2331=1116%:%
%:%2332=1116%:%
%:%2333=1117%:%
%:%2334=1117%:%
%:%2335=1118%:%
%:%2336=1118%:%
%:%2337=1118%:%
%:%2338=1119%:%
%:%2339=1119%:%
%:%2340=1120%:%
%:%2341=1120%:%
%:%2342=1121%:%
%:%2343=1121%:%
%:%2344=1122%:%
%:%2345=1122%:%
%:%2346=1122%:%
%:%2347=1123%:%
%:%2348=1123%:%
%:%2349=1124%:%
%:%2350=1124%:%
%:%2351=1124%:%
%:%2352=1125%:%
%:%2353=1125%:%
%:%2354=1126%:%
%:%2355=1126%:%
%:%2356=1126%:%
%:%2357=1127%:%
%:%2358=1127%:%
%:%2359=1128%:%
%:%2360=1128%:%
%:%2361=1128%:%
%:%2362=1129%:%
%:%2363=1129%:%
%:%2364=1130%:%
%:%2365=1130%:%
%:%2366=1131%:%
%:%2367=1131%:%
%:%2368=1132%:%
%:%2374=1132%:%
%:%2377=1133%:%
%:%2378=1134%:%
%:%2379=1134%:%
%:%2386=1135%:%
%:%2387=1135%:%
%:%2388=1136%:%
%:%2389=1136%:%
%:%2390=1137%:%
%:%2391=1137%:%
%:%2392=1137%:%
%:%2393=1137%:%
%:%2394=1138%:%
%:%2395=1138%:%
%:%2396=1139%:%
%:%2397=1139%:%
%:%2398=1140%:%
%:%2399=1140%:%
%:%2400=1140%:%
%:%2401=1140%:%
%:%2402=1141%:%
%:%2403=1141%:%
%:%2404=1142%:%
%:%2405=1142%:%
%:%2406=1143%:%
%:%2407=1143%:%
%:%2408=1143%:%
%:%2409=1143%:%
%:%2410=1144%:%
%:%2411=1144%:%
%:%2412=1145%:%
%:%2413=1145%:%
%:%2414=1146%:%
%:%2415=1146%:%
%:%2416=1146%:%
%:%2417=1146%:%
%:%2418=1147%:%
%:%2419=1147%:%
%:%2420=1148%:%
%:%2421=1148%:%
%:%2422=1149%:%
%:%2423=1149%:%
%:%2424=1149%:%
%:%2425=1149%:%
%:%2426=1150%:%
%:%2427=1150%:%
%:%2428=1151%:%
%:%2429=1151%:%
%:%2430=1152%:%
%:%2431=1152%:%
%:%2432=1152%:%
%:%2433=1152%:%
%:%2434=1153%:%
%:%2444=1155%:%
%:%2446=1157%:%
%:%2447=1157%:%
%:%2450=1158%:%
%:%2454=1158%:%
%:%2455=1158%:%
%:%2464=1160%:%
%:%2465=1161%:%
%:%2466=1162%:%
%:%2467=1163%:%
%:%2468=1164%:%
%:%2469=1165%:%
%:%2470=1166%:%
%:%2471=1167%:%
%:%2472=1168%:%
%:%2473=1169%:%
%:%2474=1170%:%
%:%2475=1171%:%
%:%2476=1172%:%
%:%2477=1173%:%
%:%2478=1174%:%
%:%2479=1175%:%
%:%2480=1176%:%
%:%2481=1177%:%
%:%2482=1178%:%
%:%2483=1179%:%
%:%2484=1180%:%
%:%2485=1181%:%
%:%2486=1182%:%
%:%2487=1183%:%
%:%2488=1184%:%
%:%2489=1185%:%
%:%2490=1186%:%
%:%2491=1187%:%
%:%2492=1188%:%
%:%2493=1189%:%
%:%2494=1190%:%
%:%2495=1191%:%
%:%2496=1192%:%
%:%2497=1193%:%
%:%2498=1194%:%
%:%2499=1195%:%
%:%2500=1196%:%
%:%2501=1197%:%
%:%2502=1198%:%
%:%2503=1199%:%
%:%2504=1200%:%
%:%2505=1201%:%
%:%2506=1202%:%
%:%2507=1203%:%
%:%2508=1204%:%
%:%2509=1205%:%
%:%2510=1206%:%
%:%2514=1208%:%
%:%2515=1209%:%
%:%2517=1211%:%
%:%2518=1211%:%
%:%2519=1212%:%
%:%2520=1213%:%
%:%2527=1214%:%
%:%2528=1214%:%
%:%2529=1215%:%
%:%2530=1215%:%
%:%2531=1215%:%
%:%2532=1216%:%
%:%2533=1216%:%
%:%2534=1217%:%
%:%2535=1217%:%
%:%2536=1217%:%
%:%2537=1218%:%
%:%2538=1218%:%
%:%2539=1219%:%
%:%2540=1219%:%
%:%2541=1219%:%
%:%2542=1220%:%
%:%2543=1220%:%
%:%2544=1221%:%
%:%2550=1221%:%
%:%2553=1222%:%
%:%2554=1223%:%
%:%2555=1223%:%
%:%2556=1224%:%
%:%2557=1225%:%
%:%2564=1226%:%
%:%2565=1226%:%
%:%2566=1227%:%
%:%2567=1227%:%
%:%2568=1228%:%
%:%2569=1228%:%
%:%2570=1229%:%
%:%2571=1229%:%
%:%2572=1230%:%
%:%2573=1230%:%
%:%2574=1230%:%
%:%2575=1231%:%
%:%2576=1231%:%
%:%2577=1232%:%
%:%2578=1232%:%
%:%2579=1232%:%
%:%2580=1233%:%
%:%2581=1233%:%
%:%2582=1234%:%
%:%2588=1234%:%
%:%2591=1235%:%
%:%2592=1236%:%
%:%2593=1236%:%
%:%2594=1237%:%
%:%2595=1238%:%
%:%2596=1239%:%
%:%2603=1240%:%
%:%2604=1240%:%
%:%2605=1241%:%
%:%2606=1241%:%
%:%2607=1242%:%
%:%2608=1242%:%
%:%2609=1243%:%
%:%2610=1243%:%
%:%2611=1244%:%
%:%2612=1244%:%
%:%2613=1244%:%
%:%2614=1245%:%
%:%2615=1245%:%
%:%2616=1246%:%
%:%2617=1246%:%
%:%2618=1246%:%
%:%2619=1247%:%
%:%2620=1247%:%
%:%2621=1248%:%
%:%2622=1248%:%
%:%2623=1249%:%
%:%2624=1249%:%
%:%2625=1249%:%
%:%2626=1250%:%
%:%2627=1250%:%
%:%2628=1251%:%
%:%2629=1251%:%
%:%2630=1251%:%
%:%2631=1252%:%
%:%2632=1252%:%
%:%2633=1253%:%
%:%2634=1253%:%
%:%2635=1254%:%
%:%2636=1254%:%
%:%2637=1255%:%
%:%2638=1255%:%
%:%2639=1255%:%
%:%2640=1256%:%
%:%2641=1256%:%
%:%2642=1257%:%
%:%2643=1257%:%
%:%2644=1257%:%
%:%2645=1258%:%
%:%2646=1258%:%
%:%2647=1259%:%
%:%2648=1259%:%
%:%2649=1259%:%
%:%2650=1260%:%
%:%2651=1260%:%
%:%2652=1260%:%
%:%2653=1261%:%
%:%2654=1261%:%
%:%2655=1262%:%
%:%2661=1262%:%
%:%2664=1263%:%
%:%2665=1264%:%
%:%2666=1264%:%
%:%2667=1265%:%
%:%2668=1266%:%
%:%2669=1267%:%
%:%2670=1268%:%
%:%2671=1269%:%
%:%2672=1270%:%
%:%2679=1271%:%
%:%2680=1271%:%
%:%2681=1272%:%
%:%2682=1272%:%
%:%2683=1273%:%
%:%2684=1273%:%
%:%2685=1274%:%
%:%2686=1274%:%
%:%2687=1275%:%
%:%2688=1275%:%
%:%2689=1275%:%
%:%2690=1276%:%
%:%2691=1276%:%
%:%2692=1277%:%
%:%2693=1277%:%
%:%2694=1277%:%
%:%2695=1278%:%
%:%2696=1278%:%
%:%2697=1279%:%
%:%2698=1279%:%
%:%2699=1280%:%
%:%2700=1280%:%
%:%2701=1280%:%
%:%2702=1281%:%
%:%2703=1281%:%
%:%2704=1282%:%
%:%2705=1282%:%
%:%2706=1282%:%
%:%2707=1283%:%
%:%2708=1283%:%
%:%2709=1284%:%
%:%2710=1284%:%
%:%2711=1285%:%
%:%2712=1285%:%
%:%2713=1286%:%
%:%2714=1286%:%
%:%2715=1286%:%
%:%2716=1287%:%
%:%2717=1287%:%
%:%2718=1288%:%
%:%2719=1288%:%
%:%2720=1288%:%
%:%2721=1289%:%
%:%2722=1289%:%
%:%2723=1290%:%
%:%2724=1290%:%
%:%2725=1291%:%
%:%2726=1291%:%
%:%2727=1291%:%
%:%2728=1292%:%
%:%2729=1292%:%
%:%2730=1293%:%
%:%2731=1293%:%
%:%2732=1293%:%
%:%2733=1294%:%
%:%2734=1294%:%
%:%2735=1295%:%
%:%2736=1295%:%
%:%2737=1295%:%
%:%2738=1296%:%
%:%2739=1296%:%
%:%2740=1297%:%
%:%2741=1297%:%
%:%2742=1297%:%
%:%2743=1298%:%
%:%2744=1298%:%
%:%2745=1299%:%
%:%2746=1299%:%
%:%2747=1300%:%
%:%2748=1300%:%
%:%2749=1301%:%
%:%2750=1301%:%
%:%2751=1301%:%
%:%2752=1302%:%
%:%2753=1302%:%
%:%2754=1303%:%
%:%2755=1303%:%
%:%2756=1303%:%
%:%2757=1304%:%
%:%2758=1304%:%
%:%2759=1305%:%
%:%2760=1305%:%
%:%2761=1305%:%
%:%2762=1306%:%
%:%2763=1306%:%
%:%2764=1307%:%
%:%2765=1307%:%
%:%2766=1307%:%
%:%2767=1308%:%
%:%2768=1308%:%
%:%2769=1309%:%
%:%2770=1309%:%
%:%2771=1310%:%
%:%2772=1310%:%
%:%2773=1311%:%
%:%2779=1311%:%
%:%2782=1312%:%
%:%2783=1313%:%
%:%2784=1313%:%
%:%2785=1314%:%
%:%2786=1315%:%
%:%2787=1316%:%
%:%2788=1317%:%
%:%2789=1318%:%
%:%2790=1319%:%
%:%2797=1320%:%
%:%2798=1320%:%
%:%2799=1321%:%
%:%2800=1321%:%
%:%2801=1322%:%
%:%2802=1322%:%
%:%2803=1323%:%
%:%2804=1323%:%
%:%2805=1324%:%
%:%2806=1324%:%
%:%2807=1324%:%
%:%2808=1325%:%
%:%2809=1325%:%
%:%2810=1326%:%
%:%2811=1326%:%
%:%2812=1326%:%
%:%2813=1327%:%
%:%2814=1327%:%
%:%2815=1328%:%
%:%2816=1328%:%
%:%2817=1329%:%
%:%2818=1329%:%
%:%2819=1329%:%
%:%2820=1330%:%
%:%2821=1330%:%
%:%2822=1331%:%
%:%2823=1331%:%
%:%2824=1331%:%
%:%2825=1332%:%
%:%2826=1332%:%
%:%2827=1333%:%
%:%2828=1333%:%
%:%2829=1334%:%
%:%2830=1334%:%
%:%2831=1335%:%
%:%2832=1335%:%
%:%2833=1335%:%
%:%2834=1336%:%
%:%2835=1336%:%
%:%2836=1337%:%
%:%2837=1337%:%
%:%2838=1337%:%
%:%2839=1338%:%
%:%2840=1338%:%
%:%2841=1339%:%
%:%2842=1339%:%
%:%2843=1340%:%
%:%2844=1340%:%
%:%2845=1340%:%
%:%2846=1341%:%
%:%2847=1341%:%
%:%2848=1342%:%
%:%2849=1342%:%
%:%2850=1342%:%
%:%2851=1343%:%
%:%2852=1343%:%
%:%2853=1344%:%
%:%2854=1344%:%
%:%2855=1344%:%
%:%2856=1345%:%
%:%2857=1345%:%
%:%2858=1346%:%
%:%2859=1346%:%
%:%2860=1346%:%
%:%2861=1347%:%
%:%2862=1347%:%
%:%2863=1348%:%
%:%2864=1348%:%
%:%2865=1349%:%
%:%2866=1349%:%
%:%2867=1350%:%
%:%2868=1350%:%
%:%2869=1350%:%
%:%2870=1351%:%
%:%2871=1351%:%
%:%2872=1352%:%
%:%2873=1352%:%
%:%2874=1352%:%
%:%2875=1353%:%
%:%2876=1353%:%
%:%2877=1354%:%
%:%2878=1354%:%
%:%2879=1354%:%
%:%2880=1355%:%
%:%2881=1355%:%
%:%2882=1356%:%
%:%2883=1356%:%
%:%2884=1356%:%
%:%2885=1357%:%
%:%2886=1357%:%
%:%2887=1358%:%
%:%2888=1358%:%
%:%2889=1359%:%
%:%2890=1359%:%
%:%2891=1360%:%
%:%2897=1360%:%
%:%2900=1361%:%
%:%2901=1362%:%
%:%2902=1362%:%
%:%2903=1363%:%
%:%2904=1364%:%
%:%2905=1365%:%
%:%2906=1366%:%
%:%2907=1367%:%
%:%2908=1368%:%
%:%2915=1369%:%
%:%2916=1369%:%
%:%2917=1370%:%
%:%2918=1370%:%
%:%2919=1371%:%
%:%2920=1371%:%
%:%2921=1372%:%
%:%2922=1372%:%
%:%2923=1373%:%
%:%2924=1373%:%
%:%2925=1373%:%
%:%2926=1374%:%
%:%2927=1374%:%
%:%2928=1375%:%
%:%2929=1375%:%
%:%2930=1375%:%
%:%2931=1376%:%
%:%2932=1376%:%
%:%2933=1377%:%
%:%2934=1377%:%
%:%2935=1378%:%
%:%2936=1378%:%
%:%2937=1378%:%
%:%2938=1379%:%
%:%2939=1379%:%
%:%2940=1380%:%
%:%2941=1380%:%
%:%2942=1380%:%
%:%2943=1381%:%
%:%2944=1381%:%
%:%2945=1382%:%
%:%2946=1382%:%
%:%2947=1383%:%
%:%2948=1383%:%
%:%2949=1384%:%
%:%2950=1384%:%
%:%2951=1384%:%
%:%2952=1385%:%
%:%2953=1385%:%
%:%2954=1386%:%
%:%2955=1386%:%
%:%2956=1386%:%
%:%2957=1387%:%
%:%2958=1387%:%
%:%2959=1388%:%
%:%2960=1388%:%
%:%2961=1389%:%
%:%2962=1389%:%
%:%2963=1389%:%
%:%2964=1390%:%
%:%2965=1390%:%
%:%2966=1391%:%
%:%2967=1391%:%
%:%2968=1391%:%
%:%2969=1392%:%
%:%2970=1392%:%
%:%2971=1393%:%
%:%2972=1393%:%
%:%2973=1393%:%
%:%2974=1394%:%
%:%2975=1394%:%
%:%2976=1395%:%
%:%2977=1395%:%
%:%2978=1395%:%
%:%2979=1396%:%
%:%2980=1396%:%
%:%2981=1397%:%
%:%2982=1397%:%
%:%2983=1398%:%
%:%2984=1398%:%
%:%2985=1399%:%
%:%2986=1399%:%
%:%2987=1399%:%
%:%2988=1400%:%
%:%2989=1400%:%
%:%2990=1401%:%
%:%2991=1401%:%
%:%2992=1401%:%
%:%2993=1402%:%
%:%2994=1402%:%
%:%2995=1403%:%
%:%2996=1403%:%
%:%2997=1403%:%
%:%2998=1404%:%
%:%2999=1404%:%
%:%3000=1405%:%
%:%3001=1405%:%
%:%3002=1405%:%
%:%3003=1406%:%
%:%3004=1406%:%
%:%3005=1407%:%
%:%3006=1407%:%
%:%3007=1408%:%
%:%3008=1408%:%
%:%3009=1409%:%
%:%3015=1409%:%
%:%3018=1410%:%
%:%3019=1411%:%
%:%3020=1411%:%
%:%3021=1412%:%
%:%3028=1413%:%
%:%3029=1413%:%
%:%3030=1414%:%
%:%3031=1414%:%
%:%3032=1415%:%
%:%3033=1415%:%
%:%3034=1415%:%
%:%3035=1415%:%
%:%3036=1416%:%
%:%3037=1416%:%
%:%3038=1417%:%
%:%3039=1417%:%
%:%3040=1418%:%
%:%3041=1418%:%
%:%3042=1418%:%
%:%3043=1418%:%
%:%3044=1419%:%
%:%3045=1419%:%
%:%3046=1420%:%
%:%3047=1420%:%
%:%3048=1421%:%
%:%3049=1421%:%
%:%3050=1421%:%
%:%3051=1421%:%
%:%3052=1422%:%
%:%3053=1422%:%
%:%3054=1423%:%
%:%3055=1423%:%
%:%3056=1424%:%
%:%3057=1424%:%
%:%3058=1424%:%
%:%3059=1424%:%
%:%3060=1425%:%
%:%3061=1425%:%
%:%3062=1426%:%
%:%3063=1426%:%
%:%3064=1427%:%
%:%3065=1427%:%
%:%3066=1427%:%
%:%3067=1427%:%
%:%3068=1428%:%
%:%3069=1428%:%
%:%3070=1429%:%
%:%3071=1429%:%
%:%3072=1430%:%
%:%3073=1430%:%
%:%3074=1430%:%
%:%3075=1430%:%
%:%3076=1431%:%
%:%3086=1433%:%
%:%3088=1435%:%
%:%3089=1435%:%
%:%3090=1436%:%
%:%3093=1437%:%
%:%3097=1437%:%
%:%3098=1437%:%
%:%3107=1439%:%
%:%3108=1440%:%
%:%3109=1441%:%
%:%3110=1442%:%
%:%3111=1443%:%
%:%3112=1444%:%
%:%3113=1445%:%
%:%3114=1446%:%
%:%3115=1447%:%
%:%3116=1448%:%
%:%3117=1449%:%
%:%3118=1450%:%
%:%3119=1451%:%
%:%3120=1452%:%
%:%3121=1453%:%
%:%3122=1454%:%
%:%3123=1455%:%
%:%3124=1456%:%
%:%3125=1457%:%
%:%3126=1458%:%
%:%3127=1459%:%
%:%3128=1460%:%
%:%3129=1461%:%
%:%3131=1463%:%
%:%3132=1463%:%
%:%3133=1464%:%
%:%3134=1465%:%
%:%3135=1466%:%
%:%3142=1467%:%
%:%3143=1467%:%
%:%3144=1468%:%
%:%3145=1468%:%
%:%3146=1468%:%
%:%3147=1469%:%
%:%3148=1469%:%
%:%3149=1470%:%
%:%3150=1470%:%
%:%3151=1470%:%
%:%3152=1471%:%
%:%3153=1471%:%
%:%3154=1472%:%
%:%3155=1472%:%
%:%3156=1472%:%
%:%3157=1472%:%
%:%3158=1473%:%
%:%3159=1473%:%
%:%3160=1474%:%
%:%3161=1474%:%
%:%3162=1475%:%
%:%3163=1475%:%
%:%3164=1476%:%
%:%3165=1476%:%
%:%3166=1476%:%
%:%3167=1477%:%
%:%3168=1477%:%
%:%3169=1478%:%
%:%3170=1478%:%
%:%3171=1479%:%
%:%3177=1479%:%
%:%3180=1480%:%
%:%3181=1481%:%
%:%3182=1481%:%
%:%3183=1482%:%
%:%3185=1484%:%
%:%3188=1485%:%
%:%3192=1485%:%
%:%3193=1485%:%
%:%3194=1485%:%
%:%3203=1487%:%
%:%3207=1489%:%
%:%3208=1490%:%
%:%3210=1492%:%
%:%3211=1492%:%
%:%3212=1493%:%
%:%3213=1494%:%
%:%3214=1495%:%
%:%3215=1496%:%
%:%3216=1497%:%
%:%3217=1498%:%
%:%3218=1499%:%
%:%3221=1500%:%
%:%3225=1500%:%
%:%3235=1502%:%
%:%3236=1503%:%
%:%3237=1504%:%
%:%3238=1505%:%
%:%3239=1506%:%
%:%3240=1507%:%
%:%3242=1509%:%
%:%3243=1509%:%
%:%3244=1510%:%
%:%3245=1511%:%
%:%3252=1512%:%
%:%3253=1512%:%
%:%3254=1513%:%
%:%3255=1513%:%
%:%3256=1514%:%
%:%3257=1514%:%
%:%3258=1514%:%
%:%3259=1514%:%
%:%3260=1515%:%
%:%3261=1515%:%
%:%3262=1515%:%
%:%3263=1516%:%
%:%3264=1516%:%
%:%3265=1516%:%
%:%3266=1516%:%
%:%3267=1517%:%
%:%3268=1517%:%
%:%3269=1517%:%
%:%3270=1518%:%
%:%3271=1518%:%
%:%3272=1519%:%
%:%3278=1519%:%
%:%3281=1520%:%
%:%3282=1521%:%
%:%3283=1521%:%
%:%3284=1522%:%
%:%3285=1523%:%
%:%3292=1524%:%
%:%3293=1524%:%
%:%3294=1525%:%
%:%3295=1525%:%
%:%3296=1526%:%
%:%3297=1527%:%
%:%3298=1527%:%
%:%3299=1528%:%
%:%3300=1528%:%
%:%3301=1528%:%
%:%3302=1529%:%
%:%3303=1529%:%
%:%3304=1529%:%
%:%3305=1529%:%
%:%3306=1530%:%
%:%3307=1530%:%
%:%3308=1531%:%
%:%3309=1531%:%
%:%3310=1532%:%
%:%3311=1532%:%
%:%3312=1532%:%
%:%3313=1532%:%
%:%3314=1533%:%
%:%3315=1533%:%
%:%3316=1533%:%
%:%3317=1533%:%
%:%3318=1534%:%
%:%3319=1534%:%
%:%3320=1534%:%
%:%3321=1534%:%
%:%3322=1535%:%
%:%3332=1537%:%
%:%3334=1539%:%
%:%3335=1539%:%
%:%3336=1540%:%
%:%3337=1541%:%
%:%3338=1542%:%
%:%3339=1543%:%
%:%3340=1544%:%
%:%3341=1545%:%
%:%3342=1546%:%
%:%3345=1547%:%
%:%3349=1547%:%
%:%3350=1547%:%
%:%3351=1547%:%
%:%3352=1548%:%
%:%3362=1550%:%
%:%3366=1552%:%
%:%3370=1554%:%
%:%3371=1555%:%
%:%3372=1556%:%
%:%3374=1558%:%
%:%3375=1558%:%
%:%3378=1559%:%
%:%3382=1559%:%
%:%3392=1561%:%
%:%3393=1562%:%
%:%3394=1563%:%
%:%3395=1564%:%
%:%3396=1565%:%
%:%3398=1567%:%
%:%3399=1567%:%
%:%3400=1568%:%
%:%3401=1569%:%
%:%3402=1570%:%
%:%3403=1570%:%
%:%3404=1571%:%
%:%3405=1572%:%
%:%3406=1573%:%
%:%3407=1573%:%
%:%3408=1574%:%
%:%3409=1575%:%
%:%3412=1575%:%
%:%3416=1575%:%
%:%3424=1575%:%
%:%3425=1576%:%
%:%3426=1577%:%
%:%3429=1579%:%
%:%3430=1580%:%
%:%3431=1581%:%
%:%3432=1582%:%
%:%3433=1583%:%
%:%3434=1584%:%
%:%3435=1585%:%
%:%3436=1586%:%
%:%3437=1587%:%
%:%3438=1588%:%
%:%3439=1589%:%
%:%3440=1590%:%
%:%3441=1591%:%
%:%3443=1593%:%
%:%3444=1593%:%
%:%3445=1594%:%
%:%3446=1594%:%
%:%3448=1597%:%
%:%3449=1598%:%
%:%3450=1599%:%
%:%3451=1600%:%
%:%3453=1602%:%
%:%3454=1602%:%
%:%3455=1603%:%
%:%3457=1605%:%
%:%3458=1606%:%
%:%3459=1607%:%
%:%3460=1608%:%
%:%3461=1609%:%
%:%3462=1610%:%
%:%3463=1611%:%
%:%3464=1612%:%
%:%3466=1614%:%
%:%3467=1614%:%
%:%3470=1615%:%
%:%3474=1615%:%
%:%3475=1615%:%
%:%3484=1617%:%
%:%3485=1618%:%
%:%3494=1620%:%
%:%3506=1622%:%
%:%3507=1623%:%
%:%3508=1624%:%
%:%3509=1625%:%
%:%3510=1626%:%
%:%3511=1627%:%
%:%3512=1628%:%
%:%3513=1629%:%
%:%3514=1630%:%
%:%3516=1632%:%
%:%3517=1632%:%
%:%3518=1633%:%
%:%3520=1635%:%
%:%3521=1636%:%
%:%3522=1637%:%
%:%3523=1638%:%
%:%3524=1639%:%
%:%3525=1640%:%
%:%3526=1641%:%
%:%3528=1643%:%
%:%3529=1643%:%
%:%3532=1644%:%
%:%3536=1644%:%
%:%3537=1644%:%
%:%3546=1646%:%
%:%3550=1648%:%
%:%3554=1650%:%
%:%3555=1651%:%
%:%3556=1652%:%
%:%3557=1653%:%
%:%3558=1654%:%
%:%3559=1655%:%
%:%3560=1656%:%
%:%3561=1657%:%
%:%3562=1658%:%
%:%3563=1659%:%
%:%3564=1660%:%
%:%3565=1661%:%
%:%3566=1662%:%
%:%3567=1663%:%
%:%3568=1664%:%
%:%3569=1665%:%
%:%3570=1666%:%
%:%3571=1667%:%
%:%3572=1668%:%
%:%3573=1669%:%
%:%3574=1670%:%
%:%3575=1671%:%
%:%3576=1672%:%
%:%3577=1673%:%
%:%3578=1674%:%
%:%3579=1675%:%
%:%3580=1676%:%
%:%3581=1677%:%
%:%3582=1678%:%
%:%3583=1679%:%
%:%3584=1680%:%
%:%3585=1681%:%
%:%3586=1682%:%
%:%3587=1683%:%
%:%3588=1684%:%
%:%3589=1685%:%
%:%3590=1686%:%
%:%3591=1687%:%
%:%3592=1688%:%
%:%3593=1689%:%
%:%3594=1690%:%
%:%3595=1691%:%
%:%3596=1692%:%
%:%3597=1693%:%
%:%3598=1694%:%
%:%3599=1695%:%
%:%3603=1705%:%
%:%3604=1706%:%
%:%3605=1707%:%
%:%3606=1708%:%
%:%3607=1709%:%
%:%3608=1710%:%
%:%3609=1711%:%
%:%3610=1712%:%
%:%3611=1713%:%
%:%3612=1714%:%
%:%3613=1715%:%
%:%3614=1716%:%
%:%3615=1717%:%
%:%3616=1718%:%
%:%3617=1719%:%
%:%3618=1720%:%
%:%3619=1721%:%
%:%3620=1722%:%
%:%3621=1723%:%
%:%3622=1724%:%
%:%3623=1725%:%
%:%3624=1726%:%
%:%3625=1727%:%
%:%3626=1728%:%
%:%3627=1729%:%
%:%3628=1730%:%
%:%3629=1731%:%


\appendix
\chapter{Lemas de HOL usados}
%
\begin{isabellebody}%
\setisabellecontext{Glosario}%
%
\isadelimtheory
%
\endisadelimtheory
%
\isatagtheory
%
\endisatagtheory
{\isafoldtheory}%
%
\isadelimtheory
%
\endisadelimtheory
%
\isadelimdocument
%
\endisadelimdocument
%
\isatagdocument
%
\isamarkupsection{Glosario de reglas%
}
\isamarkuptrue%
%
\isamarkupsubsection{Teoría de conjuntos finitos%
}
\isamarkuptrue%
%
\endisatagdocument
{\isafolddocument}%
%
\isadelimdocument
%
\endisadelimdocument
%
\begin{isamarkuptext}%
A continuación se muestran resultamos relativos a la teoría 
  \href{https://n9.cl/x86r}{FiniteSet.thy}. Dicha teoría se basa en la definición recursiva de
  \isa{finite}, que aparece retratada en la sección de \isa{Sintaxis}. Además, hemos empleado los
  siguientes resultados. 

  \begin{itemize}
    \item[] \isa{\mbox{}\inferrule{\mbox{finite\ F\ {\isasymand}\ finite\ G}}{\mbox{finite\ {\isacharparenleft}F\ {\isasymunion}\ G{\isacharparenright}}}} 
      \hfill (\isa{finite{\isacharunderscore}UnI})
  \end{itemize}%
\end{isamarkuptext}\isamarkuptrue%
%
\isadelimdocument
%
\endisadelimdocument
%
\isatagdocument
%
\isamarkupsubsection{Teoría de listas%
}
\isamarkuptrue%
%
\endisatagdocument
{\isafolddocument}%
%
\isadelimdocument
%
\endisadelimdocument
%
\begin{isamarkuptext}%
La teoría de listas en Isabelle corresponde a \href{http://bit.ly/2se9Oy0}{List.thy}. 
  Esta se fundamenta en la definición recursiva de \isa{list}.\\

\isa{datatype\ {\isacharparenleft}set{\isacharprime}{\isacharcolon}\ {\isacharprime}a{\isacharparenright}\ list{\isacharprime}\ {\isacharequal}{\isacharbackslash}{\isacharbackslash}\ Nil{\isacharprime}\ \ {\isacharparenleft}{\isachardoublequote}{\isacharbrackleft}{\isacharbrackright}{\isachardoublequote}{\isacharparenright}{\isacharbackslash}{\isacharbackslash}\ {\isacharbar}\ Cons{\isacharprime}\ {\isacharparenleft}hd{\isacharcolon}\ {\isacharprime}a{\isacharparenright}\ {\isacharparenleft}tl{\isacharcolon}\ {\isachardoublequote}{\isacharprime}a\ list{\isacharprime}{\isachardoublequote}{\isacharparenright}\ \ {\isacharparenleft}infixr\ {\isachardoublequote}{\isacharhash}{\isachardoublequote}\ {\isadigit{6}}{\isadigit{5}}{\isacharparenright}{\isacharbackslash}{\isacharbackslash}\ for{\isacharbackslash}{\isacharbackslash}\ map{\isacharcolon}\ map{\isacharbackslash}{\isacharbackslash}\ rel{\isacharcolon}\ list{\isacharunderscore}all{\isadigit{2}}{\isacharbackslash}{\isacharbackslash}\ pred{\isacharcolon}\ list{\isacharunderscore}all{\isacharbackslash}{\isacharbackslash}\ where{\isacharbackslash}{\isacharbackslash}\ {\isachardoublequote}tl\ {\isacharbrackleft}{\isacharbrackright}\ {\isacharequal}\ {\isacharbrackleft}{\isacharbrackright}{\isachardoublequote}{\isacharbackslash}{\isacharbackslash}}

COMENTARIO: NO ME PERMITE PONERLO FUERA DEL ENTORNO DE TEXTO, NI CAMBIANDO EL NOMBRE \\

Como es habitual, hemos cambiado la notación de la definición a \isa{list{\isacharprime}} para no 
  definir dos veces de manera idéntica la misma noción. Simultáneamente se define la función
  de conjuntos \isa{set} (idéntica a \isa{set{\isacharprime}}), una función \isa{map}, una relación
  \isa{rel} y un predicado \isa{pred}. Para dicha definción hemos empleado los operadores
  sobre listas \isa{hd} y \isa{tl}.
  De este modo, \isa{hd} aplicado a una lista de elementos de un tipo cualquiera \isa{{\isacharprime}a} nos 
  devuelve el primer elemento de la misma, y \isa{tl}  nos 
  devuelve la lista quitando el primer elmento.
 
  Además, hemos utilizado las siguientes propiedades sobre listas.

  \begin{itemize}
    \item[] \isa{{\isacharbraceleft}a{\isacharbraceright}\ {\isasymunion}\ B\ {\isasymunion}\ C\ {\isacharequal}\ {\isacharbraceleft}a{\isacharbraceright}\ {\isasymunion}\ {\isacharparenleft}B\ {\isasymunion}\ C{\isacharparenright}} 
    \hfill (\isa{Un{\isacharunderscore}insert{\isacharunderscore}left})
  \end{itemize}%
\end{isamarkuptext}\isamarkuptrue%
%
\isadelimdocument
%
\endisadelimdocument
%
\isatagdocument
%
\isamarkupsubsection{Teoría de conjuntos%
}
\isamarkuptrue%
%
\endisatagdocument
{\isafolddocument}%
%
\isadelimdocument
%
\endisadelimdocument
%
\begin{isamarkuptext}%
Los siguientes resultados empleados en el análisis hecho sobre la lógica proposicional 
  corresponden a la teoría de conjuntos de Isabelle: \href{https://n9.cl/qatp}{Set.thy}.

  \begin{itemize}
    \item[] \isa{xs\ \isacharat\ ys\ {\isacharequal}\ xs\ {\isasymunion}\ ys} 
      \hfill (\isa{set{\isacharunderscore}append})
    \item[] \isa{a\ {\isasymin}\ {\isacharbraceleft}a{\isacharbraceright}} 
      \hfill (\isa{singletonI})
    \item[] \isa{a\ {\isasymin}\ {\isacharbraceleft}a{\isacharbraceright}\ {\isasymunion}\ B} 
      \hfill (\isa{insertI{\isadigit{1}}})
    \item[] \isa{A\ {\isasymunion}\ {\isasymemptyset}\ {\isacharequal}\ A} 
      \hfill (\isa{Un{\isacharunderscore}empty{\isacharunderscore}right})
    \item[] \isa{\mbox{}\inferrule{\mbox{A\ {\isasymsubseteq}\ B\ {\isasymand}\ B\ {\isasymsubseteq}\ C}}{\mbox{A\ {\isasymsubseteq}\ C}}} 
      \hfill (\isa{subset{\isacharunderscore}trans})
    \item[] \isa{\mbox{}\inferrule{\mbox{c\ {\isasymin}\ A\ {\isasymand}\ A\ {\isasymsubseteq}\ B}}{\mbox{c\ {\isasymin}\ B}}} 
      \hfill (\isa{rev{\isacharunderscore}subsetD})
    \item[] \isa{\mbox{}\inferrule{\mbox{A\ {\isasymsubseteq}\ C\ {\isasymand}\ B\ {\isasymsubseteq}\ D}}{\mbox{A\ {\isasymunion}\ B\ {\isasymsubseteq}\ C\ {\isasymunion}\ D}}} 
      \hfill (\isa{Un{\isacharunderscore}mono})
    \item[] \isa{A\ {\isasymsubseteq}\ A\ {\isasymunion}\ B} 
      \hfill (\isa{Un{\isacharunderscore}upper{\isadigit{1}}})
    \item[] \isa{B\ {\isasymsubseteq}\ A\ {\isasymunion}\ B} 
      \hfill (\isa{Un{\isacharunderscore}upper{\isadigit{2}}})
    \item[] \isa{A\ {\isasymsubseteq}\ A} 
      \hfill (\isa{subset{\isacharunderscore}refl})
    \item[] \isa{{\isasymemptyset}\ {\isasymsubseteq}\ A} 
      \hfill (\isa{empty{\isacharunderscore}subsetI})
    \item[] \isa{\mbox{}\inferrule{\mbox{b\ {\isasymin}\ {\isacharbraceleft}a{\isacharbraceright}}}{\mbox{b\ {\isacharequal}\ a}}} 
      \hfill (\isa{singletonD})
    \item[] \isa{{\isacharparenleft}c\ {\isasymin}\ A\ {\isasymunion}\ B{\isacharparenright}\ {\isacharequal}\ {\isacharparenleft}c\ {\isasymin}\ A\ {\isasymor}\ c\ {\isasymin}\ B{\isacharparenright}} 
      \hfill (\isa{Un{\isacharunderscore}iff})
  \end{itemize}%
\end{isamarkuptext}\isamarkuptrue%
%
\isadelimdocument
%
\endisadelimdocument
%
\isatagdocument
%
\isamarkupsubsection{Lógica de primer orden%
}
\isamarkuptrue%
%
\endisatagdocument
{\isafolddocument}%
%
\isadelimdocument
%
\endisadelimdocument
%
\begin{isamarkuptext}%
En Isabelle corresponde a la teoría \href{http://bit.ly/38iFKlA}{HOL.thy}
  Los resultados empleados son los siguientes.

  \begin{itemize}
    \item[] \isa{\mbox{}\inferrule{\mbox{P\ {\isasymand}\ Q}}{\mbox{P}}} 
      \hfill (\isa{conjunct{\isadigit{1}}})
    \item[] \isa{\mbox{}\inferrule{\mbox{P\ {\isasymand}\ Q}}{\mbox{Q}}} 
      \hfill (\isa{conjunct{\isadigit{2}}})
  \end{itemize}%
\end{isamarkuptext}\isamarkuptrue%
%
\isadelimtheory
%
\endisadelimtheory
%
\isatagtheory
%
\endisatagtheory
{\isafoldtheory}%
%
\isadelimtheory
%
\endisadelimtheory
%
\end{isabellebody}%
\endinput
%:%file=~/Logica_Proposicional/Glosario.thy%:%
%:%24=11%:%
%:%28=13%:%
%:%40=15%:%
%:%41=16%:%
%:%42=17%:%
%:%43=18%:%
%:%44=19%:%
%:%45=20%:%
%:%46=21%:%
%:%47=22%:%
%:%48=23%:%
%:%57=25%:%
%:%69=27%:%
%:%70=28%:%
%:%71=29%:%
%:%72=38%:%
%:%73=39%:%
%:%74=40%:%
%:%75=41%:%
%:%76=42%:%
%:%77=43%:%
%:%78=44%:%
%:%79=45%:%
%:%80=46%:%
%:%81=47%:%
%:%82=48%:%
%:%83=49%:%
%:%84=50%:%
%:%85=51%:%
%:%86=52%:%
%:%87=53%:%
%:%88=54%:%
%:%89=55%:%
%:%90=56%:%
%:%99=58%:%
%:%111=60%:%
%:%112=61%:%
%:%113=62%:%
%:%114=63%:%
%:%115=64%:%
%:%116=65%:%
%:%117=66%:%
%:%118=67%:%
%:%119=68%:%
%:%120=69%:%
%:%121=70%:%
%:%122=71%:%
%:%123=72%:%
%:%124=73%:%
%:%125=74%:%
%:%126=75%:%
%:%127=76%:%
%:%128=77%:%
%:%129=78%:%
%:%130=79%:%
%:%131=80%:%
%:%132=81%:%
%:%133=82%:%
%:%134=83%:%
%:%135=84%:%
%:%136=85%:%
%:%137=86%:%
%:%138=87%:%
%:%139=88%:%
%:%140=89%:%
%:%141=90%:%
%:%150=93%:%
%:%162=95%:%
%:%163=96%:%
%:%164=97%:%
%:%165=98%:%
%:%166=99%:%
%:%167=100%:%
%:%168=101%:%
%:%169=102%:%
%:%170=103%:%

% optional bibliography
\nocite{LMF, tutorial,fitting1996first}
\bibliographystyle{plain}
\bibliography{root}

\comentario{Añadir el artículo que se usa como base.}

% Pendientes
\todototoc
\listoftodos

\end{document}

%%% Local Variables:
%%% mode: latex
%%% TeX-master: t
%%% End:
