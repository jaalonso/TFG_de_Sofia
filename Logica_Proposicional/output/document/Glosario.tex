%
\begin{isabellebody}%
\setisabellecontext{Glosario}%
%
\isadelimtheory
%
\endisadelimtheory
%
\isatagtheory
%
\endisatagtheory
{\isafoldtheory}%
%
\isadelimtheory
%
\endisadelimtheory
%
\isadelimdocument
%
\endisadelimdocument
%
\isatagdocument
%
\isamarkupsection{Glosario de reglas%
}
\isamarkuptrue%
%
\isamarkupsubsection{Teoría de conjuntos finitos%
}
\isamarkuptrue%
%
\endisatagdocument
{\isafolddocument}%
%
\isadelimdocument
%
\endisadelimdocument
%
\begin{isamarkuptext}%
A continuación se muestran resultamos relativos a la teoría 
  \href{https://n9.cl/x86r}{FiniteSet.thy}. Dicha teoría se basa en la definición recursiva de
  \isa{finite}, que aparece retratada en la sección de \isa{Sintaxis}. Además, hemos empleado los
  siguientes resultados. 

  \begin{itemize}
    \item[] \isa{\mbox{}\inferrule{\mbox{finite\ F\ {\isasymand}\ finite\ G}}{\mbox{finite\ {\isacharparenleft}F\ {\isasymunion}\ G{\isacharparenright}}}} 
      \hfill (\isa{finite{\isacharunderscore}UnI})
  \end{itemize}%
\end{isamarkuptext}\isamarkuptrue%
%
\isadelimdocument
%
\endisadelimdocument
%
\isatagdocument
%
\isamarkupsubsection{Teoría de listas%
}
\isamarkuptrue%
%
\endisatagdocument
{\isafolddocument}%
%
\isadelimdocument
%
\endisadelimdocument
%
\begin{isamarkuptext}%
La teoría de listas en Isabelle corresponde a \href{http://bit.ly/2se9Oy0}{List.thy}. 
  Esta se fundamenta en la definición recursiva de \isa{list}.\\

\isa{datatype\ {\isacharparenleft}set{\isacharprime}{\isacharcolon}\ {\isacharprime}a{\isacharparenright}\ list{\isacharprime}\ {\isacharequal}{\isacharbackslash}{\isacharbackslash}\ Nil{\isacharprime}\ \ {\isacharparenleft}{\isachardoublequote}{\isacharbrackleft}{\isacharbrackright}{\isachardoublequote}{\isacharparenright}{\isacharbackslash}{\isacharbackslash}\ {\isacharbar}\ Cons{\isacharprime}\ {\isacharparenleft}hd{\isacharcolon}\ {\isacharprime}a{\isacharparenright}\ {\isacharparenleft}tl{\isacharcolon}\ {\isachardoublequote}{\isacharprime}a\ list{\isacharprime}{\isachardoublequote}{\isacharparenright}\ \ {\isacharparenleft}infixr\ {\isachardoublequote}{\isacharhash}{\isachardoublequote}\ {\isadigit{6}}{\isadigit{5}}{\isacharparenright}{\isacharbackslash}{\isacharbackslash}\ for{\isacharbackslash}{\isacharbackslash}\ map{\isacharcolon}\ map{\isacharbackslash}{\isacharbackslash}\ rel{\isacharcolon}\ list{\isacharunderscore}all{\isadigit{2}}{\isacharbackslash}{\isacharbackslash}\ pred{\isacharcolon}\ list{\isacharunderscore}all{\isacharbackslash}{\isacharbackslash}\ where{\isacharbackslash}{\isacharbackslash}\ {\isachardoublequote}tl\ {\isacharbrackleft}{\isacharbrackright}\ {\isacharequal}\ {\isacharbrackleft}{\isacharbrackright}{\isachardoublequote}{\isacharbackslash}{\isacharbackslash}}

COMENTARIO: NO ME PERMITE PONERLO FUERA DEL ENTORNO DE TEXTO, NI CAMBIANDO EL NOMBRE \\

Como es habitual, hemos cambiado la notación de la definición a \isa{list{\isacharprime}} para no 
  definir dos veces de manera idéntica la misma noción. Simultáneamente se define la función
  de conjuntos \isa{set} (idéntica a \isa{set{\isacharprime}}), una función \isa{map}, una relación
  \isa{rel} y un predicado \isa{pred}. Para dicha definción hemos empleado los operadores
  sobre listas \isa{hd} y \isa{tl}.
  De este modo, \isa{hd} aplicado a una lista de elementos de un tipo cualquiera \isa{{\isacharprime}a} nos 
  devuelve el primer elemento de la misma, y \isa{tl}  nos 
  devuelve la lista quitando el primer elmento.
 
  Además, hemos utilizado las siguientes propiedades sobre listas.

  \begin{itemize}
    \item[] \isa{{\isacharbraceleft}a{\isacharbraceright}\ {\isasymunion}\ B\ {\isasymunion}\ C\ {\isacharequal}\ {\isacharbraceleft}a{\isacharbraceright}\ {\isasymunion}\ {\isacharparenleft}B\ {\isasymunion}\ C{\isacharparenright}} 
    \hfill (\isa{Un{\isacharunderscore}insert{\isacharunderscore}left})
  \end{itemize}%
\end{isamarkuptext}\isamarkuptrue%
%
\isadelimdocument
%
\endisadelimdocument
%
\isatagdocument
%
\isamarkupsubsection{Teoría de conjuntos%
}
\isamarkuptrue%
%
\endisatagdocument
{\isafolddocument}%
%
\isadelimdocument
%
\endisadelimdocument
%
\begin{isamarkuptext}%
Los siguientes resultados empleados en el análisis hecho sobre la lógica proposicional 
  corresponden a la teoría de conjuntos de Isabelle: \href{https://n9.cl/qatp}{Set.thy}.

  \begin{itemize}
    \item[] \isa{xs\ \isacharat\ ys\ {\isacharequal}\ xs\ {\isasymunion}\ ys} 
      \hfill (\isa{set{\isacharunderscore}append})
    \item[] \isa{a\ {\isasymin}\ {\isacharbraceleft}a{\isacharbraceright}} 
      \hfill (\isa{singletonI})
    \item[] \isa{a\ {\isasymin}\ {\isacharbraceleft}a{\isacharbraceright}\ {\isasymunion}\ B} 
      \hfill (\isa{insertI{\isadigit{1}}})
    \item[] \isa{A\ {\isasymunion}\ {\isasymemptyset}\ {\isacharequal}\ A} 
      \hfill (\isa{Un{\isacharunderscore}empty{\isacharunderscore}right})
    \item[] \isa{\mbox{}\inferrule{\mbox{A\ {\isasymsubseteq}\ B\ {\isasymand}\ B\ {\isasymsubseteq}\ C}}{\mbox{A\ {\isasymsubseteq}\ C}}} 
      \hfill (\isa{subset{\isacharunderscore}trans})
    \item[] \isa{\mbox{}\inferrule{\mbox{c\ {\isasymin}\ A\ {\isasymand}\ A\ {\isasymsubseteq}\ B}}{\mbox{c\ {\isasymin}\ B}}} 
      \hfill (\isa{rev{\isacharunderscore}subsetD})
    \item[] \isa{\mbox{}\inferrule{\mbox{A\ {\isasymsubseteq}\ C\ {\isasymand}\ B\ {\isasymsubseteq}\ D}}{\mbox{A\ {\isasymunion}\ B\ {\isasymsubseteq}\ C\ {\isasymunion}\ D}}} 
      \hfill (\isa{Un{\isacharunderscore}mono})
    \item[] \isa{A\ {\isasymsubseteq}\ A\ {\isasymunion}\ B} 
      \hfill (\isa{Un{\isacharunderscore}upper{\isadigit{1}}})
    \item[] \isa{B\ {\isasymsubseteq}\ A\ {\isasymunion}\ B} 
      \hfill (\isa{Un{\isacharunderscore}upper{\isadigit{2}}})
    \item[] \isa{A\ {\isasymsubseteq}\ A} 
      \hfill (\isa{subset{\isacharunderscore}refl})
    \item[] \isa{{\isasymemptyset}\ {\isasymsubseteq}\ A} 
      \hfill (\isa{empty{\isacharunderscore}subsetI})
    \item[] \isa{\mbox{}\inferrule{\mbox{b\ {\isasymin}\ {\isacharbraceleft}a{\isacharbraceright}}}{\mbox{b\ {\isacharequal}\ a}}} 
      \hfill (\isa{singletonD})
    \item[] \isa{{\isacharparenleft}c\ {\isasymin}\ A\ {\isasymunion}\ B{\isacharparenright}\ {\isacharequal}\ {\isacharparenleft}c\ {\isasymin}\ A\ {\isasymor}\ c\ {\isasymin}\ B{\isacharparenright}} 
      \hfill (\isa{Un{\isacharunderscore}iff})
  \end{itemize}%
\end{isamarkuptext}\isamarkuptrue%
%
\isadelimdocument
%
\endisadelimdocument
%
\isatagdocument
%
\isamarkupsubsection{Lógica de primer orden%
}
\isamarkuptrue%
%
\endisatagdocument
{\isafolddocument}%
%
\isadelimdocument
%
\endisadelimdocument
%
\begin{isamarkuptext}%
En Isabelle corresponde a la teoría \href{http://bit.ly/38iFKlA}{HOL.thy}
  Los resultados empleados son los siguientes.

  \begin{itemize}
    \item[] \isa{\mbox{}\inferrule{\mbox{P\ {\isasymand}\ Q}}{\mbox{P}}} 
      \hfill (\isa{conjunct{\isadigit{1}}})
    \item[] \isa{\mbox{}\inferrule{\mbox{P\ {\isasymand}\ Q}}{\mbox{Q}}} 
      \hfill (\isa{conjunct{\isadigit{2}}})
  \end{itemize}%
\end{isamarkuptext}\isamarkuptrue%
%
\isadelimtheory
%
\endisadelimtheory
%
\isatagtheory
%
\endisatagtheory
{\isafoldtheory}%
%
\isadelimtheory
%
\endisadelimtheory
%
\end{isabellebody}%
\endinput
%:%file=~/Desktop/LogicaProposicional/Glosario.thy%:%
%:%24=11%:%
%:%28=13%:%
%:%40=15%:%
%:%41=16%:%
%:%42=17%:%
%:%43=18%:%
%:%44=19%:%
%:%45=20%:%
%:%46=21%:%
%:%47=22%:%
%:%48=23%:%
%:%57=25%:%
%:%69=27%:%
%:%70=28%:%
%:%71=29%:%
%:%72=38%:%
%:%73=39%:%
%:%74=40%:%
%:%75=41%:%
%:%76=42%:%
%:%77=43%:%
%:%78=44%:%
%:%79=45%:%
%:%80=46%:%
%:%81=47%:%
%:%82=48%:%
%:%83=49%:%
%:%84=50%:%
%:%85=51%:%
%:%86=52%:%
%:%87=53%:%
%:%88=54%:%
%:%89=55%:%
%:%90=56%:%
%:%99=58%:%
%:%111=60%:%
%:%112=61%:%
%:%113=62%:%
%:%114=63%:%
%:%115=64%:%
%:%116=65%:%
%:%117=66%:%
%:%118=67%:%
%:%119=68%:%
%:%120=69%:%
%:%121=70%:%
%:%122=71%:%
%:%123=72%:%
%:%124=73%:%
%:%125=74%:%
%:%126=75%:%
%:%127=76%:%
%:%128=77%:%
%:%129=78%:%
%:%130=79%:%
%:%131=80%:%
%:%132=81%:%
%:%133=82%:%
%:%134=83%:%
%:%135=84%:%
%:%136=85%:%
%:%137=86%:%
%:%138=87%:%
%:%139=88%:%
%:%140=89%:%
%:%141=90%:%
%:%150=93%:%
%:%162=95%:%
%:%163=96%:%
%:%164=97%:%
%:%165=98%:%
%:%166=99%:%
%:%167=100%:%
%:%168=101%:%
%:%169=102%:%
%:%170=103%:%