%
\begin{isabellebody}%
\setisabellecontext{Sintaxis}%
%
\isadelimtheory
%
\endisadelimtheory
%
\isatagtheory
%
\endisatagtheory
{\isafoldtheory}%
%
\isadelimtheory
%
\endisadelimtheory
%
\isadelimdocument
%
\endisadelimdocument
%
\isatagdocument
%
\isamarkupsection{Sintaxis%
}
\isamarkuptrue%
%
\isamarkupsubsection{Fórmulas%
}
\isamarkuptrue%
%
\endisatagdocument
{\isafolddocument}%
%
\isadelimdocument
%
\endisadelimdocument
%
\begin{isamarkuptext}%
\comentario{Explicar la siguiente notación y recolocarla donde se
  use por primera vez.}%
\end{isamarkuptext}\isamarkuptrue%
\isacommand{notation}\isamarkupfalse%
\ insert\ {\isacharparenleft}{\isachardoublequoteopen}{\isacharunderscore}\ {\isasymtriangleright}\ {\isacharunderscore}{\isachardoublequoteclose}\ {\isacharbrackleft}{\isadigit{5}}{\isadigit{6}}{\isacharcomma}{\isadigit{5}}{\isadigit{5}}{\isacharbrackright}\ {\isadigit{5}}{\isadigit{5}}{\isacharparenright}%
\begin{isamarkuptext}%
En esta sección presentaremos una formalización en Isabelle de la 
  sintaxis de la lógica proposicional, junto con resultados y pruebas 
  sobre la misma. En líneas generales, primero daremos las nociones de 
  forma clásica y, a continuación, su correspondiente formalización.

  En primer lugar, supondremos que disponemos de los siguientes 
  elementos:
  \begin{description}
    \item[Alfabeto:] Es una lista infinita de variables proposicionales. 
      También pueden ser llamadas átomos o símbolos proposicionales.
    \item[Conectivas:] Conjunto finito cuyos elementos interactúan con 
      las variables. Pueden ser monarias que afectan a un único elemento 
      o binarias que afectan a dos. En el primer grupo se encuentra le 
      negación (\isa{{\isasymnot}}) y en el segundo la conjunción (\isa{{\isasymand}}), la disyunción 
      (\isa{{\isasymor}}) y la implicación (\isa{{\isasymlongrightarrow}}).
  \end{description}

  A continuación definiremos la estructura de fórmula sobre los 
  elementos anteriores. Para ello daremos una definición recursiva 
  basada en dos elementos: un conjunto de fórmulas básicas y una serie 
  de procedimientos de definición de fórmulas a partir de otras. El 
  conjunto de las fórmulas será el menor conjunto de estructuras 
  sinctáticas con dicho alfabeto y conectivas que contiene a las básicas 
  y es cerrado mediante los procedimientos de definición que mostraremos 
  a continuación.

  \begin{definicion}
    El conjunto de las fórmulas proposicionales está formado por las 
    siguientes:
    \begin{itemize}
      \item Las fórmulas atómicas, constituidas únicamente por una 
        variable del alfabeto. 
      \item La constante \isa{{\isasymbottom}}.
      \item Dada una fórmula \isa{F}, la negación \isa{{\isasymnot}\ F} es una fórmula.
      \item Dadas dos fórmulas \isa{F} y \isa{G}, la conjunción \isa{F\ {\isasymand}\ G} es una
        fórmula.
      \item Dadas dos fórmulas \isa{F} y \isa{G}, la disyunción \isa{F\ {\isasymor}\ G} es una
        fórmula.
      \item Dadas dos fórmulas \isa{F} y \isa{G}, la implicación \isa{F\ {\isasymlongrightarrow}\ G} es 
        una fórmula.
    \end{itemize}
  \end{definicion}

 Intuitivamente, las fórmulas proposicionales son entendidas como un 
 tipo de árbol sintáctico cuyos nodos son las conectivas y sus hojas las 
  fórmulas atómicas.

 \comentario{Incluir el árbol de formación.}

  A continuación, veamos su representación en Isabelle%
\end{isamarkuptext}\isamarkuptrue%
\isacommand{datatype}\isamarkupfalse%
\ {\isacharparenleft}atoms{\isacharcolon}\ {\isacharprime}a{\isacharparenright}\ formula\ {\isacharequal}\ \isanewline
\ \ Atom\ {\isacharprime}a\isanewline
{\isacharbar}\ Bot\ \ \ \ \ \ \ \ \ \ \ \ \ \ \ \ \ \ \ \ \ \ \ \ \ \ \ \ \ \ {\isacharparenleft}{\isachardoublequoteopen}{\isasymbottom}{\isachardoublequoteclose}{\isacharparenright}\ \ \isanewline
{\isacharbar}\ Not\ {\isachardoublequoteopen}{\isacharprime}a\ formula{\isachardoublequoteclose}\ \ \ \ \ \ \ \ \ \ \ \ \ \ \ \ \ {\isacharparenleft}{\isachardoublequoteopen}\isactrlbold {\isasymnot}{\isachardoublequoteclose}{\isacharparenright}\isanewline
{\isacharbar}\ And\ {\isachardoublequoteopen}{\isacharprime}a\ formula{\isachardoublequoteclose}\ {\isachardoublequoteopen}{\isacharprime}a\ formula{\isachardoublequoteclose}\ \ \ \ {\isacharparenleft}\isakeyword{infix}\ {\isachardoublequoteopen}\isactrlbold {\isasymand}{\isachardoublequoteclose}\ {\isadigit{6}}{\isadigit{8}}{\isacharparenright}\isanewline
{\isacharbar}\ Or\ {\isachardoublequoteopen}{\isacharprime}a\ formula{\isachardoublequoteclose}\ {\isachardoublequoteopen}{\isacharprime}a\ formula{\isachardoublequoteclose}\ \ \ \ \ {\isacharparenleft}\isakeyword{infix}\ {\isachardoublequoteopen}\isactrlbold {\isasymor}{\isachardoublequoteclose}\ {\isadigit{6}}{\isadigit{8}}{\isacharparenright}\isanewline
{\isacharbar}\ Imp\ {\isachardoublequoteopen}{\isacharprime}a\ formula{\isachardoublequoteclose}\ {\isachardoublequoteopen}{\isacharprime}a\ formula{\isachardoublequoteclose}\ \ \ \ {\isacharparenleft}\isakeyword{infixr}\ {\isachardoublequoteopen}\isactrlbold {\isasymrightarrow}{\isachardoublequoteclose}\ {\isadigit{6}}{\isadigit{8}}{\isacharparenright}%
\begin{isamarkuptext}%
Como podemos observar en la definición, \isa{formula} es un 
  tipo de datos recursivo que se entiende como un árbol que relaciona 
  elementos de un tipo \isa{{\isacharprime}a} cualquiera del alfabeto proposicional. En 
  ella, los constructores del tipo son los siguientes:

  \begin{description}
    \item[Fórmulas básicas]:  
      \begin{itemize}
        \item \isa{Atom\ {\isacharcolon}{\isacharcolon}\ {\isacharprime}a\ {\isasymRightarrow}\ {\isacharprime}a\ formula}
        \item \isa{{\isasymbottom}\ {\isacharcolon}{\isacharcolon}\ {\isacharprime}a\ formula}
      \end{itemize}
    \item [Fórmulas compuestas]:
      \begin{itemize}
        \item \isa{\isactrlbold {\isasymnot}\ {\isacharcolon}{\isacharcolon}\ {\isacharprime}a\ formula\ {\isasymRightarrow}\ {\isacharprime}a\ formula}
        \item \isa{{\isacharparenleft}\isactrlbold {\isasymand}{\isacharparenright}\ {\isacharcolon}{\isacharcolon}\ {\isacharprime}a\ formula\ {\isasymRightarrow}\ {\isacharprime}a\ formula\ {\isasymRightarrow}\ {\isacharprime}a\ formula}
        \item \isa{{\isacharparenleft}\isactrlbold {\isasymor}{\isacharparenright}\ {\isacharcolon}{\isacharcolon}\ {\isacharprime}a\ formula\ {\isasymRightarrow}\ {\isacharprime}a\ formula\ {\isasymRightarrow}\ {\isacharprime}a\ formula}
        \item \isa{{\isacharparenleft}\isactrlbold {\isasymrightarrow}{\isacharparenright}\ {\isacharcolon}{\isacharcolon}\ {\isacharprime}a\ formula\ {\isasymRightarrow}\ {\isacharprime}a\ formula\ {\isasymRightarrow}\ {\isacharprime}a\ formula}
      \end{itemize}
  \end{description}

  Cabe señalar que los términos \isa{infix} e \isa{infixr} que preceden a los
  símbolos de notación de los nodos nos señalan que son infijos, en 
  particular, \isa{infixr} se trata de un infijo asociado a la derecha.

  Además se define simultáneamente la función \isa{atoms\ {\isacharcolon}{\isacharcolon}\ {\isacharprime}a\ formula\ {\isasymRightarrow}\ {\isacharprime}a\ set}, que 
  obtiene el conjunto de variables proposicionales de una fórmula. De 
  manera equivalente, daremos la siguiente definición.

  \begin{definicion}
    Sea \isa{F} una fórmula proposicional. Entonces, se define 
    \isa{conjAtoms{\isacharparenleft}F{\isacharparenright}} como el conjunto de los átomos que aparecen en \isa{F}.
  \end{definicion}

  Por otro lado, la definición de \isa{formula} genera 
  automáticamente los siguientes lemas sobre la función de conjuntos 
  \isa{atoms} en Isabelle.
  
  \begin{itemize}
    \item[] \isa{atoms\ {\isacharparenleft}Atom\ x{\isadigit{1}}{\isacharparenright}\ {\isacharequal}\ {\isacharbraceleft}x{\isadigit{1}}{\isacharbraceright}\isasep\isanewline%
atoms\ {\isasymbottom}\ {\isacharequal}\ {\isasymemptyset}\isasep\isanewline%
atoms\ {\isacharparenleft}\isactrlbold {\isasymnot}\ x{\isadigit{3}}{\isacharparenright}\ {\isacharequal}\ atoms\ x{\isadigit{3}}\isasep\isanewline%
atoms\ {\isacharparenleft}x{\isadigit{4}}{\isadigit{1}}\ \isactrlbold {\isasymand}\ x{\isadigit{4}}{\isadigit{2}}{\isacharparenright}\ {\isacharequal}\ atoms\ x{\isadigit{4}}{\isadigit{1}}\ {\isasymunion}\ atoms\ x{\isadigit{4}}{\isadigit{2}}\isasep\isanewline%
atoms\ {\isacharparenleft}x{\isadigit{5}}{\isadigit{1}}\ \isactrlbold {\isasymor}\ x{\isadigit{5}}{\isadigit{2}}{\isacharparenright}\ {\isacharequal}\ atoms\ x{\isadigit{5}}{\isadigit{1}}\ {\isasymunion}\ atoms\ x{\isadigit{5}}{\isadigit{2}}\isasep\isanewline%
atoms\ {\isacharparenleft}x{\isadigit{6}}{\isadigit{1}}\ \isactrlbold {\isasymrightarrow}\ x{\isadigit{6}}{\isadigit{2}}{\isacharparenright}\ {\isacharequal}\ atoms\ x{\isadigit{6}}{\isadigit{1}}\ {\isasymunion}\ atoms\ x{\isadigit{6}}{\isadigit{2}}}
  \end{itemize} 

  A continuación veremos varios ejemplos de fórmulas y el conjunto de 
  sus variables proposicionales obtenido mediante \isa{atoms}. Se 
  observa que, por definición de conjunto, no contiene 
  elementos repetidos.%
\end{isamarkuptext}\isamarkuptrue%
\isacommand{notepad}\isamarkupfalse%
\ \isanewline
\isakeyword{begin}\isanewline
%
\isadelimproof
\ \ %
\endisadelimproof
%
\isatagproof
\isacommand{fix}\isamarkupfalse%
\ p\ q\ r\ {\isacharcolon}{\isacharcolon}\ {\isacharprime}a\isanewline
\isanewline
\ \ \isacommand{have}\isamarkupfalse%
\ {\isachardoublequoteopen}atoms\ {\isacharparenleft}Atom\ p{\isacharparenright}\ {\isacharequal}\ {\isacharbraceleft}p{\isacharbraceright}{\isachardoublequoteclose}\isanewline
\ \ \ \ \isacommand{by}\isamarkupfalse%
\ {\isacharparenleft}simp\ only{\isacharcolon}\ formula{\isachardot}set{\isacharparenright}\isanewline
\isanewline
\ \ \isacommand{have}\isamarkupfalse%
\ {\isachardoublequoteopen}atoms\ {\isacharparenleft}\isactrlbold {\isasymnot}\ {\isacharparenleft}Atom\ p{\isacharparenright}{\isacharparenright}\ {\isacharequal}\ {\isacharbraceleft}p{\isacharbraceright}{\isachardoublequoteclose}\isanewline
\ \ \ \ \isacommand{by}\isamarkupfalse%
\ {\isacharparenleft}simp\ only{\isacharcolon}\ formula{\isachardot}set{\isacharparenright}\isanewline
\isanewline
\ \ \isacommand{have}\isamarkupfalse%
\ {\isachardoublequoteopen}atoms\ {\isacharparenleft}{\isacharparenleft}Atom\ p\ \isactrlbold {\isasymrightarrow}\ Atom\ q{\isacharparenright}\ \isactrlbold {\isasymor}\ Atom\ r{\isacharparenright}\ {\isacharequal}\ {\isacharbraceleft}p{\isacharcomma}q{\isacharcomma}r{\isacharbraceright}{\isachardoublequoteclose}\isanewline
\ \ \ \ \isacommand{by}\isamarkupfalse%
\ auto\isanewline
\isanewline
\ \ \isacommand{have}\isamarkupfalse%
\ {\isachardoublequoteopen}atoms\ {\isacharparenleft}{\isacharparenleft}Atom\ p\ \isactrlbold {\isasymrightarrow}\ Atom\ p{\isacharparenright}\ \isactrlbold {\isasymor}\ Atom\ r{\isacharparenright}\ {\isacharequal}\ {\isacharbraceleft}p{\isacharcomma}r{\isacharbraceright}{\isachardoublequoteclose}\isanewline
\ \ \ \ \isacommand{by}\isamarkupfalse%
\ auto%
\endisatagproof
{\isafoldproof}%
%
\isadelimproof
\ \ \isanewline
%
\endisadelimproof
\isanewline
\isacommand{end}\isamarkupfalse%
%
\begin{isamarkuptext}%
En particular, el conjunto de símbolos proposicionales de la 
  fórmula \isa{Bot} es vacío. Además, para calcular esta constante es 
  necesario especificar el tipo sobre el que se construye la fórmula.%
\end{isamarkuptext}\isamarkuptrue%
\isacommand{notepad}\isamarkupfalse%
\ \isanewline
\isakeyword{begin}\isanewline
%
\isadelimproof
\ \ %
\endisadelimproof
%
\isatagproof
\isacommand{fix}\isamarkupfalse%
\ p\ {\isacharcolon}{\isacharcolon}\ {\isacharprime}a\isanewline
\isanewline
\ \ \isacommand{have}\isamarkupfalse%
\ {\isachardoublequoteopen}atoms\ {\isasymbottom}\ {\isacharequal}\ {\isasymemptyset}{\isachardoublequoteclose}\isanewline
\ \ \ \ \isacommand{by}\isamarkupfalse%
\ {\isacharparenleft}simp\ only{\isacharcolon}\ formula{\isachardot}set{\isacharparenright}\isanewline
\isanewline
\ \ \isacommand{have}\isamarkupfalse%
\ {\isachardoublequoteopen}atoms\ {\isacharparenleft}Atom\ p\ \isactrlbold {\isasymor}\ {\isasymbottom}{\isacharparenright}\ {\isacharequal}\ {\isacharbraceleft}p{\isacharbraceright}{\isachardoublequoteclose}\isanewline
\ \ \isacommand{proof}\isamarkupfalse%
\ {\isacharminus}\isanewline
\ \ \ \ \isacommand{have}\isamarkupfalse%
\ {\isachardoublequoteopen}atoms\ {\isacharparenleft}Atom\ p\ \isactrlbold {\isasymor}\ {\isasymbottom}{\isacharparenright}\ {\isacharequal}\ atoms\ {\isacharparenleft}Atom\ p{\isacharparenright}\ {\isasymunion}\ atoms\ Bot{\isachardoublequoteclose}\isanewline
\ \ \ \ \ \ \isacommand{by}\isamarkupfalse%
\ {\isacharparenleft}simp\ only{\isacharcolon}\ formula{\isachardot}set{\isacharparenleft}{\isadigit{5}}{\isacharparenright}{\isacharparenright}\isanewline
\ \ \ \ \isacommand{also}\isamarkupfalse%
\ \isacommand{have}\isamarkupfalse%
\ {\isachardoublequoteopen}{\isasymdots}\ {\isacharequal}\ {\isacharbraceleft}p{\isacharbraceright}\ {\isasymunion}\ atoms\ Bot{\isachardoublequoteclose}\isanewline
\ \ \ \ \ \ \isacommand{by}\isamarkupfalse%
\ {\isacharparenleft}simp\ only{\isacharcolon}\ formula{\isachardot}set{\isacharparenleft}{\isadigit{1}}{\isacharparenright}{\isacharparenright}\isanewline
\ \ \ \ \isacommand{also}\isamarkupfalse%
\ \isacommand{have}\isamarkupfalse%
\ {\isachardoublequoteopen}{\isasymdots}\ {\isacharequal}\ {\isacharbraceleft}p{\isacharbraceright}\ {\isasymunion}\ {\isasymemptyset}{\isachardoublequoteclose}\isanewline
\ \ \ \ \ \ \isacommand{by}\isamarkupfalse%
\ {\isacharparenleft}simp\ only{\isacharcolon}\ formula{\isachardot}set{\isacharparenleft}{\isadigit{2}}{\isacharparenright}{\isacharparenright}\isanewline
\ \ \ \ \isacommand{also}\isamarkupfalse%
\ \isacommand{have}\isamarkupfalse%
\ {\isachardoublequoteopen}{\isasymdots}\ {\isacharequal}\ {\isacharbraceleft}p{\isacharbraceright}{\isachardoublequoteclose}\isanewline
\ \ \ \ \ \ \isacommand{by}\isamarkupfalse%
\ {\isacharparenleft}simp\ only{\isacharcolon}\ Un{\isacharunderscore}empty{\isacharunderscore}right{\isacharparenright}\isanewline
\ \ \ \ \isacommand{finally}\isamarkupfalse%
\ \isacommand{show}\isamarkupfalse%
\ {\isachardoublequoteopen}atoms\ {\isacharparenleft}Atom\ p\ \isactrlbold {\isasymor}\ {\isasymbottom}{\isacharparenright}\ {\isacharequal}\ {\isacharbraceleft}p{\isacharbraceright}{\isachardoublequoteclose}\isanewline
\ \ \ \ \ \ \isacommand{by}\isamarkupfalse%
\ this\isanewline
\ \ \isacommand{qed}\isamarkupfalse%
\isanewline
\isanewline
\ \ \isacommand{have}\isamarkupfalse%
\ {\isachardoublequoteopen}atoms\ {\isacharparenleft}Atom\ p\ \isactrlbold {\isasymor}\ {\isasymbottom}{\isacharparenright}\ {\isacharequal}\ {\isacharbraceleft}p{\isacharbraceright}{\isachardoublequoteclose}\isanewline
\ \ \ \ \isacommand{by}\isamarkupfalse%
\ {\isacharparenleft}simp\ only{\isacharcolon}\ formula{\isachardot}set\ Un{\isacharunderscore}empty{\isacharunderscore}right{\isacharparenright}%
\endisatagproof
{\isafoldproof}%
%
\isadelimproof
\isanewline
%
\endisadelimproof
\isacommand{end}\isamarkupfalse%
\isanewline
\isanewline
\isacommand{value}\isamarkupfalse%
\ {\isachardoublequoteopen}{\isacharparenleft}Bot{\isacharcolon}{\isacharcolon}nat\ formula{\isacharparenright}{\isachardoublequoteclose}%
\begin{isamarkuptext}%
Una vez definida la estructura de las fórmulas, vamos a introducir 
  el método de demostración que seguirán los resultados que aquí 
  presentaremos, tanto en la teoría clásica como en Isabelle. 

  Según la definición recursiva de las fórmulas, dispondremos de un 
  esquema de inducción sobre las mismas:

  \begin{definicion}
    Sea \isa{{\isasymphi}} una propiedad sobre fórmulas que verifica las siguientes 
    condiciones:
    \begin{itemize}
      \item Las fórmulas atómicas la cumplen.
      \item La constante \isa{{\isasymbottom}} la cumple.
      \item Dada \isa{F} fórmula que la cumple, entonces \isa{{\isasymnot}\ F} la cumple.
      \item Dadas \isa{F} y \isa{G} fórmulas que la cumplen, entonces \isa{F\ {\isacharasterisk}\ G} la 
        cumple, donde \isa{{\isacharasterisk}} simboliza cualquier conectiva binaria.
    \end{itemize}
    Entonces, todas las fórmulas proposicionales tienen la propiedad 
    \isa{{\isasymphi}}.
  \end{definicion}

  Análogamente, como las fórmulas proposicionales están definidas 
  mediante un tipo de datos recursivo, Isabelle genera de forma 
  automática el esquema de inducción correspondiente. De este modo, en 
  las pruebas formalizadas utilizaremos la táctica \isa{induction}, 
  que corresponde al siguiente esquema.

  \begin{itemize}
    \item[] \isa{{\isasymlbrakk}{\isasymAnd}x{\isachardot}\ P\ {\isacharparenleft}Atom\ x{\isacharparenright}{\isacharsemicolon}\ P\ {\isasymbottom}{\isacharsemicolon}\ {\isasymAnd}x{\isachardot}\ P\ x\ {\isasymLongrightarrow}\ P\ {\isacharparenleft}\isactrlbold {\isasymnot}\ x{\isacharparenright}{\isacharsemicolon}\ {\isasymAnd}x{\isadigit{1}}a\ x{\isadigit{2}}{\isachardot}\ P\ x{\isadigit{1}}a\ {\isasymand}\ P\ x{\isadigit{2}}\ {\isasymLongrightarrow}\ P\ {\isacharparenleft}x{\isadigit{1}}a\ \isactrlbold {\isasymand}\ x{\isadigit{2}}{\isacharparenright}{\isacharsemicolon}\ {\isasymAnd}x{\isadigit{1}}a\ x{\isadigit{2}}{\isachardot}\ P\ x{\isadigit{1}}a\ {\isasymand}\ P\ x{\isadigit{2}}\ {\isasymLongrightarrow}\ P\ {\isacharparenleft}x{\isadigit{1}}a\ \isactrlbold {\isasymor}\ x{\isadigit{2}}{\isacharparenright}{\isacharsemicolon}\ {\isasymAnd}x{\isadigit{1}}a\ x{\isadigit{2}}{\isachardot}\ P\ x{\isadigit{1}}a\ {\isasymand}\ P\ x{\isadigit{2}}\ {\isasymLongrightarrow}\ P\ {\isacharparenleft}x{\isadigit{1}}a\ \isactrlbold {\isasymrightarrow}\ x{\isadigit{2}}{\isacharparenright}{\isasymrbrakk}\ {\isasymLongrightarrow}\ P\ formula}
  \end{itemize} 

  Como hemos señalado, el esquema inductivo se aplicará en cada uno de 
  los casos de los constructores, desglosándose así seis casos distintos 
  como se muestra anteriormente. Además, todas las demostraciones sobre 
  casos de conectivas binarias son equivalentes en esta sección, pues la 
  construcción sintáctica de fórmulas es idéntica entre ellas. Estas se 
  diferencian esencialmente en la connotación semántica que veremos más 
  adelante. Por tanto, para simplificar algunas demostraciones 
  sintácticas más extensas, expondremos la prueba estructurada 
  únicamente para uno de los casos de conectivas binarias.

  Llegamos así al primer resultado de este apartado:

  \begin{lema}
    El conjunto de los átomos de una fórmula proposicional es finito.
  \end{lema}

  Para proceder a la demostración, vamos a dar una definición inductiva 
  de conjunto finito que tendrá la clave de la prueba del lema. Cabe 
  añadir que la demostración seguirá el esquema inductivo relativo a la 
  estructura de fórmula, y no el que resulta de esta definición.

  \begin{definicion}
    Los conjuntos finitos son:
      \begin{itemize}
        \item El vacío.
        \item Dado un conjunto finito \isa{A} y un elemento cualquiera \isa{a}, 
          entonces \isa{{\isacharbraceleft}a{\isacharbraceright}\ {\isasymunion}\ A} es finito.
      \end{itemize}
  \end{definicion}


  En Isabelle, podemos formalizar el lema como sigue.%
\end{isamarkuptext}\isamarkuptrue%
\isacommand{lemma}\isamarkupfalse%
\ {\isachardoublequoteopen}finite\ {\isacharparenleft}atoms\ F{\isacharparenright}{\isachardoublequoteclose}\isanewline
%
\isadelimproof
\ \ %
\endisadelimproof
%
\isatagproof
\isacommand{oops}\isamarkupfalse%
%
\endisatagproof
{\isafoldproof}%
%
\isadelimproof
%
\endisadelimproof
%
\begin{isamarkuptext}%
Análogamente, el enunciado formalizado contiene la defición 
  \isa{finite\ S}, perteneciente a la teoría 
  \href{https://n9.cl/x86r}{FiniteSet.thy}.%
\end{isamarkuptext}\isamarkuptrue%
\isacommand{inductive}\isamarkupfalse%
\ finite{\isacharprime}\ {\isacharcolon}{\isacharcolon}\ {\isachardoublequoteopen}{\isacharprime}a\ set\ {\isasymRightarrow}\ bool{\isachardoublequoteclose}\ \isakeyword{where}\isanewline
\ \ emptyI{\isacharprime}\ {\isacharbrackleft}simp{\isacharcomma}\ intro{\isacharbang}{\isacharbrackright}{\isacharcolon}\ {\isachardoublequoteopen}finite{\isacharprime}\ {\isacharbraceleft}{\isacharbraceright}{\isachardoublequoteclose}\isanewline
{\isacharbar}\ insertI{\isacharprime}\ {\isacharbrackleft}simp{\isacharcomma}\ intro{\isacharbang}{\isacharbrackright}{\isacharcolon}\ {\isachardoublequoteopen}finite{\isacharprime}\ A\ {\isasymLongrightarrow}\ finite{\isacharprime}\ {\isacharparenleft}insert\ a\ A{\isacharparenright}{\isachardoublequoteclose}%
\begin{isamarkuptext}%
Observemos que la definición anterior corresponde a 
  \isa{finite{\isacharprime}}. Sin embargo, es equivalente a \isa{finite} de la 
  teoría original. Este cambio de notación es necesario para no definir 
  dos veces de manera idéntica la misma noción en Isabelle. Por otra 
  parte, esta definición permitiría la demostración del lema por 
  simplificacion pues, dentro de ella las reglas que especifica se han 
  añadido como tácticas de \isa{simp} e \isa{intro{\isacharbang}}. Sin embargo, conforme al 
  objetivo de este análisis, detallaremos dónde es usada cada una de las 
  reglas en la prueba detallada. 

  A continuación, veamos en primer lugar la demostración clásica del 
  lema. 

  \begin{demostracion}
  La prueba es por inducción sobre el tipo recursivo de las fórmulas. 
  Veamos cada caso.
  
  Consideremos una fórmula atómica \isa{p} cualquiera. Entonces, 
  \isa{conjAtoms{\isacharparenleft}Atom\ p{\isacharparenright}\ {\isacharequal}\ {\isacharbraceleft}p{\isacharbraceright}\ {\isacharequal}\ {\isacharbraceleft}p{\isacharbraceright}\ {\isasymunion}\ {\isasymemptyset}} es finito.

  Sea la fórmula \isa{{\isasymbottom}}. Entonces, \isa{conjAtoms{\isacharparenleft}{\isasymbottom}{\isacharparenright}\ {\isacharequal}\ {\isasymemptyset}} y, por lo tanto, 
  finito.
  
  Sea \isa{F} una fórmula tal que \isa{conjAtoms{\isacharparenleft}F{\isacharparenright}} es finito. Entonces, por 
  definición, \isa{conjAtoms{\isacharparenleft}{\isasymnot}\ F{\isacharparenright}\ {\isacharequal}\ conjAtoms{\isacharparenleft}F{\isacharparenright}} y, por hipótesis de 
  inducción, es finito.

  Consideremos las fórmulas \isa{F} y \isa{G} cuyos conjuntos de átomos 
  \isa{conjAtoms{\isacharparenleft}F{\isacharparenright}} y \isa{conjAtoms{\isacharparenleft}G{\isacharparenright}} son finitos. Por construcción, 
  \isa{conjAtoms{\isacharparenleft}F{\isacharasterisk}G{\isacharparenright}\ {\isacharequal}\ conjAtoms{\isacharparenleft}F{\isacharparenright}\ {\isasymunion}\ conjAtoms{\isacharparenleft}G{\isacharparenright}} para cualquier \isa{{\isacharasterisk}} 
  conectiva binaria. Por lo tanto, por hipótesis de inducción, 
  \isa{conjAtoms{\isacharparenleft}F{\isacharasterisk}G{\isacharparenright}} es finito. 
  \end{demostracion} 

  Veamos ahora la prueba detallada en Isabelle del resultado que, aunque 
  es sencillo, nos muestra un ejemplo claro de la estructura inductiva 
  que nos acompañará en las siguientes demostraciones. En este primer 
  lema mostraré con detalle de todos los casos de conectivas binarias, 
  aunque se puede observar que son completamente equivalentes. Para 
  facilitar la lectura, primero demostraremos por separado cada uno de 
  los casos según el esquema inductivo de fórmulas, y finalmente 
  añadiremos la prueba para una fórmula cualquiera a partir de los 
  anteriores.%
\end{isamarkuptext}\isamarkuptrue%
\isacommand{lemma}\isamarkupfalse%
\ atoms{\isacharunderscore}finite{\isacharunderscore}atom{\isacharcolon}\isanewline
\ \ {\isachardoublequoteopen}finite\ {\isacharparenleft}atoms\ {\isacharparenleft}Atom\ x{\isacharparenright}{\isacharparenright}{\isachardoublequoteclose}\isanewline
%
\isadelimproof
%
\endisadelimproof
%
\isatagproof
\isacommand{proof}\isamarkupfalse%
\ {\isacharminus}\isanewline
\ \ \isacommand{have}\isamarkupfalse%
\ {\isachardoublequoteopen}finite\ {\isasymemptyset}{\isachardoublequoteclose}\isanewline
\ \ \ \ \isacommand{by}\isamarkupfalse%
\ {\isacharparenleft}simp\ only{\isacharcolon}\ finite{\isachardot}emptyI{\isacharparenright}\isanewline
\ \ \isacommand{then}\isamarkupfalse%
\ \isacommand{have}\isamarkupfalse%
\ {\isachardoublequoteopen}finite\ {\isacharbraceleft}x{\isacharbraceright}{\isachardoublequoteclose}\isanewline
\ \ \ \ \isacommand{by}\isamarkupfalse%
\ {\isacharparenleft}simp\ only{\isacharcolon}\ finite{\isacharunderscore}insert{\isacharparenright}\isanewline
\ \ \isacommand{then}\isamarkupfalse%
\ \isacommand{show}\isamarkupfalse%
\ {\isachardoublequoteopen}finite\ {\isacharparenleft}atoms\ {\isacharparenleft}Atom\ x{\isacharparenright}{\isacharparenright}{\isachardoublequoteclose}\isanewline
\ \ \ \ \isacommand{by}\isamarkupfalse%
\ {\isacharparenleft}simp\ only{\isacharcolon}\ formula{\isachardot}set{\isacharparenleft}{\isadigit{1}}{\isacharparenright}{\isacharparenright}\ \isanewline
\isacommand{qed}\isamarkupfalse%
%
\endisatagproof
{\isafoldproof}%
%
\isadelimproof
\isanewline
%
\endisadelimproof
\isanewline
\isacommand{lemma}\isamarkupfalse%
\ atoms{\isacharunderscore}finite{\isacharunderscore}bot{\isacharcolon}\isanewline
\ \ {\isachardoublequoteopen}finite\ {\isacharparenleft}atoms\ {\isasymbottom}{\isacharparenright}{\isachardoublequoteclose}\isanewline
%
\isadelimproof
%
\endisadelimproof
%
\isatagproof
\isacommand{proof}\isamarkupfalse%
\ {\isacharminus}\isanewline
\ \ \isacommand{have}\isamarkupfalse%
\ {\isachardoublequoteopen}finite\ {\isasymemptyset}{\isachardoublequoteclose}\isanewline
\ \ \ \ \isacommand{by}\isamarkupfalse%
\ {\isacharparenleft}simp\ only{\isacharcolon}\ finite{\isachardot}emptyI{\isacharparenright}\isanewline
\ \ \isacommand{then}\isamarkupfalse%
\ \isacommand{show}\isamarkupfalse%
\ {\isachardoublequoteopen}finite\ {\isacharparenleft}atoms\ {\isasymbottom}{\isacharparenright}{\isachardoublequoteclose}\isanewline
\ \ \ \ \isacommand{by}\isamarkupfalse%
\ {\isacharparenleft}simp\ only{\isacharcolon}\ formula{\isachardot}set{\isacharparenleft}{\isadigit{2}}{\isacharparenright}{\isacharparenright}\ \isanewline
\isacommand{qed}\isamarkupfalse%
%
\endisatagproof
{\isafoldproof}%
%
\isadelimproof
\isanewline
%
\endisadelimproof
\isanewline
\isacommand{lemma}\isamarkupfalse%
\ atoms{\isacharunderscore}finite{\isacharunderscore}not{\isacharcolon}\isanewline
\ \ \isakeyword{assumes}\ {\isachardoublequoteopen}finite\ {\isacharparenleft}atoms\ F{\isacharparenright}{\isachardoublequoteclose}\ \isanewline
\ \ \isakeyword{shows}\ \ \ {\isachardoublequoteopen}finite\ {\isacharparenleft}atoms\ {\isacharparenleft}\isactrlbold {\isasymnot}\ F{\isacharparenright}{\isacharparenright}{\isachardoublequoteclose}\isanewline
%
\isadelimproof
\ \ %
\endisadelimproof
%
\isatagproof
\isacommand{using}\isamarkupfalse%
\ assms\isanewline
\ \ \isacommand{by}\isamarkupfalse%
\ {\isacharparenleft}simp\ only{\isacharcolon}\ formula{\isachardot}set{\isacharparenleft}{\isadigit{3}}{\isacharparenright}{\isacharparenright}%
\endisatagproof
{\isafoldproof}%
%
\isadelimproof
\ \isanewline
%
\endisadelimproof
\isanewline
\isacommand{lemma}\isamarkupfalse%
\ atoms{\isacharunderscore}finite{\isacharunderscore}and{\isacharcolon}\isanewline
\ \ \isakeyword{assumes}\ {\isachardoublequoteopen}finite\ {\isacharparenleft}atoms\ F{\isadigit{1}}{\isacharparenright}{\isachardoublequoteclose}\isanewline
\ \ \ \ \ \ \ \ \ \ {\isachardoublequoteopen}finite\ {\isacharparenleft}atoms\ F{\isadigit{2}}{\isacharparenright}{\isachardoublequoteclose}\isanewline
\ \ \isakeyword{shows}\ \ \ {\isachardoublequoteopen}finite\ {\isacharparenleft}atoms\ {\isacharparenleft}F{\isadigit{1}}\ \isactrlbold {\isasymand}\ F{\isadigit{2}}{\isacharparenright}{\isacharparenright}{\isachardoublequoteclose}\isanewline
%
\isadelimproof
%
\endisadelimproof
%
\isatagproof
\isacommand{proof}\isamarkupfalse%
\ {\isacharminus}\isanewline
\ \ \isacommand{have}\isamarkupfalse%
\ {\isachardoublequoteopen}finite\ {\isacharparenleft}atoms\ F{\isadigit{1}}\ {\isasymunion}\ atoms\ F{\isadigit{2}}{\isacharparenright}{\isachardoublequoteclose}\isanewline
\ \ \ \ \isacommand{using}\isamarkupfalse%
\ assms\isanewline
\ \ \ \ \isacommand{by}\isamarkupfalse%
\ {\isacharparenleft}simp\ only{\isacharcolon}\ finite{\isacharunderscore}UnI{\isacharparenright}\isanewline
\ \ \isacommand{then}\isamarkupfalse%
\ \isacommand{show}\isamarkupfalse%
\ {\isachardoublequoteopen}finite\ {\isacharparenleft}atoms\ {\isacharparenleft}F{\isadigit{1}}\ \isactrlbold {\isasymand}\ F{\isadigit{2}}{\isacharparenright}{\isacharparenright}{\isachardoublequoteclose}\ \ \isanewline
\ \ \ \ \isacommand{by}\isamarkupfalse%
\ {\isacharparenleft}simp\ only{\isacharcolon}\ formula{\isachardot}set{\isacharparenleft}{\isadigit{4}}{\isacharparenright}{\isacharparenright}\isanewline
\isacommand{qed}\isamarkupfalse%
%
\endisatagproof
{\isafoldproof}%
%
\isadelimproof
\isanewline
%
\endisadelimproof
\isanewline
\isacommand{lemma}\isamarkupfalse%
\ atoms{\isacharunderscore}finite{\isacharunderscore}or{\isacharcolon}\isanewline
\ \ \isakeyword{assumes}\ {\isachardoublequoteopen}finite\ {\isacharparenleft}atoms\ F{\isadigit{1}}{\isacharparenright}{\isachardoublequoteclose}\isanewline
\ \ \ \ \ \ \ \ \ \ {\isachardoublequoteopen}finite\ {\isacharparenleft}atoms\ F{\isadigit{2}}{\isacharparenright}{\isachardoublequoteclose}\isanewline
\ \ \isakeyword{shows}\ \ \ {\isachardoublequoteopen}finite\ {\isacharparenleft}atoms\ {\isacharparenleft}F{\isadigit{1}}\ \isactrlbold {\isasymor}\ F{\isadigit{2}}{\isacharparenright}{\isacharparenright}{\isachardoublequoteclose}\isanewline
%
\isadelimproof
%
\endisadelimproof
%
\isatagproof
\isacommand{proof}\isamarkupfalse%
\ {\isacharminus}\isanewline
\ \ \isacommand{have}\isamarkupfalse%
\ {\isachardoublequoteopen}finite\ {\isacharparenleft}atoms\ F{\isadigit{1}}\ {\isasymunion}\ atoms\ F{\isadigit{2}}{\isacharparenright}{\isachardoublequoteclose}\isanewline
\ \ \ \ \isacommand{using}\isamarkupfalse%
\ assms\isanewline
\ \ \ \ \isacommand{by}\isamarkupfalse%
\ {\isacharparenleft}simp\ only{\isacharcolon}\ finite{\isacharunderscore}UnI{\isacharparenright}\isanewline
\ \ \isacommand{then}\isamarkupfalse%
\ \isacommand{show}\isamarkupfalse%
\ {\isachardoublequoteopen}finite\ {\isacharparenleft}atoms\ {\isacharparenleft}F{\isadigit{1}}\ \isactrlbold {\isasymor}\ F{\isadigit{2}}{\isacharparenright}{\isacharparenright}{\isachardoublequoteclose}\ \ \isanewline
\ \ \ \ \isacommand{by}\isamarkupfalse%
\ {\isacharparenleft}simp\ only{\isacharcolon}\ formula{\isachardot}set{\isacharparenleft}{\isadigit{5}}{\isacharparenright}{\isacharparenright}\isanewline
\isacommand{qed}\isamarkupfalse%
%
\endisatagproof
{\isafoldproof}%
%
\isadelimproof
\isanewline
%
\endisadelimproof
\isanewline
\isacommand{lemma}\isamarkupfalse%
\ atoms{\isacharunderscore}finite{\isacharunderscore}imp{\isacharcolon}\isanewline
\ \ \isakeyword{assumes}\ {\isachardoublequoteopen}finite\ {\isacharparenleft}atoms\ F{\isadigit{1}}{\isacharparenright}{\isachardoublequoteclose}\isanewline
\ \ \ \ \ \ \ \ \ \ {\isachardoublequoteopen}finite\ {\isacharparenleft}atoms\ F{\isadigit{2}}{\isacharparenright}{\isachardoublequoteclose}\isanewline
\ \ \isakeyword{shows}\ \ \ {\isachardoublequoteopen}finite\ {\isacharparenleft}atoms\ {\isacharparenleft}F{\isadigit{1}}\ \isactrlbold {\isasymrightarrow}\ F{\isadigit{2}}{\isacharparenright}{\isacharparenright}{\isachardoublequoteclose}\isanewline
%
\isadelimproof
%
\endisadelimproof
%
\isatagproof
\isacommand{proof}\isamarkupfalse%
\ {\isacharminus}\isanewline
\ \ \isacommand{have}\isamarkupfalse%
\ {\isachardoublequoteopen}finite\ {\isacharparenleft}atoms\ F{\isadigit{1}}\ {\isasymunion}\ atoms\ F{\isadigit{2}}{\isacharparenright}{\isachardoublequoteclose}\isanewline
\ \ \ \ \isacommand{using}\isamarkupfalse%
\ assms\isanewline
\ \ \ \ \isacommand{by}\isamarkupfalse%
\ {\isacharparenleft}simp\ only{\isacharcolon}\ finite{\isacharunderscore}UnI{\isacharparenright}\isanewline
\ \ \isacommand{then}\isamarkupfalse%
\ \isacommand{show}\isamarkupfalse%
\ {\isachardoublequoteopen}finite\ {\isacharparenleft}atoms\ {\isacharparenleft}F{\isadigit{1}}\ \isactrlbold {\isasymrightarrow}\ F{\isadigit{2}}{\isacharparenright}{\isacharparenright}{\isachardoublequoteclose}\ \ \isanewline
\ \ \ \ \isacommand{by}\isamarkupfalse%
\ {\isacharparenleft}simp\ only{\isacharcolon}\ formula{\isachardot}set{\isacharparenleft}{\isadigit{6}}{\isacharparenright}{\isacharparenright}\isanewline
\isacommand{qed}\isamarkupfalse%
%
\endisatagproof
{\isafoldproof}%
%
\isadelimproof
\isanewline
%
\endisadelimproof
\isanewline
\isacommand{lemma}\isamarkupfalse%
\ atoms{\isacharunderscore}finite{\isacharcolon}\ {\isachardoublequoteopen}finite\ {\isacharparenleft}atoms\ F{\isacharparenright}{\isachardoublequoteclose}\isanewline
%
\isadelimproof
%
\endisadelimproof
%
\isatagproof
\isacommand{proof}\isamarkupfalse%
\ {\isacharparenleft}induction\ F{\isacharparenright}\isanewline
\ \ \isacommand{case}\isamarkupfalse%
\ {\isacharparenleft}Atom\ x{\isacharparenright}\isanewline
\ \ \isacommand{then}\isamarkupfalse%
\ \isacommand{show}\isamarkupfalse%
\ {\isacharquery}case\ \isacommand{by}\isamarkupfalse%
\ {\isacharparenleft}simp\ only{\isacharcolon}\ atoms{\isacharunderscore}finite{\isacharunderscore}atom{\isacharparenright}\isanewline
\isacommand{next}\isamarkupfalse%
\isanewline
\ \ \isacommand{case}\isamarkupfalse%
\ Bot\isanewline
\ \ \isacommand{then}\isamarkupfalse%
\ \isacommand{show}\isamarkupfalse%
\ {\isacharquery}case\ \isacommand{by}\isamarkupfalse%
\ {\isacharparenleft}simp\ only{\isacharcolon}\ atoms{\isacharunderscore}finite{\isacharunderscore}bot{\isacharparenright}\isanewline
\isacommand{next}\isamarkupfalse%
\isanewline
\ \ \isacommand{case}\isamarkupfalse%
\ {\isacharparenleft}Not\ F{\isacharparenright}\isanewline
\ \ \isacommand{then}\isamarkupfalse%
\ \isacommand{show}\isamarkupfalse%
\ {\isacharquery}case\ \isacommand{by}\isamarkupfalse%
\ {\isacharparenleft}simp\ only{\isacharcolon}\ atoms{\isacharunderscore}finite{\isacharunderscore}not{\isacharparenright}\isanewline
\isacommand{next}\isamarkupfalse%
\isanewline
\ \ \isacommand{case}\isamarkupfalse%
\ {\isacharparenleft}And\ F{\isadigit{1}}\ F{\isadigit{2}}{\isacharparenright}\isanewline
\ \ \isacommand{then}\isamarkupfalse%
\ \isacommand{show}\isamarkupfalse%
\ {\isacharquery}case\ \isacommand{by}\isamarkupfalse%
\ {\isacharparenleft}simp\ only{\isacharcolon}\ atoms{\isacharunderscore}finite{\isacharunderscore}and{\isacharparenright}\isanewline
\isacommand{next}\isamarkupfalse%
\isanewline
\ \ \isacommand{case}\isamarkupfalse%
\ {\isacharparenleft}Or\ F{\isadigit{1}}\ F{\isadigit{2}}{\isacharparenright}\isanewline
\ \ \isacommand{then}\isamarkupfalse%
\ \isacommand{show}\isamarkupfalse%
\ {\isacharquery}case\ \isacommand{by}\isamarkupfalse%
\ {\isacharparenleft}simp\ only{\isacharcolon}\ atoms{\isacharunderscore}finite{\isacharunderscore}or{\isacharparenright}\isanewline
\isacommand{next}\isamarkupfalse%
\isanewline
\ \ \isacommand{case}\isamarkupfalse%
\ {\isacharparenleft}Imp\ F{\isadigit{1}}\ F{\isadigit{2}}{\isacharparenright}\isanewline
\ \ \isacommand{then}\isamarkupfalse%
\ \isacommand{show}\isamarkupfalse%
\ {\isacharquery}case\ \isacommand{by}\isamarkupfalse%
\ {\isacharparenleft}simp\ only{\isacharcolon}\ atoms{\isacharunderscore}finite{\isacharunderscore}imp{\isacharparenright}\isanewline
\isacommand{qed}\isamarkupfalse%
%
\endisatagproof
{\isafoldproof}%
%
\isadelimproof
%
\endisadelimproof
%
\begin{isamarkuptext}%
Su demostración automática es la siguiente.%
\end{isamarkuptext}\isamarkuptrue%
\isacommand{lemma}\isamarkupfalse%
\ {\isachardoublequoteopen}finite\ {\isacharparenleft}atoms\ F{\isacharparenright}{\isachardoublequoteclose}\ \isanewline
%
\isadelimproof
\ \ %
\endisadelimproof
%
\isatagproof
\isacommand{by}\isamarkupfalse%
\ {\isacharparenleft}induction\ F{\isacharparenright}\ simp{\isacharunderscore}all%
\endisatagproof
{\isafoldproof}%
%
\isadelimproof
%
\endisadelimproof
%
\isadelimdocument
%
\endisadelimdocument
%
\isatagdocument
%
\isamarkupsubsection{Subfórmulas%
}
\isamarkuptrue%
%
\endisatagdocument
{\isafolddocument}%
%
\isadelimdocument
%
\endisadelimdocument
%
\begin{isamarkuptext}%
Veamos la noción de subfórmulas.

  \begin{definicion}
  El conjunto de subfórmulas de una fórmula \isa{F}, notada \isa{Subf{\isacharparenleft}F{\isacharparenright}}, se 
  define recursivamente como:
    \begin{itemize}
      \item \isa{{\isacharbraceleft}{\isasymbottom}{\isacharbraceright}} si \isa{F} es \isa{{\isasymbottom}}.
      \item \isa{{\isacharbraceleft}F{\isacharbraceright}} si \isa{F} es una fórmula atómica.
      \item \isa{{\isacharbraceleft}F{\isacharbraceright}\ {\isasymunion}\ Subf{\isacharparenleft}G{\isacharparenright}} si \isa{F} es \isa{{\isasymnot}G}.
      \item \isa{{\isacharbraceleft}F{\isacharbraceright}\ {\isasymunion}\ Subf{\isacharparenleft}G{\isacharparenright}\ {\isasymunion}\ Subf{\isacharparenleft}H{\isacharparenright}} si \isa{F} es \isa{G{\isacharasterisk}H} donde \isa{{\isacharasterisk}} es 
        cualquier conectiva binaria.
    \end{itemize}
  \end{definicion}

  Para proceder a la formalización de Isabelle, seguiremos dos etapas. 
  En primer lugar, definimos la función primitiva recursiva 
  \isa{subformulae}. Esta nos devolverá la lista de todas las 
  subfórmulas de una fórmula original obtenidas recursivamente.%
\end{isamarkuptext}\isamarkuptrue%
\isacommand{primrec}\isamarkupfalse%
\ subformulae\ {\isacharcolon}{\isacharcolon}\ {\isachardoublequoteopen}{\isacharprime}a\ formula\ {\isasymRightarrow}\ {\isacharprime}a\ formula\ list{\isachardoublequoteclose}\ \isakeyword{where}\isanewline
\ \ {\isachardoublequoteopen}subformulae\ {\isacharparenleft}Atom\ k{\isacharparenright}\ {\isacharequal}\ {\isacharbrackleft}Atom\ k{\isacharbrackright}{\isachardoublequoteclose}\ \isanewline
{\isacharbar}\ {\isachardoublequoteopen}subformulae\ {\isasymbottom}\ \ \ \ \ \ \ \ {\isacharequal}\ {\isacharbrackleft}{\isasymbottom}{\isacharbrackright}{\isachardoublequoteclose}\ \isanewline
{\isacharbar}\ {\isachardoublequoteopen}subformulae\ {\isacharparenleft}\isactrlbold {\isasymnot}\ F{\isacharparenright}\ \ \ \ {\isacharequal}\ {\isacharparenleft}\isactrlbold {\isasymnot}\ F{\isacharparenright}\ {\isacharhash}\ subformulae\ F{\isachardoublequoteclose}\ \isanewline
{\isacharbar}\ {\isachardoublequoteopen}subformulae\ {\isacharparenleft}F\ \isactrlbold {\isasymand}\ G{\isacharparenright}\ \ {\isacharequal}\ {\isacharparenleft}F\ \isactrlbold {\isasymand}\ G{\isacharparenright}\ {\isacharhash}\ subformulae\ F\ {\isacharat}\ subformulae\ G{\isachardoublequoteclose}\ \isanewline
{\isacharbar}\ {\isachardoublequoteopen}subformulae\ {\isacharparenleft}F\ \isactrlbold {\isasymor}\ G{\isacharparenright}\ \ {\isacharequal}\ {\isacharparenleft}F\ \isactrlbold {\isasymor}\ G{\isacharparenright}\ {\isacharhash}\ subformulae\ F\ {\isacharat}\ subformulae\ G{\isachardoublequoteclose}\isanewline
{\isacharbar}\ {\isachardoublequoteopen}subformulae\ {\isacharparenleft}F\ \isactrlbold {\isasymrightarrow}\ G{\isacharparenright}\ {\isacharequal}\ {\isacharparenleft}F\ \isactrlbold {\isasymrightarrow}\ G{\isacharparenright}\ {\isacharhash}\ subformulae\ F\ {\isacharat}\ subformulae\ G{\isachardoublequoteclose}%
\begin{isamarkuptext}%
Observemos que, en la definición anterior, \isa{{\isacharhash}} es el operador que 
  añade un elemento al comienzo de una lista y \isa{{\isacharat}} concatena varias 
  listas. Siguiendo con los ejemplos, apliquemos \isa{subformulae} en 
  las distintas fórmulas. En particular, al tratarse de una lista pueden 
  aparecer elementos repetidos como se muestra a continuación.%
\end{isamarkuptext}\isamarkuptrue%
\isacommand{notepad}\isamarkupfalse%
\isanewline
\isakeyword{begin}\isanewline
%
\isadelimproof
\ \ %
\endisadelimproof
%
\isatagproof
\isacommand{fix}\isamarkupfalse%
\ p\ {\isacharcolon}{\isacharcolon}\ {\isacharprime}a\isanewline
\isanewline
\ \ \isacommand{have}\isamarkupfalse%
\ {\isachardoublequoteopen}subformulae\ {\isacharparenleft}Atom\ p{\isacharparenright}\ {\isacharequal}\ {\isacharbrackleft}Atom\ p{\isacharbrackright}{\isachardoublequoteclose}\isanewline
\ \ \ \ \isacommand{by}\isamarkupfalse%
\ simp\isanewline
\isanewline
\ \ \isacommand{have}\isamarkupfalse%
\ {\isachardoublequoteopen}subformulae\ {\isacharparenleft}\isactrlbold {\isasymnot}\ {\isacharparenleft}Atom\ p{\isacharparenright}{\isacharparenright}\ {\isacharequal}\ {\isacharbrackleft}\isactrlbold {\isasymnot}\ {\isacharparenleft}Atom\ p{\isacharparenright}{\isacharcomma}\ Atom\ p{\isacharbrackright}{\isachardoublequoteclose}\isanewline
\ \ \ \ \isacommand{by}\isamarkupfalse%
\ simp\isanewline
\isanewline
\ \ \isacommand{have}\isamarkupfalse%
\ {\isachardoublequoteopen}subformulae\ {\isacharparenleft}{\isacharparenleft}Atom\ p\ \isactrlbold {\isasymrightarrow}\ Atom\ q{\isacharparenright}\ \isactrlbold {\isasymor}\ Atom\ r{\isacharparenright}\ {\isacharequal}\ \isanewline
\ \ \ \ \ \ \ {\isacharbrackleft}{\isacharparenleft}Atom\ p\ \isactrlbold {\isasymrightarrow}\ Atom\ q{\isacharparenright}\ \isactrlbold {\isasymor}\ Atom\ r{\isacharcomma}\ Atom\ p\ \isactrlbold {\isasymrightarrow}\ Atom\ q{\isacharcomma}\ Atom\ p{\isacharcomma}\ Atom\ q{\isacharcomma}\ \isanewline
\ \ \ \ \ \ \ \ Atom\ r{\isacharbrackright}{\isachardoublequoteclose}\isanewline
\ \ \ \ \isacommand{by}\isamarkupfalse%
\ simp\isanewline
\isanewline
\ \ \isacommand{have}\isamarkupfalse%
\ {\isachardoublequoteopen}subformulae\ {\isacharparenleft}Atom\ p\ \isactrlbold {\isasymand}\ {\isasymbottom}{\isacharparenright}\ {\isacharequal}\ {\isacharbrackleft}Atom\ p\ \isactrlbold {\isasymand}\ {\isasymbottom}{\isacharcomma}\ Atom\ p{\isacharcomma}\ {\isasymbottom}{\isacharbrackright}{\isachardoublequoteclose}\isanewline
\ \ \ \ \isacommand{by}\isamarkupfalse%
\ simp\isanewline
\isanewline
\ \ \isacommand{have}\isamarkupfalse%
\ {\isachardoublequoteopen}subformulae\ {\isacharparenleft}Atom\ p\ \isactrlbold {\isasymor}\ Atom\ p{\isacharparenright}\ {\isacharequal}\ \isanewline
\ \ \ \ \ \ \ {\isacharbrackleft}Atom\ p\ \isactrlbold {\isasymor}\ Atom\ p{\isacharcomma}\ Atom\ p{\isacharcomma}\ Atom\ p{\isacharbrackright}{\isachardoublequoteclose}\isanewline
\ \ \ \ \isacommand{by}\isamarkupfalse%
\ simp%
\endisatagproof
{\isafoldproof}%
%
\isadelimproof
\isanewline
%
\endisadelimproof
\isacommand{end}\isamarkupfalse%
%
\begin{isamarkuptext}%
En la segunda etapa de formalización, definimos 
  \isa{setSubformulae}, que convierte al tipo conjunto la lista de 
  subfórmulas anterior.%
\end{isamarkuptext}\isamarkuptrue%
\isacommand{abbreviation}\isamarkupfalse%
\ setSubformulae\ {\isacharcolon}{\isacharcolon}\ {\isachardoublequoteopen}{\isacharprime}a\ formula\ {\isasymRightarrow}\ {\isacharprime}a\ formula\ set{\isachardoublequoteclose}\ \isakeyword{where}\isanewline
\ \ {\isachardoublequoteopen}setSubformulae\ F\ {\isasymequiv}\ set\ {\isacharparenleft}subformulae\ F{\isacharparenright}{\isachardoublequoteclose}%
\begin{isamarkuptext}%
De este modo, \isa{Subf{\isacharparenleft}·{\isacharparenright}} es equivalente a esta nueva definición. La 
  justificación para este cambio en el tipo reside en las propiedades 
  sobre conjuntos que facilitan las demostraciones de los resultados que 
  mostraremos a continuación, frente a las listas. Algunas de estas 
  ventajas son la eliminación de elementos repetidos o las operaciones 
  propias de teoría de conjuntos. Observemos los siguientes ejemplos con 
  el tipo de conjuntos.%
\end{isamarkuptext}\isamarkuptrue%
\isacommand{notepad}\isamarkupfalse%
\isanewline
\isakeyword{begin}\isanewline
%
\isadelimproof
\ \ %
\endisadelimproof
%
\isatagproof
\isacommand{fix}\isamarkupfalse%
\ p\ q\ r\ {\isacharcolon}{\isacharcolon}\ {\isacharprime}a\isanewline
\isanewline
\ \ \isacommand{have}\isamarkupfalse%
\ {\isachardoublequoteopen}setSubformulae\ {\isacharparenleft}Atom\ p\ \isactrlbold {\isasymor}\ Atom\ p{\isacharparenright}\ {\isacharequal}\ {\isacharbraceleft}Atom\ p\ \isactrlbold {\isasymor}\ Atom\ p{\isacharcomma}\ Atom\ p{\isacharbraceright}{\isachardoublequoteclose}\isanewline
\ \ \ \ \isacommand{by}\isamarkupfalse%
\ simp\isanewline
\ \ \isanewline
\ \ \isacommand{have}\isamarkupfalse%
\ {\isachardoublequoteopen}setSubformulae\ {\isacharparenleft}{\isacharparenleft}Atom\ p\ \isactrlbold {\isasymrightarrow}\ Atom\ q{\isacharparenright}\ \isactrlbold {\isasymor}\ Atom\ r{\isacharparenright}\ {\isacharequal}\isanewline
\ \ \ \ \ \ \ \ {\isacharbraceleft}{\isacharparenleft}Atom\ p\ \isactrlbold {\isasymrightarrow}\ Atom\ q{\isacharparenright}\ \isactrlbold {\isasymor}\ Atom\ r{\isacharcomma}\ Atom\ p\ \isactrlbold {\isasymrightarrow}\ Atom\ q{\isacharcomma}\ Atom\ p{\isacharcomma}\ Atom\ q{\isacharcomma}\ \isanewline
\ \ \ \ \ \ \ \ \ \ Atom\ r{\isacharbraceright}{\isachardoublequoteclose}\isanewline
\ \ \isacommand{by}\isamarkupfalse%
\ auto%
\endisatagproof
{\isafoldproof}%
%
\isadelimproof
\ \ \ \isanewline
%
\endisadelimproof
\isacommand{end}\isamarkupfalse%
%
\begin{isamarkuptext}%
Por otro lado, debemos señalar que el uso de 
  \isa{abbreviation} para definir \isa{setSubformulae} no es 
  arbitrario. Esta elección se debe a que el tipo \isa{abbreviation} 
  se trata de un sinónimo para una expresión cuyo tipo ya existe (en 
  nuestro caso, convertir en conjunto la lista obtenida con 
  \isa{subformulae}). No es una definición propiamente dicha, sino 
  una forma de nombrar la composición de las funciones \isa{set} y 
  \isa{subformulae}.

  En primer lugar, vamos a probar que \isa{setSubformulae} es 
  equivalente a \isa{Subf} en Isabelle. Para ello utilizaremos el 
  siguiente resultado sobre listas, probado como sigue.%
\end{isamarkuptext}\isamarkuptrue%
\isacommand{lemma}\isamarkupfalse%
\ set{\isacharunderscore}insert{\isacharcolon}\ {\isachardoublequoteopen}set\ {\isacharparenleft}x\ {\isacharhash}\ ys{\isacharparenright}\ {\isacharequal}\ {\isacharbraceleft}x{\isacharbraceright}\ {\isasymunion}\ set\ ys{\isachardoublequoteclose}\isanewline
%
\isadelimproof
\ \ %
\endisadelimproof
%
\isatagproof
\isacommand{by}\isamarkupfalse%
\ {\isacharparenleft}simp\ only{\isacharcolon}\ list{\isachardot}set{\isacharparenleft}{\isadigit{2}}{\isacharparenright}\ Un{\isacharunderscore}insert{\isacharunderscore}left\ sup{\isacharunderscore}bot{\isachardot}left{\isacharunderscore}neutral{\isacharparenright}%
\endisatagproof
{\isafoldproof}%
%
\isadelimproof
%
\endisadelimproof
%
\begin{isamarkuptext}%
Por tanto, obtenemos la equivalencia como resultado de los 
  siguientes lemas, que aparecen demostrados de manera detallada.%
\end{isamarkuptext}\isamarkuptrue%
\isacommand{lemma}\isamarkupfalse%
\ setSubformulae{\isacharunderscore}atom{\isacharcolon}\isanewline
\ \ {\isachardoublequoteopen}setSubformulae\ {\isacharparenleft}Atom\ p{\isacharparenright}\ {\isacharequal}\ {\isacharbraceleft}Atom\ p{\isacharbraceright}{\isachardoublequoteclose}\isanewline
%
\isadelimproof
\ \ \ \ %
\endisadelimproof
%
\isatagproof
\isacommand{by}\isamarkupfalse%
\ {\isacharparenleft}simp\ only{\isacharcolon}\ subformulae{\isachardot}simps{\isacharparenleft}{\isadigit{1}}{\isacharparenright}{\isacharcomma}\ simp\ only{\isacharcolon}\ list{\isachardot}set{\isacharparenright}%
\endisatagproof
{\isafoldproof}%
%
\isadelimproof
\isanewline
%
\endisadelimproof
\isanewline
\isacommand{lemma}\isamarkupfalse%
\ setSubformulae{\isacharunderscore}bot{\isacharcolon}\isanewline
\ \ {\isachardoublequoteopen}setSubformulae\ {\isacharparenleft}{\isasymbottom}{\isacharparenright}\ {\isacharequal}\ {\isacharbraceleft}{\isasymbottom}{\isacharbraceright}{\isachardoublequoteclose}\isanewline
%
\isadelimproof
\ \ \ \ %
\endisadelimproof
%
\isatagproof
\isacommand{by}\isamarkupfalse%
\ {\isacharparenleft}simp\ only{\isacharcolon}\ subformulae{\isachardot}simps{\isacharparenleft}{\isadigit{2}}{\isacharparenright}{\isacharcomma}\ simp\ only{\isacharcolon}\ list{\isachardot}set{\isacharparenright}%
\endisatagproof
{\isafoldproof}%
%
\isadelimproof
\isanewline
%
\endisadelimproof
\isanewline
\isacommand{lemma}\isamarkupfalse%
\ setSubformulae{\isacharunderscore}not{\isacharcolon}\isanewline
\ \ \isakeyword{shows}\ {\isachardoublequoteopen}setSubformulae\ {\isacharparenleft}\isactrlbold {\isasymnot}\ F{\isacharparenright}\ {\isacharequal}\ {\isacharbraceleft}\isactrlbold {\isasymnot}\ F{\isacharbraceright}\ {\isasymunion}\ setSubformulae\ F{\isachardoublequoteclose}\isanewline
%
\isadelimproof
%
\endisadelimproof
%
\isatagproof
\isacommand{proof}\isamarkupfalse%
\ {\isacharminus}\isanewline
\ \ \isacommand{have}\isamarkupfalse%
\ {\isachardoublequoteopen}setSubformulae\ {\isacharparenleft}\isactrlbold {\isasymnot}\ F{\isacharparenright}\ {\isacharequal}\ set\ {\isacharparenleft}\isactrlbold {\isasymnot}\ F\ {\isacharhash}\ subformulae\ F{\isacharparenright}{\isachardoublequoteclose}\isanewline
\ \ \ \ \isacommand{by}\isamarkupfalse%
\ {\isacharparenleft}simp\ only{\isacharcolon}\ subformulae{\isachardot}simps{\isacharparenleft}{\isadigit{3}}{\isacharparenright}{\isacharparenright}\isanewline
\ \ \isacommand{also}\isamarkupfalse%
\ \isacommand{have}\isamarkupfalse%
\ {\isachardoublequoteopen}{\isasymdots}\ {\isacharequal}\ {\isacharbraceleft}\isactrlbold {\isasymnot}\ F{\isacharbraceright}\ {\isasymunion}\ set\ {\isacharparenleft}subformulae\ F{\isacharparenright}{\isachardoublequoteclose}\isanewline
\ \ \ \ \isacommand{by}\isamarkupfalse%
\ {\isacharparenleft}simp\ only{\isacharcolon}\ set{\isacharunderscore}insert{\isacharparenright}\isanewline
\ \ \isacommand{finally}\isamarkupfalse%
\ \isacommand{show}\isamarkupfalse%
\ {\isacharquery}thesis\isanewline
\ \ \ \ \isacommand{by}\isamarkupfalse%
\ this\isanewline
\isacommand{qed}\isamarkupfalse%
%
\endisatagproof
{\isafoldproof}%
%
\isadelimproof
\isanewline
%
\endisadelimproof
\isanewline
\isacommand{lemma}\isamarkupfalse%
\ setSubformulae{\isacharunderscore}and{\isacharcolon}\ \isanewline
\ \ {\isachardoublequoteopen}setSubformulae\ {\isacharparenleft}F{\isadigit{1}}\ \isactrlbold {\isasymand}\ F{\isadigit{2}}{\isacharparenright}\ \isanewline
\ \ \ {\isacharequal}\ {\isacharbraceleft}F{\isadigit{1}}\ \isactrlbold {\isasymand}\ F{\isadigit{2}}{\isacharbraceright}\ {\isasymunion}\ {\isacharparenleft}setSubformulae\ F{\isadigit{1}}\ {\isasymunion}\ setSubformulae\ F{\isadigit{2}}{\isacharparenright}{\isachardoublequoteclose}\isanewline
%
\isadelimproof
%
\endisadelimproof
%
\isatagproof
\isacommand{proof}\isamarkupfalse%
\ {\isacharminus}\isanewline
\ \ \isacommand{have}\isamarkupfalse%
\ {\isachardoublequoteopen}setSubformulae\ {\isacharparenleft}F{\isadigit{1}}\ \isactrlbold {\isasymand}\ F{\isadigit{2}}{\isacharparenright}\ \isanewline
\ \ \ \ \ \ \ \ {\isacharequal}\ set\ {\isacharparenleft}{\isacharparenleft}F{\isadigit{1}}\ \isactrlbold {\isasymand}\ F{\isadigit{2}}{\isacharparenright}\ {\isacharhash}\ {\isacharparenleft}subformulae\ F{\isadigit{1}}\ {\isacharat}\ subformulae\ F{\isadigit{2}}{\isacharparenright}{\isacharparenright}{\isachardoublequoteclose}\isanewline
\ \ \ \ \isacommand{by}\isamarkupfalse%
\ {\isacharparenleft}simp\ only{\isacharcolon}\ subformulae{\isachardot}simps{\isacharparenleft}{\isadigit{4}}{\isacharparenright}{\isacharparenright}\isanewline
\ \ \isacommand{also}\isamarkupfalse%
\ \isacommand{have}\isamarkupfalse%
\ {\isachardoublequoteopen}{\isasymdots}\ {\isacharequal}\ {\isacharbraceleft}F{\isadigit{1}}\ \isactrlbold {\isasymand}\ F{\isadigit{2}}{\isacharbraceright}\ {\isasymunion}\ {\isacharparenleft}set\ {\isacharparenleft}subformulae\ F{\isadigit{1}}\ {\isacharat}\ subformulae\ F{\isadigit{2}}{\isacharparenright}{\isacharparenright}{\isachardoublequoteclose}\isanewline
\ \ \ \ \isacommand{by}\isamarkupfalse%
\ {\isacharparenleft}simp\ only{\isacharcolon}\ set{\isacharunderscore}insert{\isacharparenright}\isanewline
\ \ \isacommand{also}\isamarkupfalse%
\ \isacommand{have}\isamarkupfalse%
\ {\isachardoublequoteopen}{\isasymdots}\ {\isacharequal}\ {\isacharbraceleft}F{\isadigit{1}}\ \isactrlbold {\isasymand}\ F{\isadigit{2}}{\isacharbraceright}\ {\isasymunion}\ {\isacharparenleft}setSubformulae\ F{\isadigit{1}}\ {\isasymunion}\ setSubformulae\ F{\isadigit{2}}{\isacharparenright}{\isachardoublequoteclose}\isanewline
\ \ \ \ \isacommand{by}\isamarkupfalse%
\ {\isacharparenleft}simp\ only{\isacharcolon}\ set{\isacharunderscore}append{\isacharparenright}\isanewline
\ \ \isacommand{finally}\isamarkupfalse%
\ \isacommand{show}\isamarkupfalse%
\ {\isacharquery}thesis\isanewline
\ \ \ \ \isacommand{by}\isamarkupfalse%
\ this\isanewline
\isacommand{qed}\isamarkupfalse%
%
\endisatagproof
{\isafoldproof}%
%
\isadelimproof
\isanewline
%
\endisadelimproof
\isanewline
\isacommand{lemma}\isamarkupfalse%
\ setSubformulae{\isacharunderscore}or{\isacharcolon}\ \isanewline
\ \ {\isachardoublequoteopen}setSubformulae\ {\isacharparenleft}F{\isadigit{1}}\ \isactrlbold {\isasymor}\ F{\isadigit{2}}{\isacharparenright}\ \isanewline
\ \ \ {\isacharequal}\ {\isacharbraceleft}F{\isadigit{1}}\ \isactrlbold {\isasymor}\ F{\isadigit{2}}{\isacharbraceright}\ {\isasymunion}\ {\isacharparenleft}setSubformulae\ F{\isadigit{1}}\ {\isasymunion}\ setSubformulae\ F{\isadigit{2}}{\isacharparenright}{\isachardoublequoteclose}\isanewline
%
\isadelimproof
%
\endisadelimproof
%
\isatagproof
\isacommand{proof}\isamarkupfalse%
\ {\isacharminus}\isanewline
\ \ \isacommand{have}\isamarkupfalse%
\ {\isachardoublequoteopen}setSubformulae\ {\isacharparenleft}F{\isadigit{1}}\ \isactrlbold {\isasymor}\ F{\isadigit{2}}{\isacharparenright}\ \isanewline
\ \ \ \ \ \ \ \ {\isacharequal}\ set\ {\isacharparenleft}{\isacharparenleft}F{\isadigit{1}}\ \isactrlbold {\isasymor}\ F{\isadigit{2}}{\isacharparenright}\ {\isacharhash}\ {\isacharparenleft}subformulae\ F{\isadigit{1}}\ {\isacharat}\ subformulae\ F{\isadigit{2}}{\isacharparenright}{\isacharparenright}{\isachardoublequoteclose}\isanewline
\ \ \ \ \isacommand{by}\isamarkupfalse%
\ {\isacharparenleft}simp\ only{\isacharcolon}\ subformulae{\isachardot}simps{\isacharparenleft}{\isadigit{5}}{\isacharparenright}{\isacharparenright}\isanewline
\ \ \isacommand{also}\isamarkupfalse%
\ \isacommand{have}\isamarkupfalse%
\ {\isachardoublequoteopen}{\isasymdots}\ {\isacharequal}\ {\isacharbraceleft}F{\isadigit{1}}\ \isactrlbold {\isasymor}\ F{\isadigit{2}}{\isacharbraceright}\ {\isasymunion}\ {\isacharparenleft}set\ {\isacharparenleft}subformulae\ F{\isadigit{1}}\ {\isacharat}\ subformulae\ F{\isadigit{2}}{\isacharparenright}{\isacharparenright}{\isachardoublequoteclose}\isanewline
\ \ \ \ \isacommand{by}\isamarkupfalse%
\ {\isacharparenleft}simp\ only{\isacharcolon}\ set{\isacharunderscore}insert{\isacharparenright}\isanewline
\ \ \isacommand{also}\isamarkupfalse%
\ \isacommand{have}\isamarkupfalse%
\ {\isachardoublequoteopen}{\isasymdots}\ {\isacharequal}\ {\isacharbraceleft}F{\isadigit{1}}\ \isactrlbold {\isasymor}\ F{\isadigit{2}}{\isacharbraceright}\ {\isasymunion}\ {\isacharparenleft}setSubformulae\ F{\isadigit{1}}\ {\isasymunion}\ setSubformulae\ F{\isadigit{2}}{\isacharparenright}{\isachardoublequoteclose}\isanewline
\ \ \ \ \isacommand{by}\isamarkupfalse%
\ {\isacharparenleft}simp\ only{\isacharcolon}\ set{\isacharunderscore}append{\isacharparenright}\isanewline
\ \ \isacommand{finally}\isamarkupfalse%
\ \isacommand{show}\isamarkupfalse%
\ {\isacharquery}thesis\isanewline
\ \ \ \ \isacommand{by}\isamarkupfalse%
\ this\isanewline
\isacommand{qed}\isamarkupfalse%
%
\endisatagproof
{\isafoldproof}%
%
\isadelimproof
\isanewline
%
\endisadelimproof
\isanewline
\isacommand{lemma}\isamarkupfalse%
\ setSubformulae{\isacharunderscore}imp{\isacharcolon}\ \isanewline
\ \ {\isachardoublequoteopen}setSubformulae\ {\isacharparenleft}F{\isadigit{1}}\ \isactrlbold {\isasymrightarrow}\ F{\isadigit{2}}{\isacharparenright}\ \isanewline
\ \ \ {\isacharequal}\ {\isacharbraceleft}F{\isadigit{1}}\ \isactrlbold {\isasymrightarrow}\ F{\isadigit{2}}{\isacharbraceright}\ {\isasymunion}\ {\isacharparenleft}setSubformulae\ F{\isadigit{1}}\ {\isasymunion}\ setSubformulae\ F{\isadigit{2}}{\isacharparenright}{\isachardoublequoteclose}\isanewline
%
\isadelimproof
%
\endisadelimproof
%
\isatagproof
\isacommand{proof}\isamarkupfalse%
\ {\isacharminus}\isanewline
\ \ \isacommand{have}\isamarkupfalse%
\ {\isachardoublequoteopen}setSubformulae\ {\isacharparenleft}F{\isadigit{1}}\ \isactrlbold {\isasymrightarrow}\ F{\isadigit{2}}{\isacharparenright}\ \isanewline
\ \ \ \ \ \ \ \ {\isacharequal}\ set\ {\isacharparenleft}{\isacharparenleft}F{\isadigit{1}}\ \isactrlbold {\isasymrightarrow}\ F{\isadigit{2}}{\isacharparenright}\ {\isacharhash}\ {\isacharparenleft}subformulae\ F{\isadigit{1}}\ {\isacharat}\ subformulae\ F{\isadigit{2}}{\isacharparenright}{\isacharparenright}{\isachardoublequoteclose}\isanewline
\ \ \ \ \isacommand{by}\isamarkupfalse%
\ {\isacharparenleft}simp\ only{\isacharcolon}\ subformulae{\isachardot}simps{\isacharparenleft}{\isadigit{6}}{\isacharparenright}{\isacharparenright}\isanewline
\ \ \isacommand{also}\isamarkupfalse%
\ \isacommand{have}\isamarkupfalse%
\ {\isachardoublequoteopen}{\isasymdots}\ {\isacharequal}\ {\isacharbraceleft}F{\isadigit{1}}\ \isactrlbold {\isasymrightarrow}\ F{\isadigit{2}}{\isacharbraceright}\ {\isasymunion}\ {\isacharparenleft}set\ {\isacharparenleft}subformulae\ F{\isadigit{1}}\ {\isacharat}\ subformulae\ F{\isadigit{2}}{\isacharparenright}{\isacharparenright}{\isachardoublequoteclose}\isanewline
\ \ \ \ \isacommand{by}\isamarkupfalse%
\ {\isacharparenleft}simp\ only{\isacharcolon}\ set{\isacharunderscore}insert{\isacharparenright}\isanewline
\ \ \isacommand{also}\isamarkupfalse%
\ \isacommand{have}\isamarkupfalse%
\ {\isachardoublequoteopen}{\isasymdots}\ {\isacharequal}\ {\isacharbraceleft}F{\isadigit{1}}\ \isactrlbold {\isasymrightarrow}\ F{\isadigit{2}}{\isacharbraceright}\ {\isasymunion}\ {\isacharparenleft}setSubformulae\ F{\isadigit{1}}\ {\isasymunion}\ setSubformulae\ F{\isadigit{2}}{\isacharparenright}{\isachardoublequoteclose}\isanewline
\ \ \ \ \isacommand{by}\isamarkupfalse%
\ {\isacharparenleft}simp\ only{\isacharcolon}\ set{\isacharunderscore}append{\isacharparenright}\isanewline
\ \ \isacommand{finally}\isamarkupfalse%
\ \isacommand{show}\isamarkupfalse%
\ {\isacharquery}thesis\isanewline
\ \ \ \ \isacommand{by}\isamarkupfalse%
\ this\isanewline
\isacommand{qed}\isamarkupfalse%
%
\endisatagproof
{\isafoldproof}%
%
\isadelimproof
%
\endisadelimproof
%
\begin{isamarkuptext}%
Una vez probada la equivalencia, comencemos con los resultados 
  correspondientes a las subfórmulas. En primer lugar, tenemos la 
  siguiente propiedad como consecuencia directa de la equivalencia de 
  funciones anterior.

  \begin{lema}
    \isa{F\ {\isasymin}\ Subf{\isacharparenleft}F{\isacharparenright}}.
  \end{lema}

  \begin{demostracion}
    Procedamos por inducción sobre la estructura de fórmula probando los 
    correspondientes tipos.
  
    Sea \isa{p} fórmula atómica para \isa{p} variable proposicional cualquiera. 
    Por definición de \isa{Subf} tenemos que \isa{Subf{\isacharparenleft}Atom\ p{\isacharparenright}\ {\isacharequal}\ {\isacharbraceleft}Atom\ p{\isacharbraceright}}, 
    luego se tiene la propiedad.
  
    Sea la fórmula \isa{{\isasymbottom}}. Como \isa{Subf{\isacharparenleft}{\isasymbottom}{\isacharparenright}\ {\isacharequal}\ {\isacharbraceleft}{\isasymbottom}{\isacharbraceright}}, se verifica el resultado.

    Por definición del conjunto de subfórmulas de \isa{Subf{\isacharparenleft}{\isasymnot}\ F{\isacharparenright}} se tiene 
    la propiedad para este caso, pues 
    \isa{Subf{\isacharparenleft}{\isasymnot}\ F{\isacharparenright}\ {\isacharequal}\ {\isacharbraceleft}{\isasymnot}\ F{\isacharbraceright}\ {\isasymunion}\ Subf{\isacharparenleft}F{\isacharparenright}\ {\isasymLongrightarrow}\ {\isasymnot}\ F\ {\isasymin}\ Subf{\isacharparenleft}{\isasymnot}\ F{\isacharparenright}} como queríamos 
    ver.

    Análogamente, para cualquier conectiva binaria \isa{{\isacharasterisk}} y fórmulas \isa{F} y 
    \isa{G} se cumple \isa{Subf{\isacharparenleft}F{\isacharasterisk}G{\isacharparenright}\ {\isacharequal}\ {\isacharbraceleft}F{\isacharasterisk}G{\isacharbraceright}\ {\isasymunion}\ Subf{\isacharparenleft}F{\isacharparenright}\ {\isasymunion}\ Subf{\isacharparenleft}G{\isacharparenright}}, luego se 
    verifica análogamente.
  \end{demostracion}

  Formalicemos ahora el lema con su correspondiente demostración 
  detallada.%
\end{isamarkuptext}\isamarkuptrue%
\ \isanewline
\isacommand{lemma}\isamarkupfalse%
\ subformulae{\isacharunderscore}self{\isacharcolon}\ {\isachardoublequoteopen}F\ {\isasymin}\ setSubformulae\ F{\isachardoublequoteclose}\isanewline
%
\isadelimproof
%
\endisadelimproof
%
\isatagproof
\isacommand{proof}\isamarkupfalse%
\ {\isacharparenleft}induction\ F{\isacharparenright}\ \isanewline
\ \ \isacommand{case}\isamarkupfalse%
\ {\isacharparenleft}Atom\ x{\isacharparenright}\ \isanewline
\ \ \isacommand{then}\isamarkupfalse%
\ \isacommand{show}\isamarkupfalse%
\ {\isacharquery}case\ \isanewline
\ \ \ \ \isacommand{by}\isamarkupfalse%
\ {\isacharparenleft}simp\ only{\isacharcolon}\ singletonI\ setSubformulae{\isacharunderscore}atom{\isacharparenright}\isanewline
\isacommand{next}\isamarkupfalse%
\isanewline
\ \ \isacommand{case}\isamarkupfalse%
\ Bot\isanewline
\ \ \isacommand{then}\isamarkupfalse%
\ \isacommand{show}\isamarkupfalse%
\ {\isacharquery}case\ \isanewline
\ \ \ \ \isacommand{by}\isamarkupfalse%
\ {\isacharparenleft}simp\ only{\isacharcolon}\ singletonI\ setSubformulae{\isacharunderscore}bot{\isacharparenright}\isanewline
\isacommand{next}\isamarkupfalse%
\isanewline
\ \ \isacommand{case}\isamarkupfalse%
\ {\isacharparenleft}Not\ F{\isacharparenright}\isanewline
\ \ \isacommand{then}\isamarkupfalse%
\ \isacommand{show}\isamarkupfalse%
\ {\isacharquery}case\ \isanewline
\ \ \ \ \isacommand{by}\isamarkupfalse%
\ {\isacharparenleft}simp\ add{\isacharcolon}\ insertI{\isadigit{1}}\ setSubformulae{\isacharunderscore}not{\isacharparenright}\isanewline
\isacommand{next}\isamarkupfalse%
\isanewline
\isacommand{case}\isamarkupfalse%
\ {\isacharparenleft}And\ F{\isadigit{1}}\ F{\isadigit{2}}{\isacharparenright}\isanewline
\ \ \isacommand{then}\isamarkupfalse%
\ \isacommand{show}\isamarkupfalse%
\ {\isacharquery}case\ \isanewline
\ \ \ \ \isacommand{by}\isamarkupfalse%
\ {\isacharparenleft}simp\ add{\isacharcolon}\ insertI{\isadigit{1}}\ setSubformulae{\isacharunderscore}and{\isacharparenright}\isanewline
\isacommand{next}\isamarkupfalse%
\isanewline
\isacommand{case}\isamarkupfalse%
\ {\isacharparenleft}Or\ F{\isadigit{1}}\ F{\isadigit{2}}{\isacharparenright}\isanewline
\ \ \isacommand{then}\isamarkupfalse%
\ \isacommand{show}\isamarkupfalse%
\ {\isacharquery}case\ \isanewline
\ \ \ \ \isacommand{by}\isamarkupfalse%
\ {\isacharparenleft}simp\ add{\isacharcolon}\ insertI{\isadigit{1}}\ setSubformulae{\isacharunderscore}or{\isacharparenright}\isanewline
\isacommand{next}\isamarkupfalse%
\isanewline
\isacommand{case}\isamarkupfalse%
\ {\isacharparenleft}Imp\ F{\isadigit{1}}\ F{\isadigit{2}}{\isacharparenright}\isanewline
\ \ \isacommand{then}\isamarkupfalse%
\ \isacommand{show}\isamarkupfalse%
\ {\isacharquery}case\ \isanewline
\ \ \ \ \isacommand{by}\isamarkupfalse%
\ {\isacharparenleft}simp\ add{\isacharcolon}\ insertI{\isadigit{1}}\ setSubformulae{\isacharunderscore}imp{\isacharparenright}\isanewline
\isacommand{qed}\isamarkupfalse%
%
\endisatagproof
{\isafoldproof}%
%
\isadelimproof
%
\endisadelimproof
%
\begin{isamarkuptext}%
La demostración automática es la siguiente.%
\end{isamarkuptext}\isamarkuptrue%
\isacommand{lemma}\isamarkupfalse%
\ {\isachardoublequoteopen}F\ {\isasymin}\ setSubformulae\ F{\isachardoublequoteclose}\isanewline
%
\isadelimproof
\ \ %
\endisadelimproof
%
\isatagproof
\isacommand{by}\isamarkupfalse%
\ {\isacharparenleft}induction\ F{\isacharparenright}\ simp{\isacharunderscore}all%
\endisatagproof
{\isafoldproof}%
%
\isadelimproof
%
\endisadelimproof
%
\begin{isamarkuptext}%
Procedamos con los demás resultados de la sección. Como hemos 
  señalado con anterioridad, utilizaremos varias propiedades de 
  conjuntos pertenecientes a la teoría 
  \href{https://n9.cl/qatp}{Set.thy} de Isabelle, que apareceran en 
  el glosario final. 

  Además, definiremos dos reglas adicionales que utilizaremos con 
  frecuencia.%
\end{isamarkuptext}\isamarkuptrue%
\ \isanewline
\isacommand{lemma}\isamarkupfalse%
\ subContUnionRev{\isadigit{1}}{\isacharcolon}\ \isanewline
\ \ \isakeyword{assumes}\ {\isachardoublequoteopen}A\ {\isasymunion}\ B\ {\isasymsubseteq}\ C{\isachardoublequoteclose}\ \isanewline
\ \ \isakeyword{shows}\ \ \ {\isachardoublequoteopen}A\ {\isasymsubseteq}\ C{\isachardoublequoteclose}\isanewline
%
\isadelimproof
%
\endisadelimproof
%
\isatagproof
\isacommand{proof}\isamarkupfalse%
\ {\isacharminus}\isanewline
\ \ \isacommand{have}\isamarkupfalse%
\ {\isachardoublequoteopen}A\ {\isasymsubseteq}\ C\ {\isasymand}\ B\ {\isasymsubseteq}\ C{\isachardoublequoteclose}\isanewline
\ \ \ \ \isacommand{using}\isamarkupfalse%
\ assms\isanewline
\ \ \ \ \isacommand{by}\isamarkupfalse%
\ {\isacharparenleft}simp\ only{\isacharcolon}\ sup{\isachardot}bounded{\isacharunderscore}iff{\isacharparenright}\isanewline
\ \ \isacommand{then}\isamarkupfalse%
\ \isacommand{show}\isamarkupfalse%
\ {\isachardoublequoteopen}A\ {\isasymsubseteq}\ C{\isachardoublequoteclose}\isanewline
\ \ \ \ \isacommand{by}\isamarkupfalse%
\ {\isacharparenleft}rule\ conjunct{\isadigit{1}}{\isacharparenright}\isanewline
\isacommand{qed}\isamarkupfalse%
%
\endisatagproof
{\isafoldproof}%
%
\isadelimproof
\isanewline
%
\endisadelimproof
\isanewline
\isacommand{lemma}\isamarkupfalse%
\ subContUnionRev{\isadigit{2}}{\isacharcolon}\ \isanewline
\ \ \isakeyword{assumes}\ {\isachardoublequoteopen}A\ {\isasymunion}\ B\ {\isasymsubseteq}\ C{\isachardoublequoteclose}\ \isanewline
\ \ \isakeyword{shows}\ \ \ {\isachardoublequoteopen}B\ {\isasymsubseteq}\ C{\isachardoublequoteclose}\isanewline
%
\isadelimproof
%
\endisadelimproof
%
\isatagproof
\isacommand{proof}\isamarkupfalse%
\ {\isacharminus}\isanewline
\ \ \isacommand{have}\isamarkupfalse%
\ {\isachardoublequoteopen}A\ {\isasymsubseteq}\ C\ {\isasymand}\ B\ {\isasymsubseteq}\ C{\isachardoublequoteclose}\isanewline
\ \ \ \ \isacommand{using}\isamarkupfalse%
\ assms\isanewline
\ \ \ \ \isacommand{by}\isamarkupfalse%
\ {\isacharparenleft}simp\ only{\isacharcolon}\ sup{\isachardot}bounded{\isacharunderscore}iff{\isacharparenright}\isanewline
\ \ \isacommand{then}\isamarkupfalse%
\ \isacommand{show}\isamarkupfalse%
\ {\isachardoublequoteopen}B\ {\isasymsubseteq}\ C{\isachardoublequoteclose}\isanewline
\ \ \ \ \isacommand{by}\isamarkupfalse%
\ {\isacharparenleft}rule\ conjunct{\isadigit{2}}{\isacharparenright}\isanewline
\isacommand{qed}\isamarkupfalse%
%
\endisatagproof
{\isafoldproof}%
%
\isadelimproof
%
\endisadelimproof
%
\begin{isamarkuptext}%
Sus correspondientes demostraciones automáticas se muestran a 
  continuación.%
\end{isamarkuptext}\isamarkuptrue%
\isacommand{lemma}\isamarkupfalse%
\ {\isachardoublequoteopen}A\ {\isasymunion}\ B\ {\isasymsubseteq}\ C\ {\isasymLongrightarrow}\ A\ {\isasymsubseteq}\ C{\isachardoublequoteclose}\isanewline
%
\isadelimproof
\ \ %
\endisadelimproof
%
\isatagproof
\isacommand{by}\isamarkupfalse%
\ simp%
\endisatagproof
{\isafoldproof}%
%
\isadelimproof
\isanewline
%
\endisadelimproof
\isanewline
\isacommand{lemma}\isamarkupfalse%
\ {\isachardoublequoteopen}A\ {\isasymunion}\ B\ {\isasymsubseteq}\ C\ {\isasymLongrightarrow}\ B\ {\isasymsubseteq}\ C{\isachardoublequoteclose}\isanewline
%
\isadelimproof
\ \ %
\endisadelimproof
%
\isatagproof
\isacommand{by}\isamarkupfalse%
\ simp%
\endisatagproof
{\isafoldproof}%
%
\isadelimproof
%
\endisadelimproof
%
\begin{isamarkuptext}%
Veamos ahora los distintos resultados sobre subfórmulas.

  \begin{lema}
    Sea \isa{F} una fórmula proposicional y \isa{conjAtoms{\isacharparenleft}F{\isacharparenright}} el conjunto de 
    sus variables proposicionales. Sea \isa{A\isactrlsub F} el conjunto de las fórmulas 
    atómicas formadas a partir de cada elemento de \isa{conjAtoms{\isacharparenleft}F{\isacharparenright}}. 
    Entonces, \isa{A\isactrlsub F\ {\isasymsubseteq}\ Subf{\isacharparenleft}F{\isacharparenright}}.

    Por tanto, las fórmulas atómicas son subfórmulas.
  \end{lema}

  \begin{demostracion}
    La prueba seguirá el esquema inductivo para la estructura de 
    fórmulas. Veamos cada caso:
  
    Consideremos la fórmula atómica \isa{Atom\ p} para \isa{p} una variable 
    cualquiera. Entonces, \isa{conjAtoms{\isacharparenleft}Atom\ p{\isacharparenright}\ {\isacharequal}\ {\isacharbraceleft}p{\isacharbraceright}}. De este modo, el 
    conjunto \isa{A\isactrlsub A\isactrlsub t\isactrlsub o\isactrlsub m\ \isactrlsub p} correspondiente será 
    \isa{A\isactrlsub A\isactrlsub t\isactrlsub o\isactrlsub m\ \isactrlsub p\ {\isacharequal}\ {\isacharbraceleft}Atom\ p{\isacharbraceright}\ {\isasymsubseteq}\ {\isacharbraceleft}Atom\ p{\isacharbraceright}\ {\isacharequal}\ Subf{\isacharparenleft}Atom\ p{\isacharparenright}} como queríamos 
    demostrar.

    Sea la fórmula \isa{{\isasymbottom}}. Como \isa{conjAtoms{\isacharparenleft}{\isasymbottom}{\isacharparenright}\ {\isacharequal}\ {\isasymemptyset}}, es claro que 
    \isa{A\isactrlsub {\isasymbottom}\ {\isacharequal}\ {\isasymemptyset}\ {\isasymsubseteq}\ Subf{\isacharparenleft}{\isasymbottom}{\isacharparenright}\ {\isacharequal}\ {\isasymemptyset}}.

    Sea la fórmula \isa{F} tal que \isa{A\isactrlsub F\ {\isasymsubseteq}\ Subf{\isacharparenleft}F{\isacharparenright}}. Probemos el resultado 
    para \isa{{\isasymnot}\ F}. Por definición tenemos que 
    \isa{conjAtoms{\isacharparenleft}{\isasymnot}\ F{\isacharparenright}\ {\isacharequal}\ conjAtoms{\isacharparenleft}F{\isacharparenright}}, luego \isa{A\isactrlsub {\isasymnot}\isactrlsub F\ {\isacharequal}\ A\isactrlsub F}. Además, 
    \isa{Subf{\isacharparenleft}{\isasymnot}\ F{\isacharparenright}\ {\isacharequal}\ {\isacharbraceleft}{\isasymnot}\ F{\isacharbraceright}\ {\isasymunion}\ Subf{\isacharparenleft}F{\isacharparenright}}. Por tanto, por hipótesis de 
    inducción tenemos:
    \isa{A\isactrlsub {\isasymnot}\isactrlsub F\ {\isacharequal}\ A\isactrlsub F\ {\isasymsubseteq}\ Subf{\isacharparenleft}F{\isacharparenright}\ {\isasymsubseteq}\ {\isacharbraceleft}{\isasymnot}\ F{\isacharbraceright}\ {\isasymunion}\ Subf{\isacharparenleft}F{\isacharparenright}\ {\isacharequal}\ Subf{\isacharparenleft}{\isasymnot}\ F{\isacharparenright}}, luego
    \isa{A\isactrlsub {\isasymnot}\isactrlsub F\ {\isasymsubseteq}\ Subf{\isacharparenleft}{\isasymnot}\ F{\isacharparenright}}.

    Sean las fórmulas \isa{F} y \isa{G} tales que \isa{A\isactrlsub F\ {\isasymsubseteq}\ Subf{\isacharparenleft}F{\isacharparenright}} y 
    \isa{A\isactrlsub G\ {\isasymsubseteq}\ Subf{\isacharparenleft}G{\isacharparenright}}. Probemos ahora \isa{A\isactrlsub F\isactrlsub {\isacharasterisk}\isactrlsub G\ {\isasymsubseteq}\ Subf{\isacharparenleft}F{\isacharasterisk}G{\isacharparenright}} para cualquier 
    conectiva binaria \isa{{\isacharasterisk}}. Por un lado, 
    \isa{conjAtoms{\isacharparenleft}F{\isacharasterisk}G{\isacharparenright}\ {\isacharequal}\ conjAtoms{\isacharparenleft}F{\isacharparenright}\ {\isasymunion}\ conjAtoms{\isacharparenleft}G{\isacharparenright}}, luego 
    \isa{A\isactrlsub F\isactrlsub {\isacharasterisk}\isactrlsub G\ {\isacharequal}\ A\isactrlsub F\ {\isasymunion}\ A\isactrlsub G}. Por tanto, por hipótesis de inducción y definición 
    del conjunto de subfórmulas, se tiene:

    \isa{A\isactrlsub F\isactrlsub {\isacharasterisk}\isactrlsub G\ {\isacharequal}\ A\isactrlsub F\ {\isasymunion}\ A\isactrlsub G\ {\isasymsubseteq}\ conjAtoms{\isacharparenleft}F{\isacharparenright}\ {\isasymunion}\ conjAtoms{\isacharparenleft}G{\isacharparenright}\ {\isasymsubseteq}\ {\isacharbraceleft}F{\isacharasterisk}G{\isacharbraceright}\ {\isasymunion}\ conjAtoms{\isacharparenleft}F{\isacharparenright}\ {\isasymunion}\ conjAtoms{\isacharparenleft}G{\isacharparenright}\ {\isacharequal}\ conjAtoms{\isacharparenleft}F{\isacharasterisk}G{\isacharparenright}}
    Luego, \isa{A\isactrlsub F\isactrlsub {\isacharasterisk}\isactrlsub G\ {\isasymsubseteq}\ conjAtoms{\isacharparenleft}F{\isacharasterisk}G{\isacharparenright}} como queríamos demostrar.  
  \end{demostracion}

  En Isabelle, se especifica como sigue.%
\end{isamarkuptext}\isamarkuptrue%
\isacommand{lemma}\isamarkupfalse%
\ atoms{\isacharunderscore}are{\isacharunderscore}subformulae{\isacharcolon}\ {\isachardoublequoteopen}Atom\ {\isacharbackquote}\ atoms\ F\ {\isasymsubseteq}\ setSubformulae\ F{\isachardoublequoteclose}\isanewline
%
\isadelimproof
\ \ %
\endisadelimproof
%
\isatagproof
\isacommand{oops}\isamarkupfalse%
%
\endisatagproof
{\isafoldproof}%
%
\isadelimproof
%
\endisadelimproof
%
\begin{isamarkuptext}%
Debemos observar que \isa{Atom\ {\isacharbackquote}\ atoms\ F} construye las fórmulas 
  atómicas a partir de cada uno de los elementos de \isa{atoms\ F}, creando 
  un conjunto de fórmulas atómicas. Dicho conjunto es equivalente al 
  conjunto \isa{A\isactrlsub F} del enunciado del lema. Para ello emplea el infijo \isa{{\isacharbackquote}} 
  definido como notación abreviada de \isa{{\isacharparenleft}{\isacharbackquote}{\isacharparenright}} que calcula la 
  imagen de un conjunto en la teoría \href{https://n9.cl/qatp}{Set.thy}.

  \begin{itemize}
    \item[] \isa{f\ {\isacharbackquote}\ A\ {\isacharequal}\ {\isacharbraceleft}y\ {\isacharbar}\ {\isasymexists}x{\isasymin}A{\isachardot}\ y\ {\isacharequal}\ f\ x{\isacharbraceright}} \hfill (\isa{image{\isacharunderscore}def})
  \end{itemize}

  Para aclarar su funcionamiento, veamos ejemplos para distintos casos 
  de fórmulas.%
\end{isamarkuptext}\isamarkuptrue%
\isacommand{notepad}\isamarkupfalse%
\isanewline
\isakeyword{begin}\isanewline
%
\isadelimproof
\ \ %
\endisadelimproof
%
\isatagproof
\isacommand{fix}\isamarkupfalse%
\ p\ q\ r\ {\isacharcolon}{\isacharcolon}\ {\isacharprime}a\isanewline
\isanewline
\ \ \isacommand{have}\isamarkupfalse%
\ {\isachardoublequoteopen}Atom\ {\isacharbackquote}atoms\ {\isacharparenleft}Atom\ p\ \isactrlbold {\isasymor}\ {\isasymbottom}{\isacharparenright}\ {\isacharequal}\ {\isacharbraceleft}Atom\ p{\isacharbraceright}{\isachardoublequoteclose}\isanewline
\ \ \ \ \isacommand{by}\isamarkupfalse%
\ simp\isanewline
\isanewline
\ \ \isacommand{have}\isamarkupfalse%
\ {\isachardoublequoteopen}Atom\ {\isacharbackquote}atoms\ {\isacharparenleft}{\isacharparenleft}Atom\ p\ \isactrlbold {\isasymrightarrow}\ Atom\ q{\isacharparenright}\ \isactrlbold {\isasymor}\ Atom\ r{\isacharparenright}\ {\isacharequal}\ \isanewline
\ \ \ \ \ \ \ {\isacharbraceleft}Atom\ p{\isacharcomma}\ Atom\ q{\isacharcomma}\ Atom\ r{\isacharbraceright}{\isachardoublequoteclose}\isanewline
\ \ \ \ \isacommand{by}\isamarkupfalse%
\ auto\ \isanewline
\isanewline
\ \ \isacommand{have}\isamarkupfalse%
\ {\isachardoublequoteopen}Atom\ {\isacharbackquote}atoms\ {\isacharparenleft}{\isacharparenleft}Atom\ p\ \isactrlbold {\isasymrightarrow}\ Atom\ p{\isacharparenright}\ \isactrlbold {\isasymor}\ Atom\ r{\isacharparenright}\ {\isacharequal}\ {\isacharbraceleft}Atom\ p{\isacharcomma}\ Atom\ r{\isacharbraceright}{\isachardoublequoteclose}\isanewline
\ \ \ \ \isacommand{by}\isamarkupfalse%
\ auto%
\endisatagproof
{\isafoldproof}%
%
\isadelimproof
\isanewline
%
\endisadelimproof
\isacommand{end}\isamarkupfalse%
%
\begin{isamarkuptext}%
Además, esta función tiene las siguientes propiedades sobre 
  conjuntos que utilizaremos en la demostración.

  \begin{itemize}
    \item[] \isa{f\ {\isacharbackquote}\ {\isacharparenleft}A\ {\isasymunion}\ B{\isacharparenright}\ {\isacharequal}\ f\ {\isacharbackquote}\ A\ {\isasymunion}\ f\ {\isacharbackquote}\ B} 
      \hfill (\isa{image{\isacharunderscore}Un})
    \item[] \isa{f\ {\isacharbackquote}\ {\isacharparenleft}{\isacharbraceleft}a{\isacharbraceright}\ {\isasymunion}\ B{\isacharparenright}\ {\isacharequal}\ {\isacharbraceleft}f\ a{\isacharbraceright}\ {\isasymunion}\ f\ {\isacharbackquote}\ B} 
      \hfill (\isa{image{\isacharunderscore}insert})
    \item[] \isa{f\ {\isacharbackquote}\ {\isasymemptyset}\ {\isacharequal}\ {\isasymemptyset}} 
      \hfill (\isa{image{\isacharunderscore}empty})
  \end{itemize}

  Una vez hechas las aclaraciones necesarias, comencemos la demostración 
  estructurada. Esta seguirá el esquema inductivo señalado con 
  anterioridad. Debido a la extensión de la prueba demostraremos de 
  manera detallada únicamente el caso de conectiva binaria de la 
  conjunción. El resto son totalmente equivalentes y los dejaré 
  indicados de manera automática. Observemos que los casos básicos de 
  \isa{Atom\ x} y \isa{{\isasymbottom}} podrían demostrarse de manera directa 
  únicamente mediante simplificación.%
\end{isamarkuptext}\isamarkuptrue%
\isacommand{lemma}\isamarkupfalse%
\ atoms{\isacharunderscore}are{\isacharunderscore}subformulae{\isacharunderscore}atom{\isacharcolon}\ \isanewline
\ \ {\isachardoublequoteopen}Atom\ {\isacharbackquote}\ atoms\ {\isacharparenleft}Atom\ x{\isacharparenright}\ {\isasymsubseteq}\ setSubformulae\ {\isacharparenleft}Atom\ x{\isacharparenright}{\isachardoublequoteclose}\ \isanewline
%
\isadelimproof
%
\endisadelimproof
%
\isatagproof
\isacommand{proof}\isamarkupfalse%
\ {\isacharminus}\isanewline
\ \ \isacommand{have}\isamarkupfalse%
\ {\isachardoublequoteopen}Atom\ {\isacharbackquote}\ atoms\ {\isacharparenleft}Atom\ x{\isacharparenright}\ {\isacharequal}\ Atom\ {\isacharbackquote}\ {\isacharbraceleft}x{\isacharbraceright}{\isachardoublequoteclose}\isanewline
\ \ \ \ \isacommand{by}\isamarkupfalse%
\ {\isacharparenleft}simp\ only{\isacharcolon}\ formula{\isachardot}set{\isacharparenleft}{\isadigit{1}}{\isacharparenright}{\isacharparenright}\isanewline
\ \ \isacommand{also}\isamarkupfalse%
\ \isacommand{have}\isamarkupfalse%
\ {\isachardoublequoteopen}{\isasymdots}\ {\isacharequal}\ {\isacharbraceleft}Atom\ x{\isacharbraceright}{\isachardoublequoteclose}\isanewline
\ \ \ \ \isacommand{by}\isamarkupfalse%
\ {\isacharparenleft}simp\ only{\isacharcolon}\ image{\isacharunderscore}insert\ image{\isacharunderscore}empty{\isacharparenright}\isanewline
\ \ \isacommand{also}\isamarkupfalse%
\ \isacommand{have}\isamarkupfalse%
\ {\isachardoublequoteopen}{\isasymdots}\ {\isacharequal}\ set\ {\isacharbrackleft}Atom\ x{\isacharbrackright}{\isachardoublequoteclose}\isanewline
\ \ \ \ \isacommand{by}\isamarkupfalse%
\ {\isacharparenleft}simp\ only{\isacharcolon}\ list{\isachardot}set{\isacharparenleft}{\isadigit{1}}{\isacharparenright}\ list{\isachardot}set{\isacharparenleft}{\isadigit{2}}{\isacharparenright}{\isacharparenright}\isanewline
\ \ \isacommand{also}\isamarkupfalse%
\ \isacommand{have}\isamarkupfalse%
\ {\isachardoublequoteopen}{\isasymdots}\ {\isacharequal}\ set\ {\isacharparenleft}subformulae\ {\isacharparenleft}Atom\ x{\isacharparenright}{\isacharparenright}{\isachardoublequoteclose}\isanewline
\ \ \ \ \isacommand{by}\isamarkupfalse%
\ {\isacharparenleft}simp\ only{\isacharcolon}\ subformulae{\isachardot}simps{\isacharparenleft}{\isadigit{1}}{\isacharparenright}{\isacharparenright}\isanewline
\ \ \isacommand{finally}\isamarkupfalse%
\ \isacommand{have}\isamarkupfalse%
\ {\isachardoublequoteopen}Atom\ {\isacharbackquote}\ atoms\ {\isacharparenleft}Atom\ x{\isacharparenright}\ {\isacharequal}\ set\ {\isacharparenleft}subformulae\ {\isacharparenleft}Atom\ x{\isacharparenright}{\isacharparenright}{\isachardoublequoteclose}\isanewline
\ \ \ \ \isacommand{by}\isamarkupfalse%
\ this\isanewline
\ \ \isacommand{then}\isamarkupfalse%
\ \isacommand{show}\isamarkupfalse%
\ {\isacharquery}thesis\ \isanewline
\ \ \ \ \isacommand{by}\isamarkupfalse%
\ {\isacharparenleft}simp\ only{\isacharcolon}\ subset{\isacharunderscore}refl{\isacharparenright}\isanewline
\isacommand{qed}\isamarkupfalse%
%
\endisatagproof
{\isafoldproof}%
%
\isadelimproof
\isanewline
%
\endisadelimproof
\isanewline
\isacommand{lemma}\isamarkupfalse%
\ atoms{\isacharunderscore}are{\isacharunderscore}subformulae{\isacharunderscore}bot{\isacharcolon}\ \isanewline
\ \ {\isachardoublequoteopen}Atom\ {\isacharbackquote}\ atoms\ {\isasymbottom}\ {\isasymsubseteq}\ setSubformulae\ {\isasymbottom}{\isachardoublequoteclose}\ \ \isanewline
%
\isadelimproof
%
\endisadelimproof
%
\isatagproof
\isacommand{proof}\isamarkupfalse%
\ {\isacharminus}\isanewline
\ \ \isacommand{have}\isamarkupfalse%
\ {\isachardoublequoteopen}Atom\ {\isacharbackquote}\ atoms\ {\isasymbottom}\ {\isacharequal}\ Atom\ {\isacharbackquote}\ {\isasymemptyset}{\isachardoublequoteclose}\isanewline
\ \ \ \ \isacommand{by}\isamarkupfalse%
\ {\isacharparenleft}simp\ only{\isacharcolon}\ formula{\isachardot}set{\isacharparenleft}{\isadigit{2}}{\isacharparenright}{\isacharparenright}\isanewline
\ \ \isacommand{also}\isamarkupfalse%
\ \isacommand{have}\isamarkupfalse%
\ {\isachardoublequoteopen}{\isasymdots}\ {\isacharequal}\ {\isasymemptyset}{\isachardoublequoteclose}\isanewline
\ \ \ \ \isacommand{by}\isamarkupfalse%
\ {\isacharparenleft}simp\ only{\isacharcolon}\ image{\isacharunderscore}empty{\isacharparenright}\isanewline
\ \ \isacommand{also}\isamarkupfalse%
\ \isacommand{have}\isamarkupfalse%
\ {\isachardoublequoteopen}{\isasymdots}\ {\isasymsubseteq}\ setSubformulae\ {\isasymbottom}{\isachardoublequoteclose}\isanewline
\ \ \ \ \isacommand{by}\isamarkupfalse%
\ {\isacharparenleft}simp\ only{\isacharcolon}\ empty{\isacharunderscore}subsetI{\isacharparenright}\isanewline
\ \ \isacommand{finally}\isamarkupfalse%
\ \isacommand{show}\isamarkupfalse%
\ {\isacharquery}thesis\isanewline
\ \ \ \ \isacommand{by}\isamarkupfalse%
\ this\isanewline
\isacommand{qed}\isamarkupfalse%
%
\endisatagproof
{\isafoldproof}%
%
\isadelimproof
\isanewline
%
\endisadelimproof
\isanewline
\isacommand{lemma}\isamarkupfalse%
\ atoms{\isacharunderscore}are{\isacharunderscore}subformulae{\isacharunderscore}not{\isacharcolon}\ \isanewline
\ \ \isakeyword{assumes}\ {\isachardoublequoteopen}Atom\ {\isacharbackquote}\ atoms\ F\ {\isasymsubseteq}\ setSubformulae\ F{\isachardoublequoteclose}\ \isanewline
\ \ \isakeyword{shows}\ \ \ {\isachardoublequoteopen}Atom\ {\isacharbackquote}\ atoms\ {\isacharparenleft}\isactrlbold {\isasymnot}\ F{\isacharparenright}\ {\isasymsubseteq}\ setSubformulae\ {\isacharparenleft}\isactrlbold {\isasymnot}\ F{\isacharparenright}{\isachardoublequoteclose}\isanewline
%
\isadelimproof
%
\endisadelimproof
%
\isatagproof
\isacommand{proof}\isamarkupfalse%
\ {\isacharminus}\isanewline
\ \ \isacommand{have}\isamarkupfalse%
\ {\isachardoublequoteopen}Atom\ {\isacharbackquote}\ atoms\ {\isacharparenleft}\isactrlbold {\isasymnot}\ F{\isacharparenright}\ {\isacharequal}\ Atom\ {\isacharbackquote}\ atoms\ F{\isachardoublequoteclose}\isanewline
\ \ \ \ \isacommand{by}\isamarkupfalse%
\ {\isacharparenleft}simp\ only{\isacharcolon}\ formula{\isachardot}set{\isacharparenleft}{\isadigit{3}}{\isacharparenright}{\isacharparenright}\isanewline
\ \ \isacommand{also}\isamarkupfalse%
\ \isacommand{have}\isamarkupfalse%
\ {\isachardoublequoteopen}{\isasymdots}\ {\isasymsubseteq}\ setSubformulae\ F{\isachardoublequoteclose}\isanewline
\ \ \ \ \isacommand{by}\isamarkupfalse%
\ {\isacharparenleft}simp\ only{\isacharcolon}\ assms{\isacharparenright}\isanewline
\ \ \isacommand{also}\isamarkupfalse%
\ \isacommand{have}\isamarkupfalse%
\ {\isachardoublequoteopen}{\isasymdots}\ {\isasymsubseteq}\ {\isacharbraceleft}\isactrlbold {\isasymnot}\ F{\isacharbraceright}\ {\isasymunion}\ setSubformulae\ F{\isachardoublequoteclose}\isanewline
\ \ \ \ \isacommand{by}\isamarkupfalse%
\ {\isacharparenleft}simp\ only{\isacharcolon}\ Un{\isacharunderscore}upper{\isadigit{2}}{\isacharparenright}\isanewline
\ \ \isacommand{also}\isamarkupfalse%
\ \isacommand{have}\isamarkupfalse%
\ {\isachardoublequoteopen}{\isasymdots}\ {\isacharequal}\ setSubformulae\ {\isacharparenleft}\isactrlbold {\isasymnot}\ F{\isacharparenright}{\isachardoublequoteclose}\isanewline
\ \ \ \ \isacommand{by}\isamarkupfalse%
\ {\isacharparenleft}simp\ only{\isacharcolon}\ setSubformulae{\isacharunderscore}not{\isacharparenright}\isanewline
\ \ \isacommand{finally}\isamarkupfalse%
\ \isacommand{show}\isamarkupfalse%
\ {\isacharquery}thesis\isanewline
\ \ \ \ \isacommand{by}\isamarkupfalse%
\ this\isanewline
\isacommand{qed}\isamarkupfalse%
%
\endisatagproof
{\isafoldproof}%
%
\isadelimproof
\isanewline
%
\endisadelimproof
\isanewline
\isacommand{lemma}\isamarkupfalse%
\ atoms{\isacharunderscore}are{\isacharunderscore}subformulae{\isacharunderscore}and{\isacharcolon}\ \isanewline
\ \ \isakeyword{assumes}\ {\isachardoublequoteopen}Atom\ {\isacharbackquote}\ atoms\ F{\isadigit{1}}\ {\isasymsubseteq}\ setSubformulae\ F{\isadigit{1}}{\isachardoublequoteclose}\isanewline
\ \ \ \ \ \ \ \ \ \ {\isachardoublequoteopen}Atom\ {\isacharbackquote}\ atoms\ F{\isadigit{2}}\ {\isasymsubseteq}\ setSubformulae\ F{\isadigit{2}}{\isachardoublequoteclose}\isanewline
\ \ \isakeyword{shows}\ \ \ {\isachardoublequoteopen}Atom\ {\isacharbackquote}\ atoms\ {\isacharparenleft}F{\isadigit{1}}\ \isactrlbold {\isasymand}\ F{\isadigit{2}}{\isacharparenright}\ {\isasymsubseteq}\ setSubformulae\ {\isacharparenleft}F{\isadigit{1}}\ \isactrlbold {\isasymand}\ F{\isadigit{2}}{\isacharparenright}{\isachardoublequoteclose}\isanewline
%
\isadelimproof
%
\endisadelimproof
%
\isatagproof
\isacommand{proof}\isamarkupfalse%
\ {\isacharminus}\isanewline
\ \ \isacommand{have}\isamarkupfalse%
\ {\isachardoublequoteopen}Atom\ {\isacharbackquote}\ atoms\ {\isacharparenleft}F{\isadigit{1}}\ \isactrlbold {\isasymand}\ F{\isadigit{2}}{\isacharparenright}\ {\isacharequal}\ Atom\ {\isacharbackquote}\ {\isacharparenleft}atoms\ F{\isadigit{1}}\ {\isasymunion}\ atoms\ F{\isadigit{2}}{\isacharparenright}{\isachardoublequoteclose}\isanewline
\ \ \ \ \isacommand{by}\isamarkupfalse%
\ {\isacharparenleft}simp\ only{\isacharcolon}\ formula{\isachardot}set{\isacharparenleft}{\isadigit{4}}{\isacharparenright}{\isacharparenright}\isanewline
\ \ \isacommand{also}\isamarkupfalse%
\ \isacommand{have}\isamarkupfalse%
\ {\isachardoublequoteopen}{\isasymdots}\ {\isacharequal}\ Atom\ {\isacharbackquote}\ atoms\ F{\isadigit{1}}\ {\isasymunion}\ Atom\ {\isacharbackquote}\ atoms\ F{\isadigit{2}}{\isachardoublequoteclose}\ \isanewline
\ \ \ \ \isacommand{by}\isamarkupfalse%
\ {\isacharparenleft}rule\ image{\isacharunderscore}Un{\isacharparenright}\isanewline
\ \ \isacommand{also}\isamarkupfalse%
\ \isacommand{have}\isamarkupfalse%
\ {\isachardoublequoteopen}{\isasymdots}\ {\isasymsubseteq}\ setSubformulae\ F{\isadigit{1}}\ {\isasymunion}\ setSubformulae\ F{\isadigit{2}}{\isachardoublequoteclose}\isanewline
\ \ \ \ \isacommand{using}\isamarkupfalse%
\ assms\isanewline
\ \ \ \ \isacommand{by}\isamarkupfalse%
\ {\isacharparenleft}rule\ Un{\isacharunderscore}mono{\isacharparenright}\isanewline
\ \ \isacommand{also}\isamarkupfalse%
\ \isacommand{have}\isamarkupfalse%
\ {\isachardoublequoteopen}{\isasymdots}\ {\isasymsubseteq}\ {\isacharbraceleft}F{\isadigit{1}}\ \isactrlbold {\isasymand}\ F{\isadigit{2}}{\isacharbraceright}\ {\isasymunion}\ {\isacharparenleft}setSubformulae\ F{\isadigit{1}}\ {\isasymunion}\ setSubformulae\ F{\isadigit{2}}{\isacharparenright}{\isachardoublequoteclose}\isanewline
\ \ \ \ \isacommand{by}\isamarkupfalse%
\ {\isacharparenleft}simp\ only{\isacharcolon}\ Un{\isacharunderscore}upper{\isadigit{2}}{\isacharparenright}\isanewline
\ \ \isacommand{also}\isamarkupfalse%
\ \isacommand{have}\isamarkupfalse%
\ {\isachardoublequoteopen}{\isasymdots}\ {\isacharequal}\ setSubformulae\ {\isacharparenleft}F{\isadigit{1}}\ \isactrlbold {\isasymand}\ F{\isadigit{2}}{\isacharparenright}{\isachardoublequoteclose}\isanewline
\ \ \ \ \isacommand{by}\isamarkupfalse%
\ {\isacharparenleft}simp\ only{\isacharcolon}\ setSubformulae{\isacharunderscore}and{\isacharparenright}\isanewline
\ \ \isacommand{finally}\isamarkupfalse%
\ \isacommand{show}\isamarkupfalse%
\ {\isacharquery}thesis\isanewline
\ \ \ \ \isacommand{by}\isamarkupfalse%
\ this\isanewline
\isacommand{qed}\isamarkupfalse%
%
\endisatagproof
{\isafoldproof}%
%
\isadelimproof
\isanewline
%
\endisadelimproof
\isanewline
\isacommand{lemma}\isamarkupfalse%
\ atoms{\isacharunderscore}are{\isacharunderscore}subformulae{\isacharcolon}\ \isanewline
\ \ {\isachardoublequoteopen}Atom\ {\isacharbackquote}\ atoms\ F\ {\isasymsubseteq}\ setSubformulae\ F{\isachardoublequoteclose}\isanewline
%
\isadelimproof
%
\endisadelimproof
%
\isatagproof
\isacommand{proof}\isamarkupfalse%
\ {\isacharparenleft}induction\ F{\isacharparenright}\isanewline
\ \ \isacommand{case}\isamarkupfalse%
\ {\isacharparenleft}Atom\ x{\isacharparenright}\isanewline
\ \ \isacommand{then}\isamarkupfalse%
\ \isacommand{show}\isamarkupfalse%
\ {\isacharquery}case\ \isacommand{by}\isamarkupfalse%
\ {\isacharparenleft}simp\ only{\isacharcolon}\ atoms{\isacharunderscore}are{\isacharunderscore}subformulae{\isacharunderscore}atom{\isacharparenright}\ \isanewline
\isacommand{next}\isamarkupfalse%
\isanewline
\ \ \isacommand{case}\isamarkupfalse%
\ Bot\isanewline
\ \ \isacommand{then}\isamarkupfalse%
\ \isacommand{show}\isamarkupfalse%
\ {\isacharquery}case\ \isacommand{by}\isamarkupfalse%
\ {\isacharparenleft}simp\ only{\isacharcolon}\ atoms{\isacharunderscore}are{\isacharunderscore}subformulae{\isacharunderscore}bot{\isacharparenright}\ \isanewline
\isacommand{next}\isamarkupfalse%
\isanewline
\ \ \isacommand{case}\isamarkupfalse%
\ {\isacharparenleft}Not\ F{\isacharparenright}\isanewline
\ \ \isacommand{then}\isamarkupfalse%
\ \isacommand{show}\isamarkupfalse%
\ {\isacharquery}case\ \isacommand{by}\isamarkupfalse%
\ {\isacharparenleft}simp\ only{\isacharcolon}\ atoms{\isacharunderscore}are{\isacharunderscore}subformulae{\isacharunderscore}not{\isacharparenright}\ \isanewline
\isacommand{next}\isamarkupfalse%
\isanewline
\ \ \isacommand{case}\isamarkupfalse%
\ {\isacharparenleft}And\ F{\isadigit{1}}\ F{\isadigit{2}}{\isacharparenright}\isanewline
\ \ \isacommand{then}\isamarkupfalse%
\ \isacommand{show}\isamarkupfalse%
\ {\isacharquery}case\ \isacommand{by}\isamarkupfalse%
\ {\isacharparenleft}simp\ only{\isacharcolon}\ atoms{\isacharunderscore}are{\isacharunderscore}subformulae{\isacharunderscore}and{\isacharparenright}\ \isanewline
\isacommand{next}\isamarkupfalse%
\isanewline
\ \ \isacommand{case}\isamarkupfalse%
\ {\isacharparenleft}Or\ F{\isadigit{1}}\ F{\isadigit{2}}{\isacharparenright}\isanewline
\ \ \isacommand{then}\isamarkupfalse%
\ \isacommand{show}\isamarkupfalse%
\ {\isacharquery}case\ \isacommand{by}\isamarkupfalse%
\ auto\isanewline
\isacommand{next}\isamarkupfalse%
\isanewline
\ \ \isacommand{case}\isamarkupfalse%
\ {\isacharparenleft}Imp\ F{\isadigit{1}}\ F{\isadigit{2}}{\isacharparenright}\isanewline
\ \ \isacommand{then}\isamarkupfalse%
\ \isacommand{show}\isamarkupfalse%
\ {\isacharquery}case\ \isacommand{by}\isamarkupfalse%
\ auto\isanewline
\isacommand{qed}\isamarkupfalse%
%
\endisatagproof
{\isafoldproof}%
%
\isadelimproof
%
\endisadelimproof
%
\begin{isamarkuptext}%
La demostración automática queda igualmente expuesta a 
  continuación.%
\end{isamarkuptext}\isamarkuptrue%
\isacommand{lemma}\isamarkupfalse%
\ {\isachardoublequoteopen}Atom\ {\isacharbackquote}\ atoms\ F\ {\isasymsubseteq}\ setSubformulae\ F{\isachardoublequoteclose}\isanewline
%
\isadelimproof
\ \ %
\endisadelimproof
%
\isatagproof
\isacommand{by}\isamarkupfalse%
\ {\isacharparenleft}induction\ F{\isacharparenright}\ \ auto%
\endisatagproof
{\isafoldproof}%
%
\isadelimproof
%
\endisadelimproof
%
\begin{isamarkuptext}%
La siguiente propiedad declara que el conjunto de átomos de una 
  subfórmula está contenido en el conjunto de átomos de la propia 
  fórmula.
  \begin{lema}
    Sea \isa{G\ {\isasymin}\ Subf{\isacharparenleft}F{\isacharparenright}}, entonces el \isa{conjAtoms{\isacharparenleft}G{\isacharparenright}\ {\isasymsubseteq}\ conjAtoms{\isacharparenleft}F{\isacharparenright}}.
  \end{lema}

  \begin{demostracion}
  Procedemos mediante inducción en la estructura de las fórmulas según 
  los distintos casos:

    Sea \isa{Atom\ p} una fórmula atómica cualquiera. Si \isa{G\ {\isasymin}\ Subf{\isacharparenleft}Atom\ p{\isacharparenright}}, 
    como \isa{conjAtoms{\isacharparenleft}Atom\ p{\isacharparenright}\ {\isacharequal}\ {\isacharbraceleft}Atom\ p{\isacharbraceright}}, se tiene \isa{G\ {\isacharequal}\ Atom\ p}. 
    Por tanto, \isa{conjAtoms{\isacharparenleft}G{\isacharparenright}\ {\isacharequal}\ conjAtoms{\isacharparenleft}Atom\ p{\isacharparenright}\ {\isasymsubseteq}\ conjAtoms{\isacharparenleft}Atom\ p{\isacharparenright}}.

    Sea la fórmula \isa{{\isasymbottom}}. Si \isa{G\ {\isasymin}\ Subf{\isacharparenleft}{\isasymbottom}{\isacharparenright}}, como \isa{conjAtoms{\isacharparenleft}{\isasymbottom}{\isacharparenright}\ {\isacharequal}\ {\isacharbraceleft}{\isasymbottom}{\isacharbraceright}}, 
    se tiene \isa{G\ {\isacharequal}\ {\isasymbottom}}. Por tanto, 
    \isa{conjAtoms{\isacharparenleft}G{\isacharparenright}\ {\isacharequal}\ conjAtoms{\isacharparenleft}{\isasymbottom}{\isacharparenright}\ {\isasymsubseteq}\ conjAtoms{\isacharparenleft}{\isasymbottom}{\isacharparenright}}.

    Sea la fórmula \isa{F} cualquiera tal que para cualquier subfórmula 
    \isa{G\ {\isasymin}\ Subf{\isacharparenleft}F{\isacharparenright}} se verifica \isa{conjAtoms{\isacharparenleft}G{\isacharparenright}\ {\isasymsubseteq}\ conjAtoms{\isacharparenleft}F{\isacharparenright}}. Supongamos 
    \isa{G{\isacharprime}\ {\isasymin}\ Subf{\isacharparenleft}{\isasymnot}\ F{\isacharparenright}} cualquiera, probemos que 
    \isa{conjAtoms{\isacharparenleft}G{\isacharprime}{\isacharparenright}\ {\isasymsubseteq}\ conjAtoms{\isacharparenleft}{\isasymnot}\ F{\isacharparenright}}.
    Por definición, tenemos que \isa{Subf{\isacharparenleft}{\isasymnot}\ F{\isacharparenright}\ {\isacharequal}\ {\isacharbraceleft}{\isasymnot}\ F{\isacharbraceright}\ {\isasymunion}\ Subf{\isacharparenleft}F{\isacharparenright}}. De este 
    modo, tenemos dos opciones:
    \isa{G{\isacharprime}\ {\isasymin}\ {\isacharbraceleft}{\isasymnot}\ F{\isacharbraceright}} o \isa{G{\isacharprime}\ {\isasymin}\ Subf{\isacharparenleft}F{\isacharparenright}}. Del primer caso se deduce \isa{G{\isacharprime}\ {\isacharequal}\ {\isasymnot}\ F} 
    y, por tanto, se tiene el
    resultado. Observando el segundo caso, por hipótesis de inducción, 
    se tiene \isa{conjAtoms{\isacharparenleft}G{\isacharprime}{\isacharparenright}\ {\isasymsubseteq}\ conjAtoms{\isacharparenleft}F{\isacharparenright}}. Además, como 
    \isa{conjAtoms{\isacharparenleft}{\isasymnot}\ F{\isacharparenright}\ {\isacharequal}\ conjAtoms{\isacharparenleft}F{\isacharparenright}}, se obtiene 
    \isa{conjAtoms{\isacharparenleft}G{\isacharprime}{\isacharparenright}\ {\isasymsubseteq}\ conjAtoms{\isacharparenleft}{\isasymnot}\ F{\isacharparenright}} como queríamos probar.

    Sea \isa{F{\isadigit{1}}} fórmula proposicional tal que para cualquier \isa{G\ {\isasymin}\ Subf{\isacharparenleft}F{\isadigit{1}}{\isacharparenright}} 
    se tiene \isa{conjAtoms{\isacharparenleft}G{\isacharparenright}\ {\isasymsubseteq}\ conjAtoms{\isacharparenleft}F{\isadigit{1}}{\isacharparenright}}. Sea también \isa{F{\isadigit{2}}} tal que 
    dada \isa{G\ {\isasymin}\ Subf{\isacharparenleft}F{\isadigit{2}}{\isacharparenright}} cualquiera se tiene también 
    \isa{conjAtoms{\isacharparenleft}G{\isacharparenright}\ {\isasymsubseteq}\ conjAtoms{\isacharparenleft}F{\isadigit{2}}{\isacharparenright}}. Supongamos \isa{G{\isacharprime}\ {\isasymin}\ Subf{\isacharparenleft}F{\isadigit{1}}{\isacharasterisk}F{\isadigit{2}}{\isacharparenright}} donde 
    \isa{{\isacharasterisk}} es cualquier conectiva binaria. Vamos a probar que 
    \isa{conjAtoms{\isacharparenleft}G{\isacharprime}{\isacharparenright}\ {\isasymsubseteq}\ conjAtoms{\isacharparenleft}F{\isadigit{1}}{\isacharasterisk}F{\isadigit{2}}{\isacharparenright}}.

    En primer lugar, como 
    \isa{Subf{\isacharparenleft}F{\isadigit{1}}{\isacharasterisk}F{\isadigit{2}}{\isacharparenright}\ {\isacharequal}\ {\isacharbraceleft}F{\isadigit{1}}{\isacharasterisk}F{\isadigit{2}}{\isacharbraceright}\ {\isasymunion}\ {\isacharparenleft}Subf{\isacharparenleft}F{\isadigit{1}}{\isacharparenright}\ {\isasymunion}\ Subf{\isacharparenleft}F{\isadigit{2}}{\isacharparenright}{\isacharparenright}}, se desglosan tres
    casos posibles para \isa{G{\isacharprime}}:
    Si \isa{G{\isacharprime}\ {\isasymin}\ {\isacharbraceleft}F{\isadigit{1}}{\isacharasterisk}F{\isadigit{2}}{\isacharbraceright}}, entonces \isa{G{\isacharprime}\ {\isacharequal}\ F{\isadigit{1}}{\isacharasterisk}F{\isadigit{2}}} y se tiene la propiedad.
    Si \isa{G{\isacharprime}\ {\isasymin}\ Subf{\isacharparenleft}F{\isadigit{1}}{\isacharparenright}\ {\isasymunion}\ Subf{\isacharparenleft}F{\isadigit{2}}{\isacharparenright}}, entonces por propiedades de 
    conjuntos:
    \isa{G{\isacharprime}\ {\isasymin}\ Subf{\isacharparenleft}F{\isadigit{1}}{\isacharparenright}\ {\isasymor}\ G{\isacharprime}\ {\isasymin}\ Subf{\isacharparenleft}F{\isadigit{2}}{\isacharparenright}}. Si \isa{G{\isacharprime}\ {\isasymin}\ Subf{\isacharparenleft}F{\isadigit{1}}{\isacharparenright}}, por hipótesis 
    de inducción se tiene \isa{conjAtoms{\isacharparenleft}G{\isacharprime}{\isacharparenright}\ {\isasymsubseteq}\ conjAtoms{\isacharparenleft}F{\isadigit{1}}{\isacharparenright}}. Como 
    \isa{conjAtoms{\isacharparenleft}F{\isadigit{1}}{\isacharasterisk}F{\isadigit{2}}{\isacharparenright}\ {\isacharequal}\ conjAtoms{\isacharparenleft}F{\isadigit{1}}{\isacharparenright}\ {\isasymunion}\ conjAtoms{\isacharparenleft}F{\isadigit{2}}{\isacharparenright}}, se obtiene el 
    resultado como consecuencia de la transitividad de contención para 
    conjuntos. El caso \isa{G{\isacharprime}\ {\isasymin}\ Subf{\isacharparenleft}F{\isadigit{2}}{\isacharparenright}} se demuestra de la misma forma.      
  \end{demostracion}

  Formalizado en Isabelle:%
\end{isamarkuptext}\isamarkuptrue%
\isacommand{lemma}\isamarkupfalse%
\ subformula{\isacharunderscore}atoms{\isacharcolon}\ {\isachardoublequoteopen}G\ {\isasymin}\ setSubformulae\ F\ {\isasymLongrightarrow}\ atoms\ G\ {\isasymsubseteq}\ atoms\ F{\isachardoublequoteclose}\isanewline
%
\isadelimproof
\ \ %
\endisadelimproof
%
\isatagproof
\isacommand{oops}\isamarkupfalse%
%
\endisatagproof
{\isafoldproof}%
%
\isadelimproof
%
\endisadelimproof
%
\begin{isamarkuptext}%
Veamos su demostración estructurada. Desarrollaré la disyunción como representante del caso
  de las conectivas binarias, pues los demás son equivalentes.%
\end{isamarkuptext}\isamarkuptrue%
\isacommand{lemma}\isamarkupfalse%
\ subformulas{\isacharunderscore}atoms{\isacharunderscore}atom{\isacharcolon}\isanewline
\ \ \isakeyword{assumes}\ {\isachardoublequoteopen}G\ {\isasymin}\ setSubformulae\ {\isacharparenleft}Atom\ x{\isacharparenright}{\isachardoublequoteclose}\ \isanewline
\ \ \isakeyword{shows}\ \ \ {\isachardoublequoteopen}atoms\ G\ {\isasymsubseteq}\ atoms\ {\isacharparenleft}Atom\ x{\isacharparenright}{\isachardoublequoteclose}\isanewline
%
\isadelimproof
%
\endisadelimproof
%
\isatagproof
\isacommand{proof}\isamarkupfalse%
\ {\isacharminus}\isanewline
\ \ \isacommand{have}\isamarkupfalse%
\ {\isachardoublequoteopen}G\ {\isasymin}\ {\isacharbraceleft}Atom\ x{\isacharbraceright}{\isachardoublequoteclose}\isanewline
\ \ \ \ \isacommand{using}\isamarkupfalse%
\ assms\isanewline
\ \ \ \ \isacommand{by}\isamarkupfalse%
\ {\isacharparenleft}simp\ only{\isacharcolon}\ setSubformulae{\isacharunderscore}atom{\isacharparenright}\isanewline
\ \ \isacommand{then}\isamarkupfalse%
\ \isacommand{have}\isamarkupfalse%
\ {\isachardoublequoteopen}G\ {\isacharequal}\ Atom\ x{\isachardoublequoteclose}\isanewline
\ \ \ \ \isacommand{by}\isamarkupfalse%
\ {\isacharparenleft}simp\ only{\isacharcolon}\ singletonD{\isacharparenright}\isanewline
\ \ \isacommand{then}\isamarkupfalse%
\ \isacommand{show}\isamarkupfalse%
\ {\isacharquery}thesis\isanewline
\ \ \ \ \isacommand{by}\isamarkupfalse%
\ {\isacharparenleft}simp\ only{\isacharcolon}\ subset{\isacharunderscore}refl{\isacharparenright}\isanewline
\isacommand{qed}\isamarkupfalse%
%
\endisatagproof
{\isafoldproof}%
%
\isadelimproof
\isanewline
%
\endisadelimproof
\isanewline
\isacommand{lemma}\isamarkupfalse%
\ subformulas{\isacharunderscore}atoms{\isacharunderscore}bot{\isacharcolon}\isanewline
\ \ \isakeyword{assumes}\ {\isachardoublequoteopen}G\ {\isasymin}\ setSubformulae\ {\isasymbottom}{\isachardoublequoteclose}\ \isanewline
\ \ \isakeyword{shows}\ \ \ {\isachardoublequoteopen}atoms\ G\ {\isasymsubseteq}\ atoms\ {\isasymbottom}{\isachardoublequoteclose}\isanewline
%
\isadelimproof
%
\endisadelimproof
%
\isatagproof
\isacommand{proof}\isamarkupfalse%
\ {\isacharminus}\isanewline
\ \ \isacommand{have}\isamarkupfalse%
\ {\isachardoublequoteopen}G\ {\isasymin}\ {\isacharbraceleft}{\isasymbottom}{\isacharbraceright}{\isachardoublequoteclose}\isanewline
\ \ \ \ \isacommand{using}\isamarkupfalse%
\ assms\isanewline
\ \ \ \ \isacommand{by}\isamarkupfalse%
\ {\isacharparenleft}simp\ only{\isacharcolon}\ setSubformulae{\isacharunderscore}bot{\isacharparenright}\isanewline
\ \ \isacommand{then}\isamarkupfalse%
\ \isacommand{have}\isamarkupfalse%
\ {\isachardoublequoteopen}G\ {\isacharequal}\ {\isasymbottom}{\isachardoublequoteclose}\isanewline
\ \ \ \ \isacommand{by}\isamarkupfalse%
\ {\isacharparenleft}simp\ only{\isacharcolon}\ singletonD{\isacharparenright}\isanewline
\ \ \isacommand{then}\isamarkupfalse%
\ \isacommand{show}\isamarkupfalse%
\ {\isacharquery}thesis\isanewline
\ \ \ \ \isacommand{by}\isamarkupfalse%
\ {\isacharparenleft}simp\ only{\isacharcolon}\ subset{\isacharunderscore}refl{\isacharparenright}\isanewline
\isacommand{qed}\isamarkupfalse%
%
\endisatagproof
{\isafoldproof}%
%
\isadelimproof
\isanewline
%
\endisadelimproof
\isanewline
\isacommand{lemma}\isamarkupfalse%
\ subformulas{\isacharunderscore}atoms{\isacharunderscore}not{\isacharcolon}\isanewline
\ \ \isakeyword{assumes}\ {\isachardoublequoteopen}G\ {\isasymin}\ setSubformulae\ F\ {\isasymLongrightarrow}\ atoms\ G\ {\isasymsubseteq}\ atoms\ F{\isachardoublequoteclose}\isanewline
\ \ \ \ \ \ \ \ \ \ {\isachardoublequoteopen}G\ {\isasymin}\ setSubformulae\ {\isacharparenleft}\isactrlbold {\isasymnot}\ F{\isacharparenright}{\isachardoublequoteclose}\isanewline
\ \ \isakeyword{shows}\ \ \ {\isachardoublequoteopen}atoms\ G\ {\isasymsubseteq}\ atoms\ {\isacharparenleft}\isactrlbold {\isasymnot}\ F{\isacharparenright}{\isachardoublequoteclose}\isanewline
%
\isadelimproof
%
\endisadelimproof
%
\isatagproof
\isacommand{proof}\isamarkupfalse%
\ {\isacharminus}\isanewline
\ \ \isacommand{have}\isamarkupfalse%
\ {\isachardoublequoteopen}G\ {\isasymin}\ {\isacharbraceleft}\isactrlbold {\isasymnot}\ F{\isacharbraceright}\ {\isasymunion}\ setSubformulae\ F{\isachardoublequoteclose}\isanewline
\ \ \ \ \isacommand{using}\isamarkupfalse%
\ assms{\isacharparenleft}{\isadigit{2}}{\isacharparenright}\isanewline
\ \ \ \ \isacommand{by}\isamarkupfalse%
\ {\isacharparenleft}simp\ only{\isacharcolon}\ setSubformulae{\isacharunderscore}not{\isacharparenright}\ \isanewline
\ \ \isacommand{then}\isamarkupfalse%
\ \isacommand{have}\isamarkupfalse%
\ {\isachardoublequoteopen}G\ {\isasymin}\ {\isacharbraceleft}\isactrlbold {\isasymnot}\ F{\isacharbraceright}\ {\isasymor}\ G\ {\isasymin}\ setSubformulae\ F{\isachardoublequoteclose}\isanewline
\ \ \ \ \isacommand{by}\isamarkupfalse%
\ {\isacharparenleft}simp\ only{\isacharcolon}\ Un{\isacharunderscore}iff{\isacharparenright}\isanewline
\ \ \isacommand{then}\isamarkupfalse%
\ \isacommand{show}\isamarkupfalse%
\ {\isachardoublequoteopen}atoms\ G\ {\isasymsubseteq}\ atoms\ {\isacharparenleft}\isactrlbold {\isasymnot}\ F{\isacharparenright}{\isachardoublequoteclose}\isanewline
\ \ \isacommand{proof}\isamarkupfalse%
\isanewline
\ \ \ \ \isacommand{assume}\isamarkupfalse%
\ {\isachardoublequoteopen}G\ {\isasymin}\ {\isacharbraceleft}\isactrlbold {\isasymnot}\ F{\isacharbraceright}{\isachardoublequoteclose}\isanewline
\ \ \ \ \isacommand{then}\isamarkupfalse%
\ \isacommand{have}\isamarkupfalse%
\ {\isachardoublequoteopen}G\ {\isacharequal}\ \isactrlbold {\isasymnot}\ F{\isachardoublequoteclose}\isanewline
\ \ \ \ \ \ \isacommand{by}\isamarkupfalse%
\ {\isacharparenleft}simp\ only{\isacharcolon}\ singletonD{\isacharparenright}\isanewline
\ \ \ \ \isacommand{then}\isamarkupfalse%
\ \isacommand{show}\isamarkupfalse%
\ {\isacharquery}thesis\isanewline
\ \ \ \ \ \ \isacommand{by}\isamarkupfalse%
\ {\isacharparenleft}simp\ only{\isacharcolon}\ subset{\isacharunderscore}refl{\isacharparenright}\isanewline
\ \ \isacommand{next}\isamarkupfalse%
\isanewline
\ \ \ \ \isacommand{assume}\isamarkupfalse%
\ {\isachardoublequoteopen}G\ {\isasymin}\ setSubformulae\ F{\isachardoublequoteclose}\isanewline
\ \ \ \ \isacommand{then}\isamarkupfalse%
\ \isacommand{have}\isamarkupfalse%
\ {\isachardoublequoteopen}atoms\ G\ {\isasymsubseteq}\ atoms\ F{\isachardoublequoteclose}\isanewline
\ \ \ \ \ \ \isacommand{by}\isamarkupfalse%
\ {\isacharparenleft}simp\ only{\isacharcolon}\ assms{\isacharparenleft}{\isadigit{1}}{\isacharparenright}{\isacharparenright}\isanewline
\ \ \ \ \isacommand{also}\isamarkupfalse%
\ \isacommand{have}\isamarkupfalse%
\ {\isachardoublequoteopen}{\isasymdots}\ {\isacharequal}\ atoms\ {\isacharparenleft}\isactrlbold {\isasymnot}\ F{\isacharparenright}{\isachardoublequoteclose}\isanewline
\ \ \ \ \ \ \isacommand{by}\isamarkupfalse%
\ {\isacharparenleft}simp\ only{\isacharcolon}\ formula{\isachardot}set{\isacharparenleft}{\isadigit{3}}{\isacharparenright}{\isacharparenright}\isanewline
\ \ \ \ \isacommand{finally}\isamarkupfalse%
\ \isacommand{show}\isamarkupfalse%
\ {\isacharquery}thesis\isanewline
\ \ \ \ \ \ \isacommand{by}\isamarkupfalse%
\ this\isanewline
\ \ \isacommand{qed}\isamarkupfalse%
\isanewline
\isacommand{qed}\isamarkupfalse%
%
\endisatagproof
{\isafoldproof}%
%
\isadelimproof
\isanewline
%
\endisadelimproof
\isanewline
\isacommand{lemma}\isamarkupfalse%
\ subformulas{\isacharunderscore}atoms{\isacharunderscore}or{\isacharcolon}\isanewline
\ \ \isakeyword{assumes}\ {\isachardoublequoteopen}G\ {\isasymin}\ setSubformulae\ F{\isadigit{1}}\ {\isasymLongrightarrow}\ atoms\ G\ {\isasymsubseteq}\ atoms\ F{\isadigit{1}}{\isachardoublequoteclose}\isanewline
\ \ \ \ \ \ \ \ \ \ {\isachardoublequoteopen}G\ {\isasymin}\ setSubformulae\ F{\isadigit{2}}\ {\isasymLongrightarrow}\ atoms\ G\ {\isasymsubseteq}\ atoms\ F{\isadigit{2}}{\isachardoublequoteclose}\isanewline
\ \ \ \ \ \ \ \ \ \ {\isachardoublequoteopen}G\ {\isasymin}\ setSubformulae\ {\isacharparenleft}F{\isadigit{1}}\ \isactrlbold {\isasymor}\ F{\isadigit{2}}{\isacharparenright}{\isachardoublequoteclose}\isanewline
\ \ \isakeyword{shows}\ \ \ {\isachardoublequoteopen}atoms\ G\ {\isasymsubseteq}\ atoms\ {\isacharparenleft}F{\isadigit{1}}\ \isactrlbold {\isasymor}\ F{\isadigit{2}}{\isacharparenright}{\isachardoublequoteclose}\isanewline
%
\isadelimproof
%
\endisadelimproof
%
\isatagproof
\isacommand{proof}\isamarkupfalse%
\ {\isacharminus}\isanewline
\ \ \isacommand{have}\isamarkupfalse%
\ {\isachardoublequoteopen}G\ {\isasymin}\ {\isacharbraceleft}F{\isadigit{1}}\ \isactrlbold {\isasymor}\ F{\isadigit{2}}{\isacharbraceright}\ {\isasymunion}\ {\isacharparenleft}setSubformulae\ F{\isadigit{1}}\ {\isasymunion}\ setSubformulae\ F{\isadigit{2}}{\isacharparenright}{\isachardoublequoteclose}\isanewline
\ \ \ \ \isacommand{using}\isamarkupfalse%
\ assms{\isacharparenleft}{\isadigit{3}}{\isacharparenright}\ \isanewline
\ \ \ \ \isacommand{by}\isamarkupfalse%
\ {\isacharparenleft}simp\ only{\isacharcolon}\ setSubformulae{\isacharunderscore}or{\isacharparenright}\isanewline
\ \ \isacommand{then}\isamarkupfalse%
\ \isacommand{have}\isamarkupfalse%
\ {\isachardoublequoteopen}G\ {\isasymin}\ {\isacharbraceleft}F{\isadigit{1}}\ \isactrlbold {\isasymor}\ F{\isadigit{2}}{\isacharbraceright}\ {\isasymor}\ G\ {\isasymin}\ setSubformulae\ F{\isadigit{1}}\ {\isasymunion}\ setSubformulae\ F{\isadigit{2}}{\isachardoublequoteclose}\isanewline
\ \ \ \ \isacommand{by}\isamarkupfalse%
\ {\isacharparenleft}simp\ only{\isacharcolon}\ Un{\isacharunderscore}iff{\isacharparenright}\isanewline
\ \ \isacommand{then}\isamarkupfalse%
\ \isacommand{show}\isamarkupfalse%
\ {\isacharquery}thesis\isanewline
\ \ \isacommand{proof}\isamarkupfalse%
\ \isanewline
\ \ \ \ \isacommand{assume}\isamarkupfalse%
\ {\isachardoublequoteopen}G\ {\isasymin}\ {\isacharbraceleft}F{\isadigit{1}}\ \isactrlbold {\isasymor}\ F{\isadigit{2}}{\isacharbraceright}{\isachardoublequoteclose}\isanewline
\ \ \ \ \isacommand{then}\isamarkupfalse%
\ \isacommand{have}\isamarkupfalse%
\ {\isachardoublequoteopen}G\ {\isacharequal}\ F{\isadigit{1}}\ \isactrlbold {\isasymor}\ F{\isadigit{2}}{\isachardoublequoteclose}\isanewline
\ \ \ \ \ \ \isacommand{by}\isamarkupfalse%
\ {\isacharparenleft}simp\ only{\isacharcolon}\ singletonD{\isacharparenright}\isanewline
\ \ \ \ \isacommand{then}\isamarkupfalse%
\ \isacommand{show}\isamarkupfalse%
\ {\isacharquery}thesis\isanewline
\ \ \ \ \ \ \isacommand{by}\isamarkupfalse%
\ {\isacharparenleft}simp\ only{\isacharcolon}\ subset{\isacharunderscore}refl{\isacharparenright}\isanewline
\ \ \isacommand{next}\isamarkupfalse%
\isanewline
\ \ \ \ \isacommand{assume}\isamarkupfalse%
\ {\isachardoublequoteopen}G\ {\isasymin}\ setSubformulae\ F{\isadigit{1}}\ {\isasymunion}\ setSubformulae\ F{\isadigit{2}}{\isachardoublequoteclose}\isanewline
\ \ \ \ \isacommand{then}\isamarkupfalse%
\ \isacommand{have}\isamarkupfalse%
\ {\isachardoublequoteopen}G\ {\isasymin}\ setSubformulae\ F{\isadigit{1}}\ {\isasymor}\ G\ {\isasymin}\ setSubformulae\ F{\isadigit{2}}{\isachardoublequoteclose}\ \ \isanewline
\ \ \ \ \ \ \isacommand{by}\isamarkupfalse%
\ {\isacharparenleft}simp\ only{\isacharcolon}\ Un{\isacharunderscore}iff{\isacharparenright}\isanewline
\ \ \ \ \isacommand{then}\isamarkupfalse%
\ \isacommand{show}\isamarkupfalse%
\ {\isacharquery}thesis\isanewline
\ \ \ \ \isacommand{proof}\isamarkupfalse%
\ \isanewline
\ \ \ \ \ \ \isacommand{assume}\isamarkupfalse%
\ {\isachardoublequoteopen}G\ {\isasymin}\ setSubformulae\ F{\isadigit{1}}{\isachardoublequoteclose}\isanewline
\ \ \ \ \ \ \isacommand{then}\isamarkupfalse%
\ \isacommand{have}\isamarkupfalse%
\ {\isachardoublequoteopen}atoms\ G\ {\isasymsubseteq}\ atoms\ F{\isadigit{1}}{\isachardoublequoteclose}\isanewline
\ \ \ \ \ \ \ \ \isacommand{by}\isamarkupfalse%
\ {\isacharparenleft}rule\ assms{\isacharparenleft}{\isadigit{1}}{\isacharparenright}{\isacharparenright}\isanewline
\ \ \ \ \ \ \isacommand{also}\isamarkupfalse%
\ \isacommand{have}\isamarkupfalse%
\ {\isachardoublequoteopen}{\isasymdots}\ {\isasymsubseteq}\ atoms\ F{\isadigit{1}}\ {\isasymunion}\ atoms\ F{\isadigit{2}}{\isachardoublequoteclose}\isanewline
\ \ \ \ \ \ \ \ \isacommand{by}\isamarkupfalse%
\ {\isacharparenleft}simp\ only{\isacharcolon}\ Un{\isacharunderscore}upper{\isadigit{1}}{\isacharparenright}\isanewline
\ \ \ \ \ \ \isacommand{also}\isamarkupfalse%
\ \isacommand{have}\isamarkupfalse%
\ {\isachardoublequoteopen}{\isasymdots}\ {\isacharequal}\ atoms\ {\isacharparenleft}F{\isadigit{1}}\ \isactrlbold {\isasymor}\ F{\isadigit{2}}{\isacharparenright}{\isachardoublequoteclose}\isanewline
\ \ \ \ \ \ \ \ \isacommand{by}\isamarkupfalse%
\ {\isacharparenleft}simp\ only{\isacharcolon}\ formula{\isachardot}set{\isacharparenleft}{\isadigit{5}}{\isacharparenright}{\isacharparenright}\isanewline
\ \ \ \ \ \ \isacommand{finally}\isamarkupfalse%
\ \isacommand{show}\isamarkupfalse%
\ {\isacharquery}thesis\isanewline
\ \ \ \ \ \ \ \ \isacommand{by}\isamarkupfalse%
\ this\isanewline
\ \ \ \ \isacommand{next}\isamarkupfalse%
\isanewline
\ \ \ \ \ \ \isacommand{assume}\isamarkupfalse%
\ {\isachardoublequoteopen}G\ {\isasymin}\ setSubformulae\ F{\isadigit{2}}{\isachardoublequoteclose}\isanewline
\ \ \ \ \ \ \isacommand{then}\isamarkupfalse%
\ \isacommand{have}\isamarkupfalse%
\ {\isachardoublequoteopen}atoms\ G\ {\isasymsubseteq}\ atoms\ F{\isadigit{2}}{\isachardoublequoteclose}\isanewline
\ \ \ \ \ \ \ \ \isacommand{by}\isamarkupfalse%
\ {\isacharparenleft}rule\ assms{\isacharparenleft}{\isadigit{2}}{\isacharparenright}{\isacharparenright}\isanewline
\ \ \ \ \ \ \isacommand{also}\isamarkupfalse%
\ \isacommand{have}\isamarkupfalse%
\ {\isachardoublequoteopen}{\isasymdots}\ {\isasymsubseteq}\ atoms\ F{\isadigit{1}}\ {\isasymunion}\ atoms\ F{\isadigit{2}}{\isachardoublequoteclose}\isanewline
\ \ \ \ \ \ \ \ \isacommand{by}\isamarkupfalse%
\ {\isacharparenleft}simp\ only{\isacharcolon}\ Un{\isacharunderscore}upper{\isadigit{2}}{\isacharparenright}\isanewline
\ \ \ \ \ \ \isacommand{also}\isamarkupfalse%
\ \isacommand{have}\isamarkupfalse%
\ {\isachardoublequoteopen}{\isasymdots}\ {\isacharequal}\ atoms\ {\isacharparenleft}F{\isadigit{1}}\ \isactrlbold {\isasymor}\ F{\isadigit{2}}{\isacharparenright}{\isachardoublequoteclose}\isanewline
\ \ \ \ \ \ \ \ \isacommand{by}\isamarkupfalse%
\ {\isacharparenleft}simp\ only{\isacharcolon}\ formula{\isachardot}set{\isacharparenleft}{\isadigit{5}}{\isacharparenright}{\isacharparenright}\isanewline
\ \ \ \ \ \ \isacommand{finally}\isamarkupfalse%
\ \isacommand{show}\isamarkupfalse%
\ {\isacharquery}thesis\isanewline
\ \ \ \ \ \ \ \ \isacommand{by}\isamarkupfalse%
\ this\isanewline
\ \ \ \ \isacommand{qed}\isamarkupfalse%
\isanewline
\ \ \isacommand{qed}\isamarkupfalse%
\isanewline
\isacommand{qed}\isamarkupfalse%
%
\endisatagproof
{\isafoldproof}%
%
\isadelimproof
\isanewline
%
\endisadelimproof
\isanewline
\isacommand{lemma}\isamarkupfalse%
\ subformulas{\isacharunderscore}atoms{\isacharcolon}\isanewline
\ \ {\isachardoublequoteopen}G\ {\isasymin}\ setSubformulae\ F\ {\isasymLongrightarrow}\ atoms\ G\ {\isasymsubseteq}\ atoms\ F{\isachardoublequoteclose}\isanewline
%
\isadelimproof
%
\endisadelimproof
%
\isatagproof
\isacommand{proof}\isamarkupfalse%
\ {\isacharparenleft}induction\ F{\isacharparenright}\isanewline
\ \ \isacommand{case}\isamarkupfalse%
\ {\isacharparenleft}Atom\ x{\isacharparenright}\isanewline
\ \ \isacommand{then}\isamarkupfalse%
\ \isacommand{show}\isamarkupfalse%
\ {\isacharquery}case\ \isacommand{by}\isamarkupfalse%
\ {\isacharparenleft}simp\ only{\isacharcolon}\ subformulas{\isacharunderscore}atoms{\isacharunderscore}atom{\isacharparenright}\ \isanewline
\isacommand{next}\isamarkupfalse%
\isanewline
\ \ \isacommand{case}\isamarkupfalse%
\ Bot\isanewline
\ \ \isacommand{then}\isamarkupfalse%
\ \isacommand{show}\isamarkupfalse%
\ {\isacharquery}case\ \isacommand{by}\isamarkupfalse%
\ {\isacharparenleft}simp\ only{\isacharcolon}\ subformulas{\isacharunderscore}atoms{\isacharunderscore}bot{\isacharparenright}\isanewline
\isacommand{next}\isamarkupfalse%
\isanewline
\ \ \isacommand{case}\isamarkupfalse%
\ {\isacharparenleft}Not\ F{\isacharparenright}\isanewline
\ \ \isacommand{then}\isamarkupfalse%
\ \isacommand{show}\isamarkupfalse%
\ {\isacharquery}case\ \isacommand{by}\isamarkupfalse%
\ {\isacharparenleft}simp\ only{\isacharcolon}\ subformulas{\isacharunderscore}atoms{\isacharunderscore}not{\isacharparenright}\isanewline
\isacommand{next}\isamarkupfalse%
\isanewline
\ \ \isacommand{case}\isamarkupfalse%
\ {\isacharparenleft}And\ F{\isadigit{1}}\ F{\isadigit{2}}{\isacharparenright}\isanewline
\ \ \isacommand{then}\isamarkupfalse%
\ \isacommand{show}\isamarkupfalse%
\ {\isacharquery}case\ \isacommand{by}\isamarkupfalse%
\ auto\isanewline
\isacommand{next}\isamarkupfalse%
\isanewline
\ \ \isacommand{case}\isamarkupfalse%
\ {\isacharparenleft}Or\ F{\isadigit{1}}\ F{\isadigit{2}}{\isacharparenright}\isanewline
\ \ \isacommand{then}\isamarkupfalse%
\ \isacommand{show}\isamarkupfalse%
\ {\isacharquery}case\ \isacommand{by}\isamarkupfalse%
\ {\isacharparenleft}simp\ only{\isacharcolon}\ subformulas{\isacharunderscore}atoms{\isacharunderscore}or{\isacharparenright}\isanewline
\isacommand{next}\isamarkupfalse%
\isanewline
\ \ \isacommand{case}\isamarkupfalse%
\ {\isacharparenleft}Imp\ F{\isadigit{1}}\ F{\isadigit{2}}{\isacharparenright}\isanewline
\ \ \isacommand{then}\isamarkupfalse%
\ \isacommand{show}\isamarkupfalse%
\ {\isacharquery}case\ \isacommand{by}\isamarkupfalse%
\ auto\isanewline
\isacommand{qed}\isamarkupfalse%
%
\endisatagproof
{\isafoldproof}%
%
\isadelimproof
%
\endisadelimproof
%
\begin{isamarkuptext}%
Por último, su demostración aplicativa automática.%
\end{isamarkuptext}\isamarkuptrue%
\isacommand{lemma}\isamarkupfalse%
\ subformula{\isacharunderscore}atoms{\isacharcolon}\ {\isachardoublequoteopen}G\ {\isasymin}\ setSubformulae\ F\ {\isasymLongrightarrow}\ atoms\ G\ {\isasymsubseteq}\ atoms\ F{\isachardoublequoteclose}\isanewline
%
\isadelimproof
\ \ %
\endisadelimproof
%
\isatagproof
\isacommand{by}\isamarkupfalse%
\ {\isacharparenleft}induction\ F{\isacharparenright}\ auto%
\endisatagproof
{\isafoldproof}%
%
\isadelimproof
%
\endisadelimproof
%
\begin{isamarkuptext}%
CORREGIDO HASTA AQUÍ%
\end{isamarkuptext}\isamarkuptrue%
%
\isadelimtheory
%
\endisadelimtheory
%
\isatagtheory
%
\endisatagtheory
{\isafoldtheory}%
%
\isadelimtheory
%
\endisadelimtheory
%
\end{isabellebody}%
\endinput
%:%file=~/Desktop/LogicaProposicional/Sintaxis.thy%:%
%:%24=11%:%
%:%28=13%:%
%:%40=15%:%
%:%41=16%:%
%:%43=18%:%
%:%44=18%:%
%:%46=20%:%
%:%47=21%:%
%:%48=22%:%
%:%49=23%:%
%:%50=24%:%
%:%51=25%:%
%:%52=26%:%
%:%53=27%:%
%:%54=28%:%
%:%55=29%:%
%:%56=30%:%
%:%57=31%:%
%:%58=32%:%
%:%59=33%:%
%:%60=34%:%
%:%61=35%:%
%:%62=36%:%
%:%63=37%:%
%:%64=38%:%
%:%65=39%:%
%:%66=40%:%
%:%67=41%:%
%:%68=42%:%
%:%69=43%:%
%:%70=44%:%
%:%71=45%:%
%:%72=46%:%
%:%73=47%:%
%:%74=48%:%
%:%75=49%:%
%:%76=50%:%
%:%77=51%:%
%:%78=52%:%
%:%79=53%:%
%:%80=54%:%
%:%81=55%:%
%:%82=56%:%
%:%83=57%:%
%:%84=58%:%
%:%85=59%:%
%:%86=60%:%
%:%87=61%:%
%:%88=62%:%
%:%89=63%:%
%:%90=64%:%
%:%91=65%:%
%:%92=66%:%
%:%93=67%:%
%:%94=68%:%
%:%95=69%:%
%:%97=71%:%
%:%98=71%:%
%:%99=72%:%
%:%100=73%:%
%:%101=74%:%
%:%102=75%:%
%:%103=76%:%
%:%104=77%:%
%:%106=79%:%
%:%107=80%:%
%:%108=81%:%
%:%109=82%:%
%:%110=83%:%
%:%111=84%:%
%:%112=85%:%
%:%113=86%:%
%:%114=87%:%
%:%115=88%:%
%:%116=89%:%
%:%117=90%:%
%:%118=91%:%
%:%119=92%:%
%:%120=93%:%
%:%121=94%:%
%:%122=95%:%
%:%123=96%:%
%:%124=97%:%
%:%125=98%:%
%:%126=99%:%
%:%127=100%:%
%:%128=101%:%
%:%129=102%:%
%:%130=103%:%
%:%131=104%:%
%:%132=105%:%
%:%133=106%:%
%:%134=107%:%
%:%135=108%:%
%:%136=109%:%
%:%137=110%:%
%:%138=111%:%
%:%139=112%:%
%:%140=113%:%
%:%141=114%:%
%:%142=115%:%
%:%143=116%:%
%:%144=117%:%
%:%149=117%:%
%:%150=118%:%
%:%151=119%:%
%:%152=120%:%
%:%153=121%:%
%:%154=122%:%
%:%155=123%:%
%:%157=125%:%
%:%158=125%:%
%:%159=126%:%
%:%162=127%:%
%:%166=127%:%
%:%167=127%:%
%:%168=128%:%
%:%169=129%:%
%:%170=129%:%
%:%171=130%:%
%:%172=130%:%
%:%173=131%:%
%:%174=132%:%
%:%175=132%:%
%:%176=133%:%
%:%177=133%:%
%:%178=134%:%
%:%179=135%:%
%:%180=135%:%
%:%181=136%:%
%:%182=136%:%
%:%183=137%:%
%:%184=138%:%
%:%185=138%:%
%:%186=139%:%
%:%187=139%:%
%:%192=139%:%
%:%195=140%:%
%:%196=141%:%
%:%199=143%:%
%:%200=144%:%
%:%201=145%:%
%:%203=147%:%
%:%204=147%:%
%:%205=148%:%
%:%208=149%:%
%:%212=149%:%
%:%213=149%:%
%:%214=150%:%
%:%215=151%:%
%:%216=151%:%
%:%217=152%:%
%:%218=152%:%
%:%219=153%:%
%:%220=154%:%
%:%221=154%:%
%:%222=155%:%
%:%223=155%:%
%:%224=156%:%
%:%225=156%:%
%:%226=157%:%
%:%227=157%:%
%:%228=158%:%
%:%229=158%:%
%:%230=158%:%
%:%231=159%:%
%:%232=159%:%
%:%233=160%:%
%:%234=160%:%
%:%235=160%:%
%:%236=161%:%
%:%237=161%:%
%:%238=162%:%
%:%239=162%:%
%:%240=162%:%
%:%241=163%:%
%:%242=163%:%
%:%243=164%:%
%:%244=164%:%
%:%245=164%:%
%:%246=165%:%
%:%247=165%:%
%:%248=166%:%
%:%249=166%:%
%:%250=167%:%
%:%251=168%:%
%:%252=168%:%
%:%253=169%:%
%:%254=169%:%
%:%259=169%:%
%:%262=170%:%
%:%263=170%:%
%:%264=171%:%
%:%265=172%:%
%:%266=172%:%
%:%268=174%:%
%:%269=175%:%
%:%270=176%:%
%:%271=177%:%
%:%272=178%:%
%:%273=179%:%
%:%274=180%:%
%:%275=181%:%
%:%276=182%:%
%:%277=183%:%
%:%278=184%:%
%:%279=185%:%
%:%280=186%:%
%:%281=187%:%
%:%282=188%:%
%:%283=189%:%
%:%284=190%:%
%:%285=191%:%
%:%286=192%:%
%:%287=193%:%
%:%288=194%:%
%:%289=195%:%
%:%290=196%:%
%:%291=197%:%
%:%292=198%:%
%:%293=199%:%
%:%294=200%:%
%:%295=201%:%
%:%296=202%:%
%:%297=203%:%
%:%298=204%:%
%:%299=205%:%
%:%300=206%:%
%:%301=207%:%
%:%302=208%:%
%:%303=209%:%
%:%304=210%:%
%:%305=211%:%
%:%306=212%:%
%:%307=213%:%
%:%308=214%:%
%:%309=215%:%
%:%310=216%:%
%:%311=217%:%
%:%312=218%:%
%:%313=219%:%
%:%314=220%:%
%:%315=221%:%
%:%316=222%:%
%:%317=223%:%
%:%318=224%:%
%:%319=225%:%
%:%320=226%:%
%:%321=227%:%
%:%322=228%:%
%:%323=229%:%
%:%324=230%:%
%:%325=231%:%
%:%326=232%:%
%:%327=233%:%
%:%328=234%:%
%:%329=235%:%
%:%330=236%:%
%:%332=238%:%
%:%333=238%:%
%:%336=239%:%
%:%340=239%:%
%:%350=241%:%
%:%351=242%:%
%:%352=243%:%
%:%354=245%:%
%:%355=245%:%
%:%356=246%:%
%:%357=247%:%
%:%359=249%:%
%:%360=250%:%
%:%361=251%:%
%:%362=252%:%
%:%363=253%:%
%:%364=254%:%
%:%365=255%:%
%:%366=256%:%
%:%367=257%:%
%:%368=258%:%
%:%369=259%:%
%:%370=260%:%
%:%371=261%:%
%:%372=262%:%
%:%373=263%:%
%:%374=264%:%
%:%375=265%:%
%:%376=266%:%
%:%377=267%:%
%:%378=268%:%
%:%379=269%:%
%:%380=270%:%
%:%381=271%:%
%:%382=272%:%
%:%383=273%:%
%:%384=274%:%
%:%385=275%:%
%:%386=276%:%
%:%387=277%:%
%:%388=278%:%
%:%389=279%:%
%:%390=280%:%
%:%391=281%:%
%:%392=282%:%
%:%393=283%:%
%:%394=284%:%
%:%395=285%:%
%:%396=286%:%
%:%397=287%:%
%:%398=288%:%
%:%399=289%:%
%:%400=290%:%
%:%401=291%:%
%:%403=294%:%
%:%404=294%:%
%:%405=295%:%
%:%412=296%:%
%:%413=296%:%
%:%414=297%:%
%:%415=297%:%
%:%416=298%:%
%:%417=298%:%
%:%418=299%:%
%:%419=299%:%
%:%420=299%:%
%:%421=300%:%
%:%422=300%:%
%:%423=301%:%
%:%424=301%:%
%:%425=301%:%
%:%426=302%:%
%:%427=302%:%
%:%428=303%:%
%:%434=303%:%
%:%437=304%:%
%:%438=305%:%
%:%439=305%:%
%:%440=306%:%
%:%447=307%:%
%:%448=307%:%
%:%449=308%:%
%:%450=308%:%
%:%451=309%:%
%:%452=309%:%
%:%453=310%:%
%:%454=310%:%
%:%455=310%:%
%:%456=311%:%
%:%457=311%:%
%:%458=312%:%
%:%464=312%:%
%:%467=313%:%
%:%468=314%:%
%:%469=314%:%
%:%470=315%:%
%:%471=316%:%
%:%474=317%:%
%:%478=317%:%
%:%479=317%:%
%:%480=318%:%
%:%481=318%:%
%:%486=318%:%
%:%489=319%:%
%:%490=320%:%
%:%491=320%:%
%:%492=321%:%
%:%493=322%:%
%:%494=323%:%
%:%501=324%:%
%:%502=324%:%
%:%503=325%:%
%:%504=325%:%
%:%505=326%:%
%:%506=326%:%
%:%507=327%:%
%:%508=327%:%
%:%509=328%:%
%:%510=328%:%
%:%511=328%:%
%:%512=329%:%
%:%513=329%:%
%:%514=330%:%
%:%520=330%:%
%:%523=331%:%
%:%524=332%:%
%:%525=332%:%
%:%526=333%:%
%:%527=334%:%
%:%528=335%:%
%:%535=336%:%
%:%536=336%:%
%:%537=337%:%
%:%538=337%:%
%:%539=338%:%
%:%540=338%:%
%:%541=339%:%
%:%542=339%:%
%:%543=340%:%
%:%544=340%:%
%:%545=340%:%
%:%546=341%:%
%:%547=341%:%
%:%548=342%:%
%:%554=342%:%
%:%557=343%:%
%:%558=344%:%
%:%559=344%:%
%:%560=345%:%
%:%561=346%:%
%:%562=347%:%
%:%569=348%:%
%:%570=348%:%
%:%571=349%:%
%:%572=349%:%
%:%573=350%:%
%:%574=350%:%
%:%575=351%:%
%:%576=351%:%
%:%577=352%:%
%:%578=352%:%
%:%579=352%:%
%:%580=353%:%
%:%581=353%:%
%:%582=354%:%
%:%588=354%:%
%:%591=355%:%
%:%592=356%:%
%:%593=356%:%
%:%600=357%:%
%:%601=357%:%
%:%602=358%:%
%:%603=358%:%
%:%604=359%:%
%:%605=359%:%
%:%606=359%:%
%:%607=359%:%
%:%608=360%:%
%:%609=360%:%
%:%610=361%:%
%:%611=361%:%
%:%612=362%:%
%:%613=362%:%
%:%614=362%:%
%:%615=362%:%
%:%616=363%:%
%:%617=363%:%
%:%618=364%:%
%:%619=364%:%
%:%620=365%:%
%:%621=365%:%
%:%622=365%:%
%:%623=365%:%
%:%624=366%:%
%:%625=366%:%
%:%626=367%:%
%:%627=367%:%
%:%628=368%:%
%:%629=368%:%
%:%630=368%:%
%:%631=368%:%
%:%632=369%:%
%:%633=369%:%
%:%634=370%:%
%:%635=370%:%
%:%636=371%:%
%:%637=371%:%
%:%638=371%:%
%:%639=371%:%
%:%640=372%:%
%:%641=372%:%
%:%642=373%:%
%:%643=373%:%
%:%644=374%:%
%:%645=374%:%
%:%646=374%:%
%:%647=374%:%
%:%648=375%:%
%:%658=377%:%
%:%660=379%:%
%:%661=379%:%
%:%664=380%:%
%:%668=380%:%
%:%669=380%:%
%:%683=382%:%
%:%695=384%:%
%:%696=385%:%
%:%697=386%:%
%:%698=387%:%
%:%699=388%:%
%:%700=389%:%
%:%701=390%:%
%:%702=391%:%
%:%703=392%:%
%:%704=393%:%
%:%705=394%:%
%:%706=395%:%
%:%707=396%:%
%:%708=397%:%
%:%709=398%:%
%:%710=399%:%
%:%711=400%:%
%:%712=401%:%
%:%714=403%:%
%:%715=403%:%
%:%716=404%:%
%:%717=405%:%
%:%718=406%:%
%:%719=407%:%
%:%720=408%:%
%:%721=409%:%
%:%723=411%:%
%:%724=412%:%
%:%725=413%:%
%:%726=414%:%
%:%727=415%:%
%:%729=417%:%
%:%730=417%:%
%:%731=418%:%
%:%734=419%:%
%:%738=419%:%
%:%739=419%:%
%:%740=420%:%
%:%741=421%:%
%:%742=421%:%
%:%743=422%:%
%:%744=422%:%
%:%745=423%:%
%:%746=424%:%
%:%747=424%:%
%:%748=425%:%
%:%749=425%:%
%:%750=426%:%
%:%751=427%:%
%:%752=427%:%
%:%754=429%:%
%:%755=430%:%
%:%756=430%:%
%:%757=431%:%
%:%758=432%:%
%:%759=432%:%
%:%760=433%:%
%:%761=433%:%
%:%762=434%:%
%:%763=435%:%
%:%764=435%:%
%:%765=436%:%
%:%766=437%:%
%:%767=437%:%
%:%772=437%:%
%:%775=438%:%
%:%778=440%:%
%:%779=441%:%
%:%780=442%:%
%:%782=444%:%
%:%783=444%:%
%:%784=445%:%
%:%786=447%:%
%:%787=448%:%
%:%788=449%:%
%:%789=450%:%
%:%790=451%:%
%:%791=452%:%
%:%792=453%:%
%:%794=455%:%
%:%795=455%:%
%:%796=456%:%
%:%799=457%:%
%:%803=457%:%
%:%804=457%:%
%:%805=458%:%
%:%806=459%:%
%:%807=459%:%
%:%808=460%:%
%:%809=460%:%
%:%810=461%:%
%:%811=462%:%
%:%812=462%:%
%:%814=464%:%
%:%815=465%:%
%:%816=465%:%
%:%821=465%:%
%:%824=466%:%
%:%827=468%:%
%:%828=469%:%
%:%829=470%:%
%:%830=471%:%
%:%831=472%:%
%:%832=473%:%
%:%833=474%:%
%:%834=475%:%
%:%835=476%:%
%:%836=477%:%
%:%837=478%:%
%:%838=479%:%
%:%840=481%:%
%:%841=481%:%
%:%844=482%:%
%:%848=482%:%
%:%849=482%:%
%:%858=484%:%
%:%859=485%:%
%:%861=487%:%
%:%862=487%:%
%:%863=488%:%
%:%866=489%:%
%:%870=489%:%
%:%871=489%:%
%:%876=489%:%
%:%879=490%:%
%:%880=491%:%
%:%881=491%:%
%:%882=492%:%
%:%885=493%:%
%:%889=493%:%
%:%890=493%:%
%:%895=493%:%
%:%898=494%:%
%:%899=495%:%
%:%900=495%:%
%:%901=496%:%
%:%908=497%:%
%:%909=497%:%
%:%910=498%:%
%:%911=498%:%
%:%912=499%:%
%:%913=499%:%
%:%914=500%:%
%:%915=500%:%
%:%916=500%:%
%:%917=501%:%
%:%918=501%:%
%:%919=502%:%
%:%920=502%:%
%:%921=502%:%
%:%922=503%:%
%:%923=503%:%
%:%924=504%:%
%:%930=504%:%
%:%933=505%:%
%:%934=506%:%
%:%935=506%:%
%:%936=507%:%
%:%937=508%:%
%:%944=509%:%
%:%945=509%:%
%:%946=510%:%
%:%947=510%:%
%:%948=511%:%
%:%949=512%:%
%:%950=512%:%
%:%951=513%:%
%:%952=513%:%
%:%953=513%:%
%:%954=514%:%
%:%955=514%:%
%:%956=515%:%
%:%957=515%:%
%:%958=515%:%
%:%959=516%:%
%:%960=516%:%
%:%961=517%:%
%:%962=517%:%
%:%963=517%:%
%:%964=518%:%
%:%965=518%:%
%:%966=519%:%
%:%972=519%:%
%:%975=520%:%
%:%976=521%:%
%:%977=521%:%
%:%978=522%:%
%:%979=523%:%
%:%986=524%:%
%:%987=524%:%
%:%988=525%:%
%:%989=525%:%
%:%990=526%:%
%:%991=527%:%
%:%992=527%:%
%:%993=528%:%
%:%994=528%:%
%:%995=528%:%
%:%996=529%:%
%:%997=529%:%
%:%998=530%:%
%:%999=530%:%
%:%1000=530%:%
%:%1001=531%:%
%:%1002=531%:%
%:%1003=532%:%
%:%1004=532%:%
%:%1005=532%:%
%:%1006=533%:%
%:%1007=533%:%
%:%1008=534%:%
%:%1014=534%:%
%:%1017=535%:%
%:%1018=536%:%
%:%1019=536%:%
%:%1020=537%:%
%:%1021=538%:%
%:%1028=539%:%
%:%1029=539%:%
%:%1030=540%:%
%:%1031=540%:%
%:%1032=541%:%
%:%1033=542%:%
%:%1034=542%:%
%:%1035=543%:%
%:%1036=543%:%
%:%1037=543%:%
%:%1038=544%:%
%:%1039=544%:%
%:%1040=545%:%
%:%1041=545%:%
%:%1042=545%:%
%:%1043=546%:%
%:%1044=546%:%
%:%1045=547%:%
%:%1046=547%:%
%:%1047=547%:%
%:%1048=548%:%
%:%1049=548%:%
%:%1050=549%:%
%:%1060=551%:%
%:%1061=552%:%
%:%1062=553%:%
%:%1063=554%:%
%:%1064=555%:%
%:%1065=556%:%
%:%1066=557%:%
%:%1067=558%:%
%:%1068=559%:%
%:%1069=560%:%
%:%1070=561%:%
%:%1071=562%:%
%:%1072=563%:%
%:%1073=564%:%
%:%1074=565%:%
%:%1075=566%:%
%:%1076=567%:%
%:%1077=568%:%
%:%1078=569%:%
%:%1079=570%:%
%:%1080=571%:%
%:%1081=572%:%
%:%1082=573%:%
%:%1083=574%:%
%:%1084=575%:%
%:%1085=576%:%
%:%1086=577%:%
%:%1087=578%:%
%:%1088=579%:%
%:%1089=580%:%
%:%1090=581%:%
%:%1092=583%:%
%:%1092=584%:%
%:%1093=585%:%
%:%1094=585%:%
%:%1101=586%:%
%:%1102=586%:%
%:%1103=587%:%
%:%1104=587%:%
%:%1105=588%:%
%:%1106=588%:%
%:%1107=588%:%
%:%1108=589%:%
%:%1109=589%:%
%:%1110=590%:%
%:%1111=590%:%
%:%1112=591%:%
%:%1113=591%:%
%:%1114=592%:%
%:%1115=592%:%
%:%1116=592%:%
%:%1117=593%:%
%:%1118=593%:%
%:%1119=594%:%
%:%1120=594%:%
%:%1121=595%:%
%:%1122=595%:%
%:%1123=596%:%
%:%1124=596%:%
%:%1125=596%:%
%:%1126=597%:%
%:%1127=597%:%
%:%1128=598%:%
%:%1129=598%:%
%:%1130=599%:%
%:%1131=599%:%
%:%1132=600%:%
%:%1133=600%:%
%:%1134=600%:%
%:%1135=601%:%
%:%1136=601%:%
%:%1137=602%:%
%:%1138=602%:%
%:%1139=603%:%
%:%1140=603%:%
%:%1141=604%:%
%:%1142=604%:%
%:%1143=604%:%
%:%1144=605%:%
%:%1145=605%:%
%:%1146=606%:%
%:%1147=606%:%
%:%1148=607%:%
%:%1149=607%:%
%:%1150=608%:%
%:%1151=608%:%
%:%1152=608%:%
%:%1153=609%:%
%:%1154=609%:%
%:%1155=610%:%
%:%1165=612%:%
%:%1167=614%:%
%:%1168=614%:%
%:%1171=615%:%
%:%1175=615%:%
%:%1176=615%:%
%:%1185=617%:%
%:%1186=618%:%
%:%1187=619%:%
%:%1188=620%:%
%:%1189=621%:%
%:%1190=622%:%
%:%1191=623%:%
%:%1192=624%:%
%:%1194=626%:%
%:%1194=627%:%
%:%1195=628%:%
%:%1196=628%:%
%:%1197=629%:%
%:%1198=630%:%
%:%1205=631%:%
%:%1206=631%:%
%:%1207=632%:%
%:%1208=632%:%
%:%1209=633%:%
%:%1210=633%:%
%:%1211=634%:%
%:%1212=634%:%
%:%1213=635%:%
%:%1214=635%:%
%:%1215=635%:%
%:%1216=636%:%
%:%1217=636%:%
%:%1218=637%:%
%:%1224=637%:%
%:%1227=638%:%
%:%1228=639%:%
%:%1229=639%:%
%:%1230=640%:%
%:%1231=641%:%
%:%1238=642%:%
%:%1239=642%:%
%:%1240=643%:%
%:%1241=643%:%
%:%1242=644%:%
%:%1243=644%:%
%:%1244=645%:%
%:%1245=645%:%
%:%1246=646%:%
%:%1247=646%:%
%:%1248=646%:%
%:%1249=647%:%
%:%1250=647%:%
%:%1251=648%:%
%:%1261=650%:%
%:%1262=651%:%
%:%1264=653%:%
%:%1265=653%:%
%:%1268=654%:%
%:%1272=654%:%
%:%1273=654%:%
%:%1278=654%:%
%:%1281=655%:%
%:%1282=656%:%
%:%1283=656%:%
%:%1286=657%:%
%:%1290=657%:%
%:%1291=657%:%
%:%1300=659%:%
%:%1301=660%:%
%:%1302=661%:%
%:%1303=662%:%
%:%1304=663%:%
%:%1305=664%:%
%:%1306=665%:%
%:%1307=666%:%
%:%1308=667%:%
%:%1309=668%:%
%:%1310=669%:%
%:%1311=670%:%
%:%1312=671%:%
%:%1313=672%:%
%:%1314=673%:%
%:%1315=674%:%
%:%1316=675%:%
%:%1317=676%:%
%:%1318=677%:%
%:%1319=678%:%
%:%1320=679%:%
%:%1321=680%:%
%:%1322=681%:%
%:%1323=682%:%
%:%1324=683%:%
%:%1325=684%:%
%:%1326=685%:%
%:%1327=686%:%
%:%1328=687%:%
%:%1329=688%:%
%:%1330=689%:%
%:%1331=690%:%
%:%1332=691%:%
%:%1333=692%:%
%:%1334=693%:%
%:%1335=694%:%
%:%1336=695%:%
%:%1337=696%:%
%:%1338=697%:%
%:%1339=698%:%
%:%1339=699%:%
%:%1340=700%:%
%:%1341=701%:%
%:%1342=702%:%
%:%1343=703%:%
%:%1345=705%:%
%:%1346=705%:%
%:%1349=706%:%
%:%1353=706%:%
%:%1363=708%:%
%:%1364=709%:%
%:%1365=710%:%
%:%1366=711%:%
%:%1367=712%:%
%:%1368=713%:%
%:%1369=714%:%
%:%1370=715%:%
%:%1371=716%:%
%:%1372=717%:%
%:%1373=718%:%
%:%1374=719%:%
%:%1375=720%:%
%:%1377=722%:%
%:%1378=722%:%
%:%1379=723%:%
%:%1382=724%:%
%:%1386=724%:%
%:%1387=724%:%
%:%1388=725%:%
%:%1389=726%:%
%:%1390=726%:%
%:%1391=727%:%
%:%1392=727%:%
%:%1393=728%:%
%:%1394=729%:%
%:%1395=729%:%
%:%1396=730%:%
%:%1397=731%:%
%:%1398=731%:%
%:%1399=732%:%
%:%1400=733%:%
%:%1401=733%:%
%:%1402=734%:%
%:%1403=734%:%
%:%1408=734%:%
%:%1411=735%:%
%:%1414=737%:%
%:%1415=738%:%
%:%1416=739%:%
%:%1417=740%:%
%:%1418=741%:%
%:%1419=742%:%
%:%1420=743%:%
%:%1421=744%:%
%:%1422=745%:%
%:%1423=746%:%
%:%1424=747%:%
%:%1425=748%:%
%:%1426=749%:%
%:%1427=750%:%
%:%1428=751%:%
%:%1429=752%:%
%:%1430=753%:%
%:%1431=754%:%
%:%1432=755%:%
%:%1433=756%:%
%:%1435=758%:%
%:%1436=758%:%
%:%1437=759%:%
%:%1444=760%:%
%:%1445=760%:%
%:%1446=761%:%
%:%1447=761%:%
%:%1448=762%:%
%:%1449=762%:%
%:%1450=763%:%
%:%1451=763%:%
%:%1452=763%:%
%:%1453=764%:%
%:%1454=764%:%
%:%1455=765%:%
%:%1456=765%:%
%:%1457=765%:%
%:%1458=766%:%
%:%1459=766%:%
%:%1460=767%:%
%:%1461=767%:%
%:%1462=767%:%
%:%1463=768%:%
%:%1464=768%:%
%:%1465=769%:%
%:%1466=769%:%
%:%1467=769%:%
%:%1468=770%:%
%:%1469=770%:%
%:%1470=771%:%
%:%1471=771%:%
%:%1472=771%:%
%:%1473=772%:%
%:%1474=772%:%
%:%1475=773%:%
%:%1481=773%:%
%:%1484=774%:%
%:%1485=775%:%
%:%1486=775%:%
%:%1487=776%:%
%:%1494=777%:%
%:%1495=777%:%
%:%1496=778%:%
%:%1497=778%:%
%:%1498=779%:%
%:%1499=779%:%
%:%1500=780%:%
%:%1501=780%:%
%:%1502=780%:%
%:%1503=781%:%
%:%1504=781%:%
%:%1505=782%:%
%:%1506=782%:%
%:%1507=782%:%
%:%1508=783%:%
%:%1509=783%:%
%:%1510=784%:%
%:%1511=784%:%
%:%1512=784%:%
%:%1513=785%:%
%:%1514=785%:%
%:%1515=786%:%
%:%1521=786%:%
%:%1524=787%:%
%:%1525=788%:%
%:%1526=788%:%
%:%1527=789%:%
%:%1528=790%:%
%:%1535=791%:%
%:%1536=791%:%
%:%1537=792%:%
%:%1538=792%:%
%:%1539=793%:%
%:%1540=793%:%
%:%1541=794%:%
%:%1542=794%:%
%:%1543=794%:%
%:%1544=795%:%
%:%1545=795%:%
%:%1546=796%:%
%:%1547=796%:%
%:%1548=796%:%
%:%1549=797%:%
%:%1550=797%:%
%:%1551=798%:%
%:%1552=798%:%
%:%1553=798%:%
%:%1554=799%:%
%:%1555=799%:%
%:%1556=800%:%
%:%1557=800%:%
%:%1558=800%:%
%:%1559=801%:%
%:%1560=801%:%
%:%1561=802%:%
%:%1567=802%:%
%:%1570=803%:%
%:%1571=804%:%
%:%1572=804%:%
%:%1573=805%:%
%:%1574=806%:%
%:%1575=807%:%
%:%1582=808%:%
%:%1583=808%:%
%:%1584=809%:%
%:%1585=809%:%
%:%1586=810%:%
%:%1587=810%:%
%:%1588=811%:%
%:%1589=811%:%
%:%1590=811%:%
%:%1591=812%:%
%:%1592=812%:%
%:%1593=813%:%
%:%1594=813%:%
%:%1595=813%:%
%:%1596=814%:%
%:%1597=814%:%
%:%1598=815%:%
%:%1599=815%:%
%:%1600=816%:%
%:%1601=816%:%
%:%1602=816%:%
%:%1603=817%:%
%:%1604=817%:%
%:%1605=818%:%
%:%1606=818%:%
%:%1607=818%:%
%:%1608=819%:%
%:%1609=819%:%
%:%1610=820%:%
%:%1611=820%:%
%:%1612=820%:%
%:%1613=821%:%
%:%1614=821%:%
%:%1615=822%:%
%:%1621=822%:%
%:%1624=823%:%
%:%1625=824%:%
%:%1626=824%:%
%:%1627=825%:%
%:%1634=826%:%
%:%1635=826%:%
%:%1636=827%:%
%:%1637=827%:%
%:%1638=828%:%
%:%1639=828%:%
%:%1640=828%:%
%:%1641=828%:%
%:%1642=829%:%
%:%1643=829%:%
%:%1644=830%:%
%:%1645=830%:%
%:%1646=831%:%
%:%1647=831%:%
%:%1648=831%:%
%:%1649=831%:%
%:%1650=832%:%
%:%1651=832%:%
%:%1652=833%:%
%:%1653=833%:%
%:%1654=834%:%
%:%1655=834%:%
%:%1656=834%:%
%:%1657=834%:%
%:%1658=835%:%
%:%1659=835%:%
%:%1660=836%:%
%:%1661=836%:%
%:%1662=837%:%
%:%1663=837%:%
%:%1664=837%:%
%:%1665=837%:%
%:%1666=838%:%
%:%1667=838%:%
%:%1668=839%:%
%:%1669=839%:%
%:%1670=840%:%
%:%1671=840%:%
%:%1672=840%:%
%:%1673=840%:%
%:%1674=841%:%
%:%1675=841%:%
%:%1676=842%:%
%:%1677=842%:%
%:%1678=843%:%
%:%1679=843%:%
%:%1680=843%:%
%:%1681=843%:%
%:%1682=844%:%
%:%1692=846%:%
%:%1693=847%:%
%:%1695=849%:%
%:%1696=849%:%
%:%1699=850%:%
%:%1703=850%:%
%:%1704=850%:%
%:%1713=852%:%
%:%1714=853%:%
%:%1715=854%:%
%:%1716=855%:%
%:%1717=856%:%
%:%1718=857%:%
%:%1719=858%:%
%:%1720=859%:%
%:%1721=860%:%
%:%1722=861%:%
%:%1723=862%:%
%:%1724=863%:%
%:%1725=864%:%
%:%1726=865%:%
%:%1727=866%:%
%:%1728=867%:%
%:%1729=868%:%
%:%1730=869%:%
%:%1731=870%:%
%:%1732=871%:%
%:%1733=872%:%
%:%1734=873%:%
%:%1735=874%:%
%:%1736=875%:%
%:%1737=876%:%
%:%1738=877%:%
%:%1739=878%:%
%:%1740=879%:%
%:%1741=880%:%
%:%1742=881%:%
%:%1743=882%:%
%:%1744=883%:%
%:%1745=884%:%
%:%1746=885%:%
%:%1747=886%:%
%:%1748=887%:%
%:%1749=888%:%
%:%1750=889%:%
%:%1751=890%:%
%:%1752=891%:%
%:%1753=892%:%
%:%1754=893%:%
%:%1755=894%:%
%:%1756=895%:%
%:%1757=896%:%
%:%1758=897%:%
%:%1759=898%:%
%:%1760=899%:%
%:%1761=900%:%
%:%1762=901%:%
%:%1763=902%:%
%:%1764=903%:%
%:%1765=904%:%
%:%1767=906%:%
%:%1768=906%:%
%:%1771=907%:%
%:%1775=907%:%
%:%1785=910%:%
%:%1786=911%:%
%:%1788=913%:%
%:%1789=913%:%
%:%1790=914%:%
%:%1791=915%:%
%:%1798=916%:%
%:%1799=916%:%
%:%1800=917%:%
%:%1801=917%:%
%:%1802=918%:%
%:%1803=918%:%
%:%1804=919%:%
%:%1805=919%:%
%:%1806=920%:%
%:%1807=920%:%
%:%1808=920%:%
%:%1809=921%:%
%:%1810=921%:%
%:%1811=922%:%
%:%1812=922%:%
%:%1813=922%:%
%:%1814=923%:%
%:%1815=923%:%
%:%1816=924%:%
%:%1822=924%:%
%:%1825=925%:%
%:%1826=926%:%
%:%1827=926%:%
%:%1828=927%:%
%:%1829=928%:%
%:%1836=929%:%
%:%1837=929%:%
%:%1838=930%:%
%:%1839=930%:%
%:%1840=931%:%
%:%1841=931%:%
%:%1842=932%:%
%:%1843=932%:%
%:%1844=933%:%
%:%1845=933%:%
%:%1846=933%:%
%:%1847=934%:%
%:%1848=934%:%
%:%1849=935%:%
%:%1850=935%:%
%:%1851=935%:%
%:%1852=936%:%
%:%1853=936%:%
%:%1854=937%:%
%:%1860=937%:%
%:%1863=938%:%
%:%1864=939%:%
%:%1865=939%:%
%:%1866=940%:%
%:%1867=941%:%
%:%1868=942%:%
%:%1875=943%:%
%:%1876=943%:%
%:%1877=944%:%
%:%1878=944%:%
%:%1879=945%:%
%:%1880=945%:%
%:%1881=946%:%
%:%1882=946%:%
%:%1883=947%:%
%:%1884=947%:%
%:%1885=947%:%
%:%1886=948%:%
%:%1887=948%:%
%:%1888=949%:%
%:%1889=949%:%
%:%1890=949%:%
%:%1891=950%:%
%:%1892=950%:%
%:%1893=951%:%
%:%1894=951%:%
%:%1895=952%:%
%:%1896=952%:%
%:%1897=952%:%
%:%1898=953%:%
%:%1899=953%:%
%:%1900=954%:%
%:%1901=954%:%
%:%1902=954%:%
%:%1903=955%:%
%:%1904=955%:%
%:%1905=956%:%
%:%1906=956%:%
%:%1907=957%:%
%:%1908=957%:%
%:%1909=958%:%
%:%1910=958%:%
%:%1911=958%:%
%:%1912=959%:%
%:%1913=959%:%
%:%1914=960%:%
%:%1915=960%:%
%:%1916=960%:%
%:%1917=961%:%
%:%1918=961%:%
%:%1919=962%:%
%:%1920=962%:%
%:%1921=962%:%
%:%1922=963%:%
%:%1923=963%:%
%:%1924=964%:%
%:%1925=964%:%
%:%1926=965%:%
%:%1932=965%:%
%:%1935=966%:%
%:%1936=967%:%
%:%1937=967%:%
%:%1938=968%:%
%:%1939=969%:%
%:%1940=970%:%
%:%1941=971%:%
%:%1948=972%:%
%:%1949=972%:%
%:%1950=973%:%
%:%1951=973%:%
%:%1952=974%:%
%:%1953=974%:%
%:%1954=975%:%
%:%1955=975%:%
%:%1956=976%:%
%:%1957=976%:%
%:%1958=976%:%
%:%1959=977%:%
%:%1960=977%:%
%:%1961=978%:%
%:%1962=978%:%
%:%1963=978%:%
%:%1964=979%:%
%:%1965=979%:%
%:%1966=980%:%
%:%1967=980%:%
%:%1968=981%:%
%:%1969=981%:%
%:%1970=981%:%
%:%1971=982%:%
%:%1972=982%:%
%:%1973=983%:%
%:%1974=983%:%
%:%1975=983%:%
%:%1976=984%:%
%:%1977=984%:%
%:%1978=985%:%
%:%1979=985%:%
%:%1980=986%:%
%:%1981=986%:%
%:%1982=987%:%
%:%1983=987%:%
%:%1984=987%:%
%:%1985=988%:%
%:%1986=988%:%
%:%1987=989%:%
%:%1988=989%:%
%:%1989=989%:%
%:%1990=990%:%
%:%1991=990%:%
%:%1992=991%:%
%:%1993=991%:%
%:%1994=992%:%
%:%1995=992%:%
%:%1996=992%:%
%:%1997=993%:%
%:%1998=993%:%
%:%1999=994%:%
%:%2000=994%:%
%:%2001=994%:%
%:%2002=995%:%
%:%2003=995%:%
%:%2004=996%:%
%:%2005=996%:%
%:%2006=996%:%
%:%2007=997%:%
%:%2008=997%:%
%:%2009=998%:%
%:%2010=998%:%
%:%2011=998%:%
%:%2012=999%:%
%:%2013=999%:%
%:%2014=1000%:%
%:%2015=1000%:%
%:%2016=1001%:%
%:%2017=1001%:%
%:%2018=1002%:%
%:%2019=1002%:%
%:%2020=1002%:%
%:%2021=1003%:%
%:%2022=1003%:%
%:%2023=1004%:%
%:%2024=1004%:%
%:%2025=1004%:%
%:%2026=1005%:%
%:%2027=1005%:%
%:%2028=1006%:%
%:%2029=1006%:%
%:%2030=1006%:%
%:%2031=1007%:%
%:%2032=1007%:%
%:%2033=1008%:%
%:%2034=1008%:%
%:%2035=1008%:%
%:%2036=1009%:%
%:%2037=1009%:%
%:%2038=1010%:%
%:%2039=1010%:%
%:%2040=1011%:%
%:%2041=1011%:%
%:%2042=1012%:%
%:%2048=1012%:%
%:%2051=1013%:%
%:%2052=1014%:%
%:%2053=1014%:%
%:%2054=1015%:%
%:%2061=1016%:%
%:%2062=1016%:%
%:%2063=1017%:%
%:%2064=1017%:%
%:%2065=1018%:%
%:%2066=1018%:%
%:%2067=1018%:%
%:%2068=1018%:%
%:%2069=1019%:%
%:%2070=1019%:%
%:%2071=1020%:%
%:%2072=1020%:%
%:%2073=1021%:%
%:%2074=1021%:%
%:%2075=1021%:%
%:%2076=1021%:%
%:%2077=1022%:%
%:%2078=1022%:%
%:%2079=1023%:%
%:%2080=1023%:%
%:%2081=1024%:%
%:%2082=1024%:%
%:%2083=1024%:%
%:%2084=1024%:%
%:%2085=1025%:%
%:%2086=1025%:%
%:%2087=1026%:%
%:%2088=1026%:%
%:%2089=1027%:%
%:%2090=1027%:%
%:%2091=1027%:%
%:%2092=1027%:%
%:%2093=1028%:%
%:%2094=1028%:%
%:%2095=1029%:%
%:%2096=1029%:%
%:%2097=1030%:%
%:%2098=1030%:%
%:%2099=1030%:%
%:%2100=1030%:%
%:%2101=1031%:%
%:%2102=1031%:%
%:%2103=1032%:%
%:%2104=1032%:%
%:%2105=1033%:%
%:%2106=1033%:%
%:%2107=1033%:%
%:%2108=1033%:%
%:%2109=1034%:%
%:%2119=1036%:%
%:%2121=1038%:%
%:%2122=1038%:%
%:%2125=1039%:%
%:%2129=1039%:%
%:%2130=1039%:%
%:%2139=1041%:%