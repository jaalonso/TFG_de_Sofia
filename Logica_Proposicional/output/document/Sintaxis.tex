%
\begin{isabellebody}%
\setisabellecontext{Sintaxis}%
%
\isadelimtheory
%
\endisadelimtheory
%
\isatagtheory
%
\endisatagtheory
{\isafoldtheory}%
%
\isadelimtheory
%
\endisadelimtheory
%
\isadelimdocument
%
\endisadelimdocument
%
\isatagdocument
%
\isamarkupsection{Fórmulas%
}
\isamarkuptrue%
%
\endisatagdocument
{\isafolddocument}%
%
\isadelimdocument
%
\endisadelimdocument
%
\begin{isamarkuptext}%
\comentario{Explicar la siguiente notación y recolocarla donde se
  use por primera vez.}

  \comentario{He quitado la palabra "continuación" del fichero 
  castellano.tex ya que no dejaba cargar el documento}%
\end{isamarkuptext}\isamarkuptrue%
\isacommand{notation}\isamarkupfalse%
\ insert\ {\isacharparenleft}{\isachardoublequoteopen}{\isacharunderscore}\ {\isasymtriangleright}\ {\isacharunderscore}{\isachardoublequoteclose}\ {\isacharbrackleft}{\isadigit{5}}{\isadigit{6}}{\isacharcomma}{\isadigit{5}}{\isadigit{5}}{\isacharbrackright}\ {\isadigit{5}}{\isadigit{5}}{\isacharparenright}%
\begin{isamarkuptext}%
En esta sección presentaremos una formalización en Isabelle de la 
  sintaxis de la lógica proposicional, junto con resultados y pruebas 
  sobre la misma. En líneas generales, primero daremos las nociones de 
  forma clásica y, a continuación, su correspondiente formalización.

  En primer lugar, supondremos que disponemos de los siguientes 
  elementos:
  \begin{description}
    \item[Alfabeto:] Es una lista infinita de variables proposicionales. 
      También pueden ser llamadas átomos o símbolos proposicionales.
    \item[Conectivas:] Conjunto finito cuyos elementos interactúan con 
      las variables. Pueden ser monarias que afectan a un único elemento 
      o binarias que afectan a dos. En el primer grupo se encuentra le 
      negación (\isa{{\isasymnot}}) y en el segundo la conjunción (\isa{{\isasymand}}), la disyunción 
      (\isa{{\isasymor}}) y la implicación (\isa{{\isasymlongrightarrow}}).
  \end{description}

  A continuación definiremos la estructura de fórmula sobre los 
  elementos anteriores. Para ello daremos una definición recursiva 
  basada en dos elementos: un conjunto de fórmulas básicas y una serie 
  de procedimientos de definición de fórmulas a partir de otras. El 
  conjunto de las fórmulas será el menor conjunto de estructuras 
  sinctáticas con dicho alfabeto y conectivas que contiene a las básicas 
  y es cerrado mediante los procedimientos de definición que mostraremos 
  a continuación.

  \begin{definicion}
    El conjunto de las fórmulas proposicionales está formado por las 
    siguientes:
    \begin{itemize}
      \item Las fórmulas atómicas, constituidas únicamente por una 
        variable del alfabeto. 
      \item La constante \isa{{\isasymbottom}}.
      \item Dada una fórmula \isa{F}, la negación \isa{{\isasymnot}\ F} es una fórmula.
      \item Dadas dos fórmulas \isa{F} y \isa{G}, la conjunción \isa{F\ {\isasymand}\ G} es una
        fórmula.
      \item Dadas dos fórmulas \isa{F} y \isa{G}, la disyunción \isa{F\ {\isasymor}\ G} es una
        fórmula.
      \item Dadas dos fórmulas \isa{F} y \isa{G}, la implicación \isa{F\ {\isasymlongrightarrow}\ G} es 
        una fórmula.
    \end{itemize}
  La demostración estructurada se hace de manera sencilla con el 
  resultado introducido anteriormente. 

  La demostración estructurada se hace de manera sencilla con el 
  resultado introducido anteriormente. 

  \end{definicion}

  Intuitivamente, las fórmulas proposicionales son entendidas como un 
  tipo de árbol sintáctico cuyos nodos son las conectivas y sus hojas
  las fórmulas atómicas.

  \comentario{Incluir el árbol de formación.}

  A continuación, veamos su representación en Isabelle%
\end{isamarkuptext}\isamarkuptrue%
\isacommand{datatype}\isamarkupfalse%
\ {\isacharparenleft}atoms{\isacharcolon}\ {\isacharprime}a{\isacharparenright}\ formula\ {\isacharequal}\ \isanewline
\ \ Atom\ {\isacharprime}a\isanewline
{\isacharbar}\ Bot\ \ \ \ \ \ \ \ \ \ \ \ \ \ \ \ \ \ \ \ \ \ \ \ \ \ \ \ \ \ {\isacharparenleft}{\isachardoublequoteopen}{\isasymbottom}{\isachardoublequoteclose}{\isacharparenright}\ \ \isanewline
{\isacharbar}\ Not\ {\isachardoublequoteopen}{\isacharprime}a\ formula{\isachardoublequoteclose}\ \ \ \ \ \ \ \ \ \ \ \ \ \ \ \ \ {\isacharparenleft}{\isachardoublequoteopen}\isactrlbold {\isasymnot}{\isachardoublequoteclose}{\isacharparenright}\isanewline
{\isacharbar}\ And\ {\isachardoublequoteopen}{\isacharprime}a\ formula{\isachardoublequoteclose}\ {\isachardoublequoteopen}{\isacharprime}a\ formula{\isachardoublequoteclose}\ \ \ \ {\isacharparenleft}\isakeyword{infix}\ {\isachardoublequoteopen}\isactrlbold {\isasymand}{\isachardoublequoteclose}\ {\isadigit{6}}{\isadigit{8}}{\isacharparenright}\isanewline
{\isacharbar}\ Or\ {\isachardoublequoteopen}{\isacharprime}a\ formula{\isachardoublequoteclose}\ {\isachardoublequoteopen}{\isacharprime}a\ formula{\isachardoublequoteclose}\ \ \ \ \ {\isacharparenleft}\isakeyword{infix}\ {\isachardoublequoteopen}\isactrlbold {\isasymor}{\isachardoublequoteclose}\ {\isadigit{6}}{\isadigit{8}}{\isacharparenright}\isanewline
{\isacharbar}\ Imp\ {\isachardoublequoteopen}{\isacharprime}a\ formula{\isachardoublequoteclose}\ {\isachardoublequoteopen}{\isacharprime}a\ formula{\isachardoublequoteclose}\ \ \ \ {\isacharparenleft}\isakeyword{infixr}\ {\isachardoublequoteopen}\isactrlbold {\isasymrightarrow}{\isachardoublequoteclose}\ {\isadigit{6}}{\isadigit{8}}{\isacharparenright}%
\begin{isamarkuptext}%
Como podemos observar representamos las fórmulas proposicionales
  mediante un tipo de dato recursivo, \isa{formula}, con los 
  siguientes constructures sobre un tipo \isa{{\isacharprime}a} cualquiera:

  \begin{description}
    \item[Fórmulas básicas:]  
      \begin{itemize}
        \item \isa{Atom\ {\isacharcolon}{\isacharcolon}\ {\isacharprime}a\ {\isasymRightarrow}\ {\isacharprime}a\ formula}
        \item \isa{{\isasymbottom}\ {\isacharcolon}{\isacharcolon}\ {\isacharprime}a\ formula}
      \end{itemize}
    \item [Fórmulas compuestas:]
      \begin{itemize}
        \item \isa{\isactrlbold {\isasymnot}\ {\isacharcolon}{\isacharcolon}\ {\isacharprime}a\ formula\ {\isasymRightarrow}\ {\isacharprime}a\ formula}
        \item \isa{{\isacharparenleft}\isactrlbold {\isasymand}{\isacharparenright}\ {\isacharcolon}{\isacharcolon}\ {\isacharprime}a\ formula\ {\isasymRightarrow}\ {\isacharprime}a\ formula\ {\isasymRightarrow}\ {\isacharprime}a\ formula}
        \item \isa{{\isacharparenleft}\isactrlbold {\isasymor}{\isacharparenright}\ {\isacharcolon}{\isacharcolon}\ {\isacharprime}a\ formula\ {\isasymRightarrow}\ {\isacharprime}a\ formula\ {\isasymRightarrow}\ {\isacharprime}a\ formula}
        \item \isa{{\isacharparenleft}\isactrlbold {\isasymrightarrow}{\isacharparenright}\ {\isacharcolon}{\isacharcolon}\ {\isacharprime}a\ formula\ {\isasymRightarrow}\ {\isacharprime}a\ formula\ {\isasymRightarrow}\ {\isacharprime}a\ formula}
      \end{itemize}
  \end{description}

  Cabe señalar que los términos \isa{infix} e \isa{infixr} nos señalan que 
  los constructores que representan a las conectivas se pueden usar de
  forma infija. En particular, \isa{infixr} se trata de un infijo asociado a 
  la derecha.

  Además se define simultáneamente la función \isa{atoms\ {\isacharcolon}{\isacharcolon}\ {\isacharprime}a\ formula\ {\isasymRightarrow}\ {\isacharprime}a\ set}, que 
  obtiene el conjunto de variables proposicionales de una fórmula. 

  Por otro lado, la definición de \isa{formula} genera 
  automáticamente los siguientes lemas sobre la función de conjuntos 
  \isa{atoms} en Isabelle.
  
  \begin{itemize}
    \item[] \isa{atoms\ {\isacharparenleft}Atom\ x{\isadigit{1}}{\isachardot}{\isadigit{0}}{\isacharparenright}\ {\isacharequal}\ {\isacharbraceleft}x{\isadigit{1}}{\isachardot}{\isadigit{0}}{\isacharbraceright}\isasep\isanewline%
atoms\ {\isasymbottom}\ {\isacharequal}\ {\isasymemptyset}\isasep\isanewline%
atoms\ {\isacharparenleft}\isactrlbold {\isasymnot}\ x{\isadigit{3}}{\isachardot}{\isadigit{0}}{\isacharparenright}\ {\isacharequal}\ atoms\ x{\isadigit{3}}{\isachardot}{\isadigit{0}}\isasep\isanewline%
atoms\ {\isacharparenleft}x{\isadigit{4}}{\isadigit{1}}{\isachardot}{\isadigit{0}}\ \isactrlbold {\isasymand}\ x{\isadigit{4}}{\isadigit{2}}{\isachardot}{\isadigit{0}}{\isacharparenright}\ {\isacharequal}\ atoms\ x{\isadigit{4}}{\isadigit{1}}{\isachardot}{\isadigit{0}}\ {\isasymunion}\ atoms\ x{\isadigit{4}}{\isadigit{2}}{\isachardot}{\isadigit{0}}\isasep\isanewline%
atoms\ {\isacharparenleft}x{\isadigit{5}}{\isadigit{1}}{\isachardot}{\isadigit{0}}\ \isactrlbold {\isasymor}\ x{\isadigit{5}}{\isadigit{2}}{\isachardot}{\isadigit{0}}{\isacharparenright}\ {\isacharequal}\ atoms\ x{\isadigit{5}}{\isadigit{1}}{\isachardot}{\isadigit{0}}\ {\isasymunion}\ atoms\ x{\isadigit{5}}{\isadigit{2}}{\isachardot}{\isadigit{0}}\isasep\isanewline%
atoms\ {\isacharparenleft}x{\isadigit{6}}{\isadigit{1}}{\isachardot}{\isadigit{0}}\ \isactrlbold {\isasymrightarrow}\ x{\isadigit{6}}{\isadigit{2}}{\isachardot}{\isadigit{0}}{\isacharparenright}\ {\isacharequal}\ atoms\ x{\isadigit{6}}{\isadigit{1}}{\isachardot}{\isadigit{0}}\ {\isasymunion}\ atoms\ x{\isadigit{6}}{\isadigit{2}}{\isachardot}{\isadigit{0}}}
  \end{itemize} 

  A continuación veremos varios ejemplos de fórmulas y el conjunto de 
  sus variables proposicionales obtenido mediante \isa{atoms}. Se 
  observa que, por definición de conjunto, no contiene 
  elementos repetidos.%
\end{isamarkuptext}\isamarkuptrue%
\isacommand{notepad}\isamarkupfalse%
\ \isanewline
\isakeyword{begin}\isanewline
%
\isadelimproof
\ \ %
\endisadelimproof
%
\isatagproof
\isacommand{fix}\isamarkupfalse%
\ p\ q\ r\ {\isacharcolon}{\isacharcolon}\ {\isacharprime}a\isanewline
\isanewline
\ \ \isacommand{have}\isamarkupfalse%
\ {\isachardoublequoteopen}atoms\ {\isacharparenleft}Atom\ p{\isacharparenright}\ {\isacharequal}\ {\isacharbraceleft}p{\isacharbraceright}{\isachardoublequoteclose}\isanewline
\ \ \ \ \isacommand{by}\isamarkupfalse%
\ {\isacharparenleft}simp\ only{\isacharcolon}\ formula{\isachardot}set{\isacharparenright}\isanewline
\isanewline
\ \ \isacommand{have}\isamarkupfalse%
\ {\isachardoublequoteopen}atoms\ {\isacharparenleft}\isactrlbold {\isasymnot}\ {\isacharparenleft}Atom\ p{\isacharparenright}{\isacharparenright}\ {\isacharequal}\ {\isacharbraceleft}p{\isacharbraceright}{\isachardoublequoteclose}\isanewline
\ \ \ \ \isacommand{by}\isamarkupfalse%
\ {\isacharparenleft}simp\ only{\isacharcolon}\ formula{\isachardot}set{\isacharparenright}\isanewline
\isanewline
\ \ \isacommand{have}\isamarkupfalse%
\ {\isachardoublequoteopen}atoms\ {\isacharparenleft}{\isacharparenleft}Atom\ p\ \isactrlbold {\isasymrightarrow}\ Atom\ q{\isacharparenright}\ \isactrlbold {\isasymor}\ Atom\ r{\isacharparenright}\ {\isacharequal}\ {\isacharbraceleft}p{\isacharcomma}q{\isacharcomma}r{\isacharbraceright}{\isachardoublequoteclose}\isanewline
\ \ \ \ \isacommand{by}\isamarkupfalse%
\ auto\isanewline
\isanewline
\ \ \isacommand{have}\isamarkupfalse%
\ {\isachardoublequoteopen}atoms\ {\isacharparenleft}{\isacharparenleft}Atom\ p\ \isactrlbold {\isasymrightarrow}\ Atom\ p{\isacharparenright}\ \isactrlbold {\isasymor}\ Atom\ r{\isacharparenright}\ {\isacharequal}\ {\isacharbraceleft}p{\isacharcomma}r{\isacharbraceright}{\isachardoublequoteclose}\isanewline
\ \ \ \ \isacommand{by}\isamarkupfalse%
\ auto%
\endisatagproof
{\isafoldproof}%
%
\isadelimproof
\ \ \isanewline
%
\endisadelimproof
\isacommand{end}\isamarkupfalse%
%
\begin{isamarkuptext}%
En particular, el conjunto de símbolos proposicionales de la 
  fórmula \isa{Bot} es vacío. Además, para calcular esta constante es 
  necesario especificar el tipo sobre el que se construye la fórmula.%
\end{isamarkuptext}\isamarkuptrue%
\isacommand{notepad}\isamarkupfalse%
\ \isanewline
\isakeyword{begin}\isanewline
%
\isadelimproof
\ \ %
\endisadelimproof
%
\isatagproof
\isacommand{fix}\isamarkupfalse%
\ p\ {\isacharcolon}{\isacharcolon}\ {\isacharprime}a\isanewline
\isanewline
\ \ \isacommand{have}\isamarkupfalse%
\ {\isachardoublequoteopen}atoms\ {\isasymbottom}\ {\isacharequal}\ {\isasymemptyset}{\isachardoublequoteclose}\isanewline
\ \ \ \ \isacommand{by}\isamarkupfalse%
\ {\isacharparenleft}simp\ only{\isacharcolon}\ formula{\isachardot}set{\isacharparenright}\isanewline
\isanewline
\ \ \isacommand{have}\isamarkupfalse%
\ {\isachardoublequoteopen}atoms\ {\isacharparenleft}Atom\ p\ \isactrlbold {\isasymor}\ {\isasymbottom}{\isacharparenright}\ {\isacharequal}\ {\isacharbraceleft}p{\isacharbraceright}{\isachardoublequoteclose}\isanewline
\ \ \isacommand{proof}\isamarkupfalse%
\ {\isacharminus}\isanewline
\ \ \ \ \isacommand{have}\isamarkupfalse%
\ {\isachardoublequoteopen}atoms\ {\isacharparenleft}Atom\ p\ \isactrlbold {\isasymor}\ {\isasymbottom}{\isacharparenright}\ {\isacharequal}\ atoms\ {\isacharparenleft}Atom\ p{\isacharparenright}\ {\isasymunion}\ atoms\ Bot{\isachardoublequoteclose}\isanewline
\ \ \ \ \ \ \isacommand{by}\isamarkupfalse%
\ {\isacharparenleft}simp\ only{\isacharcolon}\ formula{\isachardot}set{\isacharparenleft}{\isadigit{5}}{\isacharparenright}{\isacharparenright}\isanewline
\ \ \ \ \isacommand{also}\isamarkupfalse%
\ \isacommand{have}\isamarkupfalse%
\ {\isachardoublequoteopen}{\isasymdots}\ {\isacharequal}\ {\isacharbraceleft}p{\isacharbraceright}\ {\isasymunion}\ atoms\ Bot{\isachardoublequoteclose}\isanewline
\ \ \ \ \ \ \isacommand{by}\isamarkupfalse%
\ {\isacharparenleft}simp\ only{\isacharcolon}\ formula{\isachardot}set{\isacharparenleft}{\isadigit{1}}{\isacharparenright}{\isacharparenright}\isanewline
\ \ \ \ \isacommand{also}\isamarkupfalse%
\ \isacommand{have}\isamarkupfalse%
\ {\isachardoublequoteopen}{\isasymdots}\ {\isacharequal}\ {\isacharbraceleft}p{\isacharbraceright}\ {\isasymunion}\ {\isasymemptyset}{\isachardoublequoteclose}\isanewline
\ \ \ \ \ \ \isacommand{by}\isamarkupfalse%
\ {\isacharparenleft}simp\ only{\isacharcolon}\ formula{\isachardot}set{\isacharparenleft}{\isadigit{2}}{\isacharparenright}{\isacharparenright}\isanewline
\ \ \ \ \isacommand{also}\isamarkupfalse%
\ \isacommand{have}\isamarkupfalse%
\ {\isachardoublequoteopen}{\isasymdots}\ {\isacharequal}\ {\isacharbraceleft}p{\isacharbraceright}{\isachardoublequoteclose}\isanewline
\ \ \ \ \ \ \isacommand{by}\isamarkupfalse%
\ {\isacharparenleft}simp\ only{\isacharcolon}\ Un{\isacharunderscore}empty{\isacharunderscore}right{\isacharparenright}\isanewline
\ \ \ \ \isacommand{finally}\isamarkupfalse%
\ \isacommand{show}\isamarkupfalse%
\ {\isachardoublequoteopen}atoms\ {\isacharparenleft}Atom\ p\ \isactrlbold {\isasymor}\ {\isasymbottom}{\isacharparenright}\ {\isacharequal}\ {\isacharbraceleft}p{\isacharbraceright}{\isachardoublequoteclose}\isanewline
\ \ \ \ \ \ \isacommand{by}\isamarkupfalse%
\ this\isanewline
\ \ \isacommand{qed}\isamarkupfalse%
\isanewline
\isanewline
\ \ \isacommand{have}\isamarkupfalse%
\ {\isachardoublequoteopen}atoms\ {\isacharparenleft}Atom\ p\ \isactrlbold {\isasymor}\ {\isasymbottom}{\isacharparenright}\ {\isacharequal}\ {\isacharbraceleft}p{\isacharbraceright}{\isachardoublequoteclose}\isanewline
\ \ \ \ \isacommand{by}\isamarkupfalse%
\ {\isacharparenleft}simp\ only{\isacharcolon}\ formula{\isachardot}set\ Un{\isacharunderscore}empty{\isacharunderscore}right{\isacharparenright}%
\endisatagproof
{\isafoldproof}%
%
\isadelimproof
\isanewline
%
\endisadelimproof
\isacommand{end}\isamarkupfalse%
\isanewline
\isanewline
\isacommand{value}\isamarkupfalse%
\ {\isachardoublequoteopen}{\isacharparenleft}Bot{\isacharcolon}{\isacharcolon}nat\ formula{\isacharparenright}{\isachardoublequoteclose}%
\begin{isamarkuptext}%
Una vez definida la estructura de las fórmulas, vamos a introducir 
  el método de demostración que seguirán los resultados que aquí 
  presentaremos, tanto en la teoría clásica como en Isabelle. 

  Según la definición recursiva de las fórmulas, dispondremos de un 
  esquema de inducción sobre las mismas:

  \begin{definicion}
    Sea \isa{{\isasymphi}} una propiedad sobre fórmulas que verifica las siguientes 
    condiciones:
    \begin{itemize}
      \item Las fórmulas atómicas la cumplen.
      \item La constante \isa{{\isasymbottom}} la cumple.
      \item Dada \isa{F} fórmula que la cumple, entonces \isa{{\isasymnot}\ F} la cumple.
      \item Dadas \isa{F} y \isa{G} fórmulas que la cumplen, entonces \isa{F\ {\isacharasterisk}\ G} la 
        cumple, donde \isa{{\isacharasterisk}} simboliza cualquier conectiva binaria.
    \end{itemize}
    Entonces, todas las fórmulas proposicionales tienen la propiedad 
    \isa{{\isasymphi}}.
  \end{definicion}

  Análogamente, como las fórmulas proposicionales están definidas 
  mediante un tipo de datos recursivo, Isabelle genera de forma 
  automática el esquema de inducción correspondiente. De este modo, en 
  las pruebas formalizadas utilizaremos la táctica \isa{induction}, 
  que corresponde al siguiente esquema.

  \comentario{Poner bien cada regla.}

  \begin{itemize}
    \item[] \isa{\mbox{}\inferrule{\mbox{{\isasymAnd}x{\isachardot}\ P\ {\isacharparenleft}Atom\ x{\isacharparenright}}\\\ \mbox{P\ {\isasymbottom}}\\\ \mbox{{\isasymAnd}x{\isachardot}\ \mbox{}\inferrule{\mbox{P\ x}}{\mbox{P\ {\isacharparenleft}\isactrlbold {\isasymnot}\ x{\isacharparenright}}}}\\\ \mbox{{\isasymAnd}x{\isadigit{1}}a\ x{\isadigit{2}}{\isachardot}\ \mbox{}\inferrule{\mbox{P\ x{\isadigit{1}}a\ {\isasymand}\ P\ x{\isadigit{2}}}}{\mbox{P\ {\isacharparenleft}x{\isadigit{1}}a\ \isactrlbold {\isasymand}\ x{\isadigit{2}}{\isacharparenright}}}}\\\ \mbox{{\isasymAnd}x{\isadigit{1}}a\ x{\isadigit{2}}{\isachardot}\ \mbox{}\inferrule{\mbox{P\ x{\isadigit{1}}a\ {\isasymand}\ P\ x{\isadigit{2}}}}{\mbox{P\ {\isacharparenleft}x{\isadigit{1}}a\ \isactrlbold {\isasymor}\ x{\isadigit{2}}{\isacharparenright}}}}\\\ \mbox{{\isasymAnd}x{\isadigit{1}}a\ x{\isadigit{2}}{\isachardot}\ \mbox{}\inferrule{\mbox{P\ x{\isadigit{1}}a\ {\isasymand}\ P\ x{\isadigit{2}}}}{\mbox{P\ {\isacharparenleft}x{\isadigit{1}}a\ \isactrlbold {\isasymrightarrow}\ x{\isadigit{2}}{\isacharparenright}}}}}{\mbox{P\ formula}}}
  \end{itemize} 

  Como hemos señalado, el esquema inductivo se aplicará en cada uno de 
  los casos de los constructores, desglosándose así seis casos distintos 
  como se muestra anteriormente. Además, todas las demostraciones sobre 
  casos de conectivas binarias son equivalentes en esta sección, pues la 
  construcción sintáctica de fórmulas es idéntica entre ellas. Estas se 
  diferencian esencialmente en la connotación semántica que veremos más 
  adelante.

  Llegamos así al primer resultado de este apartado:

  \begin{lema}
    El conjunto de los átomos de una fórmula proposicional es finito.
  \end{lema}

  Para proceder a la demostración, vamos a dar una definición inductiva 
  de conjunto finito. Cabe añadir que la demostración seguirá el esquema 
  inductivo relativo a la estructura de fórmula, y no el que induce esta
  última definición.

  \begin{definicion}
    Los conjuntos finitos son:
      \begin{itemize}
        \item El vacío.
        \item Dado un conjunto finito \isa{A} y un elemento cualquiera \isa{a}, 
          entonces \isa{{\isacharbraceleft}a{\isacharbraceright}\ {\isasymunion}\ A} es finito.
      \end{itemize}
  \end{definicion}

  En Isabelle, podemos formalizar el lema como sigue.%
\end{isamarkuptext}\isamarkuptrue%
\isacommand{lemma}\isamarkupfalse%
\ {\isachardoublequoteopen}finite\ {\isacharparenleft}atoms\ F{\isacharparenright}{\isachardoublequoteclose}\isanewline
%
\isadelimproof
\ \ %
\endisadelimproof
%
\isatagproof
\isacommand{oops}\isamarkupfalse%
%
\endisatagproof
{\isafoldproof}%
%
\isadelimproof
%
\endisadelimproof
%
\begin{isamarkuptext}%
Análogamente, el enunciado formalizado contiene la definición 
  \isa{finite\ S}, perteneciente a la teoría 
  \href{https://n9.cl/x86r}{FiniteSet.thy}.%
\end{isamarkuptext}\isamarkuptrue%
\isacommand{inductive}\isamarkupfalse%
\ finite{\isacharprime}\ {\isacharcolon}{\isacharcolon}\ {\isachardoublequoteopen}{\isacharprime}a\ set\ {\isasymRightarrow}\ bool{\isachardoublequoteclose}\ \isakeyword{where}\isanewline
\ \ emptyI{\isacharprime}\ {\isacharbrackleft}simp{\isacharcomma}\ intro{\isacharbang}{\isacharbrackright}{\isacharcolon}\ {\isachardoublequoteopen}finite{\isacharprime}\ {\isacharbraceleft}{\isacharbraceright}{\isachardoublequoteclose}\isanewline
{\isacharbar}\ insertI{\isacharprime}\ {\isacharbrackleft}simp{\isacharcomma}\ intro{\isacharbang}{\isacharbrackright}{\isacharcolon}\ {\isachardoublequoteopen}finite{\isacharprime}\ A\ {\isasymLongrightarrow}\ finite{\isacharprime}\ {\isacharparenleft}insert\ a\ A{\isacharparenright}{\isachardoublequoteclose}%
\begin{isamarkuptext}%
Observemos que la definición anterior corresponde a 
  \isa{finite{\isacharprime}}. Sin embargo, es análoga a \isa{finite} de la 
  teoría original. Este cambio de notación es necesario para no definir 
  dos veces de manera idéntica la misma noción en Isabelle. Por otra 
  parte, esta definición permitiría la demostración del lema por 
  simplificacion pues, dentro de ella las reglas que especifica se han 
  añadido como tácticas de \isa{simp} e \isa{intro{\isacharbang}}. Sin embargo, conforme al 
  objetivo de este análisis, detallaremos dónde es usada cada una de las 
  reglas en la prueba detallada. 

  A continuación, veamos en primer lugar la demostración clásica del 
  lema. 

  \begin{demostracion}
  La prueba es por inducción sobre el tipo recursivo de las fórmulas. 
  Veamos cada caso.
  
  Consideremos una fórmula atómica \isa{p} cualquiera. Entonces, 
  su conjunto de variables proposicionales es \isa{{\isacharbraceleft}p{\isacharbraceright}}, finito.

  Sea la fórmula \isa{{\isasymbottom}}. Entonces, su conjunto de átomos es vacío y, por 
  lo tanto, finito.
  
  Sea \isa{F} una fórmula cuyo conjunto de variables proposicionales sea 
  finito. Entonces, por definición, \isa{{\isasymnot}\ F} y \isa{F} tienen igual conjunto de
  átomos y, por hipótesis de inducción, es finito.

  Consideremos las fórmulas \isa{F} y \isa{G} cuyos conjuntos de átomos son 
  finitos. Por construcción, el conjunto de variables de \isa{F{\isacharasterisk}G} es la 
  unión de sus respectivos conjuntos de átomos para cualquier \isa{{\isacharasterisk}} 
  conectiva binaria. Por lo tanto, usando la hipótesis de inducción, 
  dicho conjunto es finito. 
  \end{demostracion} 

  Veamos ahora la prueba detallada en Isabelle. Mostraremos con detalle 
  todos los casos de conectivas binarias, aunque se puede observar que 
  son completamente análogos. Para facilitar la lectura, primero 
  demostraremos por separado cada uno de los casos según el esquema 
  inductivo de fórmulas, y finalmente añadiremos la prueba para una 
  fórmula cualquiera a partir de los anteriores.%
\end{isamarkuptext}\isamarkuptrue%
\isacommand{lemma}\isamarkupfalse%
\ atoms{\isacharunderscore}finite{\isacharunderscore}atom{\isacharcolon}\isanewline
\ \ {\isachardoublequoteopen}finite\ {\isacharparenleft}atoms\ {\isacharparenleft}Atom\ x{\isacharparenright}{\isacharparenright}{\isachardoublequoteclose}\isanewline
%
\isadelimproof
%
\endisadelimproof
%
\isatagproof
\isacommand{proof}\isamarkupfalse%
\ {\isacharminus}\isanewline
\ \ \isacommand{have}\isamarkupfalse%
\ {\isachardoublequoteopen}finite\ {\isasymemptyset}{\isachardoublequoteclose}\isanewline
\ \ \ \ \isacommand{by}\isamarkupfalse%
\ {\isacharparenleft}simp\ only{\isacharcolon}\ finite{\isachardot}emptyI{\isacharparenright}\isanewline
\ \ \isacommand{then}\isamarkupfalse%
\ \isacommand{have}\isamarkupfalse%
\ {\isachardoublequoteopen}finite\ {\isacharbraceleft}x{\isacharbraceright}{\isachardoublequoteclose}\isanewline
\ \ \ \ \isacommand{by}\isamarkupfalse%
\ {\isacharparenleft}simp\ only{\isacharcolon}\ finite{\isacharunderscore}insert{\isacharparenright}\isanewline
\ \ \isacommand{then}\isamarkupfalse%
\ \isacommand{show}\isamarkupfalse%
\ {\isachardoublequoteopen}finite\ {\isacharparenleft}atoms\ {\isacharparenleft}Atom\ x{\isacharparenright}{\isacharparenright}{\isachardoublequoteclose}\isanewline
\ \ \ \ \isacommand{by}\isamarkupfalse%
\ {\isacharparenleft}simp\ only{\isacharcolon}\ formula{\isachardot}set{\isacharparenleft}{\isadigit{1}}{\isacharparenright}{\isacharparenright}\ \isanewline
\isacommand{qed}\isamarkupfalse%
%
\endisatagproof
{\isafoldproof}%
%
\isadelimproof
\isanewline
%
\endisadelimproof
\isanewline
\isacommand{lemma}\isamarkupfalse%
\ atoms{\isacharunderscore}finite{\isacharunderscore}bot{\isacharcolon}\isanewline
\ \ {\isachardoublequoteopen}finite\ {\isacharparenleft}atoms\ {\isasymbottom}{\isacharparenright}{\isachardoublequoteclose}\isanewline
%
\isadelimproof
%
\endisadelimproof
%
\isatagproof
\isacommand{proof}\isamarkupfalse%
\ {\isacharminus}\isanewline
\ \ \isacommand{have}\isamarkupfalse%
\ {\isachardoublequoteopen}finite\ {\isasymemptyset}{\isachardoublequoteclose}\isanewline
\ \ \ \ \isacommand{by}\isamarkupfalse%
\ {\isacharparenleft}simp\ only{\isacharcolon}\ finite{\isachardot}emptyI{\isacharparenright}\isanewline
\ \ \isacommand{then}\isamarkupfalse%
\ \isacommand{show}\isamarkupfalse%
\ {\isachardoublequoteopen}finite\ {\isacharparenleft}atoms\ {\isasymbottom}{\isacharparenright}{\isachardoublequoteclose}\isanewline
\ \ \ \ \isacommand{by}\isamarkupfalse%
\ {\isacharparenleft}simp\ only{\isacharcolon}\ formula{\isachardot}set{\isacharparenleft}{\isadigit{2}}{\isacharparenright}{\isacharparenright}\ \isanewline
\isacommand{qed}\isamarkupfalse%
%
\endisatagproof
{\isafoldproof}%
%
\isadelimproof
\isanewline
%
\endisadelimproof
\isanewline
\isacommand{lemma}\isamarkupfalse%
\ atoms{\isacharunderscore}finite{\isacharunderscore}not{\isacharcolon}\isanewline
\ \ \isakeyword{assumes}\ {\isachardoublequoteopen}finite\ {\isacharparenleft}atoms\ F{\isacharparenright}{\isachardoublequoteclose}\ \isanewline
\ \ \isakeyword{shows}\ \ \ {\isachardoublequoteopen}finite\ {\isacharparenleft}atoms\ {\isacharparenleft}\isactrlbold {\isasymnot}\ F{\isacharparenright}{\isacharparenright}{\isachardoublequoteclose}\isanewline
%
\isadelimproof
\ \ %
\endisadelimproof
%
\isatagproof
\isacommand{using}\isamarkupfalse%
\ assms\isanewline
\ \ \isacommand{by}\isamarkupfalse%
\ {\isacharparenleft}simp\ only{\isacharcolon}\ formula{\isachardot}set{\isacharparenleft}{\isadigit{3}}{\isacharparenright}{\isacharparenright}%
\endisatagproof
{\isafoldproof}%
%
\isadelimproof
\ \isanewline
%
\endisadelimproof
\isanewline
\isacommand{lemma}\isamarkupfalse%
\ atoms{\isacharunderscore}finite{\isacharunderscore}and{\isacharcolon}\isanewline
\ \ \isakeyword{assumes}\ {\isachardoublequoteopen}finite\ {\isacharparenleft}atoms\ F{\isadigit{1}}{\isacharparenright}{\isachardoublequoteclose}\isanewline
\ \ \ \ \ \ \ \ \ \ {\isachardoublequoteopen}finite\ {\isacharparenleft}atoms\ F{\isadigit{2}}{\isacharparenright}{\isachardoublequoteclose}\isanewline
\ \ \isakeyword{shows}\ \ \ {\isachardoublequoteopen}finite\ {\isacharparenleft}atoms\ {\isacharparenleft}F{\isadigit{1}}\ \isactrlbold {\isasymand}\ F{\isadigit{2}}{\isacharparenright}{\isacharparenright}{\isachardoublequoteclose}\isanewline
%
\isadelimproof
%
\endisadelimproof
%
\isatagproof
\isacommand{proof}\isamarkupfalse%
\ {\isacharminus}\isanewline
\ \ \isacommand{have}\isamarkupfalse%
\ {\isachardoublequoteopen}finite\ {\isacharparenleft}atoms\ F{\isadigit{1}}\ {\isasymunion}\ atoms\ F{\isadigit{2}}{\isacharparenright}{\isachardoublequoteclose}\isanewline
\ \ \ \ \isacommand{using}\isamarkupfalse%
\ assms\isanewline
\ \ \ \ \isacommand{by}\isamarkupfalse%
\ {\isacharparenleft}simp\ only{\isacharcolon}\ finite{\isacharunderscore}UnI{\isacharparenright}\isanewline
\ \ \isacommand{then}\isamarkupfalse%
\ \isacommand{show}\isamarkupfalse%
\ {\isachardoublequoteopen}finite\ {\isacharparenleft}atoms\ {\isacharparenleft}F{\isadigit{1}}\ \isactrlbold {\isasymand}\ F{\isadigit{2}}{\isacharparenright}{\isacharparenright}{\isachardoublequoteclose}\ \ \isanewline
\ \ \ \ \isacommand{by}\isamarkupfalse%
\ {\isacharparenleft}simp\ only{\isacharcolon}\ formula{\isachardot}set{\isacharparenleft}{\isadigit{4}}{\isacharparenright}{\isacharparenright}\isanewline
\isacommand{qed}\isamarkupfalse%
%
\endisatagproof
{\isafoldproof}%
%
\isadelimproof
\isanewline
%
\endisadelimproof
\isanewline
\isacommand{lemma}\isamarkupfalse%
\ atoms{\isacharunderscore}finite{\isacharunderscore}or{\isacharcolon}\isanewline
\ \ \isakeyword{assumes}\ {\isachardoublequoteopen}finite\ {\isacharparenleft}atoms\ F{\isadigit{1}}{\isacharparenright}{\isachardoublequoteclose}\isanewline
\ \ \ \ \ \ \ \ \ \ {\isachardoublequoteopen}finite\ {\isacharparenleft}atoms\ F{\isadigit{2}}{\isacharparenright}{\isachardoublequoteclose}\isanewline
\ \ \isakeyword{shows}\ \ \ {\isachardoublequoteopen}finite\ {\isacharparenleft}atoms\ {\isacharparenleft}F{\isadigit{1}}\ \isactrlbold {\isasymor}\ F{\isadigit{2}}{\isacharparenright}{\isacharparenright}{\isachardoublequoteclose}\isanewline
%
\isadelimproof
%
\endisadelimproof
%
\isatagproof
\isacommand{proof}\isamarkupfalse%
\ {\isacharminus}\isanewline
\ \ \isacommand{have}\isamarkupfalse%
\ {\isachardoublequoteopen}finite\ {\isacharparenleft}atoms\ F{\isadigit{1}}\ {\isasymunion}\ atoms\ F{\isadigit{2}}{\isacharparenright}{\isachardoublequoteclose}\isanewline
\ \ \ \ \isacommand{using}\isamarkupfalse%
\ assms\isanewline
\ \ \ \ \isacommand{by}\isamarkupfalse%
\ {\isacharparenleft}simp\ only{\isacharcolon}\ finite{\isacharunderscore}UnI{\isacharparenright}\isanewline
\ \ \isacommand{then}\isamarkupfalse%
\ \isacommand{show}\isamarkupfalse%
\ {\isachardoublequoteopen}finite\ {\isacharparenleft}atoms\ {\isacharparenleft}F{\isadigit{1}}\ \isactrlbold {\isasymor}\ F{\isadigit{2}}{\isacharparenright}{\isacharparenright}{\isachardoublequoteclose}\ \ \isanewline
\ \ \ \ \isacommand{by}\isamarkupfalse%
\ {\isacharparenleft}simp\ only{\isacharcolon}\ formula{\isachardot}set{\isacharparenleft}{\isadigit{5}}{\isacharparenright}{\isacharparenright}\isanewline
\isacommand{qed}\isamarkupfalse%
%
\endisatagproof
{\isafoldproof}%
%
\isadelimproof
\isanewline
%
\endisadelimproof
\isanewline
\isacommand{lemma}\isamarkupfalse%
\ atoms{\isacharunderscore}finite{\isacharunderscore}imp{\isacharcolon}\isanewline
\ \ \isakeyword{assumes}\ {\isachardoublequoteopen}finite\ {\isacharparenleft}atoms\ F{\isadigit{1}}{\isacharparenright}{\isachardoublequoteclose}\isanewline
\ \ \ \ \ \ \ \ \ \ {\isachardoublequoteopen}finite\ {\isacharparenleft}atoms\ F{\isadigit{2}}{\isacharparenright}{\isachardoublequoteclose}\isanewline
\ \ \isakeyword{shows}\ \ \ {\isachardoublequoteopen}finite\ {\isacharparenleft}atoms\ {\isacharparenleft}F{\isadigit{1}}\ \isactrlbold {\isasymrightarrow}\ F{\isadigit{2}}{\isacharparenright}{\isacharparenright}{\isachardoublequoteclose}\isanewline
%
\isadelimproof
%
\endisadelimproof
%
\isatagproof
\isacommand{proof}\isamarkupfalse%
\ {\isacharminus}\isanewline
\ \ \isacommand{have}\isamarkupfalse%
\ {\isachardoublequoteopen}finite\ {\isacharparenleft}atoms\ F{\isadigit{1}}\ {\isasymunion}\ atoms\ F{\isadigit{2}}{\isacharparenright}{\isachardoublequoteclose}\isanewline
\ \ \ \ \isacommand{using}\isamarkupfalse%
\ assms\isanewline
\ \ \ \ \isacommand{by}\isamarkupfalse%
\ {\isacharparenleft}simp\ only{\isacharcolon}\ finite{\isacharunderscore}UnI{\isacharparenright}\isanewline
\ \ \isacommand{then}\isamarkupfalse%
\ \isacommand{show}\isamarkupfalse%
\ {\isachardoublequoteopen}finite\ {\isacharparenleft}atoms\ {\isacharparenleft}F{\isadigit{1}}\ \isactrlbold {\isasymrightarrow}\ F{\isadigit{2}}{\isacharparenright}{\isacharparenright}{\isachardoublequoteclose}\ \ \isanewline
\ \ \ \ \isacommand{by}\isamarkupfalse%
\ {\isacharparenleft}simp\ only{\isacharcolon}\ formula{\isachardot}set{\isacharparenleft}{\isadigit{6}}{\isacharparenright}{\isacharparenright}\isanewline
\isacommand{qed}\isamarkupfalse%
%
\endisatagproof
{\isafoldproof}%
%
\isadelimproof
\isanewline
%
\endisadelimproof
\isanewline
\isacommand{lemma}\isamarkupfalse%
\ atoms{\isacharunderscore}finite{\isacharcolon}\ {\isachardoublequoteopen}finite\ {\isacharparenleft}atoms\ F{\isacharparenright}{\isachardoublequoteclose}\isanewline
%
\isadelimproof
%
\endisadelimproof
%
\isatagproof
\isacommand{proof}\isamarkupfalse%
\ {\isacharparenleft}induction\ F{\isacharparenright}\isanewline
\ \ \isacommand{case}\isamarkupfalse%
\ {\isacharparenleft}Atom\ x{\isacharparenright}\isanewline
\ \ \isacommand{then}\isamarkupfalse%
\ \isacommand{show}\isamarkupfalse%
\ {\isacharquery}case\ \isacommand{by}\isamarkupfalse%
\ {\isacharparenleft}simp\ only{\isacharcolon}\ atoms{\isacharunderscore}finite{\isacharunderscore}atom{\isacharparenright}\isanewline
\isacommand{next}\isamarkupfalse%
\isanewline
\ \ \isacommand{case}\isamarkupfalse%
\ Bot\isanewline
\ \ \isacommand{then}\isamarkupfalse%
\ \isacommand{show}\isamarkupfalse%
\ {\isacharquery}case\ \isacommand{by}\isamarkupfalse%
\ {\isacharparenleft}simp\ only{\isacharcolon}\ atoms{\isacharunderscore}finite{\isacharunderscore}bot{\isacharparenright}\isanewline
\isacommand{next}\isamarkupfalse%
\isanewline
\ \ \isacommand{case}\isamarkupfalse%
\ {\isacharparenleft}Not\ F{\isacharparenright}\isanewline
\ \ \isacommand{then}\isamarkupfalse%
\ \isacommand{show}\isamarkupfalse%
\ {\isacharquery}case\ \isacommand{by}\isamarkupfalse%
\ {\isacharparenleft}simp\ only{\isacharcolon}\ atoms{\isacharunderscore}finite{\isacharunderscore}not{\isacharparenright}\isanewline
\isacommand{next}\isamarkupfalse%
\isanewline
\ \ \isacommand{case}\isamarkupfalse%
\ {\isacharparenleft}And\ F{\isadigit{1}}\ F{\isadigit{2}}{\isacharparenright}\isanewline
\ \ \isacommand{then}\isamarkupfalse%
\ \isacommand{show}\isamarkupfalse%
\ {\isacharquery}case\ \isacommand{by}\isamarkupfalse%
\ {\isacharparenleft}simp\ only{\isacharcolon}\ atoms{\isacharunderscore}finite{\isacharunderscore}and{\isacharparenright}\isanewline
\isacommand{next}\isamarkupfalse%
\isanewline
\ \ \isacommand{case}\isamarkupfalse%
\ {\isacharparenleft}Or\ F{\isadigit{1}}\ F{\isadigit{2}}{\isacharparenright}\isanewline
\ \ \isacommand{then}\isamarkupfalse%
\ \isacommand{show}\isamarkupfalse%
\ {\isacharquery}case\ \isacommand{by}\isamarkupfalse%
\ {\isacharparenleft}simp\ only{\isacharcolon}\ atoms{\isacharunderscore}finite{\isacharunderscore}or{\isacharparenright}\isanewline
\isacommand{next}\isamarkupfalse%
\isanewline
\ \ \isacommand{case}\isamarkupfalse%
\ {\isacharparenleft}Imp\ F{\isadigit{1}}\ F{\isadigit{2}}{\isacharparenright}\isanewline
\ \ \isacommand{then}\isamarkupfalse%
\ \isacommand{show}\isamarkupfalse%
\ {\isacharquery}case\ \isacommand{by}\isamarkupfalse%
\ {\isacharparenleft}simp\ only{\isacharcolon}\ atoms{\isacharunderscore}finite{\isacharunderscore}imp{\isacharparenright}\isanewline
\isacommand{qed}\isamarkupfalse%
%
\endisatagproof
{\isafoldproof}%
%
\isadelimproof
%
\endisadelimproof
%
\begin{isamarkuptext}%
Su demostración automática es la siguiente.%
\end{isamarkuptext}\isamarkuptrue%
\isacommand{lemma}\isamarkupfalse%
\ {\isachardoublequoteopen}finite\ {\isacharparenleft}atoms\ F{\isacharparenright}{\isachardoublequoteclose}\ \isanewline
%
\isadelimproof
\ \ %
\endisadelimproof
%
\isatagproof
\isacommand{by}\isamarkupfalse%
\ {\isacharparenleft}induction\ F{\isacharparenright}\ simp{\isacharunderscore}all%
\endisatagproof
{\isafoldproof}%
%
\isadelimproof
%
\endisadelimproof
%
\isadelimdocument
%
\endisadelimdocument
%
\isatagdocument
%
\isamarkupsection{Subfórmulas%
}
\isamarkuptrue%
%
\endisatagdocument
{\isafolddocument}%
%
\isadelimdocument
%
\endisadelimdocument
%
\begin{isamarkuptext}%
Veamos la noción de subfórmulas.

  \begin{definicion}
  El conjunto de subfórmulas de una fórmula \isa{F}, notada \isa{Subf{\isacharparenleft}F{\isacharparenright}}, se 
  define recursivamente como:
    \begin{itemize}
      \item \isa{{\isacharbraceleft}F{\isacharbraceright}} si \isa{F} es una fórmula atómica.
      \item \isa{{\isacharbraceleft}{\isasymbottom}{\isacharbraceright}} si \isa{F} es \isa{{\isasymbottom}}.
      \item \isa{{\isacharbraceleft}F{\isacharbraceright}\ {\isasymunion}\ Subf{\isacharparenleft}G{\isacharparenright}} si \isa{F} es \isa{{\isasymnot}G}.
      \item \isa{{\isacharbraceleft}F{\isacharbraceright}\ {\isasymunion}\ Subf{\isacharparenleft}G{\isacharparenright}\ {\isasymunion}\ Subf{\isacharparenleft}H{\isacharparenright}} si \isa{F} es \isa{G{\isacharasterisk}H} donde \isa{{\isacharasterisk}} es 
        cualquier conectiva binaria.
    \end{itemize}
  \end{definicion}

  Para proceder a la formalización de Isabelle, seguiremos dos etapas. 
  En primer lugar, definimos la función primitiva recursiva 
  \isa{subformulae}. Esta nos devolverá la lista de todas las 
  subfórmulas de una fórmula original obtenidas recursivamente.%
\end{isamarkuptext}\isamarkuptrue%
\isacommand{primrec}\isamarkupfalse%
\ subformulae\ {\isacharcolon}{\isacharcolon}\ {\isachardoublequoteopen}{\isacharprime}a\ formula\ {\isasymRightarrow}\ {\isacharprime}a\ formula\ list{\isachardoublequoteclose}\ \isakeyword{where}\isanewline
\ \ {\isachardoublequoteopen}subformulae\ {\isacharparenleft}Atom\ k{\isacharparenright}\ {\isacharequal}\ {\isacharbrackleft}Atom\ k{\isacharbrackright}{\isachardoublequoteclose}\ \isanewline
{\isacharbar}\ {\isachardoublequoteopen}subformulae\ {\isasymbottom}\ \ \ \ \ \ \ \ {\isacharequal}\ {\isacharbrackleft}{\isasymbottom}{\isacharbrackright}{\isachardoublequoteclose}\ \isanewline
{\isacharbar}\ {\isachardoublequoteopen}subformulae\ {\isacharparenleft}\isactrlbold {\isasymnot}\ F{\isacharparenright}\ \ \ \ {\isacharequal}\ {\isacharparenleft}\isactrlbold {\isasymnot}\ F{\isacharparenright}\ {\isacharhash}\ subformulae\ F{\isachardoublequoteclose}\ \isanewline
{\isacharbar}\ {\isachardoublequoteopen}subformulae\ {\isacharparenleft}F\ \isactrlbold {\isasymand}\ G{\isacharparenright}\ \ {\isacharequal}\ {\isacharparenleft}F\ \isactrlbold {\isasymand}\ G{\isacharparenright}\ {\isacharhash}\ subformulae\ F\ {\isacharat}\ subformulae\ G{\isachardoublequoteclose}\ \isanewline
{\isacharbar}\ {\isachardoublequoteopen}subformulae\ {\isacharparenleft}F\ \isactrlbold {\isasymor}\ G{\isacharparenright}\ \ {\isacharequal}\ {\isacharparenleft}F\ \isactrlbold {\isasymor}\ G{\isacharparenright}\ {\isacharhash}\ subformulae\ F\ {\isacharat}\ subformulae\ G{\isachardoublequoteclose}\isanewline
{\isacharbar}\ {\isachardoublequoteopen}subformulae\ {\isacharparenleft}F\ \isactrlbold {\isasymrightarrow}\ G{\isacharparenright}\ {\isacharequal}\ {\isacharparenleft}F\ \isactrlbold {\isasymrightarrow}\ G{\isacharparenright}\ {\isacharhash}\ subformulae\ F\ {\isacharat}\ subformulae\ G{\isachardoublequoteclose}%
\begin{isamarkuptext}%
Observemos que, en la definición anterior, \isa{{\isacharhash}} es el operador que 
  añade un elemento al comienzo de una lista y \isa{{\isacharat}} concatena varias 
  listas. Siguiendo con los ejemplos, apliquemos \isa{subformulae} en 
  las distintas fórmulas. En particular, al tratarse de una lista pueden 
  aparecer elementos repetidos como se muestra a continuación.%
\end{isamarkuptext}\isamarkuptrue%
\isacommand{notepad}\isamarkupfalse%
\isanewline
\isakeyword{begin}\isanewline
%
\isadelimproof
\ \ %
\endisadelimproof
%
\isatagproof
\isacommand{fix}\isamarkupfalse%
\ p\ {\isacharcolon}{\isacharcolon}\ {\isacharprime}a\isanewline
\isanewline
\ \ \isacommand{have}\isamarkupfalse%
\ {\isachardoublequoteopen}subformulae\ {\isacharparenleft}Atom\ p{\isacharparenright}\ {\isacharequal}\ {\isacharbrackleft}Atom\ p{\isacharbrackright}{\isachardoublequoteclose}\isanewline
\ \ \ \ \isacommand{by}\isamarkupfalse%
\ simp\isanewline
\isanewline
\ \ \isacommand{have}\isamarkupfalse%
\ {\isachardoublequoteopen}subformulae\ {\isacharparenleft}\isactrlbold {\isasymnot}\ {\isacharparenleft}Atom\ p{\isacharparenright}{\isacharparenright}\ {\isacharequal}\ {\isacharbrackleft}\isactrlbold {\isasymnot}\ {\isacharparenleft}Atom\ p{\isacharparenright}{\isacharcomma}\ Atom\ p{\isacharbrackright}{\isachardoublequoteclose}\isanewline
\ \ \ \ \isacommand{by}\isamarkupfalse%
\ simp\isanewline
\isanewline
\ \ \isacommand{have}\isamarkupfalse%
\ {\isachardoublequoteopen}subformulae\ {\isacharparenleft}{\isacharparenleft}Atom\ p\ \isactrlbold {\isasymrightarrow}\ Atom\ q{\isacharparenright}\ \isactrlbold {\isasymor}\ Atom\ r{\isacharparenright}\ {\isacharequal}\ \isanewline
\ \ \ \ \ \ \ {\isacharbrackleft}{\isacharparenleft}Atom\ p\ \isactrlbold {\isasymrightarrow}\ Atom\ q{\isacharparenright}\ \isactrlbold {\isasymor}\ Atom\ r{\isacharcomma}\ Atom\ p\ \isactrlbold {\isasymrightarrow}\ Atom\ q{\isacharcomma}\ Atom\ p{\isacharcomma}\ \isanewline
\ \ \ \ \ \ \ \ Atom\ q{\isacharcomma}\ Atom\ r{\isacharbrackright}{\isachardoublequoteclose}\isanewline
\ \ \ \ \isacommand{by}\isamarkupfalse%
\ simp\isanewline
\isanewline
\ \ \isacommand{have}\isamarkupfalse%
\ {\isachardoublequoteopen}subformulae\ {\isacharparenleft}Atom\ p\ \isactrlbold {\isasymand}\ {\isasymbottom}{\isacharparenright}\ {\isacharequal}\ {\isacharbrackleft}Atom\ p\ \isactrlbold {\isasymand}\ {\isasymbottom}{\isacharcomma}\ Atom\ p{\isacharcomma}\ {\isasymbottom}{\isacharbrackright}{\isachardoublequoteclose}\isanewline
\ \ \ \ \isacommand{by}\isamarkupfalse%
\ simp\isanewline
\isanewline
\ \ \isacommand{have}\isamarkupfalse%
\ {\isachardoublequoteopen}subformulae\ {\isacharparenleft}Atom\ p\ \isactrlbold {\isasymor}\ Atom\ p{\isacharparenright}\ {\isacharequal}\ \isanewline
\ \ \ \ \ \ \ {\isacharbrackleft}Atom\ p\ \isactrlbold {\isasymor}\ Atom\ p{\isacharcomma}\ Atom\ p{\isacharcomma}\ Atom\ p{\isacharbrackright}{\isachardoublequoteclose}\isanewline
\ \ \ \ \isacommand{by}\isamarkupfalse%
\ simp%
\endisatagproof
{\isafoldproof}%
%
\isadelimproof
\isanewline
%
\endisadelimproof
\isacommand{end}\isamarkupfalse%
%
\begin{isamarkuptext}%
En la segunda etapa de formalización, definimos 
  \isa{setSubformulae}, que convierte al tipo conjunto la lista de 
  subfórmulas anterior.%
\end{isamarkuptext}\isamarkuptrue%
\isacommand{abbreviation}\isamarkupfalse%
\ setSubformulae\ {\isacharcolon}{\isacharcolon}\ {\isachardoublequoteopen}{\isacharprime}a\ formula\ {\isasymRightarrow}\ {\isacharprime}a\ formula\ set{\isachardoublequoteclose}\ \isakeyword{where}\isanewline
\ \ {\isachardoublequoteopen}setSubformulae\ F\ {\isasymequiv}\ set\ {\isacharparenleft}subformulae\ F{\isacharparenright}{\isachardoublequoteclose}%
\begin{isamarkuptext}%
De este modo, la función \isa{setSubformulae} es la formalización
  en Isabelle de \isa{Subf{\isacharparenleft}·{\isacharparenright}}. En Isabelle, primero hemos definido la lista 
  de subfórmulas pues, en algunos casos, es más sencilla la prueba de 
  resultados sobre este tipo. Sin embargo, el tipo de conjuntos facilita
  las pruebas de los resultados de esta sección. Algunas de las
  ventajas del tipo conjuntos son la eliminación de elementos repetidos 
  o las operaciones propias de teoría de conjuntos. Observemos los 
  siguientes ejemplos con el tipo de conjuntos.%
\end{isamarkuptext}\isamarkuptrue%
\isacommand{notepad}\isamarkupfalse%
\isanewline
\isakeyword{begin}\isanewline
%
\isadelimproof
\ \ %
\endisadelimproof
%
\isatagproof
\isacommand{fix}\isamarkupfalse%
\ p\ q\ r\ {\isacharcolon}{\isacharcolon}\ {\isacharprime}a\isanewline
\isanewline
\ \ \isacommand{have}\isamarkupfalse%
\ {\isachardoublequoteopen}setSubformulae\ {\isacharparenleft}Atom\ p\ \isactrlbold {\isasymor}\ Atom\ p{\isacharparenright}\ {\isacharequal}\ {\isacharbraceleft}Atom\ p\ \isactrlbold {\isasymor}\ Atom\ p{\isacharcomma}\ Atom\ p{\isacharbraceright}{\isachardoublequoteclose}\isanewline
\ \ \ \ \isacommand{by}\isamarkupfalse%
\ simp\isanewline
\ \ \isanewline
\ \ \isacommand{have}\isamarkupfalse%
\ {\isachardoublequoteopen}setSubformulae\ {\isacharparenleft}{\isacharparenleft}Atom\ p\ \isactrlbold {\isasymrightarrow}\ Atom\ q{\isacharparenright}\ \isactrlbold {\isasymor}\ Atom\ r{\isacharparenright}\ {\isacharequal}\isanewline
\ \ \ \ \ \ \ \ {\isacharbraceleft}{\isacharparenleft}Atom\ p\ \isactrlbold {\isasymrightarrow}\ Atom\ q{\isacharparenright}\ \isactrlbold {\isasymor}\ Atom\ r{\isacharcomma}\ Atom\ p\ \isactrlbold {\isasymrightarrow}\ Atom\ q{\isacharcomma}\ Atom\ p{\isacharcomma}\ \isanewline
\ \ \ \ \ \ \ \ \ Atom\ q{\isacharcomma}\ Atom\ r{\isacharbraceright}{\isachardoublequoteclose}\isanewline
\ \ \isacommand{by}\isamarkupfalse%
\ auto%
\endisatagproof
{\isafoldproof}%
%
\isadelimproof
\ \ \ \isanewline
%
\endisadelimproof
\isacommand{end}\isamarkupfalse%
%
\begin{isamarkuptext}%
Por otro lado, debemos señalar que el uso de 
  \isa{abbreviation} para definir \isa{setSubformulae} no es 
  arbitrario. Esta elección se debe a que el tipo \isa{abbreviation} 
  se trata de un sinónimo para una expresión cuyo tipo ya existe (en 
  nuestro caso, convertir en conjunto la lista obtenida con 
  \isa{subformulae}). No es una definición propiamente dicha, sino 
  una forma de nombrar la composición de las funciones \isa{set} y 
  \isa{subformulae}.

  En primer lugar, veamos que \isa{setSubformulae} es una
  formalización de \isa{Subf} en Isabelle. Para ello 
  utilizaremos el siguiente resultado sobre listas, probado como sigue.%
\end{isamarkuptext}\isamarkuptrue%
\isacommand{lemma}\isamarkupfalse%
\ set{\isacharunderscore}insert{\isacharcolon}\ {\isachardoublequoteopen}set\ {\isacharparenleft}x\ {\isacharhash}\ ys{\isacharparenright}\ {\isacharequal}\ {\isacharbraceleft}x{\isacharbraceright}\ {\isasymunion}\ set\ ys{\isachardoublequoteclose}\isanewline
%
\isadelimproof
\ \ %
\endisadelimproof
%
\isatagproof
\isacommand{by}\isamarkupfalse%
\ {\isacharparenleft}simp\ only{\isacharcolon}\ list{\isachardot}set{\isacharparenleft}{\isadigit{2}}{\isacharparenright}\ Un{\isacharunderscore}insert{\isacharunderscore}left\ sup{\isacharunderscore}bot{\isachardot}left{\isacharunderscore}neutral{\isacharparenright}%
\endisatagproof
{\isafoldproof}%
%
\isadelimproof
%
\endisadelimproof
%
\begin{isamarkuptext}%
Por tanto, obtenemos la equivalencia como resultado de los 
  siguientes lemas, que aparecen demostrados de manera detallada.%
\end{isamarkuptext}\isamarkuptrue%
\isacommand{lemma}\isamarkupfalse%
\ setSubformulae{\isacharunderscore}atom{\isacharcolon}\isanewline
\ \ {\isachardoublequoteopen}setSubformulae\ {\isacharparenleft}Atom\ p{\isacharparenright}\ {\isacharequal}\ {\isacharbraceleft}Atom\ p{\isacharbraceright}{\isachardoublequoteclose}\isanewline
%
\isadelimproof
\ \ \ \ %
\endisadelimproof
%
\isatagproof
\isacommand{by}\isamarkupfalse%
\ {\isacharparenleft}simp\ only{\isacharcolon}\ subformulae{\isachardot}simps{\isacharparenleft}{\isadigit{1}}{\isacharparenright}\ list{\isachardot}set{\isacharparenright}%
\endisatagproof
{\isafoldproof}%
%
\isadelimproof
\isanewline
%
\endisadelimproof
\isanewline
\isacommand{lemma}\isamarkupfalse%
\ setSubformulae{\isacharunderscore}bot{\isacharcolon}\isanewline
\ \ {\isachardoublequoteopen}setSubformulae\ {\isacharparenleft}{\isasymbottom}{\isacharparenright}\ {\isacharequal}\ {\isacharbraceleft}{\isasymbottom}{\isacharbraceright}{\isachardoublequoteclose}\isanewline
%
\isadelimproof
\ \ \ \ %
\endisadelimproof
%
\isatagproof
\isacommand{by}\isamarkupfalse%
\ {\isacharparenleft}simp\ only{\isacharcolon}\ subformulae{\isachardot}simps{\isacharparenleft}{\isadigit{2}}{\isacharparenright}\ list{\isachardot}set{\isacharparenright}%
\endisatagproof
{\isafoldproof}%
%
\isadelimproof
\isanewline
%
\endisadelimproof
\isanewline
\isacommand{lemma}\isamarkupfalse%
\ setSubformulae{\isacharunderscore}not{\isacharcolon}\isanewline
\ \ \isakeyword{shows}\ {\isachardoublequoteopen}setSubformulae\ {\isacharparenleft}\isactrlbold {\isasymnot}\ F{\isacharparenright}\ {\isacharequal}\ {\isacharbraceleft}\isactrlbold {\isasymnot}\ F{\isacharbraceright}\ {\isasymunion}\ setSubformulae\ F{\isachardoublequoteclose}\isanewline
%
\isadelimproof
%
\endisadelimproof
%
\isatagproof
\isacommand{proof}\isamarkupfalse%
\ {\isacharminus}\isanewline
\ \ \isacommand{have}\isamarkupfalse%
\ {\isachardoublequoteopen}setSubformulae\ {\isacharparenleft}\isactrlbold {\isasymnot}\ F{\isacharparenright}\ {\isacharequal}\ set\ {\isacharparenleft}\isactrlbold {\isasymnot}\ F\ {\isacharhash}\ subformulae\ F{\isacharparenright}{\isachardoublequoteclose}\isanewline
\ \ \ \ \isacommand{by}\isamarkupfalse%
\ {\isacharparenleft}simp\ only{\isacharcolon}\ subformulae{\isachardot}simps{\isacharparenleft}{\isadigit{3}}{\isacharparenright}{\isacharparenright}\isanewline
\ \ \isacommand{also}\isamarkupfalse%
\ \isacommand{have}\isamarkupfalse%
\ {\isachardoublequoteopen}{\isasymdots}\ {\isacharequal}\ {\isacharbraceleft}\isactrlbold {\isasymnot}\ F{\isacharbraceright}\ {\isasymunion}\ set\ {\isacharparenleft}subformulae\ F{\isacharparenright}{\isachardoublequoteclose}\isanewline
\ \ \ \ \isacommand{by}\isamarkupfalse%
\ {\isacharparenleft}simp\ only{\isacharcolon}\ set{\isacharunderscore}insert{\isacharparenright}\isanewline
\ \ \isacommand{finally}\isamarkupfalse%
\ \isacommand{show}\isamarkupfalse%
\ {\isacharquery}thesis\isanewline
\ \ \ \ \isacommand{by}\isamarkupfalse%
\ this\isanewline
\isacommand{qed}\isamarkupfalse%
%
\endisatagproof
{\isafoldproof}%
%
\isadelimproof
\isanewline
%
\endisadelimproof
\isanewline
\isacommand{lemma}\isamarkupfalse%
\ setSubformulae{\isacharunderscore}and{\isacharcolon}\ \isanewline
\ \ {\isachardoublequoteopen}setSubformulae\ {\isacharparenleft}F{\isadigit{1}}\ \isactrlbold {\isasymand}\ F{\isadigit{2}}{\isacharparenright}\ \isanewline
\ \ \ {\isacharequal}\ {\isacharbraceleft}F{\isadigit{1}}\ \isactrlbold {\isasymand}\ F{\isadigit{2}}{\isacharbraceright}\ {\isasymunion}\ {\isacharparenleft}setSubformulae\ F{\isadigit{1}}\ {\isasymunion}\ setSubformulae\ F{\isadigit{2}}{\isacharparenright}{\isachardoublequoteclose}\isanewline
%
\isadelimproof
%
\endisadelimproof
%
\isatagproof
\isacommand{proof}\isamarkupfalse%
\ {\isacharminus}\isanewline
\ \ \isacommand{have}\isamarkupfalse%
\ {\isachardoublequoteopen}setSubformulae\ {\isacharparenleft}F{\isadigit{1}}\ \isactrlbold {\isasymand}\ F{\isadigit{2}}{\isacharparenright}\ \isanewline
\ \ \ \ \ \ \ \ {\isacharequal}\ set\ {\isacharparenleft}{\isacharparenleft}F{\isadigit{1}}\ \isactrlbold {\isasymand}\ F{\isadigit{2}}{\isacharparenright}\ {\isacharhash}\ {\isacharparenleft}subformulae\ F{\isadigit{1}}\ {\isacharat}\ subformulae\ F{\isadigit{2}}{\isacharparenright}{\isacharparenright}{\isachardoublequoteclose}\isanewline
\ \ \ \ \isacommand{by}\isamarkupfalse%
\ {\isacharparenleft}simp\ only{\isacharcolon}\ subformulae{\isachardot}simps{\isacharparenleft}{\isadigit{4}}{\isacharparenright}{\isacharparenright}\isanewline
\ \ \isacommand{also}\isamarkupfalse%
\ \isacommand{have}\isamarkupfalse%
\ {\isachardoublequoteopen}{\isasymdots}\ {\isacharequal}\ {\isacharbraceleft}F{\isadigit{1}}\ \isactrlbold {\isasymand}\ F{\isadigit{2}}{\isacharbraceright}\ {\isasymunion}\ {\isacharparenleft}set\ {\isacharparenleft}subformulae\ F{\isadigit{1}}\ {\isacharat}\ subformulae\ F{\isadigit{2}}{\isacharparenright}{\isacharparenright}{\isachardoublequoteclose}\isanewline
\ \ \ \ \isacommand{by}\isamarkupfalse%
\ {\isacharparenleft}simp\ only{\isacharcolon}\ set{\isacharunderscore}insert{\isacharparenright}\isanewline
\ \ \isacommand{also}\isamarkupfalse%
\ \isacommand{have}\isamarkupfalse%
\ {\isachardoublequoteopen}{\isasymdots}\ {\isacharequal}\ {\isacharbraceleft}F{\isadigit{1}}\ \isactrlbold {\isasymand}\ F{\isadigit{2}}{\isacharbraceright}\ {\isasymunion}\ {\isacharparenleft}setSubformulae\ F{\isadigit{1}}\ {\isasymunion}\ setSubformulae\ F{\isadigit{2}}{\isacharparenright}{\isachardoublequoteclose}\isanewline
\ \ \ \ \isacommand{by}\isamarkupfalse%
\ {\isacharparenleft}simp\ only{\isacharcolon}\ set{\isacharunderscore}append{\isacharparenright}\isanewline
\ \ \isacommand{finally}\isamarkupfalse%
\ \isacommand{show}\isamarkupfalse%
\ {\isacharquery}thesis\isanewline
\ \ \ \ \isacommand{by}\isamarkupfalse%
\ this\isanewline
\isacommand{qed}\isamarkupfalse%
%
\endisatagproof
{\isafoldproof}%
%
\isadelimproof
\isanewline
%
\endisadelimproof
\isanewline
\isacommand{lemma}\isamarkupfalse%
\ setSubformulae{\isacharunderscore}or{\isacharcolon}\ \isanewline
\ \ {\isachardoublequoteopen}setSubformulae\ {\isacharparenleft}F{\isadigit{1}}\ \isactrlbold {\isasymor}\ F{\isadigit{2}}{\isacharparenright}\ \isanewline
\ \ \ {\isacharequal}\ {\isacharbraceleft}F{\isadigit{1}}\ \isactrlbold {\isasymor}\ F{\isadigit{2}}{\isacharbraceright}\ {\isasymunion}\ {\isacharparenleft}setSubformulae\ F{\isadigit{1}}\ {\isasymunion}\ setSubformulae\ F{\isadigit{2}}{\isacharparenright}{\isachardoublequoteclose}\isanewline
%
\isadelimproof
%
\endisadelimproof
%
\isatagproof
\isacommand{proof}\isamarkupfalse%
\ {\isacharminus}\isanewline
\ \ \isacommand{have}\isamarkupfalse%
\ {\isachardoublequoteopen}setSubformulae\ {\isacharparenleft}F{\isadigit{1}}\ \isactrlbold {\isasymor}\ F{\isadigit{2}}{\isacharparenright}\ \isanewline
\ \ \ \ \ \ \ \ {\isacharequal}\ set\ {\isacharparenleft}{\isacharparenleft}F{\isadigit{1}}\ \isactrlbold {\isasymor}\ F{\isadigit{2}}{\isacharparenright}\ {\isacharhash}\ {\isacharparenleft}subformulae\ F{\isadigit{1}}\ {\isacharat}\ subformulae\ F{\isadigit{2}}{\isacharparenright}{\isacharparenright}{\isachardoublequoteclose}\isanewline
\ \ \ \ \isacommand{by}\isamarkupfalse%
\ {\isacharparenleft}simp\ only{\isacharcolon}\ subformulae{\isachardot}simps{\isacharparenleft}{\isadigit{5}}{\isacharparenright}{\isacharparenright}\isanewline
\ \ \isacommand{also}\isamarkupfalse%
\ \isacommand{have}\isamarkupfalse%
\ {\isachardoublequoteopen}{\isasymdots}\ {\isacharequal}\ {\isacharbraceleft}F{\isadigit{1}}\ \isactrlbold {\isasymor}\ F{\isadigit{2}}{\isacharbraceright}\ {\isasymunion}\ {\isacharparenleft}set\ {\isacharparenleft}subformulae\ F{\isadigit{1}}\ {\isacharat}\ subformulae\ F{\isadigit{2}}{\isacharparenright}{\isacharparenright}{\isachardoublequoteclose}\isanewline
\ \ \ \ \isacommand{by}\isamarkupfalse%
\ {\isacharparenleft}simp\ only{\isacharcolon}\ set{\isacharunderscore}insert{\isacharparenright}\isanewline
\ \ \isacommand{also}\isamarkupfalse%
\ \isacommand{have}\isamarkupfalse%
\ {\isachardoublequoteopen}{\isasymdots}\ {\isacharequal}\ {\isacharbraceleft}F{\isadigit{1}}\ \isactrlbold {\isasymor}\ F{\isadigit{2}}{\isacharbraceright}\ {\isasymunion}\ {\isacharparenleft}setSubformulae\ F{\isadigit{1}}\ {\isasymunion}\ setSubformulae\ F{\isadigit{2}}{\isacharparenright}{\isachardoublequoteclose}\isanewline
\ \ \ \ \isacommand{by}\isamarkupfalse%
\ {\isacharparenleft}simp\ only{\isacharcolon}\ set{\isacharunderscore}append{\isacharparenright}\isanewline
\ \ \isacommand{finally}\isamarkupfalse%
\ \isacommand{show}\isamarkupfalse%
\ {\isacharquery}thesis\isanewline
\ \ \ \ \isacommand{by}\isamarkupfalse%
\ this\isanewline
\isacommand{qed}\isamarkupfalse%
%
\endisatagproof
{\isafoldproof}%
%
\isadelimproof
\isanewline
%
\endisadelimproof
\isanewline
\isacommand{lemma}\isamarkupfalse%
\ setSubformulae{\isacharunderscore}imp{\isacharcolon}\ \isanewline
\ \ {\isachardoublequoteopen}setSubformulae\ {\isacharparenleft}F{\isadigit{1}}\ \isactrlbold {\isasymrightarrow}\ F{\isadigit{2}}{\isacharparenright}\ \isanewline
\ \ \ {\isacharequal}\ {\isacharbraceleft}F{\isadigit{1}}\ \isactrlbold {\isasymrightarrow}\ F{\isadigit{2}}{\isacharbraceright}\ {\isasymunion}\ {\isacharparenleft}setSubformulae\ F{\isadigit{1}}\ {\isasymunion}\ setSubformulae\ F{\isadigit{2}}{\isacharparenright}{\isachardoublequoteclose}\isanewline
%
\isadelimproof
%
\endisadelimproof
%
\isatagproof
\isacommand{proof}\isamarkupfalse%
\ {\isacharminus}\isanewline
\ \ \isacommand{have}\isamarkupfalse%
\ {\isachardoublequoteopen}setSubformulae\ {\isacharparenleft}F{\isadigit{1}}\ \isactrlbold {\isasymrightarrow}\ F{\isadigit{2}}{\isacharparenright}\ \isanewline
\ \ \ \ \ \ \ \ {\isacharequal}\ set\ {\isacharparenleft}{\isacharparenleft}F{\isadigit{1}}\ \isactrlbold {\isasymrightarrow}\ F{\isadigit{2}}{\isacharparenright}\ {\isacharhash}\ {\isacharparenleft}subformulae\ F{\isadigit{1}}\ {\isacharat}\ subformulae\ F{\isadigit{2}}{\isacharparenright}{\isacharparenright}{\isachardoublequoteclose}\isanewline
\ \ \ \ \isacommand{by}\isamarkupfalse%
\ {\isacharparenleft}simp\ only{\isacharcolon}\ subformulae{\isachardot}simps{\isacharparenleft}{\isadigit{6}}{\isacharparenright}{\isacharparenright}\isanewline
\ \ \isacommand{also}\isamarkupfalse%
\ \isacommand{have}\isamarkupfalse%
\ {\isachardoublequoteopen}{\isasymdots}\ {\isacharequal}\ {\isacharbraceleft}F{\isadigit{1}}\ \isactrlbold {\isasymrightarrow}\ F{\isadigit{2}}{\isacharbraceright}\ {\isasymunion}\ {\isacharparenleft}set\ {\isacharparenleft}subformulae\ F{\isadigit{1}}\ {\isacharat}\ subformulae\ F{\isadigit{2}}{\isacharparenright}{\isacharparenright}{\isachardoublequoteclose}\isanewline
\ \ \ \ \isacommand{by}\isamarkupfalse%
\ {\isacharparenleft}simp\ only{\isacharcolon}\ set{\isacharunderscore}insert{\isacharparenright}\isanewline
\ \ \isacommand{also}\isamarkupfalse%
\ \isacommand{have}\isamarkupfalse%
\ {\isachardoublequoteopen}{\isasymdots}\ {\isacharequal}\ {\isacharbraceleft}F{\isadigit{1}}\ \isactrlbold {\isasymrightarrow}\ F{\isadigit{2}}{\isacharbraceright}\ {\isasymunion}\ {\isacharparenleft}setSubformulae\ F{\isadigit{1}}\ {\isasymunion}\ setSubformulae\ F{\isadigit{2}}{\isacharparenright}{\isachardoublequoteclose}\isanewline
\ \ \ \ \isacommand{by}\isamarkupfalse%
\ {\isacharparenleft}simp\ only{\isacharcolon}\ set{\isacharunderscore}append{\isacharparenright}\isanewline
\ \ \isacommand{finally}\isamarkupfalse%
\ \isacommand{show}\isamarkupfalse%
\ {\isacharquery}thesis\isanewline
\ \ \ \ \isacommand{by}\isamarkupfalse%
\ this\isanewline
\isacommand{qed}\isamarkupfalse%
%
\endisatagproof
{\isafoldproof}%
%
\isadelimproof
%
\endisadelimproof
%
\begin{isamarkuptext}%
Una vez probada la equivalencia, comencemos con los resultados 
  correspondientes a las subfórmulas. En primer lugar, tenemos la 
  siguiente propiedad como consecuencia directa de la equivalencia de 
  funciones anterior.

  \comentario{Reescribir el siguiente enunciado y demostración.}

  \begin{lema}
    \isa{F\ {\isasymin}\ Subf{\isacharparenleft}F{\isacharparenright}}.
  \end{lema}

  \begin{demostracion}
    Por inducción en la estructura de las fórmulas. Se tienen los
    siguientes casos:
  
    Sea \isa{p} fórmula atómica cualquiera. Por definición de \isa{Subf} tenemos 
    que \isa{Subf{\isacharparenleft}p{\isacharparenright}\ {\isacharequal}\ {\isacharbraceleft}p{\isacharbraceright}}, luego se tiene la propiedad.
  
    Sea la fórmula \isa{{\isasymbottom}}. Como \isa{Subf{\isacharparenleft}{\isasymbottom}{\isacharparenright}\ {\isacharequal}\ {\isacharbraceleft}{\isasymbottom}{\isacharbraceright}}, se verifica el resultado.

    Por definición del conjunto de subfórmulas de \isa{Subf{\isacharparenleft}{\isasymnot}\ F{\isacharparenright}} se tiene 
    la propiedad para este caso, pues 
    \isa{Subf{\isacharparenleft}{\isasymnot}\ F{\isacharparenright}\ {\isacharequal}\ {\isacharbraceleft}{\isasymnot}\ F{\isacharbraceright}\ {\isasymunion}\ Subf{\isacharparenleft}F{\isacharparenright}\ {\isasymLongrightarrow}\ {\isasymnot}\ F\ {\isasymin}\ Subf{\isacharparenleft}{\isasymnot}\ F{\isacharparenright}} como queríamos 
    ver.

    Análogamente, para cualquier conectiva binaria \isa{{\isacharasterisk}} y fórmulas \isa{F} y 
    \isa{G} se cumple \isa{Subf{\isacharparenleft}F{\isacharasterisk}G{\isacharparenright}\ {\isacharequal}\ {\isacharbraceleft}F{\isacharasterisk}G{\isacharbraceright}\ {\isasymunion}\ Subf{\isacharparenleft}F{\isacharparenright}\ {\isasymunion}\ Subf{\isacharparenleft}G{\isacharparenright}}, luego se 
    cumple la propiedad.
  \end{demostracion}

  Formalicemos ahora el lema con su correspondiente demostración 
  detallada.%
\end{isamarkuptext}\isamarkuptrue%
\isacommand{lemma}\isamarkupfalse%
\ subformulae{\isacharunderscore}self{\isacharcolon}\ {\isachardoublequoteopen}F\ {\isasymin}\ setSubformulae\ F{\isachardoublequoteclose}\isanewline
%
\isadelimproof
%
\endisadelimproof
%
\isatagproof
\isacommand{proof}\isamarkupfalse%
\ {\isacharparenleft}induction\ F{\isacharparenright}\ \isanewline
\ \ \isacommand{case}\isamarkupfalse%
\ {\isacharparenleft}Atom\ x{\isacharparenright}\ \isanewline
\ \ \isacommand{then}\isamarkupfalse%
\ \isacommand{show}\isamarkupfalse%
\ {\isacharquery}case\ \isanewline
\ \ \ \ \isacommand{by}\isamarkupfalse%
\ {\isacharparenleft}simp\ only{\isacharcolon}\ singletonI\ setSubformulae{\isacharunderscore}atom{\isacharparenright}\isanewline
\isacommand{next}\isamarkupfalse%
\isanewline
\ \ \isacommand{case}\isamarkupfalse%
\ Bot\isanewline
\ \ \isacommand{then}\isamarkupfalse%
\ \isacommand{show}\isamarkupfalse%
\ {\isacharquery}case\ \isanewline
\ \ \ \ \isacommand{by}\isamarkupfalse%
\ {\isacharparenleft}simp\ only{\isacharcolon}\ singletonI\ setSubformulae{\isacharunderscore}bot{\isacharparenright}\isanewline
\isacommand{next}\isamarkupfalse%
\isanewline
\ \ \isacommand{case}\isamarkupfalse%
\ {\isacharparenleft}Not\ F{\isacharparenright}\isanewline
\ \ \isacommand{then}\isamarkupfalse%
\ \isacommand{show}\isamarkupfalse%
\ {\isacharquery}case\ \isanewline
\ \ \ \ \isacommand{by}\isamarkupfalse%
\ {\isacharparenleft}simp\ add{\isacharcolon}\ insertI{\isadigit{1}}\ setSubformulae{\isacharunderscore}not{\isacharparenright}\isanewline
\isacommand{next}\isamarkupfalse%
\isanewline
\isacommand{case}\isamarkupfalse%
\ {\isacharparenleft}And\ F{\isadigit{1}}\ F{\isadigit{2}}{\isacharparenright}\isanewline
\ \ \isacommand{then}\isamarkupfalse%
\ \isacommand{show}\isamarkupfalse%
\ {\isacharquery}case\ \isanewline
\ \ \ \ \isacommand{by}\isamarkupfalse%
\ {\isacharparenleft}simp\ add{\isacharcolon}\ insertI{\isadigit{1}}\ setSubformulae{\isacharunderscore}and{\isacharparenright}\isanewline
\isacommand{next}\isamarkupfalse%
\isanewline
\isacommand{case}\isamarkupfalse%
\ {\isacharparenleft}Or\ F{\isadigit{1}}\ F{\isadigit{2}}{\isacharparenright}\isanewline
\ \ \isacommand{then}\isamarkupfalse%
\ \isacommand{show}\isamarkupfalse%
\ {\isacharquery}case\ \isanewline
\ \ \ \ \isacommand{by}\isamarkupfalse%
\ {\isacharparenleft}simp\ add{\isacharcolon}\ insertI{\isadigit{1}}\ setSubformulae{\isacharunderscore}or{\isacharparenright}\isanewline
\isacommand{next}\isamarkupfalse%
\isanewline
\isacommand{case}\isamarkupfalse%
\ {\isacharparenleft}Imp\ F{\isadigit{1}}\ F{\isadigit{2}}{\isacharparenright}\isanewline
\ \ \isacommand{then}\isamarkupfalse%
\ \isacommand{show}\isamarkupfalse%
\ {\isacharquery}case\ \isanewline
\ \ \ \ \isacommand{by}\isamarkupfalse%
\ {\isacharparenleft}simp\ add{\isacharcolon}\ insertI{\isadigit{1}}\ setSubformulae{\isacharunderscore}imp{\isacharparenright}\isanewline
\isacommand{qed}\isamarkupfalse%
%
\endisatagproof
{\isafoldproof}%
%
\isadelimproof
%
\endisadelimproof
%
\begin{isamarkuptext}%
La demostración automática es la siguiente.%
\end{isamarkuptext}\isamarkuptrue%
\isacommand{lemma}\isamarkupfalse%
\ {\isachardoublequoteopen}F\ {\isasymin}\ setSubformulae\ F{\isachardoublequoteclose}\isanewline
%
\isadelimproof
\ \ %
\endisadelimproof
%
\isatagproof
\isacommand{by}\isamarkupfalse%
\ {\isacharparenleft}induction\ F{\isacharparenright}\ simp{\isacharunderscore}all%
\endisatagproof
{\isafoldproof}%
%
\isadelimproof
%
\endisadelimproof
%
\begin{isamarkuptext}%
Procedamos con los demás resultados de la sección. Como hemos 
  señalado con anterioridad, utilizaremos varias propiedades de 
  conjuntos pertenecientes a la teoría 
  \href{https://n9.cl/qatp}{Set.thy} de Isabelle, que apareceran en 
  el glosario final. 

  Además, definiremos dos reglas adicionales que utilizaremos con 
  frecuencia.%
\end{isamarkuptext}\isamarkuptrue%
\isacommand{lemma}\isamarkupfalse%
\ subContUnionRev{\isadigit{1}}{\isacharcolon}\ \isanewline
\ \ \isakeyword{assumes}\ {\isachardoublequoteopen}A\ {\isasymunion}\ B\ {\isasymsubseteq}\ C{\isachardoublequoteclose}\ \isanewline
\ \ \isakeyword{shows}\ \ \ {\isachardoublequoteopen}A\ {\isasymsubseteq}\ C{\isachardoublequoteclose}\isanewline
%
\isadelimproof
%
\endisadelimproof
%
\isatagproof
\isacommand{proof}\isamarkupfalse%
\ {\isacharminus}\isanewline
\ \ \isacommand{have}\isamarkupfalse%
\ {\isachardoublequoteopen}A\ {\isasymsubseteq}\ C\ {\isasymand}\ B\ {\isasymsubseteq}\ C{\isachardoublequoteclose}\isanewline
\ \ \ \ \isacommand{using}\isamarkupfalse%
\ assms\isanewline
\ \ \ \ \isacommand{by}\isamarkupfalse%
\ {\isacharparenleft}simp\ only{\isacharcolon}\ sup{\isachardot}bounded{\isacharunderscore}iff{\isacharparenright}\isanewline
\ \ \isacommand{then}\isamarkupfalse%
\ \isacommand{show}\isamarkupfalse%
\ {\isachardoublequoteopen}A\ {\isasymsubseteq}\ C{\isachardoublequoteclose}\isanewline
\ \ \ \ \isacommand{by}\isamarkupfalse%
\ {\isacharparenleft}rule\ conjunct{\isadigit{1}}{\isacharparenright}\isanewline
\isacommand{qed}\isamarkupfalse%
%
\endisatagproof
{\isafoldproof}%
%
\isadelimproof
\isanewline
%
\endisadelimproof
\isanewline
\isacommand{lemma}\isamarkupfalse%
\ subContUnionRev{\isadigit{2}}{\isacharcolon}\ \isanewline
\ \ \isakeyword{assumes}\ {\isachardoublequoteopen}A\ {\isasymunion}\ B\ {\isasymsubseteq}\ C{\isachardoublequoteclose}\ \isanewline
\ \ \isakeyword{shows}\ \ \ {\isachardoublequoteopen}B\ {\isasymsubseteq}\ C{\isachardoublequoteclose}\isanewline
%
\isadelimproof
%
\endisadelimproof
%
\isatagproof
\isacommand{proof}\isamarkupfalse%
\ {\isacharminus}\isanewline
\ \ \isacommand{have}\isamarkupfalse%
\ {\isachardoublequoteopen}A\ {\isasymsubseteq}\ C\ {\isasymand}\ B\ {\isasymsubseteq}\ C{\isachardoublequoteclose}\isanewline
\ \ \ \ \isacommand{using}\isamarkupfalse%
\ assms\isanewline
\ \ \ \ \isacommand{by}\isamarkupfalse%
\ {\isacharparenleft}simp\ only{\isacharcolon}\ sup{\isachardot}bounded{\isacharunderscore}iff{\isacharparenright}\isanewline
\ \ \isacommand{then}\isamarkupfalse%
\ \isacommand{show}\isamarkupfalse%
\ {\isachardoublequoteopen}B\ {\isasymsubseteq}\ C{\isachardoublequoteclose}\isanewline
\ \ \ \ \isacommand{by}\isamarkupfalse%
\ {\isacharparenleft}rule\ conjunct{\isadigit{2}}{\isacharparenright}\isanewline
\isacommand{qed}\isamarkupfalse%
%
\endisatagproof
{\isafoldproof}%
%
\isadelimproof
%
\endisadelimproof
%
\begin{isamarkuptext}%
Sus correspondientes demostraciones automáticas se muestran a 
  continuación.%
\end{isamarkuptext}\isamarkuptrue%
\isacommand{lemma}\isamarkupfalse%
\ {\isachardoublequoteopen}A\ {\isasymunion}\ B\ {\isasymsubseteq}\ C\ {\isasymLongrightarrow}\ A\ {\isasymsubseteq}\ C{\isachardoublequoteclose}\isanewline
%
\isadelimproof
\ \ %
\endisadelimproof
%
\isatagproof
\isacommand{by}\isamarkupfalse%
\ simp%
\endisatagproof
{\isafoldproof}%
%
\isadelimproof
\isanewline
%
\endisadelimproof
\isanewline
\isacommand{lemma}\isamarkupfalse%
\ {\isachardoublequoteopen}A\ {\isasymunion}\ B\ {\isasymsubseteq}\ C\ {\isasymLongrightarrow}\ B\ {\isasymsubseteq}\ C{\isachardoublequoteclose}\isanewline
%
\isadelimproof
\ \ %
\endisadelimproof
%
\isatagproof
\isacommand{by}\isamarkupfalse%
\ simp%
\endisatagproof
{\isafoldproof}%
%
\isadelimproof
%
\endisadelimproof
%
\begin{isamarkuptext}%
Veamos ahora los distintos resultados sobre subfórmulas.

  \comentario{Reescribir el siguiente enunciado y su demostración.}

  \begin{lema}
    Sean \isa{F} una fórmula proposicional y \isa{A\isactrlsub F} el conjunto de las 
    fórmulas atómicas formadas a partir de cada elemento del conjunto 
    de variables proposicionales de \isa{F}. 
    Entonces, \isa{A\isactrlsub F\ {\isasymsubseteq}\ Subf{\isacharparenleft}F{\isacharparenright}}.

    Por tanto, las fórmulas atómicas son subfórmulas.
  \end{lema}

  \begin{demostracion}
    La prueba seguirá el esquema inductivo para la estructura de 
    fórmulas. Veamos cada caso:
  
    Consideremos la fórmula atómica \isa{p} cualquiera. Entonces, su
    conjunto de átomos es \isa{{\isacharbraceleft}p{\isacharbraceright}}. De este modo, el conjunto \isa{A\isactrlsub p} 
    correspondiente será \isa{A\isactrlsub p\ {\isacharequal}\ {\isacharbraceleft}p{\isacharbraceright}\ {\isasymsubseteq}\ {\isacharbraceleft}p{\isacharbraceright}\ {\isacharequal}\ Subf{\isacharparenleft}Atom\ p{\isacharparenright}} como 
    queríamos 
    demostrar.

    Sea la fórmula \isa{{\isasymbottom}}. Como su connjunto de átomos es vacío, es claro 
    que \isa{A\isactrlsub {\isasymbottom}\ {\isacharequal}\ {\isasymemptyset}\ {\isasymsubseteq}\ Subf{\isacharparenleft}{\isasymbottom}{\isacharparenright}\ {\isacharequal}\ {\isasymemptyset}}.

    Sea la fórmula \isa{F} tal que \isa{A\isactrlsub F\ {\isasymsubseteq}\ Subf{\isacharparenleft}F{\isacharparenright}}. Probemos el resultado 
    para \isa{{\isasymnot}\ F}. Por definición tenemos que los conjunto de variables 
    proposicionales de \isa{F} y \isa{{\isasymnot}\ F} coinciden, luego \isa{A\isactrlsub {\isasymnot}\isactrlsub F\ {\isacharequal}\ A\isactrlsub F}. Además, 
    \isa{Subf{\isacharparenleft}{\isasymnot}\ F{\isacharparenright}\ {\isacharequal}\ {\isacharbraceleft}{\isasymnot}\ F{\isacharbraceright}\ {\isasymunion}\ Subf{\isacharparenleft}F{\isacharparenright}}. Por tanto, por hipótesis de 
    inducción tenemos:
    \isa{A\isactrlsub {\isasymnot}\isactrlsub F\ {\isacharequal}\ A\isactrlsub F\ {\isasymsubseteq}\ Subf{\isacharparenleft}F{\isacharparenright}\ {\isasymsubseteq}\ {\isacharbraceleft}{\isasymnot}\ F{\isacharbraceright}\ {\isasymunion}\ Subf{\isacharparenleft}F{\isacharparenright}\ {\isacharequal}\ Subf{\isacharparenleft}{\isasymnot}\ F{\isacharparenright}}, luego
    \isa{A\isactrlsub {\isasymnot}\isactrlsub F\ {\isasymsubseteq}\ Subf{\isacharparenleft}{\isasymnot}\ F{\isacharparenright}}.

    Sean las fórmulas \isa{F} y \isa{G} tales que \isa{A\isactrlsub F\ {\isasymsubseteq}\ Subf{\isacharparenleft}F{\isacharparenright}} y 
    \isa{A\isactrlsub G\ {\isasymsubseteq}\ Subf{\isacharparenleft}G{\isacharparenright}}. Probemos ahora \isa{A\isactrlsub F\isactrlsub {\isacharasterisk}\isactrlsub G\ {\isasymsubseteq}\ Subf{\isacharparenleft}F{\isacharasterisk}G{\isacharparenright}} para cualquier 
    conectiva binaria \isa{{\isacharasterisk}}. Por un lado, el conjunto de átomos de \isa{F{\isacharasterisk}G}
    es la unión de sus correspondientes conjuntos de átomos, luego 
    \isa{A\isactrlsub F\isactrlsub {\isacharasterisk}\isactrlsub G\ {\isacharequal}\ A\isactrlsub F\ {\isasymunion}\ A\isactrlsub G}. Por tanto, por hipótesis de inducción y definición 
    del conjunto de subfórmulas, se tiene:
    \isa{A\isactrlsub F\isactrlsub {\isacharasterisk}\isactrlsub G\ {\isacharequal}\ A\isactrlsub F\ {\isasymunion}\ A\isactrlsub G\ {\isasymsubseteq}\ Subf{\isacharparenleft}F{\isacharparenright}\ {\isasymunion}\ Subf{\isacharparenleft}G{\isacharparenright}\ {\isasymsubseteq}\ {\isacharbraceleft}F{\isacharasterisk}G{\isacharbraceright}\ {\isasymunion}\ Subf{\isacharparenleft}F{\isacharparenright}\ {\isasymunion}\ Subf{\isacharparenleft}G{\isacharparenright}\ {\isacharequal}\ Subf{\isacharparenleft}F{\isacharasterisk}G{\isacharparenright}}
    Luego, \isa{A\isactrlsub F\isactrlsub {\isacharasterisk}\isactrlsub G\ {\isasymsubseteq}\ Subf{\isacharparenleft}F{\isacharasterisk}G{\isacharparenright}} como queríamos demostrar.  
  \end{demostracion}

  En Isabelle, se especifica como sigue.%
\end{isamarkuptext}\isamarkuptrue%
\isacommand{lemma}\isamarkupfalse%
\ {\isachardoublequoteopen}Atom\ {\isacharbackquote}\ atoms\ F\ {\isasymsubseteq}\ setSubformulae\ F{\isachardoublequoteclose}\isanewline
%
\isadelimproof
\ \ %
\endisadelimproof
%
\isatagproof
\isacommand{oops}\isamarkupfalse%
%
\endisatagproof
{\isafoldproof}%
%
\isadelimproof
%
\endisadelimproof
%
\begin{isamarkuptext}%
Debemos observar que \isa{Atom\ {\isacharbackquote}\ atoms\ F} construye las fórmulas 
  atómicas a partir de cada uno de los elementos de \isa{atoms\ F}, creando 
  un conjunto de fórmulas atómicas. Dicho conjunto es equivalente al 
  conjunto \isa{A\isactrlsub F} del enunciado del lema. Para ello emplea el infijo \isa{{\isacharbackquote}} 
  definido como notación abreviada de \isa{{\isacharparenleft}{\isacharbackquote}{\isacharparenright}} que calcula la 
  imagen de un conjunto en la teoría \href{https://n9.cl/qatp}{Set.thy}.

  \begin{itemize}
    \item[] \isa{f\ {\isacharbackquote}\ A\ {\isacharequal}\ {\isacharbraceleft}y\ {\isacharbar}\ {\isasymexists}x{\isasymin}A{\isachardot}\ y\ {\isacharequal}\ f\ x{\isacharbraceright}} 
      \hfill (\isa{image{\isacharunderscore}def})
  \end{itemize}

  Para aclarar su funcionamiento, veamos ejemplos para distintos casos 
  de fórmulas.%
\end{isamarkuptext}\isamarkuptrue%
\isacommand{notepad}\isamarkupfalse%
\isanewline
\isakeyword{begin}\isanewline
%
\isadelimproof
\ \ %
\endisadelimproof
%
\isatagproof
\isacommand{fix}\isamarkupfalse%
\ p\ q\ r\ {\isacharcolon}{\isacharcolon}\ {\isacharprime}a\isanewline
\isanewline
\ \ \isacommand{have}\isamarkupfalse%
\ {\isachardoublequoteopen}Atom\ {\isacharbackquote}atoms\ {\isacharparenleft}Atom\ p\ \isactrlbold {\isasymor}\ {\isasymbottom}{\isacharparenright}\ {\isacharequal}\ {\isacharbraceleft}Atom\ p{\isacharbraceright}{\isachardoublequoteclose}\isanewline
\ \ \ \ \isacommand{by}\isamarkupfalse%
\ simp\isanewline
\isanewline
\ \ \isacommand{have}\isamarkupfalse%
\ {\isachardoublequoteopen}Atom\ {\isacharbackquote}atoms\ {\isacharparenleft}{\isacharparenleft}Atom\ p\ \isactrlbold {\isasymrightarrow}\ Atom\ q{\isacharparenright}\ \isactrlbold {\isasymor}\ Atom\ r{\isacharparenright}\ {\isacharequal}\ \isanewline
\ \ \ \ \ \ \ {\isacharbraceleft}Atom\ p{\isacharcomma}\ Atom\ q{\isacharcomma}\ Atom\ r{\isacharbraceright}{\isachardoublequoteclose}\isanewline
\ \ \ \ \isacommand{by}\isamarkupfalse%
\ auto\ \isanewline
\isanewline
\ \ \isacommand{have}\isamarkupfalse%
\ {\isachardoublequoteopen}Atom\ {\isacharbackquote}atoms\ {\isacharparenleft}{\isacharparenleft}Atom\ p\ \isactrlbold {\isasymrightarrow}\ Atom\ p{\isacharparenright}\ \isactrlbold {\isasymor}\ Atom\ r{\isacharparenright}\ {\isacharequal}\ \isanewline
\ \ \ \ \ \ \ {\isacharbraceleft}Atom\ p{\isacharcomma}\ Atom\ r{\isacharbraceright}{\isachardoublequoteclose}\isanewline
\ \ \ \ \isacommand{by}\isamarkupfalse%
\ auto%
\endisatagproof
{\isafoldproof}%
%
\isadelimproof
\isanewline
%
\endisadelimproof
\isacommand{end}\isamarkupfalse%
%
\begin{isamarkuptext}%
Además, esta función tiene las siguientes propiedades sobre 
  conjuntos que utilizaremos en la demostración.

  \begin{itemize}
    \item[] \isa{f\ {\isacharbackquote}\ {\isacharparenleft}A\ {\isasymunion}\ B{\isacharparenright}\ {\isacharequal}\ f\ {\isacharbackquote}\ A\ {\isasymunion}\ f\ {\isacharbackquote}\ B} 
      \hfill (\isa{image{\isacharunderscore}Un})
    \item[] \isa{f\ {\isacharbackquote}\ {\isacharparenleft}{\isacharbraceleft}a{\isacharbraceright}\ {\isasymunion}\ B{\isacharparenright}\ {\isacharequal}\ {\isacharbraceleft}f\ a{\isacharbraceright}\ {\isasymunion}\ f\ {\isacharbackquote}\ B} 
      \hfill (\isa{image{\isacharunderscore}insert})
    \item[] \isa{f\ {\isacharbackquote}\ {\isasymemptyset}\ {\isacharequal}\ {\isasymemptyset}} 
      \hfill (\isa{image{\isacharunderscore}empty})
  \end{itemize}

  Una vez hechas las aclaraciones necesarias, comencemos la demostración 
  estructurada. Esta seguirá el esquema inductivo señalado con 
  anterioridad.%
\end{isamarkuptext}\isamarkuptrue%
\isacommand{lemma}\isamarkupfalse%
\ atoms{\isacharunderscore}are{\isacharunderscore}subformulae{\isacharunderscore}atom{\isacharcolon}\ \isanewline
\ \ {\isachardoublequoteopen}Atom\ {\isacharbackquote}\ atoms\ {\isacharparenleft}Atom\ x{\isacharparenright}\ {\isasymsubseteq}\ setSubformulae\ {\isacharparenleft}Atom\ x{\isacharparenright}{\isachardoublequoteclose}\ \isanewline
%
\isadelimproof
%
\endisadelimproof
%
\isatagproof
\isacommand{proof}\isamarkupfalse%
\ {\isacharminus}\isanewline
\ \ \isacommand{have}\isamarkupfalse%
\ {\isachardoublequoteopen}Atom\ {\isacharbackquote}\ atoms\ {\isacharparenleft}Atom\ x{\isacharparenright}\ {\isacharequal}\ Atom\ {\isacharbackquote}\ {\isacharbraceleft}x{\isacharbraceright}{\isachardoublequoteclose}\isanewline
\ \ \ \ \isacommand{by}\isamarkupfalse%
\ {\isacharparenleft}simp\ only{\isacharcolon}\ formula{\isachardot}set{\isacharparenleft}{\isadigit{1}}{\isacharparenright}{\isacharparenright}\isanewline
\ \ \isacommand{also}\isamarkupfalse%
\ \isacommand{have}\isamarkupfalse%
\ {\isachardoublequoteopen}{\isasymdots}\ {\isacharequal}\ {\isacharbraceleft}Atom\ x{\isacharbraceright}{\isachardoublequoteclose}\isanewline
\ \ \ \ \isacommand{by}\isamarkupfalse%
\ {\isacharparenleft}simp\ only{\isacharcolon}\ image{\isacharunderscore}insert\ image{\isacharunderscore}empty{\isacharparenright}\isanewline
\ \ \isacommand{also}\isamarkupfalse%
\ \isacommand{have}\isamarkupfalse%
\ {\isachardoublequoteopen}{\isasymdots}\ {\isacharequal}\ set\ {\isacharbrackleft}Atom\ x{\isacharbrackright}{\isachardoublequoteclose}\isanewline
\ \ \ \ \isacommand{by}\isamarkupfalse%
\ {\isacharparenleft}simp\ only{\isacharcolon}\ list{\isachardot}set{\isacharparenleft}{\isadigit{1}}{\isacharparenright}\ list{\isachardot}set{\isacharparenleft}{\isadigit{2}}{\isacharparenright}{\isacharparenright}\isanewline
\ \ \isacommand{also}\isamarkupfalse%
\ \isacommand{have}\isamarkupfalse%
\ {\isachardoublequoteopen}{\isasymdots}\ {\isacharequal}\ set\ {\isacharparenleft}subformulae\ {\isacharparenleft}Atom\ x{\isacharparenright}{\isacharparenright}{\isachardoublequoteclose}\isanewline
\ \ \ \ \isacommand{by}\isamarkupfalse%
\ {\isacharparenleft}simp\ only{\isacharcolon}\ subformulae{\isachardot}simps{\isacharparenleft}{\isadigit{1}}{\isacharparenright}{\isacharparenright}\isanewline
\ \ \isacommand{finally}\isamarkupfalse%
\ \isacommand{have}\isamarkupfalse%
\ {\isachardoublequoteopen}Atom\ {\isacharbackquote}\ atoms\ {\isacharparenleft}Atom\ x{\isacharparenright}\ {\isacharequal}\ set\ {\isacharparenleft}subformulae\ {\isacharparenleft}Atom\ x{\isacharparenright}{\isacharparenright}{\isachardoublequoteclose}\isanewline
\ \ \ \ \isacommand{by}\isamarkupfalse%
\ this\isanewline
\ \ \isacommand{then}\isamarkupfalse%
\ \isacommand{show}\isamarkupfalse%
\ {\isacharquery}thesis\ \isanewline
\ \ \ \ \isacommand{by}\isamarkupfalse%
\ {\isacharparenleft}simp\ only{\isacharcolon}\ subset{\isacharunderscore}refl{\isacharparenright}\isanewline
\isacommand{qed}\isamarkupfalse%
%
\endisatagproof
{\isafoldproof}%
%
\isadelimproof
\isanewline
%
\endisadelimproof
\isanewline
\isacommand{lemma}\isamarkupfalse%
\ atoms{\isacharunderscore}are{\isacharunderscore}subformulae{\isacharunderscore}bot{\isacharcolon}\ \isanewline
\ \ {\isachardoublequoteopen}Atom\ {\isacharbackquote}\ atoms\ {\isasymbottom}\ {\isasymsubseteq}\ setSubformulae\ {\isasymbottom}{\isachardoublequoteclose}\ \ \isanewline
%
\isadelimproof
%
\endisadelimproof
%
\isatagproof
\isacommand{proof}\isamarkupfalse%
\ {\isacharminus}\isanewline
\ \ \isacommand{have}\isamarkupfalse%
\ {\isachardoublequoteopen}Atom\ {\isacharbackquote}\ atoms\ {\isasymbottom}\ {\isacharequal}\ Atom\ {\isacharbackquote}\ {\isasymemptyset}{\isachardoublequoteclose}\isanewline
\ \ \ \ \isacommand{by}\isamarkupfalse%
\ {\isacharparenleft}simp\ only{\isacharcolon}\ formula{\isachardot}set{\isacharparenleft}{\isadigit{2}}{\isacharparenright}{\isacharparenright}\isanewline
\ \ \isacommand{also}\isamarkupfalse%
\ \isacommand{have}\isamarkupfalse%
\ {\isachardoublequoteopen}{\isasymdots}\ {\isacharequal}\ {\isasymemptyset}{\isachardoublequoteclose}\isanewline
\ \ \ \ \isacommand{by}\isamarkupfalse%
\ {\isacharparenleft}simp\ only{\isacharcolon}\ image{\isacharunderscore}empty{\isacharparenright}\isanewline
\ \ \isacommand{also}\isamarkupfalse%
\ \isacommand{have}\isamarkupfalse%
\ {\isachardoublequoteopen}{\isasymdots}\ {\isasymsubseteq}\ setSubformulae\ {\isasymbottom}{\isachardoublequoteclose}\isanewline
\ \ \ \ \isacommand{by}\isamarkupfalse%
\ {\isacharparenleft}simp\ only{\isacharcolon}\ empty{\isacharunderscore}subsetI{\isacharparenright}\isanewline
\ \ \isacommand{finally}\isamarkupfalse%
\ \isacommand{show}\isamarkupfalse%
\ {\isacharquery}thesis\isanewline
\ \ \ \ \isacommand{by}\isamarkupfalse%
\ this\isanewline
\isacommand{qed}\isamarkupfalse%
%
\endisatagproof
{\isafoldproof}%
%
\isadelimproof
\isanewline
%
\endisadelimproof
\isanewline
\isacommand{lemma}\isamarkupfalse%
\ atoms{\isacharunderscore}are{\isacharunderscore}subformulae{\isacharunderscore}not{\isacharcolon}\ \isanewline
\ \ \isakeyword{assumes}\ {\isachardoublequoteopen}Atom\ {\isacharbackquote}\ atoms\ F\ {\isasymsubseteq}\ setSubformulae\ F{\isachardoublequoteclose}\ \isanewline
\ \ \isakeyword{shows}\ \ \ {\isachardoublequoteopen}Atom\ {\isacharbackquote}\ atoms\ {\isacharparenleft}\isactrlbold {\isasymnot}\ F{\isacharparenright}\ {\isasymsubseteq}\ setSubformulae\ {\isacharparenleft}\isactrlbold {\isasymnot}\ F{\isacharparenright}{\isachardoublequoteclose}\isanewline
%
\isadelimproof
%
\endisadelimproof
%
\isatagproof
\isacommand{proof}\isamarkupfalse%
\ {\isacharminus}\isanewline
\ \ \isacommand{have}\isamarkupfalse%
\ {\isachardoublequoteopen}Atom\ {\isacharbackquote}\ atoms\ {\isacharparenleft}\isactrlbold {\isasymnot}\ F{\isacharparenright}\ {\isacharequal}\ Atom\ {\isacharbackquote}\ atoms\ F{\isachardoublequoteclose}\isanewline
\ \ \ \ \isacommand{by}\isamarkupfalse%
\ {\isacharparenleft}simp\ only{\isacharcolon}\ formula{\isachardot}set{\isacharparenleft}{\isadigit{3}}{\isacharparenright}{\isacharparenright}\isanewline
\ \ \isacommand{also}\isamarkupfalse%
\ \isacommand{have}\isamarkupfalse%
\ {\isachardoublequoteopen}{\isasymdots}\ {\isasymsubseteq}\ setSubformulae\ F{\isachardoublequoteclose}\isanewline
\ \ \ \ \isacommand{by}\isamarkupfalse%
\ {\isacharparenleft}simp\ only{\isacharcolon}\ assms{\isacharparenright}\isanewline
\ \ \isacommand{also}\isamarkupfalse%
\ \isacommand{have}\isamarkupfalse%
\ {\isachardoublequoteopen}{\isasymdots}\ {\isasymsubseteq}\ {\isacharbraceleft}\isactrlbold {\isasymnot}\ F{\isacharbraceright}\ {\isasymunion}\ setSubformulae\ F{\isachardoublequoteclose}\isanewline
\ \ \ \ \isacommand{by}\isamarkupfalse%
\ {\isacharparenleft}simp\ only{\isacharcolon}\ Un{\isacharunderscore}upper{\isadigit{2}}{\isacharparenright}\isanewline
\ \ \isacommand{also}\isamarkupfalse%
\ \isacommand{have}\isamarkupfalse%
\ {\isachardoublequoteopen}{\isasymdots}\ {\isacharequal}\ setSubformulae\ {\isacharparenleft}\isactrlbold {\isasymnot}\ F{\isacharparenright}{\isachardoublequoteclose}\isanewline
\ \ \ \ \isacommand{by}\isamarkupfalse%
\ {\isacharparenleft}simp\ only{\isacharcolon}\ setSubformulae{\isacharunderscore}not{\isacharparenright}\isanewline
\ \ \isacommand{finally}\isamarkupfalse%
\ \isacommand{show}\isamarkupfalse%
\ {\isacharquery}thesis\isanewline
\ \ \ \ \isacommand{by}\isamarkupfalse%
\ this\isanewline
\isacommand{qed}\isamarkupfalse%
%
\endisatagproof
{\isafoldproof}%
%
\isadelimproof
\isanewline
%
\endisadelimproof
\isanewline
\isacommand{lemma}\isamarkupfalse%
\ atoms{\isacharunderscore}are{\isacharunderscore}subformulae{\isacharunderscore}and{\isacharcolon}\ \isanewline
\ \ \isakeyword{assumes}\ {\isachardoublequoteopen}Atom\ {\isacharbackquote}\ atoms\ F{\isadigit{1}}\ {\isasymsubseteq}\ setSubformulae\ F{\isadigit{1}}{\isachardoublequoteclose}\isanewline
\ \ \ \ \ \ \ \ \ \ {\isachardoublequoteopen}Atom\ {\isacharbackquote}\ atoms\ F{\isadigit{2}}\ {\isasymsubseteq}\ setSubformulae\ F{\isadigit{2}}{\isachardoublequoteclose}\isanewline
\ \ \isakeyword{shows}\ \ \ {\isachardoublequoteopen}Atom\ {\isacharbackquote}\ atoms\ {\isacharparenleft}F{\isadigit{1}}\ \isactrlbold {\isasymand}\ F{\isadigit{2}}{\isacharparenright}\ {\isasymsubseteq}\ setSubformulae\ {\isacharparenleft}F{\isadigit{1}}\ \isactrlbold {\isasymand}\ F{\isadigit{2}}{\isacharparenright}{\isachardoublequoteclose}\isanewline
%
\isadelimproof
%
\endisadelimproof
%
\isatagproof
\isacommand{proof}\isamarkupfalse%
\ {\isacharminus}\isanewline
\ \ \isacommand{have}\isamarkupfalse%
\ {\isachardoublequoteopen}Atom\ {\isacharbackquote}\ atoms\ {\isacharparenleft}F{\isadigit{1}}\ \isactrlbold {\isasymand}\ F{\isadigit{2}}{\isacharparenright}\ {\isacharequal}\ Atom\ {\isacharbackquote}\ {\isacharparenleft}atoms\ F{\isadigit{1}}\ {\isasymunion}\ atoms\ F{\isadigit{2}}{\isacharparenright}{\isachardoublequoteclose}\isanewline
\ \ \ \ \isacommand{by}\isamarkupfalse%
\ {\isacharparenleft}simp\ only{\isacharcolon}\ formula{\isachardot}set{\isacharparenleft}{\isadigit{4}}{\isacharparenright}{\isacharparenright}\isanewline
\ \ \isacommand{also}\isamarkupfalse%
\ \isacommand{have}\isamarkupfalse%
\ {\isachardoublequoteopen}{\isasymdots}\ {\isacharequal}\ Atom\ {\isacharbackquote}\ atoms\ F{\isadigit{1}}\ {\isasymunion}\ Atom\ {\isacharbackquote}\ atoms\ F{\isadigit{2}}{\isachardoublequoteclose}\ \isanewline
\ \ \ \ \isacommand{by}\isamarkupfalse%
\ {\isacharparenleft}rule\ image{\isacharunderscore}Un{\isacharparenright}\isanewline
\ \ \isacommand{also}\isamarkupfalse%
\ \isacommand{have}\isamarkupfalse%
\ {\isachardoublequoteopen}{\isasymdots}\ {\isasymsubseteq}\ setSubformulae\ F{\isadigit{1}}\ {\isasymunion}\ setSubformulae\ F{\isadigit{2}}{\isachardoublequoteclose}\isanewline
\ \ \ \ \isacommand{using}\isamarkupfalse%
\ assms\isanewline
\ \ \ \ \isacommand{by}\isamarkupfalse%
\ {\isacharparenleft}rule\ Un{\isacharunderscore}mono{\isacharparenright}\isanewline
\ \ \isacommand{also}\isamarkupfalse%
\ \isacommand{have}\isamarkupfalse%
\ {\isachardoublequoteopen}{\isasymdots}\ {\isasymsubseteq}\ {\isacharbraceleft}F{\isadigit{1}}\ \isactrlbold {\isasymand}\ F{\isadigit{2}}{\isacharbraceright}\ {\isasymunion}\ {\isacharparenleft}setSubformulae\ F{\isadigit{1}}\ {\isasymunion}\ setSubformulae\ F{\isadigit{2}}{\isacharparenright}{\isachardoublequoteclose}\isanewline
\ \ \ \ \isacommand{by}\isamarkupfalse%
\ {\isacharparenleft}simp\ only{\isacharcolon}\ Un{\isacharunderscore}upper{\isadigit{2}}{\isacharparenright}\isanewline
\ \ \isacommand{also}\isamarkupfalse%
\ \isacommand{have}\isamarkupfalse%
\ {\isachardoublequoteopen}{\isasymdots}\ {\isacharequal}\ setSubformulae\ {\isacharparenleft}F{\isadigit{1}}\ \isactrlbold {\isasymand}\ F{\isadigit{2}}{\isacharparenright}{\isachardoublequoteclose}\isanewline
\ \ \ \ \isacommand{by}\isamarkupfalse%
\ {\isacharparenleft}simp\ only{\isacharcolon}\ setSubformulae{\isacharunderscore}and{\isacharparenright}\isanewline
\ \ \isacommand{finally}\isamarkupfalse%
\ \isacommand{show}\isamarkupfalse%
\ {\isacharquery}thesis\isanewline
\ \ \ \ \isacommand{by}\isamarkupfalse%
\ this\isanewline
\isacommand{qed}\isamarkupfalse%
%
\endisatagproof
{\isafoldproof}%
%
\isadelimproof
\isanewline
%
\endisadelimproof
\isanewline
\isacommand{lemma}\isamarkupfalse%
\ atoms{\isacharunderscore}are{\isacharunderscore}subformulae{\isacharunderscore}or{\isacharcolon}\ \isanewline
\ \ \isakeyword{assumes}\ {\isachardoublequoteopen}Atom\ {\isacharbackquote}\ atoms\ F{\isadigit{1}}\ {\isasymsubseteq}\ setSubformulae\ F{\isadigit{1}}{\isachardoublequoteclose}\isanewline
\ \ \ \ \ \ \ \ \ \ {\isachardoublequoteopen}Atom\ {\isacharbackquote}\ atoms\ F{\isadigit{2}}\ {\isasymsubseteq}\ setSubformulae\ F{\isadigit{2}}{\isachardoublequoteclose}\isanewline
\ \ \isakeyword{shows}\ \ \ {\isachardoublequoteopen}Atom\ {\isacharbackquote}\ atoms\ {\isacharparenleft}F{\isadigit{1}}\ \isactrlbold {\isasymor}\ F{\isadigit{2}}{\isacharparenright}\ {\isasymsubseteq}\ setSubformulae\ {\isacharparenleft}F{\isadigit{1}}\ \isactrlbold {\isasymor}\ F{\isadigit{2}}{\isacharparenright}{\isachardoublequoteclose}\isanewline
%
\isadelimproof
%
\endisadelimproof
%
\isatagproof
\isacommand{proof}\isamarkupfalse%
\ {\isacharminus}\isanewline
\ \ \isacommand{have}\isamarkupfalse%
\ {\isachardoublequoteopen}Atom\ {\isacharbackquote}\ atoms\ {\isacharparenleft}F{\isadigit{1}}\ \isactrlbold {\isasymor}\ F{\isadigit{2}}{\isacharparenright}\ {\isacharequal}\ Atom\ {\isacharbackquote}\ {\isacharparenleft}atoms\ F{\isadigit{1}}\ {\isasymunion}\ atoms\ F{\isadigit{2}}{\isacharparenright}{\isachardoublequoteclose}\isanewline
\ \ \ \ \isacommand{by}\isamarkupfalse%
\ {\isacharparenleft}simp\ only{\isacharcolon}\ formula{\isachardot}set{\isacharparenleft}{\isadigit{5}}{\isacharparenright}{\isacharparenright}\isanewline
\ \ \isacommand{also}\isamarkupfalse%
\ \isacommand{have}\isamarkupfalse%
\ {\isachardoublequoteopen}{\isasymdots}\ {\isacharequal}\ Atom\ {\isacharbackquote}\ atoms\ F{\isadigit{1}}\ {\isasymunion}\ Atom\ {\isacharbackquote}\ atoms\ F{\isadigit{2}}{\isachardoublequoteclose}\ \isanewline
\ \ \ \ \isacommand{by}\isamarkupfalse%
\ {\isacharparenleft}rule\ image{\isacharunderscore}Un{\isacharparenright}\isanewline
\ \ \isacommand{also}\isamarkupfalse%
\ \isacommand{have}\isamarkupfalse%
\ {\isachardoublequoteopen}{\isasymdots}\ {\isasymsubseteq}\ setSubformulae\ F{\isadigit{1}}\ {\isasymunion}\ setSubformulae\ F{\isadigit{2}}{\isachardoublequoteclose}\isanewline
\ \ \ \ \isacommand{using}\isamarkupfalse%
\ assms\isanewline
\ \ \ \ \isacommand{by}\isamarkupfalse%
\ {\isacharparenleft}rule\ Un{\isacharunderscore}mono{\isacharparenright}\isanewline
\ \ \isacommand{also}\isamarkupfalse%
\ \isacommand{have}\isamarkupfalse%
\ {\isachardoublequoteopen}{\isasymdots}\ {\isasymsubseteq}\ {\isacharbraceleft}F{\isadigit{1}}\ \isactrlbold {\isasymor}\ F{\isadigit{2}}{\isacharbraceright}\ {\isasymunion}\ {\isacharparenleft}setSubformulae\ F{\isadigit{1}}\ {\isasymunion}\ setSubformulae\ F{\isadigit{2}}{\isacharparenright}{\isachardoublequoteclose}\isanewline
\ \ \ \ \isacommand{by}\isamarkupfalse%
\ {\isacharparenleft}simp\ only{\isacharcolon}\ Un{\isacharunderscore}upper{\isadigit{2}}{\isacharparenright}\isanewline
\ \ \isacommand{also}\isamarkupfalse%
\ \isacommand{have}\isamarkupfalse%
\ {\isachardoublequoteopen}{\isasymdots}\ {\isacharequal}\ setSubformulae\ {\isacharparenleft}F{\isadigit{1}}\ \isactrlbold {\isasymor}\ F{\isadigit{2}}{\isacharparenright}{\isachardoublequoteclose}\isanewline
\ \ \ \ \isacommand{by}\isamarkupfalse%
\ {\isacharparenleft}simp\ only{\isacharcolon}\ setSubformulae{\isacharunderscore}or{\isacharparenright}\isanewline
\ \ \isacommand{finally}\isamarkupfalse%
\ \isacommand{show}\isamarkupfalse%
\ {\isacharquery}thesis\isanewline
\ \ \ \ \isacommand{by}\isamarkupfalse%
\ this\isanewline
\isacommand{qed}\isamarkupfalse%
%
\endisatagproof
{\isafoldproof}%
%
\isadelimproof
\isanewline
%
\endisadelimproof
\isanewline
\isacommand{lemma}\isamarkupfalse%
\ atoms{\isacharunderscore}are{\isacharunderscore}subformulae{\isacharunderscore}imp{\isacharcolon}\ \isanewline
\ \ \isakeyword{assumes}\ {\isachardoublequoteopen}Atom\ {\isacharbackquote}\ atoms\ F{\isadigit{1}}\ {\isasymsubseteq}\ setSubformulae\ F{\isadigit{1}}{\isachardoublequoteclose}\isanewline
\ \ \ \ \ \ \ \ \ \ {\isachardoublequoteopen}Atom\ {\isacharbackquote}\ atoms\ F{\isadigit{2}}\ {\isasymsubseteq}\ setSubformulae\ F{\isadigit{2}}{\isachardoublequoteclose}\isanewline
\ \ \isakeyword{shows}\ \ \ {\isachardoublequoteopen}Atom\ {\isacharbackquote}\ atoms\ {\isacharparenleft}F{\isadigit{1}}\ \isactrlbold {\isasymrightarrow}\ F{\isadigit{2}}{\isacharparenright}\ {\isasymsubseteq}\ setSubformulae\ {\isacharparenleft}F{\isadigit{1}}\ \isactrlbold {\isasymrightarrow}\ F{\isadigit{2}}{\isacharparenright}{\isachardoublequoteclose}\isanewline
%
\isadelimproof
%
\endisadelimproof
%
\isatagproof
\isacommand{proof}\isamarkupfalse%
\ {\isacharminus}\isanewline
\ \ \isacommand{have}\isamarkupfalse%
\ {\isachardoublequoteopen}Atom\ {\isacharbackquote}\ atoms\ {\isacharparenleft}F{\isadigit{1}}\ \isactrlbold {\isasymrightarrow}\ F{\isadigit{2}}{\isacharparenright}\ {\isacharequal}\ Atom\ {\isacharbackquote}\ {\isacharparenleft}atoms\ F{\isadigit{1}}\ {\isasymunion}\ atoms\ F{\isadigit{2}}{\isacharparenright}{\isachardoublequoteclose}\isanewline
\ \ \ \ \isacommand{by}\isamarkupfalse%
\ {\isacharparenleft}simp\ only{\isacharcolon}\ formula{\isachardot}set{\isacharparenleft}{\isadigit{6}}{\isacharparenright}{\isacharparenright}\isanewline
\ \ \isacommand{also}\isamarkupfalse%
\ \isacommand{have}\isamarkupfalse%
\ {\isachardoublequoteopen}{\isasymdots}\ {\isacharequal}\ Atom\ {\isacharbackquote}\ atoms\ F{\isadigit{1}}\ {\isasymunion}\ Atom\ {\isacharbackquote}\ atoms\ F{\isadigit{2}}{\isachardoublequoteclose}\ \isanewline
\ \ \ \ \isacommand{by}\isamarkupfalse%
\ {\isacharparenleft}rule\ image{\isacharunderscore}Un{\isacharparenright}\isanewline
\ \ \isacommand{also}\isamarkupfalse%
\ \isacommand{have}\isamarkupfalse%
\ {\isachardoublequoteopen}{\isasymdots}\ {\isasymsubseteq}\ setSubformulae\ F{\isadigit{1}}\ {\isasymunion}\ setSubformulae\ F{\isadigit{2}}{\isachardoublequoteclose}\isanewline
\ \ \ \ \isacommand{using}\isamarkupfalse%
\ assms\isanewline
\ \ \ \ \isacommand{by}\isamarkupfalse%
\ {\isacharparenleft}rule\ Un{\isacharunderscore}mono{\isacharparenright}\isanewline
\ \ \isacommand{also}\isamarkupfalse%
\ \isacommand{have}\isamarkupfalse%
\ {\isachardoublequoteopen}{\isasymdots}\ {\isasymsubseteq}\ {\isacharbraceleft}F{\isadigit{1}}\ \isactrlbold {\isasymrightarrow}\ F{\isadigit{2}}{\isacharbraceright}\ {\isasymunion}\ {\isacharparenleft}setSubformulae\ F{\isadigit{1}}\ {\isasymunion}\ setSubformulae\ F{\isadigit{2}}{\isacharparenright}{\isachardoublequoteclose}\isanewline
\ \ \ \ \isacommand{by}\isamarkupfalse%
\ {\isacharparenleft}simp\ only{\isacharcolon}\ Un{\isacharunderscore}upper{\isadigit{2}}{\isacharparenright}\isanewline
\ \ \isacommand{also}\isamarkupfalse%
\ \isacommand{have}\isamarkupfalse%
\ {\isachardoublequoteopen}{\isasymdots}\ {\isacharequal}\ setSubformulae\ {\isacharparenleft}F{\isadigit{1}}\ \isactrlbold {\isasymrightarrow}\ F{\isadigit{2}}{\isacharparenright}{\isachardoublequoteclose}\isanewline
\ \ \ \ \isacommand{by}\isamarkupfalse%
\ {\isacharparenleft}simp\ only{\isacharcolon}\ setSubformulae{\isacharunderscore}imp{\isacharparenright}\isanewline
\ \ \isacommand{finally}\isamarkupfalse%
\ \isacommand{show}\isamarkupfalse%
\ {\isacharquery}thesis\isanewline
\ \ \ \ \isacommand{by}\isamarkupfalse%
\ this\isanewline
\isacommand{qed}\isamarkupfalse%
%
\endisatagproof
{\isafoldproof}%
%
\isadelimproof
\isanewline
%
\endisadelimproof
\isanewline
\isacommand{lemma}\isamarkupfalse%
\ atoms{\isacharunderscore}are{\isacharunderscore}subformulae{\isacharcolon}\ \isanewline
\ \ {\isachardoublequoteopen}Atom\ {\isacharbackquote}\ atoms\ F\ {\isasymsubseteq}\ setSubformulae\ F{\isachardoublequoteclose}\isanewline
%
\isadelimproof
%
\endisadelimproof
%
\isatagproof
\isacommand{proof}\isamarkupfalse%
\ {\isacharparenleft}induction\ F{\isacharparenright}\isanewline
\ \ \isacommand{case}\isamarkupfalse%
\ {\isacharparenleft}Atom\ x{\isacharparenright}\isanewline
\ \ \isacommand{then}\isamarkupfalse%
\ \isacommand{show}\isamarkupfalse%
\ {\isacharquery}case\ \isacommand{by}\isamarkupfalse%
\ {\isacharparenleft}simp\ only{\isacharcolon}\ atoms{\isacharunderscore}are{\isacharunderscore}subformulae{\isacharunderscore}atom{\isacharparenright}\ \isanewline
\isacommand{next}\isamarkupfalse%
\isanewline
\ \ \isacommand{case}\isamarkupfalse%
\ Bot\isanewline
\ \ \isacommand{then}\isamarkupfalse%
\ \isacommand{show}\isamarkupfalse%
\ {\isacharquery}case\ \isacommand{by}\isamarkupfalse%
\ {\isacharparenleft}simp\ only{\isacharcolon}\ atoms{\isacharunderscore}are{\isacharunderscore}subformulae{\isacharunderscore}bot{\isacharparenright}\ \isanewline
\isacommand{next}\isamarkupfalse%
\isanewline
\ \ \isacommand{case}\isamarkupfalse%
\ {\isacharparenleft}Not\ F{\isacharparenright}\isanewline
\ \ \isacommand{then}\isamarkupfalse%
\ \isacommand{show}\isamarkupfalse%
\ {\isacharquery}case\ \isacommand{by}\isamarkupfalse%
\ {\isacharparenleft}simp\ only{\isacharcolon}\ atoms{\isacharunderscore}are{\isacharunderscore}subformulae{\isacharunderscore}not{\isacharparenright}\ \isanewline
\isacommand{next}\isamarkupfalse%
\isanewline
\ \ \isacommand{case}\isamarkupfalse%
\ {\isacharparenleft}And\ F{\isadigit{1}}\ F{\isadigit{2}}{\isacharparenright}\isanewline
\ \ \isacommand{then}\isamarkupfalse%
\ \isacommand{show}\isamarkupfalse%
\ {\isacharquery}case\ \isacommand{by}\isamarkupfalse%
\ {\isacharparenleft}simp\ only{\isacharcolon}\ atoms{\isacharunderscore}are{\isacharunderscore}subformulae{\isacharunderscore}and{\isacharparenright}\ \isanewline
\isacommand{next}\isamarkupfalse%
\isanewline
\ \ \isacommand{case}\isamarkupfalse%
\ {\isacharparenleft}Or\ F{\isadigit{1}}\ F{\isadigit{2}}{\isacharparenright}\isanewline
\ \ \isacommand{then}\isamarkupfalse%
\ \isacommand{show}\isamarkupfalse%
\ {\isacharquery}case\ \isacommand{by}\isamarkupfalse%
\ {\isacharparenleft}simp\ only{\isacharcolon}\ atoms{\isacharunderscore}are{\isacharunderscore}subformulae{\isacharunderscore}or{\isacharparenright}\isanewline
\isacommand{next}\isamarkupfalse%
\isanewline
\ \ \isacommand{case}\isamarkupfalse%
\ {\isacharparenleft}Imp\ F{\isadigit{1}}\ F{\isadigit{2}}{\isacharparenright}\isanewline
\ \ \isacommand{then}\isamarkupfalse%
\ \isacommand{show}\isamarkupfalse%
\ {\isacharquery}case\ \isacommand{by}\isamarkupfalse%
\ {\isacharparenleft}simp\ only{\isacharcolon}\ atoms{\isacharunderscore}are{\isacharunderscore}subformulae{\isacharunderscore}imp{\isacharparenright}\isanewline
\isacommand{qed}\isamarkupfalse%
%
\endisatagproof
{\isafoldproof}%
%
\isadelimproof
%
\endisadelimproof
%
\begin{isamarkuptext}%
La demostración automática queda igualmente expuesta a 
  continuación.%
\end{isamarkuptext}\isamarkuptrue%
\isacommand{lemma}\isamarkupfalse%
\ {\isachardoublequoteopen}Atom\ {\isacharbackquote}\ atoms\ F\ {\isasymsubseteq}\ setSubformulae\ F{\isachardoublequoteclose}\isanewline
%
\isadelimproof
\ \ %
\endisadelimproof
%
\isatagproof
\isacommand{by}\isamarkupfalse%
\ {\isacharparenleft}induction\ F{\isacharparenright}\ \ auto%
\endisatagproof
{\isafoldproof}%
%
\isadelimproof
%
\endisadelimproof
%
\begin{isamarkuptext}%
La siguiente propiedad declara que el conjunto de átomos de una 
  subfórmula está contenido en el conjunto de átomos de la propia 
  fórmula.
  \begin{lema}
    Sea \isa{G\ {\isasymin}\ Subf{\isacharparenleft}F{\isacharparenright}}, entonces el conjunto de átomos de \isa{G} está
    contenido en el de \isa{F}.
  \end{lema}

  \begin{demostracion}
  Procedemos mediante inducción en la estructura de las fórmulas según 
  los distintos casos:

  Sea \isa{p} una fórmula atómica cualquiera. Si \isa{G\ {\isasymin}\ Subf{\isacharparenleft}p{\isacharparenright}}, 
  como su conjunto de variables es \isa{{\isacharbraceleft}p{\isacharbraceright}}, se tiene \isa{G\ {\isacharequal}\ p}. 
  Por tanto, se tiene el resultado.

  Sea la fórmula \isa{{\isasymbottom}}. Si \isa{G\ {\isasymin}\ Subf{\isacharparenleft}{\isasymbottom}{\isacharparenright}}, como  su conjunto de átomos es
  \isa{{\isacharbraceleft}{\isasymbottom}{\isacharbraceright}}, se tiene \isa{G\ {\isacharequal}\ {\isasymbottom}}. Por tanto, se cumple la propiedad.

  Sea la fórmula \isa{F} cualquiera tal que para cualquier subfórmula 
  \isa{G\ {\isasymin}\ Subf{\isacharparenleft}F{\isacharparenright}} se verifica que el conjunto de átomos de \isa{G} está 
  contenido en el de \isa{F}. Supongamos \isa{G{\isacharprime}\ {\isasymin}\ Subf{\isacharparenleft}{\isasymnot}\ F{\isacharparenright}} cualquiera, 
  probemos que \isa{conjAtoms{\isacharparenleft}G{\isacharprime}{\isacharparenright}\ {\isasymsubseteq}\ conjAtoms{\isacharparenleft}{\isasymnot}\ F{\isacharparenright}}.
  Por definición, tenemos que \isa{Subf{\isacharparenleft}{\isasymnot}\ F{\isacharparenright}\ {\isacharequal}\ {\isacharbraceleft}{\isasymnot}\ F{\isacharbraceright}\ {\isasymunion}\ Subf{\isacharparenleft}F{\isacharparenright}}. De este 
  modo, tenemos dos opciones:
  \isa{G{\isacharprime}\ {\isasymin}\ {\isacharbraceleft}{\isasymnot}\ F{\isacharbraceright}} o \isa{G{\isacharprime}\ {\isasymin}\ Subf{\isacharparenleft}F{\isacharparenright}}. Del primer caso se deduce \isa{G{\isacharprime}\ {\isacharequal}\ {\isasymnot}\ F} 
  y, por tanto, se verifica el resultado. Observando el segundo caso, 
  por hipótesis de inducción, se tiene que el conjunto de átomos de \isa{G{\isacharprime}}
  está contenido en el de \isa{F}. Además, como el conjunto de átomos de 
  \isa{F} y \isa{{\isasymnot}\ F} coinciden, se verifica el resultado.

  Sea \isa{F{\isadigit{1}}} fórmula proposicional tal que para cualquier \isa{G\ {\isasymin}\ Subf{\isacharparenleft}F{\isadigit{1}}{\isacharparenright}} 
  se tiene que el conjunto de átomos de \isa{G} está contenido en el de 
  \isa{F{\isadigit{1}}}. Sea también \isa{F{\isadigit{2}}} tal que dada \isa{G\ {\isasymin}\ Subf{\isacharparenleft}F{\isadigit{2}}{\isacharparenright}} cualquiera se 
  verifica también la hipótesis de inducción en su caso. Supongamos 
  \isa{G{\isacharprime}\ {\isasymin}\ Subf{\isacharparenleft}F{\isadigit{1}}{\isacharasterisk}F{\isadigit{2}}{\isacharparenright}} donde \isa{{\isacharasterisk}} es cualquier conectiva binaria. Vamos a 
  probar que el conjunto de átomos de \isa{G} está contenido en el de 
  \isa{F{\isadigit{1}}{\isacharasterisk}F{\isadigit{2}}}.

  En primer lugar, como 
  \isa{Subf{\isacharparenleft}F{\isadigit{1}}{\isacharasterisk}F{\isadigit{2}}{\isacharparenright}\ {\isacharequal}\ {\isacharbraceleft}F{\isadigit{1}}{\isacharasterisk}F{\isadigit{2}}{\isacharbraceright}\ {\isasymunion}\ {\isacharparenleft}Subf{\isacharparenleft}F{\isadigit{1}}{\isacharparenright}\ {\isasymunion}\ Subf{\isacharparenleft}F{\isadigit{2}}{\isacharparenright}{\isacharparenright}}, se desglosan tres
  casos posibles para \isa{G{\isacharprime}}:
  Si \isa{G{\isacharprime}\ {\isasymin}\ {\isacharbraceleft}F{\isadigit{1}}{\isacharasterisk}F{\isadigit{2}}{\isacharbraceright}}, entonces \isa{G{\isacharprime}\ {\isacharequal}\ F{\isadigit{1}}{\isacharasterisk}F{\isadigit{2}}} y se tiene la propiedad.
  Si \isa{G{\isacharprime}\ {\isasymin}\ Subf{\isacharparenleft}F{\isadigit{1}}{\isacharparenright}\ {\isasymunion}\ Subf{\isacharparenleft}F{\isadigit{2}}{\isacharparenright}}, entonces por propiedades de 
  conjuntos:
  \isa{G{\isacharprime}\ {\isasymin}\ Subf{\isacharparenleft}F{\isadigit{1}}{\isacharparenright}} o \isa{G{\isacharprime}\ {\isasymin}\ Subf{\isacharparenleft}F{\isadigit{2}}{\isacharparenright}}. Si \isa{G{\isacharprime}\ {\isasymin}\ Subf{\isacharparenleft}F{\isadigit{1}}{\isacharparenright}}, por hipótesis 
  de inducción se tiene el resultado. Como el conjunto de átomos de
  \isa{F{\isadigit{1}}{\isacharasterisk}F{\isadigit{2}}} es la unión de los conjuntos de átomos de \isa{F{\isadigit{1}}} y \isa{F{\isadigit{2}}}, se 
  obtiene el resultado como consecuencia de la transitividad de 
  contención para conjuntos. El caso \isa{G{\isacharprime}\ {\isasymin}\ Subf{\isacharparenleft}F{\isadigit{2}}{\isacharparenright}} se demuestra de la 
  misma forma.      
  \end{demostracion}

  Formalizado en Isabelle:%
\end{isamarkuptext}\isamarkuptrue%
\isacommand{lemma}\isamarkupfalse%
\ {\isachardoublequoteopen}G\ {\isasymin}\ setSubformulae\ F\ {\isasymLongrightarrow}\ atoms\ G\ {\isasymsubseteq}\ atoms\ F{\isachardoublequoteclose}\isanewline
%
\isadelimproof
\ \ %
\endisadelimproof
%
\isatagproof
\isacommand{oops}\isamarkupfalse%
%
\endisatagproof
{\isafoldproof}%
%
\isadelimproof
%
\endisadelimproof
%
\begin{isamarkuptext}%
Veamos su demostración estructurada.%
\end{isamarkuptext}\isamarkuptrue%
\isacommand{lemma}\isamarkupfalse%
\ subformulas{\isacharunderscore}atoms{\isacharunderscore}atom{\isacharcolon}\isanewline
\ \ \isakeyword{assumes}\ {\isachardoublequoteopen}G\ {\isasymin}\ setSubformulae\ {\isacharparenleft}Atom\ x{\isacharparenright}{\isachardoublequoteclose}\ \isanewline
\ \ \isakeyword{shows}\ \ \ {\isachardoublequoteopen}atoms\ G\ {\isasymsubseteq}\ atoms\ {\isacharparenleft}Atom\ x{\isacharparenright}{\isachardoublequoteclose}\isanewline
%
\isadelimproof
%
\endisadelimproof
%
\isatagproof
\isacommand{proof}\isamarkupfalse%
\ {\isacharminus}\isanewline
\ \ \isacommand{have}\isamarkupfalse%
\ {\isachardoublequoteopen}G\ {\isasymin}\ {\isacharbraceleft}Atom\ x{\isacharbraceright}{\isachardoublequoteclose}\isanewline
\ \ \ \ \isacommand{using}\isamarkupfalse%
\ assms\isanewline
\ \ \ \ \isacommand{by}\isamarkupfalse%
\ {\isacharparenleft}simp\ only{\isacharcolon}\ setSubformulae{\isacharunderscore}atom{\isacharparenright}\isanewline
\ \ \isacommand{then}\isamarkupfalse%
\ \isacommand{have}\isamarkupfalse%
\ {\isachardoublequoteopen}G\ {\isacharequal}\ Atom\ x{\isachardoublequoteclose}\isanewline
\ \ \ \ \isacommand{by}\isamarkupfalse%
\ {\isacharparenleft}simp\ only{\isacharcolon}\ singletonD{\isacharparenright}\isanewline
\ \ \isacommand{then}\isamarkupfalse%
\ \isacommand{show}\isamarkupfalse%
\ {\isacharquery}thesis\isanewline
\ \ \ \ \isacommand{by}\isamarkupfalse%
\ {\isacharparenleft}simp\ only{\isacharcolon}\ subset{\isacharunderscore}refl{\isacharparenright}\isanewline
\isacommand{qed}\isamarkupfalse%
%
\endisatagproof
{\isafoldproof}%
%
\isadelimproof
\isanewline
%
\endisadelimproof
\isanewline
\isacommand{lemma}\isamarkupfalse%
\ subformulas{\isacharunderscore}atoms{\isacharunderscore}bot{\isacharcolon}\isanewline
\ \ \isakeyword{assumes}\ {\isachardoublequoteopen}G\ {\isasymin}\ setSubformulae\ {\isasymbottom}{\isachardoublequoteclose}\ \isanewline
\ \ \isakeyword{shows}\ \ \ {\isachardoublequoteopen}atoms\ G\ {\isasymsubseteq}\ atoms\ {\isasymbottom}{\isachardoublequoteclose}\isanewline
%
\isadelimproof
%
\endisadelimproof
%
\isatagproof
\isacommand{proof}\isamarkupfalse%
\ {\isacharminus}\isanewline
\ \ \isacommand{have}\isamarkupfalse%
\ {\isachardoublequoteopen}G\ {\isasymin}\ {\isacharbraceleft}{\isasymbottom}{\isacharbraceright}{\isachardoublequoteclose}\isanewline
\ \ \ \ \isacommand{using}\isamarkupfalse%
\ assms\isanewline
\ \ \ \ \isacommand{by}\isamarkupfalse%
\ {\isacharparenleft}simp\ only{\isacharcolon}\ setSubformulae{\isacharunderscore}bot{\isacharparenright}\isanewline
\ \ \isacommand{then}\isamarkupfalse%
\ \isacommand{have}\isamarkupfalse%
\ {\isachardoublequoteopen}G\ {\isacharequal}\ {\isasymbottom}{\isachardoublequoteclose}\isanewline
\ \ \ \ \isacommand{by}\isamarkupfalse%
\ {\isacharparenleft}simp\ only{\isacharcolon}\ singletonD{\isacharparenright}\isanewline
\ \ \isacommand{then}\isamarkupfalse%
\ \isacommand{show}\isamarkupfalse%
\ {\isacharquery}thesis\isanewline
\ \ \ \ \isacommand{by}\isamarkupfalse%
\ {\isacharparenleft}simp\ only{\isacharcolon}\ subset{\isacharunderscore}refl{\isacharparenright}\isanewline
\isacommand{qed}\isamarkupfalse%
%
\endisatagproof
{\isafoldproof}%
%
\isadelimproof
\isanewline
%
\endisadelimproof
\isanewline
\isacommand{lemma}\isamarkupfalse%
\ subformulas{\isacharunderscore}atoms{\isacharunderscore}not{\isacharcolon}\isanewline
\ \ \isakeyword{assumes}\ {\isachardoublequoteopen}G\ {\isasymin}\ setSubformulae\ F\ {\isasymLongrightarrow}\ atoms\ G\ {\isasymsubseteq}\ atoms\ F{\isachardoublequoteclose}\isanewline
\ \ \ \ \ \ \ \ \ \ {\isachardoublequoteopen}G\ {\isasymin}\ setSubformulae\ {\isacharparenleft}\isactrlbold {\isasymnot}\ F{\isacharparenright}{\isachardoublequoteclose}\isanewline
\ \ \isakeyword{shows}\ \ \ {\isachardoublequoteopen}atoms\ G\ {\isasymsubseteq}\ atoms\ {\isacharparenleft}\isactrlbold {\isasymnot}\ F{\isacharparenright}{\isachardoublequoteclose}\isanewline
%
\isadelimproof
%
\endisadelimproof
%
\isatagproof
\isacommand{proof}\isamarkupfalse%
\ {\isacharminus}\isanewline
\ \ \isacommand{have}\isamarkupfalse%
\ {\isachardoublequoteopen}G\ {\isasymin}\ {\isacharbraceleft}\isactrlbold {\isasymnot}\ F{\isacharbraceright}\ {\isasymunion}\ setSubformulae\ F{\isachardoublequoteclose}\isanewline
\ \ \ \ \isacommand{using}\isamarkupfalse%
\ assms{\isacharparenleft}{\isadigit{2}}{\isacharparenright}\isanewline
\ \ \ \ \isacommand{by}\isamarkupfalse%
\ {\isacharparenleft}simp\ only{\isacharcolon}\ setSubformulae{\isacharunderscore}not{\isacharparenright}\ \isanewline
\ \ \isacommand{then}\isamarkupfalse%
\ \isacommand{have}\isamarkupfalse%
\ {\isachardoublequoteopen}G\ {\isasymin}\ {\isacharbraceleft}\isactrlbold {\isasymnot}\ F{\isacharbraceright}\ {\isasymor}\ G\ {\isasymin}\ setSubformulae\ F{\isachardoublequoteclose}\isanewline
\ \ \ \ \isacommand{by}\isamarkupfalse%
\ {\isacharparenleft}simp\ only{\isacharcolon}\ Un{\isacharunderscore}iff{\isacharparenright}\isanewline
\ \ \isacommand{then}\isamarkupfalse%
\ \isacommand{show}\isamarkupfalse%
\ {\isachardoublequoteopen}atoms\ G\ {\isasymsubseteq}\ atoms\ {\isacharparenleft}\isactrlbold {\isasymnot}\ F{\isacharparenright}{\isachardoublequoteclose}\isanewline
\ \ \isacommand{proof}\isamarkupfalse%
\isanewline
\ \ \ \ \isacommand{assume}\isamarkupfalse%
\ {\isachardoublequoteopen}G\ {\isasymin}\ {\isacharbraceleft}\isactrlbold {\isasymnot}\ F{\isacharbraceright}{\isachardoublequoteclose}\isanewline
\ \ \ \ \isacommand{then}\isamarkupfalse%
\ \isacommand{have}\isamarkupfalse%
\ {\isachardoublequoteopen}G\ {\isacharequal}\ \isactrlbold {\isasymnot}\ F{\isachardoublequoteclose}\isanewline
\ \ \ \ \ \ \isacommand{by}\isamarkupfalse%
\ {\isacharparenleft}simp\ only{\isacharcolon}\ singletonD{\isacharparenright}\isanewline
\ \ \ \ \isacommand{then}\isamarkupfalse%
\ \isacommand{show}\isamarkupfalse%
\ {\isacharquery}thesis\isanewline
\ \ \ \ \ \ \isacommand{by}\isamarkupfalse%
\ {\isacharparenleft}simp\ only{\isacharcolon}\ subset{\isacharunderscore}refl{\isacharparenright}\isanewline
\ \ \isacommand{next}\isamarkupfalse%
\isanewline
\ \ \ \ \isacommand{assume}\isamarkupfalse%
\ {\isachardoublequoteopen}G\ {\isasymin}\ setSubformulae\ F{\isachardoublequoteclose}\isanewline
\ \ \ \ \isacommand{then}\isamarkupfalse%
\ \isacommand{have}\isamarkupfalse%
\ {\isachardoublequoteopen}atoms\ G\ {\isasymsubseteq}\ atoms\ F{\isachardoublequoteclose}\isanewline
\ \ \ \ \ \ \isacommand{by}\isamarkupfalse%
\ {\isacharparenleft}simp\ only{\isacharcolon}\ assms{\isacharparenleft}{\isadigit{1}}{\isacharparenright}{\isacharparenright}\isanewline
\ \ \ \ \isacommand{also}\isamarkupfalse%
\ \isacommand{have}\isamarkupfalse%
\ {\isachardoublequoteopen}{\isasymdots}\ {\isacharequal}\ atoms\ {\isacharparenleft}\isactrlbold {\isasymnot}\ F{\isacharparenright}{\isachardoublequoteclose}\isanewline
\ \ \ \ \ \ \isacommand{by}\isamarkupfalse%
\ {\isacharparenleft}simp\ only{\isacharcolon}\ formula{\isachardot}set{\isacharparenleft}{\isadigit{3}}{\isacharparenright}{\isacharparenright}\isanewline
\ \ \ \ \isacommand{finally}\isamarkupfalse%
\ \isacommand{show}\isamarkupfalse%
\ {\isacharquery}thesis\isanewline
\ \ \ \ \ \ \isacommand{by}\isamarkupfalse%
\ this\isanewline
\ \ \isacommand{qed}\isamarkupfalse%
\isanewline
\isacommand{qed}\isamarkupfalse%
%
\endisatagproof
{\isafoldproof}%
%
\isadelimproof
\isanewline
%
\endisadelimproof
\isanewline
\isacommand{lemma}\isamarkupfalse%
\ subformulas{\isacharunderscore}atoms{\isacharunderscore}and{\isacharcolon}\isanewline
\ \ \isakeyword{assumes}\ {\isachardoublequoteopen}G\ {\isasymin}\ setSubformulae\ F{\isadigit{1}}\ {\isasymLongrightarrow}\ atoms\ G\ {\isasymsubseteq}\ atoms\ F{\isadigit{1}}{\isachardoublequoteclose}\isanewline
\ \ \ \ \ \ \ \ \ \ {\isachardoublequoteopen}G\ {\isasymin}\ setSubformulae\ F{\isadigit{2}}\ {\isasymLongrightarrow}\ atoms\ G\ {\isasymsubseteq}\ atoms\ F{\isadigit{2}}{\isachardoublequoteclose}\isanewline
\ \ \ \ \ \ \ \ \ \ {\isachardoublequoteopen}G\ {\isasymin}\ setSubformulae\ {\isacharparenleft}F{\isadigit{1}}\ \isactrlbold {\isasymand}\ F{\isadigit{2}}{\isacharparenright}{\isachardoublequoteclose}\isanewline
\ \ \isakeyword{shows}\ \ \ {\isachardoublequoteopen}atoms\ G\ {\isasymsubseteq}\ atoms\ {\isacharparenleft}F{\isadigit{1}}\ \isactrlbold {\isasymand}\ F{\isadigit{2}}{\isacharparenright}{\isachardoublequoteclose}\isanewline
%
\isadelimproof
%
\endisadelimproof
%
\isatagproof
\isacommand{proof}\isamarkupfalse%
\ {\isacharminus}\isanewline
\ \ \isacommand{have}\isamarkupfalse%
\ {\isachardoublequoteopen}G\ {\isasymin}\ {\isacharbraceleft}F{\isadigit{1}}\ \isactrlbold {\isasymand}\ F{\isadigit{2}}{\isacharbraceright}\ {\isasymunion}\ {\isacharparenleft}setSubformulae\ F{\isadigit{1}}\ {\isasymunion}\ setSubformulae\ F{\isadigit{2}}{\isacharparenright}{\isachardoublequoteclose}\isanewline
\ \ \ \ \isacommand{using}\isamarkupfalse%
\ assms{\isacharparenleft}{\isadigit{3}}{\isacharparenright}\ \isanewline
\ \ \ \ \isacommand{by}\isamarkupfalse%
\ {\isacharparenleft}simp\ only{\isacharcolon}\ setSubformulae{\isacharunderscore}and{\isacharparenright}\isanewline
\ \ \isacommand{then}\isamarkupfalse%
\ \isacommand{have}\isamarkupfalse%
\ {\isachardoublequoteopen}G\ {\isasymin}\ {\isacharbraceleft}F{\isadigit{1}}\ \isactrlbold {\isasymand}\ F{\isadigit{2}}{\isacharbraceright}\ {\isasymor}\ G\ {\isasymin}\ setSubformulae\ F{\isadigit{1}}\ {\isasymunion}\ setSubformulae\ F{\isadigit{2}}{\isachardoublequoteclose}\isanewline
\ \ \ \ \isacommand{by}\isamarkupfalse%
\ {\isacharparenleft}simp\ only{\isacharcolon}\ Un{\isacharunderscore}iff{\isacharparenright}\isanewline
\ \ \isacommand{then}\isamarkupfalse%
\ \isacommand{show}\isamarkupfalse%
\ {\isacharquery}thesis\isanewline
\ \ \isacommand{proof}\isamarkupfalse%
\ \isanewline
\ \ \ \ \isacommand{assume}\isamarkupfalse%
\ {\isachardoublequoteopen}G\ {\isasymin}\ {\isacharbraceleft}F{\isadigit{1}}\ \isactrlbold {\isasymand}\ F{\isadigit{2}}{\isacharbraceright}{\isachardoublequoteclose}\isanewline
\ \ \ \ \isacommand{then}\isamarkupfalse%
\ \isacommand{have}\isamarkupfalse%
\ {\isachardoublequoteopen}G\ {\isacharequal}\ F{\isadigit{1}}\ \isactrlbold {\isasymand}\ F{\isadigit{2}}{\isachardoublequoteclose}\isanewline
\ \ \ \ \ \ \isacommand{by}\isamarkupfalse%
\ {\isacharparenleft}simp\ only{\isacharcolon}\ singletonD{\isacharparenright}\isanewline
\ \ \ \ \isacommand{then}\isamarkupfalse%
\ \isacommand{show}\isamarkupfalse%
\ {\isacharquery}thesis\isanewline
\ \ \ \ \ \ \isacommand{by}\isamarkupfalse%
\ {\isacharparenleft}simp\ only{\isacharcolon}\ subset{\isacharunderscore}refl{\isacharparenright}\isanewline
\ \ \isacommand{next}\isamarkupfalse%
\isanewline
\ \ \ \ \isacommand{assume}\isamarkupfalse%
\ {\isachardoublequoteopen}G\ {\isasymin}\ setSubformulae\ F{\isadigit{1}}\ {\isasymunion}\ setSubformulae\ F{\isadigit{2}}{\isachardoublequoteclose}\isanewline
\ \ \ \ \isacommand{then}\isamarkupfalse%
\ \isacommand{have}\isamarkupfalse%
\ {\isachardoublequoteopen}G\ {\isasymin}\ setSubformulae\ F{\isadigit{1}}\ {\isasymor}\ G\ {\isasymin}\ setSubformulae\ F{\isadigit{2}}{\isachardoublequoteclose}\ \ \isanewline
\ \ \ \ \ \ \isacommand{by}\isamarkupfalse%
\ {\isacharparenleft}simp\ only{\isacharcolon}\ Un{\isacharunderscore}iff{\isacharparenright}\isanewline
\ \ \ \ \isacommand{then}\isamarkupfalse%
\ \isacommand{show}\isamarkupfalse%
\ {\isacharquery}thesis\isanewline
\ \ \ \ \isacommand{proof}\isamarkupfalse%
\ \isanewline
\ \ \ \ \ \ \isacommand{assume}\isamarkupfalse%
\ {\isachardoublequoteopen}G\ {\isasymin}\ setSubformulae\ F{\isadigit{1}}{\isachardoublequoteclose}\isanewline
\ \ \ \ \ \ \isacommand{then}\isamarkupfalse%
\ \isacommand{have}\isamarkupfalse%
\ {\isachardoublequoteopen}atoms\ G\ {\isasymsubseteq}\ atoms\ F{\isadigit{1}}{\isachardoublequoteclose}\isanewline
\ \ \ \ \ \ \ \ \isacommand{by}\isamarkupfalse%
\ {\isacharparenleft}rule\ assms{\isacharparenleft}{\isadigit{1}}{\isacharparenright}{\isacharparenright}\isanewline
\ \ \ \ \ \ \isacommand{also}\isamarkupfalse%
\ \isacommand{have}\isamarkupfalse%
\ {\isachardoublequoteopen}{\isasymdots}\ {\isasymsubseteq}\ atoms\ F{\isadigit{1}}\ {\isasymunion}\ atoms\ F{\isadigit{2}}{\isachardoublequoteclose}\isanewline
\ \ \ \ \ \ \ \ \isacommand{by}\isamarkupfalse%
\ {\isacharparenleft}simp\ only{\isacharcolon}\ Un{\isacharunderscore}upper{\isadigit{1}}{\isacharparenright}\isanewline
\ \ \ \ \ \ \isacommand{also}\isamarkupfalse%
\ \isacommand{have}\isamarkupfalse%
\ {\isachardoublequoteopen}{\isasymdots}\ {\isacharequal}\ atoms\ {\isacharparenleft}F{\isadigit{1}}\ \isactrlbold {\isasymand}\ F{\isadigit{2}}{\isacharparenright}{\isachardoublequoteclose}\isanewline
\ \ \ \ \ \ \ \ \isacommand{by}\isamarkupfalse%
\ {\isacharparenleft}simp\ only{\isacharcolon}\ formula{\isachardot}set{\isacharparenleft}{\isadigit{4}}{\isacharparenright}{\isacharparenright}\isanewline
\ \ \ \ \ \ \isacommand{finally}\isamarkupfalse%
\ \isacommand{show}\isamarkupfalse%
\ {\isacharquery}thesis\isanewline
\ \ \ \ \ \ \ \ \isacommand{by}\isamarkupfalse%
\ this\isanewline
\ \ \ \ \isacommand{next}\isamarkupfalse%
\isanewline
\ \ \ \ \ \ \isacommand{assume}\isamarkupfalse%
\ {\isachardoublequoteopen}G\ {\isasymin}\ setSubformulae\ F{\isadigit{2}}{\isachardoublequoteclose}\isanewline
\ \ \ \ \ \ \isacommand{then}\isamarkupfalse%
\ \isacommand{have}\isamarkupfalse%
\ {\isachardoublequoteopen}atoms\ G\ {\isasymsubseteq}\ atoms\ F{\isadigit{2}}{\isachardoublequoteclose}\isanewline
\ \ \ \ \ \ \ \ \isacommand{by}\isamarkupfalse%
\ {\isacharparenleft}rule\ assms{\isacharparenleft}{\isadigit{2}}{\isacharparenright}{\isacharparenright}\isanewline
\ \ \ \ \ \ \isacommand{also}\isamarkupfalse%
\ \isacommand{have}\isamarkupfalse%
\ {\isachardoublequoteopen}{\isasymdots}\ {\isasymsubseteq}\ atoms\ F{\isadigit{1}}\ {\isasymunion}\ atoms\ F{\isadigit{2}}{\isachardoublequoteclose}\isanewline
\ \ \ \ \ \ \ \ \isacommand{by}\isamarkupfalse%
\ {\isacharparenleft}simp\ only{\isacharcolon}\ Un{\isacharunderscore}upper{\isadigit{2}}{\isacharparenright}\isanewline
\ \ \ \ \ \ \isacommand{also}\isamarkupfalse%
\ \isacommand{have}\isamarkupfalse%
\ {\isachardoublequoteopen}{\isasymdots}\ {\isacharequal}\ atoms\ {\isacharparenleft}F{\isadigit{1}}\ \isactrlbold {\isasymand}\ F{\isadigit{2}}{\isacharparenright}{\isachardoublequoteclose}\isanewline
\ \ \ \ \ \ \ \ \isacommand{by}\isamarkupfalse%
\ {\isacharparenleft}simp\ only{\isacharcolon}\ formula{\isachardot}set{\isacharparenleft}{\isadigit{4}}{\isacharparenright}{\isacharparenright}\isanewline
\ \ \ \ \ \ \isacommand{finally}\isamarkupfalse%
\ \isacommand{show}\isamarkupfalse%
\ {\isacharquery}thesis\isanewline
\ \ \ \ \ \ \ \ \isacommand{by}\isamarkupfalse%
\ this\isanewline
\ \ \ \ \isacommand{qed}\isamarkupfalse%
\isanewline
\ \ \isacommand{qed}\isamarkupfalse%
\isanewline
\isacommand{qed}\isamarkupfalse%
%
\endisatagproof
{\isafoldproof}%
%
\isadelimproof
\isanewline
%
\endisadelimproof
\isanewline
\isacommand{lemma}\isamarkupfalse%
\ subformulas{\isacharunderscore}atoms{\isacharunderscore}or{\isacharcolon}\isanewline
\ \ \isakeyword{assumes}\ {\isachardoublequoteopen}G\ {\isasymin}\ setSubformulae\ F{\isadigit{1}}\ {\isasymLongrightarrow}\ atoms\ G\ {\isasymsubseteq}\ atoms\ F{\isadigit{1}}{\isachardoublequoteclose}\isanewline
\ \ \ \ \ \ \ \ \ \ {\isachardoublequoteopen}G\ {\isasymin}\ setSubformulae\ F{\isadigit{2}}\ {\isasymLongrightarrow}\ atoms\ G\ {\isasymsubseteq}\ atoms\ F{\isadigit{2}}{\isachardoublequoteclose}\isanewline
\ \ \ \ \ \ \ \ \ \ {\isachardoublequoteopen}G\ {\isasymin}\ setSubformulae\ {\isacharparenleft}F{\isadigit{1}}\ \isactrlbold {\isasymor}\ F{\isadigit{2}}{\isacharparenright}{\isachardoublequoteclose}\isanewline
\ \ \isakeyword{shows}\ \ \ {\isachardoublequoteopen}atoms\ G\ {\isasymsubseteq}\ atoms\ {\isacharparenleft}F{\isadigit{1}}\ \isactrlbold {\isasymor}\ F{\isadigit{2}}{\isacharparenright}{\isachardoublequoteclose}\isanewline
%
\isadelimproof
%
\endisadelimproof
%
\isatagproof
\isacommand{proof}\isamarkupfalse%
\ {\isacharminus}\isanewline
\ \ \isacommand{have}\isamarkupfalse%
\ {\isachardoublequoteopen}G\ {\isasymin}\ {\isacharbraceleft}F{\isadigit{1}}\ \isactrlbold {\isasymor}\ F{\isadigit{2}}{\isacharbraceright}\ {\isasymunion}\ {\isacharparenleft}setSubformulae\ F{\isadigit{1}}\ {\isasymunion}\ setSubformulae\ F{\isadigit{2}}{\isacharparenright}{\isachardoublequoteclose}\isanewline
\ \ \ \ \isacommand{using}\isamarkupfalse%
\ assms{\isacharparenleft}{\isadigit{3}}{\isacharparenright}\ \isanewline
\ \ \ \ \isacommand{by}\isamarkupfalse%
\ {\isacharparenleft}simp\ only{\isacharcolon}\ setSubformulae{\isacharunderscore}or{\isacharparenright}\isanewline
\ \ \isacommand{then}\isamarkupfalse%
\ \isacommand{have}\isamarkupfalse%
\ {\isachardoublequoteopen}G\ {\isasymin}\ {\isacharbraceleft}F{\isadigit{1}}\ \isactrlbold {\isasymor}\ F{\isadigit{2}}{\isacharbraceright}\ {\isasymor}\ G\ {\isasymin}\ setSubformulae\ F{\isadigit{1}}\ {\isasymunion}\ setSubformulae\ F{\isadigit{2}}{\isachardoublequoteclose}\isanewline
\ \ \ \ \isacommand{by}\isamarkupfalse%
\ {\isacharparenleft}simp\ only{\isacharcolon}\ Un{\isacharunderscore}iff{\isacharparenright}\isanewline
\ \ \isacommand{then}\isamarkupfalse%
\ \isacommand{show}\isamarkupfalse%
\ {\isacharquery}thesis\isanewline
\ \ \isacommand{proof}\isamarkupfalse%
\ \isanewline
\ \ \ \ \isacommand{assume}\isamarkupfalse%
\ {\isachardoublequoteopen}G\ {\isasymin}\ {\isacharbraceleft}F{\isadigit{1}}\ \isactrlbold {\isasymor}\ F{\isadigit{2}}{\isacharbraceright}{\isachardoublequoteclose}\isanewline
\ \ \ \ \isacommand{then}\isamarkupfalse%
\ \isacommand{have}\isamarkupfalse%
\ {\isachardoublequoteopen}G\ {\isacharequal}\ F{\isadigit{1}}\ \isactrlbold {\isasymor}\ F{\isadigit{2}}{\isachardoublequoteclose}\isanewline
\ \ \ \ \ \ \isacommand{by}\isamarkupfalse%
\ {\isacharparenleft}simp\ only{\isacharcolon}\ singletonD{\isacharparenright}\isanewline
\ \ \ \ \isacommand{then}\isamarkupfalse%
\ \isacommand{show}\isamarkupfalse%
\ {\isacharquery}thesis\isanewline
\ \ \ \ \ \ \isacommand{by}\isamarkupfalse%
\ {\isacharparenleft}simp\ only{\isacharcolon}\ subset{\isacharunderscore}refl{\isacharparenright}\isanewline
\ \ \isacommand{next}\isamarkupfalse%
\isanewline
\ \ \ \ \isacommand{assume}\isamarkupfalse%
\ {\isachardoublequoteopen}G\ {\isasymin}\ setSubformulae\ F{\isadigit{1}}\ {\isasymunion}\ setSubformulae\ F{\isadigit{2}}{\isachardoublequoteclose}\isanewline
\ \ \ \ \isacommand{then}\isamarkupfalse%
\ \isacommand{have}\isamarkupfalse%
\ {\isachardoublequoteopen}G\ {\isasymin}\ setSubformulae\ F{\isadigit{1}}\ {\isasymor}\ G\ {\isasymin}\ setSubformulae\ F{\isadigit{2}}{\isachardoublequoteclose}\ \ \isanewline
\ \ \ \ \ \ \isacommand{by}\isamarkupfalse%
\ {\isacharparenleft}simp\ only{\isacharcolon}\ Un{\isacharunderscore}iff{\isacharparenright}\isanewline
\ \ \ \ \isacommand{then}\isamarkupfalse%
\ \isacommand{show}\isamarkupfalse%
\ {\isacharquery}thesis\isanewline
\ \ \ \ \isacommand{proof}\isamarkupfalse%
\ \isanewline
\ \ \ \ \ \ \isacommand{assume}\isamarkupfalse%
\ {\isachardoublequoteopen}G\ {\isasymin}\ setSubformulae\ F{\isadigit{1}}{\isachardoublequoteclose}\isanewline
\ \ \ \ \ \ \isacommand{then}\isamarkupfalse%
\ \isacommand{have}\isamarkupfalse%
\ {\isachardoublequoteopen}atoms\ G\ {\isasymsubseteq}\ atoms\ F{\isadigit{1}}{\isachardoublequoteclose}\isanewline
\ \ \ \ \ \ \ \ \isacommand{by}\isamarkupfalse%
\ {\isacharparenleft}rule\ assms{\isacharparenleft}{\isadigit{1}}{\isacharparenright}{\isacharparenright}\isanewline
\ \ \ \ \ \ \isacommand{also}\isamarkupfalse%
\ \isacommand{have}\isamarkupfalse%
\ {\isachardoublequoteopen}{\isasymdots}\ {\isasymsubseteq}\ atoms\ F{\isadigit{1}}\ {\isasymunion}\ atoms\ F{\isadigit{2}}{\isachardoublequoteclose}\isanewline
\ \ \ \ \ \ \ \ \isacommand{by}\isamarkupfalse%
\ {\isacharparenleft}simp\ only{\isacharcolon}\ Un{\isacharunderscore}upper{\isadigit{1}}{\isacharparenright}\isanewline
\ \ \ \ \ \ \isacommand{also}\isamarkupfalse%
\ \isacommand{have}\isamarkupfalse%
\ {\isachardoublequoteopen}{\isasymdots}\ {\isacharequal}\ atoms\ {\isacharparenleft}F{\isadigit{1}}\ \isactrlbold {\isasymor}\ F{\isadigit{2}}{\isacharparenright}{\isachardoublequoteclose}\isanewline
\ \ \ \ \ \ \ \ \isacommand{by}\isamarkupfalse%
\ {\isacharparenleft}simp\ only{\isacharcolon}\ formula{\isachardot}set{\isacharparenleft}{\isadigit{5}}{\isacharparenright}{\isacharparenright}\isanewline
\ \ \ \ \ \ \isacommand{finally}\isamarkupfalse%
\ \isacommand{show}\isamarkupfalse%
\ {\isacharquery}thesis\isanewline
\ \ \ \ \ \ \ \ \isacommand{by}\isamarkupfalse%
\ this\isanewline
\ \ \ \ \isacommand{next}\isamarkupfalse%
\isanewline
\ \ \ \ \ \ \isacommand{assume}\isamarkupfalse%
\ {\isachardoublequoteopen}G\ {\isasymin}\ setSubformulae\ F{\isadigit{2}}{\isachardoublequoteclose}\isanewline
\ \ \ \ \ \ \isacommand{then}\isamarkupfalse%
\ \isacommand{have}\isamarkupfalse%
\ {\isachardoublequoteopen}atoms\ G\ {\isasymsubseteq}\ atoms\ F{\isadigit{2}}{\isachardoublequoteclose}\isanewline
\ \ \ \ \ \ \ \ \isacommand{by}\isamarkupfalse%
\ {\isacharparenleft}rule\ assms{\isacharparenleft}{\isadigit{2}}{\isacharparenright}{\isacharparenright}\isanewline
\ \ \ \ \ \ \isacommand{also}\isamarkupfalse%
\ \isacommand{have}\isamarkupfalse%
\ {\isachardoublequoteopen}{\isasymdots}\ {\isasymsubseteq}\ atoms\ F{\isadigit{1}}\ {\isasymunion}\ atoms\ F{\isadigit{2}}{\isachardoublequoteclose}\isanewline
\ \ \ \ \ \ \ \ \isacommand{by}\isamarkupfalse%
\ {\isacharparenleft}simp\ only{\isacharcolon}\ Un{\isacharunderscore}upper{\isadigit{2}}{\isacharparenright}\isanewline
\ \ \ \ \ \ \isacommand{also}\isamarkupfalse%
\ \isacommand{have}\isamarkupfalse%
\ {\isachardoublequoteopen}{\isasymdots}\ {\isacharequal}\ atoms\ {\isacharparenleft}F{\isadigit{1}}\ \isactrlbold {\isasymor}\ F{\isadigit{2}}{\isacharparenright}{\isachardoublequoteclose}\isanewline
\ \ \ \ \ \ \ \ \isacommand{by}\isamarkupfalse%
\ {\isacharparenleft}simp\ only{\isacharcolon}\ formula{\isachardot}set{\isacharparenleft}{\isadigit{5}}{\isacharparenright}{\isacharparenright}\isanewline
\ \ \ \ \ \ \isacommand{finally}\isamarkupfalse%
\ \isacommand{show}\isamarkupfalse%
\ {\isacharquery}thesis\isanewline
\ \ \ \ \ \ \ \ \isacommand{by}\isamarkupfalse%
\ this\isanewline
\ \ \ \ \isacommand{qed}\isamarkupfalse%
\isanewline
\ \ \isacommand{qed}\isamarkupfalse%
\isanewline
\isacommand{qed}\isamarkupfalse%
%
\endisatagproof
{\isafoldproof}%
%
\isadelimproof
\isanewline
%
\endisadelimproof
\isanewline
\isacommand{lemma}\isamarkupfalse%
\ subformulas{\isacharunderscore}atoms{\isacharunderscore}imp{\isacharcolon}\isanewline
\ \ \isakeyword{assumes}\ {\isachardoublequoteopen}G\ {\isasymin}\ setSubformulae\ F{\isadigit{1}}\ {\isasymLongrightarrow}\ atoms\ G\ {\isasymsubseteq}\ atoms\ F{\isadigit{1}}{\isachardoublequoteclose}\isanewline
\ \ \ \ \ \ \ \ \ \ {\isachardoublequoteopen}G\ {\isasymin}\ setSubformulae\ F{\isadigit{2}}\ {\isasymLongrightarrow}\ atoms\ G\ {\isasymsubseteq}\ atoms\ F{\isadigit{2}}{\isachardoublequoteclose}\isanewline
\ \ \ \ \ \ \ \ \ \ {\isachardoublequoteopen}G\ {\isasymin}\ setSubformulae\ {\isacharparenleft}F{\isadigit{1}}\ \isactrlbold {\isasymrightarrow}\ F{\isadigit{2}}{\isacharparenright}{\isachardoublequoteclose}\isanewline
\ \ \isakeyword{shows}\ \ \ {\isachardoublequoteopen}atoms\ G\ {\isasymsubseteq}\ atoms\ {\isacharparenleft}F{\isadigit{1}}\ \isactrlbold {\isasymrightarrow}\ F{\isadigit{2}}{\isacharparenright}{\isachardoublequoteclose}\isanewline
%
\isadelimproof
%
\endisadelimproof
%
\isatagproof
\isacommand{proof}\isamarkupfalse%
\ {\isacharminus}\isanewline
\ \ \isacommand{have}\isamarkupfalse%
\ {\isachardoublequoteopen}G\ {\isasymin}\ {\isacharbraceleft}F{\isadigit{1}}\ \isactrlbold {\isasymrightarrow}\ F{\isadigit{2}}{\isacharbraceright}\ {\isasymunion}\ {\isacharparenleft}setSubformulae\ F{\isadigit{1}}\ {\isasymunion}\ setSubformulae\ F{\isadigit{2}}{\isacharparenright}{\isachardoublequoteclose}\isanewline
\ \ \ \ \isacommand{using}\isamarkupfalse%
\ assms{\isacharparenleft}{\isadigit{3}}{\isacharparenright}\ \isanewline
\ \ \ \ \isacommand{by}\isamarkupfalse%
\ {\isacharparenleft}simp\ only{\isacharcolon}\ setSubformulae{\isacharunderscore}imp{\isacharparenright}\isanewline
\ \ \isacommand{then}\isamarkupfalse%
\ \isacommand{have}\isamarkupfalse%
\ {\isachardoublequoteopen}G\ {\isasymin}\ {\isacharbraceleft}F{\isadigit{1}}\ \isactrlbold {\isasymrightarrow}\ F{\isadigit{2}}{\isacharbraceright}\ {\isasymor}\ G\ {\isasymin}\ setSubformulae\ F{\isadigit{1}}\ {\isasymunion}\ setSubformulae\ F{\isadigit{2}}{\isachardoublequoteclose}\isanewline
\ \ \ \ \isacommand{by}\isamarkupfalse%
\ {\isacharparenleft}simp\ only{\isacharcolon}\ Un{\isacharunderscore}iff{\isacharparenright}\isanewline
\ \ \isacommand{then}\isamarkupfalse%
\ \isacommand{show}\isamarkupfalse%
\ {\isacharquery}thesis\isanewline
\ \ \isacommand{proof}\isamarkupfalse%
\ \isanewline
\ \ \ \ \isacommand{assume}\isamarkupfalse%
\ {\isachardoublequoteopen}G\ {\isasymin}\ {\isacharbraceleft}F{\isadigit{1}}\ \isactrlbold {\isasymrightarrow}\ F{\isadigit{2}}{\isacharbraceright}{\isachardoublequoteclose}\isanewline
\ \ \ \ \isacommand{then}\isamarkupfalse%
\ \isacommand{have}\isamarkupfalse%
\ {\isachardoublequoteopen}G\ {\isacharequal}\ F{\isadigit{1}}\ \isactrlbold {\isasymrightarrow}\ F{\isadigit{2}}{\isachardoublequoteclose}\isanewline
\ \ \ \ \ \ \isacommand{by}\isamarkupfalse%
\ {\isacharparenleft}simp\ only{\isacharcolon}\ singletonD{\isacharparenright}\isanewline
\ \ \ \ \isacommand{then}\isamarkupfalse%
\ \isacommand{show}\isamarkupfalse%
\ {\isacharquery}thesis\isanewline
\ \ \ \ \ \ \isacommand{by}\isamarkupfalse%
\ {\isacharparenleft}simp\ only{\isacharcolon}\ subset{\isacharunderscore}refl{\isacharparenright}\isanewline
\ \ \isacommand{next}\isamarkupfalse%
\isanewline
\ \ \ \ \isacommand{assume}\isamarkupfalse%
\ {\isachardoublequoteopen}G\ {\isasymin}\ setSubformulae\ F{\isadigit{1}}\ {\isasymunion}\ setSubformulae\ F{\isadigit{2}}{\isachardoublequoteclose}\isanewline
\ \ \ \ \isacommand{then}\isamarkupfalse%
\ \isacommand{have}\isamarkupfalse%
\ {\isachardoublequoteopen}G\ {\isasymin}\ setSubformulae\ F{\isadigit{1}}\ {\isasymor}\ G\ {\isasymin}\ setSubformulae\ F{\isadigit{2}}{\isachardoublequoteclose}\ \ \isanewline
\ \ \ \ \ \ \isacommand{by}\isamarkupfalse%
\ {\isacharparenleft}simp\ only{\isacharcolon}\ Un{\isacharunderscore}iff{\isacharparenright}\isanewline
\ \ \ \ \isacommand{then}\isamarkupfalse%
\ \isacommand{show}\isamarkupfalse%
\ {\isacharquery}thesis\isanewline
\ \ \ \ \isacommand{proof}\isamarkupfalse%
\ \isanewline
\ \ \ \ \ \ \isacommand{assume}\isamarkupfalse%
\ {\isachardoublequoteopen}G\ {\isasymin}\ setSubformulae\ F{\isadigit{1}}{\isachardoublequoteclose}\isanewline
\ \ \ \ \ \ \isacommand{then}\isamarkupfalse%
\ \isacommand{have}\isamarkupfalse%
\ {\isachardoublequoteopen}atoms\ G\ {\isasymsubseteq}\ atoms\ F{\isadigit{1}}{\isachardoublequoteclose}\isanewline
\ \ \ \ \ \ \ \ \isacommand{by}\isamarkupfalse%
\ {\isacharparenleft}rule\ assms{\isacharparenleft}{\isadigit{1}}{\isacharparenright}{\isacharparenright}\isanewline
\ \ \ \ \ \ \isacommand{also}\isamarkupfalse%
\ \isacommand{have}\isamarkupfalse%
\ {\isachardoublequoteopen}{\isasymdots}\ {\isasymsubseteq}\ atoms\ F{\isadigit{1}}\ {\isasymunion}\ atoms\ F{\isadigit{2}}{\isachardoublequoteclose}\isanewline
\ \ \ \ \ \ \ \ \isacommand{by}\isamarkupfalse%
\ {\isacharparenleft}simp\ only{\isacharcolon}\ Un{\isacharunderscore}upper{\isadigit{1}}{\isacharparenright}\isanewline
\ \ \ \ \ \ \isacommand{also}\isamarkupfalse%
\ \isacommand{have}\isamarkupfalse%
\ {\isachardoublequoteopen}{\isasymdots}\ {\isacharequal}\ atoms\ {\isacharparenleft}F{\isadigit{1}}\ \isactrlbold {\isasymrightarrow}\ F{\isadigit{2}}{\isacharparenright}{\isachardoublequoteclose}\isanewline
\ \ \ \ \ \ \ \ \isacommand{by}\isamarkupfalse%
\ {\isacharparenleft}simp\ only{\isacharcolon}\ formula{\isachardot}set{\isacharparenleft}{\isadigit{6}}{\isacharparenright}{\isacharparenright}\isanewline
\ \ \ \ \ \ \isacommand{finally}\isamarkupfalse%
\ \isacommand{show}\isamarkupfalse%
\ {\isacharquery}thesis\isanewline
\ \ \ \ \ \ \ \ \isacommand{by}\isamarkupfalse%
\ this\isanewline
\ \ \ \ \isacommand{next}\isamarkupfalse%
\isanewline
\ \ \ \ \ \ \isacommand{assume}\isamarkupfalse%
\ {\isachardoublequoteopen}G\ {\isasymin}\ setSubformulae\ F{\isadigit{2}}{\isachardoublequoteclose}\isanewline
\ \ \ \ \ \ \isacommand{then}\isamarkupfalse%
\ \isacommand{have}\isamarkupfalse%
\ {\isachardoublequoteopen}atoms\ G\ {\isasymsubseteq}\ atoms\ F{\isadigit{2}}{\isachardoublequoteclose}\isanewline
\ \ \ \ \ \ \ \ \isacommand{by}\isamarkupfalse%
\ {\isacharparenleft}rule\ assms{\isacharparenleft}{\isadigit{2}}{\isacharparenright}{\isacharparenright}\isanewline
\ \ \ \ \ \ \isacommand{also}\isamarkupfalse%
\ \isacommand{have}\isamarkupfalse%
\ {\isachardoublequoteopen}{\isasymdots}\ {\isasymsubseteq}\ atoms\ F{\isadigit{1}}\ {\isasymunion}\ atoms\ F{\isadigit{2}}{\isachardoublequoteclose}\isanewline
\ \ \ \ \ \ \ \ \isacommand{by}\isamarkupfalse%
\ {\isacharparenleft}simp\ only{\isacharcolon}\ Un{\isacharunderscore}upper{\isadigit{2}}{\isacharparenright}\isanewline
\ \ \ \ \ \ \isacommand{also}\isamarkupfalse%
\ \isacommand{have}\isamarkupfalse%
\ {\isachardoublequoteopen}{\isasymdots}\ {\isacharequal}\ atoms\ {\isacharparenleft}F{\isadigit{1}}\ \isactrlbold {\isasymrightarrow}\ F{\isadigit{2}}{\isacharparenright}{\isachardoublequoteclose}\isanewline
\ \ \ \ \ \ \ \ \isacommand{by}\isamarkupfalse%
\ {\isacharparenleft}simp\ only{\isacharcolon}\ formula{\isachardot}set{\isacharparenleft}{\isadigit{6}}{\isacharparenright}{\isacharparenright}\isanewline
\ \ \ \ \ \ \isacommand{finally}\isamarkupfalse%
\ \isacommand{show}\isamarkupfalse%
\ {\isacharquery}thesis\isanewline
\ \ \ \ \ \ \ \ \isacommand{by}\isamarkupfalse%
\ this\isanewline
\ \ \ \ \isacommand{qed}\isamarkupfalse%
\isanewline
\ \ \isacommand{qed}\isamarkupfalse%
\isanewline
\isacommand{qed}\isamarkupfalse%
%
\endisatagproof
{\isafoldproof}%
%
\isadelimproof
\isanewline
%
\endisadelimproof
\isanewline
\isacommand{lemma}\isamarkupfalse%
\ subformulae{\isacharunderscore}atoms{\isacharcolon}\ {\isachardoublequoteopen}G\ {\isasymin}\ setSubformulae\ F\ {\isasymLongrightarrow}\ atoms\ G\ {\isasymsubseteq}\ atoms\ F{\isachardoublequoteclose}\isanewline
%
\isadelimproof
%
\endisadelimproof
%
\isatagproof
\isacommand{proof}\isamarkupfalse%
\ {\isacharparenleft}induction\ F{\isacharparenright}\isanewline
\ \ \isacommand{case}\isamarkupfalse%
\ {\isacharparenleft}Atom\ x{\isacharparenright}\isanewline
\ \ \isacommand{then}\isamarkupfalse%
\ \isacommand{show}\isamarkupfalse%
\ {\isacharquery}case\ \isacommand{by}\isamarkupfalse%
\ {\isacharparenleft}simp\ only{\isacharcolon}\ subformulas{\isacharunderscore}atoms{\isacharunderscore}atom{\isacharparenright}\ \isanewline
\isacommand{next}\isamarkupfalse%
\isanewline
\ \ \isacommand{case}\isamarkupfalse%
\ Bot\isanewline
\ \ \isacommand{then}\isamarkupfalse%
\ \isacommand{show}\isamarkupfalse%
\ {\isacharquery}case\ \isacommand{by}\isamarkupfalse%
\ {\isacharparenleft}simp\ only{\isacharcolon}\ subformulas{\isacharunderscore}atoms{\isacharunderscore}bot{\isacharparenright}\isanewline
\isacommand{next}\isamarkupfalse%
\isanewline
\ \ \isacommand{case}\isamarkupfalse%
\ {\isacharparenleft}Not\ F{\isacharparenright}\isanewline
\ \ \isacommand{then}\isamarkupfalse%
\ \isacommand{show}\isamarkupfalse%
\ {\isacharquery}case\ \isacommand{by}\isamarkupfalse%
\ {\isacharparenleft}simp\ only{\isacharcolon}\ subformulas{\isacharunderscore}atoms{\isacharunderscore}not{\isacharparenright}\isanewline
\isacommand{next}\isamarkupfalse%
\isanewline
\ \ \isacommand{case}\isamarkupfalse%
\ {\isacharparenleft}And\ F{\isadigit{1}}\ F{\isadigit{2}}{\isacharparenright}\isanewline
\ \ \isacommand{then}\isamarkupfalse%
\ \isacommand{show}\isamarkupfalse%
\ {\isacharquery}case\ \isacommand{by}\isamarkupfalse%
\ {\isacharparenleft}simp\ only{\isacharcolon}\ subformulas{\isacharunderscore}atoms{\isacharunderscore}and{\isacharparenright}\isanewline
\isacommand{next}\isamarkupfalse%
\isanewline
\ \ \isacommand{case}\isamarkupfalse%
\ {\isacharparenleft}Or\ F{\isadigit{1}}\ F{\isadigit{2}}{\isacharparenright}\isanewline
\ \ \isacommand{then}\isamarkupfalse%
\ \isacommand{show}\isamarkupfalse%
\ {\isacharquery}case\ \isacommand{by}\isamarkupfalse%
\ {\isacharparenleft}simp\ only{\isacharcolon}\ subformulas{\isacharunderscore}atoms{\isacharunderscore}or{\isacharparenright}\isanewline
\isacommand{next}\isamarkupfalse%
\isanewline
\ \ \isacommand{case}\isamarkupfalse%
\ {\isacharparenleft}Imp\ F{\isadigit{1}}\ F{\isadigit{2}}{\isacharparenright}\isanewline
\ \ \isacommand{then}\isamarkupfalse%
\ \isacommand{show}\isamarkupfalse%
\ {\isacharquery}case\ \isacommand{by}\isamarkupfalse%
\ {\isacharparenleft}simp\ only{\isacharcolon}\ subformulas{\isacharunderscore}atoms{\isacharunderscore}imp{\isacharparenright}\isanewline
\isacommand{qed}\isamarkupfalse%
%
\endisatagproof
{\isafoldproof}%
%
\isadelimproof
%
\endisadelimproof
%
\begin{isamarkuptext}%
Por último, su demostración aplicativa automática.%
\end{isamarkuptext}\isamarkuptrue%
\isacommand{lemma}\isamarkupfalse%
\ {\isachardoublequoteopen}G\ {\isasymin}\ setSubformulae\ F\ {\isasymLongrightarrow}\ atoms\ G\ {\isasymsubseteq}\ atoms\ F{\isachardoublequoteclose}\isanewline
%
\isadelimproof
\ \ %
\endisadelimproof
%
\isatagproof
\isacommand{by}\isamarkupfalse%
\ {\isacharparenleft}induction\ F{\isacharparenright}\ auto%
\endisatagproof
{\isafoldproof}%
%
\isadelimproof
%
\endisadelimproof
%
\begin{isamarkuptext}%
A continuación voy a introducir un lema que no pertenece a la 
  teoría original de Isabelle pero facilita las siguientes 
  demostraciones detalladas mediante contenciones en cadena.

  \begin{lema}
    Sea \isa{G} subfórmula de \isa{F}, entonces el conjunto de subfórmulas de 
    \isa{G} está contenido en el de \isa{F}.
  \end{lema} 

  \begin{demostracion}
  La prueba es por inducción en la estructura de fórmula.
  
  Sea \isa{p} una fórmula atómica cualquiera. Entonces, bajo las
  condiciones del lema se tiene que \isa{G\ {\isacharequal}\ p}. Por lo tanto, tienen igual
  conjunto de subfórmulas.

  Sea la fórmula \isa{{\isasymbottom}}. Entonces, \isa{G\ {\isacharequal}\ {\isasymbottom}} y tienen igual conjunto de
  subfórmulas.

  Sea una fórmula \isa{F} tal que para toda subfórmula \isa{G}, se tiene que el
  conjunto de subfórmulas de \isa{G} está contenido en el de \isa{F}. Veamos la
  propiedad para \isa{{\isasymnot}\ F}. Sea \isa{G{\isacharprime}\ {\isasymin}\ Subf{\isacharparenleft}{\isasymnot}\ F{\isacharparenright}\ {\isacharequal}\ {\isacharbraceleft}{\isasymnot}\ F{\isacharbraceright}\ {\isasymunion}\ Subf{\isacharparenleft}F{\isacharparenright}}. 
  Entonces \isa{G{\isacharprime}\ {\isasymin}\ {\isacharbraceleft}{\isasymnot}\ F{\isacharbraceright}} o \isa{G{\isacharprime}\ {\isasymin}\ Subf{\isacharparenleft}F{\isacharparenright}}. 
  Del primer caso se obtiene que \isa{G{\isacharprime}\ {\isacharequal}\ {\isasymnot}\ F} y, por tanto, tienen igual 
  conjunto de subfórmulas. Del segundo caso se tiene \isa{G{\isacharprime}\ {\isasymin}\ Subf{\isacharparenleft}F{\isacharparenright}} y, 
  por hipótesis de inducción, el conjunto de subfórmulas de \isa{G{\isacharprime}} está 
  contenido en el de \isa{F}. Como, a su vez, el conjunto de subfórmulas 
  de \isa{F} está contenido en el de \isa{{\isasymnot}\ F} por definición, se verifica la
  propiedad como consecuencia de la transitividad de la contención.

  Sean las fórmulas \isa{F{\isadigit{1}}} y \isa{F{\isadigit{2}}} tales que para cualquier subfórmula \isa{G{\isadigit{1}}}
  de \isa{F{\isadigit{1}}} el conjunto de subfórmulas de \isa{G{\isadigit{1}}} está contenido en el de 
  \isa{F{\isadigit{1}}}, y para cualquier subfórmula \isa{G{\isadigit{2}}} de \isa{F{\isadigit{2}}} el conjunto de 
  subfórmulas de \isa{G{\isadigit{2}}} está contenido en el de \isa{F{\isadigit{2}}}. Veamos que se 
  verifica la propiedad para \isa{F{\isadigit{1}}{\isacharasterisk}F{\isadigit{2}}} donde \isa{{\isacharasterisk}} es cualquier conectiva 
  binaria. 
  Sea \isa{G{\isacharprime}\ {\isasymin}\ Subf{\isacharparenleft}F{\isadigit{1}}{\isacharasterisk}F{\isadigit{2}}{\isacharparenright}\ {\isacharequal}\ {\isacharbraceleft}F{\isadigit{1}}{\isacharasterisk}F{\isadigit{2}}{\isacharbraceright}\ {\isasymunion}\ Subf{\isacharparenleft}F{\isadigit{1}}{\isacharparenright}\ {\isasymunion}\ Subf{\isacharparenleft}F{\isadigit{2}}{\isacharparenright}}. De este modo,
  tenemos tres casos: \isa{G{\isacharprime}\ {\isasymin}\ {\isacharbraceleft}F{\isadigit{1}}{\isacharasterisk}F{\isadigit{2}}{\isacharbraceright}} o \isa{G{\isacharprime}\ {\isasymin}\ Subf{\isacharparenleft}F{\isadigit{1}}{\isacharparenright}} o 
  \isa{G{\isacharprime}\ {\isasymin}\ Subf{\isacharparenleft}F{\isadigit{2}}{\isacharparenright}}. De la primera opción se deduce \isa{G{\isacharprime}\ {\isacharequal}\ F{\isadigit{1}}{\isacharasterisk}F{\isadigit{2}}} y, por
  tanto, tienen igual conjunto de subfórmulas. Por otro lado, si 
  \isa{G{\isacharprime}\ {\isasymin}\ Subf{\isacharparenleft}F{\isadigit{1}}{\isacharparenright}}, por hipótesis de inducción se tiene que el conjunto
  de subfórmulas de \isa{G{\isacharprime}} está contenido en el de \isa{F{\isadigit{1}}}. Por tanto, 
  como el conjunto de subfórmulas de \isa{F{\isadigit{1}}} está a su vez contenido en el 
  de \isa{F{\isadigit{1}}{\isacharasterisk}F{\isadigit{2}}}, se verifica la propiedad por la transitividad de la 
  contención en cadena. El caso \isa{G{\isacharprime}\ {\isasymin}\ Subf{\isacharparenleft}F{\isadigit{2}}{\isacharparenright}} es análogo cambiando el 
  índice de la fórmula.   
  \end{demostracion}%
\end{isamarkuptext}\isamarkuptrue%
%
\begin{isamarkuptext}%
Veamos su formalización en Isabelle junto con su demostración 
  estructurada.%
\end{isamarkuptext}\isamarkuptrue%
\isacommand{lemma}\isamarkupfalse%
\ subContsubformulae{\isacharunderscore}atom{\isacharcolon}\ \isanewline
\ \ \isakeyword{assumes}\ {\isachardoublequoteopen}G\ {\isasymin}\ setSubformulae\ {\isacharparenleft}Atom\ x{\isacharparenright}{\isachardoublequoteclose}\ \isanewline
\ \ \isakeyword{shows}\ {\isachardoublequoteopen}setSubformulae\ G\ {\isasymsubseteq}\ setSubformulae\ {\isacharparenleft}Atom\ x{\isacharparenright}{\isachardoublequoteclose}\isanewline
%
\isadelimproof
%
\endisadelimproof
%
\isatagproof
\isacommand{proof}\isamarkupfalse%
\ {\isacharminus}\ \isanewline
\ \ \isacommand{have}\isamarkupfalse%
\ {\isachardoublequoteopen}G\ {\isasymin}\ {\isacharbraceleft}Atom\ x{\isacharbraceright}{\isachardoublequoteclose}\ \isacommand{using}\isamarkupfalse%
\ assms\ \isanewline
\ \ \ \ \isacommand{by}\isamarkupfalse%
\ {\isacharparenleft}simp\ only{\isacharcolon}\ setSubformulae{\isacharunderscore}atom{\isacharparenright}\isanewline
\ \ \isacommand{then}\isamarkupfalse%
\ \isacommand{have}\isamarkupfalse%
\ {\isachardoublequoteopen}G\ {\isacharequal}\ Atom\ x{\isachardoublequoteclose}\isanewline
\ \ \ \ \isacommand{by}\isamarkupfalse%
\ {\isacharparenleft}simp\ only{\isacharcolon}\ singletonD{\isacharparenright}\isanewline
\ \ \isacommand{then}\isamarkupfalse%
\ \isacommand{show}\isamarkupfalse%
\ {\isacharquery}thesis\isanewline
\ \ \ \ \isacommand{by}\isamarkupfalse%
\ {\isacharparenleft}simp\ only{\isacharcolon}\ subset{\isacharunderscore}refl{\isacharparenright}\isanewline
\isacommand{qed}\isamarkupfalse%
%
\endisatagproof
{\isafoldproof}%
%
\isadelimproof
\isanewline
%
\endisadelimproof
\isanewline
\isacommand{lemma}\isamarkupfalse%
\ subContsubformulae{\isacharunderscore}bot{\isacharcolon}\isanewline
\ \ \isakeyword{assumes}\ {\isachardoublequoteopen}G\ {\isasymin}\ setSubformulae\ {\isasymbottom}{\isachardoublequoteclose}\ \isanewline
\ \ \isakeyword{shows}\ \ \ {\isachardoublequoteopen}setSubformulae\ G\ {\isasymsubseteq}\ setSubformulae\ {\isasymbottom}{\isachardoublequoteclose}\isanewline
%
\isadelimproof
%
\endisadelimproof
%
\isatagproof
\isacommand{proof}\isamarkupfalse%
\ {\isacharminus}\isanewline
\ \ \isacommand{have}\isamarkupfalse%
\ {\isachardoublequoteopen}G\ {\isasymin}\ {\isacharbraceleft}{\isasymbottom}{\isacharbraceright}{\isachardoublequoteclose}\isanewline
\ \ \ \ \isacommand{using}\isamarkupfalse%
\ assms\isanewline
\ \ \ \ \isacommand{by}\isamarkupfalse%
\ {\isacharparenleft}simp\ only{\isacharcolon}\ setSubformulae{\isacharunderscore}bot{\isacharparenright}\isanewline
\ \ \isacommand{then}\isamarkupfalse%
\ \isacommand{have}\isamarkupfalse%
\ {\isachardoublequoteopen}G\ {\isacharequal}\ {\isasymbottom}{\isachardoublequoteclose}\isanewline
\ \ \ \ \isacommand{by}\isamarkupfalse%
\ {\isacharparenleft}simp\ only{\isacharcolon}\ singletonD{\isacharparenright}\isanewline
\ \ \isacommand{then}\isamarkupfalse%
\ \isacommand{show}\isamarkupfalse%
\ {\isacharquery}thesis\isanewline
\ \ \ \ \isacommand{by}\isamarkupfalse%
\ {\isacharparenleft}simp\ only{\isacharcolon}\ subset{\isacharunderscore}refl{\isacharparenright}\isanewline
\isacommand{qed}\isamarkupfalse%
%
\endisatagproof
{\isafoldproof}%
%
\isadelimproof
\isanewline
%
\endisadelimproof
\isanewline
\isacommand{lemma}\isamarkupfalse%
\ subContsubformulae{\isacharunderscore}not{\isacharcolon}\isanewline
\ \ \isakeyword{assumes}\ {\isachardoublequoteopen}G\ {\isasymin}\ setSubformulae\ F\ {\isasymLongrightarrow}\ setSubformulae\ G\ {\isasymsubseteq}\ setSubformulae\ F{\isachardoublequoteclose}\isanewline
\ \ \ \ \ \ \ \ \ \ {\isachardoublequoteopen}G\ {\isasymin}\ setSubformulae\ {\isacharparenleft}\isactrlbold {\isasymnot}\ F{\isacharparenright}{\isachardoublequoteclose}\isanewline
\ \ \isakeyword{shows}\ \ \ {\isachardoublequoteopen}setSubformulae\ G\ {\isasymsubseteq}\ setSubformulae\ {\isacharparenleft}\isactrlbold {\isasymnot}\ F{\isacharparenright}{\isachardoublequoteclose}\isanewline
%
\isadelimproof
%
\endisadelimproof
%
\isatagproof
\isacommand{proof}\isamarkupfalse%
\ {\isacharminus}\isanewline
\ \ \isacommand{have}\isamarkupfalse%
\ {\isachardoublequoteopen}G\ {\isasymin}\ {\isacharbraceleft}\isactrlbold {\isasymnot}\ F{\isacharbraceright}\ {\isasymunion}\ setSubformulae\ F{\isachardoublequoteclose}\isanewline
\ \ \ \ \isacommand{using}\isamarkupfalse%
\ assms{\isacharparenleft}{\isadigit{2}}{\isacharparenright}\isanewline
\ \ \ \ \isacommand{by}\isamarkupfalse%
\ {\isacharparenleft}simp\ only{\isacharcolon}\ setSubformulae{\isacharunderscore}not{\isacharparenright}\ \isanewline
\ \ \isacommand{then}\isamarkupfalse%
\ \isacommand{have}\isamarkupfalse%
\ {\isachardoublequoteopen}G\ {\isasymin}\ {\isacharbraceleft}\isactrlbold {\isasymnot}\ F{\isacharbraceright}\ {\isasymor}\ G\ {\isasymin}\ setSubformulae\ F{\isachardoublequoteclose}\isanewline
\ \ \ \ \isacommand{by}\isamarkupfalse%
\ {\isacharparenleft}simp\ only{\isacharcolon}\ Un{\isacharunderscore}iff{\isacharparenright}\isanewline
\ \ \isacommand{then}\isamarkupfalse%
\ \isacommand{show}\isamarkupfalse%
\ {\isachardoublequoteopen}setSubformulae\ G\ {\isasymsubseteq}\ setSubformulae\ {\isacharparenleft}\isactrlbold {\isasymnot}\ F{\isacharparenright}{\isachardoublequoteclose}\isanewline
\ \ \isacommand{proof}\isamarkupfalse%
\isanewline
\ \ \ \ \isacommand{assume}\isamarkupfalse%
\ {\isachardoublequoteopen}G\ {\isasymin}\ {\isacharbraceleft}\isactrlbold {\isasymnot}\ F{\isacharbraceright}{\isachardoublequoteclose}\isanewline
\ \ \ \ \isacommand{then}\isamarkupfalse%
\ \isacommand{have}\isamarkupfalse%
\ {\isachardoublequoteopen}G\ {\isacharequal}\ \isactrlbold {\isasymnot}\ F{\isachardoublequoteclose}\isanewline
\ \ \ \ \ \ \isacommand{by}\isamarkupfalse%
\ {\isacharparenleft}simp\ only{\isacharcolon}\ singletonD{\isacharparenright}\isanewline
\ \ \ \ \isacommand{then}\isamarkupfalse%
\ \isacommand{show}\isamarkupfalse%
\ {\isacharquery}thesis\isanewline
\ \ \ \ \ \ \isacommand{by}\isamarkupfalse%
\ {\isacharparenleft}simp\ only{\isacharcolon}\ subset{\isacharunderscore}refl{\isacharparenright}\isanewline
\ \ \isacommand{next}\isamarkupfalse%
\isanewline
\ \ \ \ \isacommand{assume}\isamarkupfalse%
\ {\isachardoublequoteopen}G\ {\isasymin}\ setSubformulae\ F{\isachardoublequoteclose}\isanewline
\ \ \ \ \isacommand{then}\isamarkupfalse%
\ \isacommand{have}\isamarkupfalse%
\ {\isadigit{1}}{\isacharcolon}{\isachardoublequoteopen}setSubformulae\ G\ {\isasymsubseteq}\ setSubformulae\ F{\isachardoublequoteclose}\isanewline
\ \ \ \ \ \ \isacommand{by}\isamarkupfalse%
\ {\isacharparenleft}simp\ only{\isacharcolon}\ assms{\isacharparenleft}{\isadigit{1}}{\isacharparenright}{\isacharparenright}\isanewline
\ \ \ \ \isacommand{also}\isamarkupfalse%
\ \isacommand{have}\isamarkupfalse%
\ {\isadigit{2}}{\isacharcolon}{\isachardoublequoteopen}setSubformulae\ F\ {\isasymsubseteq}\ setSubformulae\ {\isacharparenleft}\isactrlbold {\isasymnot}\ F{\isacharparenright}{\isachardoublequoteclose}\isanewline
\ \ \ \ \ \ \isacommand{by}\isamarkupfalse%
\ {\isacharparenleft}simp\ only{\isacharcolon}\ setSubformulae{\isacharunderscore}not\ Un{\isacharunderscore}upper{\isadigit{2}}{\isacharparenright}\isanewline
\ \ \ \ \isacommand{finally}\isamarkupfalse%
\ \isacommand{show}\isamarkupfalse%
\ {\isacharquery}thesis\isanewline
\ \ \ \ \ \ \isacommand{using}\isamarkupfalse%
\ {\isadigit{1}}\ {\isadigit{2}}\ \isacommand{by}\isamarkupfalse%
\ {\isacharparenleft}simp\ only{\isacharcolon}\ subset{\isacharunderscore}trans{\isacharparenright}\isanewline
\ \ \isacommand{qed}\isamarkupfalse%
\isanewline
\isacommand{qed}\isamarkupfalse%
%
\endisatagproof
{\isafoldproof}%
%
\isadelimproof
\isanewline
%
\endisadelimproof
\isanewline
\isacommand{lemma}\isamarkupfalse%
\ subContsubformulae{\isacharunderscore}and{\isacharcolon}\isanewline
\ \ \isakeyword{assumes}\ {\isachardoublequoteopen}G\ {\isasymin}\ setSubformulae\ F{\isadigit{1}}\ \isanewline
\ \ \ \ \ \ \ \ \ \ \ \ {\isasymLongrightarrow}\ setSubformulae\ G\ {\isasymsubseteq}\ setSubformulae\ F{\isadigit{1}}{\isachardoublequoteclose}\isanewline
\ \ \ \ \ \ \ \ \ \ {\isachardoublequoteopen}G\ {\isasymin}\ setSubformulae\ F{\isadigit{2}}\ \isanewline
\ \ \ \ \ \ \ \ \ \ \ \ {\isasymLongrightarrow}\ setSubformulae\ G\ {\isasymsubseteq}\ setSubformulae\ F{\isadigit{2}}{\isachardoublequoteclose}\isanewline
\ \ \ \ \ \ \ \ \ \ {\isachardoublequoteopen}G\ {\isasymin}\ setSubformulae\ {\isacharparenleft}F{\isadigit{1}}\ \isactrlbold {\isasymand}\ F{\isadigit{2}}{\isacharparenright}{\isachardoublequoteclose}\isanewline
\ \ \isakeyword{shows}\ \ \ {\isachardoublequoteopen}setSubformulae\ G\ {\isasymsubseteq}\ setSubformulae\ {\isacharparenleft}F{\isadigit{1}}\ \isactrlbold {\isasymand}\ F{\isadigit{2}}{\isacharparenright}{\isachardoublequoteclose}\isanewline
%
\isadelimproof
%
\endisadelimproof
%
\isatagproof
\isacommand{proof}\isamarkupfalse%
\ {\isacharminus}\isanewline
\ \ \isacommand{have}\isamarkupfalse%
\ {\isachardoublequoteopen}G\ {\isasymin}\ {\isacharbraceleft}F{\isadigit{1}}\ \isactrlbold {\isasymand}\ F{\isadigit{2}}{\isacharbraceright}\ {\isasymunion}\ {\isacharparenleft}setSubformulae\ F{\isadigit{1}}\ {\isasymunion}\ setSubformulae\ F{\isadigit{2}}{\isacharparenright}{\isachardoublequoteclose}\isanewline
\ \ \ \ \isacommand{using}\isamarkupfalse%
\ assms{\isacharparenleft}{\isadigit{3}}{\isacharparenright}\ \isanewline
\ \ \ \ \isacommand{by}\isamarkupfalse%
\ {\isacharparenleft}simp\ only{\isacharcolon}\ setSubformulae{\isacharunderscore}and{\isacharparenright}\isanewline
\ \ \isacommand{then}\isamarkupfalse%
\ \isacommand{have}\isamarkupfalse%
\ {\isachardoublequoteopen}G\ {\isasymin}\ {\isacharbraceleft}F{\isadigit{1}}\ \isactrlbold {\isasymand}\ F{\isadigit{2}}{\isacharbraceright}\ {\isasymor}\ G\ {\isasymin}\ setSubformulae\ F{\isadigit{1}}\ {\isasymunion}\ setSubformulae\ F{\isadigit{2}}{\isachardoublequoteclose}\isanewline
\ \ \ \ \isacommand{by}\isamarkupfalse%
\ {\isacharparenleft}simp\ only{\isacharcolon}\ Un{\isacharunderscore}iff{\isacharparenright}\isanewline
\ \ \isacommand{then}\isamarkupfalse%
\ \isacommand{show}\isamarkupfalse%
\ {\isacharquery}thesis\isanewline
\ \ \isacommand{proof}\isamarkupfalse%
\ \isanewline
\ \ \ \ \isacommand{assume}\isamarkupfalse%
\ {\isachardoublequoteopen}G\ {\isasymin}\ {\isacharbraceleft}F{\isadigit{1}}\ \isactrlbold {\isasymand}\ F{\isadigit{2}}{\isacharbraceright}{\isachardoublequoteclose}\isanewline
\ \ \ \ \isacommand{then}\isamarkupfalse%
\ \isacommand{have}\isamarkupfalse%
\ {\isachardoublequoteopen}G\ {\isacharequal}\ F{\isadigit{1}}\ \isactrlbold {\isasymand}\ F{\isadigit{2}}{\isachardoublequoteclose}\isanewline
\ \ \ \ \ \ \isacommand{by}\isamarkupfalse%
\ {\isacharparenleft}simp\ only{\isacharcolon}\ singletonD{\isacharparenright}\isanewline
\ \ \ \ \isacommand{then}\isamarkupfalse%
\ \isacommand{show}\isamarkupfalse%
\ {\isacharquery}thesis\isanewline
\ \ \ \ \ \ \isacommand{by}\isamarkupfalse%
\ {\isacharparenleft}simp\ only{\isacharcolon}\ subset{\isacharunderscore}refl{\isacharparenright}\isanewline
\ \ \isacommand{next}\isamarkupfalse%
\isanewline
\ \ \ \ \isacommand{assume}\isamarkupfalse%
\ {\isachardoublequoteopen}G\ {\isasymin}\ setSubformulae\ F{\isadigit{1}}\ {\isasymunion}\ setSubformulae\ F{\isadigit{2}}{\isachardoublequoteclose}\isanewline
\ \ \ \ \isacommand{then}\isamarkupfalse%
\ \isacommand{have}\isamarkupfalse%
\ {\isachardoublequoteopen}G\ {\isasymin}\ setSubformulae\ F{\isadigit{1}}\ {\isasymor}\ G\ {\isasymin}\ setSubformulae\ F{\isadigit{2}}{\isachardoublequoteclose}\ \ \isanewline
\ \ \ \ \ \ \isacommand{by}\isamarkupfalse%
\ {\isacharparenleft}simp\ only{\isacharcolon}\ Un{\isacharunderscore}iff{\isacharparenright}\isanewline
\ \ \ \ \isacommand{then}\isamarkupfalse%
\ \isacommand{show}\isamarkupfalse%
\ {\isacharquery}thesis\isanewline
\ \ \ \ \isacommand{proof}\isamarkupfalse%
\ \isanewline
\ \ \ \ \ \ \isacommand{assume}\isamarkupfalse%
\ {\isachardoublequoteopen}G\ {\isasymin}\ setSubformulae\ F{\isadigit{1}}{\isachardoublequoteclose}\isanewline
\ \ \ \ \ \ \isacommand{then}\isamarkupfalse%
\ \isacommand{have}\isamarkupfalse%
\ {\isachardoublequoteopen}setSubformulae\ G\ {\isasymsubseteq}\ setSubformulae\ F{\isadigit{1}}{\isachardoublequoteclose}\isanewline
\ \ \ \ \ \ \ \ \isacommand{by}\isamarkupfalse%
\ {\isacharparenleft}simp\ only{\isacharcolon}\ assms{\isacharparenleft}{\isadigit{1}}{\isacharparenright}{\isacharparenright}\isanewline
\ \ \ \ \ \ \isacommand{also}\isamarkupfalse%
\ \isacommand{have}\isamarkupfalse%
\ {\isachardoublequoteopen}{\isasymdots}\ {\isasymsubseteq}\ setSubformulae\ F{\isadigit{1}}\ {\isasymunion}\ setSubformulae\ F{\isadigit{2}}{\isachardoublequoteclose}\isanewline
\ \ \ \ \ \ \ \ \isacommand{by}\isamarkupfalse%
\ {\isacharparenleft}simp\ only{\isacharcolon}\ Un{\isacharunderscore}upper{\isadigit{1}}{\isacharparenright}\isanewline
\ \ \ \ \ \ \isacommand{also}\isamarkupfalse%
\ \isacommand{have}\isamarkupfalse%
\ {\isachardoublequoteopen}{\isasymdots}\ {\isasymsubseteq}\ setSubformulae\ {\isacharparenleft}F{\isadigit{1}}\ \isactrlbold {\isasymand}\ F{\isadigit{2}}{\isacharparenright}{\isachardoublequoteclose}\isanewline
\ \ \ \ \ \ \ \ \isacommand{by}\isamarkupfalse%
\ {\isacharparenleft}simp\ only{\isacharcolon}\ setSubformulae{\isacharunderscore}and\ Un{\isacharunderscore}upper{\isadigit{2}}{\isacharparenright}\isanewline
\ \ \ \ \ \ \isacommand{finally}\isamarkupfalse%
\ \isacommand{show}\isamarkupfalse%
\ {\isacharquery}thesis\isanewline
\ \ \ \ \ \ \ \ \isacommand{by}\isamarkupfalse%
\ this\isanewline
\ \ \ \ \isacommand{next}\isamarkupfalse%
\isanewline
\ \ \ \ \ \ \isacommand{assume}\isamarkupfalse%
\ {\isachardoublequoteopen}G\ {\isasymin}\ setSubformulae\ F{\isadigit{2}}{\isachardoublequoteclose}\isanewline
\ \ \ \ \ \ \isacommand{then}\isamarkupfalse%
\ \isacommand{have}\isamarkupfalse%
\ {\isachardoublequoteopen}setSubformulae\ G\ {\isasymsubseteq}\ setSubformulae\ F{\isadigit{2}}{\isachardoublequoteclose}\isanewline
\ \ \ \ \ \ \ \ \isacommand{by}\isamarkupfalse%
\ {\isacharparenleft}rule\ assms{\isacharparenleft}{\isadigit{2}}{\isacharparenright}{\isacharparenright}\isanewline
\ \ \ \ \ \ \isacommand{also}\isamarkupfalse%
\ \isacommand{have}\isamarkupfalse%
\ {\isachardoublequoteopen}{\isasymdots}\ {\isasymsubseteq}\ setSubformulae\ F{\isadigit{1}}\ {\isasymunion}\ setSubformulae\ F{\isadigit{2}}{\isachardoublequoteclose}\isanewline
\ \ \ \ \ \ \ \ \isacommand{by}\isamarkupfalse%
\ {\isacharparenleft}simp\ only{\isacharcolon}\ Un{\isacharunderscore}upper{\isadigit{2}}{\isacharparenright}\isanewline
\ \ \ \ \ \ \isacommand{also}\isamarkupfalse%
\ \isacommand{have}\isamarkupfalse%
\ {\isachardoublequoteopen}{\isasymdots}\ {\isasymsubseteq}\ setSubformulae\ {\isacharparenleft}F{\isadigit{1}}\ \isactrlbold {\isasymand}\ F{\isadigit{2}}{\isacharparenright}{\isachardoublequoteclose}\isanewline
\ \ \ \ \ \ \ \ \isacommand{by}\isamarkupfalse%
\ {\isacharparenleft}simp\ only{\isacharcolon}\ setSubformulae{\isacharunderscore}and\ Un{\isacharunderscore}upper{\isadigit{2}}{\isacharparenright}\isanewline
\ \ \ \ \ \ \isacommand{finally}\isamarkupfalse%
\ \isacommand{show}\isamarkupfalse%
\ {\isacharquery}thesis\isanewline
\ \ \ \ \ \ \ \ \isacommand{by}\isamarkupfalse%
\ this\isanewline
\ \ \ \ \isacommand{qed}\isamarkupfalse%
\isanewline
\ \ \isacommand{qed}\isamarkupfalse%
\isanewline
\isacommand{qed}\isamarkupfalse%
%
\endisatagproof
{\isafoldproof}%
%
\isadelimproof
\isanewline
%
\endisadelimproof
\isanewline
\isacommand{lemma}\isamarkupfalse%
\ subContsubformulae{\isacharunderscore}or{\isacharcolon}\isanewline
\ \ \isakeyword{assumes}\ {\isachardoublequoteopen}G\ {\isasymin}\ setSubformulae\ F{\isadigit{1}}\ \isanewline
\ \ \ \ \ \ \ \ \ \ \ \ {\isasymLongrightarrow}\ setSubformulae\ G\ {\isasymsubseteq}\ setSubformulae\ F{\isadigit{1}}{\isachardoublequoteclose}\isanewline
\ \ \ \ \ \ \ \ \ \ {\isachardoublequoteopen}G\ {\isasymin}\ setSubformulae\ F{\isadigit{2}}\ \isanewline
\ \ \ \ \ \ \ \ \ \ \ \ {\isasymLongrightarrow}\ setSubformulae\ G\ {\isasymsubseteq}\ setSubformulae\ F{\isadigit{2}}{\isachardoublequoteclose}\isanewline
\ \ \ \ \ \ \ \ \ \ {\isachardoublequoteopen}G\ {\isasymin}\ setSubformulae\ {\isacharparenleft}F{\isadigit{1}}\ \isactrlbold {\isasymor}\ F{\isadigit{2}}{\isacharparenright}{\isachardoublequoteclose}\isanewline
\ \ \isakeyword{shows}\ \ \ {\isachardoublequoteopen}setSubformulae\ G\ {\isasymsubseteq}\ setSubformulae\ {\isacharparenleft}F{\isadigit{1}}\ \isactrlbold {\isasymor}\ F{\isadigit{2}}{\isacharparenright}{\isachardoublequoteclose}\isanewline
%
\isadelimproof
%
\endisadelimproof
%
\isatagproof
\isacommand{proof}\isamarkupfalse%
\ {\isacharminus}\isanewline
\ \ \isacommand{have}\isamarkupfalse%
\ {\isachardoublequoteopen}G\ {\isasymin}\ {\isacharbraceleft}F{\isadigit{1}}\ \isactrlbold {\isasymor}\ F{\isadigit{2}}{\isacharbraceright}\ {\isasymunion}\ {\isacharparenleft}setSubformulae\ F{\isadigit{1}}\ {\isasymunion}\ setSubformulae\ F{\isadigit{2}}{\isacharparenright}{\isachardoublequoteclose}\isanewline
\ \ \ \ \isacommand{using}\isamarkupfalse%
\ assms{\isacharparenleft}{\isadigit{3}}{\isacharparenright}\ \isanewline
\ \ \ \ \isacommand{by}\isamarkupfalse%
\ {\isacharparenleft}simp\ only{\isacharcolon}\ setSubformulae{\isacharunderscore}or{\isacharparenright}\isanewline
\ \ \isacommand{then}\isamarkupfalse%
\ \isacommand{have}\isamarkupfalse%
\ {\isachardoublequoteopen}G\ {\isasymin}\ {\isacharbraceleft}F{\isadigit{1}}\ \isactrlbold {\isasymor}\ F{\isadigit{2}}{\isacharbraceright}\ {\isasymor}\ G\ {\isasymin}\ setSubformulae\ F{\isadigit{1}}\ {\isasymunion}\ setSubformulae\ F{\isadigit{2}}{\isachardoublequoteclose}\isanewline
\ \ \ \ \isacommand{by}\isamarkupfalse%
\ {\isacharparenleft}simp\ only{\isacharcolon}\ Un{\isacharunderscore}iff{\isacharparenright}\isanewline
\ \ \isacommand{then}\isamarkupfalse%
\ \isacommand{show}\isamarkupfalse%
\ {\isacharquery}thesis\isanewline
\ \ \isacommand{proof}\isamarkupfalse%
\ \isanewline
\ \ \ \ \isacommand{assume}\isamarkupfalse%
\ {\isachardoublequoteopen}G\ {\isasymin}\ {\isacharbraceleft}F{\isadigit{1}}\ \isactrlbold {\isasymor}\ F{\isadigit{2}}{\isacharbraceright}{\isachardoublequoteclose}\isanewline
\ \ \ \ \isacommand{then}\isamarkupfalse%
\ \isacommand{have}\isamarkupfalse%
\ {\isachardoublequoteopen}G\ {\isacharequal}\ F{\isadigit{1}}\ \isactrlbold {\isasymor}\ F{\isadigit{2}}{\isachardoublequoteclose}\isanewline
\ \ \ \ \ \ \isacommand{by}\isamarkupfalse%
\ {\isacharparenleft}simp\ only{\isacharcolon}\ singletonD{\isacharparenright}\isanewline
\ \ \ \ \isacommand{then}\isamarkupfalse%
\ \isacommand{show}\isamarkupfalse%
\ {\isacharquery}thesis\isanewline
\ \ \ \ \ \ \isacommand{by}\isamarkupfalse%
\ {\isacharparenleft}simp\ only{\isacharcolon}\ subset{\isacharunderscore}refl{\isacharparenright}\isanewline
\ \ \isacommand{next}\isamarkupfalse%
\isanewline
\ \ \ \ \isacommand{assume}\isamarkupfalse%
\ {\isachardoublequoteopen}G\ {\isasymin}\ setSubformulae\ F{\isadigit{1}}\ {\isasymunion}\ setSubformulae\ F{\isadigit{2}}{\isachardoublequoteclose}\isanewline
\ \ \ \ \isacommand{then}\isamarkupfalse%
\ \isacommand{have}\isamarkupfalse%
\ {\isachardoublequoteopen}G\ {\isasymin}\ setSubformulae\ F{\isadigit{1}}\ {\isasymor}\ G\ {\isasymin}\ setSubformulae\ F{\isadigit{2}}{\isachardoublequoteclose}\ \ \isanewline
\ \ \ \ \ \ \isacommand{by}\isamarkupfalse%
\ {\isacharparenleft}simp\ only{\isacharcolon}\ Un{\isacharunderscore}iff{\isacharparenright}\isanewline
\ \ \ \ \isacommand{then}\isamarkupfalse%
\ \isacommand{show}\isamarkupfalse%
\ {\isacharquery}thesis\isanewline
\ \ \ \ \isacommand{proof}\isamarkupfalse%
\ \isanewline
\ \ \ \ \ \ \isacommand{assume}\isamarkupfalse%
\ {\isachardoublequoteopen}G\ {\isasymin}\ setSubformulae\ F{\isadigit{1}}{\isachardoublequoteclose}\isanewline
\ \ \ \ \ \ \isacommand{then}\isamarkupfalse%
\ \isacommand{have}\isamarkupfalse%
\ {\isachardoublequoteopen}setSubformulae\ G\ {\isasymsubseteq}\ setSubformulae\ F{\isadigit{1}}{\isachardoublequoteclose}\isanewline
\ \ \ \ \ \ \ \ \isacommand{by}\isamarkupfalse%
\ {\isacharparenleft}simp\ only{\isacharcolon}\ assms{\isacharparenleft}{\isadigit{1}}{\isacharparenright}{\isacharparenright}\isanewline
\ \ \ \ \ \ \isacommand{also}\isamarkupfalse%
\ \isacommand{have}\isamarkupfalse%
\ {\isachardoublequoteopen}{\isasymdots}\ {\isasymsubseteq}\ setSubformulae\ F{\isadigit{1}}\ {\isasymunion}\ setSubformulae\ F{\isadigit{2}}{\isachardoublequoteclose}\isanewline
\ \ \ \ \ \ \ \ \isacommand{by}\isamarkupfalse%
\ {\isacharparenleft}simp\ only{\isacharcolon}\ Un{\isacharunderscore}upper{\isadigit{1}}{\isacharparenright}\isanewline
\ \ \ \ \ \ \isacommand{also}\isamarkupfalse%
\ \isacommand{have}\isamarkupfalse%
\ {\isachardoublequoteopen}{\isasymdots}\ {\isasymsubseteq}\ setSubformulae\ {\isacharparenleft}F{\isadigit{1}}\ \isactrlbold {\isasymor}\ F{\isadigit{2}}{\isacharparenright}{\isachardoublequoteclose}\isanewline
\ \ \ \ \ \ \ \ \isacommand{by}\isamarkupfalse%
\ {\isacharparenleft}simp\ only{\isacharcolon}\ setSubformulae{\isacharunderscore}or\ Un{\isacharunderscore}upper{\isadigit{2}}{\isacharparenright}\isanewline
\ \ \ \ \ \ \isacommand{finally}\isamarkupfalse%
\ \isacommand{show}\isamarkupfalse%
\ {\isacharquery}thesis\isanewline
\ \ \ \ \ \ \ \ \isacommand{by}\isamarkupfalse%
\ this\isanewline
\ \ \ \ \isacommand{next}\isamarkupfalse%
\isanewline
\ \ \ \ \ \ \isacommand{assume}\isamarkupfalse%
\ {\isachardoublequoteopen}G\ {\isasymin}\ setSubformulae\ F{\isadigit{2}}{\isachardoublequoteclose}\isanewline
\ \ \ \ \ \ \isacommand{then}\isamarkupfalse%
\ \isacommand{have}\isamarkupfalse%
\ {\isachardoublequoteopen}setSubformulae\ G\ {\isasymsubseteq}\ setSubformulae\ F{\isadigit{2}}{\isachardoublequoteclose}\isanewline
\ \ \ \ \ \ \ \ \isacommand{by}\isamarkupfalse%
\ {\isacharparenleft}rule\ assms{\isacharparenleft}{\isadigit{2}}{\isacharparenright}{\isacharparenright}\isanewline
\ \ \ \ \ \ \isacommand{also}\isamarkupfalse%
\ \isacommand{have}\isamarkupfalse%
\ {\isachardoublequoteopen}{\isasymdots}\ {\isasymsubseteq}\ setSubformulae\ F{\isadigit{1}}\ {\isasymunion}\ setSubformulae\ F{\isadigit{2}}{\isachardoublequoteclose}\isanewline
\ \ \ \ \ \ \ \ \isacommand{by}\isamarkupfalse%
\ {\isacharparenleft}simp\ only{\isacharcolon}\ Un{\isacharunderscore}upper{\isadigit{2}}{\isacharparenright}\isanewline
\ \ \ \ \ \ \isacommand{also}\isamarkupfalse%
\ \isacommand{have}\isamarkupfalse%
\ {\isachardoublequoteopen}{\isasymdots}\ {\isasymsubseteq}\ setSubformulae\ {\isacharparenleft}F{\isadigit{1}}\ \isactrlbold {\isasymor}\ F{\isadigit{2}}{\isacharparenright}{\isachardoublequoteclose}\isanewline
\ \ \ \ \ \ \ \ \isacommand{by}\isamarkupfalse%
\ {\isacharparenleft}simp\ only{\isacharcolon}\ setSubformulae{\isacharunderscore}or\ Un{\isacharunderscore}upper{\isadigit{2}}{\isacharparenright}\isanewline
\ \ \ \ \ \ \isacommand{finally}\isamarkupfalse%
\ \isacommand{show}\isamarkupfalse%
\ {\isacharquery}thesis\isanewline
\ \ \ \ \ \ \ \ \isacommand{by}\isamarkupfalse%
\ this\isanewline
\ \ \ \ \isacommand{qed}\isamarkupfalse%
\isanewline
\ \ \isacommand{qed}\isamarkupfalse%
\isanewline
\isacommand{qed}\isamarkupfalse%
%
\endisatagproof
{\isafoldproof}%
%
\isadelimproof
\isanewline
%
\endisadelimproof
\isanewline
\isacommand{lemma}\isamarkupfalse%
\ subContsubformulae{\isacharunderscore}imp{\isacharcolon}\isanewline
\ \ \isakeyword{assumes}\ {\isachardoublequoteopen}G\ {\isasymin}\ setSubformulae\ F{\isadigit{1}}\ \isanewline
\ \ \ \ \ \ \ \ \ \ \ \ {\isasymLongrightarrow}\ setSubformulae\ G\ {\isasymsubseteq}\ setSubformulae\ F{\isadigit{1}}{\isachardoublequoteclose}\isanewline
\ \ \ \ \ \ \ \ \ \ {\isachardoublequoteopen}G\ {\isasymin}\ setSubformulae\ F{\isadigit{2}}\ \isanewline
\ \ \ \ \ \ \ \ \ \ \ \ {\isasymLongrightarrow}\ setSubformulae\ G\ {\isasymsubseteq}\ setSubformulae\ F{\isadigit{2}}{\isachardoublequoteclose}\isanewline
\ \ \ \ \ \ \ \ \ \ {\isachardoublequoteopen}G\ {\isasymin}\ setSubformulae\ {\isacharparenleft}F{\isadigit{1}}\ \isactrlbold {\isasymrightarrow}\ F{\isadigit{2}}{\isacharparenright}{\isachardoublequoteclose}\isanewline
\ \ \isakeyword{shows}\ \ \ {\isachardoublequoteopen}setSubformulae\ G\ {\isasymsubseteq}\ setSubformulae\ {\isacharparenleft}F{\isadigit{1}}\ \isactrlbold {\isasymrightarrow}\ F{\isadigit{2}}{\isacharparenright}{\isachardoublequoteclose}\isanewline
%
\isadelimproof
%
\endisadelimproof
%
\isatagproof
\isacommand{proof}\isamarkupfalse%
\ {\isacharminus}\isanewline
\ \ \isacommand{have}\isamarkupfalse%
\ {\isachardoublequoteopen}G\ {\isasymin}\ {\isacharbraceleft}F{\isadigit{1}}\ \isactrlbold {\isasymrightarrow}\ F{\isadigit{2}}{\isacharbraceright}\ {\isasymunion}\ {\isacharparenleft}setSubformulae\ F{\isadigit{1}}\ {\isasymunion}\ setSubformulae\ F{\isadigit{2}}{\isacharparenright}{\isachardoublequoteclose}\isanewline
\ \ \ \ \isacommand{using}\isamarkupfalse%
\ assms{\isacharparenleft}{\isadigit{3}}{\isacharparenright}\ \isanewline
\ \ \ \ \isacommand{by}\isamarkupfalse%
\ {\isacharparenleft}simp\ only{\isacharcolon}\ setSubformulae{\isacharunderscore}imp{\isacharparenright}\isanewline
\ \ \isacommand{then}\isamarkupfalse%
\ \isacommand{have}\isamarkupfalse%
\ {\isachardoublequoteopen}G\ {\isasymin}\ {\isacharbraceleft}F{\isadigit{1}}\ \isactrlbold {\isasymrightarrow}\ F{\isadigit{2}}{\isacharbraceright}\ {\isasymor}\ G\ {\isasymin}\ setSubformulae\ F{\isadigit{1}}\ {\isasymunion}\ setSubformulae\ F{\isadigit{2}}{\isachardoublequoteclose}\isanewline
\ \ \ \ \isacommand{by}\isamarkupfalse%
\ {\isacharparenleft}simp\ only{\isacharcolon}\ Un{\isacharunderscore}iff{\isacharparenright}\isanewline
\ \ \isacommand{then}\isamarkupfalse%
\ \isacommand{show}\isamarkupfalse%
\ {\isacharquery}thesis\isanewline
\ \ \isacommand{proof}\isamarkupfalse%
\ \isanewline
\ \ \ \ \isacommand{assume}\isamarkupfalse%
\ {\isachardoublequoteopen}G\ {\isasymin}\ {\isacharbraceleft}F{\isadigit{1}}\ \isactrlbold {\isasymrightarrow}\ F{\isadigit{2}}{\isacharbraceright}{\isachardoublequoteclose}\isanewline
\ \ \ \ \isacommand{then}\isamarkupfalse%
\ \isacommand{have}\isamarkupfalse%
\ {\isachardoublequoteopen}G\ {\isacharequal}\ F{\isadigit{1}}\ \isactrlbold {\isasymrightarrow}\ F{\isadigit{2}}{\isachardoublequoteclose}\isanewline
\ \ \ \ \ \ \isacommand{by}\isamarkupfalse%
\ {\isacharparenleft}simp\ only{\isacharcolon}\ singletonD{\isacharparenright}\isanewline
\ \ \ \ \isacommand{then}\isamarkupfalse%
\ \isacommand{show}\isamarkupfalse%
\ {\isacharquery}thesis\isanewline
\ \ \ \ \ \ \isacommand{by}\isamarkupfalse%
\ {\isacharparenleft}simp\ only{\isacharcolon}\ subset{\isacharunderscore}refl{\isacharparenright}\isanewline
\ \ \isacommand{next}\isamarkupfalse%
\isanewline
\ \ \ \ \isacommand{assume}\isamarkupfalse%
\ {\isachardoublequoteopen}G\ {\isasymin}\ setSubformulae\ F{\isadigit{1}}\ {\isasymunion}\ setSubformulae\ F{\isadigit{2}}{\isachardoublequoteclose}\isanewline
\ \ \ \ \isacommand{then}\isamarkupfalse%
\ \isacommand{have}\isamarkupfalse%
\ {\isachardoublequoteopen}G\ {\isasymin}\ setSubformulae\ F{\isadigit{1}}\ {\isasymor}\ G\ {\isasymin}\ setSubformulae\ F{\isadigit{2}}{\isachardoublequoteclose}\ \ \isanewline
\ \ \ \ \ \ \isacommand{by}\isamarkupfalse%
\ {\isacharparenleft}simp\ only{\isacharcolon}\ Un{\isacharunderscore}iff{\isacharparenright}\isanewline
\ \ \ \ \isacommand{then}\isamarkupfalse%
\ \isacommand{show}\isamarkupfalse%
\ {\isacharquery}thesis\isanewline
\ \ \ \ \isacommand{proof}\isamarkupfalse%
\ \isanewline
\ \ \ \ \ \ \isacommand{assume}\isamarkupfalse%
\ {\isachardoublequoteopen}G\ {\isasymin}\ setSubformulae\ F{\isadigit{1}}{\isachardoublequoteclose}\isanewline
\ \ \ \ \ \ \isacommand{then}\isamarkupfalse%
\ \isacommand{have}\isamarkupfalse%
\ {\isachardoublequoteopen}setSubformulae\ G\ {\isasymsubseteq}\ setSubformulae\ F{\isadigit{1}}{\isachardoublequoteclose}\isanewline
\ \ \ \ \ \ \ \ \isacommand{by}\isamarkupfalse%
\ {\isacharparenleft}simp\ only{\isacharcolon}\ assms{\isacharparenleft}{\isadigit{1}}{\isacharparenright}{\isacharparenright}\isanewline
\ \ \ \ \ \ \isacommand{also}\isamarkupfalse%
\ \isacommand{have}\isamarkupfalse%
\ {\isachardoublequoteopen}{\isasymdots}\ {\isasymsubseteq}\ setSubformulae\ F{\isadigit{1}}\ {\isasymunion}\ setSubformulae\ F{\isadigit{2}}{\isachardoublequoteclose}\isanewline
\ \ \ \ \ \ \ \ \isacommand{by}\isamarkupfalse%
\ {\isacharparenleft}simp\ only{\isacharcolon}\ Un{\isacharunderscore}upper{\isadigit{1}}{\isacharparenright}\isanewline
\ \ \ \ \ \ \isacommand{also}\isamarkupfalse%
\ \isacommand{have}\isamarkupfalse%
\ {\isachardoublequoteopen}{\isasymdots}\ {\isasymsubseteq}\ setSubformulae\ {\isacharparenleft}F{\isadigit{1}}\ \isactrlbold {\isasymrightarrow}\ F{\isadigit{2}}{\isacharparenright}{\isachardoublequoteclose}\isanewline
\ \ \ \ \ \ \ \ \isacommand{by}\isamarkupfalse%
\ {\isacharparenleft}simp\ only{\isacharcolon}\ setSubformulae{\isacharunderscore}imp\ Un{\isacharunderscore}upper{\isadigit{2}}{\isacharparenright}\isanewline
\ \ \ \ \ \ \isacommand{finally}\isamarkupfalse%
\ \isacommand{show}\isamarkupfalse%
\ {\isacharquery}thesis\isanewline
\ \ \ \ \ \ \ \ \isacommand{by}\isamarkupfalse%
\ this\isanewline
\ \ \ \ \isacommand{next}\isamarkupfalse%
\isanewline
\ \ \ \ \ \ \isacommand{assume}\isamarkupfalse%
\ {\isachardoublequoteopen}G\ {\isasymin}\ setSubformulae\ F{\isadigit{2}}{\isachardoublequoteclose}\isanewline
\ \ \ \ \ \ \isacommand{then}\isamarkupfalse%
\ \isacommand{have}\isamarkupfalse%
\ {\isachardoublequoteopen}setSubformulae\ G\ {\isasymsubseteq}\ setSubformulae\ F{\isadigit{2}}{\isachardoublequoteclose}\isanewline
\ \ \ \ \ \ \ \ \isacommand{by}\isamarkupfalse%
\ {\isacharparenleft}rule\ assms{\isacharparenleft}{\isadigit{2}}{\isacharparenright}{\isacharparenright}\isanewline
\ \ \ \ \ \ \isacommand{also}\isamarkupfalse%
\ \isacommand{have}\isamarkupfalse%
\ {\isachardoublequoteopen}{\isasymdots}\ {\isasymsubseteq}\ setSubformulae\ F{\isadigit{1}}\ {\isasymunion}\ setSubformulae\ F{\isadigit{2}}{\isachardoublequoteclose}\isanewline
\ \ \ \ \ \ \ \ \isacommand{by}\isamarkupfalse%
\ {\isacharparenleft}simp\ only{\isacharcolon}\ Un{\isacharunderscore}upper{\isadigit{2}}{\isacharparenright}\isanewline
\ \ \ \ \ \ \isacommand{also}\isamarkupfalse%
\ \isacommand{have}\isamarkupfalse%
\ {\isachardoublequoteopen}{\isasymdots}\ {\isasymsubseteq}\ setSubformulae\ {\isacharparenleft}F{\isadigit{1}}\ \isactrlbold {\isasymrightarrow}\ F{\isadigit{2}}{\isacharparenright}{\isachardoublequoteclose}\isanewline
\ \ \ \ \ \ \ \ \isacommand{by}\isamarkupfalse%
\ {\isacharparenleft}simp\ only{\isacharcolon}\ setSubformulae{\isacharunderscore}imp\ Un{\isacharunderscore}upper{\isadigit{2}}{\isacharparenright}\isanewline
\ \ \ \ \ \ \isacommand{finally}\isamarkupfalse%
\ \isacommand{show}\isamarkupfalse%
\ {\isacharquery}thesis\isanewline
\ \ \ \ \ \ \ \ \isacommand{by}\isamarkupfalse%
\ this\isanewline
\ \ \ \ \isacommand{qed}\isamarkupfalse%
\isanewline
\ \ \isacommand{qed}\isamarkupfalse%
\isanewline
\isacommand{qed}\isamarkupfalse%
%
\endisatagproof
{\isafoldproof}%
%
\isadelimproof
\isanewline
%
\endisadelimproof
\isanewline
\isacommand{lemma}\isamarkupfalse%
\isanewline
\ \ {\isachardoublequoteopen}G\ {\isasymin}\ setSubformulae\ F\ {\isasymLongrightarrow}\ setSubformulae\ G\ {\isasymsubseteq}\ setSubformulae\ F{\isachardoublequoteclose}\isanewline
%
\isadelimproof
%
\endisadelimproof
%
\isatagproof
\isacommand{proof}\isamarkupfalse%
\ {\isacharparenleft}induction\ F{\isacharparenright}\isanewline
\isacommand{case}\isamarkupfalse%
\ {\isacharparenleft}Atom\ x{\isacharparenright}\isanewline
\ \ \isacommand{then}\isamarkupfalse%
\ \isacommand{show}\isamarkupfalse%
\ {\isacharquery}case\ \isacommand{by}\isamarkupfalse%
\ {\isacharparenleft}rule\ subContsubformulae{\isacharunderscore}atom{\isacharparenright}\isanewline
\isacommand{next}\isamarkupfalse%
\isanewline
\ \ \isacommand{case}\isamarkupfalse%
\ Bot\isanewline
\ \ \isacommand{then}\isamarkupfalse%
\ \isacommand{show}\isamarkupfalse%
\ {\isacharquery}case\ \isacommand{by}\isamarkupfalse%
\ {\isacharparenleft}rule\ subContsubformulae{\isacharunderscore}bot{\isacharparenright}\isanewline
\isacommand{next}\isamarkupfalse%
\isanewline
\isacommand{case}\isamarkupfalse%
\ {\isacharparenleft}Not\ F{\isacharparenright}\isanewline
\ \ \isacommand{then}\isamarkupfalse%
\ \isacommand{show}\isamarkupfalse%
\ {\isacharquery}case\ \isacommand{by}\isamarkupfalse%
\ {\isacharparenleft}rule\ subContsubformulae{\isacharunderscore}not{\isacharparenright}\isanewline
\isacommand{next}\isamarkupfalse%
\isanewline
\ \ \isacommand{case}\isamarkupfalse%
\ {\isacharparenleft}And\ F{\isadigit{1}}\ F{\isadigit{2}}{\isacharparenright}\isanewline
\ \ \isacommand{then}\isamarkupfalse%
\ \isacommand{show}\isamarkupfalse%
\ {\isacharquery}case\ \isacommand{by}\isamarkupfalse%
\ {\isacharparenleft}rule\ subContsubformulae{\isacharunderscore}and{\isacharparenright}\isanewline
\isacommand{next}\isamarkupfalse%
\isanewline
\ \ \isacommand{case}\isamarkupfalse%
\ {\isacharparenleft}Or\ F{\isadigit{1}}\ F{\isadigit{2}}{\isacharparenright}\isanewline
\ \ \isacommand{then}\isamarkupfalse%
\ \isacommand{show}\isamarkupfalse%
\ {\isacharquery}case\ \isacommand{by}\isamarkupfalse%
\ {\isacharparenleft}rule\ subContsubformulae{\isacharunderscore}or{\isacharparenright}\isanewline
\isacommand{next}\isamarkupfalse%
\isanewline
\ \ \isacommand{case}\isamarkupfalse%
\ {\isacharparenleft}Imp\ F{\isadigit{1}}\ F{\isadigit{2}}{\isacharparenright}\isanewline
\ \ \isacommand{then}\isamarkupfalse%
\ \isacommand{show}\isamarkupfalse%
\ {\isacharquery}case\ \isacommand{by}\isamarkupfalse%
\ {\isacharparenleft}rule\ subContsubformulae{\isacharunderscore}imp{\isacharparenright}\isanewline
\isacommand{qed}\isamarkupfalse%
%
\endisatagproof
{\isafoldproof}%
%
\isadelimproof
%
\endisadelimproof
%
\begin{isamarkuptext}%
Finalmente, su demostración automática se muestra a continuación.%
\end{isamarkuptext}\isamarkuptrue%
\isacommand{lemma}\isamarkupfalse%
\ subContsubformulae{\isacharcolon}\isanewline
\ \ {\isachardoublequoteopen}G\ {\isasymin}\ setSubformulae\ F\ {\isasymLongrightarrow}\ setSubformulae\ G\ {\isasymsubseteq}\ setSubformulae\ F{\isachardoublequoteclose}\isanewline
%
\isadelimproof
\ \ %
\endisadelimproof
%
\isatagproof
\isacommand{by}\isamarkupfalse%
\ {\isacharparenleft}induction\ F{\isacharparenright}\ auto%
\endisatagproof
{\isafoldproof}%
%
\isadelimproof
%
\endisadelimproof
%
\begin{isamarkuptext}%
El siguiente lema nos da la noción de transitividad de contención 
  en cadena de las subfórmulas, de modo que la subfórmula de una 
  subfórmula es del mismo modo subfórmula de la mayor.

  \begin{lema}
    Sean \isa{G} subfórmula de \isa{F} y \isa{H} subfórmula de \isa{G}, entonces 
    \isa{H} es subfórmula de \isa{F}.
  \end{lema}

  \begin{demostracion}
  La prueba está basada en el lema anterior. Hemos demostrado que como 
  \isa{G} es subfórmula de \isa{F}, entonces el conjunto de átomos de \isa{G} está
  contenido en el de \isa{F}. Del mismo modo, como \isa{H} es subfórmula de
  \isa{G}, su conjunto de átomos está contenido en el de \isa{G}. Por la
  transitividad de la contención, tenemos que el conjunto de átomos de 
  \isa{H} está contenido en el de \isa{F}. Por otro lema anterior, tenemos que
  \isa{H} pertenece a su propio conjunto de subfórmulas. Por tanto,
  \isa{H\ {\isasymin}\ Subf{\isacharparenleft}H{\isacharparenright}\ {\isasymsubseteq}\ Subf{\isacharparenleft}F{\isacharparenright}\ {\isasymLongrightarrow}\ H\ {\isasymin}\ Subf{\isacharparenleft}F{\isacharparenright}}.
  \end{demostracion}

  Veamos su formalización y prueba estructurada en Isabelle.

  Veamos su prueba según las distintas tácticas.%
\end{isamarkuptext}\isamarkuptrue%
\isacommand{lemma}\isamarkupfalse%
\isanewline
\ \ \isakeyword{assumes}\ {\isachardoublequoteopen}G\ {\isasymin}\ setSubformulae\ F{\isachardoublequoteclose}\ \isanewline
\ \ \ \ \ \ \ \ \ \ {\isachardoublequoteopen}H\ {\isasymin}\ setSubformulae\ G{\isachardoublequoteclose}\isanewline
\ \ \isakeyword{shows}\ \ \ {\isachardoublequoteopen}H\ {\isasymin}\ setSubformulae\ F{\isachardoublequoteclose}\isanewline
%
\isadelimproof
%
\endisadelimproof
%
\isatagproof
\isacommand{proof}\isamarkupfalse%
\ {\isacharminus}\isanewline
\ \ \isacommand{have}\isamarkupfalse%
\ {\isadigit{1}}{\isacharcolon}{\isachardoublequoteopen}setSubformulae\ G\ {\isasymsubseteq}\ setSubformulae\ F{\isachardoublequoteclose}\ \isacommand{using}\isamarkupfalse%
\ assms{\isacharparenleft}{\isadigit{1}}{\isacharparenright}\ \isanewline
\ \ \ \ \isacommand{by}\isamarkupfalse%
\ {\isacharparenleft}rule\ subContsubformulae{\isacharparenright}\isanewline
\ \ \isacommand{have}\isamarkupfalse%
\ {\isachardoublequoteopen}setSubformulae\ H\ {\isasymsubseteq}\ setSubformulae\ G{\isachardoublequoteclose}\ \isacommand{using}\isamarkupfalse%
\ assms{\isacharparenleft}{\isadigit{2}}{\isacharparenright}\ \isanewline
\ \ \ \ \isacommand{by}\isamarkupfalse%
\ {\isacharparenleft}rule\ subContsubformulae{\isacharparenright}\isanewline
\ \ \isacommand{then}\isamarkupfalse%
\ \isacommand{have}\isamarkupfalse%
\ {\isadigit{2}}{\isacharcolon}{\isachardoublequoteopen}setSubformulae\ H\ {\isasymsubseteq}\ setSubformulae\ F{\isachardoublequoteclose}\ \isacommand{using}\isamarkupfalse%
\ {\isadigit{1}}\ \isanewline
\ \ \ \ \isacommand{by}\isamarkupfalse%
\ {\isacharparenleft}rule\ subset{\isacharunderscore}trans{\isacharparenright}\isanewline
\ \ \isacommand{have}\isamarkupfalse%
\ {\isachardoublequoteopen}H\ {\isasymin}\ setSubformulae\ H{\isachardoublequoteclose}\ \isanewline
\ \ \ \ \isacommand{by}\isamarkupfalse%
\ {\isacharparenleft}simp\ only{\isacharcolon}\ subformulae{\isacharunderscore}self{\isacharparenright}\isanewline
\ \ \isacommand{then}\isamarkupfalse%
\ \isacommand{show}\isamarkupfalse%
\ {\isachardoublequoteopen}H\ {\isasymin}\ setSubformulae\ F{\isachardoublequoteclose}\ \isanewline
\ \ \ \ \isacommand{using}\isamarkupfalse%
\ {\isadigit{2}}\ \isanewline
\ \ \ \ \isacommand{by}\isamarkupfalse%
\ {\isacharparenleft}rule\ rev{\isacharunderscore}subsetD{\isacharparenright}\isanewline
\isacommand{qed}\isamarkupfalse%
%
\endisatagproof
{\isafoldproof}%
%
\isadelimproof
\isanewline
%
\endisadelimproof
\isanewline
\isacommand{lemma}\isamarkupfalse%
\ subsubformulae{\isacharcolon}\ \isanewline
\ \ {\isachardoublequoteopen}G\ {\isasymin}\ setSubformulae\ F\ \isanewline
\ \ \ {\isasymLongrightarrow}\ H\ {\isasymin}\ setSubformulae\ G\ \isanewline
\ \ \ {\isasymLongrightarrow}\ H\ {\isasymin}\ setSubformulae\ F{\isachardoublequoteclose}\isanewline
%
\isadelimproof
\ \ %
\endisadelimproof
%
\isatagproof
\isacommand{using}\isamarkupfalse%
\ subContsubformulae\ \isacommand{by}\isamarkupfalse%
\ blast%
\endisatagproof
{\isafoldproof}%
%
\isadelimproof
%
\endisadelimproof
%
\begin{isamarkuptext}%
comentario{Explicar el cambio de enunciado}%
\end{isamarkuptext}\isamarkuptrue%
%
\begin{isamarkuptext}%
A continuación presentamos otro resultado que relaciona los 
  conjuntos de subfórmulas según las conectivas que operen.%
\end{isamarkuptext}\isamarkuptrue%
\isacommand{lemma}\isamarkupfalse%
\ subformulas{\isacharunderscore}in{\isacharunderscore}subformulas{\isacharcolon}\isanewline
\ \ {\isachardoublequoteopen}G\ \isactrlbold {\isasymand}\ H\ {\isasymin}\ setSubformulae\ F\ \isanewline
\ \ {\isasymLongrightarrow}\ G\ {\isasymin}\ setSubformulae\ F\ {\isasymand}\ H\ {\isasymin}\ setSubformulae\ F{\isachardoublequoteclose}\isanewline
\ \ {\isachardoublequoteopen}G\ \isactrlbold {\isasymor}\ H\ {\isasymin}\ setSubformulae\ F\ \isanewline
\ \ {\isasymLongrightarrow}\ G\ {\isasymin}\ setSubformulae\ F\ {\isasymand}\ H\ {\isasymin}\ setSubformulae\ F{\isachardoublequoteclose}\isanewline
\ \ {\isachardoublequoteopen}G\ \isactrlbold {\isasymrightarrow}\ H\ {\isasymin}\ setSubformulae\ F\ \isanewline
\ \ {\isasymLongrightarrow}\ G\ {\isasymin}\ setSubformulae\ F\ {\isasymand}\ H\ {\isasymin}\ setSubformulae\ F{\isachardoublequoteclose}\isanewline
\ \ {\isachardoublequoteopen}\isactrlbold {\isasymnot}\ G\ {\isasymin}\ setSubformulae\ F\ {\isasymLongrightarrow}\ G\ {\isasymin}\ setSubformulae\ F{\isachardoublequoteclose}\isanewline
%
\isadelimproof
\ \ %
\endisadelimproof
%
\isatagproof
\isacommand{oops}\isamarkupfalse%
%
\endisatagproof
{\isafoldproof}%
%
\isadelimproof
%
\endisadelimproof
%
\begin{isamarkuptext}%
Como podemos observar, el resultado es análogo en todas las 
  conectivas binarias aunque aparezcan definidas por separado, por tanto 
  haré la demostración estructurada para una de ellas pues el resto son 
  análogas. 

  Nos basaremos en el lema anterior \isa{subsubformulae}.%
\end{isamarkuptext}\isamarkuptrue%
\isacommand{lemma}\isamarkupfalse%
\ subformulas{\isacharunderscore}in{\isacharunderscore}subformulas{\isacharunderscore}not{\isacharcolon}\isanewline
\ \ \isakeyword{assumes}\ {\isachardoublequoteopen}Not\ G\ {\isasymin}\ setSubformulae\ F{\isachardoublequoteclose}\isanewline
\ \ \isakeyword{shows}\ {\isachardoublequoteopen}G\ {\isasymin}\ setSubformulae\ F{\isachardoublequoteclose}\isanewline
%
\isadelimproof
%
\endisadelimproof
%
\isatagproof
\isacommand{proof}\isamarkupfalse%
\ {\isacharminus}\isanewline
\ \ \isacommand{have}\isamarkupfalse%
\ {\isachardoublequoteopen}setSubformulae\ {\isacharparenleft}Not\ G{\isacharparenright}\ {\isacharequal}\ {\isacharbraceleft}Not\ G{\isacharbraceright}\ {\isasymunion}\ setSubformulae\ G{\isachardoublequoteclose}\ \isanewline
\ \ \ \ \isacommand{by}\isamarkupfalse%
\ simp\ %
\isamarkupcmt{Pendiente%
}\isanewline
\ \ \isacommand{then}\isamarkupfalse%
\ \isacommand{have}\isamarkupfalse%
\ {\isadigit{1}}{\isacharcolon}{\isachardoublequoteopen}G\ {\isasymin}\ setSubformulae\ {\isacharparenleft}Not\ G{\isacharparenright}{\isachardoublequoteclose}\ \isanewline
\ \ \ \ \isacommand{by}\isamarkupfalse%
\ {\isacharparenleft}simp\ add{\isacharcolon}\ subformulae{\isacharunderscore}self{\isacharparenright}\ %
\isamarkupcmt{Pendiente%
}\isanewline
\ \ \isacommand{show}\isamarkupfalse%
\ {\isachardoublequoteopen}G\ {\isasymin}\ setSubformulae\ F{\isachardoublequoteclose}\ \isacommand{using}\isamarkupfalse%
\ assms\ {\isadigit{1}}\ \isanewline
\ \ \ \ \isacommand{by}\isamarkupfalse%
\ {\isacharparenleft}rule\ subsubformulae{\isacharparenright}\isanewline
\isacommand{qed}\isamarkupfalse%
%
\endisatagproof
{\isafoldproof}%
%
\isadelimproof
\isanewline
%
\endisadelimproof
\isanewline
\isacommand{lemma}\isamarkupfalse%
\ subformulas{\isacharunderscore}in{\isacharunderscore}subformulas{\isacharunderscore}and{\isacharcolon}\isanewline
\ \ \isakeyword{assumes}\ {\isachardoublequoteopen}G\ \isactrlbold {\isasymand}\ H\ {\isasymin}\ setSubformulae\ F{\isachardoublequoteclose}\ \isanewline
\ \ \isakeyword{shows}\ {\isachardoublequoteopen}G\ {\isasymin}\ setSubformulae\ F\ {\isasymand}\ H\ {\isasymin}\ setSubformulae\ F{\isachardoublequoteclose}\isanewline
%
\isadelimproof
%
\endisadelimproof
%
\isatagproof
\isacommand{proof}\isamarkupfalse%
\ {\isacharparenleft}rule\ conjI{\isacharparenright}\isanewline
\ \ \isacommand{have}\isamarkupfalse%
\ {\isadigit{1}}{\isacharcolon}\ {\isachardoublequoteopen}setSubformulae\ {\isacharparenleft}G\ \isactrlbold {\isasymand}\ H{\isacharparenright}\ {\isacharequal}\ \isanewline
\ \ \ \ \ \ \ \ \ \ {\isacharbraceleft}G\ \isactrlbold {\isasymand}\ H{\isacharbraceright}\ {\isasymunion}\ {\isacharparenleft}setSubformulae\ G\ {\isasymunion}\ setSubformulae\ H{\isacharparenright}{\isachardoublequoteclose}\ \isanewline
\ \ \ \ \isacommand{by}\isamarkupfalse%
\ {\isacharparenleft}simp\ only{\isacharcolon}\ setSubformulae{\isacharunderscore}and{\isacharparenright}\isanewline
\ \ \isacommand{then}\isamarkupfalse%
\ \isacommand{have}\isamarkupfalse%
\ {\isadigit{2}}{\isacharcolon}\ {\isachardoublequoteopen}G\ {\isasymin}\ setSubformulae\ {\isacharparenleft}G\ \isactrlbold {\isasymand}\ H{\isacharparenright}{\isachardoublequoteclose}\ \isanewline
\ \ \ \ \isacommand{by}\isamarkupfalse%
\ {\isacharparenleft}simp\ add{\isacharcolon}\ subformulae{\isacharunderscore}self{\isacharparenright}\ %
\isamarkupcmt{Pendiente%
}\ \isanewline
\ \ \isacommand{have}\isamarkupfalse%
\ {\isadigit{3}}{\isacharcolon}\ {\isachardoublequoteopen}H\ {\isasymin}\ setSubformulae\ {\isacharparenleft}G\ \isactrlbold {\isasymand}\ H{\isacharparenright}{\isachardoublequoteclose}\ \isanewline
\ \ \ \ \isacommand{using}\isamarkupfalse%
\ {\isadigit{1}}\ \isanewline
\ \ \ \ \isacommand{by}\isamarkupfalse%
\ {\isacharparenleft}simp\ add{\isacharcolon}\ subformulae{\isacharunderscore}self{\isacharparenright}\ %
\isamarkupcmt{Pendiente%
}\ \isanewline
\ \ \isacommand{show}\isamarkupfalse%
\ {\isachardoublequoteopen}G\ {\isasymin}\ setSubformulae\ F{\isachardoublequoteclose}\ \isacommand{using}\isamarkupfalse%
\ assms\ {\isadigit{2}}\ \isacommand{by}\isamarkupfalse%
\ {\isacharparenleft}rule\ subsubformulae{\isacharparenright}\isanewline
\ \ \isacommand{show}\isamarkupfalse%
\ {\isachardoublequoteopen}H\ {\isasymin}\ setSubformulae\ F{\isachardoublequoteclose}\ \isacommand{using}\isamarkupfalse%
\ assms\ {\isadigit{3}}\ \isacommand{by}\isamarkupfalse%
\ {\isacharparenleft}rule\ subsubformulae{\isacharparenright}\isanewline
\isacommand{qed}\isamarkupfalse%
%
\endisatagproof
{\isafoldproof}%
%
\isadelimproof
%
\endisadelimproof
%
\begin{isamarkuptext}%
Mostremos ahora la demostración automática.%
\end{isamarkuptext}\isamarkuptrue%
\isacommand{lemma}\isamarkupfalse%
\ subformulas{\isacharunderscore}in{\isacharunderscore}subformulas{\isacharcolon}\isanewline
\ \ {\isachardoublequoteopen}G\ \isactrlbold {\isasymand}\ H\ {\isasymin}\ setSubformulae\ F\ \isanewline
\ \ \ {\isasymLongrightarrow}\ G\ {\isasymin}\ setSubformulae\ F\ {\isasymand}\ H\ {\isasymin}\ setSubformulae\ F{\isachardoublequoteclose}\isanewline
\ \ {\isachardoublequoteopen}G\ \isactrlbold {\isasymor}\ H\ {\isasymin}\ setSubformulae\ F\ \isanewline
\ \ \ {\isasymLongrightarrow}\ G\ {\isasymin}\ setSubformulae\ F\ {\isasymand}\ H\ {\isasymin}\ setSubformulae\ F{\isachardoublequoteclose}\isanewline
\ \ {\isachardoublequoteopen}G\ \isactrlbold {\isasymrightarrow}\ H\ {\isasymin}\ setSubformulae\ F\ \isanewline
\ \ \ {\isasymLongrightarrow}\ G\ {\isasymin}\ setSubformulae\ F\ {\isasymand}\ H\ {\isasymin}\ setSubformulae\ F{\isachardoublequoteclose}\isanewline
\ \ {\isachardoublequoteopen}\isactrlbold {\isasymnot}\ G\ {\isasymin}\ setSubformulae\ F\ {\isasymLongrightarrow}\ G\ {\isasymin}\ setSubformulae\ F{\isachardoublequoteclose}\isanewline
%
\isadelimproof
\ \ %
\endisadelimproof
%
\isatagproof
\isacommand{using}\isamarkupfalse%
\ subformulae{\isacharunderscore}self\ subsubformulae\ \isacommand{apply}\isamarkupfalse%
\ force\isanewline
\ \ \isacommand{oops}\isamarkupfalse%
%
\endisatagproof
{\isafoldproof}%
%
\isadelimproof
%
\endisadelimproof
%
\begin{isamarkuptext}%
\comentario{Completar la prueba anterior.}%
\end{isamarkuptext}\isamarkuptrue%
%
\begin{isamarkuptext}%
HASTA AQUÍ HE TRABAJADO: 24/11/19%
\end{isamarkuptext}\isamarkuptrue%
%
\begin{isamarkuptext}%
Concluimos la sección de subfórmulas con un resultado que 
  relaciona varias funciones sobre la longitud de la lista 
  \isa{subformulae\ F} de una fórmula \isa{F} cualquiera.%
\end{isamarkuptext}\isamarkuptrue%
\isacommand{lemma}\isamarkupfalse%
\ length{\isacharunderscore}subformulae{\isacharcolon}\ {\isachardoublequoteopen}length\ {\isacharparenleft}subformulae\ F{\isacharparenright}\ {\isacharequal}\ size\ F{\isachardoublequoteclose}\ \isanewline
%
\isadelimproof
\ \ %
\endisadelimproof
%
\isatagproof
\isacommand{oops}\isamarkupfalse%
%
\endisatagproof
{\isafoldproof}%
%
\isadelimproof
%
\endisadelimproof
%
\begin{isamarkuptext}%
En primer lugar aparece la clase \isa{size} de la teoría de 
  números naturales ....

  Vamos a definir \isa{size{\isadigit{1}}} de manera idéntica a como aparece 
  \isa{size} en la teoría.%
\end{isamarkuptext}\isamarkuptrue%
\isacommand{class}\isamarkupfalse%
\ size{\isadigit{1}}\ {\isacharequal}\isanewline
\ \ \isakeyword{fixes}\ size{\isadigit{1}}\ {\isacharcolon}{\isacharcolon}\ {\isachardoublequoteopen}{\isacharprime}a\ {\isasymRightarrow}\ nat{\isachardoublequoteclose}\ \isanewline
\isanewline
\isacommand{instantiation}\isamarkupfalse%
\ nat\ {\isacharcolon}{\isacharcolon}\ size{\isadigit{1}}\isanewline
\isakeyword{begin}\isanewline
\isanewline
\isacommand{definition}\isamarkupfalse%
\ size{\isadigit{1}}{\isacharunderscore}nat\ \isakeyword{where}\ {\isacharbrackleft}simp{\isacharcomma}\ code{\isacharbrackright}{\isacharcolon}\ {\isachardoublequoteopen}size{\isadigit{1}}\ {\isacharparenleft}n{\isacharcolon}{\isacharcolon}nat{\isacharparenright}\ {\isacharequal}\ n{\isachardoublequoteclose}\isanewline
\isanewline
\isacommand{instance}\isamarkupfalse%
%
\isadelimproof
\ %
\endisadelimproof
%
\isatagproof
\isacommand{{\isachardot}{\isachardot}}\isamarkupfalse%
%
\endisatagproof
{\isafoldproof}%
%
\isadelimproof
%
\endisadelimproof
\isanewline
\isanewline
\isacommand{end}\isamarkupfalse%
%
\begin{isamarkuptext}%
Como podemos observar, se trata de una clase que actúa sobre un 
  parámetro global de tipo \isa{{\isacharprime}a} cualquiera. Por otro lado, 
  \isa{instantation} define una clase de parámetros, en este caso los 
  números naturales \isa{nat} que devuelve como resultado. Incluye una 
  definición concreta del operador \isa{size{\isadigit{1}}} sobre dichos parámetros. 
  Además, el último \isa{instance} abre una prueba que afirma que los 
  parámetros dados conforman la clase especificada en la definición. 
  Esta prueba que nos afirma que está bien definida la clase aparece
  omitida utilizando \isa{{\isachardot}{\isachardot}} .

  En particular, sobre una fórmula nos devuelve el número de elementos 
  de la lista cuyos elementos son los nodos y las hojas de su árbol de 
  formación.%
\end{isamarkuptext}\isamarkuptrue%
\isacommand{value}\isamarkupfalse%
\ {\isachardoublequoteopen}size\ {\isacharparenleft}n{\isacharcolon}{\isacharcolon}nat{\isacharparenright}\ {\isacharequal}\ n{\isachardoublequoteclose}\isanewline
\isacommand{value}\isamarkupfalse%
\ {\isachardoublequoteopen}size\ {\isacharparenleft}{\isadigit{5}}{\isacharcolon}{\isacharcolon}nat{\isacharparenright}\ {\isacharequal}\ {\isadigit{5}}{\isachardoublequoteclose}%
\begin{isamarkuptext}%
Por otro lado, la función \isa{length} de la teoría 
  \href{http://cort.as/-Stfm}{List.thy} nos indica la longitud de una 
  lista cualquiera de elementos, definiéndose utilizando el comando
  \isa{size} visto anteriormente.%
\end{isamarkuptext}\isamarkuptrue%
\isacommand{abbreviation}\isamarkupfalse%
\ length{\isacharprime}\ {\isacharcolon}{\isacharcolon}\ {\isachardoublequoteopen}{\isacharprime}a\ list\ {\isasymRightarrow}\ nat{\isachardoublequoteclose}\ \isakeyword{where}\isanewline
\ \ {\isachardoublequoteopen}length{\isacharprime}\ {\isasymequiv}\ size{\isachardoublequoteclose}%
\begin{isamarkuptext}%
Las demostración del resultado se vuelve a basar en la inducción 
  que nos despliega seis casos. 

  La prueba estructurada no resulta interesante, pues todos los casos 
  son inmediatos por simplificación como en el primer lema de esta 
  sección. 

  Incluimos a continuación la prueba automática.%
\end{isamarkuptext}\isamarkuptrue%
\isacommand{lemma}\isamarkupfalse%
\ length{\isacharunderscore}subformulae{\isacharcolon}\ {\isachardoublequoteopen}length\ {\isacharparenleft}subformulae\ F{\isacharparenright}\ {\isacharequal}\ size\ F{\isachardoublequoteclose}\ \isanewline
%
\isadelimproof
\ \ %
\endisadelimproof
%
\isatagproof
\isacommand{by}\isamarkupfalse%
\ {\isacharparenleft}induction\ F{\isacharsemicolon}\ simp{\isacharparenright}%
\endisatagproof
{\isafoldproof}%
%
\isadelimproof
%
\endisadelimproof
%
\begin{isamarkuptext}%
\comentario{Hacer la prueba detallada para mostrar los teoremas 
  utilizados.}%
\end{isamarkuptext}\isamarkuptrue%
%
\isadelimdocument
%
\endisadelimdocument
%
\isatagdocument
%
\isamarkupsection{Conectivas derivadas%
}
\isamarkuptrue%
%
\endisatagdocument
{\isafolddocument}%
%
\isadelimdocument
%
\endisadelimdocument
%
\begin{isamarkuptext}%
En esta sección definiremos nuevas conectivas y elementos a partir 
  de los ya definidos en el apartado anterior. Además veremos varios 
  resultados sobre los mismos.%
\end{isamarkuptext}\isamarkuptrue%
%
\begin{isamarkuptext}%
En primer lugar, vamos a definir \isa{Top{\isacharcolon}{\isacharcolon}\ {\isacharprime}a\ formula\ {\isasymRightarrow}\ bool} como 
  la constante  que devuelve el booleano contrario a \isa{Bot}. Se trata, 
  por tanto, de una constante de la misma naturaleza que la ya definida 
  para \isa{Bot}. De este modo, \isa{Top} será equivalente a \isa{Verdadero}, y 
  \isa{Bot} a \isa{Falso}, según se muestra en la siguiente ecuación. Su símbolo 
  queda igualmente retratado a continuación.%
\end{isamarkuptext}\isamarkuptrue%
\isacommand{definition}\isamarkupfalse%
\ Top\ {\isacharparenleft}{\isachardoublequoteopen}{\isasymtop}{\isachardoublequoteclose}{\isacharparenright}\ \isakeyword{where}\isanewline
\ \ {\isachardoublequoteopen}{\isasymtop}\ {\isasymequiv}\ {\isasymbottom}\ \isactrlbold {\isasymrightarrow}\ {\isasymbottom}{\isachardoublequoteclose}%
\begin{isamarkuptext}%
\comentario{Añadir la doble implicación como conectiva derivada.}%
\end{isamarkuptext}\isamarkuptrue%
%
\begin{isamarkuptext}%
Por la propia definición y naturaleza de \isa{Top}, verifica que no 
  contiene ninguna variable del alfabeto, como ya sabíamos análogamente 
  para \isa{Bot}. Tenemos así la siguiente propiedad.%
\end{isamarkuptext}\isamarkuptrue%
\isacommand{lemma}\isamarkupfalse%
\ top{\isacharunderscore}atoms{\isacharunderscore}simp{\isacharcolon}\ {\isachardoublequoteopen}atoms\ {\isasymtop}\ {\isacharequal}\ {\isacharbraceleft}{\isacharbraceright}{\isachardoublequoteclose}\ \isanewline
%
\isadelimproof
\ \ %
\endisadelimproof
%
\isatagproof
\isacommand{unfolding}\isamarkupfalse%
\ Top{\isacharunderscore}def\ \isacommand{by}\isamarkupfalse%
\ simp%
\endisatagproof
{\isafoldproof}%
%
\isadelimproof
%
\endisadelimproof
%
\begin{isamarkuptext}%
A continuación vamos a definir dos conectivas que generalizarán la 
  conjunción y la disyunción para una lista finita de fórmulas.%
\end{isamarkuptext}\isamarkuptrue%
\isacommand{primrec}\isamarkupfalse%
\ BigAnd\ {\isacharcolon}{\isacharcolon}\ {\isachardoublequoteopen}{\isacharprime}a\ formula\ list\ {\isasymRightarrow}\ {\isacharprime}a\ formula{\isachardoublequoteclose}\ {\isacharparenleft}{\isachardoublequoteopen}\isactrlbold {\isasymAnd}{\isacharunderscore}{\isachardoublequoteclose}{\isacharparenright}\ \isakeyword{where}\isanewline
\ \ {\isachardoublequoteopen}\isactrlbold {\isasymAnd}Nil\ {\isacharequal}\ {\isacharparenleft}\isactrlbold {\isasymnot}{\isasymbottom}{\isacharparenright}{\isachardoublequoteclose}\ \isanewline
{\isacharbar}\ {\isachardoublequoteopen}\isactrlbold {\isasymAnd}{\isacharparenleft}F{\isacharhash}Fs{\isacharparenright}\ {\isacharequal}\ F\ \isactrlbold {\isasymand}\ \isactrlbold {\isasymAnd}Fs{\isachardoublequoteclose}\isanewline
\isanewline
\isacommand{primrec}\isamarkupfalse%
\ BigOr\ {\isacharcolon}{\isacharcolon}\ {\isachardoublequoteopen}{\isacharprime}a\ formula\ list\ {\isasymRightarrow}\ {\isacharprime}a\ formula{\isachardoublequoteclose}\ {\isacharparenleft}{\isachardoublequoteopen}\isactrlbold {\isasymOr}{\isacharunderscore}{\isachardoublequoteclose}{\isacharparenright}\ \isakeyword{where}\isanewline
\ \ {\isachardoublequoteopen}\isactrlbold {\isasymOr}Nil\ {\isacharequal}\ {\isasymbottom}{\isachardoublequoteclose}\ \isanewline
{\isacharbar}\ {\isachardoublequoteopen}\isactrlbold {\isasymOr}{\isacharparenleft}F{\isacharhash}Fs{\isacharparenright}\ {\isacharequal}\ F\ \isactrlbold {\isasymor}\ \isactrlbold {\isasymOr}Fs{\isachardoublequoteclose}%
\begin{isamarkuptext}%
Ambas nuevas conectivas se caracterizarán por ser del tipo 
  funciones primitivas recursivas. Por tanto, sus definiciones se basan 
  en dos casos:
  \begin{description}
    \item[Lista vacía:] Representada como \isa{Nil}. En este caso, la 
      conjunción plural aplicada a la lista vacía nos devuelve la 
      negación de \isa{Bot}, es decir, \isa{Verdadero}, y la disyunción plural 
      sobre la lista vacía nos da simplemente \isa{Bot}, luego \isa{Falso}. 
    \item[Lista recursiva:] En este caso actúa sobre \isa{F{\isacharhash}Fs} donde \isa{F} 
      es una fórmula concatenada a la lista de fórmulas \isa{Fs}. Como es 
      lógico, \isa{BigAnd} nos devuelve la conjunción de todas las 
      fórmulas de la lista y \isa{BigOr} nos devuelve su disyunción.
  \end{description}

  Además, se observa en cada función el símbolo de notación que aparece 
  entre paréntesis.

  La conjunción plural nos da el siguiente resultado.%
\end{isamarkuptext}\isamarkuptrue%
\isacommand{lemma}\isamarkupfalse%
\ atoms{\isacharunderscore}BigAnd{\isacharbrackleft}simp{\isacharbrackright}{\isacharcolon}\ {\isachardoublequoteopen}atoms\ {\isacharparenleft}\isactrlbold {\isasymAnd}Fs{\isacharparenright}\ {\isacharequal}\ {\isasymUnion}{\isacharparenleft}atoms\ {\isacharbackquote}\ set\ Fs{\isacharparenright}{\isachardoublequoteclose}\isanewline
%
\isadelimproof
\ \ %
\endisadelimproof
%
\isatagproof
\isacommand{by}\isamarkupfalse%
\ {\isacharparenleft}induction\ Fs{\isacharsemicolon}\ simp{\isacharparenright}\isanewline
\isanewline
%
\endisatagproof
{\isafoldproof}%
%
\isadelimproof
%
\endisadelimproof
%
\isadelimtheory
%
\endisadelimtheory
%
\isatagtheory
%
\endisatagtheory
{\isafoldtheory}%
%
\isadelimtheory
%
\endisadelimtheory
%
\end{isabellebody}%
\endinput
%:%file=~/TFG/Logica_Proposicional/Sintaxis.thy%:%
%:%24=13%:%
%:%36=15%:%
%:%37=16%:%
%:%38=17%:%
%:%39=18%:%
%:%40=19%:%
%:%42=21%:%
%:%43=21%:%
%:%45=23%:%
%:%46=24%:%
%:%47=25%:%
%:%48=26%:%
%:%49=27%:%
%:%50=28%:%
%:%51=29%:%
%:%52=30%:%
%:%53=31%:%
%:%54=32%:%
%:%55=33%:%
%:%56=34%:%
%:%57=35%:%
%:%58=36%:%
%:%59=37%:%
%:%60=38%:%
%:%61=39%:%
%:%62=40%:%
%:%63=41%:%
%:%64=42%:%
%:%65=43%:%
%:%66=44%:%
%:%67=45%:%
%:%68=46%:%
%:%69=47%:%
%:%70=48%:%
%:%71=49%:%
%:%72=50%:%
%:%73=51%:%
%:%74=52%:%
%:%75=53%:%
%:%76=54%:%
%:%77=55%:%
%:%78=56%:%
%:%79=57%:%
%:%80=58%:%
%:%81=59%:%
%:%82=60%:%
%:%83=61%:%
%:%84=62%:%
%:%85=63%:%
%:%86=64%:%
%:%87=65%:%
%:%88=66%:%
%:%89=67%:%
%:%90=68%:%
%:%91=69%:%
%:%92=70%:%
%:%93=71%:%
%:%94=72%:%
%:%95=73%:%
%:%96=74%:%
%:%97=75%:%
%:%98=76%:%
%:%99=77%:%
%:%100=78%:%
%:%102=80%:%
%:%103=80%:%
%:%104=81%:%
%:%105=82%:%
%:%106=83%:%
%:%107=84%:%
%:%108=85%:%
%:%109=86%:%
%:%111=88%:%
%:%112=89%:%
%:%113=90%:%
%:%114=91%:%
%:%115=92%:%
%:%116=93%:%
%:%117=94%:%
%:%118=95%:%
%:%119=96%:%
%:%120=97%:%
%:%121=98%:%
%:%122=99%:%
%:%123=100%:%
%:%124=101%:%
%:%125=102%:%
%:%126=103%:%
%:%127=104%:%
%:%128=105%:%
%:%129=106%:%
%:%130=107%:%
%:%131=108%:%
%:%132=109%:%
%:%133=110%:%
%:%134=111%:%
%:%135=112%:%
%:%136=113%:%
%:%137=114%:%
%:%138=115%:%
%:%139=116%:%
%:%140=117%:%
%:%141=118%:%
%:%142=119%:%
%:%143=120%:%
%:%148=120%:%
%:%149=121%:%
%:%150=122%:%
%:%151=123%:%
%:%152=124%:%
%:%153=125%:%
%:%154=126%:%
%:%156=128%:%
%:%157=128%:%
%:%158=129%:%
%:%161=130%:%
%:%165=130%:%
%:%166=130%:%
%:%167=131%:%
%:%168=132%:%
%:%169=132%:%
%:%170=133%:%
%:%171=133%:%
%:%172=134%:%
%:%173=135%:%
%:%174=135%:%
%:%175=136%:%
%:%176=136%:%
%:%177=137%:%
%:%178=138%:%
%:%179=138%:%
%:%180=139%:%
%:%181=139%:%
%:%182=140%:%
%:%183=141%:%
%:%184=141%:%
%:%185=142%:%
%:%186=142%:%
%:%191=142%:%
%:%194=143%:%
%:%197=145%:%
%:%198=146%:%
%:%199=147%:%
%:%201=149%:%
%:%202=149%:%
%:%203=150%:%
%:%206=151%:%
%:%210=151%:%
%:%211=151%:%
%:%212=152%:%
%:%213=153%:%
%:%214=153%:%
%:%215=154%:%
%:%216=154%:%
%:%217=155%:%
%:%218=156%:%
%:%219=156%:%
%:%220=157%:%
%:%221=157%:%
%:%222=158%:%
%:%223=158%:%
%:%224=159%:%
%:%225=159%:%
%:%226=160%:%
%:%227=160%:%
%:%228=160%:%
%:%229=161%:%
%:%230=161%:%
%:%231=162%:%
%:%232=162%:%
%:%233=162%:%
%:%234=163%:%
%:%235=163%:%
%:%236=164%:%
%:%237=164%:%
%:%238=164%:%
%:%239=165%:%
%:%240=165%:%
%:%241=166%:%
%:%242=166%:%
%:%243=166%:%
%:%244=167%:%
%:%245=167%:%
%:%246=168%:%
%:%247=168%:%
%:%248=169%:%
%:%249=170%:%
%:%250=170%:%
%:%251=171%:%
%:%252=171%:%
%:%257=171%:%
%:%260=172%:%
%:%261=172%:%
%:%262=173%:%
%:%263=174%:%
%:%264=174%:%
%:%266=176%:%
%:%267=177%:%
%:%268=178%:%
%:%269=179%:%
%:%270=180%:%
%:%271=181%:%
%:%272=182%:%
%:%273=183%:%
%:%274=184%:%
%:%275=185%:%
%:%276=186%:%
%:%277=187%:%
%:%278=188%:%
%:%279=189%:%
%:%280=190%:%
%:%281=191%:%
%:%282=192%:%
%:%283=193%:%
%:%284=194%:%
%:%285=195%:%
%:%286=196%:%
%:%287=197%:%
%:%288=198%:%
%:%289=199%:%
%:%290=200%:%
%:%291=201%:%
%:%292=202%:%
%:%293=203%:%
%:%294=204%:%
%:%295=205%:%
%:%296=206%:%
%:%297=207%:%
%:%298=208%:%
%:%299=209%:%
%:%300=210%:%
%:%301=211%:%
%:%302=212%:%
%:%303=213%:%
%:%304=214%:%
%:%305=215%:%
%:%306=216%:%
%:%307=217%:%
%:%308=218%:%
%:%309=219%:%
%:%310=220%:%
%:%311=221%:%
%:%312=222%:%
%:%313=223%:%
%:%314=224%:%
%:%315=225%:%
%:%316=226%:%
%:%317=227%:%
%:%318=228%:%
%:%319=229%:%
%:%320=230%:%
%:%321=231%:%
%:%322=232%:%
%:%323=233%:%
%:%324=234%:%
%:%325=235%:%
%:%326=236%:%
%:%327=237%:%
%:%329=239%:%
%:%330=239%:%
%:%333=240%:%
%:%337=240%:%
%:%347=242%:%
%:%348=243%:%
%:%349=244%:%
%:%351=246%:%
%:%352=246%:%
%:%353=247%:%
%:%354=248%:%
%:%356=250%:%
%:%357=251%:%
%:%358=252%:%
%:%359=253%:%
%:%360=254%:%
%:%361=255%:%
%:%362=256%:%
%:%363=257%:%
%:%364=258%:%
%:%365=259%:%
%:%366=260%:%
%:%367=261%:%
%:%368=262%:%
%:%369=263%:%
%:%370=264%:%
%:%371=265%:%
%:%372=266%:%
%:%373=267%:%
%:%374=268%:%
%:%375=269%:%
%:%376=270%:%
%:%377=271%:%
%:%378=272%:%
%:%379=273%:%
%:%380=274%:%
%:%381=275%:%
%:%382=276%:%
%:%383=277%:%
%:%384=278%:%
%:%385=279%:%
%:%386=280%:%
%:%387=281%:%
%:%388=282%:%
%:%389=283%:%
%:%390=284%:%
%:%391=285%:%
%:%392=286%:%
%:%393=287%:%
%:%394=288%:%
%:%395=289%:%
%:%397=291%:%
%:%398=291%:%
%:%399=292%:%
%:%406=293%:%
%:%407=293%:%
%:%408=294%:%
%:%409=294%:%
%:%410=295%:%
%:%411=295%:%
%:%412=296%:%
%:%413=296%:%
%:%414=296%:%
%:%415=297%:%
%:%416=297%:%
%:%417=298%:%
%:%418=298%:%
%:%419=298%:%
%:%420=299%:%
%:%421=299%:%
%:%422=300%:%
%:%428=300%:%
%:%431=301%:%
%:%432=302%:%
%:%433=302%:%
%:%434=303%:%
%:%441=304%:%
%:%442=304%:%
%:%443=305%:%
%:%444=305%:%
%:%445=306%:%
%:%446=306%:%
%:%447=307%:%
%:%448=307%:%
%:%449=307%:%
%:%450=308%:%
%:%451=308%:%
%:%452=309%:%
%:%458=309%:%
%:%461=310%:%
%:%462=311%:%
%:%463=311%:%
%:%464=312%:%
%:%465=313%:%
%:%468=314%:%
%:%472=314%:%
%:%473=314%:%
%:%474=315%:%
%:%475=315%:%
%:%480=315%:%
%:%483=316%:%
%:%484=317%:%
%:%485=317%:%
%:%486=318%:%
%:%487=319%:%
%:%488=320%:%
%:%495=321%:%
%:%496=321%:%
%:%497=322%:%
%:%498=322%:%
%:%499=323%:%
%:%500=323%:%
%:%501=324%:%
%:%502=324%:%
%:%503=325%:%
%:%504=325%:%
%:%505=325%:%
%:%506=326%:%
%:%507=326%:%
%:%508=327%:%
%:%514=327%:%
%:%517=328%:%
%:%518=329%:%
%:%519=329%:%
%:%520=330%:%
%:%521=331%:%
%:%522=332%:%
%:%529=333%:%
%:%530=333%:%
%:%531=334%:%
%:%532=334%:%
%:%533=335%:%
%:%534=335%:%
%:%535=336%:%
%:%536=336%:%
%:%537=337%:%
%:%538=337%:%
%:%539=337%:%
%:%540=338%:%
%:%541=338%:%
%:%542=339%:%
%:%548=339%:%
%:%551=340%:%
%:%552=341%:%
%:%553=341%:%
%:%554=342%:%
%:%555=343%:%
%:%556=344%:%
%:%563=345%:%
%:%564=345%:%
%:%565=346%:%
%:%566=346%:%
%:%567=347%:%
%:%568=347%:%
%:%569=348%:%
%:%570=348%:%
%:%571=349%:%
%:%572=349%:%
%:%573=349%:%
%:%574=350%:%
%:%575=350%:%
%:%576=351%:%
%:%582=351%:%
%:%585=352%:%
%:%586=353%:%
%:%587=353%:%
%:%594=354%:%
%:%595=354%:%
%:%596=355%:%
%:%597=355%:%
%:%598=356%:%
%:%599=356%:%
%:%600=356%:%
%:%601=356%:%
%:%602=357%:%
%:%603=357%:%
%:%604=358%:%
%:%605=358%:%
%:%606=359%:%
%:%607=359%:%
%:%608=359%:%
%:%609=359%:%
%:%610=360%:%
%:%611=360%:%
%:%612=361%:%
%:%613=361%:%
%:%614=362%:%
%:%615=362%:%
%:%616=362%:%
%:%617=362%:%
%:%618=363%:%
%:%619=363%:%
%:%620=364%:%
%:%621=364%:%
%:%622=365%:%
%:%623=365%:%
%:%624=365%:%
%:%625=365%:%
%:%626=366%:%
%:%627=366%:%
%:%628=367%:%
%:%629=367%:%
%:%630=368%:%
%:%631=368%:%
%:%632=368%:%
%:%633=368%:%
%:%634=369%:%
%:%635=369%:%
%:%636=370%:%
%:%637=370%:%
%:%638=371%:%
%:%639=371%:%
%:%640=371%:%
%:%641=371%:%
%:%642=372%:%
%:%652=374%:%
%:%654=376%:%
%:%655=376%:%
%:%658=377%:%
%:%662=377%:%
%:%663=377%:%
%:%677=379%:%
%:%689=381%:%
%:%690=382%:%
%:%691=383%:%
%:%692=384%:%
%:%693=385%:%
%:%694=386%:%
%:%695=387%:%
%:%696=388%:%
%:%697=389%:%
%:%698=390%:%
%:%699=391%:%
%:%700=392%:%
%:%701=393%:%
%:%702=394%:%
%:%703=395%:%
%:%704=396%:%
%:%705=397%:%
%:%706=398%:%
%:%708=400%:%
%:%709=400%:%
%:%710=401%:%
%:%711=402%:%
%:%712=403%:%
%:%713=404%:%
%:%714=405%:%
%:%715=406%:%
%:%717=408%:%
%:%718=409%:%
%:%719=410%:%
%:%720=411%:%
%:%721=412%:%
%:%723=414%:%
%:%724=414%:%
%:%725=415%:%
%:%728=416%:%
%:%732=416%:%
%:%733=416%:%
%:%734=417%:%
%:%735=418%:%
%:%736=418%:%
%:%737=419%:%
%:%738=419%:%
%:%739=420%:%
%:%740=421%:%
%:%741=421%:%
%:%742=422%:%
%:%743=422%:%
%:%744=423%:%
%:%745=424%:%
%:%746=424%:%
%:%748=426%:%
%:%749=427%:%
%:%750=427%:%
%:%751=428%:%
%:%752=429%:%
%:%753=429%:%
%:%754=430%:%
%:%755=430%:%
%:%756=431%:%
%:%757=432%:%
%:%758=432%:%
%:%759=433%:%
%:%760=434%:%
%:%761=434%:%
%:%766=434%:%
%:%769=435%:%
%:%772=437%:%
%:%773=438%:%
%:%774=439%:%
%:%776=441%:%
%:%777=441%:%
%:%778=442%:%
%:%780=444%:%
%:%781=445%:%
%:%782=446%:%
%:%783=447%:%
%:%784=448%:%
%:%785=449%:%
%:%786=450%:%
%:%787=451%:%
%:%789=453%:%
%:%790=453%:%
%:%791=454%:%
%:%794=455%:%
%:%798=455%:%
%:%799=455%:%
%:%800=456%:%
%:%801=457%:%
%:%802=457%:%
%:%803=458%:%
%:%804=458%:%
%:%805=459%:%
%:%806=460%:%
%:%807=460%:%
%:%809=462%:%
%:%810=463%:%
%:%811=463%:%
%:%816=463%:%
%:%819=464%:%
%:%822=466%:%
%:%823=467%:%
%:%824=468%:%
%:%825=469%:%
%:%826=470%:%
%:%827=471%:%
%:%828=472%:%
%:%829=473%:%
%:%830=474%:%
%:%831=475%:%
%:%832=476%:%
%:%833=477%:%
%:%835=479%:%
%:%836=479%:%
%:%839=480%:%
%:%843=480%:%
%:%844=480%:%
%:%853=482%:%
%:%854=483%:%
%:%856=485%:%
%:%857=485%:%
%:%858=486%:%
%:%861=487%:%
%:%865=487%:%
%:%866=487%:%
%:%871=487%:%
%:%874=488%:%
%:%875=489%:%
%:%876=489%:%
%:%877=490%:%
%:%880=491%:%
%:%884=491%:%
%:%885=491%:%
%:%890=491%:%
%:%893=492%:%
%:%894=493%:%
%:%895=493%:%
%:%896=494%:%
%:%903=495%:%
%:%904=495%:%
%:%905=496%:%
%:%906=496%:%
%:%907=497%:%
%:%908=497%:%
%:%909=498%:%
%:%910=498%:%
%:%911=498%:%
%:%912=499%:%
%:%913=499%:%
%:%914=500%:%
%:%915=500%:%
%:%916=500%:%
%:%917=501%:%
%:%918=501%:%
%:%919=502%:%
%:%925=502%:%
%:%928=503%:%
%:%929=504%:%
%:%930=504%:%
%:%931=505%:%
%:%932=506%:%
%:%939=507%:%
%:%940=507%:%
%:%941=508%:%
%:%942=508%:%
%:%943=509%:%
%:%944=510%:%
%:%945=510%:%
%:%946=511%:%
%:%947=511%:%
%:%948=511%:%
%:%949=512%:%
%:%950=512%:%
%:%951=513%:%
%:%952=513%:%
%:%953=513%:%
%:%954=514%:%
%:%955=514%:%
%:%956=515%:%
%:%957=515%:%
%:%958=515%:%
%:%959=516%:%
%:%960=516%:%
%:%961=517%:%
%:%967=517%:%
%:%970=518%:%
%:%971=519%:%
%:%972=519%:%
%:%973=520%:%
%:%974=521%:%
%:%981=522%:%
%:%982=522%:%
%:%983=523%:%
%:%984=523%:%
%:%985=524%:%
%:%986=525%:%
%:%987=525%:%
%:%988=526%:%
%:%989=526%:%
%:%990=526%:%
%:%991=527%:%
%:%992=527%:%
%:%993=528%:%
%:%994=528%:%
%:%995=528%:%
%:%996=529%:%
%:%997=529%:%
%:%998=530%:%
%:%999=530%:%
%:%1000=530%:%
%:%1001=531%:%
%:%1002=531%:%
%:%1003=532%:%
%:%1009=532%:%
%:%1012=533%:%
%:%1013=534%:%
%:%1014=534%:%
%:%1015=535%:%
%:%1016=536%:%
%:%1023=537%:%
%:%1024=537%:%
%:%1025=538%:%
%:%1026=538%:%
%:%1027=539%:%
%:%1028=540%:%
%:%1029=540%:%
%:%1030=541%:%
%:%1031=541%:%
%:%1032=541%:%
%:%1033=542%:%
%:%1034=542%:%
%:%1035=543%:%
%:%1036=543%:%
%:%1037=543%:%
%:%1038=544%:%
%:%1039=544%:%
%:%1040=545%:%
%:%1041=545%:%
%:%1042=545%:%
%:%1043=546%:%
%:%1044=546%:%
%:%1045=547%:%
%:%1055=549%:%
%:%1056=550%:%
%:%1057=551%:%
%:%1058=552%:%
%:%1059=553%:%
%:%1060=554%:%
%:%1061=555%:%
%:%1062=556%:%
%:%1063=557%:%
%:%1064=558%:%
%:%1065=559%:%
%:%1066=560%:%
%:%1067=561%:%
%:%1068=562%:%
%:%1069=563%:%
%:%1070=564%:%
%:%1071=565%:%
%:%1072=566%:%
%:%1073=567%:%
%:%1074=568%:%
%:%1075=569%:%
%:%1076=570%:%
%:%1077=571%:%
%:%1078=572%:%
%:%1079=573%:%
%:%1080=574%:%
%:%1081=575%:%
%:%1082=576%:%
%:%1083=577%:%
%:%1084=578%:%
%:%1085=579%:%
%:%1086=580%:%
%:%1088=582%:%
%:%1089=582%:%
%:%1096=583%:%
%:%1097=583%:%
%:%1098=584%:%
%:%1099=584%:%
%:%1100=585%:%
%:%1101=585%:%
%:%1102=585%:%
%:%1103=586%:%
%:%1104=586%:%
%:%1105=587%:%
%:%1106=587%:%
%:%1107=588%:%
%:%1108=588%:%
%:%1109=589%:%
%:%1110=589%:%
%:%1111=589%:%
%:%1112=590%:%
%:%1113=590%:%
%:%1114=591%:%
%:%1115=591%:%
%:%1116=592%:%
%:%1117=592%:%
%:%1118=593%:%
%:%1119=593%:%
%:%1120=593%:%
%:%1121=594%:%
%:%1122=594%:%
%:%1123=595%:%
%:%1124=595%:%
%:%1125=596%:%
%:%1126=596%:%
%:%1127=597%:%
%:%1128=597%:%
%:%1129=597%:%
%:%1130=598%:%
%:%1131=598%:%
%:%1132=599%:%
%:%1133=599%:%
%:%1134=600%:%
%:%1135=600%:%
%:%1136=601%:%
%:%1137=601%:%
%:%1138=601%:%
%:%1139=602%:%
%:%1140=602%:%
%:%1141=603%:%
%:%1142=603%:%
%:%1143=604%:%
%:%1144=604%:%
%:%1145=605%:%
%:%1146=605%:%
%:%1147=605%:%
%:%1148=606%:%
%:%1149=606%:%
%:%1150=607%:%
%:%1160=609%:%
%:%1162=611%:%
%:%1163=611%:%
%:%1166=612%:%
%:%1170=612%:%
%:%1171=612%:%
%:%1180=614%:%
%:%1181=615%:%
%:%1182=616%:%
%:%1183=617%:%
%:%1184=618%:%
%:%1185=619%:%
%:%1186=620%:%
%:%1187=621%:%
%:%1189=623%:%
%:%1190=623%:%
%:%1191=624%:%
%:%1192=625%:%
%:%1199=626%:%
%:%1200=626%:%
%:%1201=627%:%
%:%1202=627%:%
%:%1203=628%:%
%:%1204=628%:%
%:%1205=629%:%
%:%1206=629%:%
%:%1207=630%:%
%:%1208=630%:%
%:%1209=630%:%
%:%1210=631%:%
%:%1211=631%:%
%:%1212=632%:%
%:%1218=632%:%
%:%1221=633%:%
%:%1222=634%:%
%:%1223=634%:%
%:%1224=635%:%
%:%1225=636%:%
%:%1232=637%:%
%:%1233=637%:%
%:%1234=638%:%
%:%1235=638%:%
%:%1236=639%:%
%:%1237=639%:%
%:%1238=640%:%
%:%1239=640%:%
%:%1240=641%:%
%:%1241=641%:%
%:%1242=641%:%
%:%1243=642%:%
%:%1244=642%:%
%:%1245=643%:%
%:%1255=645%:%
%:%1256=646%:%
%:%1258=648%:%
%:%1259=648%:%
%:%1262=649%:%
%:%1266=649%:%
%:%1267=649%:%
%:%1272=649%:%
%:%1275=650%:%
%:%1276=651%:%
%:%1277=651%:%
%:%1280=652%:%
%:%1284=652%:%
%:%1285=652%:%
%:%1294=654%:%
%:%1295=655%:%
%:%1296=656%:%
%:%1297=657%:%
%:%1298=658%:%
%:%1299=659%:%
%:%1300=660%:%
%:%1301=661%:%
%:%1302=662%:%
%:%1303=663%:%
%:%1304=664%:%
%:%1305=665%:%
%:%1306=666%:%
%:%1307=667%:%
%:%1308=668%:%
%:%1309=669%:%
%:%1310=670%:%
%:%1311=671%:%
%:%1312=672%:%
%:%1313=673%:%
%:%1314=674%:%
%:%1315=675%:%
%:%1316=676%:%
%:%1317=677%:%
%:%1318=678%:%
%:%1319=679%:%
%:%1320=680%:%
%:%1321=681%:%
%:%1322=682%:%
%:%1323=683%:%
%:%1324=684%:%
%:%1325=685%:%
%:%1326=686%:%
%:%1327=687%:%
%:%1328=688%:%
%:%1329=689%:%
%:%1330=690%:%
%:%1331=691%:%
%:%1332=692%:%
%:%1333=693%:%
%:%1334=694%:%
%:%1334=695%:%
%:%1335=696%:%
%:%1336=697%:%
%:%1337=698%:%
%:%1338=699%:%
%:%1340=701%:%
%:%1341=701%:%
%:%1344=702%:%
%:%1348=702%:%
%:%1358=704%:%
%:%1359=705%:%
%:%1360=706%:%
%:%1361=707%:%
%:%1362=708%:%
%:%1363=709%:%
%:%1364=710%:%
%:%1365=711%:%
%:%1366=712%:%
%:%1367=713%:%
%:%1368=714%:%
%:%1369=715%:%
%:%1370=716%:%
%:%1371=717%:%
%:%1373=719%:%
%:%1374=719%:%
%:%1375=720%:%
%:%1378=721%:%
%:%1382=721%:%
%:%1383=721%:%
%:%1384=722%:%
%:%1385=723%:%
%:%1386=723%:%
%:%1387=724%:%
%:%1388=724%:%
%:%1389=725%:%
%:%1390=726%:%
%:%1391=726%:%
%:%1392=727%:%
%:%1393=728%:%
%:%1394=728%:%
%:%1395=729%:%
%:%1396=730%:%
%:%1397=730%:%
%:%1398=731%:%
%:%1399=732%:%
%:%1400=732%:%
%:%1405=732%:%
%:%1408=733%:%
%:%1411=735%:%
%:%1412=736%:%
%:%1413=737%:%
%:%1414=738%:%
%:%1415=739%:%
%:%1416=740%:%
%:%1417=741%:%
%:%1418=742%:%
%:%1419=743%:%
%:%1420=744%:%
%:%1421=745%:%
%:%1422=746%:%
%:%1423=747%:%
%:%1424=748%:%
%:%1425=749%:%
%:%1427=751%:%
%:%1428=751%:%
%:%1429=752%:%
%:%1436=753%:%
%:%1437=753%:%
%:%1438=754%:%
%:%1439=754%:%
%:%1440=755%:%
%:%1441=755%:%
%:%1442=756%:%
%:%1443=756%:%
%:%1444=756%:%
%:%1445=757%:%
%:%1446=757%:%
%:%1447=758%:%
%:%1448=758%:%
%:%1449=758%:%
%:%1450=759%:%
%:%1451=759%:%
%:%1452=760%:%
%:%1453=760%:%
%:%1454=760%:%
%:%1455=761%:%
%:%1456=761%:%
%:%1457=762%:%
%:%1458=762%:%
%:%1459=762%:%
%:%1460=763%:%
%:%1461=763%:%
%:%1462=764%:%
%:%1463=764%:%
%:%1464=764%:%
%:%1465=765%:%
%:%1466=765%:%
%:%1467=766%:%
%:%1473=766%:%
%:%1476=767%:%
%:%1477=768%:%
%:%1478=768%:%
%:%1479=769%:%
%:%1486=770%:%
%:%1487=770%:%
%:%1488=771%:%
%:%1489=771%:%
%:%1490=772%:%
%:%1491=772%:%
%:%1492=773%:%
%:%1493=773%:%
%:%1494=773%:%
%:%1495=774%:%
%:%1496=774%:%
%:%1497=775%:%
%:%1498=775%:%
%:%1499=775%:%
%:%1500=776%:%
%:%1501=776%:%
%:%1502=777%:%
%:%1503=777%:%
%:%1504=777%:%
%:%1505=778%:%
%:%1506=778%:%
%:%1507=779%:%
%:%1513=779%:%
%:%1516=780%:%
%:%1517=781%:%
%:%1518=781%:%
%:%1519=782%:%
%:%1520=783%:%
%:%1527=784%:%
%:%1528=784%:%
%:%1529=785%:%
%:%1530=785%:%
%:%1531=786%:%
%:%1532=786%:%
%:%1533=787%:%
%:%1534=787%:%
%:%1535=787%:%
%:%1536=788%:%
%:%1537=788%:%
%:%1538=789%:%
%:%1539=789%:%
%:%1540=789%:%
%:%1541=790%:%
%:%1542=790%:%
%:%1543=791%:%
%:%1544=791%:%
%:%1545=791%:%
%:%1546=792%:%
%:%1547=792%:%
%:%1548=793%:%
%:%1549=793%:%
%:%1550=793%:%
%:%1551=794%:%
%:%1552=794%:%
%:%1553=795%:%
%:%1559=795%:%
%:%1562=796%:%
%:%1563=797%:%
%:%1564=797%:%
%:%1565=798%:%
%:%1566=799%:%
%:%1567=800%:%
%:%1574=801%:%
%:%1575=801%:%
%:%1576=802%:%
%:%1577=802%:%
%:%1578=803%:%
%:%1579=803%:%
%:%1580=804%:%
%:%1581=804%:%
%:%1582=804%:%
%:%1583=805%:%
%:%1584=805%:%
%:%1585=806%:%
%:%1586=806%:%
%:%1587=806%:%
%:%1588=807%:%
%:%1589=807%:%
%:%1590=808%:%
%:%1591=808%:%
%:%1592=809%:%
%:%1593=809%:%
%:%1594=809%:%
%:%1595=810%:%
%:%1596=810%:%
%:%1597=811%:%
%:%1598=811%:%
%:%1599=811%:%
%:%1600=812%:%
%:%1601=812%:%
%:%1602=813%:%
%:%1603=813%:%
%:%1604=813%:%
%:%1605=814%:%
%:%1606=814%:%
%:%1607=815%:%
%:%1613=815%:%
%:%1616=816%:%
%:%1617=817%:%
%:%1618=817%:%
%:%1619=818%:%
%:%1620=819%:%
%:%1621=820%:%
%:%1628=821%:%
%:%1629=821%:%
%:%1630=822%:%
%:%1631=822%:%
%:%1632=823%:%
%:%1633=823%:%
%:%1634=824%:%
%:%1635=824%:%
%:%1636=824%:%
%:%1637=825%:%
%:%1638=825%:%
%:%1639=826%:%
%:%1640=826%:%
%:%1641=826%:%
%:%1642=827%:%
%:%1643=827%:%
%:%1644=828%:%
%:%1645=828%:%
%:%1646=829%:%
%:%1647=829%:%
%:%1648=829%:%
%:%1649=830%:%
%:%1650=830%:%
%:%1651=831%:%
%:%1652=831%:%
%:%1653=831%:%
%:%1654=832%:%
%:%1655=832%:%
%:%1656=833%:%
%:%1657=833%:%
%:%1658=833%:%
%:%1659=834%:%
%:%1660=834%:%
%:%1661=835%:%
%:%1667=835%:%
%:%1670=836%:%
%:%1671=837%:%
%:%1672=837%:%
%:%1673=838%:%
%:%1674=839%:%
%:%1675=840%:%
%:%1682=841%:%
%:%1683=841%:%
%:%1684=842%:%
%:%1685=842%:%
%:%1686=843%:%
%:%1687=843%:%
%:%1688=844%:%
%:%1689=844%:%
%:%1690=844%:%
%:%1691=845%:%
%:%1692=845%:%
%:%1693=846%:%
%:%1694=846%:%
%:%1695=846%:%
%:%1696=847%:%
%:%1697=847%:%
%:%1698=848%:%
%:%1699=848%:%
%:%1700=849%:%
%:%1701=849%:%
%:%1702=849%:%
%:%1703=850%:%
%:%1704=850%:%
%:%1705=851%:%
%:%1706=851%:%
%:%1707=851%:%
%:%1708=852%:%
%:%1709=852%:%
%:%1710=853%:%
%:%1711=853%:%
%:%1712=853%:%
%:%1713=854%:%
%:%1714=854%:%
%:%1715=855%:%
%:%1721=855%:%
%:%1724=856%:%
%:%1725=857%:%
%:%1726=857%:%
%:%1727=858%:%
%:%1734=859%:%
%:%1735=859%:%
%:%1736=860%:%
%:%1737=860%:%
%:%1738=861%:%
%:%1739=861%:%
%:%1740=861%:%
%:%1741=861%:%
%:%1742=862%:%
%:%1743=862%:%
%:%1744=863%:%
%:%1745=863%:%
%:%1746=864%:%
%:%1747=864%:%
%:%1748=864%:%
%:%1749=864%:%
%:%1750=865%:%
%:%1751=865%:%
%:%1752=866%:%
%:%1753=866%:%
%:%1754=867%:%
%:%1755=867%:%
%:%1756=867%:%
%:%1757=867%:%
%:%1758=868%:%
%:%1759=868%:%
%:%1760=869%:%
%:%1761=869%:%
%:%1762=870%:%
%:%1763=870%:%
%:%1764=870%:%
%:%1765=870%:%
%:%1766=871%:%
%:%1767=871%:%
%:%1768=872%:%
%:%1769=872%:%
%:%1770=873%:%
%:%1771=873%:%
%:%1772=873%:%
%:%1773=873%:%
%:%1774=874%:%
%:%1775=874%:%
%:%1776=875%:%
%:%1777=875%:%
%:%1778=876%:%
%:%1779=876%:%
%:%1780=876%:%
%:%1781=876%:%
%:%1782=877%:%
%:%1792=879%:%
%:%1793=880%:%
%:%1795=882%:%
%:%1796=882%:%
%:%1799=883%:%
%:%1803=883%:%
%:%1804=883%:%
%:%1813=885%:%
%:%1814=886%:%
%:%1815=887%:%
%:%1816=888%:%
%:%1817=889%:%
%:%1818=890%:%
%:%1819=891%:%
%:%1820=892%:%
%:%1821=893%:%
%:%1822=894%:%
%:%1823=895%:%
%:%1824=896%:%
%:%1825=897%:%
%:%1826=898%:%
%:%1827=899%:%
%:%1828=900%:%
%:%1829=901%:%
%:%1830=902%:%
%:%1831=903%:%
%:%1832=904%:%
%:%1833=905%:%
%:%1834=906%:%
%:%1835=907%:%
%:%1836=908%:%
%:%1837=909%:%
%:%1838=910%:%
%:%1839=911%:%
%:%1840=912%:%
%:%1841=913%:%
%:%1842=914%:%
%:%1843=915%:%
%:%1844=916%:%
%:%1845=917%:%
%:%1846=918%:%
%:%1847=919%:%
%:%1848=920%:%
%:%1849=921%:%
%:%1850=922%:%
%:%1851=923%:%
%:%1852=924%:%
%:%1853=925%:%
%:%1854=926%:%
%:%1855=927%:%
%:%1856=928%:%
%:%1857=929%:%
%:%1858=930%:%
%:%1859=931%:%
%:%1860=932%:%
%:%1861=933%:%
%:%1862=934%:%
%:%1863=935%:%
%:%1864=936%:%
%:%1865=937%:%
%:%1866=938%:%
%:%1868=940%:%
%:%1869=940%:%
%:%1872=941%:%
%:%1876=941%:%
%:%1886=943%:%
%:%1888=945%:%
%:%1889=945%:%
%:%1890=946%:%
%:%1891=947%:%
%:%1898=948%:%
%:%1899=948%:%
%:%1900=949%:%
%:%1901=949%:%
%:%1902=950%:%
%:%1903=950%:%
%:%1904=951%:%
%:%1905=951%:%
%:%1906=952%:%
%:%1907=952%:%
%:%1908=952%:%
%:%1909=953%:%
%:%1910=953%:%
%:%1911=954%:%
%:%1912=954%:%
%:%1913=954%:%
%:%1914=955%:%
%:%1915=955%:%
%:%1916=956%:%
%:%1922=956%:%
%:%1925=957%:%
%:%1926=958%:%
%:%1927=958%:%
%:%1928=959%:%
%:%1929=960%:%
%:%1936=961%:%
%:%1937=961%:%
%:%1938=962%:%
%:%1939=962%:%
%:%1940=963%:%
%:%1941=963%:%
%:%1942=964%:%
%:%1943=964%:%
%:%1944=965%:%
%:%1945=965%:%
%:%1946=965%:%
%:%1947=966%:%
%:%1948=966%:%
%:%1949=967%:%
%:%1950=967%:%
%:%1951=967%:%
%:%1952=968%:%
%:%1953=968%:%
%:%1954=969%:%
%:%1960=969%:%
%:%1963=970%:%
%:%1964=971%:%
%:%1965=971%:%
%:%1966=972%:%
%:%1967=973%:%
%:%1968=974%:%
%:%1975=975%:%
%:%1976=975%:%
%:%1977=976%:%
%:%1978=976%:%
%:%1979=977%:%
%:%1980=977%:%
%:%1981=978%:%
%:%1982=978%:%
%:%1983=979%:%
%:%1984=979%:%
%:%1985=979%:%
%:%1986=980%:%
%:%1987=980%:%
%:%1988=981%:%
%:%1989=981%:%
%:%1990=981%:%
%:%1991=982%:%
%:%1992=982%:%
%:%1993=983%:%
%:%1994=983%:%
%:%1995=984%:%
%:%1996=984%:%
%:%1997=984%:%
%:%1998=985%:%
%:%1999=985%:%
%:%2000=986%:%
%:%2001=986%:%
%:%2002=986%:%
%:%2003=987%:%
%:%2004=987%:%
%:%2005=988%:%
%:%2006=988%:%
%:%2007=989%:%
%:%2008=989%:%
%:%2009=990%:%
%:%2010=990%:%
%:%2011=990%:%
%:%2012=991%:%
%:%2013=991%:%
%:%2014=992%:%
%:%2015=992%:%
%:%2016=992%:%
%:%2017=993%:%
%:%2018=993%:%
%:%2019=994%:%
%:%2020=994%:%
%:%2021=994%:%
%:%2022=995%:%
%:%2023=995%:%
%:%2024=996%:%
%:%2025=996%:%
%:%2026=997%:%
%:%2032=997%:%
%:%2035=998%:%
%:%2036=999%:%
%:%2037=999%:%
%:%2038=1000%:%
%:%2039=1001%:%
%:%2040=1002%:%
%:%2041=1003%:%
%:%2048=1004%:%
%:%2049=1004%:%
%:%2050=1005%:%
%:%2051=1005%:%
%:%2052=1006%:%
%:%2053=1006%:%
%:%2054=1007%:%
%:%2055=1007%:%
%:%2056=1008%:%
%:%2057=1008%:%
%:%2058=1008%:%
%:%2059=1009%:%
%:%2060=1009%:%
%:%2061=1010%:%
%:%2062=1010%:%
%:%2063=1010%:%
%:%2064=1011%:%
%:%2065=1011%:%
%:%2066=1012%:%
%:%2067=1012%:%
%:%2068=1013%:%
%:%2069=1013%:%
%:%2070=1013%:%
%:%2071=1014%:%
%:%2072=1014%:%
%:%2073=1015%:%
%:%2074=1015%:%
%:%2075=1015%:%
%:%2076=1016%:%
%:%2077=1016%:%
%:%2078=1017%:%
%:%2079=1017%:%
%:%2080=1018%:%
%:%2081=1018%:%
%:%2082=1019%:%
%:%2083=1019%:%
%:%2084=1019%:%
%:%2085=1020%:%
%:%2086=1020%:%
%:%2087=1021%:%
%:%2088=1021%:%
%:%2089=1021%:%
%:%2090=1022%:%
%:%2091=1022%:%
%:%2092=1023%:%
%:%2093=1023%:%
%:%2094=1024%:%
%:%2095=1024%:%
%:%2096=1024%:%
%:%2097=1025%:%
%:%2098=1025%:%
%:%2099=1026%:%
%:%2100=1026%:%
%:%2101=1026%:%
%:%2102=1027%:%
%:%2103=1027%:%
%:%2104=1028%:%
%:%2105=1028%:%
%:%2106=1028%:%
%:%2107=1029%:%
%:%2108=1029%:%
%:%2109=1030%:%
%:%2110=1030%:%
%:%2111=1030%:%
%:%2112=1031%:%
%:%2113=1031%:%
%:%2114=1032%:%
%:%2115=1032%:%
%:%2116=1033%:%
%:%2117=1033%:%
%:%2118=1034%:%
%:%2119=1034%:%
%:%2120=1034%:%
%:%2121=1035%:%
%:%2122=1035%:%
%:%2123=1036%:%
%:%2124=1036%:%
%:%2125=1036%:%
%:%2126=1037%:%
%:%2127=1037%:%
%:%2128=1038%:%
%:%2129=1038%:%
%:%2130=1038%:%
%:%2131=1039%:%
%:%2132=1039%:%
%:%2133=1040%:%
%:%2134=1040%:%
%:%2135=1040%:%
%:%2136=1041%:%
%:%2137=1041%:%
%:%2138=1042%:%
%:%2139=1042%:%
%:%2140=1043%:%
%:%2141=1043%:%
%:%2142=1044%:%
%:%2148=1044%:%
%:%2151=1045%:%
%:%2152=1046%:%
%:%2153=1046%:%
%:%2154=1047%:%
%:%2155=1048%:%
%:%2156=1049%:%
%:%2157=1050%:%
%:%2164=1051%:%
%:%2165=1051%:%
%:%2166=1052%:%
%:%2167=1052%:%
%:%2168=1053%:%
%:%2169=1053%:%
%:%2170=1054%:%
%:%2171=1054%:%
%:%2172=1055%:%
%:%2173=1055%:%
%:%2174=1055%:%
%:%2175=1056%:%
%:%2176=1056%:%
%:%2177=1057%:%
%:%2178=1057%:%
%:%2179=1057%:%
%:%2180=1058%:%
%:%2181=1058%:%
%:%2182=1059%:%
%:%2183=1059%:%
%:%2184=1060%:%
%:%2185=1060%:%
%:%2186=1060%:%
%:%2187=1061%:%
%:%2188=1061%:%
%:%2189=1062%:%
%:%2190=1062%:%
%:%2191=1062%:%
%:%2192=1063%:%
%:%2193=1063%:%
%:%2194=1064%:%
%:%2195=1064%:%
%:%2196=1065%:%
%:%2197=1065%:%
%:%2198=1066%:%
%:%2199=1066%:%
%:%2200=1066%:%
%:%2201=1067%:%
%:%2202=1067%:%
%:%2203=1068%:%
%:%2204=1068%:%
%:%2205=1068%:%
%:%2206=1069%:%
%:%2207=1069%:%
%:%2208=1070%:%
%:%2209=1070%:%
%:%2210=1071%:%
%:%2211=1071%:%
%:%2212=1071%:%
%:%2213=1072%:%
%:%2214=1072%:%
%:%2215=1073%:%
%:%2216=1073%:%
%:%2217=1073%:%
%:%2218=1074%:%
%:%2219=1074%:%
%:%2220=1075%:%
%:%2221=1075%:%
%:%2222=1075%:%
%:%2223=1076%:%
%:%2224=1076%:%
%:%2225=1077%:%
%:%2226=1077%:%
%:%2227=1077%:%
%:%2228=1078%:%
%:%2229=1078%:%
%:%2230=1079%:%
%:%2231=1079%:%
%:%2232=1080%:%
%:%2233=1080%:%
%:%2234=1081%:%
%:%2235=1081%:%
%:%2236=1081%:%
%:%2237=1082%:%
%:%2238=1082%:%
%:%2239=1083%:%
%:%2240=1083%:%
%:%2241=1083%:%
%:%2242=1084%:%
%:%2243=1084%:%
%:%2244=1085%:%
%:%2245=1085%:%
%:%2246=1085%:%
%:%2247=1086%:%
%:%2248=1086%:%
%:%2249=1087%:%
%:%2250=1087%:%
%:%2251=1087%:%
%:%2252=1088%:%
%:%2253=1088%:%
%:%2254=1089%:%
%:%2255=1089%:%
%:%2256=1090%:%
%:%2257=1090%:%
%:%2258=1091%:%
%:%2264=1091%:%
%:%2267=1092%:%
%:%2268=1093%:%
%:%2269=1093%:%
%:%2270=1094%:%
%:%2271=1095%:%
%:%2272=1096%:%
%:%2273=1097%:%
%:%2280=1098%:%
%:%2281=1098%:%
%:%2282=1099%:%
%:%2283=1099%:%
%:%2284=1100%:%
%:%2285=1100%:%
%:%2286=1101%:%
%:%2287=1101%:%
%:%2288=1102%:%
%:%2289=1102%:%
%:%2290=1102%:%
%:%2291=1103%:%
%:%2292=1103%:%
%:%2293=1104%:%
%:%2294=1104%:%
%:%2295=1104%:%
%:%2296=1105%:%
%:%2297=1105%:%
%:%2298=1106%:%
%:%2299=1106%:%
%:%2300=1107%:%
%:%2301=1107%:%
%:%2302=1107%:%
%:%2303=1108%:%
%:%2304=1108%:%
%:%2305=1109%:%
%:%2306=1109%:%
%:%2307=1109%:%
%:%2308=1110%:%
%:%2309=1110%:%
%:%2310=1111%:%
%:%2311=1111%:%
%:%2312=1112%:%
%:%2313=1112%:%
%:%2314=1113%:%
%:%2315=1113%:%
%:%2316=1113%:%
%:%2317=1114%:%
%:%2318=1114%:%
%:%2319=1115%:%
%:%2320=1115%:%
%:%2321=1115%:%
%:%2322=1116%:%
%:%2323=1116%:%
%:%2324=1117%:%
%:%2325=1117%:%
%:%2326=1118%:%
%:%2327=1118%:%
%:%2328=1118%:%
%:%2329=1119%:%
%:%2330=1119%:%
%:%2331=1120%:%
%:%2332=1120%:%
%:%2333=1120%:%
%:%2334=1121%:%
%:%2335=1121%:%
%:%2336=1122%:%
%:%2337=1122%:%
%:%2338=1122%:%
%:%2339=1123%:%
%:%2340=1123%:%
%:%2341=1124%:%
%:%2342=1124%:%
%:%2343=1124%:%
%:%2344=1125%:%
%:%2345=1125%:%
%:%2346=1126%:%
%:%2347=1126%:%
%:%2348=1127%:%
%:%2349=1127%:%
%:%2350=1128%:%
%:%2351=1128%:%
%:%2352=1128%:%
%:%2353=1129%:%
%:%2354=1129%:%
%:%2355=1130%:%
%:%2356=1130%:%
%:%2357=1130%:%
%:%2358=1131%:%
%:%2359=1131%:%
%:%2360=1132%:%
%:%2361=1132%:%
%:%2362=1132%:%
%:%2363=1133%:%
%:%2364=1133%:%
%:%2365=1134%:%
%:%2366=1134%:%
%:%2367=1134%:%
%:%2368=1135%:%
%:%2369=1135%:%
%:%2370=1136%:%
%:%2371=1136%:%
%:%2372=1137%:%
%:%2373=1137%:%
%:%2374=1138%:%
%:%2380=1138%:%
%:%2383=1139%:%
%:%2384=1140%:%
%:%2385=1140%:%
%:%2392=1141%:%
%:%2393=1141%:%
%:%2394=1142%:%
%:%2395=1142%:%
%:%2396=1143%:%
%:%2397=1143%:%
%:%2398=1143%:%
%:%2399=1143%:%
%:%2400=1144%:%
%:%2401=1144%:%
%:%2402=1145%:%
%:%2403=1145%:%
%:%2404=1146%:%
%:%2405=1146%:%
%:%2406=1146%:%
%:%2407=1146%:%
%:%2408=1147%:%
%:%2409=1147%:%
%:%2410=1148%:%
%:%2411=1148%:%
%:%2412=1149%:%
%:%2413=1149%:%
%:%2414=1149%:%
%:%2415=1149%:%
%:%2416=1150%:%
%:%2417=1150%:%
%:%2418=1151%:%
%:%2419=1151%:%
%:%2420=1152%:%
%:%2421=1152%:%
%:%2422=1152%:%
%:%2423=1152%:%
%:%2424=1153%:%
%:%2425=1153%:%
%:%2426=1154%:%
%:%2427=1154%:%
%:%2428=1155%:%
%:%2429=1155%:%
%:%2430=1155%:%
%:%2431=1155%:%
%:%2432=1156%:%
%:%2433=1156%:%
%:%2434=1157%:%
%:%2435=1157%:%
%:%2436=1158%:%
%:%2437=1158%:%
%:%2438=1158%:%
%:%2439=1158%:%
%:%2440=1159%:%
%:%2450=1161%:%
%:%2452=1163%:%
%:%2453=1163%:%
%:%2456=1164%:%
%:%2460=1164%:%
%:%2461=1164%:%
%:%2470=1166%:%
%:%2471=1167%:%
%:%2472=1168%:%
%:%2473=1169%:%
%:%2474=1170%:%
%:%2475=1171%:%
%:%2476=1172%:%
%:%2477=1173%:%
%:%2478=1174%:%
%:%2479=1175%:%
%:%2480=1176%:%
%:%2481=1177%:%
%:%2482=1178%:%
%:%2483=1179%:%
%:%2484=1180%:%
%:%2485=1181%:%
%:%2486=1182%:%
%:%2487=1183%:%
%:%2488=1184%:%
%:%2489=1185%:%
%:%2490=1186%:%
%:%2491=1187%:%
%:%2492=1188%:%
%:%2493=1189%:%
%:%2494=1190%:%
%:%2495=1191%:%
%:%2496=1192%:%
%:%2497=1193%:%
%:%2498=1194%:%
%:%2499=1195%:%
%:%2500=1196%:%
%:%2501=1197%:%
%:%2502=1198%:%
%:%2503=1199%:%
%:%2504=1200%:%
%:%2505=1201%:%
%:%2506=1202%:%
%:%2507=1203%:%
%:%2508=1204%:%
%:%2509=1205%:%
%:%2510=1206%:%
%:%2511=1207%:%
%:%2512=1208%:%
%:%2513=1209%:%
%:%2514=1210%:%
%:%2515=1211%:%
%:%2516=1212%:%
%:%2520=1214%:%
%:%2521=1215%:%
%:%2523=1217%:%
%:%2524=1217%:%
%:%2525=1218%:%
%:%2526=1219%:%
%:%2533=1220%:%
%:%2534=1220%:%
%:%2535=1221%:%
%:%2536=1221%:%
%:%2537=1221%:%
%:%2538=1222%:%
%:%2539=1222%:%
%:%2540=1223%:%
%:%2541=1223%:%
%:%2542=1223%:%
%:%2543=1224%:%
%:%2544=1224%:%
%:%2545=1225%:%
%:%2546=1225%:%
%:%2547=1225%:%
%:%2548=1226%:%
%:%2549=1226%:%
%:%2550=1227%:%
%:%2556=1227%:%
%:%2559=1228%:%
%:%2560=1229%:%
%:%2561=1229%:%
%:%2562=1230%:%
%:%2563=1231%:%
%:%2570=1232%:%
%:%2571=1232%:%
%:%2572=1233%:%
%:%2573=1233%:%
%:%2574=1234%:%
%:%2575=1234%:%
%:%2576=1235%:%
%:%2577=1235%:%
%:%2578=1236%:%
%:%2579=1236%:%
%:%2580=1236%:%
%:%2581=1237%:%
%:%2582=1237%:%
%:%2583=1238%:%
%:%2584=1238%:%
%:%2585=1238%:%
%:%2586=1239%:%
%:%2587=1239%:%
%:%2588=1240%:%
%:%2594=1240%:%
%:%2597=1241%:%
%:%2598=1242%:%
%:%2599=1242%:%
%:%2600=1243%:%
%:%2601=1244%:%
%:%2602=1245%:%
%:%2609=1246%:%
%:%2610=1246%:%
%:%2611=1247%:%
%:%2612=1247%:%
%:%2613=1248%:%
%:%2614=1248%:%
%:%2615=1249%:%
%:%2616=1249%:%
%:%2617=1250%:%
%:%2618=1250%:%
%:%2619=1250%:%
%:%2620=1251%:%
%:%2621=1251%:%
%:%2622=1252%:%
%:%2623=1252%:%
%:%2624=1252%:%
%:%2625=1253%:%
%:%2626=1253%:%
%:%2627=1254%:%
%:%2628=1254%:%
%:%2629=1255%:%
%:%2630=1255%:%
%:%2631=1255%:%
%:%2632=1256%:%
%:%2633=1256%:%
%:%2634=1257%:%
%:%2635=1257%:%
%:%2636=1257%:%
%:%2637=1258%:%
%:%2638=1258%:%
%:%2639=1259%:%
%:%2640=1259%:%
%:%2641=1260%:%
%:%2642=1260%:%
%:%2643=1261%:%
%:%2644=1261%:%
%:%2645=1261%:%
%:%2646=1262%:%
%:%2647=1262%:%
%:%2648=1263%:%
%:%2649=1263%:%
%:%2650=1263%:%
%:%2651=1264%:%
%:%2652=1264%:%
%:%2653=1265%:%
%:%2654=1265%:%
%:%2655=1265%:%
%:%2656=1266%:%
%:%2657=1266%:%
%:%2658=1266%:%
%:%2659=1267%:%
%:%2660=1267%:%
%:%2661=1268%:%
%:%2667=1268%:%
%:%2670=1269%:%
%:%2671=1270%:%
%:%2672=1270%:%
%:%2673=1271%:%
%:%2674=1272%:%
%:%2675=1273%:%
%:%2676=1274%:%
%:%2677=1275%:%
%:%2678=1276%:%
%:%2685=1277%:%
%:%2686=1277%:%
%:%2687=1278%:%
%:%2688=1278%:%
%:%2689=1279%:%
%:%2690=1279%:%
%:%2691=1280%:%
%:%2692=1280%:%
%:%2693=1281%:%
%:%2694=1281%:%
%:%2695=1281%:%
%:%2696=1282%:%
%:%2697=1282%:%
%:%2698=1283%:%
%:%2699=1283%:%
%:%2700=1283%:%
%:%2701=1284%:%
%:%2702=1284%:%
%:%2703=1285%:%
%:%2704=1285%:%
%:%2705=1286%:%
%:%2706=1286%:%
%:%2707=1286%:%
%:%2708=1287%:%
%:%2709=1287%:%
%:%2710=1288%:%
%:%2711=1288%:%
%:%2712=1288%:%
%:%2713=1289%:%
%:%2714=1289%:%
%:%2715=1290%:%
%:%2716=1290%:%
%:%2717=1291%:%
%:%2718=1291%:%
%:%2719=1292%:%
%:%2720=1292%:%
%:%2721=1292%:%
%:%2722=1293%:%
%:%2723=1293%:%
%:%2724=1294%:%
%:%2725=1294%:%
%:%2726=1294%:%
%:%2727=1295%:%
%:%2728=1295%:%
%:%2729=1296%:%
%:%2730=1296%:%
%:%2731=1297%:%
%:%2732=1297%:%
%:%2733=1297%:%
%:%2734=1298%:%
%:%2735=1298%:%
%:%2736=1299%:%
%:%2737=1299%:%
%:%2738=1299%:%
%:%2739=1300%:%
%:%2740=1300%:%
%:%2741=1301%:%
%:%2742=1301%:%
%:%2743=1301%:%
%:%2744=1302%:%
%:%2745=1302%:%
%:%2746=1303%:%
%:%2747=1303%:%
%:%2748=1303%:%
%:%2749=1304%:%
%:%2750=1304%:%
%:%2751=1305%:%
%:%2752=1305%:%
%:%2753=1306%:%
%:%2754=1306%:%
%:%2755=1307%:%
%:%2756=1307%:%
%:%2757=1307%:%
%:%2758=1308%:%
%:%2759=1308%:%
%:%2760=1309%:%
%:%2761=1309%:%
%:%2762=1309%:%
%:%2763=1310%:%
%:%2764=1310%:%
%:%2765=1311%:%
%:%2766=1311%:%
%:%2767=1311%:%
%:%2768=1312%:%
%:%2769=1312%:%
%:%2770=1313%:%
%:%2771=1313%:%
%:%2772=1313%:%
%:%2773=1314%:%
%:%2774=1314%:%
%:%2775=1315%:%
%:%2776=1315%:%
%:%2777=1316%:%
%:%2778=1316%:%
%:%2779=1317%:%
%:%2785=1317%:%
%:%2788=1318%:%
%:%2789=1319%:%
%:%2790=1319%:%
%:%2791=1320%:%
%:%2792=1321%:%
%:%2793=1322%:%
%:%2794=1323%:%
%:%2795=1324%:%
%:%2796=1325%:%
%:%2803=1326%:%
%:%2804=1326%:%
%:%2805=1327%:%
%:%2806=1327%:%
%:%2807=1328%:%
%:%2808=1328%:%
%:%2809=1329%:%
%:%2810=1329%:%
%:%2811=1330%:%
%:%2812=1330%:%
%:%2813=1330%:%
%:%2814=1331%:%
%:%2815=1331%:%
%:%2816=1332%:%
%:%2817=1332%:%
%:%2818=1332%:%
%:%2819=1333%:%
%:%2820=1333%:%
%:%2821=1334%:%
%:%2822=1334%:%
%:%2823=1335%:%
%:%2824=1335%:%
%:%2825=1335%:%
%:%2826=1336%:%
%:%2827=1336%:%
%:%2828=1337%:%
%:%2829=1337%:%
%:%2830=1337%:%
%:%2831=1338%:%
%:%2832=1338%:%
%:%2833=1339%:%
%:%2834=1339%:%
%:%2835=1340%:%
%:%2836=1340%:%
%:%2837=1341%:%
%:%2838=1341%:%
%:%2839=1341%:%
%:%2840=1342%:%
%:%2841=1342%:%
%:%2842=1343%:%
%:%2843=1343%:%
%:%2844=1343%:%
%:%2845=1344%:%
%:%2846=1344%:%
%:%2847=1345%:%
%:%2848=1345%:%
%:%2849=1346%:%
%:%2850=1346%:%
%:%2851=1346%:%
%:%2852=1347%:%
%:%2853=1347%:%
%:%2854=1348%:%
%:%2855=1348%:%
%:%2856=1348%:%
%:%2857=1349%:%
%:%2858=1349%:%
%:%2859=1350%:%
%:%2860=1350%:%
%:%2861=1350%:%
%:%2862=1351%:%
%:%2863=1351%:%
%:%2864=1352%:%
%:%2865=1352%:%
%:%2866=1352%:%
%:%2867=1353%:%
%:%2868=1353%:%
%:%2869=1354%:%
%:%2870=1354%:%
%:%2871=1355%:%
%:%2872=1355%:%
%:%2873=1356%:%
%:%2874=1356%:%
%:%2875=1356%:%
%:%2876=1357%:%
%:%2877=1357%:%
%:%2878=1358%:%
%:%2879=1358%:%
%:%2880=1358%:%
%:%2881=1359%:%
%:%2882=1359%:%
%:%2883=1360%:%
%:%2884=1360%:%
%:%2885=1360%:%
%:%2886=1361%:%
%:%2887=1361%:%
%:%2888=1362%:%
%:%2889=1362%:%
%:%2890=1362%:%
%:%2891=1363%:%
%:%2892=1363%:%
%:%2893=1364%:%
%:%2894=1364%:%
%:%2895=1365%:%
%:%2896=1365%:%
%:%2897=1366%:%
%:%2903=1366%:%
%:%2906=1367%:%
%:%2907=1368%:%
%:%2908=1368%:%
%:%2909=1369%:%
%:%2910=1370%:%
%:%2911=1371%:%
%:%2912=1372%:%
%:%2913=1373%:%
%:%2914=1374%:%
%:%2921=1375%:%
%:%2922=1375%:%
%:%2923=1376%:%
%:%2924=1376%:%
%:%2925=1377%:%
%:%2926=1377%:%
%:%2927=1378%:%
%:%2928=1378%:%
%:%2929=1379%:%
%:%2930=1379%:%
%:%2931=1379%:%
%:%2932=1380%:%
%:%2933=1380%:%
%:%2934=1381%:%
%:%2935=1381%:%
%:%2936=1381%:%
%:%2937=1382%:%
%:%2938=1382%:%
%:%2939=1383%:%
%:%2940=1383%:%
%:%2941=1384%:%
%:%2942=1384%:%
%:%2943=1384%:%
%:%2944=1385%:%
%:%2945=1385%:%
%:%2946=1386%:%
%:%2947=1386%:%
%:%2948=1386%:%
%:%2949=1387%:%
%:%2950=1387%:%
%:%2951=1388%:%
%:%2952=1388%:%
%:%2953=1389%:%
%:%2954=1389%:%
%:%2955=1390%:%
%:%2956=1390%:%
%:%2957=1390%:%
%:%2958=1391%:%
%:%2959=1391%:%
%:%2960=1392%:%
%:%2961=1392%:%
%:%2962=1392%:%
%:%2963=1393%:%
%:%2964=1393%:%
%:%2965=1394%:%
%:%2966=1394%:%
%:%2967=1395%:%
%:%2968=1395%:%
%:%2969=1395%:%
%:%2970=1396%:%
%:%2971=1396%:%
%:%2972=1397%:%
%:%2973=1397%:%
%:%2974=1397%:%
%:%2975=1398%:%
%:%2976=1398%:%
%:%2977=1399%:%
%:%2978=1399%:%
%:%2979=1399%:%
%:%2980=1400%:%
%:%2981=1400%:%
%:%2982=1401%:%
%:%2983=1401%:%
%:%2984=1401%:%
%:%2985=1402%:%
%:%2986=1402%:%
%:%2987=1403%:%
%:%2988=1403%:%
%:%2989=1404%:%
%:%2990=1404%:%
%:%2991=1405%:%
%:%2992=1405%:%
%:%2993=1405%:%
%:%2994=1406%:%
%:%2995=1406%:%
%:%2996=1407%:%
%:%2997=1407%:%
%:%2998=1407%:%
%:%2999=1408%:%
%:%3000=1408%:%
%:%3001=1409%:%
%:%3002=1409%:%
%:%3003=1409%:%
%:%3004=1410%:%
%:%3005=1410%:%
%:%3006=1411%:%
%:%3007=1411%:%
%:%3008=1411%:%
%:%3009=1412%:%
%:%3010=1412%:%
%:%3011=1413%:%
%:%3012=1413%:%
%:%3013=1414%:%
%:%3014=1414%:%
%:%3015=1415%:%
%:%3021=1415%:%
%:%3024=1416%:%
%:%3025=1417%:%
%:%3026=1417%:%
%:%3027=1418%:%
%:%3034=1419%:%
%:%3035=1419%:%
%:%3036=1420%:%
%:%3037=1420%:%
%:%3038=1421%:%
%:%3039=1421%:%
%:%3040=1421%:%
%:%3041=1421%:%
%:%3042=1422%:%
%:%3043=1422%:%
%:%3044=1423%:%
%:%3045=1423%:%
%:%3046=1424%:%
%:%3047=1424%:%
%:%3048=1424%:%
%:%3049=1424%:%
%:%3050=1425%:%
%:%3051=1425%:%
%:%3052=1426%:%
%:%3053=1426%:%
%:%3054=1427%:%
%:%3055=1427%:%
%:%3056=1427%:%
%:%3057=1427%:%
%:%3058=1428%:%
%:%3059=1428%:%
%:%3060=1429%:%
%:%3061=1429%:%
%:%3062=1430%:%
%:%3063=1430%:%
%:%3064=1430%:%
%:%3065=1430%:%
%:%3066=1431%:%
%:%3067=1431%:%
%:%3068=1432%:%
%:%3069=1432%:%
%:%3070=1433%:%
%:%3071=1433%:%
%:%3072=1433%:%
%:%3073=1433%:%
%:%3074=1434%:%
%:%3075=1434%:%
%:%3076=1435%:%
%:%3077=1435%:%
%:%3078=1436%:%
%:%3079=1436%:%
%:%3080=1436%:%
%:%3081=1436%:%
%:%3082=1437%:%
%:%3092=1439%:%
%:%3094=1441%:%
%:%3095=1441%:%
%:%3096=1442%:%
%:%3099=1443%:%
%:%3103=1443%:%
%:%3104=1443%:%
%:%3113=1445%:%
%:%3114=1446%:%
%:%3115=1447%:%
%:%3116=1448%:%
%:%3117=1449%:%
%:%3118=1450%:%
%:%3119=1451%:%
%:%3120=1452%:%
%:%3121=1453%:%
%:%3122=1454%:%
%:%3123=1455%:%
%:%3124=1456%:%
%:%3125=1457%:%
%:%3126=1458%:%
%:%3127=1459%:%
%:%3128=1460%:%
%:%3129=1461%:%
%:%3130=1462%:%
%:%3131=1463%:%
%:%3132=1464%:%
%:%3133=1465%:%
%:%3134=1466%:%
%:%3135=1467%:%
%:%3137=1469%:%
%:%3138=1469%:%
%:%3139=1470%:%
%:%3140=1471%:%
%:%3141=1472%:%
%:%3148=1473%:%
%:%3149=1473%:%
%:%3150=1474%:%
%:%3151=1474%:%
%:%3152=1474%:%
%:%3153=1475%:%
%:%3154=1475%:%
%:%3155=1476%:%
%:%3156=1476%:%
%:%3157=1476%:%
%:%3158=1477%:%
%:%3159=1477%:%
%:%3160=1478%:%
%:%3161=1478%:%
%:%3162=1478%:%
%:%3163=1478%:%
%:%3164=1479%:%
%:%3165=1479%:%
%:%3166=1480%:%
%:%3167=1480%:%
%:%3168=1481%:%
%:%3169=1481%:%
%:%3170=1482%:%
%:%3171=1482%:%
%:%3172=1482%:%
%:%3173=1483%:%
%:%3174=1483%:%
%:%3175=1484%:%
%:%3176=1484%:%
%:%3177=1485%:%
%:%3183=1485%:%
%:%3186=1486%:%
%:%3187=1487%:%
%:%3188=1487%:%
%:%3189=1488%:%
%:%3191=1490%:%
%:%3194=1491%:%
%:%3198=1491%:%
%:%3199=1491%:%
%:%3200=1491%:%
%:%3209=1493%:%
%:%3213=1495%:%
%:%3214=1496%:%
%:%3216=1498%:%
%:%3217=1498%:%
%:%3218=1499%:%
%:%3219=1500%:%
%:%3220=1501%:%
%:%3221=1502%:%
%:%3222=1503%:%
%:%3223=1504%:%
%:%3224=1505%:%
%:%3227=1506%:%
%:%3231=1506%:%
%:%3241=1508%:%
%:%3242=1509%:%
%:%3243=1510%:%
%:%3244=1511%:%
%:%3245=1512%:%
%:%3246=1513%:%
%:%3248=1515%:%
%:%3249=1515%:%
%:%3250=1516%:%
%:%3251=1517%:%
%:%3258=1518%:%
%:%3259=1518%:%
%:%3260=1519%:%
%:%3261=1519%:%
%:%3262=1520%:%
%:%3263=1520%:%
%:%3264=1520%:%
%:%3265=1520%:%
%:%3266=1521%:%
%:%3267=1521%:%
%:%3268=1521%:%
%:%3269=1522%:%
%:%3270=1522%:%
%:%3271=1522%:%
%:%3272=1522%:%
%:%3273=1523%:%
%:%3274=1523%:%
%:%3275=1523%:%
%:%3276=1524%:%
%:%3277=1524%:%
%:%3278=1525%:%
%:%3284=1525%:%
%:%3287=1526%:%
%:%3288=1527%:%
%:%3289=1527%:%
%:%3290=1528%:%
%:%3291=1529%:%
%:%3298=1530%:%
%:%3299=1530%:%
%:%3300=1531%:%
%:%3301=1531%:%
%:%3302=1532%:%
%:%3303=1533%:%
%:%3304=1533%:%
%:%3305=1534%:%
%:%3306=1534%:%
%:%3307=1534%:%
%:%3308=1535%:%
%:%3309=1535%:%
%:%3310=1535%:%
%:%3311=1535%:%
%:%3312=1536%:%
%:%3313=1536%:%
%:%3314=1537%:%
%:%3315=1537%:%
%:%3316=1538%:%
%:%3317=1538%:%
%:%3318=1538%:%
%:%3319=1538%:%
%:%3320=1539%:%
%:%3321=1539%:%
%:%3322=1539%:%
%:%3323=1539%:%
%:%3324=1540%:%
%:%3325=1540%:%
%:%3326=1540%:%
%:%3327=1540%:%
%:%3328=1541%:%
%:%3338=1543%:%
%:%3340=1545%:%
%:%3341=1545%:%
%:%3342=1546%:%
%:%3343=1547%:%
%:%3344=1548%:%
%:%3345=1549%:%
%:%3346=1550%:%
%:%3347=1551%:%
%:%3348=1552%:%
%:%3351=1553%:%
%:%3355=1553%:%
%:%3356=1553%:%
%:%3357=1553%:%
%:%3358=1554%:%
%:%3368=1556%:%
%:%3372=1558%:%
%:%3376=1560%:%
%:%3377=1561%:%
%:%3378=1562%:%
%:%3380=1564%:%
%:%3381=1564%:%
%:%3384=1565%:%
%:%3388=1565%:%
%:%3398=1567%:%
%:%3399=1568%:%
%:%3400=1569%:%
%:%3401=1570%:%
%:%3402=1571%:%
%:%3404=1573%:%
%:%3405=1573%:%
%:%3406=1574%:%
%:%3407=1575%:%
%:%3408=1576%:%
%:%3409=1576%:%
%:%3410=1577%:%
%:%3411=1578%:%
%:%3412=1579%:%
%:%3413=1579%:%
%:%3414=1580%:%
%:%3415=1581%:%
%:%3418=1581%:%
%:%3422=1581%:%
%:%3430=1581%:%
%:%3431=1582%:%
%:%3432=1583%:%
%:%3435=1585%:%
%:%3436=1586%:%
%:%3437=1587%:%
%:%3438=1588%:%
%:%3439=1589%:%
%:%3440=1590%:%
%:%3441=1591%:%
%:%3442=1592%:%
%:%3443=1593%:%
%:%3444=1594%:%
%:%3445=1595%:%
%:%3446=1596%:%
%:%3447=1597%:%
%:%3449=1599%:%
%:%3450=1599%:%
%:%3451=1600%:%
%:%3452=1600%:%
%:%3454=1603%:%
%:%3455=1604%:%
%:%3456=1605%:%
%:%3457=1606%:%
%:%3459=1608%:%
%:%3460=1608%:%
%:%3461=1609%:%
%:%3463=1611%:%
%:%3464=1612%:%
%:%3465=1613%:%
%:%3466=1614%:%
%:%3467=1615%:%
%:%3468=1616%:%
%:%3469=1617%:%
%:%3470=1618%:%
%:%3472=1620%:%
%:%3473=1620%:%
%:%3476=1621%:%
%:%3480=1621%:%
%:%3481=1621%:%
%:%3490=1623%:%
%:%3491=1624%:%
%:%3500=1626%:%
%:%3512=1628%:%
%:%3513=1629%:%
%:%3514=1630%:%
%:%3518=1632%:%
%:%3519=1633%:%
%:%3520=1634%:%
%:%3521=1635%:%
%:%3522=1636%:%
%:%3523=1637%:%
%:%3525=1639%:%
%:%3526=1639%:%
%:%3527=1640%:%
%:%3529=1642%:%
%:%3533=1644%:%
%:%3534=1645%:%
%:%3535=1646%:%
%:%3537=1648%:%
%:%3538=1648%:%
%:%3541=1649%:%
%:%3545=1649%:%
%:%3546=1649%:%
%:%3547=1649%:%
%:%3556=1651%:%
%:%3557=1652%:%
%:%3559=1654%:%
%:%3560=1654%:%
%:%3561=1655%:%
%:%3562=1656%:%
%:%3563=1657%:%
%:%3564=1658%:%
%:%3565=1658%:%
%:%3566=1659%:%
%:%3567=1660%:%
%:%3569=1662%:%
%:%3570=1663%:%
%:%3571=1664%:%
%:%3572=1665%:%
%:%3573=1666%:%
%:%3574=1667%:%
%:%3575=1668%:%
%:%3576=1669%:%
%:%3577=1670%:%
%:%3578=1671%:%
%:%3579=1672%:%
%:%3580=1673%:%
%:%3581=1674%:%
%:%3582=1675%:%
%:%3583=1676%:%
%:%3584=1677%:%
%:%3585=1678%:%
%:%3586=1679%:%
%:%3588=1681%:%
%:%3589=1681%:%
%:%3592=1682%:%
%:%3596=1682%:%
%:%3597=1682%:%
%:%3598=1683%:%