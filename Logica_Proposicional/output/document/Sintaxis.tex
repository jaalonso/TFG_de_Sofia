%
\begin{isabellebody}%
\setisabellecontext{Sintaxis}%
%
\isadelimtheory
%
\endisadelimtheory
%
\isatagtheory
%
\endisatagtheory
{\isafoldtheory}%
%
\isadelimtheory
%
\endisadelimtheory
%
\isadelimdocument
%
\endisadelimdocument
%
\isatagdocument
%
\isamarkupsection{Sintaxis%
}
\isamarkuptrue%
%
\endisatagdocument
{\isafolddocument}%
%
\isadelimdocument
%
\endisadelimdocument
\isacommand{notation}\isamarkupfalse%
\ insert\ {\isacharparenleft}{\isachardoublequoteopen}{\isacharunderscore}\ {\isasymtriangleright}\ {\isacharunderscore}{\isachardoublequoteclose}\ {\isacharbrackleft}{\isadigit{5}}{\isadigit{6}}{\isacharcomma}{\isadigit{5}}{\isadigit{5}}{\isacharbrackright}\ {\isadigit{5}}{\isadigit{5}}{\isacharparenright}%
\begin{isamarkuptext}%
En este apartado vamos a desarrollar los elementos de la sintaxis junto con varios resultados 
sobre los mismos. La lógica proposicional cuenta, fundamentalmente con:
\begin{description}
    \item[Alfabeto:] Se trata de una lista infinita de variables proposicionales, consideradas como 
expresiones cuya estructura interna no vamos a analizar.
    \item[Conectivas:] Conectores que interactúan con los elementos del alfabeto. Pueden ser monarias que afectan a 
un único elemento o binarias que afectan a dos. En el primer grupo se encuentra le negación (\isa{{\isasymnot}}) y 
en el segundo vamos a considerar la conjunción (\isa{{\isasymand}}), la disyunción (\isa{{\isasymor}}) y la implicación (\isa{{\isasymlongrightarrow}}).
\end{description}

De este modo, informalmente diremos que una fórmula es el resultado de relacionar los elementos del 
alfabeto mediante las conectivas definidas anteriormente. En primer lugar, podemos definir las 
fórmulas atómicas como el tipo de fórmulas más básico, formadas únicamente por una variable 
porposicional del alfabeto. Por abuso de notación llamaremos \isa{átomos} a las variables proposicionales
del alfabeto. Aunque son intuitivamente equivalentes a las fórmulas atómicas, debemos recalcar 
su diferencia, pues los átomos son los elementos del alfabeto y las fórmulas atómicas son
construcciones básicas de ellos. Este apunte es fundamental para entender el tipo correspondiente
de fórmulas en nuestro programa.%
\end{isamarkuptext}\isamarkuptrue%
%
\begin{isamarkuptext}%
En Isabelle, las fórmulas son entendidas como un árbol con distintos tipos de nodos, que
son las conectivas, y hojas, que serán las fórmulas atómicas. De este modo, se define el tipo de 
las fórmulas como sigue.%
\end{isamarkuptext}\isamarkuptrue%
\isacommand{datatype}\isamarkupfalse%
\ {\isacharparenleft}atoms{\isacharcolon}\ {\isacharprime}a{\isacharparenright}\ formula\ {\isacharequal}\ \isanewline
\ \ \ \ Atom\ {\isacharprime}a\isanewline
\ \ {\isacharbar}\ Bot\ \ \ \ \ \ \ \ \ \ \ \ \ \ \ \ \ \ \ \ \ \ \ \ \ \ \ \ \ \ {\isacharparenleft}{\isachardoublequoteopen}{\isasymbottom}{\isachardoublequoteclose}{\isacharparenright}\ \ \isanewline
\ \ {\isacharbar}\ Not\ {\isachardoublequoteopen}{\isacharprime}a\ formula{\isachardoublequoteclose}\ \ \ \ \ \ \ \ \ \ \ \ \ \ \ \ \ {\isacharparenleft}{\isachardoublequoteopen}\isactrlbold {\isasymnot}{\isachardoublequoteclose}{\isacharparenright}\isanewline
\ \ {\isacharbar}\ And\ {\isachardoublequoteopen}{\isacharprime}a\ formula{\isachardoublequoteclose}\ {\isachardoublequoteopen}{\isacharprime}a\ formula{\isachardoublequoteclose}\ \ \ \ {\isacharparenleft}\isakeyword{infix}\ {\isachardoublequoteopen}\isactrlbold {\isasymand}{\isachardoublequoteclose}\ {\isadigit{6}}{\isadigit{8}}{\isacharparenright}\isanewline
\ \ {\isacharbar}\ Or\ {\isachardoublequoteopen}{\isacharprime}a\ formula{\isachardoublequoteclose}\ {\isachardoublequoteopen}{\isacharprime}a\ formula{\isachardoublequoteclose}\ \ \ \ \ {\isacharparenleft}\isakeyword{infix}\ {\isachardoublequoteopen}\isactrlbold {\isasymor}{\isachardoublequoteclose}\ {\isadigit{6}}{\isadigit{8}}{\isacharparenright}\isanewline
\ \ {\isacharbar}\ Imp\ {\isachardoublequoteopen}{\isacharprime}a\ formula{\isachardoublequoteclose}\ {\isachardoublequoteopen}{\isacharprime}a\ formula{\isachardoublequoteclose}\ \ \ \ {\isacharparenleft}\isakeyword{infixr}\ {\isachardoublequoteopen}\isactrlbold {\isasymrightarrow}{\isachardoublequoteclose}\ {\isadigit{6}}{\isadigit{8}}{\isacharparenright}%
\begin{isamarkuptext}%
Como podemos observar en la definición, \isa{formula} es un tipo de datos recursivo que se 
entiende como un árbol que relaciona elementos de un tipo \isa{{\isacharprime}a} cualquiera del alfabeto proposicional. 
En ella aparecen los siguientes elementos:
\begin{description}
\item[Constructores]:  
  \begin{enumerate} 
  \item \isa{Atom\ {\isacharcolon}{\isacharcolon}\ {\isacharprime}a\ formula}  
  \item \isa{Not\ {\isacharcolon}{\isacharcolon}\ {\isacharprime}a\ {\isasymRightarrow}\ {\isacharprime}a\ formula\ {\isasymRightarrow}\ {\isacharprime}a\ formula}
  \item \isa{And\ {\isacharcolon}{\isacharcolon}\ {\isacharprime}a\ {\isasymRightarrow}\ {\isacharprime}a\ formula\ {\isasymRightarrow}\ {\isacharprime}a\ formula\ {\isasymRightarrow}\ {\isacharprime}a\ formula}
  \item \isa{Or\ {\isacharcolon}{\isacharcolon}\ {\isacharprime}a\ {\isasymRightarrow}\ {\isacharprime}a\ formula\ {\isasymRightarrow}\ {\isacharprime}a\ formula\ {\isasymRightarrow}\ {\isacharprime}a\ formula}
  \item \isa{Imp\ {\isacharcolon}{\isacharcolon}\ {\isacharprime}a\ {\isasymRightarrow}\ {\isacharprime}a\ formula\ {\isasymRightarrow}\ {\isacharprime}a\ formula\ {\isasymRightarrow}\ {\isacharprime}a\ formula}
  \item \isa{Bot\ {\isacharcolon}{\isacharcolon}\ {\isacharprime}a\ formula}
  \end{enumerate}
\item[Función de conjunto]: \isa{atoms\ {\isacharcolon}{\isacharcolon}\ ′a\ formula\ {\isasymRightarrow}\ ′a\ set}
\end{description}

Podemos observar que se define simultáneamente la función \isa{atoms} de modo que al aplicarla 
sobre una fórmula nos devuelve el conjunto de los átomos que la componen.
En particular, \isa{Atom\ {\isacharprime}a} es la construcción de las fórmulas atómicas. \isa{Bot} es un constructor que 
para cada tipo \isa{{\isacharprime}a} cualquiera construye la fórmula falsa cuyo símbolo queda retratado en la definición.
Observemos que para emplear \isa{Bot} en solitario es necesario especificar el tipo \isa{{\isacharprime}a}.%
\end{isamarkuptext}\isamarkuptrue%
\isacommand{value}\isamarkupfalse%
{\isachardoublequoteopen}{\isacharparenleft}Bot{\isacharcolon}{\isacharcolon}nat\ formula{\isacharparenright}{\isachardoublequoteclose}%
\begin{isamarkuptext}%
Veamos ejemplos de obtención del conjunto de las variables proposicionales de las fórmulas atómicas y 
de negación.%
\end{isamarkuptext}\isamarkuptrue%
\isacommand{value}\isamarkupfalse%
\ {\isachardoublequoteopen}atoms\ {\isacharparenleft}Atom\ p{\isacharparenright}\ {\isacharequal}\ {\isacharbraceleft}p{\isacharbraceright}{\isachardoublequoteclose}\isanewline
\isanewline
\isacommand{value}\isamarkupfalse%
\ {\isachardoublequoteopen}atoms\ {\isacharparenleft}Not\ {\isacharparenleft}Atom\ p{\isacharparenright}{\isacharparenright}\ {\isacharequal}\ {\isacharbraceleft}p{\isacharbraceright}{\isachardoublequoteclose}%
\begin{isamarkuptext}%
En particular, al aplicar \isa{atoms} sobre la construcción \isa{Bot} nos devuelve el conjunto
vacío, pues como hemos señalado, no contiene ninguna variable del alfabeto.%
\end{isamarkuptext}\isamarkuptrue%
\isacommand{lemma}\isamarkupfalse%
\ {\isachardoublequoteopen}atoms\ Bot\ {\isacharequal}\ {\isacharbraceleft}{\isacharbraceright}{\isachardoublequoteclose}\isanewline
%
\isadelimproof
\ \ %
\endisadelimproof
%
\isatagproof
\isacommand{by}\isamarkupfalse%
\ auto%
\endisatagproof
{\isafoldproof}%
%
\isadelimproof
\isanewline
%
\endisadelimproof
\isanewline
\isacommand{lemma}\isamarkupfalse%
\ {\isachardoublequoteopen}atoms\ {\isacharparenleft}Or\ {\isacharparenleft}Atom\ p{\isacharparenright}\ Bot{\isacharparenright}\ {\isacharequal}\ {\isacharbraceleft}p{\isacharbraceright}{\isachardoublequoteclose}\isanewline
%
\isadelimproof
\ \ %
\endisadelimproof
%
\isatagproof
\isacommand{by}\isamarkupfalse%
\ auto%
\endisatagproof
{\isafoldproof}%
%
\isadelimproof
%
\endisadelimproof
%
\begin{isamarkuptext}%
El resto de elementos que aparecen son equivalentes a las conectivas binarias y la negación. 
Cabe señalar que el término \isa{infix} que precede al símbolo de notación de los nodos nos señala que 
son infijos, y \isa{infixr} se trata de un infijo asociado a la derecha.
A continuación vamos a incluir el ejemplo de fórmula: \isa{{\isacharparenleft}p\ {\isasymlongrightarrow}\ q{\isacharparenright}\ {\isasymor}\ r} y su árbol de formación 
correspondiente.%
\end{isamarkuptext}\isamarkuptrue%
\isacommand{value}\isamarkupfalse%
\ {\isachardoublequoteopen}Or\ {\isacharparenleft}Imp\ {\isacharparenleft}Atom\ p{\isacharparenright}\ {\isacharparenleft}Atom\ q{\isacharparenright}{\isacharparenright}\ {\isacharparenleft}Atom\ r{\isacharparenright}{\isachardoublequoteclose}%
\begin{isamarkuptext}%
(Aquí debería salir el árbol pero no sé hacerlo)%
\end{isamarkuptext}\isamarkuptrue%
%
\begin{isamarkuptext}%
Por otro lado, veamos cómo actúa la función \isa{atoms} sobre fórmulas construidas con 
conectivas binarias, señalando los casos en que interactúan variables distintas y repetidas. 
Como se observa, por definición de conjunto, no contiene elementos repetidos.%
\end{isamarkuptext}\isamarkuptrue%
\isacommand{lemma}\isamarkupfalse%
\ {\isachardoublequoteopen}atoms\ {\isacharparenleft}Or\ {\isacharparenleft}Imp\ {\isacharparenleft}Atom\ p{\isacharparenright}\ {\isacharparenleft}Atom\ q{\isacharparenright}{\isacharparenright}\ {\isacharparenleft}Atom\ r{\isacharparenright}{\isacharparenright}\ {\isacharequal}\ {\isacharbraceleft}p{\isacharcomma}q{\isacharcomma}r{\isacharbraceright}{\isachardoublequoteclose}\isanewline
%
\isadelimproof
\ \ %
\endisadelimproof
%
\isatagproof
\isacommand{by}\isamarkupfalse%
\ auto%
\endisatagproof
{\isafoldproof}%
%
\isadelimproof
\isanewline
%
\endisadelimproof
\isanewline
\isacommand{lemma}\isamarkupfalse%
\ {\isachardoublequoteopen}atoms\ {\isacharparenleft}Or\ {\isacharparenleft}Imp\ {\isacharparenleft}Atom\ r{\isacharparenright}\ {\isacharparenleft}Atom\ p{\isacharparenright}{\isacharparenright}\ {\isacharparenleft}Atom\ r{\isacharparenright}{\isacharparenright}\ {\isacharequal}\ {\isacharbraceleft}p{\isacharcomma}r{\isacharbraceright}{\isachardoublequoteclose}\isanewline
%
\isadelimproof
\ \ %
\endisadelimproof
%
\isatagproof
\isacommand{by}\isamarkupfalse%
\ auto%
\endisatagproof
{\isafoldproof}%
%
\isadelimproof
%
\endisadelimproof
%
\begin{isamarkuptext}%
En esta sección, para demostrar los distintos resultados utilizaremos la táctica \isa{induction}, 
que hace uso del esquema de indución. Para el tipo de las fórmulas, el esquema inductivo
se aplicará en cada uno de los casos de los constructores, desglosándose
así seis casos correspondientes a las distintas conectivas, fórmula atómica y \isa{Bot}. Además, todas 
las demostraciones sobre casos de conectivas binarias son equivalentes en esta sección,
pues la construcción sintáctica de fórmulas es idéntica entre ellas. Estas se
diferencian esencialmente por la semántica, que veremos en la siguiente sección. Por tanto, para
simplificar las demostraciones sintácticas detalladas más extensas, daré un nuevo tipo equivalente: 
\isa{formula{\isacharunderscore}simp}.%
\end{isamarkuptext}\isamarkuptrue%
\isacommand{datatype}\isamarkupfalse%
\ {\isacharparenleft}atoms{\isacharunderscore}s{\isacharcolon}\ {\isacharprime}a{\isacharparenright}\ formula{\isacharunderscore}simp\ {\isacharequal}\ \isanewline
\ \ \ \ Atom{\isacharunderscore}s\ {\isacharprime}a\isanewline
\ \ {\isacharbar}\ Bot{\isacharunderscore}s\ \ \ \ \ \ \ \ \ \ \ \ \ \ \ \ \ \ \ \ \ \ \ \ \ \ \ \ \ \ {\isacharparenleft}{\isachardoublequoteopen}Falso{\isachardoublequoteclose}{\isacharparenright}\ \ \isanewline
\ \ {\isacharbar}\ Mon\ {\isachardoublequoteopen}{\isacharprime}a\ formula{\isacharunderscore}simp{\isachardoublequoteclose}\ \ \ \ \ \ \ \ \ \ \ \ \ \ \ \ \ {\isacharparenleft}{\isachardoublequoteopen}Neg{\isachardoublequoteclose}{\isacharparenright}\isanewline
\ \ {\isacharbar}\ Bi\ {\isachardoublequoteopen}{\isacharprime}a\ formula{\isacharunderscore}simp{\isachardoublequoteclose}\ {\isachardoublequoteopen}{\isacharprime}a\ formula{\isacharunderscore}simp{\isachardoublequoteclose}\ \ \ \ {\isacharparenleft}\isakeyword{infix}\ {\isachardoublequoteopen}\isactrlbold {\isacharasterisk}{\isachardoublequoteclose}\ {\isadigit{6}}{\isadigit{8}}{\isacharparenright}%
\begin{isamarkuptext}%
De este modo, se consideran todas las conectivas binarias dentro de un mismo caso de constructor 
\isa{Bi} con notación \isa{{\isacharasterisk}},
y la conectiva mónica como \isa{Mon}, de notación \isa{Neg}. Para que no haya confusión, he renombrado
la notación del equivalente de \isa{Bot} como \isa{Falso}.
De este modo, en la inducción sobre esta nueva definición se deglosarán únicamente cuatro casos.

Análogamente,
a lo largo de la sección definiré si es necesario la versión simplificada de otros tipos que se incluyan.
La demostración automática aparecerá enunciada y demostrada con la definición original de \isa{formula}.%
\end{isamarkuptext}\isamarkuptrue%
%
\begin{isamarkuptext}%
Llegamos así al primer resultado de este apartado:
 \begin{lema}
    Los átomos de cualquier fórmula conforman un conjunto finito.
  \end{lema}%
\end{isamarkuptext}\isamarkuptrue%
%
\begin{isamarkuptext}%
En Isabelle, se traduce del siguiente modo.%
\end{isamarkuptext}\isamarkuptrue%
\isacommand{lemma}\isamarkupfalse%
\ atoms{\isacharunderscore}finite{\isacharbrackleft}simp{\isacharcomma}intro{\isacharbang}{\isacharbrackright}{\isacharcolon}\ {\isachardoublequoteopen}finite\ {\isacharparenleft}atoms\ F{\isacharparenright}{\isachardoublequoteclose}\isanewline
%
\isadelimproof
\ \ %
\endisadelimproof
%
\isatagproof
\isacommand{oops}\isamarkupfalse%
%
\endisatagproof
{\isafoldproof}%
%
\isadelimproof
%
\endisadelimproof
%
\begin{isamarkuptext}%
El lema anterior contiene la notación \isa{simp{\isacharcomma}intro{\isacharbang}} a continuación del título para incluir este
resultado en las tácticas automática (mediante \isa{intro{\isacharbang}}) y de simplificación (mediante \isa{simp}). 
Esto ocurrirá en varios resultados de esta sección.%
\end{isamarkuptext}\isamarkuptrue%
%
\begin{isamarkuptext}%
Por otro lado, el enunciado contiene la defición \isa{finite\ S}, perteneciente a la teoría 
\href{https://n9.cl/x86r}{FiniteSet.thy}. Se trata de una definición
inductiva que nos permite la demostración del lema por simplificacion ya que dentro de ella,
las reglas que especifica se han añadido como tácticas de \isa{simp} e \isa{intro{\isacharbang}}.
\\
\isa{inductive\ finite\ {\isacharcolon}{\isacharcolon}\ {\isachardoublequote}{\isacharprime}a\ set\ {\isasymRightarrow}\ bool{\isachardoublequote}}\\
\isa{where}\\
\isa{emptyI\ {\isacharbrackleft}simp{\isacharcomma}\ intro{\isacharbang}{\isacharbrackright}{\isacharcolon}\ {\isachardoublequote}finite\ {\isacharbraceleft}{\isacharbraceright}{\isachardoublequote}}\\
\isa{{\isacharbar}\ insertI\ {\isacharbrackleft}simp{\isacharcomma}\ intro{\isacharbang}{\isacharbrackright}{\isacharcolon}\ {\isachardoublequote}finite\ A\ {\isasymLongrightarrow}\ finite\ {\isacharparenleft}insert\ a\ A{\isacharparenright}{\isachardoublequote}}\\%
\end{isamarkuptext}\isamarkuptrue%
%
\begin{isamarkuptext}%
Veamos la prueba detallada del resultado que, aunque se resulve 
fácilmente por simplificación, nos muestra un ejemplo claro de la estructura inductiva
que nos acompañará en las siguientes demostraciones de este apartado.%
\end{isamarkuptext}\isamarkuptrue%
\isacommand{lemma}\isamarkupfalse%
\ atoms{\isacharunderscore}finite{\isacharunderscore}detallada{\isacharcolon}\ {\isachardoublequoteopen}finite\ {\isacharparenleft}atoms\ F{\isacharparenright}{\isachardoublequoteclose}\isanewline
%
\isadelimproof
%
\endisadelimproof
%
\isatagproof
\isacommand{proof}\isamarkupfalse%
\ {\isacharparenleft}induction\ F{\isacharparenright}\isanewline
\isacommand{case}\isamarkupfalse%
\ {\isacharparenleft}Atom\ x{\isacharparenright}\isanewline
\isacommand{then}\isamarkupfalse%
\ \isacommand{show}\isamarkupfalse%
\ {\isacharquery}case\ \isacommand{by}\isamarkupfalse%
\ simp\isanewline
\isacommand{next}\isamarkupfalse%
\isanewline
\isacommand{case}\isamarkupfalse%
\ Bot\isanewline
\ \ \isacommand{then}\isamarkupfalse%
\ \isacommand{show}\isamarkupfalse%
\ {\isacharquery}case\ \isacommand{by}\isamarkupfalse%
\ simp\isanewline
\isacommand{next}\isamarkupfalse%
\isanewline
\ \ \isacommand{case}\isamarkupfalse%
\ {\isacharparenleft}Not\ F{\isacharparenright}\isanewline
\ \ \isacommand{then}\isamarkupfalse%
\ \isacommand{show}\isamarkupfalse%
\ {\isacharquery}case\ \isacommand{by}\isamarkupfalse%
\ simp\isanewline
\isacommand{next}\isamarkupfalse%
\isanewline
\ \ \isacommand{case}\isamarkupfalse%
\ {\isacharparenleft}And\ F{\isadigit{1}}\ F{\isadigit{2}}{\isacharparenright}\isanewline
\ \ \isacommand{then}\isamarkupfalse%
\ \isacommand{show}\isamarkupfalse%
\ {\isacharquery}case\ \isacommand{by}\isamarkupfalse%
\ simp\isanewline
\isacommand{next}\isamarkupfalse%
\isanewline
\ \ \isacommand{case}\isamarkupfalse%
\ {\isacharparenleft}Or\ F{\isadigit{1}}\ F{\isadigit{2}}{\isacharparenright}\isanewline
\ \ \isacommand{then}\isamarkupfalse%
\ \isacommand{show}\isamarkupfalse%
\ {\isacharquery}case\ \isacommand{by}\isamarkupfalse%
\ simp\isanewline
\isacommand{next}\isamarkupfalse%
\isanewline
\ \ \isacommand{case}\isamarkupfalse%
\ {\isacharparenleft}Imp\ F{\isadigit{1}}\ F{\isadigit{2}}{\isacharparenright}\isanewline
\ \ \isacommand{then}\isamarkupfalse%
\ \isacommand{show}\isamarkupfalse%
\ {\isacharquery}case\ \isacommand{by}\isamarkupfalse%
\ simp\isanewline
\isacommand{qed}\isamarkupfalse%
%
\endisatagproof
{\isafoldproof}%
%
\isadelimproof
%
\endisadelimproof
%
\begin{isamarkuptext}%
Las demostraciones aplicativa y automática son las siguientes respectivamente.%
\end{isamarkuptext}\isamarkuptrue%
\isacommand{lemma}\isamarkupfalse%
\ atoms{\isacharunderscore}finite{\isacharunderscore}aplicativa{\isacharcolon}\ {\isachardoublequoteopen}finite\ {\isacharparenleft}atoms\ F{\isacharparenright}{\isachardoublequoteclose}\ \isanewline
%
\isadelimproof
\ \ %
\endisadelimproof
%
\isatagproof
\isacommand{apply}\isamarkupfalse%
\ {\isacharparenleft}induction\ F{\isacharparenright}\isanewline
\ \ \isacommand{apply}\isamarkupfalse%
\ simp{\isacharunderscore}all\isanewline
\ \isacommand{done}\isamarkupfalse%
%
\endisatagproof
{\isafoldproof}%
%
\isadelimproof
\isanewline
%
\endisadelimproof
\isanewline
\isacommand{lemma}\isamarkupfalse%
\ atoms{\isacharunderscore}finite{\isacharbrackleft}simp{\isacharcomma}intro{\isacharbang}{\isacharbrackright}{\isacharcolon}\ {\isachardoublequoteopen}finite\ {\isacharparenleft}atoms\ F{\isacharparenright}{\isachardoublequoteclose}\ \isanewline
%
\isadelimproof
\ \ %
\endisadelimproof
%
\isatagproof
\isacommand{by}\isamarkupfalse%
\ {\isacharparenleft}induction\ F{\isacharsemicolon}\ simp{\isacharparenright}%
\endisatagproof
{\isafoldproof}%
%
\isadelimproof
%
\endisadelimproof
%
\isadelimdocument
%
\endisadelimdocument
%
\isatagdocument
%
\isamarkupsubsection{Subfórmulas%
}
\isamarkuptrue%
%
\endisatagdocument
{\isafolddocument}%
%
\isadelimdocument
%
\endisadelimdocument
%
\begin{isamarkuptext}%
Otra construcción natural a partir de la definición de fórmulas son las subfórmulas.

  \begin{definicion}
La lista de las subfórmulas de una fórmula F (\isa{Subf{\isacharparenleft}F{\isacharparenright}}) se define recursivamente por:
    \begin{enumerate}
      \item \isa{{\isacharbrackleft}F{\isacharbrackright}} si \isa{F} es atómica.
      \item \isa{{\isacharbrackleft}Bot{\isacharbrackright}} si \isa{F} es \isa{Bot}.
      \item \isa{F{\isacharhash}Subf{\isacharparenleft}G{\isacharparenright}} si \isa{F} es \isa{{\isasymnot}G}.
      \item \isa{F{\isacharhash}Subf{\isacharparenleft}G{\isacharparenright}{\isacharat}Subf{\isacharparenleft}H{\isacharparenright}} si \isa{F} es \isa{G{\isacharasterisk}H} donde \isa{{\isacharasterisk}} es cualquier conectiva binaria.
    \end{enumerate}
  \end{definicion}%
\end{isamarkuptext}\isamarkuptrue%
%
\begin{isamarkuptext}%
En la definición anterior, \isa{{\isacharhash}} es el operador que añade un elemento al comienzo de una lista
y \isa{{\isacharat}} concatena varias listas.
Análogamente, vamos a definir la función primitiva recursiva \isa{subformulae}, que nos dará
una lista de todas las subfórmulas de una fórmula original obtenidas recursivamente.%
\end{isamarkuptext}\isamarkuptrue%
\isacommand{primrec}\isamarkupfalse%
\ subformulae\ {\isacharcolon}{\isacharcolon}\ {\isachardoublequoteopen}{\isacharprime}a\ formula\ {\isasymRightarrow}\ {\isacharprime}a\ formula\ list{\isachardoublequoteclose}\ \isakeyword{where}\isanewline
{\isachardoublequoteopen}subformulae\ {\isasymbottom}\ {\isacharequal}\ {\isacharbrackleft}{\isasymbottom}{\isacharbrackright}{\isachardoublequoteclose}\ {\isacharbar}\isanewline
{\isachardoublequoteopen}subformulae\ {\isacharparenleft}Atom\ k{\isacharparenright}\ {\isacharequal}\ {\isacharbrackleft}Atom\ k{\isacharbrackright}{\isachardoublequoteclose}\ {\isacharbar}\isanewline
{\isachardoublequoteopen}subformulae\ {\isacharparenleft}Not\ F{\isacharparenright}\ {\isacharequal}\ Not\ F\ {\isacharhash}\ subformulae\ F{\isachardoublequoteclose}\ {\isacharbar}\isanewline
{\isachardoublequoteopen}subformulae\ {\isacharparenleft}And\ F\ G{\isacharparenright}\ {\isacharequal}\ And\ F\ G\ {\isacharhash}\ subformulae\ F\ {\isacharat}\ subformulae\ G{\isachardoublequoteclose}\ {\isacharbar}\isanewline
{\isachardoublequoteopen}subformulae\ {\isacharparenleft}Imp\ F\ G{\isacharparenright}\ {\isacharequal}\ Imp\ F\ G\ {\isacharhash}\ subformulae\ F\ {\isacharat}\ subformulae\ G{\isachardoublequoteclose}\ {\isacharbar}\isanewline
{\isachardoublequoteopen}subformulae\ {\isacharparenleft}Or\ F\ G{\isacharparenright}\ {\isacharequal}\ Or\ F\ G\ {\isacharhash}\ subformulae\ F\ {\isacharat}\ subformulae\ G{\isachardoublequoteclose}%
\begin{isamarkuptext}%
Su definición simplificada equivalente es la siguiente.%
\end{isamarkuptext}\isamarkuptrue%
\isacommand{primrec}\isamarkupfalse%
\ subformulae{\isacharunderscore}s\ {\isacharcolon}{\isacharcolon}\ {\isachardoublequoteopen}{\isacharprime}a\ formula{\isacharunderscore}simp\ {\isasymRightarrow}\ {\isacharprime}a\ formula{\isacharunderscore}simp\ list{\isachardoublequoteclose}\ \isakeyword{where}\isanewline
{\isachardoublequoteopen}subformulae{\isacharunderscore}s\ {\isacharparenleft}Bot{\isacharunderscore}s{\isacharparenright}\ {\isacharequal}\ {\isacharbrackleft}Bot{\isacharunderscore}s{\isacharbrackright}{\isachardoublequoteclose}\ {\isacharbar}\isanewline
{\isachardoublequoteopen}subformulae{\isacharunderscore}s\ {\isacharparenleft}Atom{\isacharunderscore}s\ k{\isacharparenright}\ {\isacharequal}\ {\isacharbrackleft}Atom{\isacharunderscore}s\ k{\isacharbrackright}{\isachardoublequoteclose}\ {\isacharbar}\isanewline
{\isachardoublequoteopen}subformulae{\isacharunderscore}s\ {\isacharparenleft}Mon\ F{\isacharparenright}\ {\isacharequal}\ Mon\ F\ {\isacharhash}\ subformulae{\isacharunderscore}s\ F{\isachardoublequoteclose}\ {\isacharbar}\isanewline
{\isachardoublequoteopen}subformulae{\isacharunderscore}s\ {\isacharparenleft}Bi\ F\ G{\isacharparenright}\ {\isacharequal}\ Bi\ F\ G\ {\isacharhash}\ subformulae{\isacharunderscore}s\ F\ {\isacharat}\ subformulae{\isacharunderscore}s\ G{\isachardoublequoteclose}%
\begin{isamarkuptext}%
Siguiendo con los ejemplos, observemos cómo actúa \isa{subformulae} en las distintas fórmulas. 
En primer lugar, veamos los casos base de fórmulas atómicas y con conectiva de negación.%
\end{isamarkuptext}\isamarkuptrue%
\isacommand{value}\isamarkupfalse%
\ {\isachardoublequoteopen}subformulae\ {\isacharparenleft}Atom\ p{\isacharparenright}\ {\isacharequal}\ {\isacharbrackleft}Atom\ p{\isacharbrackright}{\isachardoublequoteclose}\isanewline
\isanewline
\isacommand{value}\isamarkupfalse%
\ {\isachardoublequoteopen}subformulae\ {\isacharparenleft}Not\ {\isacharparenleft}Atom\ p{\isacharparenright}{\isacharparenright}\ {\isacharequal}\ {\isacharbrackleft}Not\ {\isacharparenleft}Atom\ p{\isacharparenright}{\isacharcomma}\ Atom\ p{\isacharbrackright}{\isachardoublequoteclose}%
\begin{isamarkuptext}%
A continuación, una fórmula con conectivas binarias y variables todas distintas.%
\end{isamarkuptext}\isamarkuptrue%
\isacommand{value}\isamarkupfalse%
\ {\isachardoublequoteopen}subformulae\ {\isacharparenleft}Or\ {\isacharparenleft}Imp\ {\isacharparenleft}Atom\ p{\isacharparenright}\ {\isacharparenleft}Atom\ q{\isacharparenright}{\isacharparenright}\ {\isacharparenleft}Atom\ r{\isacharparenright}{\isacharparenright}\ {\isacharequal}\ \isanewline
\ \ {\isacharbrackleft}{\isacharparenleft}Atom\ p\ \isactrlbold {\isasymrightarrow}\ Atom\ q{\isacharparenright}\ \isactrlbold {\isasymor}\ Atom\ r{\isacharcomma}\ Atom\ p\ \isactrlbold {\isasymrightarrow}\ Atom\ q{\isacharcomma}\ Atom\ p{\isacharcomma}\ Atom\ q{\isacharcomma}\ Atom\ r{\isacharbrackright}{\isachardoublequoteclose}%
\begin{isamarkuptext}%
En particular, al tratarse de una lista pueden aparecer elementos repetidos como se muestra a
continuación.%
\end{isamarkuptext}\isamarkuptrue%
\isacommand{value}\isamarkupfalse%
\ {\isachardoublequoteopen}subformulae\ {\isacharparenleft}Or\ {\isacharparenleft}Atom\ p{\isacharparenright}\ {\isacharparenleft}Atom\ p{\isacharparenright}{\isacharparenright}\ {\isacharequal}\ \ \isanewline
\ \ {\isacharbrackleft}Or\ {\isacharparenleft}Atom\ p{\isacharparenright}\ {\isacharparenleft}Atom\ p{\isacharparenright}{\isacharcomma}\ Atom\ p{\isacharcomma}\ Atom\ p{\isacharbrackright}{\isachardoublequoteclose}\isanewline
\isanewline
\isacommand{value}\isamarkupfalse%
\ {\isachardoublequoteopen}subformulae\ {\isacharparenleft}Or\ {\isacharparenleft}Atom\ p{\isacharparenright}\ {\isacharparenleft}Atom\ p{\isacharparenright}{\isacharparenright}\ {\isacharequal}\ \ \isanewline
\ \ {\isacharbrackleft}Or\ {\isacharparenleft}Atom\ p{\isacharparenright}\ {\isacharparenleft}Atom\ p{\isacharparenright}{\isacharcomma}\ Atom\ p{\isacharbrackright}\ {\isacharequal}\ False{\isachardoublequoteclose}%
\begin{isamarkuptext}%
Veamos su valor en presencia de \isa{Bot}.%
\end{isamarkuptext}\isamarkuptrue%
\isacommand{value}\isamarkupfalse%
\ {\isachardoublequoteopen}subformulae\ {\isacharparenleft}And\ {\isacharparenleft}Atom\ p{\isacharparenright}\ Bot{\isacharparenright}\ {\isacharequal}\ \isanewline
\ \ {\isacharbrackleft}And\ {\isacharparenleft}Atom\ p{\isacharparenright}\ Bot{\isacharcomma}\ Atom\ p{\isacharcomma}\ Bot{\isacharbrackright}{\isachardoublequoteclose}%
\begin{isamarkuptext}%
A partir de esta definición, aparecen varios resultados sencillos 
que demostraremos siguiendo tácticas similares a las empleadas en el 
lema anterior. Como se ha argumentado anteriormente,
para resumir las demostraciones detalladas se harán mediante las definiciones simplificadas de los tipos.
Además, trabajaremos
con conjuntos en vez de listas, pues poseen ventajas como la eliminación de elementos repetidos
o las operaciones de conjuntos.
De este modo, definimos \isa{setSubformulae}, que convierte en un tipo conjunto la lista de 
subfórmulas anterior. Añadimos también su versión simplificada.%
\end{isamarkuptext}\isamarkuptrue%
\isacommand{abbreviation}\isamarkupfalse%
\ setSubformulae\ {\isacharcolon}{\isacharcolon}\ {\isachardoublequoteopen}{\isacharprime}a\ formula\ {\isasymRightarrow}\ {\isacharprime}a\ formula\ set{\isachardoublequoteclose}\ \isakeyword{where}\isanewline
{\isachardoublequoteopen}setSubformulae\ F\ {\isasymequiv}\ set\ {\isacharparenleft}subformulae\ F{\isacharparenright}{\isachardoublequoteclose}\isanewline
\isanewline
\isacommand{abbreviation}\isamarkupfalse%
\ setSubformulae{\isacharunderscore}s\ {\isacharcolon}{\isacharcolon}\ {\isachardoublequoteopen}{\isacharprime}a\ formula{\isacharunderscore}simp\ {\isasymRightarrow}\ {\isacharprime}a\ formula{\isacharunderscore}simp\ set{\isachardoublequoteclose}\ \isakeyword{where}\isanewline
{\isachardoublequoteopen}setSubformulae{\isacharunderscore}s\ F\ {\isasymequiv}\ set\ {\isacharparenleft}subformulae{\isacharunderscore}s\ F{\isacharparenright}{\isachardoublequoteclose}%
\begin{isamarkuptext}%
De este modo, observemos la diferencia de repetición con el ejemplo anterior.%
\end{isamarkuptext}\isamarkuptrue%
\isacommand{value}\isamarkupfalse%
\ {\isachardoublequoteopen}setSubformulae\ {\isacharparenleft}Or\ {\isacharparenleft}Atom\ p{\isacharparenright}\ {\isacharparenleft}Atom\ p{\isacharparenright}{\isacharparenright}\ {\isacharequal}\ \ {\isacharbraceleft}Or\ {\isacharparenleft}Atom\ p{\isacharparenright}\ {\isacharparenleft}Atom\ p{\isacharparenright}{\isacharcomma}\ Atom\ p{\isacharbraceright}{\isachardoublequoteclose}\isanewline
\isanewline
\isacommand{lemma}\isamarkupfalse%
\ {\isachardoublequoteopen}setSubformulae\ {\isacharparenleft}Or\ {\isacharparenleft}Imp\ {\isacharparenleft}Atom\ p{\isacharparenright}\ {\isacharparenleft}Atom\ q{\isacharparenright}{\isacharparenright}\ {\isacharparenleft}Atom\ r{\isacharparenright}{\isacharparenright}\ {\isacharequal}\ \isanewline
\ \ {\isacharbraceleft}{\isacharparenleft}Atom\ p\ \isactrlbold {\isasymrightarrow}\ Atom\ q{\isacharparenright}\ \isactrlbold {\isasymor}\ Atom\ r{\isacharcomma}\ Atom\ p\ \isactrlbold {\isasymrightarrow}\ Atom\ q{\isacharcomma}\ Atom\ p{\isacharcomma}\ Atom\ q{\isacharcomma}\ Atom\ r{\isacharbraceright}{\isachardoublequoteclose}\isanewline
%
\isadelimproof
\ \ %
\endisadelimproof
%
\isatagproof
\isacommand{by}\isamarkupfalse%
\ auto%
\endisatagproof
{\isafoldproof}%
%
\isadelimproof
%
\endisadelimproof
%
\begin{isamarkuptext}%
Como hemos señalado, utilizaremos varios resultados de la teoría de conjuntos definida en Isabelle
como \href{https://n9.cl/qatp}{Set.thy}.
Voy a especificar el esquema de las usadas en esta sección que no están incluidas en las tácticas de
simplificación para aclarar las demostraciones detalladas que presentaré en este apartado.\\
 \begin{itemize}
  \item[] \isa{\mbox{}\inferrule{\mbox{A\ {\isasymsubseteq}\ B\ {\isasymand}\ B\ {\isasymsubseteq}\ C}}{\mbox{A\ {\isasymsubseteq}\ C}}} \hfill (\isa{subset{\isacharunderscore}trans})
  \end{itemize}

 \begin{itemize}
  \item[] \isa{\mbox{}\inferrule{\mbox{c\ {\isasymin}\ A\ {\isasymand}\ A\ {\isasymsubseteq}\ B}}{\mbox{c\ {\isasymin}\ B}}} \hfill (\isa{rev{\isacharunderscore}subsetD})
  \end{itemize}
Además, definiré alguna propiedad más que no aparece en la teoría de Isabelle y que utilizaremos
con frecuencia. Su demostración será la automática.%
\end{isamarkuptext}\isamarkuptrue%
\isacommand{lemma}\isamarkupfalse%
\ subContUnion{\isadigit{1}}{\isacharcolon}\ {\isachardoublequoteopen}A\ {\isacharequal}\ B\ {\isasymunion}\ C\ {\isasymLongrightarrow}\ B\ {\isasymsubseteq}\ A{\isachardoublequoteclose}\isanewline
%
\isadelimproof
\ \ %
\endisadelimproof
%
\isatagproof
\isacommand{by}\isamarkupfalse%
\ auto%
\endisatagproof
{\isafoldproof}%
%
\isadelimproof
\isanewline
%
\endisadelimproof
\isanewline
\isacommand{lemma}\isamarkupfalse%
\ subContUnion{\isadigit{2}}{\isacharcolon}\ {\isachardoublequoteopen}A\ {\isacharequal}\ B\ {\isasymunion}\ C\ {\isasymLongrightarrow}\ C\ {\isasymsubseteq}\ A{\isachardoublequoteclose}\isanewline
%
\isadelimproof
\ \ %
\endisadelimproof
%
\isatagproof
\isacommand{by}\isamarkupfalse%
\ auto%
\endisatagproof
{\isafoldproof}%
%
\isadelimproof
\isanewline
%
\endisadelimproof
\isanewline
\isacommand{lemma}\isamarkupfalse%
\ subContUnionRev{\isadigit{1}}{\isacharcolon}\ {\isachardoublequoteopen}A\ {\isasymunion}\ B\ {\isasymsubseteq}\ C\ {\isasymLongrightarrow}\ A\ {\isasymsubseteq}\ C{\isachardoublequoteclose}\isanewline
%
\isadelimproof
\ \ %
\endisadelimproof
%
\isatagproof
\isacommand{by}\isamarkupfalse%
\ auto%
\endisatagproof
{\isafoldproof}%
%
\isadelimproof
\isanewline
%
\endisadelimproof
\isanewline
\isacommand{lemma}\isamarkupfalse%
\ subContUnionRev{\isadigit{2}}{\isacharcolon}\ {\isachardoublequoteopen}A\ {\isasymunion}\ B\ {\isasymsubseteq}\ C\ {\isasymLongrightarrow}\ B\ {\isasymsubseteq}\ C{\isachardoublequoteclose}\isanewline
%
\isadelimproof
\ \ %
\endisadelimproof
%
\isatagproof
\isacommand{by}\isamarkupfalse%
\ auto%
\endisatagproof
{\isafoldproof}%
%
\isadelimproof
\isanewline
%
\endisadelimproof
\isanewline
\isacommand{lemma}\isamarkupfalse%
\ subConts{\isacharcolon}\ {\isachardoublequoteopen}{\isasymlbrakk}A\ {\isasymsubseteq}\ B{\isacharsemicolon}\ C\ {\isasymsubseteq}\ D{\isasymrbrakk}\ {\isasymLongrightarrow}\ \ A\ {\isasymunion}\ C\ {\isasymsubseteq}\ B\ {\isasymunion}\ D{\isachardoublequoteclose}\isanewline
%
\isadelimproof
\ \ %
\endisadelimproof
%
\isatagproof
\isacommand{by}\isamarkupfalse%
\ auto%
\endisatagproof
{\isafoldproof}%
%
\isadelimproof
%
\endisadelimproof
%
\begin{isamarkuptext}%
Una vez aclarada la nueva función de conjuntos, vamos a demostrar el siguiente lema sirviéndonos 
de ella.

 \begin{lema}
    El conjunto de los átomos de una fórmula está contenido en el conjunto de subfórmulas de la 
    misma, es decir, las fórmulas atómicas son subfórmulas.
  \end{lema}

En Isabelle, se especifica como sigue.%
\end{isamarkuptext}\isamarkuptrue%
\isacommand{lemma}\isamarkupfalse%
\ atoms{\isacharunderscore}are{\isacharunderscore}subformulae{\isacharcolon}\ {\isachardoublequoteopen}Atom\ {\isacharbackquote}\ atoms\ F\ {\isasymsubseteq}\ setSubformulae\ F{\isachardoublequoteclose}\isanewline
%
\isadelimproof
\ \ %
\endisadelimproof
%
\isatagproof
\isacommand{oops}\isamarkupfalse%
%
\endisatagproof
{\isafoldproof}%
%
\isadelimproof
%
\endisadelimproof
%
\begin{isamarkuptext}%
Este resultado es especialmente interesante para aclarar la naturaleza de la función 
\isa{atoms} aplicada a una fórmula. De este modo, \isa{Atom\ {\isacharbackquote}\ atoms\ F} se encarga de 
construir las fórmulas atómicas a partir de cada uno de los elementos del conjunto de átomos de la fórmula \isa{F}, 
creando un conjunto de fórmulas atómicas. Para ello emplea el infijo \isa{{\isacharbackquote}} definido como notación
abreviada de \isa{{\isacharparenleft}{\isacharbackquote}{\isacharparenright}} que calcula la imagen de un conjunto en la teoría \href{https://n9.cl/qatp}{Set.thy}.
 \begin{itemize}
  \item[] \isa{f\ {\isacharbackquote}\ A\ {\isacharequal}\ {\isacharbraceleft}y\ {\isacharbar}\ {\isasymexists}x{\isasymin}A{\isachardot}\ y\ {\isacharequal}\ f\ x{\isacharbraceright}} \hfill (\isa{image{\isacharunderscore}def})
  \end{itemize}
Para aclarar su funcionamiento, veamos ejemplos para distintos casos de fórmulas.%
\end{isamarkuptext}\isamarkuptrue%
\isacommand{value}\isamarkupfalse%
\ {\isachardoublequoteopen}Atom\ {\isacharbackquote}atoms\ {\isacharparenleft}Or\ {\isacharparenleft}Atom\ p{\isacharparenright}\ Bot{\isacharparenright}\ {\isacharequal}\ {\isacharbraceleft}Atom\ p{\isacharbraceright}{\isachardoublequoteclose}\isanewline
\isanewline
\isacommand{lemma}\isamarkupfalse%
\ {\isachardoublequoteopen}Atom\ {\isacharbackquote}atoms\ {\isacharparenleft}Or\ {\isacharparenleft}Imp\ {\isacharparenleft}Atom\ p{\isacharparenright}\ {\isacharparenleft}Atom\ q{\isacharparenright}{\isacharparenright}\ {\isacharparenleft}Atom\ r{\isacharparenright}{\isacharparenright}\ {\isacharequal}\ {\isacharbraceleft}Atom\ p{\isacharcomma}\ Atom\ q{\isacharcomma}\ Atom\ r{\isacharbraceright}{\isachardoublequoteclose}\isanewline
%
\isadelimproof
\ \ %
\endisadelimproof
%
\isatagproof
\isacommand{by}\isamarkupfalse%
\ auto%
\endisatagproof
{\isafoldproof}%
%
\isadelimproof
\isanewline
%
\endisadelimproof
\isanewline
\isacommand{lemma}\isamarkupfalse%
\ {\isachardoublequoteopen}Atom\ {\isacharbackquote}atoms\ {\isacharparenleft}Or\ {\isacharparenleft}Imp\ {\isacharparenleft}Atom\ p{\isacharparenright}\ {\isacharparenleft}Atom\ p{\isacharparenright}{\isacharparenright}\ {\isacharparenleft}Atom\ r{\isacharparenright}{\isacharparenright}\ {\isacharequal}\ {\isacharbraceleft}Atom\ p{\isacharcomma}\ Atom\ r{\isacharbraceright}{\isachardoublequoteclose}\isanewline
%
\isadelimproof
\ \ %
\endisadelimproof
%
\isatagproof
\isacommand{by}\isamarkupfalse%
\ auto%
\endisatagproof
{\isafoldproof}%
%
\isadelimproof
%
\endisadelimproof
%
\begin{isamarkuptext}%
Además, esta función tiene la siguiente propiedad sobre la unión de conjuntos que utilizaré
en la demostración.\\
 \begin{itemize}
  \item[] \isa{f\ {\isacharbackquote}\ {\isacharparenleft}A\ {\isasymunion}\ B{\isacharparenright}\ {\isacharequal}\ f\ {\isacharbackquote}\ A\ {\isasymunion}\ f\ {\isacharbackquote}\ B} \hfill (\isa{image{\isacharunderscore}Un})
  \end{itemize}%
\end{isamarkuptext}\isamarkuptrue%
%
\begin{isamarkuptext}%
Una vez hecha la aclaración anterior, comencemos la demostración detallada simplificada, 
que seguirá el esquema inductivo señalado con anterioridad.%
\end{isamarkuptext}\isamarkuptrue%
\isacommand{lemma}\isamarkupfalse%
\ atoms{\isacharunderscore}are{\isacharunderscore}subformulae{\isacharunderscore}detallada{\isacharunderscore}s{\isacharcolon}\ {\isachardoublequoteopen}Atom{\isacharunderscore}s\ {\isacharbackquote}\ atoms{\isacharunderscore}s\ F\ {\isasymsubseteq}\ setSubformulae{\isacharunderscore}s\ F{\isachardoublequoteclose}\isanewline
%
\isadelimproof
%
\endisadelimproof
%
\isatagproof
\isacommand{proof}\isamarkupfalse%
\ {\isacharparenleft}induction\ F{\isacharparenright}\isanewline
\isacommand{case}\isamarkupfalse%
\ {\isacharparenleft}Atom{\isacharunderscore}s\ x{\isacharparenright}\isanewline
\ \ \isacommand{then}\isamarkupfalse%
\ \isacommand{show}\isamarkupfalse%
\ {\isacharquery}case\ \isacommand{by}\isamarkupfalse%
\ simp\isanewline
\isacommand{next}\isamarkupfalse%
\isanewline
\ \ \isacommand{case}\isamarkupfalse%
\ Bot{\isacharunderscore}s\isanewline
\ \ \isacommand{then}\isamarkupfalse%
\ \isacommand{show}\isamarkupfalse%
\ {\isacharquery}case\ \isacommand{by}\isamarkupfalse%
\ simp\isanewline
\isacommand{next}\isamarkupfalse%
\isanewline
\ \ \isacommand{case}\isamarkupfalse%
\ {\isacharparenleft}Mon\ F{\isacharparenright}\isanewline
\ \ \isacommand{assume}\isamarkupfalse%
\ H{\isacharcolon}{\isachardoublequoteopen}Atom{\isacharunderscore}s\ {\isacharbackquote}\ atoms{\isacharunderscore}s\ F\ {\isasymsubseteq}\ setSubformulae{\isacharunderscore}s\ F{\isachardoublequoteclose}\isanewline
\ \ \isacommand{show}\isamarkupfalse%
\ {\isachardoublequoteopen}Atom{\isacharunderscore}s\ {\isacharbackquote}\ atoms{\isacharunderscore}s\ {\isacharparenleft}Mon\ F{\isacharparenright}\ {\isasymsubseteq}\ setSubformulae{\isacharunderscore}s\ {\isacharparenleft}Mon\ F{\isacharparenright}{\isachardoublequoteclose}\isanewline
\ \ \isacommand{proof}\isamarkupfalse%
\ {\isacharminus}\isanewline
\ \ \ \ \isacommand{have}\isamarkupfalse%
\ {\isachardoublequoteopen}setSubformulae{\isacharunderscore}s\ {\isacharparenleft}Mon\ F{\isacharparenright}\ {\isacharequal}\ {\isacharbraceleft}Mon\ F{\isacharbraceright}\ {\isasymunion}\ setSubformulae{\isacharunderscore}s\ F{\isachardoublequoteclose}\ \isacommand{by}\isamarkupfalse%
\ simp\isanewline
\ \ \ \ \isacommand{then}\isamarkupfalse%
\ \isacommand{have}\isamarkupfalse%
\ {\isadigit{1}}{\isacharcolon}{\isachardoublequoteopen}setSubformulae{\isacharunderscore}s\ F\ {\isasymsubseteq}\ setSubformulae{\isacharunderscore}s\ {\isacharparenleft}Mon\ F{\isacharparenright}{\isachardoublequoteclose}\ \isacommand{by}\isamarkupfalse%
\ {\isacharparenleft}rule\ subContUnion{\isadigit{2}}{\isacharparenright}\isanewline
\ \ \ \ \isacommand{also}\isamarkupfalse%
\ \isacommand{have}\isamarkupfalse%
\ {\isachardoublequoteopen}Atom{\isacharunderscore}s\ {\isacharbackquote}\ atoms{\isacharunderscore}s\ F\ {\isacharequal}\ Atom{\isacharunderscore}s\ {\isacharbackquote}\ atoms{\isacharunderscore}s\ {\isacharparenleft}Mon\ F{\isacharparenright}{\isachardoublequoteclose}\ \isacommand{by}\isamarkupfalse%
\ simp\ \isanewline
\ \ \ \ \isacommand{then}\isamarkupfalse%
\ \isacommand{have}\isamarkupfalse%
\ {\isachardoublequoteopen}Atom{\isacharunderscore}s\ {\isacharbackquote}\ atoms{\isacharunderscore}s\ {\isacharparenleft}Mon\ F{\isacharparenright}\ {\isasymsubseteq}\ setSubformulae{\isacharunderscore}s\ F{\isachardoublequoteclose}\ \isacommand{using}\isamarkupfalse%
\ H\ \isacommand{by}\isamarkupfalse%
\ simp\ \isanewline
\ \ \ \ \isacommand{then}\isamarkupfalse%
\ \isacommand{show}\isamarkupfalse%
\ {\isachardoublequoteopen}Atom{\isacharunderscore}s\ {\isacharbackquote}\ atoms{\isacharunderscore}s\ {\isacharparenleft}Mon\ F{\isacharparenright}\ {\isasymsubseteq}\ setSubformulae{\isacharunderscore}s\ {\isacharparenleft}Mon\ F{\isacharparenright}{\isachardoublequoteclose}\ \isacommand{using}\isamarkupfalse%
\ {\isadigit{1}}\ \isacommand{by}\isamarkupfalse%
\ {\isacharparenleft}rule\ subset{\isacharunderscore}trans{\isacharparenright}\isanewline
\ \ \isacommand{qed}\isamarkupfalse%
\isanewline
\isacommand{next}\isamarkupfalse%
\isanewline
\ \ \isacommand{case}\isamarkupfalse%
\ {\isacharparenleft}Bi\ F{\isadigit{1}}\ F{\isadigit{2}}{\isacharparenright}\isanewline
\ \ \isacommand{assume}\isamarkupfalse%
\ H{\isadigit{1}}{\isacharcolon}{\isachardoublequoteopen}Atom{\isacharunderscore}s\ {\isacharbackquote}\ atoms{\isacharunderscore}s\ F{\isadigit{1}}\ {\isasymsubseteq}\ setSubformulae{\isacharunderscore}s\ F{\isadigit{1}}{\isachardoublequoteclose}\isanewline
\ \ \isacommand{assume}\isamarkupfalse%
\ H{\isadigit{2}}{\isacharcolon}{\isachardoublequoteopen}Atom{\isacharunderscore}s\ {\isacharbackquote}\ atoms{\isacharunderscore}s\ F{\isadigit{2}}\ {\isasymsubseteq}\ setSubformulae{\isacharunderscore}s\ F{\isadigit{2}}{\isachardoublequoteclose}\isanewline
\ \ \isacommand{show}\isamarkupfalse%
\ {\isachardoublequoteopen}Atom{\isacharunderscore}s\ {\isacharbackquote}\ atoms{\isacharunderscore}s\ {\isacharparenleft}Bi\ F{\isadigit{1}}\ F{\isadigit{2}}{\isacharparenright}\ {\isasymsubseteq}\ setSubformulae{\isacharunderscore}s\ {\isacharparenleft}Bi\ F{\isadigit{1}}\ F{\isadigit{2}}{\isacharparenright}{\isachardoublequoteclose}\isanewline
\ \ \isacommand{proof}\isamarkupfalse%
\ {\isacharminus}\isanewline
\ \ \ \ \isacommand{have}\isamarkupfalse%
\ {\isadigit{2}}{\isacharcolon}{\isachardoublequoteopen}{\isacharparenleft}Atom{\isacharunderscore}s\ {\isacharbackquote}\ atoms{\isacharunderscore}s\ F{\isadigit{1}}{\isacharparenright}\ {\isasymunion}\ {\isacharparenleft}Atom{\isacharunderscore}s\ {\isacharbackquote}\ atoms{\isacharunderscore}s\ F{\isadigit{2}}{\isacharparenright}\ {\isasymsubseteq}\ setSubformulae{\isacharunderscore}s\ F{\isadigit{1}}\ {\isasymunion}\ setSubformulae{\isacharunderscore}s\ F{\isadigit{2}}{\isachardoublequoteclose}\ \isanewline
\ \ \ \ \ \ \isacommand{using}\isamarkupfalse%
\ H{\isadigit{1}}\ H{\isadigit{2}}\ \isacommand{by}\isamarkupfalse%
\ {\isacharparenleft}rule\ subConts{\isacharparenright}\isanewline
\ \ \ \ \isacommand{have}\isamarkupfalse%
\ {\isachardoublequoteopen}setSubformulae{\isacharunderscore}s\ {\isacharparenleft}Bi\ F{\isadigit{1}}\ F{\isadigit{2}}{\isacharparenright}\ {\isacharequal}\ {\isacharbraceleft}Bi\ F{\isadigit{1}}\ F{\isadigit{2}}{\isacharbraceright}\ {\isasymunion}\ {\isacharparenleft}setSubformulae{\isacharunderscore}s\ F{\isadigit{1}}\ {\isasymunion}\ setSubformulae{\isacharunderscore}s\ F{\isadigit{2}}{\isacharparenright}{\isachardoublequoteclose}\ \isacommand{by}\isamarkupfalse%
\ simp\isanewline
\ \ \ \ \isacommand{then}\isamarkupfalse%
\ \isacommand{have}\isamarkupfalse%
\ {\isadigit{3}}{\isacharcolon}{\isachardoublequoteopen}setSubformulae{\isacharunderscore}s\ F{\isadigit{1}}\ {\isasymunion}\ setSubformulae{\isacharunderscore}s\ F{\isadigit{2}}\ {\isasymsubseteq}\ setSubformulae{\isacharunderscore}s\ {\isacharparenleft}Bi\ F{\isadigit{1}}\ F{\isadigit{2}}{\isacharparenright}{\isachardoublequoteclose}\ \isacommand{by}\isamarkupfalse%
\ {\isacharparenleft}rule\ subContUnion{\isadigit{2}}{\isacharparenright}\isanewline
\ \ \ \ \isacommand{then}\isamarkupfalse%
\ \isacommand{have}\isamarkupfalse%
\ {\isachardoublequoteopen}setSubformulae{\isacharunderscore}s\ F{\isadigit{1}}\ {\isasymsubseteq}\ setSubformulae{\isacharunderscore}s\ {\isacharparenleft}Bi\ F{\isadigit{1}}\ F{\isadigit{2}}{\isacharparenright}{\isachardoublequoteclose}\ \isacommand{by}\isamarkupfalse%
\ simp\isanewline
\ \ \ \ \isacommand{have}\isamarkupfalse%
\ {\isachardoublequoteopen}setSubformulae{\isacharunderscore}s\ F{\isadigit{2}}\ {\isasymsubseteq}\ setSubformulae{\isacharunderscore}s\ {\isacharparenleft}Bi\ F{\isadigit{1}}\ F{\isadigit{2}}{\isacharparenright}{\isachardoublequoteclose}\ \isacommand{using}\isamarkupfalse%
\ {\isadigit{3}}\ \isacommand{by}\isamarkupfalse%
\ simp\isanewline
\ \ \ \ \isacommand{also}\isamarkupfalse%
\ \isacommand{have}\isamarkupfalse%
\ {\isachardoublequoteopen}atoms{\isacharunderscore}s\ {\isacharparenleft}Bi\ F{\isadigit{1}}\ F{\isadigit{2}}{\isacharparenright}\ {\isacharequal}\ atoms{\isacharunderscore}s\ F{\isadigit{1}}\ {\isasymunion}\ atoms{\isacharunderscore}s\ F{\isadigit{2}}{\isachardoublequoteclose}\ \isacommand{by}\isamarkupfalse%
\ simp\isanewline
\ \ \ \ \isacommand{then}\isamarkupfalse%
\ \isacommand{have}\isamarkupfalse%
\ {\isachardoublequoteopen}Atom{\isacharunderscore}s\ {\isacharbackquote}\ atoms{\isacharunderscore}s\ {\isacharparenleft}Bi\ F{\isadigit{1}}\ F{\isadigit{2}}{\isacharparenright}\ {\isacharequal}\ Atom{\isacharunderscore}s\ {\isacharbackquote}\ {\isacharparenleft}atoms{\isacharunderscore}s\ F{\isadigit{1}}\ {\isasymunion}\ atoms{\isacharunderscore}s\ F{\isadigit{2}}{\isacharparenright}{\isachardoublequoteclose}\ \isacommand{by}\isamarkupfalse%
\ simp\isanewline
\ \ \ \ \isacommand{also}\isamarkupfalse%
\ \isacommand{have}\isamarkupfalse%
\ {\isachardoublequoteopen}{\isachardot}{\isachardot}{\isachardot}\ {\isacharequal}\ Atom{\isacharunderscore}s\ {\isacharbackquote}\ atoms{\isacharunderscore}s\ F{\isadigit{1}}\ {\isasymunion}\ Atom{\isacharunderscore}s\ {\isacharbackquote}\ atoms{\isacharunderscore}s\ F{\isadigit{2}}{\isachardoublequoteclose}\ \isacommand{by}\isamarkupfalse%
\ {\isacharparenleft}rule\ image{\isacharunderscore}Un{\isacharparenright}\isanewline
\ \ \ \ \isacommand{then}\isamarkupfalse%
\ \isacommand{have}\isamarkupfalse%
\ {\isachardoublequoteopen}Atom{\isacharunderscore}s\ {\isacharbackquote}\ atoms{\isacharunderscore}s\ {\isacharparenleft}Bi\ F{\isadigit{1}}\ F{\isadigit{2}}{\isacharparenright}\ {\isacharequal}\ Atom{\isacharunderscore}s\ {\isacharbackquote}\ atoms{\isacharunderscore}s\ F{\isadigit{1}}\ {\isasymunion}\ Atom{\isacharunderscore}s\ {\isacharbackquote}\ atoms{\isacharunderscore}s\ F{\isadigit{2}}{\isachardoublequoteclose}\ \isacommand{by}\isamarkupfalse%
\ simp\isanewline
\ \ \ \ \isacommand{then}\isamarkupfalse%
\ \isacommand{have}\isamarkupfalse%
\ {\isachardoublequoteopen}Atom{\isacharunderscore}s\ {\isacharbackquote}\ atoms{\isacharunderscore}s\ {\isacharparenleft}Bi\ F{\isadigit{1}}\ F{\isadigit{2}}{\isacharparenright}\ {\isasymsubseteq}\ setSubformulae{\isacharunderscore}s\ F{\isadigit{1}}\ {\isasymunion}\ setSubformulae{\isacharunderscore}s\ F{\isadigit{2}}{\isachardoublequoteclose}\ \isacommand{using}\isamarkupfalse%
\ {\isadigit{2}}\ \isacommand{by}\isamarkupfalse%
\ simp\isanewline
\ \ \ \ \isacommand{then}\isamarkupfalse%
\ \isacommand{show}\isamarkupfalse%
\ {\isachardoublequoteopen}Atom{\isacharunderscore}s\ {\isacharbackquote}\ atoms{\isacharunderscore}s\ {\isacharparenleft}Bi\ F{\isadigit{1}}\ F{\isadigit{2}}{\isacharparenright}\ {\isasymsubseteq}\ setSubformulae{\isacharunderscore}s\ {\isacharparenleft}Bi\ F{\isadigit{1}}\ F{\isadigit{2}}{\isacharparenright}{\isachardoublequoteclose}\ \isacommand{using}\isamarkupfalse%
\ {\isadigit{3}}\ \isacommand{by}\isamarkupfalse%
\ {\isacharparenleft}rule\ subset{\isacharunderscore}trans{\isacharparenright}\isanewline
\ \ \isacommand{qed}\isamarkupfalse%
\isanewline
\isacommand{qed}\isamarkupfalse%
%
\endisatagproof
{\isafoldproof}%
%
\isadelimproof
%
\endisadelimproof
%
\begin{isamarkuptext}%
Mostremos también la demostración automática con la definición original.%
\end{isamarkuptext}\isamarkuptrue%
\isacommand{lemma}\isamarkupfalse%
\ atoms{\isacharunderscore}are{\isacharunderscore}subformulae{\isacharcolon}\ {\isachardoublequoteopen}Atom\ {\isacharbackquote}\ atoms\ F\ {\isasymsubseteq}\ setSubformulae\ F{\isachardoublequoteclose}\isanewline
%
\isadelimproof
\ \ %
\endisadelimproof
%
\isatagproof
\isacommand{by}\isamarkupfalse%
\ {\isacharparenleft}induction\ F{\isacharparenright}\ auto%
\endisatagproof
{\isafoldproof}%
%
\isadelimproof
%
\endisadelimproof
%
\begin{isamarkuptext}%
Otro resultado de esta sección es la propia pertenencia de una fórmula en el conjunto 
de sus subfórmulas. 
\begin{lema}
    La propia fórmula pertence a la lista de sus subfórmulas, es decir: \isa{F\ {\isasymin}\ Subf{\isacharparenleft}F{\isacharparenright}}.
  \end{lema}

A continuación incluimos el enunciado del lema con su demostración automática. Las demostraciones detallada
y aplicativa son análogas a las del primer lema de la sección, utilizando únicamente inducción y 
simplificación. Para facilitar pruebas posteriores, vamos a añadir la regla a las tácticas de 
simplificación.%
\end{isamarkuptext}\isamarkuptrue%
\isacommand{lemma}\isamarkupfalse%
\ subformulae{\isacharunderscore}self{\isacharbrackleft}simp{\isacharcomma}intro{\isacharbrackright}{\isacharcolon}\ {\isachardoublequoteopen}F\ {\isasymin}\ setSubformulae\ F{\isachardoublequoteclose}\isanewline
%
\isadelimproof
\ \ %
\endisadelimproof
%
\isatagproof
\isacommand{by}\isamarkupfalse%
\ {\isacharparenleft}induction\ F{\isacharparenright}\ simp{\isacharunderscore}all%
\endisatagproof
{\isafoldproof}%
%
\isadelimproof
\ \isanewline
%
\endisadelimproof
\isanewline
\isacommand{lemma}\isamarkupfalse%
\ subformulae{\isacharunderscore}self{\isacharunderscore}s{\isacharbrackleft}simp{\isacharcomma}intro{\isacharbrackright}{\isacharcolon}\ {\isachardoublequoteopen}F\ {\isasymin}\ setSubformulae{\isacharunderscore}s\ F{\isachardoublequoteclose}\isanewline
%
\isadelimproof
\ \ %
\endisadelimproof
%
\isatagproof
\isacommand{by}\isamarkupfalse%
\ {\isacharparenleft}induction\ F{\isacharparenright}\ simp{\isacharunderscore}all%
\endisatagproof
{\isafoldproof}%
%
\isadelimproof
%
\endisadelimproof
%
\begin{isamarkuptext}%
La siguiente propiedad declara que el conjunto de átomos de una subfórmula está contenido en el
conjunto de átomos de la propia fórmula.
\begin{lema}
    Sea \isa{G\ {\isasymin}\ Subf{\isacharparenleft}F{\isacharparenright}}, entonces el conjunto de los átomos de \isa{G} está contenido en los de \isa{F}.
  \end{lema}

Traducido a Isabelle:%
\end{isamarkuptext}\isamarkuptrue%
\isacommand{lemma}\isamarkupfalse%
\ subformula{\isacharunderscore}atoms{\isacharcolon}\ {\isachardoublequoteopen}G\ {\isasymin}\ setSubformulae\ F\ {\isasymLongrightarrow}\ atoms\ G\ {\isasymsubseteq}\ atoms\ F{\isachardoublequoteclose}\isanewline
%
\isadelimproof
\ \ %
\endisadelimproof
%
\isatagproof
\isacommand{oops}\isamarkupfalse%
%
\endisatagproof
{\isafoldproof}%
%
\isadelimproof
%
\endisadelimproof
%
\begin{isamarkuptext}%
Veamos su demostración estructurada con la definición simplificada para resumir los casos
de conectivas binarias.%
\end{isamarkuptext}\isamarkuptrue%
\isacommand{lemma}\isamarkupfalse%
\ subformula{\isacharunderscore}atoms{\isacharunderscore}estructurada{\isacharunderscore}s{\isacharcolon}\ {\isachardoublequoteopen}G\ {\isasymin}\ setSubformulae{\isacharunderscore}s\ F\ {\isasymLongrightarrow}\ atoms{\isacharunderscore}s\ G\ {\isasymsubseteq}\ atoms{\isacharunderscore}s\ F{\isachardoublequoteclose}\isanewline
%
\isadelimproof
%
\endisadelimproof
%
\isatagproof
\isacommand{proof}\isamarkupfalse%
\ {\isacharparenleft}induction\ F{\isacharparenright}\isanewline
\ \ \isacommand{case}\isamarkupfalse%
\ {\isacharparenleft}Atom{\isacharunderscore}s\ x{\isacharparenright}\isanewline
\ \ \isacommand{then}\isamarkupfalse%
\ \isacommand{show}\isamarkupfalse%
\ {\isacharquery}case\ \isacommand{by}\isamarkupfalse%
\ simp\ \isanewline
\isacommand{next}\isamarkupfalse%
\isanewline
\ \ \isacommand{case}\isamarkupfalse%
\ Bot{\isacharunderscore}s\isanewline
\ \ \isacommand{then}\isamarkupfalse%
\ \isacommand{show}\isamarkupfalse%
\ {\isacharquery}case\ \isacommand{by}\isamarkupfalse%
\ simp\isanewline
\isacommand{next}\isamarkupfalse%
\isanewline
\ \ \isacommand{case}\isamarkupfalse%
\ {\isacharparenleft}Mon\ F{\isacharparenright}\isanewline
\ \ \isacommand{assume}\isamarkupfalse%
\ H{\isadigit{1}}{\isacharcolon}\ {\isachardoublequoteopen}G\ {\isasymin}\ setSubformulae{\isacharunderscore}s\ {\isacharparenleft}Mon\ F{\isacharparenright}{\isachardoublequoteclose}\isanewline
\ \ \isacommand{assume}\isamarkupfalse%
\ H{\isadigit{2}}{\isacharcolon}\ {\isachardoublequoteopen}G\ {\isasymin}\ setSubformulae{\isacharunderscore}s\ F\ {\isasymLongrightarrow}\ atoms{\isacharunderscore}s\ G\ {\isasymsubseteq}\ atoms{\isacharunderscore}s\ F{\isachardoublequoteclose}\isanewline
\ \ \isacommand{show}\isamarkupfalse%
\ {\isachardoublequoteopen}atoms{\isacharunderscore}s\ G\ {\isasymsubseteq}\ atoms{\isacharunderscore}s\ {\isacharparenleft}Mon\ F{\isacharparenright}{\isachardoublequoteclose}\isanewline
\ \ \isacommand{proof}\isamarkupfalse%
\ {\isacharparenleft}cases\ {\isachardoublequoteopen}G\ {\isacharequal}\ Mon\ F{\isachardoublequoteclose}{\isacharparenright}\isanewline
\ \ \ \ \isacommand{case}\isamarkupfalse%
\ True\ \isanewline
\ \ \ \ \isacommand{then}\isamarkupfalse%
\ \isacommand{have}\isamarkupfalse%
\ {\isachardoublequoteopen}G\ {\isacharequal}\ Mon\ F{\isachardoublequoteclose}\ \isacommand{by}\isamarkupfalse%
\ simp\isanewline
\ \ \ \ \isacommand{then}\isamarkupfalse%
\ \isacommand{show}\isamarkupfalse%
\ {\isachardoublequoteopen}atoms{\isacharunderscore}s\ G\ {\isasymsubseteq}\ atoms{\isacharunderscore}s\ {\isacharparenleft}Mon\ F{\isacharparenright}{\isachardoublequoteclose}\ \isacommand{by}\isamarkupfalse%
\ simp\isanewline
\ \ \isacommand{next}\isamarkupfalse%
\isanewline
\ \ \ \ \isacommand{case}\isamarkupfalse%
\ False\isanewline
\ \ \ \ \isacommand{then}\isamarkupfalse%
\ \isacommand{have}\isamarkupfalse%
\ {\isadigit{1}}{\isacharcolon}{\isachardoublequoteopen}G\ {\isasymnoteq}\ Mon\ F{\isachardoublequoteclose}\ \isacommand{by}\isamarkupfalse%
\ simp\isanewline
\ \ \ \ \isacommand{have}\isamarkupfalse%
\ {\isachardoublequoteopen}setSubformulae{\isacharunderscore}s\ {\isacharparenleft}Mon\ F{\isacharparenright}\ {\isacharequal}\ {\isacharbraceleft}Mon\ F{\isacharbraceright}\ {\isasymunion}\ setSubformulae{\isacharunderscore}s\ F{\isachardoublequoteclose}\ \isacommand{by}\isamarkupfalse%
\ simp\isanewline
\ \ \ \ \isacommand{then}\isamarkupfalse%
\ \isacommand{have}\isamarkupfalse%
\ {\isadigit{2}}{\isacharcolon}{\isachardoublequoteopen}G\ {\isasymin}\ setSubformulae{\isacharunderscore}s\ F{\isachardoublequoteclose}\ \isacommand{using}\isamarkupfalse%
\ {\isadigit{1}}\ H{\isadigit{1}}\ \isacommand{by}\isamarkupfalse%
\ simp\isanewline
\ \ \ \ \isacommand{have}\isamarkupfalse%
\ {\isachardoublequoteopen}atoms{\isacharunderscore}s\ F\ {\isacharequal}\ atoms{\isacharunderscore}s\ {\isacharparenleft}Mon\ F{\isacharparenright}{\isachardoublequoteclose}\ \isacommand{by}\isamarkupfalse%
\ simp\isanewline
\ \ \ \ \isacommand{then}\isamarkupfalse%
\ \isacommand{show}\isamarkupfalse%
\ {\isachardoublequoteopen}atoms{\isacharunderscore}s\ G\ {\isasymsubseteq}\ atoms{\isacharunderscore}s\ {\isacharparenleft}Mon\ F{\isacharparenright}{\isachardoublequoteclose}\ \isacommand{using}\isamarkupfalse%
\ {\isadigit{2}}\ H{\isadigit{2}}\ \isacommand{by}\isamarkupfalse%
\ simp\isanewline
\ \ \isacommand{qed}\isamarkupfalse%
\ \isanewline
\isacommand{next}\isamarkupfalse%
\isanewline
\ \ \isacommand{case}\isamarkupfalse%
\ {\isacharparenleft}Bi\ F{\isadigit{1}}\ F{\isadigit{2}}{\isacharparenright}\isanewline
\ \ \isacommand{assume}\isamarkupfalse%
\ H{\isadigit{3}}{\isacharcolon}\ {\isachardoublequoteopen}G\ {\isasymin}\ setSubformulae{\isacharunderscore}s\ {\isacharparenleft}Bi\ F{\isadigit{1}}\ F{\isadigit{2}}{\isacharparenright}{\isachardoublequoteclose}\isanewline
\ \ \isacommand{assume}\isamarkupfalse%
\ H{\isadigit{4}}{\isacharcolon}\ {\isachardoublequoteopen}G\ {\isasymin}\ setSubformulae{\isacharunderscore}s\ F{\isadigit{1}}\ {\isasymLongrightarrow}\ atoms{\isacharunderscore}s\ G\ {\isasymsubseteq}\ atoms{\isacharunderscore}s\ F{\isadigit{1}}{\isachardoublequoteclose}\isanewline
\ \ \isacommand{assume}\isamarkupfalse%
\ H{\isadigit{5}}{\isacharcolon}\ {\isachardoublequoteopen}G\ {\isasymin}\ setSubformulae{\isacharunderscore}s\ F{\isadigit{2}}\ {\isasymLongrightarrow}\ atoms{\isacharunderscore}s\ G\ {\isasymsubseteq}\ atoms{\isacharunderscore}s\ F{\isadigit{2}}{\isachardoublequoteclose}\isanewline
\ \ \isacommand{then}\isamarkupfalse%
\ \isacommand{show}\isamarkupfalse%
\ {\isachardoublequoteopen}atoms{\isacharunderscore}s\ G\ {\isasymsubseteq}\ atoms{\isacharunderscore}s\ {\isacharparenleft}Bi\ F{\isadigit{1}}\ F{\isadigit{2}}{\isacharparenright}{\isachardoublequoteclose}\isanewline
\ \ \isacommand{proof}\isamarkupfalse%
\ {\isacharparenleft}cases\ {\isachardoublequoteopen}G\ {\isacharequal}\ Bi\ F{\isadigit{1}}\ F{\isadigit{2}}{\isachardoublequoteclose}{\isacharparenright}\isanewline
\ \ \ \ \isacommand{case}\isamarkupfalse%
\ True\isanewline
\ \ \ \ \isacommand{then}\isamarkupfalse%
\ \isacommand{have}\isamarkupfalse%
\ {\isachardoublequoteopen}G\ {\isacharequal}\ Bi\ F{\isadigit{1}}\ F{\isadigit{2}}{\isachardoublequoteclose}\ \isacommand{by}\isamarkupfalse%
\ simp\isanewline
\ \ \ \ \isacommand{then}\isamarkupfalse%
\ \isacommand{show}\isamarkupfalse%
\ {\isachardoublequoteopen}atoms{\isacharunderscore}s\ G\ {\isasymsubseteq}\ atoms{\isacharunderscore}s\ {\isacharparenleft}Bi\ F{\isadigit{1}}\ F{\isadigit{2}}{\isacharparenright}{\isachardoublequoteclose}\ \isacommand{by}\isamarkupfalse%
\ simp\isanewline
\ \ \isacommand{next}\isamarkupfalse%
\isanewline
\ \ \ \ \isacommand{case}\isamarkupfalse%
\ False\isanewline
\ \ \ \ \isacommand{then}\isamarkupfalse%
\ \isacommand{have}\isamarkupfalse%
\ {\isadigit{3}}{\isacharcolon}{\isachardoublequoteopen}G\ {\isasymnoteq}\ Bi\ F{\isadigit{1}}\ F{\isadigit{2}}{\isachardoublequoteclose}\ \isacommand{by}\isamarkupfalse%
\ simp\isanewline
\ \ \ \ \isacommand{have}\isamarkupfalse%
\ {\isachardoublequoteopen}setSubformulae{\isacharunderscore}s\ {\isacharparenleft}Bi\ F{\isadigit{1}}\ F{\isadigit{2}}{\isacharparenright}\ {\isacharequal}\ {\isacharbraceleft}Bi\ F{\isadigit{1}}\ F{\isadigit{2}}{\isacharbraceright}\ {\isasymunion}\ setSubformulae{\isacharunderscore}s\ F{\isadigit{1}}\ {\isasymunion}\ setSubformulae{\isacharunderscore}s\ F{\isadigit{2}}{\isachardoublequoteclose}\ \isacommand{by}\isamarkupfalse%
\ simp\isanewline
\ \ \ \ \isacommand{then}\isamarkupfalse%
\ \isacommand{have}\isamarkupfalse%
\ {\isadigit{4}}{\isacharcolon}{\isachardoublequoteopen}G\ {\isasymin}\ setSubformulae{\isacharunderscore}s\ F{\isadigit{1}}\ {\isasymunion}\ setSubformulae{\isacharunderscore}s\ F{\isadigit{2}}{\isachardoublequoteclose}\ \isacommand{using}\isamarkupfalse%
\ {\isadigit{3}}\ H{\isadigit{3}}\ \isacommand{by}\isamarkupfalse%
\ simp\isanewline
\ \ \ \ \isacommand{have}\isamarkupfalse%
\ {\isadigit{5}}{\isacharcolon}{\isachardoublequoteopen}atoms{\isacharunderscore}s\ {\isacharparenleft}Bi\ F{\isadigit{1}}\ F{\isadigit{2}}{\isacharparenright}\ {\isacharequal}\ atoms{\isacharunderscore}s\ F{\isadigit{1}}\ {\isasymunion}\ atoms{\isacharunderscore}s\ F{\isadigit{2}}{\isachardoublequoteclose}\ \isacommand{by}\isamarkupfalse%
\ simp\isanewline
\ \ \ \ \isacommand{then}\isamarkupfalse%
\ \isacommand{show}\isamarkupfalse%
\ {\isachardoublequoteopen}atoms{\isacharunderscore}s\ G\ {\isasymsubseteq}\ atoms{\isacharunderscore}s\ {\isacharparenleft}Bi\ F{\isadigit{1}}\ F{\isadigit{2}}{\isacharparenright}{\isachardoublequoteclose}\isanewline
\ \ \ \ \isacommand{proof}\isamarkupfalse%
\ {\isacharparenleft}cases\ {\isachardoublequoteopen}G\ {\isasymin}\ setSubformulae{\isacharunderscore}s\ F{\isadigit{1}}{\isachardoublequoteclose}{\isacharparenright}\isanewline
\ \ \ \ \ \ \isacommand{case}\isamarkupfalse%
\ True\isanewline
\ \ \ \ \ \ \isacommand{then}\isamarkupfalse%
\ \isacommand{have}\isamarkupfalse%
\ {\isachardoublequoteopen}G\ {\isasymin}\ setSubformulae{\isacharunderscore}s\ F{\isadigit{1}}{\isachardoublequoteclose}\ \isacommand{by}\isamarkupfalse%
\ simp\isanewline
\ \ \ \ \ \ \isacommand{then}\isamarkupfalse%
\ \isacommand{have}\isamarkupfalse%
\ {\isadigit{6}}{\isacharcolon}{\isachardoublequoteopen}atoms{\isacharunderscore}s\ G\ {\isasymsubseteq}\ atoms{\isacharunderscore}s\ F{\isadigit{1}}{\isachardoublequoteclose}\ \isacommand{using}\isamarkupfalse%
\ H{\isadigit{4}}\ \isacommand{by}\isamarkupfalse%
\ simp\isanewline
\ \ \ \ \ \ \isacommand{have}\isamarkupfalse%
\ {\isadigit{7}}{\isacharcolon}{\isachardoublequoteopen}atoms{\isacharunderscore}s\ F{\isadigit{1}}\ {\isasymsubseteq}\ atoms{\isacharunderscore}s\ {\isacharparenleft}Bi\ F{\isadigit{1}}\ F{\isadigit{2}}{\isacharparenright}{\isachardoublequoteclose}\ \isacommand{using}\isamarkupfalse%
\ {\isadigit{5}}\ \isacommand{by}\isamarkupfalse%
\ {\isacharparenleft}rule\ subContUnion{\isadigit{1}}{\isacharparenright}\isanewline
\ \ \ \ \ \ \isacommand{show}\isamarkupfalse%
\ {\isachardoublequoteopen}atoms{\isacharunderscore}s\ G\ {\isasymsubseteq}\ atoms{\isacharunderscore}s\ {\isacharparenleft}Bi\ F{\isadigit{1}}\ F{\isadigit{2}}{\isacharparenright}{\isachardoublequoteclose}\ \isacommand{using}\isamarkupfalse%
\ {\isadigit{6}}\ {\isadigit{7}}\ \isacommand{by}\isamarkupfalse%
\ {\isacharparenleft}rule\ subset{\isacharunderscore}trans{\isacharparenright}\isanewline
\ \ \ \ \isacommand{next}\isamarkupfalse%
\isanewline
\ \ \ \ \ \ \isacommand{case}\isamarkupfalse%
\ False\isanewline
\ \ \ \ \ \ \isacommand{then}\isamarkupfalse%
\ \isacommand{have}\isamarkupfalse%
\ {\isachardoublequoteopen}G\ {\isasymnotin}\ setSubformulae{\isacharunderscore}s\ F{\isadigit{1}}{\isachardoublequoteclose}\ \isacommand{by}\isamarkupfalse%
\ simp\isanewline
\ \ \ \ \ \ \isacommand{then}\isamarkupfalse%
\ \isacommand{have}\isamarkupfalse%
\ {\isachardoublequoteopen}G\ {\isasymin}\ setSubformulae{\isacharunderscore}s\ F{\isadigit{2}}{\isachardoublequoteclose}\ \isacommand{using}\isamarkupfalse%
\ {\isadigit{4}}\ \isacommand{by}\isamarkupfalse%
\ simp\isanewline
\ \ \ \ \ \ \isacommand{then}\isamarkupfalse%
\ \isacommand{have}\isamarkupfalse%
\ {\isadigit{8}}{\isacharcolon}{\isachardoublequoteopen}atoms{\isacharunderscore}s\ G\ {\isasymsubseteq}\ atoms{\isacharunderscore}s\ F{\isadigit{2}}{\isachardoublequoteclose}\ \isacommand{using}\isamarkupfalse%
\ H{\isadigit{5}}\ \isacommand{by}\isamarkupfalse%
\ simp\isanewline
\ \ \ \ \ \ \isacommand{have}\isamarkupfalse%
\ {\isadigit{9}}{\isacharcolon}{\isachardoublequoteopen}atoms{\isacharunderscore}s\ F{\isadigit{2}}\ {\isasymsubseteq}\ atoms{\isacharunderscore}s\ {\isacharparenleft}Bi\ F{\isadigit{1}}\ F{\isadigit{2}}{\isacharparenright}{\isachardoublequoteclose}\ \isacommand{using}\isamarkupfalse%
\ {\isadigit{5}}\ \isacommand{by}\isamarkupfalse%
\ simp\isanewline
\ \ \ \ \ \ \isacommand{show}\isamarkupfalse%
\ {\isachardoublequoteopen}atoms{\isacharunderscore}s\ G\ {\isasymsubseteq}\ atoms{\isacharunderscore}s\ {\isacharparenleft}Bi\ F{\isadigit{1}}\ F{\isadigit{2}}{\isacharparenright}{\isachardoublequoteclose}\ \isacommand{using}\isamarkupfalse%
\ {\isadigit{8}}\ {\isadigit{9}}\ \isacommand{by}\isamarkupfalse%
\ {\isacharparenleft}rule\ subset{\isacharunderscore}trans{\isacharparenright}\isanewline
\ \ \ \ \isacommand{qed}\isamarkupfalse%
\isanewline
\ \ \isacommand{qed}\isamarkupfalse%
\isanewline
\isacommand{qed}\isamarkupfalse%
%
\endisatagproof
{\isafoldproof}%
%
\isadelimproof
%
\endisadelimproof
%
\begin{isamarkuptext}%
Por último, su demostración aplicativa y automática con la definición no simplificada.%
\end{isamarkuptext}\isamarkuptrue%
\isacommand{lemma}\isamarkupfalse%
\ subformula{\isacharunderscore}atoms{\isacharunderscore}aplicativa{\isacharcolon}\ {\isachardoublequoteopen}G\ {\isasymin}\ setSubformulae\ F\ {\isasymLongrightarrow}\ atoms\ G\ {\isasymsubseteq}\ atoms\ F{\isachardoublequoteclose}\isanewline
%
\isadelimproof
\ \ %
\endisadelimproof
%
\isatagproof
\isacommand{apply}\isamarkupfalse%
\ {\isacharparenleft}induction\ F{\isacharparenright}\isanewline
\ \ \isacommand{apply}\isamarkupfalse%
\ auto\isanewline
\ \isacommand{done}\isamarkupfalse%
%
\endisatagproof
{\isafoldproof}%
%
\isadelimproof
\isanewline
%
\endisadelimproof
\isanewline
\isacommand{lemma}\isamarkupfalse%
\ subformula{\isacharunderscore}atoms{\isacharcolon}\ {\isachardoublequoteopen}G\ {\isasymin}\ setSubformulae\ F\ {\isasymLongrightarrow}\ atoms\ G\ {\isasymsubseteq}\ atoms\ F{\isachardoublequoteclose}\isanewline
%
\isadelimproof
\ \ %
\endisadelimproof
%
\isatagproof
\isacommand{by}\isamarkupfalse%
\ {\isacharparenleft}induction\ F{\isacharparenright}\ auto%
\endisatagproof
{\isafoldproof}%
%
\isadelimproof
%
\endisadelimproof
%
\begin{isamarkuptext}%
A continuación voy a introducir un lema que no pertenece a la teoría original de Isabelle pero
facilita las siguientes demostraciones detalladas mediante contenciones en cadena.
\begin{lema}
    Sea \isa{G\ {\isasymin}\ SubfSet{\isacharparenleft}F{\isacharparenright}} entonces \isa{SubfSet{\isacharparenleft}G{\isacharparenright}\ {\isasymsubseteq}\ SubSet{\isacharparenleft}F{\isacharparenright}}.
  \end{lema} 
Para que la propiedad de contención esté bien definida, considero \isa{SubfSet{\isacharparenleft}·{\isacharparenright}} el conjunto equivalente
a \isa{setSubformulae} aplicado a una fórmula. Veamos las demostraciones estructurada simplificada
y automática del mismo.%
\end{isamarkuptext}\isamarkuptrue%
\isacommand{lemma}\isamarkupfalse%
\ subsubformulae{\isacharunderscore}estructurada{\isacharunderscore}s{\isacharcolon}\ {\isachardoublequoteopen}G\ {\isasymin}\ setSubformulae{\isacharunderscore}s\ F\ {\isasymLongrightarrow}\ setSubformulae{\isacharunderscore}s\ G\ {\isasymsubseteq}\ setSubformulae{\isacharunderscore}s\ F{\isachardoublequoteclose}\isanewline
%
\isadelimproof
%
\endisadelimproof
%
\isatagproof
\isacommand{proof}\isamarkupfalse%
\ {\isacharparenleft}induction\ F{\isacharparenright}\isanewline
\ \ \isacommand{case}\isamarkupfalse%
\ {\isacharparenleft}Atom{\isacharunderscore}s\ x{\isacharparenright}\isanewline
\ \ \isacommand{then}\isamarkupfalse%
\ \isacommand{show}\isamarkupfalse%
\ {\isacharquery}case\ \isacommand{by}\isamarkupfalse%
\ simp\isanewline
\isacommand{next}\isamarkupfalse%
\isanewline
\ \ \isacommand{case}\isamarkupfalse%
\ Bot{\isacharunderscore}s\isanewline
\ \ \isacommand{then}\isamarkupfalse%
\ \isacommand{show}\isamarkupfalse%
\ {\isacharquery}case\ \isacommand{by}\isamarkupfalse%
\ simp\isanewline
\isacommand{next}\isamarkupfalse%
\isanewline
\ \ \isacommand{case}\isamarkupfalse%
\ {\isacharparenleft}Mon\ F{\isacharparenright}\isanewline
\ \ \isacommand{assume}\isamarkupfalse%
\ H{\isadigit{1}}{\isacharcolon}{\isachardoublequoteopen}G\ {\isasymin}\ setSubformulae{\isacharunderscore}s\ F\ {\isasymLongrightarrow}\ setSubformulae{\isacharunderscore}s\ G\ {\isasymsubseteq}\ setSubformulae{\isacharunderscore}s\ F{\isachardoublequoteclose}\isanewline
\ \ \isacommand{assume}\isamarkupfalse%
\ H{\isadigit{2}}{\isacharcolon}{\isachardoublequoteopen}G\ {\isasymin}\ setSubformulae{\isacharunderscore}s\ {\isacharparenleft}Mon\ F{\isacharparenright}{\isachardoublequoteclose}\isanewline
\ \ \isacommand{then}\isamarkupfalse%
\ \isacommand{show}\isamarkupfalse%
\ {\isachardoublequoteopen}setSubformulae{\isacharunderscore}s\ G\ {\isasymsubseteq}\ setSubformulae{\isacharunderscore}s\ {\isacharparenleft}Mon\ F{\isacharparenright}{\isachardoublequoteclose}\isanewline
\ \ \isacommand{proof}\isamarkupfalse%
\ {\isacharparenleft}cases\ {\isachardoublequoteopen}G\ {\isacharequal}\ Mon\ F{\isachardoublequoteclose}{\isacharparenright}\isanewline
\ \ \ \ \isacommand{case}\isamarkupfalse%
\ True\isanewline
\ \ \ \ \isacommand{then}\isamarkupfalse%
\ \isacommand{show}\isamarkupfalse%
\ {\isacharquery}thesis\ \isacommand{by}\isamarkupfalse%
\ simp\isanewline
\ \ \isacommand{next}\isamarkupfalse%
\isanewline
\ \ \ \ \isacommand{case}\isamarkupfalse%
\ False\isanewline
\ \ \ \ \isacommand{then}\isamarkupfalse%
\ \isacommand{have}\isamarkupfalse%
\ {\isachardoublequoteopen}G\ {\isasymnoteq}\ Mon\ F{\isachardoublequoteclose}\ \isacommand{by}\isamarkupfalse%
\ simp\isanewline
\ \ \ \ \isacommand{then}\isamarkupfalse%
\ \isacommand{have}\isamarkupfalse%
\ {\isachardoublequoteopen}G\ {\isasymin}\ setSubformulae{\isacharunderscore}s\ F{\isachardoublequoteclose}\ \isacommand{using}\isamarkupfalse%
\ H{\isadigit{2}}\ \isacommand{by}\isamarkupfalse%
\ simp\isanewline
\ \ \ \ \isacommand{then}\isamarkupfalse%
\ \isacommand{have}\isamarkupfalse%
\ {\isadigit{1}}{\isacharcolon}{\isachardoublequoteopen}setSubformulae{\isacharunderscore}s\ G\ {\isasymsubseteq}\ setSubformulae{\isacharunderscore}s\ F{\isachardoublequoteclose}\ \isacommand{using}\isamarkupfalse%
\ H{\isadigit{1}}\ \isacommand{by}\isamarkupfalse%
\ simp\isanewline
\ \ \ \ \isacommand{have}\isamarkupfalse%
\ {\isachardoublequoteopen}setSubformulae{\isacharunderscore}s\ {\isacharparenleft}Mon\ F{\isacharparenright}\ {\isacharequal}\ {\isacharbraceleft}Mon\ F{\isacharbraceright}\ {\isasymunion}\ setSubformulae{\isacharunderscore}s\ F{\isachardoublequoteclose}\ \isacommand{by}\isamarkupfalse%
\ simp\isanewline
\ \ \ \ \isacommand{then}\isamarkupfalse%
\ \isacommand{have}\isamarkupfalse%
\ {\isadigit{2}}{\isacharcolon}{\isachardoublequoteopen}setSubformulae{\isacharunderscore}s\ F\ {\isasymsubseteq}\ setSubformulae{\isacharunderscore}s\ {\isacharparenleft}Mon\ F{\isacharparenright}{\isachardoublequoteclose}\ \isacommand{by}\isamarkupfalse%
\ {\isacharparenleft}rule\ subContUnion{\isadigit{2}}{\isacharparenright}\isanewline
\ \ \ \ \isacommand{show}\isamarkupfalse%
\ {\isachardoublequoteopen}setSubformulae{\isacharunderscore}s\ G\ {\isasymsubseteq}\ setSubformulae{\isacharunderscore}s\ {\isacharparenleft}Mon\ F{\isacharparenright}{\isachardoublequoteclose}\ \isacommand{using}\isamarkupfalse%
\ {\isadigit{1}}\ {\isadigit{2}}\ \isacommand{by}\isamarkupfalse%
\ {\isacharparenleft}rule\ subset{\isacharunderscore}trans{\isacharparenright}\isanewline
\ \ \isacommand{qed}\isamarkupfalse%
\isanewline
\isacommand{next}\isamarkupfalse%
\isanewline
\ \ \isacommand{case}\isamarkupfalse%
\ {\isacharparenleft}Bi\ F{\isadigit{1}}\ F{\isadigit{2}}{\isacharparenright}\isanewline
\ \ \isacommand{assume}\isamarkupfalse%
\ H{\isadigit{3}}{\isacharcolon}{\isachardoublequoteopen}G\ {\isasymin}\ setSubformulae{\isacharunderscore}s\ F{\isadigit{1}}\ {\isasymLongrightarrow}\ setSubformulae{\isacharunderscore}s\ G\ {\isasymsubseteq}\ setSubformulae{\isacharunderscore}s\ F{\isadigit{1}}{\isachardoublequoteclose}\isanewline
\ \ \isacommand{assume}\isamarkupfalse%
\ H{\isadigit{4}}{\isacharcolon}{\isachardoublequoteopen}G\ {\isasymin}\ setSubformulae{\isacharunderscore}s\ F{\isadigit{2}}\ {\isasymLongrightarrow}\ setSubformulae{\isacharunderscore}s\ G\ {\isasymsubseteq}\ setSubformulae{\isacharunderscore}s\ F{\isadigit{2}}{\isachardoublequoteclose}\isanewline
\ \ \isacommand{assume}\isamarkupfalse%
\ H{\isadigit{5}}{\isacharcolon}{\isachardoublequoteopen}G\ {\isasymin}\ setSubformulae{\isacharunderscore}s\ {\isacharparenleft}Bi\ F{\isadigit{1}}\ F{\isadigit{2}}{\isacharparenright}{\isachardoublequoteclose}\isanewline
\ \ \isacommand{have}\isamarkupfalse%
\ {\isadigit{4}}{\isacharcolon}{\isachardoublequoteopen}setSubformulae{\isacharunderscore}s\ {\isacharparenleft}Bi\ F{\isadigit{1}}\ F{\isadigit{2}}{\isacharparenright}\ {\isacharequal}\ {\isacharbraceleft}Bi\ F{\isadigit{1}}\ F{\isadigit{2}}{\isacharbraceright}\ {\isasymunion}\ {\isacharparenleft}setSubformulae{\isacharunderscore}s\ F{\isadigit{1}}\ {\isasymunion}\ setSubformulae{\isacharunderscore}s\ F{\isadigit{2}}{\isacharparenright}{\isachardoublequoteclose}\ \isacommand{by}\isamarkupfalse%
\ simp\isanewline
\ \ \isacommand{then}\isamarkupfalse%
\ \isacommand{show}\isamarkupfalse%
\ {\isachardoublequoteopen}setSubformulae{\isacharunderscore}s\ G\ {\isasymsubseteq}\ setSubformulae{\isacharunderscore}s\ {\isacharparenleft}Bi\ F{\isadigit{1}}\ F{\isadigit{2}}{\isacharparenright}{\isachardoublequoteclose}\isanewline
\ \ \isacommand{proof}\isamarkupfalse%
\ {\isacharparenleft}cases\ {\isachardoublequoteopen}G\ {\isacharequal}\ Bi\ F{\isadigit{1}}\ F{\isadigit{2}}{\isachardoublequoteclose}{\isacharparenright}\isanewline
\ \ \ \ \isacommand{case}\isamarkupfalse%
\ True\isanewline
\ \ \ \ \isacommand{then}\isamarkupfalse%
\ \isacommand{show}\isamarkupfalse%
\ {\isacharquery}thesis\ \isacommand{by}\isamarkupfalse%
\ simp\isanewline
\ \ \isacommand{next}\isamarkupfalse%
\isanewline
\ \ \ \ \isacommand{case}\isamarkupfalse%
\ False\isanewline
\ \ \ \ \isacommand{then}\isamarkupfalse%
\ \isacommand{have}\isamarkupfalse%
\ {\isadigit{5}}{\isacharcolon}{\isachardoublequoteopen}G\ {\isasymnoteq}\ Bi\ F{\isadigit{1}}\ F{\isadigit{2}}{\isachardoublequoteclose}\ \isacommand{by}\isamarkupfalse%
\ simp\isanewline
\ \ \ \ \isacommand{have}\isamarkupfalse%
\ {\isadigit{6}}{\isacharcolon}{\isachardoublequoteopen}setSubformulae{\isacharunderscore}s\ F{\isadigit{1}}\ {\isasymunion}\ setSubformulae{\isacharunderscore}s\ F{\isadigit{2}}\ {\isasymsubseteq}\ setSubformulae{\isacharunderscore}s\ {\isacharparenleft}Bi\ F{\isadigit{1}}\ F{\isadigit{2}}{\isacharparenright}{\isachardoublequoteclose}\ \isacommand{using}\isamarkupfalse%
\ {\isadigit{4}}\ \isacommand{by}\isamarkupfalse%
\ {\isacharparenleft}rule\ subContUnion{\isadigit{2}}{\isacharparenright}\isanewline
\ \ \ \ \isacommand{then}\isamarkupfalse%
\ \isacommand{show}\isamarkupfalse%
\ {\isachardoublequoteopen}setSubformulae{\isacharunderscore}s\ G\ {\isasymsubseteq}\ setSubformulae{\isacharunderscore}s\ {\isacharparenleft}Bi\ F{\isadigit{1}}\ F{\isadigit{2}}{\isacharparenright}{\isachardoublequoteclose}\isanewline
\ \ \ \ \isacommand{proof}\isamarkupfalse%
\ {\isacharparenleft}cases\ {\isachardoublequoteopen}G\ {\isasymin}\ setSubformulae{\isacharunderscore}s\ F{\isadigit{1}}{\isachardoublequoteclose}{\isacharparenright}\isanewline
\ \ \ \ \ \ \isacommand{case}\isamarkupfalse%
\ True\isanewline
\ \ \ \ \ \ \isacommand{then}\isamarkupfalse%
\ \isacommand{have}\isamarkupfalse%
\ {\isachardoublequoteopen}G\ {\isasymin}\ setSubformulae{\isacharunderscore}s\ F{\isadigit{1}}{\isachardoublequoteclose}\ \isacommand{by}\isamarkupfalse%
\ simp\isanewline
\ \ \ \ \ \ \isacommand{then}\isamarkupfalse%
\ \isacommand{have}\isamarkupfalse%
\ {\isadigit{7}}{\isacharcolon}{\isachardoublequoteopen}setSubformulae{\isacharunderscore}s\ G\ {\isasymsubseteq}\ setSubformulae{\isacharunderscore}s\ F{\isadigit{1}}{\isachardoublequoteclose}\ \isacommand{using}\isamarkupfalse%
\ H{\isadigit{3}}\ \isacommand{by}\isamarkupfalse%
\ simp\isanewline
\ \ \ \ \ \ \isacommand{have}\isamarkupfalse%
\ {\isadigit{8}}{\isacharcolon}{\isachardoublequoteopen}setSubformulae{\isacharunderscore}s\ F{\isadigit{1}}\ {\isasymsubseteq}\ setSubformulae{\isacharunderscore}s\ {\isacharparenleft}Bi\ F{\isadigit{1}}\ F{\isadigit{2}}{\isacharparenright}{\isachardoublequoteclose}\ \isacommand{using}\isamarkupfalse%
\ {\isadigit{6}}\ \isacommand{by}\isamarkupfalse%
\ {\isacharparenleft}rule\ subContUnionRev{\isadigit{1}}{\isacharparenright}\ \ \isanewline
\ \ \ \ \ \ \isacommand{show}\isamarkupfalse%
\ {\isachardoublequoteopen}setSubformulae{\isacharunderscore}s\ G\ {\isasymsubseteq}\ setSubformulae{\isacharunderscore}s\ {\isacharparenleft}Bi\ F{\isadigit{1}}\ F{\isadigit{2}}{\isacharparenright}{\isachardoublequoteclose}\ \isacommand{using}\isamarkupfalse%
\ {\isadigit{7}}\ {\isadigit{8}}\ \isacommand{by}\isamarkupfalse%
\ {\isacharparenleft}rule\ subset{\isacharunderscore}trans{\isacharparenright}\isanewline
\ \ \ \ \isacommand{next}\isamarkupfalse%
\isanewline
\ \ \ \ \ \ \isacommand{case}\isamarkupfalse%
\ False\isanewline
\ \ \ \ \ \ \isacommand{then}\isamarkupfalse%
\ \isacommand{have}\isamarkupfalse%
\ {\isadigit{9}}{\isacharcolon}{\isachardoublequoteopen}G\ {\isasymnotin}\ setSubformulae{\isacharunderscore}s\ F{\isadigit{1}}{\isachardoublequoteclose}\ \isacommand{by}\isamarkupfalse%
\ simp\isanewline
\ \ \ \ \ \ \isacommand{have}\isamarkupfalse%
\ {\isachardoublequoteopen}G\ {\isasymin}\ setSubformulae{\isacharunderscore}s\ F{\isadigit{1}}\ {\isasymunion}\ setSubformulae{\isacharunderscore}s\ F{\isadigit{2}}{\isachardoublequoteclose}\ \isacommand{using}\isamarkupfalse%
\ {\isadigit{5}}\ H{\isadigit{5}}\ \isacommand{by}\isamarkupfalse%
\ simp\isanewline
\ \ \ \ \ \ \isacommand{then}\isamarkupfalse%
\ \isacommand{have}\isamarkupfalse%
\ {\isachardoublequoteopen}G\ {\isasymin}\ setSubformulae{\isacharunderscore}s\ F{\isadigit{2}}{\isachardoublequoteclose}\ \isacommand{using}\isamarkupfalse%
\ {\isadigit{9}}\ \isacommand{by}\isamarkupfalse%
\ simp\isanewline
\ \ \ \ \ \ \isacommand{then}\isamarkupfalse%
\ \isacommand{have}\isamarkupfalse%
\ {\isadigit{1}}{\isadigit{0}}{\isacharcolon}{\isachardoublequoteopen}setSubformulae{\isacharunderscore}s\ G\ {\isasymsubseteq}\ setSubformulae{\isacharunderscore}s\ F{\isadigit{2}}{\isachardoublequoteclose}\ \isacommand{using}\isamarkupfalse%
\ H{\isadigit{4}}\ \isacommand{by}\isamarkupfalse%
\ simp\isanewline
\ \ \ \ \ \ \isacommand{have}\isamarkupfalse%
\ {\isadigit{1}}{\isadigit{1}}{\isacharcolon}{\isachardoublequoteopen}setSubformulae{\isacharunderscore}s\ F{\isadigit{2}}\ {\isasymsubseteq}\ setSubformulae{\isacharunderscore}s\ {\isacharparenleft}Bi\ F{\isadigit{1}}\ F{\isadigit{2}}{\isacharparenright}{\isachardoublequoteclose}\ \isacommand{using}\isamarkupfalse%
\ {\isadigit{6}}\ \isacommand{by}\isamarkupfalse%
\ simp\isanewline
\ \ \ \ \ \ \isacommand{show}\isamarkupfalse%
\ {\isachardoublequoteopen}setSubformulae{\isacharunderscore}s\ G\ {\isasymsubseteq}\ setSubformulae{\isacharunderscore}s\ {\isacharparenleft}Bi\ F{\isadigit{1}}\ F{\isadigit{2}}{\isacharparenright}{\isachardoublequoteclose}\ \isacommand{using}\isamarkupfalse%
\ {\isadigit{1}}{\isadigit{0}}\ {\isadigit{1}}{\isadigit{1}}\ \isacommand{by}\isamarkupfalse%
\ {\isacharparenleft}rule\ subset{\isacharunderscore}trans{\isacharparenright}\isanewline
\ \ \ \ \isacommand{qed}\isamarkupfalse%
\isanewline
\ \ \isacommand{qed}\isamarkupfalse%
\isanewline
\isacommand{qed}\isamarkupfalse%
%
\endisatagproof
{\isafoldproof}%
%
\isadelimproof
\isanewline
%
\endisadelimproof
\isanewline
\isanewline
\isanewline
\isacommand{lemma}\isamarkupfalse%
\ subformulae{\isacharunderscore}setSubformulae{\isacharcolon}{\isachardoublequoteopen}G\ {\isasymin}\ setSubformulae\ F\ {\isasymLongrightarrow}\ setSubformulae\ G\ {\isasymsubseteq}\ setSubformulae\ F{\isachardoublequoteclose}\isanewline
%
\isadelimproof
\ \ %
\endisadelimproof
%
\isatagproof
\isacommand{by}\isamarkupfalse%
\ {\isacharparenleft}induction\ F{\isacharparenright}\ auto%
\endisatagproof
{\isafoldproof}%
%
\isadelimproof
%
\endisadelimproof
%
\begin{isamarkuptext}%
El siguiente lema nos da la noción de transitividad de contención en cadena de las subfórmulas, de modo que la
subfórmula de una subfórmula es del mismo modo subfórmula de la mayor. Es un resultado sencillo
derivado de la estructura de árbol de formación que siguen las fórmulas según las hemos definido.

\begin{lema}
    Sea \isa{G\ {\isasymin}\ Subf{\isacharparenleft}F{\isacharparenright}} y \isa{H\ {\isasymin}\ Subf{\isacharparenleft}G{\isacharparenright}}, entonces \isa{H\ {\isasymin}\ Subf{\isacharparenleft}F{\isacharparenright}}.
  \end{lema}

La demostración estructurada se hace de manera sencilla con el resultado introducido anteriormente. 
Veamos su prueba según las distintas tácticas.%
\end{isamarkuptext}\isamarkuptrue%
\isacommand{lemma}\isamarkupfalse%
\ subsubformulae{\isacharunderscore}estruct{\isacharcolon}\isanewline
\ \ \isakeyword{assumes}\ {\isachardoublequoteopen}G\ {\isasymin}\ setSubformulae\ F{\isachardoublequoteclose}\ \isanewline
\ \ \ \ \ \ \ \ \ \ {\isachardoublequoteopen}H\ {\isasymin}\ setSubformulae\ G{\isachardoublequoteclose}\isanewline
\ \ \ \ \isakeyword{shows}\ {\isachardoublequoteopen}H\ {\isasymin}\ setSubformulae\ F{\isachardoublequoteclose}\isanewline
%
\isadelimproof
%
\endisadelimproof
%
\isatagproof
\isacommand{proof}\isamarkupfalse%
\ {\isacharminus}\isanewline
\ \ \isacommand{have}\isamarkupfalse%
\ {\isadigit{1}}{\isacharcolon}{\isachardoublequoteopen}setSubformulae\ G\ {\isasymsubseteq}\ setSubformulae\ F{\isachardoublequoteclose}\ \isacommand{using}\isamarkupfalse%
\ assms{\isacharparenleft}{\isadigit{1}}{\isacharparenright}\ \isacommand{by}\isamarkupfalse%
\ {\isacharparenleft}rule\ subformulae{\isacharunderscore}setSubformulae{\isacharparenright}\isanewline
\ \ \isacommand{have}\isamarkupfalse%
\ {\isachardoublequoteopen}setSubformulae\ H\ {\isasymsubseteq}\ setSubformulae\ G{\isachardoublequoteclose}\ \isacommand{using}\isamarkupfalse%
\ assms{\isacharparenleft}{\isadigit{2}}{\isacharparenright}\ \isacommand{by}\isamarkupfalse%
\ {\isacharparenleft}rule\ subformulae{\isacharunderscore}setSubformulae{\isacharparenright}\isanewline
\ \ \isacommand{then}\isamarkupfalse%
\ \isacommand{have}\isamarkupfalse%
\ {\isadigit{2}}{\isacharcolon}{\isachardoublequoteopen}setSubformulae\ H\ {\isasymsubseteq}\ setSubformulae\ F{\isachardoublequoteclose}\ \isacommand{using}\isamarkupfalse%
\ {\isadigit{1}}\ \isacommand{by}\isamarkupfalse%
\ {\isacharparenleft}rule\ subset{\isacharunderscore}trans{\isacharparenright}\isanewline
\ \ \isacommand{have}\isamarkupfalse%
\ {\isachardoublequoteopen}H\ {\isasymin}\ setSubformulae\ H{\isachardoublequoteclose}\ \isacommand{by}\isamarkupfalse%
\ simp\isanewline
\ \ \isacommand{then}\isamarkupfalse%
\ \isacommand{show}\isamarkupfalse%
\ {\isachardoublequoteopen}H\ {\isasymin}\ setSubformulae\ F{\isachardoublequoteclose}\ \isacommand{using}\isamarkupfalse%
\ {\isadigit{2}}\ \isacommand{by}\isamarkupfalse%
\ {\isacharparenleft}rule\ rev{\isacharunderscore}subsetD{\isacharparenright}\isanewline
\isacommand{qed}\isamarkupfalse%
%
\endisatagproof
{\isafoldproof}%
%
\isadelimproof
\isanewline
%
\endisadelimproof
\isanewline
\isacommand{lemma}\isamarkupfalse%
\ subsubformulae{\isacharunderscore}aplic{\isacharcolon}\ {\isachardoublequoteopen}G\ {\isasymin}\ setSubformulae\ F\ {\isasymLongrightarrow}\ H\ {\isasymin}\ setSubformulae\ G\ {\isasymLongrightarrow}\ H\ {\isasymin}\ setSubformulae\ F{\isachardoublequoteclose}\isanewline
%
\isadelimproof
\ \ %
\endisadelimproof
%
\isatagproof
\isacommand{oops}\isamarkupfalse%
%
\endisatagproof
{\isafoldproof}%
%
\isadelimproof
\isanewline
%
\endisadelimproof
\ \ \isanewline
\isacommand{lemma}\isamarkupfalse%
\ subsubformulae{\isacharcolon}\ {\isachardoublequoteopen}G\ {\isasymin}\ setSubformulae\ F\ {\isasymLongrightarrow}\ H\ {\isasymin}\ setSubformulae\ G\ {\isasymLongrightarrow}\ H\ {\isasymin}\ setSubformulae\ F{\isachardoublequoteclose}\isanewline
%
\isadelimproof
\ \ %
\endisadelimproof
%
\isatagproof
\isacommand{by}\isamarkupfalse%
\ {\isacharparenleft}induction\ F{\isacharsemicolon}\ force{\isacharparenright}%
\endisatagproof
{\isafoldproof}%
%
\isadelimproof
%
\endisadelimproof
%
\begin{isamarkuptext}%
A continuación, la versión del lema con definiciones simplificadas, pues será utilizada para
la siguiente prueba.%
\end{isamarkuptext}\isamarkuptrue%
\isacommand{lemma}\isamarkupfalse%
\ subsubformulae{\isacharunderscore}s{\isacharcolon}\ {\isachardoublequoteopen}G\ {\isasymin}\ setSubformulae{\isacharunderscore}s\ F\ {\isasymLongrightarrow}\ H\ {\isasymin}\ setSubformulae{\isacharunderscore}s\ G\ {\isasymLongrightarrow}\ H\ {\isasymin}\ setSubformulae{\isacharunderscore}s\ F{\isachardoublequoteclose}\isanewline
%
\isadelimproof
\ \ %
\endisadelimproof
%
\isatagproof
\isacommand{by}\isamarkupfalse%
\ {\isacharparenleft}induction\ F{\isacharsemicolon}\ force{\isacharparenright}%
\endisatagproof
{\isafoldproof}%
%
\isadelimproof
%
\endisadelimproof
%
\begin{isamarkuptext}%
A continuación presentamos otro resultado que relaciona las subfórmulas de una fórmula según 
las conectivas con operaciones sobre los conjuntos de subfórmulas de cada parte.%
\end{isamarkuptext}\isamarkuptrue%
\isacommand{lemma}\isamarkupfalse%
\ subformulas{\isacharunderscore}in{\isacharunderscore}subformulas{\isacharcolon}\isanewline
\ \ {\isachardoublequoteopen}G\ \isactrlbold {\isasymand}\ H\ {\isasymin}\ setSubformulae\ F\ {\isasymLongrightarrow}\ G\ {\isasymin}\ setSubformulae\ F\ {\isasymand}\ H\ {\isasymin}\ setSubformulae\ F{\isachardoublequoteclose}\isanewline
\ \ {\isachardoublequoteopen}G\ \isactrlbold {\isasymor}\ H\ {\isasymin}\ setSubformulae\ F\ {\isasymLongrightarrow}\ G\ {\isasymin}\ setSubformulae\ F\ {\isasymand}\ H\ {\isasymin}\ setSubformulae\ F{\isachardoublequoteclose}\isanewline
\ \ {\isachardoublequoteopen}G\ \isactrlbold {\isasymrightarrow}\ H\ {\isasymin}\ setSubformulae\ F\ {\isasymLongrightarrow}\ G\ {\isasymin}\ setSubformulae\ F\ {\isasymand}\ H\ {\isasymin}\ setSubformulae\ F{\isachardoublequoteclose}\isanewline
\ \ {\isachardoublequoteopen}\isactrlbold {\isasymnot}\ G\ {\isasymin}\ setSubformulae\ F\ {\isasymLongrightarrow}\ G\ {\isasymin}\ setSubformulae\ F{\isachardoublequoteclose}\isanewline
%
\isadelimproof
\ \ %
\endisadelimproof
%
\isatagproof
\isacommand{oops}\isamarkupfalse%
%
\endisatagproof
{\isafoldproof}%
%
\isadelimproof
%
\endisadelimproof
%
\begin{isamarkuptext}%
Como podemos observar, el resultado es análogo en todas las conectivas binarias aunque
aparezcan definidas por separado, por tanto haré la demostración estructurada para
las definiciones simplificadas. Nos basaremos en el lema anterior \isa{subsubformulae}.%
\end{isamarkuptext}\isamarkuptrue%
\isacommand{lemma}\isamarkupfalse%
\ subformulas{\isacharunderscore}in{\isacharunderscore}subformulas{\isacharunderscore}estructurada{\isadigit{1}}{\isacharunderscore}s{\isacharcolon}\isanewline
\ \ \isakeyword{assumes}\ {\isachardoublequoteopen}Bi\ G\ H\ {\isasymin}\ setSubformulae{\isacharunderscore}s\ F{\isachardoublequoteclose}\ \isanewline
\ \ \isakeyword{shows}\ {\isachardoublequoteopen}G\ {\isasymin}\ setSubformulae{\isacharunderscore}s\ F\ {\isasymand}\ H\ {\isasymin}\ setSubformulae{\isacharunderscore}s\ F{\isachardoublequoteclose}\isanewline
%
\isadelimproof
%
\endisadelimproof
%
\isatagproof
\isacommand{proof}\isamarkupfalse%
\ {\isacharparenleft}rule\ conjI{\isacharparenright}\isanewline
\ \ \isacommand{have}\isamarkupfalse%
\ {\isadigit{1}}{\isacharcolon}{\isachardoublequoteopen}setSubformulae{\isacharunderscore}s\ {\isacharparenleft}Bi\ G\ H{\isacharparenright}\ {\isacharequal}\ {\isacharbraceleft}Bi\ G\ H{\isacharbraceright}\ {\isasymunion}\ setSubformulae{\isacharunderscore}s\ G\ {\isasymunion}\ setSubformulae{\isacharunderscore}s\ H{\isachardoublequoteclose}\ \isacommand{by}\isamarkupfalse%
\ simp\isanewline
\ \ \isacommand{then}\isamarkupfalse%
\ \isacommand{have}\isamarkupfalse%
\ {\isadigit{2}}{\isacharcolon}{\isachardoublequoteopen}G\ {\isasymin}\ setSubformulae{\isacharunderscore}s\ {\isacharparenleft}Bi\ G\ H{\isacharparenright}{\isachardoublequoteclose}\ \isacommand{by}\isamarkupfalse%
\ simp\isanewline
\ \ \isacommand{have}\isamarkupfalse%
\ {\isadigit{3}}{\isacharcolon}{\isachardoublequoteopen}H\ {\isasymin}\ setSubformulae{\isacharunderscore}s\ {\isacharparenleft}Bi\ G\ H{\isacharparenright}{\isachardoublequoteclose}\ \isacommand{using}\isamarkupfalse%
\ {\isadigit{1}}\ \isacommand{by}\isamarkupfalse%
\ simp\isanewline
\ \ \isacommand{show}\isamarkupfalse%
\ {\isachardoublequoteopen}G\ {\isasymin}\ setSubformulae{\isacharunderscore}s\ F{\isachardoublequoteclose}\ \isacommand{using}\isamarkupfalse%
\ assms\ {\isadigit{2}}\ \isacommand{by}\isamarkupfalse%
\ {\isacharparenleft}rule\ subsubformulae{\isacharunderscore}s{\isacharparenright}\isanewline
\ \ \isacommand{show}\isamarkupfalse%
\ {\isachardoublequoteopen}H\ {\isasymin}\ setSubformulae{\isacharunderscore}s\ F{\isachardoublequoteclose}\ \isacommand{using}\isamarkupfalse%
\ assms\ {\isadigit{3}}\ \isacommand{by}\isamarkupfalse%
\ {\isacharparenleft}rule\ subsubformulae{\isacharunderscore}s{\isacharparenright}\isanewline
\isacommand{qed}\isamarkupfalse%
%
\endisatagproof
{\isafoldproof}%
%
\isadelimproof
\isanewline
%
\endisadelimproof
\isanewline
\isacommand{lemma}\isamarkupfalse%
\ subformulas{\isacharunderscore}in{\isacharunderscore}subformulas{\isacharunderscore}negacion{\isacharunderscore}estructurada{\isacharcolon}\isanewline
\ \ \isakeyword{assumes}\ {\isachardoublequoteopen}Mon\ G\ {\isasymin}\ setSubformulae{\isacharunderscore}s\ F{\isachardoublequoteclose}\isanewline
\ \ \isakeyword{shows}\ {\isachardoublequoteopen}G\ {\isasymin}\ setSubformulae{\isacharunderscore}s\ F{\isachardoublequoteclose}\isanewline
%
\isadelimproof
%
\endisadelimproof
%
\isatagproof
\isacommand{proof}\isamarkupfalse%
\ {\isacharminus}\isanewline
\ \ \isacommand{have}\isamarkupfalse%
\ {\isachardoublequoteopen}setSubformulae{\isacharunderscore}s\ {\isacharparenleft}Mon\ G{\isacharparenright}\ {\isacharequal}\ {\isacharbraceleft}Mon\ G{\isacharbraceright}\ {\isasymunion}\ setSubformulae{\isacharunderscore}s\ G{\isachardoublequoteclose}\ \isacommand{by}\isamarkupfalse%
\ simp\ \isanewline
\ \ \isacommand{then}\isamarkupfalse%
\ \isacommand{have}\isamarkupfalse%
\ {\isadigit{1}}{\isacharcolon}{\isachardoublequoteopen}G\ {\isasymin}\ setSubformulae{\isacharunderscore}s\ {\isacharparenleft}Mon\ G{\isacharparenright}{\isachardoublequoteclose}\ \isacommand{by}\isamarkupfalse%
\ simp\isanewline
\ \ \isacommand{show}\isamarkupfalse%
\ {\isachardoublequoteopen}G\ {\isasymin}\ setSubformulae{\isacharunderscore}s\ F{\isachardoublequoteclose}\ \isacommand{using}\isamarkupfalse%
\ assms\ {\isadigit{1}}\ \isacommand{by}\isamarkupfalse%
\ {\isacharparenleft}rule\ subsubformulae{\isacharunderscore}s{\isacharparenright}\isanewline
\isacommand{qed}\isamarkupfalse%
%
\endisatagproof
{\isafoldproof}%
%
\isadelimproof
%
\endisadelimproof
%
\begin{isamarkuptext}%
Mostremos ahora la demostración aplicativa para estos casos y la automática para el lema 
completo.%
\end{isamarkuptext}\isamarkuptrue%
\isacommand{lemma}\isamarkupfalse%
\ subformulas{\isacharunderscore}in{\isacharunderscore}subformulas{\isacharunderscore}aplicativa{\isacharunderscore}s{\isacharcolon}\isanewline
\ \ {\isachardoublequoteopen}Bi\ G\ H\ {\isasymin}\ setSubformulae{\isacharunderscore}s\ F\ {\isasymLongrightarrow}\ G\ {\isasymin}\ setSubformulae{\isacharunderscore}s\ F\ {\isasymand}\ H\ {\isasymin}\ setSubformulae{\isacharunderscore}s\ F{\isachardoublequoteclose}\isanewline
\ \ {\isachardoublequoteopen}Mon\ G\ {\isasymin}\ setSubformulae{\isacharunderscore}s\ F\ {\isasymLongrightarrow}\ G\ {\isasymin}\ setSubformulae{\isacharunderscore}s\ F{\isachardoublequoteclose}\isanewline
%
\isadelimproof
\ \ \ %
\endisadelimproof
%
\isatagproof
\isacommand{apply}\isamarkupfalse%
\ {\isacharparenleft}rule\ conjI{\isacharparenright}\isanewline
\ \ \ \isacommand{apply}\isamarkupfalse%
\ {\isacharparenleft}erule\ subsubformulae{\isacharunderscore}s{\isacharcomma}simp{\isacharparenright}{\isacharplus}\isanewline
\ \ \isacommand{done}\isamarkupfalse%
%
\endisatagproof
{\isafoldproof}%
%
\isadelimproof
\isanewline
%
\endisadelimproof
\isanewline
\isacommand{lemma}\isamarkupfalse%
\ subformulas{\isacharunderscore}in{\isacharunderscore}subformulas{\isacharcolon}\isanewline
\ \ {\isachardoublequoteopen}G\ \isactrlbold {\isasymand}\ H\ {\isasymin}\ setSubformulae\ F\ {\isasymLongrightarrow}\ G\ {\isasymin}\ setSubformulae\ F\ {\isasymand}\ H\ {\isasymin}\ setSubformulae\ F{\isachardoublequoteclose}\isanewline
\ \ {\isachardoublequoteopen}G\ \isactrlbold {\isasymor}\ H\ {\isasymin}\ setSubformulae\ F\ {\isasymLongrightarrow}\ G\ {\isasymin}\ setSubformulae\ F\ {\isasymand}\ H\ {\isasymin}\ setSubformulae\ F{\isachardoublequoteclose}\isanewline
\ \ {\isachardoublequoteopen}G\ \isactrlbold {\isasymrightarrow}\ H\ {\isasymin}\ setSubformulae\ F\ {\isasymLongrightarrow}\ G\ {\isasymin}\ setSubformulae\ F\ {\isasymand}\ H\ {\isasymin}\ setSubformulae\ F{\isachardoublequoteclose}\isanewline
\ \ {\isachardoublequoteopen}\isactrlbold {\isasymnot}\ G\ {\isasymin}\ setSubformulae\ F\ {\isasymLongrightarrow}\ G\ {\isasymin}\ setSubformulae\ F{\isachardoublequoteclose}\isanewline
%
\isadelimproof
\ \ %
\endisadelimproof
%
\isatagproof
\isacommand{by}\isamarkupfalse%
\ {\isacharparenleft}fastforce\ elim{\isacharcolon}\ subsubformulae{\isacharparenright}{\isacharplus}%
\endisatagproof
{\isafoldproof}%
%
\isadelimproof
%
\endisadelimproof
%
\begin{isamarkuptext}%
Concluimos la sección de subfórmulas con un resultado que relaciona varias funciones
sobre la longitud de la lista \isa{subformulae\ F} de una fórmula \isa{F} cualquiera.%
\end{isamarkuptext}\isamarkuptrue%
\isacommand{lemma}\isamarkupfalse%
\ length{\isacharunderscore}subformulae{\isacharcolon}\ {\isachardoublequoteopen}length\ {\isacharparenleft}subformulae\ F{\isacharparenright}\ {\isacharequal}\ size\ F{\isachardoublequoteclose}\ \isanewline
%
\isadelimproof
\ \ %
\endisadelimproof
%
\isatagproof
\isacommand{oops}\isamarkupfalse%
%
\endisatagproof
{\isafoldproof}%
%
\isadelimproof
%
\endisadelimproof
%
\begin{isamarkuptext}%
En prime lugar aparece la clase \isa{size} de la teoría de números naturales ....
Vamos a definir \isa{size{\isadigit{1}}} de manera idéntica a como aparace \isa{size} en la teoría.%
\end{isamarkuptext}\isamarkuptrue%
\isacommand{class}\isamarkupfalse%
\ size{\isadigit{1}}\ {\isacharequal}\isanewline
\ \ \isakeyword{fixes}\ size{\isadigit{1}}\ {\isacharcolon}{\isacharcolon}\ {\isachardoublequoteopen}{\isacharprime}a\ {\isasymRightarrow}\ nat{\isachardoublequoteclose}\ \isanewline
\isanewline
\isacommand{instantiation}\isamarkupfalse%
\ nat\ {\isacharcolon}{\isacharcolon}\ size{\isadigit{1}}\isanewline
\isakeyword{begin}\isanewline
\isanewline
\isacommand{definition}\isamarkupfalse%
\ size{\isadigit{1}}{\isacharunderscore}nat\ \isakeyword{where}\ {\isacharbrackleft}simp{\isacharcomma}\ code{\isacharbrackright}{\isacharcolon}\ {\isachardoublequoteopen}size{\isadigit{1}}\ {\isacharparenleft}n{\isacharcolon}{\isacharcolon}nat{\isacharparenright}\ {\isacharequal}\ n{\isachardoublequoteclose}\isanewline
\isanewline
\isacommand{instance}\isamarkupfalse%
%
\isadelimproof
\ %
\endisadelimproof
%
\isatagproof
\isacommand{{\isachardot}{\isachardot}}\isamarkupfalse%
%
\endisatagproof
{\isafoldproof}%
%
\isadelimproof
%
\endisadelimproof
\isanewline
\isanewline
\isacommand{end}\isamarkupfalse%
%
\begin{isamarkuptext}%
Como podemos observar, se trata de una clase que actúa sobre un parámetro global
de tipo \isa{{\isacharprime}a} cualquiera. Por otro lado, \isa{instantation} define una clase de
parámetros, en este caso los números naturales \isa{nat} que devuelve como resultado. Incluye
una definición concreta del operador \isa{size{\isadigit{1}}} sobre dichos parámetros. Además, el 
último \isa{instance} abre una prueba que afirma que los parámetros dados conforman la clase 
especificada en la definición. Esta prueba que nos afirma que está bien definida la clase aparece
omitida utilizando \isa{{\isachardot}{\isachardot}} .
\\
En particular,
sobre una fórmula nos devuelve el número de elementos de la lista cuyos elementos son los nodos
y las hojas de su árbol de formación.%
\end{isamarkuptext}\isamarkuptrue%
\isacommand{value}\isamarkupfalse%
\ {\isachardoublequoteopen}size{\isacharparenleft}n{\isacharcolon}{\isacharcolon}nat{\isacharparenright}\ {\isacharequal}\ n{\isachardoublequoteclose}\isanewline
\isacommand{value}\isamarkupfalse%
{\isachardoublequoteopen}size{\isacharparenleft}{\isadigit{5}}{\isacharcolon}{\isacharcolon}nat{\isacharparenright}\ {\isacharequal}\ {\isadigit{5}}{\isachardoublequoteclose}%
\begin{isamarkuptext}%
Por otro lado, la función \isa{length} de la teoría \href{http://cort.as/-Stfm}{List.thy}
nos indica la longitud de una lista cualquiera de elementos, definiéndose utilizando el comando
\isa{size} visto anteriormente.%
\end{isamarkuptext}\isamarkuptrue%
\isacommand{abbreviation}\isamarkupfalse%
\ length\ {\isacharcolon}{\isacharcolon}\ {\isachardoublequoteopen}{\isacharprime}a\ list\ {\isasymRightarrow}\ nat{\isachardoublequoteclose}\ \isakeyword{where}\isanewline
{\isachardoublequoteopen}length\ {\isasymequiv}\ size{\isachardoublequoteclose}%
\begin{isamarkuptext}%
Las demostración del resultado se vuelve a basar en la inducción que nos despliega seis casos. 
La prueba estructurada no resulta interesante, pues todos los casos son
inmediatos por simplificación como en el primer lema de esta sección. 
Incluimos a continuación la prueba aplicativa y 
automática.%
\end{isamarkuptext}\isamarkuptrue%
\isacommand{lemma}\isamarkupfalse%
\ length{\isacharunderscore}subformulae{\isacharunderscore}aplicativa{\isacharcolon}\ {\isachardoublequoteopen}length\ {\isacharparenleft}subformulae\ F{\isacharparenright}\ {\isacharequal}\ size\ F{\isachardoublequoteclose}\ \isanewline
%
\isadelimproof
\ \ %
\endisadelimproof
%
\isatagproof
\isacommand{apply}\isamarkupfalse%
\ {\isacharparenleft}induction\ F{\isacharparenright}\ \isanewline
\ \ \isacommand{apply}\isamarkupfalse%
\ simp{\isacharunderscore}all\isanewline
\ \isacommand{done}\isamarkupfalse%
%
\endisatagproof
{\isafoldproof}%
%
\isadelimproof
\isanewline
%
\endisadelimproof
\isanewline
\isacommand{lemma}\isamarkupfalse%
\ length{\isacharunderscore}subformulae{\isacharcolon}\ {\isachardoublequoteopen}length\ {\isacharparenleft}subformulae\ F{\isacharparenright}\ {\isacharequal}\ size\ F{\isachardoublequoteclose}\ \isanewline
%
\isadelimproof
\ \ %
\endisadelimproof
%
\isatagproof
\isacommand{by}\isamarkupfalse%
\ {\isacharparenleft}induction\ F{\isacharsemicolon}\ simp{\isacharparenright}%
\endisatagproof
{\isafoldproof}%
%
\isadelimproof
%
\endisadelimproof
%
\isadelimdocument
%
\endisadelimdocument
%
\isatagdocument
%
\isamarkupsubsection{Conectivas Derivadas%
}
\isamarkuptrue%
%
\endisatagdocument
{\isafolddocument}%
%
\isadelimdocument
%
\endisadelimdocument
%
\begin{isamarkuptext}%
En esta sección definiremos nuevas conectivas y elementos a partir de los ya definidos en el
apartado anterior. Además veremos varios resultados sobre los mismos.%
\end{isamarkuptext}\isamarkuptrue%
%
\begin{isamarkuptext}%
En primer lugar, vamos a definir \isa{Top{\isacharcolon}{\isacharcolon}\ {\isacharprime}a\ formula\ {\isasymRightarrow}\ bool} como la constante  que devuelve
el booleano contrario a \isa{Bot}. Se trata, por tanto, de una constante de la misma naturaleza que
la ya definida para \isa{Bot}. De este modo, \isa{Top} será equivalente a \isa{Verdadero}, y \isa{Bot} a \isa{Falso},
según se muestra en la siguiente ecuación. Su símbolo queda igualmente retratado a continuación.%
\end{isamarkuptext}\isamarkuptrue%
\isacommand{definition}\isamarkupfalse%
\ Top\ {\isacharparenleft}{\isachardoublequoteopen}{\isasymtop}{\isachardoublequoteclose}{\isacharparenright}\ \isakeyword{where}\isanewline
{\isachardoublequoteopen}{\isasymtop}\ {\isasymequiv}\ {\isasymbottom}\ \isactrlbold {\isasymrightarrow}\ {\isasymbottom}{\isachardoublequoteclose}%
\begin{isamarkuptext}%
Por la propia definición y naturaleza de \isa{Top}, verifica que no contiene ninguna variable del
alfabeto, como ya sabíamos análogamente para \isa{Bot}. Tenemos así la siguiente propiedad.%
\end{isamarkuptext}\isamarkuptrue%
\isacommand{lemma}\isamarkupfalse%
\ top{\isacharunderscore}atoms{\isacharunderscore}simp{\isacharbrackleft}simp{\isacharbrackright}{\isacharcolon}\ {\isachardoublequoteopen}atoms\ {\isasymtop}\ {\isacharequal}\ {\isacharbraceleft}{\isacharbraceright}{\isachardoublequoteclose}\ \isanewline
%
\isadelimproof
\ \ %
\endisadelimproof
%
\isatagproof
\isacommand{unfolding}\isamarkupfalse%
\ Top{\isacharunderscore}def\ \isacommand{by}\isamarkupfalse%
\ simp%
\endisatagproof
{\isafoldproof}%
%
\isadelimproof
%
\endisadelimproof
%
\begin{isamarkuptext}%
A continuación vamos a definir dos conectivas que generalizarán la conjunción y la disyunción
para una lista finita de fórmulas. .%
\end{isamarkuptext}\isamarkuptrue%
\isacommand{primrec}\isamarkupfalse%
\ BigAnd\ {\isacharcolon}{\isacharcolon}\ {\isachardoublequoteopen}{\isacharprime}a\ formula\ list\ {\isasymRightarrow}\ {\isacharprime}a\ formula{\isachardoublequoteclose}\ {\isacharparenleft}{\isachardoublequoteopen}\isactrlbold {\isasymAnd}{\isacharunderscore}{\isachardoublequoteclose}{\isacharparenright}\ \isakeyword{where}\isanewline
\ \ {\isachardoublequoteopen}\isactrlbold {\isasymAnd}Nil\ {\isacharequal}\ {\isacharparenleft}\isactrlbold {\isasymnot}{\isasymbottom}{\isacharparenright}{\isachardoublequoteclose}\ \isanewline
{\isacharbar}\ {\isachardoublequoteopen}\isactrlbold {\isasymAnd}{\isacharparenleft}F{\isacharhash}Fs{\isacharparenright}\ {\isacharequal}\ F\ \isactrlbold {\isasymand}\ \isactrlbold {\isasymAnd}Fs{\isachardoublequoteclose}\isanewline
\isanewline
\isacommand{primrec}\isamarkupfalse%
\ BigOr\ {\isacharcolon}{\isacharcolon}\ {\isachardoublequoteopen}{\isacharprime}a\ formula\ list\ {\isasymRightarrow}\ {\isacharprime}a\ formula{\isachardoublequoteclose}\ {\isacharparenleft}{\isachardoublequoteopen}\isactrlbold {\isasymOr}{\isacharunderscore}{\isachardoublequoteclose}{\isacharparenright}\ \isakeyword{where}\isanewline
\ \ {\isachardoublequoteopen}\isactrlbold {\isasymOr}Nil\ {\isacharequal}\ {\isasymbottom}{\isachardoublequoteclose}\ \isanewline
{\isacharbar}\ {\isachardoublequoteopen}\isactrlbold {\isasymOr}{\isacharparenleft}F{\isacharhash}Fs{\isacharparenright}\ {\isacharequal}\ F\ \isactrlbold {\isasymor}\ \isactrlbold {\isasymOr}Fs{\isachardoublequoteclose}%
\begin{isamarkuptext}%
Ambas nuevas conectivas se caracterizarán por ser del tipo funciones primitivas recursivas. Por
tanto, sus definiciones se basan en dos casos:
  \begin{description}
  \item[Lista vacía:] Representada como \isa{Nil}. En este caso, la conjunción plural aplicada a la lista
vacía nos devuelve la negación de \isa{Bot}, es decir, \isa{Verdadero}, y la disyunción plural sobre la lista
vacía nos da simplemente \isa{Bot}, luego \isa{Falso}. 
  \item[Lista recursiva:] En este caso actúa sobre \isa{F{\isacharhash}Fs} donde \isa{F} es una fórmula concatenada a la
lista de fórmulas \isa{Fs}. Como es lógico, \isa{BigAnd} nos devuelve la conjunción de todas las fórmulas
de la lista y \isa{BigOr} nos devuelve su disyunción.
  \end{description}
Además, se observa en cada función el símbolo de notación que aparece entre paréntesis.
La conjunción plural nos da el siguiente resultado.%
\end{isamarkuptext}\isamarkuptrue%
\isacommand{lemma}\isamarkupfalse%
\ atoms{\isacharunderscore}BigAnd{\isacharbrackleft}simp{\isacharbrackright}{\isacharcolon}\ {\isachardoublequoteopen}atoms\ {\isacharparenleft}\isactrlbold {\isasymAnd}Fs{\isacharparenright}\ {\isacharequal}\ {\isasymUnion}{\isacharparenleft}atoms\ {\isacharbackquote}\ set\ Fs{\isacharparenright}{\isachardoublequoteclose}\isanewline
%
\isadelimproof
\ \ %
\endisadelimproof
%
\isatagproof
\isacommand{by}\isamarkupfalse%
{\isacharparenleft}induction\ Fs{\isacharsemicolon}\ simp{\isacharparenright}\isanewline
\isanewline
%
\endisatagproof
{\isafoldproof}%
%
\isadelimproof
%
\endisadelimproof
%
\isadelimtheory
%
\endisadelimtheory
%
\isatagtheory
%
\endisatagtheory
{\isafoldtheory}%
%
\isadelimtheory
%
\endisadelimtheory
%
\end{isabellebody}%
\endinput
%:%file=~/LogicaProposicional/Sintaxis.thy%:%
%:%24=11%:%
%:%34=13%:%
%:%35=13%:%
%:%37=15%:%
%:%38=16%:%
%:%39=17%:%
%:%40=18%:%
%:%41=19%:%
%:%42=20%:%
%:%43=21%:%
%:%44=22%:%
%:%45=23%:%
%:%46=24%:%
%:%47=25%:%
%:%48=26%:%
%:%49=27%:%
%:%50=28%:%
%:%51=29%:%
%:%52=30%:%
%:%53=31%:%
%:%54=32%:%
%:%58=34%:%
%:%59=35%:%
%:%60=36%:%
%:%62=38%:%
%:%63=38%:%
%:%64=39%:%
%:%65=40%:%
%:%66=41%:%
%:%67=42%:%
%:%68=43%:%
%:%69=44%:%
%:%71=46%:%
%:%72=47%:%
%:%73=48%:%
%:%74=49%:%
%:%75=50%:%
%:%76=51%:%
%:%77=52%:%
%:%78=53%:%
%:%79=54%:%
%:%80=55%:%
%:%81=56%:%
%:%82=57%:%
%:%83=58%:%
%:%84=59%:%
%:%85=60%:%
%:%86=61%:%
%:%87=62%:%
%:%88=63%:%
%:%89=64%:%
%:%90=65%:%
%:%91=66%:%
%:%93=68%:%
%:%94=68%:%
%:%96=70%:%
%:%97=71%:%
%:%99=73%:%
%:%100=73%:%
%:%101=74%:%
%:%102=75%:%
%:%103=75%:%
%:%105=77%:%
%:%106=78%:%
%:%108=80%:%
%:%109=80%:%
%:%112=81%:%
%:%116=81%:%
%:%117=81%:%
%:%122=81%:%
%:%125=82%:%
%:%126=83%:%
%:%127=83%:%
%:%130=84%:%
%:%134=84%:%
%:%135=84%:%
%:%144=86%:%
%:%145=87%:%
%:%146=88%:%
%:%147=89%:%
%:%148=90%:%
%:%150=93%:%
%:%151=93%:%
%:%153=96%:%
%:%157=99%:%
%:%158=100%:%
%:%159=101%:%
%:%161=103%:%
%:%162=103%:%
%:%165=104%:%
%:%169=104%:%
%:%170=104%:%
%:%175=104%:%
%:%178=105%:%
%:%179=106%:%
%:%180=106%:%
%:%183=107%:%
%:%187=107%:%
%:%188=107%:%
%:%197=109%:%
%:%198=110%:%
%:%199=111%:%
%:%200=112%:%
%:%201=113%:%
%:%202=114%:%
%:%203=115%:%
%:%204=116%:%
%:%205=117%:%
%:%207=120%:%
%:%208=120%:%
%:%209=121%:%
%:%210=122%:%
%:%211=123%:%
%:%212=124%:%
%:%214=126%:%
%:%215=127%:%
%:%216=128%:%
%:%217=129%:%
%:%218=130%:%
%:%219=131%:%
%:%220=132%:%
%:%221=133%:%
%:%222=134%:%
%:%226=136%:%
%:%227=137%:%
%:%228=138%:%
%:%229=139%:%
%:%233=142%:%
%:%235=144%:%
%:%236=144%:%
%:%239=145%:%
%:%243=145%:%
%:%253=147%:%
%:%254=148%:%
%:%255=149%:%
%:%259=151%:%
%:%260=152%:%
%:%261=153%:%
%:%262=154%:%
%:%263=155%:%
%:%264=156%:%
%:%265=157%:%
%:%266=158%:%
%:%267=159%:%
%:%271=163%:%
%:%272=164%:%
%:%273=165%:%
%:%275=167%:%
%:%276=167%:%
%:%283=168%:%
%:%284=168%:%
%:%285=169%:%
%:%286=169%:%
%:%287=170%:%
%:%288=170%:%
%:%289=170%:%
%:%290=170%:%
%:%291=171%:%
%:%292=171%:%
%:%293=172%:%
%:%294=172%:%
%:%295=173%:%
%:%296=173%:%
%:%297=173%:%
%:%298=173%:%
%:%299=174%:%
%:%300=174%:%
%:%301=175%:%
%:%302=175%:%
%:%303=176%:%
%:%304=176%:%
%:%305=176%:%
%:%306=176%:%
%:%307=177%:%
%:%308=177%:%
%:%309=178%:%
%:%310=178%:%
%:%311=179%:%
%:%312=179%:%
%:%313=179%:%
%:%314=179%:%
%:%315=180%:%
%:%316=180%:%
%:%317=181%:%
%:%318=181%:%
%:%319=182%:%
%:%320=182%:%
%:%321=182%:%
%:%322=182%:%
%:%323=183%:%
%:%324=183%:%
%:%325=184%:%
%:%326=184%:%
%:%327=185%:%
%:%328=185%:%
%:%329=185%:%
%:%330=185%:%
%:%331=186%:%
%:%341=188%:%
%:%343=190%:%
%:%344=190%:%
%:%347=191%:%
%:%351=191%:%
%:%352=191%:%
%:%353=192%:%
%:%354=192%:%
%:%355=193%:%
%:%361=193%:%
%:%364=194%:%
%:%365=195%:%
%:%366=195%:%
%:%369=196%:%
%:%373=196%:%
%:%374=196%:%
%:%388=198%:%
%:%400=200%:%
%:%401=201%:%
%:%402=202%:%
%:%403=203%:%
%:%404=204%:%
%:%405=205%:%
%:%406=206%:%
%:%407=207%:%
%:%408=208%:%
%:%409=209%:%
%:%410=210%:%
%:%414=212%:%
%:%415=213%:%
%:%416=214%:%
%:%417=215%:%
%:%419=217%:%
%:%420=217%:%
%:%421=218%:%
%:%422=219%:%
%:%423=220%:%
%:%424=221%:%
%:%425=222%:%
%:%426=223%:%
%:%428=225%:%
%:%430=227%:%
%:%431=227%:%
%:%432=228%:%
%:%433=229%:%
%:%434=230%:%
%:%435=231%:%
%:%437=233%:%
%:%438=234%:%
%:%440=236%:%
%:%441=236%:%
%:%442=237%:%
%:%443=238%:%
%:%444=238%:%
%:%446=240%:%
%:%448=242%:%
%:%449=242%:%
%:%452=245%:%
%:%453=246%:%
%:%455=248%:%
%:%456=248%:%
%:%457=249%:%
%:%458=250%:%
%:%459=251%:%
%:%460=251%:%
%:%463=254%:%
%:%465=256%:%
%:%466=256%:%
%:%469=259%:%
%:%470=260%:%
%:%471=261%:%
%:%472=262%:%
%:%473=263%:%
%:%474=264%:%
%:%475=265%:%
%:%476=266%:%
%:%477=267%:%
%:%479=269%:%
%:%480=269%:%
%:%481=270%:%
%:%482=271%:%
%:%483=272%:%
%:%484=272%:%
%:%485=273%:%
%:%487=275%:%
%:%489=277%:%
%:%490=277%:%
%:%491=278%:%
%:%492=279%:%
%:%493=279%:%
%:%494=280%:%
%:%497=281%:%
%:%501=281%:%
%:%502=281%:%
%:%511=283%:%
%:%512=284%:%
%:%513=285%:%
%:%514=286%:%
%:%515=287%:%
%:%516=288%:%
%:%517=289%:%
%:%518=290%:%
%:%519=291%:%
%:%520=292%:%
%:%521=293%:%
%:%522=294%:%
%:%523=295%:%
%:%525=297%:%
%:%526=297%:%
%:%529=298%:%
%:%533=298%:%
%:%534=298%:%
%:%539=298%:%
%:%542=299%:%
%:%543=300%:%
%:%544=300%:%
%:%547=301%:%
%:%551=301%:%
%:%552=301%:%
%:%557=301%:%
%:%560=302%:%
%:%561=303%:%
%:%562=303%:%
%:%565=304%:%
%:%569=304%:%
%:%570=304%:%
%:%575=304%:%
%:%578=305%:%
%:%579=306%:%
%:%580=306%:%
%:%583=307%:%
%:%587=307%:%
%:%588=307%:%
%:%593=307%:%
%:%596=308%:%
%:%597=309%:%
%:%598=309%:%
%:%601=310%:%
%:%605=310%:%
%:%606=310%:%
%:%615=312%:%
%:%616=313%:%
%:%617=314%:%
%:%618=315%:%
%:%619=316%:%
%:%620=317%:%
%:%621=318%:%
%:%622=319%:%
%:%623=320%:%
%:%625=323%:%
%:%626=323%:%
%:%629=324%:%
%:%633=324%:%
%:%643=326%:%
%:%644=327%:%
%:%645=328%:%
%:%646=329%:%
%:%647=330%:%
%:%648=331%:%
%:%649=332%:%
%:%650=333%:%
%:%651=334%:%
%:%653=336%:%
%:%654=336%:%
%:%655=337%:%
%:%656=338%:%
%:%657=338%:%
%:%660=339%:%
%:%664=339%:%
%:%665=339%:%
%:%670=339%:%
%:%673=340%:%
%:%674=341%:%
%:%675=341%:%
%:%678=342%:%
%:%682=342%:%
%:%683=342%:%
%:%692=344%:%
%:%693=345%:%
%:%694=346%:%
%:%695=347%:%
%:%696=348%:%
%:%700=350%:%
%:%701=351%:%
%:%703=353%:%
%:%704=353%:%
%:%711=354%:%
%:%712=354%:%
%:%713=355%:%
%:%714=355%:%
%:%715=356%:%
%:%716=356%:%
%:%717=356%:%
%:%718=356%:%
%:%719=357%:%
%:%720=357%:%
%:%721=358%:%
%:%722=358%:%
%:%723=359%:%
%:%724=359%:%
%:%725=359%:%
%:%726=359%:%
%:%727=360%:%
%:%728=360%:%
%:%729=361%:%
%:%730=361%:%
%:%731=362%:%
%:%732=362%:%
%:%733=363%:%
%:%734=363%:%
%:%735=364%:%
%:%736=364%:%
%:%737=365%:%
%:%738=365%:%
%:%739=365%:%
%:%740=366%:%
%:%741=366%:%
%:%742=366%:%
%:%743=366%:%
%:%744=367%:%
%:%745=367%:%
%:%746=367%:%
%:%747=367%:%
%:%748=368%:%
%:%749=368%:%
%:%750=368%:%
%:%751=368%:%
%:%752=368%:%
%:%753=369%:%
%:%754=369%:%
%:%755=369%:%
%:%756=369%:%
%:%757=369%:%
%:%758=370%:%
%:%759=370%:%
%:%760=371%:%
%:%761=371%:%
%:%762=372%:%
%:%763=372%:%
%:%764=373%:%
%:%765=373%:%
%:%766=374%:%
%:%767=374%:%
%:%768=375%:%
%:%769=375%:%
%:%770=376%:%
%:%771=376%:%
%:%772=377%:%
%:%773=377%:%
%:%774=378%:%
%:%775=378%:%
%:%776=378%:%
%:%777=379%:%
%:%778=379%:%
%:%779=379%:%
%:%780=380%:%
%:%781=380%:%
%:%782=380%:%
%:%783=380%:%
%:%784=381%:%
%:%785=381%:%
%:%786=381%:%
%:%787=381%:%
%:%788=382%:%
%:%789=382%:%
%:%790=382%:%
%:%791=382%:%
%:%792=383%:%
%:%793=383%:%
%:%794=383%:%
%:%795=383%:%
%:%796=384%:%
%:%797=384%:%
%:%798=384%:%
%:%799=384%:%
%:%800=385%:%
%:%801=385%:%
%:%802=385%:%
%:%803=385%:%
%:%804=386%:%
%:%805=386%:%
%:%806=386%:%
%:%807=386%:%
%:%808=387%:%
%:%809=387%:%
%:%810=387%:%
%:%811=387%:%
%:%812=387%:%
%:%813=388%:%
%:%814=388%:%
%:%815=388%:%
%:%816=388%:%
%:%817=388%:%
%:%818=389%:%
%:%819=389%:%
%:%820=390%:%
%:%830=392%:%
%:%832=394%:%
%:%833=394%:%
%:%836=395%:%
%:%840=395%:%
%:%841=395%:%
%:%850=397%:%
%:%851=398%:%
%:%852=399%:%
%:%853=400%:%
%:%854=401%:%
%:%855=402%:%
%:%856=403%:%
%:%857=404%:%
%:%858=405%:%
%:%859=406%:%
%:%861=408%:%
%:%862=408%:%
%:%865=409%:%
%:%869=409%:%
%:%870=409%:%
%:%875=409%:%
%:%878=410%:%
%:%879=411%:%
%:%880=411%:%
%:%883=412%:%
%:%887=412%:%
%:%888=412%:%
%:%897=414%:%
%:%898=415%:%
%:%899=416%:%
%:%900=417%:%
%:%901=418%:%
%:%902=419%:%
%:%903=420%:%
%:%905=422%:%
%:%906=422%:%
%:%909=423%:%
%:%913=423%:%
%:%923=425%:%
%:%924=426%:%
%:%926=428%:%
%:%927=428%:%
%:%934=429%:%
%:%935=429%:%
%:%936=430%:%
%:%937=430%:%
%:%938=431%:%
%:%939=431%:%
%:%940=431%:%
%:%941=431%:%
%:%942=432%:%
%:%943=432%:%
%:%944=433%:%
%:%945=433%:%
%:%946=434%:%
%:%947=434%:%
%:%948=434%:%
%:%949=434%:%
%:%950=435%:%
%:%951=435%:%
%:%952=436%:%
%:%953=436%:%
%:%954=437%:%
%:%955=437%:%
%:%956=438%:%
%:%957=438%:%
%:%958=439%:%
%:%959=439%:%
%:%960=440%:%
%:%961=440%:%
%:%962=441%:%
%:%963=441%:%
%:%964=442%:%
%:%965=442%:%
%:%966=442%:%
%:%967=442%:%
%:%968=443%:%
%:%969=443%:%
%:%970=443%:%
%:%971=443%:%
%:%972=444%:%
%:%973=444%:%
%:%974=445%:%
%:%975=445%:%
%:%976=446%:%
%:%977=446%:%
%:%978=446%:%
%:%979=446%:%
%:%980=447%:%
%:%981=447%:%
%:%982=447%:%
%:%983=448%:%
%:%984=448%:%
%:%985=448%:%
%:%986=448%:%
%:%987=448%:%
%:%988=449%:%
%:%989=449%:%
%:%990=449%:%
%:%991=450%:%
%:%992=450%:%
%:%993=450%:%
%:%994=450%:%
%:%995=450%:%
%:%996=451%:%
%:%997=451%:%
%:%998=452%:%
%:%999=452%:%
%:%1000=453%:%
%:%1001=453%:%
%:%1002=454%:%
%:%1003=454%:%
%:%1004=455%:%
%:%1005=455%:%
%:%1006=456%:%
%:%1007=456%:%
%:%1008=457%:%
%:%1009=457%:%
%:%1010=457%:%
%:%1011=458%:%
%:%1012=458%:%
%:%1013=459%:%
%:%1014=459%:%
%:%1015=460%:%
%:%1016=460%:%
%:%1017=460%:%
%:%1018=460%:%
%:%1019=461%:%
%:%1020=461%:%
%:%1021=461%:%
%:%1022=461%:%
%:%1023=462%:%
%:%1024=462%:%
%:%1025=463%:%
%:%1026=463%:%
%:%1027=464%:%
%:%1028=464%:%
%:%1029=464%:%
%:%1030=464%:%
%:%1031=465%:%
%:%1032=465%:%
%:%1033=465%:%
%:%1034=466%:%
%:%1035=466%:%
%:%1036=466%:%
%:%1037=466%:%
%:%1038=466%:%
%:%1039=467%:%
%:%1040=467%:%
%:%1041=467%:%
%:%1042=468%:%
%:%1043=468%:%
%:%1044=468%:%
%:%1045=469%:%
%:%1046=469%:%
%:%1047=470%:%
%:%1048=470%:%
%:%1049=471%:%
%:%1050=471%:%
%:%1051=471%:%
%:%1052=471%:%
%:%1053=472%:%
%:%1054=472%:%
%:%1055=472%:%
%:%1056=472%:%
%:%1057=472%:%
%:%1058=473%:%
%:%1059=473%:%
%:%1060=473%:%
%:%1061=473%:%
%:%1062=474%:%
%:%1063=474%:%
%:%1064=474%:%
%:%1065=474%:%
%:%1066=475%:%
%:%1067=475%:%
%:%1068=476%:%
%:%1069=476%:%
%:%1070=477%:%
%:%1071=477%:%
%:%1072=477%:%
%:%1073=477%:%
%:%1074=478%:%
%:%1075=478%:%
%:%1076=478%:%
%:%1077=478%:%
%:%1078=478%:%
%:%1079=479%:%
%:%1080=479%:%
%:%1081=479%:%
%:%1082=479%:%
%:%1083=479%:%
%:%1084=480%:%
%:%1085=480%:%
%:%1086=480%:%
%:%1087=480%:%
%:%1088=481%:%
%:%1089=481%:%
%:%1090=481%:%
%:%1091=481%:%
%:%1092=482%:%
%:%1093=482%:%
%:%1094=483%:%
%:%1095=483%:%
%:%1096=484%:%
%:%1106=486%:%
%:%1108=488%:%
%:%1109=488%:%
%:%1112=489%:%
%:%1116=489%:%
%:%1117=489%:%
%:%1118=490%:%
%:%1119=490%:%
%:%1120=491%:%
%:%1126=491%:%
%:%1129=492%:%
%:%1130=493%:%
%:%1131=493%:%
%:%1134=494%:%
%:%1138=494%:%
%:%1139=494%:%
%:%1148=496%:%
%:%1149=497%:%
%:%1150=498%:%
%:%1151=499%:%
%:%1152=500%:%
%:%1153=501%:%
%:%1154=502%:%
%:%1155=503%:%
%:%1157=505%:%
%:%1158=505%:%
%:%1165=506%:%
%:%1166=506%:%
%:%1167=507%:%
%:%1168=507%:%
%:%1169=508%:%
%:%1170=508%:%
%:%1171=508%:%
%:%1172=508%:%
%:%1173=509%:%
%:%1174=509%:%
%:%1175=510%:%
%:%1176=510%:%
%:%1177=511%:%
%:%1178=511%:%
%:%1179=511%:%
%:%1180=511%:%
%:%1181=512%:%
%:%1182=512%:%
%:%1183=513%:%
%:%1184=513%:%
%:%1185=514%:%
%:%1186=514%:%
%:%1187=515%:%
%:%1188=515%:%
%:%1189=516%:%
%:%1190=516%:%
%:%1191=516%:%
%:%1192=517%:%
%:%1193=517%:%
%:%1194=518%:%
%:%1195=518%:%
%:%1196=519%:%
%:%1197=519%:%
%:%1198=519%:%
%:%1199=519%:%
%:%1200=520%:%
%:%1201=520%:%
%:%1202=521%:%
%:%1203=521%:%
%:%1204=522%:%
%:%1205=522%:%
%:%1206=522%:%
%:%1207=522%:%
%:%1208=523%:%
%:%1209=523%:%
%:%1210=523%:%
%:%1211=523%:%
%:%1212=523%:%
%:%1213=524%:%
%:%1214=524%:%
%:%1215=524%:%
%:%1216=524%:%
%:%1217=524%:%
%:%1218=525%:%
%:%1219=525%:%
%:%1220=525%:%
%:%1221=526%:%
%:%1222=526%:%
%:%1223=526%:%
%:%1224=526%:%
%:%1225=527%:%
%:%1226=527%:%
%:%1227=527%:%
%:%1228=527%:%
%:%1229=528%:%
%:%1230=528%:%
%:%1231=529%:%
%:%1232=529%:%
%:%1233=530%:%
%:%1234=530%:%
%:%1235=531%:%
%:%1236=531%:%
%:%1237=532%:%
%:%1238=532%:%
%:%1239=533%:%
%:%1240=533%:%
%:%1241=534%:%
%:%1242=534%:%
%:%1243=534%:%
%:%1244=535%:%
%:%1245=535%:%
%:%1246=535%:%
%:%1247=536%:%
%:%1248=536%:%
%:%1249=537%:%
%:%1250=537%:%
%:%1251=538%:%
%:%1252=538%:%
%:%1253=538%:%
%:%1254=538%:%
%:%1255=539%:%
%:%1256=539%:%
%:%1257=540%:%
%:%1258=540%:%
%:%1259=541%:%
%:%1260=541%:%
%:%1261=541%:%
%:%1262=541%:%
%:%1263=542%:%
%:%1264=542%:%
%:%1265=542%:%
%:%1266=542%:%
%:%1267=543%:%
%:%1268=543%:%
%:%1269=543%:%
%:%1270=544%:%
%:%1271=544%:%
%:%1272=545%:%
%:%1273=545%:%
%:%1274=546%:%
%:%1275=546%:%
%:%1276=546%:%
%:%1277=546%:%
%:%1278=547%:%
%:%1279=547%:%
%:%1280=547%:%
%:%1281=547%:%
%:%1282=547%:%
%:%1283=548%:%
%:%1284=548%:%
%:%1285=548%:%
%:%1286=548%:%
%:%1287=549%:%
%:%1288=549%:%
%:%1289=549%:%
%:%1290=549%:%
%:%1291=550%:%
%:%1292=550%:%
%:%1293=551%:%
%:%1294=551%:%
%:%1295=552%:%
%:%1296=552%:%
%:%1297=552%:%
%:%1298=552%:%
%:%1299=553%:%
%:%1300=553%:%
%:%1301=553%:%
%:%1302=553%:%
%:%1303=554%:%
%:%1304=554%:%
%:%1305=554%:%
%:%1306=554%:%
%:%1307=554%:%
%:%1308=555%:%
%:%1309=555%:%
%:%1310=555%:%
%:%1311=555%:%
%:%1312=555%:%
%:%1313=556%:%
%:%1314=556%:%
%:%1315=556%:%
%:%1316=556%:%
%:%1317=557%:%
%:%1318=557%:%
%:%1319=557%:%
%:%1320=557%:%
%:%1321=558%:%
%:%1322=558%:%
%:%1323=559%:%
%:%1324=559%:%
%:%1325=560%:%
%:%1331=560%:%
%:%1334=561%:%
%:%1335=562%:%
%:%1336=563%:%
%:%1337=564%:%
%:%1338=564%:%
%:%1341=565%:%
%:%1345=565%:%
%:%1346=565%:%
%:%1355=567%:%
%:%1356=568%:%
%:%1357=569%:%
%:%1358=570%:%
%:%1359=571%:%
%:%1360=572%:%
%:%1361=573%:%
%:%1362=574%:%
%:%1363=575%:%
%:%1364=576%:%
%:%1366=578%:%
%:%1367=578%:%
%:%1368=579%:%
%:%1369=580%:%
%:%1370=581%:%
%:%1377=582%:%
%:%1378=582%:%
%:%1379=583%:%
%:%1380=583%:%
%:%1381=583%:%
%:%1382=583%:%
%:%1383=584%:%
%:%1384=584%:%
%:%1385=584%:%
%:%1386=584%:%
%:%1387=585%:%
%:%1388=585%:%
%:%1389=585%:%
%:%1390=585%:%
%:%1391=585%:%
%:%1392=586%:%
%:%1393=586%:%
%:%1394=586%:%
%:%1395=587%:%
%:%1396=587%:%
%:%1397=587%:%
%:%1398=587%:%
%:%1399=587%:%
%:%1400=588%:%
%:%1406=588%:%
%:%1409=589%:%
%:%1410=590%:%
%:%1411=590%:%
%:%1414=591%:%
%:%1418=591%:%
%:%1424=591%:%
%:%1427=592%:%
%:%1428=593%:%
%:%1429=593%:%
%:%1432=594%:%
%:%1436=594%:%
%:%1437=594%:%
%:%1446=596%:%
%:%1447=597%:%
%:%1449=599%:%
%:%1450=599%:%
%:%1453=600%:%
%:%1457=600%:%
%:%1458=600%:%
%:%1467=602%:%
%:%1468=603%:%
%:%1470=605%:%
%:%1471=605%:%
%:%1472=606%:%
%:%1473=607%:%
%:%1474=608%:%
%:%1475=609%:%
%:%1478=610%:%
%:%1482=610%:%
%:%1492=612%:%
%:%1493=613%:%
%:%1494=614%:%
%:%1496=616%:%
%:%1497=616%:%
%:%1498=617%:%
%:%1499=618%:%
%:%1506=619%:%
%:%1507=619%:%
%:%1508=620%:%
%:%1509=620%:%
%:%1510=620%:%
%:%1511=621%:%
%:%1512=621%:%
%:%1513=621%:%
%:%1514=621%:%
%:%1515=622%:%
%:%1516=622%:%
%:%1517=622%:%
%:%1518=622%:%
%:%1519=623%:%
%:%1520=623%:%
%:%1521=623%:%
%:%1522=623%:%
%:%1523=624%:%
%:%1524=624%:%
%:%1525=624%:%
%:%1526=624%:%
%:%1527=625%:%
%:%1533=625%:%
%:%1536=626%:%
%:%1537=627%:%
%:%1538=627%:%
%:%1539=628%:%
%:%1540=629%:%
%:%1547=630%:%
%:%1548=630%:%
%:%1549=631%:%
%:%1550=631%:%
%:%1551=631%:%
%:%1552=632%:%
%:%1553=632%:%
%:%1554=632%:%
%:%1555=632%:%
%:%1556=633%:%
%:%1557=633%:%
%:%1558=633%:%
%:%1559=633%:%
%:%1560=634%:%
%:%1570=636%:%
%:%1571=637%:%
%:%1573=639%:%
%:%1574=639%:%
%:%1575=640%:%
%:%1576=641%:%
%:%1579=642%:%
%:%1583=642%:%
%:%1584=642%:%
%:%1585=643%:%
%:%1586=643%:%
%:%1587=644%:%
%:%1593=644%:%
%:%1596=645%:%
%:%1597=646%:%
%:%1598=646%:%
%:%1599=647%:%
%:%1600=648%:%
%:%1601=649%:%
%:%1602=650%:%
%:%1605=651%:%
%:%1609=651%:%
%:%1610=651%:%
%:%1619=653%:%
%:%1620=654%:%
%:%1622=656%:%
%:%1623=656%:%
%:%1626=657%:%
%:%1630=657%:%
%:%1640=659%:%
%:%1641=660%:%
%:%1643=662%:%
%:%1644=662%:%
%:%1645=663%:%
%:%1646=664%:%
%:%1647=665%:%
%:%1648=665%:%
%:%1649=666%:%
%:%1650=667%:%
%:%1651=668%:%
%:%1652=668%:%
%:%1653=669%:%
%:%1654=670%:%
%:%1657=670%:%
%:%1661=670%:%
%:%1669=670%:%
%:%1670=671%:%
%:%1671=672%:%
%:%1674=674%:%
%:%1675=675%:%
%:%1676=676%:%
%:%1677=677%:%
%:%1678=678%:%
%:%1679=679%:%
%:%1680=680%:%
%:%1681=681%:%
%:%1682=682%:%
%:%1683=683%:%
%:%1684=684%:%
%:%1686=687%:%
%:%1687=687%:%
%:%1688=688%:%
%:%1689=688%:%
%:%1691=691%:%
%:%1692=692%:%
%:%1693=693%:%
%:%1695=695%:%
%:%1696=695%:%
%:%1697=696%:%
%:%1699=698%:%
%:%1700=699%:%
%:%1701=700%:%
%:%1702=701%:%
%:%1703=702%:%
%:%1705=704%:%
%:%1706=704%:%
%:%1709=705%:%
%:%1713=705%:%
%:%1714=705%:%
%:%1715=706%:%
%:%1716=706%:%
%:%1717=707%:%
%:%1723=707%:%
%:%1726=708%:%
%:%1727=709%:%
%:%1728=709%:%
%:%1731=710%:%
%:%1735=710%:%
%:%1736=710%:%
%:%1750=712%:%
%:%1762=714%:%
%:%1763=715%:%
%:%1767=717%:%
%:%1768=718%:%
%:%1769=719%:%
%:%1770=720%:%
%:%1772=723%:%
%:%1773=723%:%
%:%1774=724%:%
%:%1776=726%:%
%:%1777=727%:%
%:%1779=729%:%
%:%1780=729%:%
%:%1783=730%:%
%:%1787=730%:%
%:%1788=730%:%
%:%1789=730%:%
%:%1798=732%:%
%:%1799=733%:%
%:%1801=735%:%
%:%1802=735%:%
%:%1803=736%:%
%:%1804=737%:%
%:%1805=738%:%
%:%1806=739%:%
%:%1807=739%:%
%:%1808=740%:%
%:%1809=741%:%
%:%1811=743%:%
%:%1812=744%:%
%:%1813=745%:%
%:%1814=746%:%
%:%1815=747%:%
%:%1816=748%:%
%:%1817=749%:%
%:%1818=750%:%
%:%1819=751%:%
%:%1820=752%:%
%:%1821=753%:%
%:%1822=754%:%
%:%1824=757%:%
%:%1825=757%:%
%:%1828=758%:%
%:%1832=758%:%
%:%1833=758%:%
%:%1834=759%:%